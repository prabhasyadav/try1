%% Generated by Sphinx.
\def\sphinxdocclass{jupyterBook}
\documentclass[letterpaper,10pt,english]{jupyterBook}
\ifdefined\pdfpxdimen
   \let\sphinxpxdimen\pdfpxdimen\else\newdimen\sphinxpxdimen
\fi \sphinxpxdimen=.75bp\relax
\ifdefined\pdfimageresolution
    \pdfimageresolution= \numexpr \dimexpr1in\relax/\sphinxpxdimen\relax
\fi
%% let collapsible pdf bookmarks panel have high depth per default
\PassOptionsToPackage{bookmarksdepth=5}{hyperref}
%% turn off hyperref patch of \index as sphinx.xdy xindy module takes care of
%% suitable \hyperpage mark-up, working around hyperref-xindy incompatibility
\PassOptionsToPackage{hyperindex=false}{hyperref}
%% memoir class requires extra handling
\makeatletter\@ifclassloaded{memoir}
{\ifdefined\memhyperindexfalse\memhyperindexfalse\fi}{}\makeatother

\PassOptionsToPackage{warn}{textcomp}

\catcode`^^^^00a0\active\protected\def^^^^00a0{\leavevmode\nobreak\ }
\usepackage{cmap}
\usepackage{fontspec}
\defaultfontfeatures[\rmfamily,\sffamily,\ttfamily]{}
\usepackage{amsmath,amssymb,amstext}
\usepackage{polyglossia}
\setmainlanguage{english}



\setmainfont{FreeSerif}[
  Extension      = .otf,
  UprightFont    = *,
  ItalicFont     = *Italic,
  BoldFont       = *Bold,
  BoldItalicFont = *BoldItalic
]
\setsansfont{FreeSans}[
  Extension      = .otf,
  UprightFont    = *,
  ItalicFont     = *Oblique,
  BoldFont       = *Bold,
  BoldItalicFont = *BoldOblique,
]
\setmonofont{FreeMono}[
  Extension      = .otf,
  UprightFont    = *,
  ItalicFont     = *Oblique,
  BoldFont       = *Bold,
  BoldItalicFont = *BoldOblique,
]



\usepackage[Bjarne]{fncychap}
\usepackage[,numfigreset=1,mathnumfig]{sphinx}

\fvset{fontsize=\small}
\usepackage{geometry}


% Include hyperref last.
\usepackage{hyperref}
% Fix anchor placement for figures with captions.
\usepackage{hypcap}% it must be loaded after hyperref.
% Set up styles of URL: it should be placed after hyperref.
\urlstyle{same}

\addto\captionsenglish{\renewcommand{\contentsname}{Basics}}

\usepackage{sphinxmessages}



        % Start of preamble defined in sphinx-jupyterbook-latex %
         \usepackage[Latin,Greek]{ucharclasses}
        \usepackage{unicode-math}
        % fixing title of the toc
        \addto\captionsenglish{\renewcommand{\contentsname}{Contents}}
        \hypersetup{
            pdfencoding=auto,
            psdextra
        }
        % End of preamble defined in sphinx-jupyterbook-latex %
        

\title{Groundwater I}
\date{Dec 02, 2022}
\release{}
\author{P.\@{} K.\@{} Yadav, T.\@{} Reimann and several others}
\newcommand{\sphinxlogo}{\vbox{}}
\renewcommand{\releasename}{}
\makeindex
\begin{document}

\pagestyle{empty}
\sphinxmaketitle
\pagestyle{plain}
\sphinxtableofcontents
\pagestyle{normal}
\phantomsection\label{\detokenize{intro::doc}}


\sphinxAtStartPar
The site provides an interactive JUPYTER\sphinxhyphen{}notebook based book with contents typical of an \sphinxstyleemphasis{introductory} groundwater course taught at higher UG level or the early PG level at universities.

\sphinxAtStartPar
The contents/structure provided here are mostly from those developed and lectured by \sphinxstylestrong{Prof. Rudolf Liedl} at \sphinxhref{https://tu-dresden.de/}{TU Dresden}.

\sphinxAtStartPar
The contents are geared towards \sphinxstylestrong{learning through computing}. The computing part is entirely based on \sphinxcode{\sphinxupquote{Python}} programming language. Previous programming/coding experiences are not required or expected. Coding is to be learned as a \sphinxstyleemphasis{method of learning} and is considered as a skill enhancing component rather than knowledge acquiring component, which is focussed towards \sphinxstylestrong{Groundwater} learning.

\sphinxAtStartPar
The contents of the book are divided into:
\begin{enumerate}
\sphinxsetlistlabels{\arabic}{enumi}{enumii}{}{.}%
\item {} 
\sphinxAtStartPar
Lecture Parts

\item {} 
\sphinxAtStartPar
Tutorial Parts

\item {} 
\sphinxAtStartPar
Self\sphinxhyphen{}learning (simulation) tools

\end{enumerate}

\sphinxAtStartPar
The \sphinxstyleemphasis{lecture} parts are combination of \sphinxstylestrong{texts} and \sphinxstylestrong{simpler} numerical examples. Only minimum \sphinxcode{\sphinxupquote{Python}} codes are available on this part.
The \sphinxstyleemphasis{tutorial} are mostly \sphinxstylestrong{numerical} examples. This aims at teaching also the basic of \sphinxcode{\sphinxupquote{Python}} coding. The focus remains on illustrating lecture contents. \sphinxstyleemphasis{Self\sphinxhyphen{}learning tools} are interactive tools that supports the lecture and tutorial components and enhances understanding. The codes of these \sphinxstyleemphasis{tools} may require higher knowledge of \sphinxcode{\sphinxupquote{Python}} programming and therefore they are not to be mastered as part of the course.

\sphinxAtStartPar
All codes and contents provided in this interactive book are licensed under \sphinxhref{https://creativecommons.org/licenses/by/4.0/}{Creative Commons BY 4.0}
Codes are available at this \sphinxhref{https://github.com/prabhasyadav/iGW-I}{GitHUB repository}
\begin{quote}

\sphinxAtStartPar
\sphinxstylestrong{The development of the book is based on the \sphinxstylestrong{wonderful} works of the \sphinxhref{https://jupyterbook.org/intro.html}{JUPYTER Book Team}}
\end{quote}

\begin{DUlineblock}{0em}
\item[] \sphinxstylestrong{\Large Main contributors of this version of the book}
\end{DUlineblock}

\sphinxAtStartPar
The contents are developed by (not in any order):
\begin{itemize}
\item {} 
\sphinxAtStartPar
Prof. Rudolf Liedl (TU Dresden)

\item {} 
\sphinxAtStartPar
Prof. Peter Dietrich (UFZ Leipzig/Uni\sphinxhyphen{}Tübingen)

\item {} 
\sphinxAtStartPar
Prof. B. R. Chahar (Indian Institute of Technology Delhi, Delhi)

\item {} 
\sphinxAtStartPar
Prof Charles Werth (Uni\sphinxhyphen{}Texas Austin, US)

\item {} 
\sphinxAtStartPar
Dr.rer. nat. Prabhas K Yadav (TU Dresden)

\item {} 
\sphinxAtStartPar
Dr. Ing. Thomas Reimann (TU Dresden)

\item {} 
\sphinxAtStartPar
M.Sc. Hanieh Mehrdad (Student Assistant/Numerical contents\sphinxhyphen{} TU Dresden)

\end{itemize}

\sphinxAtStartPar
and several students: Anton, Abhiral, Anne, Sophie, Alexander

\begin{DUlineblock}{0em}
\item[] \sphinxstylestrong{\large Acknowledgments}
\end{DUlineblock}

\sphinxAtStartPar
This work is partly supported by:
\begin{enumerate}
\sphinxsetlistlabels{\arabic}{enumi}{enumii}{}{.}%
\item {} 
\sphinxAtStartPar
The \sphinxhref{https://tu-dresden.de/tu-dresden/organisation/rektorat/prorektor-bildung-und-internationales/zill/e-learning/multimediafonds}{Multimediafonds, TU\sphinxhyphen{}Dresden}

\item {} 
\sphinxAtStartPar
The \sphinxhref{https://www.dfg.de/}{ESTIMATE (LI 727/29\sphinxhyphen{}1), project, DFG}

\end{enumerate}

\sphinxstepscope


\part{Basics}

\sphinxstepscope


\chapter{About this Groundwater Course and Contents}
\label{\detokenize{content/background/00_general:about-this-groundwater-course-and-contents}}\label{\detokenize{content/background/00_general::doc}}
\sphinxAtStartPar
The contents provided in this website that forms an interactive book, is based on the Groundwater course created, maintained and lectured by
\sphinxstylestrong{Prof. Rudolf Liedl} for over 15 years at the \sphinxhref{https://tu-dresden.de/bu/umwelt/hydro/igw}{Institute of Groundwater Manangement} of \DUrole{xref,myst}{TU Dresden}.

\sphinxAtStartPar
The course structure and contents have been used as an \sphinxstyleemphasis{Introductory} course on Groundwater for M.Sc. level students coming from multiple academic backgrounds.

\sphinxAtStartPar
The \sphinxstylestrong{texts} provided in this interactive book is mostly based on the Prof. Liedl’s lecture slides and the \sphinxstylestrong{interactive codes} are conversion from MS Excel® spreadsheet to \sphinxstyleemphasis{Python} codes.
\begin{quote}

\sphinxAtStartPar
\sphinxstylestrong{This interactive web\sphinxhyphen{}book is dedicated to Prof. Rudolf Liedl efforts to teach, train and inspire us and other students over the years.}
\end{quote}


\section{Basic contents structure}
\label{\detokenize{content/background/00_general:basic-contents-structure}}
\sphinxAtStartPar
The contents of this interactive book can be broadly divided in the following three groups:
\begin{enumerate}
\sphinxsetlistlabels{\arabic}{enumi}{enumii}{}{.}%
\item {} 
\sphinxAtStartPar
Aquifer properties and groundwater flow

\item {} 
\sphinxAtStartPar
Transport in groundwater

\item {} 
\sphinxAtStartPar
Groundwater Modelling

\end{enumerate}

\sphinxAtStartPar
The first group (\sphinxstylestrong{aquifer properties and groundwater flow}) makes the core of this course. Here the very basic of groundwater, subsurface structure, properties that quantify groundwater mass and volume budgets, and the flow and other dynamics processes, e.g., abstraction using wells.

\sphinxAtStartPar
The \sphinxstylestrong{transport in groundwater} topics focus on the quality aspects of groundwater. In particular, transport equations with and without inclusion of chemical reactions (e.g., sorption, decay) are considered. Eventually, few analytical solutions of transport problems are discussed.

\sphinxAtStartPar
The \sphinxstylestrong{Groundwater Modelling} is for introducing the realm of computer modelling of groundwater and transport. Fundamentals of mathematical modelling, e.g., finite different methods, is introduced. The focus remains towards eventual use of MODFLOW (Flopy and Modelmuse interface), which is introduced in a short tutorial form.


\section{What do you need in this course?}
\label{\detokenize{content/background/00_general:what-do-you-need-in-this-course}}
\sphinxAtStartPar
You will need the following:
\begin{itemize}
\item {} 
\sphinxAtStartPar
Laptop/Smartphone/tablet more convenient with internet connection

\item {} 
\sphinxAtStartPar
Recommended is installed JUPYTER interface.

\end{itemize}

\sphinxAtStartPar
The course contents can be received as:
\begin{itemize}
\item {} 
\sphinxAtStartPar
Lectures: Texts reading and interactive manipulation of codes/problems. Short questions (self\sphinxhyphen{}test) should be used to check the understanding of the contents.

\item {} 
\sphinxAtStartPar
Tutorials: Should be understand by manipulating the existing codes. The tutorials should be then independently solved using \sphinxstyleemphasis{JUPYTER} interface (Mybinder or Personal system)

\item {} 
\sphinxAtStartPar
Additional tools: The course provides several simulation tools\sphinxhyphen{} e.g., sieve analysis, effective conductivity, Advection\sphinxhyphen{}dispersion etc. These tools (also \sphinxstyleemphasis{Python}) provide high\sphinxhyphen{}level interactivity. These tools should be used to enhance learning.

\item {} 
\sphinxAtStartPar
Question banks/exam questions should be used to self\sphinxhyphen{}check the level of understanding.

\end{itemize}

\sphinxstepscope


\chapter{Course Introduction \& Water Cycle}
\label{\detokenize{content/background/03_basic_hydrogeology:course-introduction-water-cycle}}\label{\detokenize{content/background/03_basic_hydrogeology::doc}}
\sphinxAtStartPar
\sphinxstyleemphasis{(The contents presented in this section were re\sphinxhyphen{}developed principally by Dr. P. K. Yadav. The original contents were developed by Prof. Rudolf Liedl)}

\begin{sphinxuseclass}{cell}
\begin{sphinxuseclass}{tag_hide-input}
\begin{sphinxuseclass}{tag_remove-input}\begin{sphinxVerbatimOutput}

\begin{sphinxuseclass}{cell_output}
\end{sphinxuseclass}\end{sphinxVerbatimOutput}

\end{sphinxuseclass}
\end{sphinxuseclass}
\end{sphinxuseclass}

\section{Contents of “Groundwater Module ”}
\label{\detokenize{content/background/03_basic_hydrogeology:contents-of-groundwater-module}}
\sphinxAtStartPar
The entire contents of this interactive book can be listed with the following points:
\begin{itemize}
\item {} 
\sphinxAtStartPar
general overview

\item {} 
\sphinxAtStartPar
types of aquifers, properties of aquifers

\item {} 
\sphinxAtStartPar
forces and pressure in the subsurface

\item {} 
\sphinxAtStartPar
laws of groundwater flow and some applications (e.g. groundwater wells)

\item {} 
\sphinxAtStartPar
quantification of aquifer parameter values via pumping tests

\item {} 
\sphinxAtStartPar
transport of chemicals (solutes) in groundwater

\item {} 
\sphinxAtStartPar
retardation and degradation of chemicals in groundwater

\item {} 
\sphinxAtStartPar
groundwater modelling

\end{itemize}


\section{Suggested Literature:}
\label{\detokenize{content/background/03_basic_hydrogeology:suggested-literature}}\begin{itemize}
\item {} 
\sphinxAtStartPar
Brassington R. (1988): Field hydrogeology, Wiley \& Sons.

\item {} 
\sphinxAtStartPar
Domenico P. A., Schwartz F. W. (1990): Physical and chemical hydrogeology, Wiley \& Sons.

\item {} 
\sphinxAtStartPar
Fetter C. W. (2001): Applied hydrogeology, Prentice Hall.

\item {} 
\sphinxAtStartPar
Freeze R. A., Cherry J. A. (1979): Groundwater, Prentice Hall.

\item {} 
\sphinxAtStartPar
Heath R. C. (1987): Basic groundwater hydrology, USGS Water Supply Paper 2220.

\item {} 
\sphinxAtStartPar
Price M. (1996): Introducing groundwater, Chapman and Hall.

\end{itemize}

\sphinxAtStartPar
Additional literature details are provided in the text when used.


\section{What is Hydrogeology?}
\label{\detokenize{content/background/03_basic_hydrogeology:what-is-hydrogeology}}
\sphinxAtStartPar
\sphinxstyleemphasis{\sphinxstylestrong{Hydrogeology}} is the study of the laws governing the movement of subterranean water, the mechanical, chemical, and thermal interaction of this water with the porous solid, and the transport of energy and chemical constituents by the flow.
(Domenico and Schwartz, 1990)

\sphinxAtStartPar
The dominant reliance of groundwater for the drinking globally has made hydrogeology a very important academic course. Also, it is a very important research field. Therefore, several \sphinxstylestrong{techniques} and \sphinxstylestrong{methods} are now available to explore and understand \sphinxstylestrong{Hydrogeological Process}. The methods and techniques can be broadly categorized to:
\begin{enumerate}
\sphinxsetlistlabels{\arabic}{enumi}{enumii}{}{.}%
\item {} 
\sphinxAtStartPar
Field works

\item {} 
\sphinxAtStartPar
Laboratory experiments

\item {} 
\sphinxAtStartPar
Computer modeling

\end{enumerate}

\sphinxAtStartPar
\sphinxstyleemphasis{Computer modelling} is often the most economical method but its usefullness rely of data obtained from \sphinxstyleemphasis{Field works} and \sphinxstyleemphasis{Laboratory experiments.} Thus, the sequence of techniques/methods to be adopted depends on the available site information.

\begin{sphinxuseclass}{cell}
\begin{sphinxuseclass}{tag_full-width}
\begin{sphinxuseclass}{tag_remove-input}\begin{sphinxVerbatimOutput}

\begin{sphinxuseclass}{cell_output}
\begin{sphinxVerbatim}[commandchars=\\\{\}]
Row
    [0] PNG(str, width=250)
    [1] PNG(str, width=275)
    [2] PNG(str, width=280)
\end{sphinxVerbatim}

\end{sphinxuseclass}\end{sphinxVerbatimOutput}

\end{sphinxuseclass}
\end{sphinxuseclass}
\end{sphinxuseclass}

\section{Example: Groundwater Extraction Well}
\label{\detokenize{content/background/03_basic_hydrogeology:example-groundwater-extraction-well}}
\sphinxAtStartPar
Groundwater is extracted using a groundwater well applying \sphinxstyleemphasis{hydrogeological} methods and techniques. The procedure followed can be summarized in the following steps:
\begin{enumerate}
\sphinxsetlistlabels{\arabic}{enumi}{enumii}{}{.}%
\item {} 
\sphinxAtStartPar
The appropriate extraction location is identified

\item {} 
\sphinxAtStartPar
Drilling machine are used to obtain sub\sphinxhyphen{}surface structure, i.e. or well logs are obtained. The process is also called well logging.

\item {} 
\sphinxAtStartPar
Well logs are studied in detail to identify the characteristics of the subsurface\sphinxhyphen{} e.g., how thick is the aquifer or identify environmental consequence of water extraction.

\item {} 
\sphinxAtStartPar
The construction of well begins

\end{enumerate}

\sphinxAtStartPar
Groundwater extraction using well is a challenge when aquifers are located very deep from the surface, e.g., in deserts.

\begin{sphinxuseclass}{cell}
\begin{sphinxuseclass}{tag_full-width}
\begin{sphinxuseclass}{tag_remove-input}\begin{sphinxVerbatimOutput}

\begin{sphinxuseclass}{cell_output}
\begin{sphinxVerbatim}[commandchars=\\\{\}]
Video(str, height=400, sizing\PYGZus{}mode=\PYGZsq{}fixed\PYGZsq{}, width=600)
\end{sphinxVerbatim}

\end{sphinxuseclass}\end{sphinxVerbatimOutput}

\end{sphinxuseclass}
\end{sphinxuseclass}
\end{sphinxuseclass}
\begin{sphinxuseclass}{cell}
\begin{sphinxuseclass}{tag_full-width}
\begin{sphinxuseclass}{tag_remove-input}\begin{sphinxVerbatimOutput}

\begin{sphinxuseclass}{cell_output}
\begin{sphinxVerbatim}[commandchars=\\\{\}]
Row
    [0] Markdown(str)
    [1] Video(str, height=400, sizing\PYGZus{}mode=\PYGZsq{}fixed\PYGZsq{}, width=600)
\end{sphinxVerbatim}

\end{sphinxuseclass}\end{sphinxVerbatimOutput}

\end{sphinxuseclass}
\end{sphinxuseclass}
\end{sphinxuseclass}

\section{Groundwater and Global Water Cycle}
\label{\detokenize{content/background/03_basic_hydrogeology:groundwater-and-global-water-cycle}}
\sphinxAtStartPar
Water bodies that exist on earth is connected, and they function as a cycle, called \sphinxstylestrong{Global Water Cycle}. It is estimated that over 57, 700 Km\(^3\) of water actively participates in the cycle each year. \sphinxstylestrong{Precipitation} and \sphinxstylestrong{evaporation} are the two main components of the cycle in which \sphinxstylestrong{temperature} plays the critical role. In the cycle, \sphinxstylestrong{Groundwater} receives water from \sphinxstyleemphasis{precipitation,} It then contributes to \sphinxstyleemphasis{evaporation} through subsurface flow or through mostly human intervention (e.g., use for drinking water).

\sphinxAtStartPar
The water cycle provides an approach to judge the sustainability of groundwater extraction. The sustainability of extraction can be obtained if extraction rate approximately equals the replenishing rate. Often the replenishing rate of groundwater is much slower and this has led to groundwater stress in many parts of the world.

\begin{sphinxuseclass}{cell}
\begin{sphinxuseclass}{tag_full-width}
\begin{sphinxuseclass}{tag_remove-input}\begin{sphinxVerbatimOutput}

\begin{sphinxuseclass}{cell_output}
\begin{sphinxVerbatim}[commandchars=\\\{\}]
Video(str, height=400, sizing\PYGZus{}mode=\PYGZsq{}fixed\PYGZsq{}, width=600)
\end{sphinxVerbatim}

\end{sphinxuseclass}\end{sphinxVerbatimOutput}

\end{sphinxuseclass}
\end{sphinxuseclass}
\end{sphinxuseclass}
\begin{sphinxuseclass}{cell}
\begin{sphinxuseclass}{tag_full-width}
\begin{sphinxuseclass}{tag_remove-input}\begin{sphinxVerbatimOutput}

\begin{sphinxuseclass}{cell_output}
\begin{sphinxVerbatim}[commandchars=\\\{\}]
Row
    [0] Markdown(str)
    [1] PNG(str, width=600)
\end{sphinxVerbatim}

\end{sphinxuseclass}\end{sphinxVerbatimOutput}

\end{sphinxuseclass}
\end{sphinxuseclass}
\end{sphinxuseclass}

\section{The Hydrological Balance}
\label{\detokenize{content/background/03_basic_hydrogeology:the-hydrological-balance}}
\sphinxAtStartPar
Since \sphinxstyleemphasis{groundwater} is part of the global water cycle, the balance of the cycle becomes an important topic. In general:
\begin{itemize}
\item {} 
\sphinxAtStartPar
The \sphinxstyleemphasis{hydrological balance} provides a relationship between various flow rates for a certain area. It is based on the conservation of water volume.

\item {} 
\sphinxAtStartPar
expressed in words:  \sphinxstyleemphasis{inflow} equals \sphinxstyleemphasis{outflow} plus \sphinxstyleemphasis{change in storage}

\item {} 
\sphinxAtStartPar
expressed by a formula:

\end{itemize}
\begin{equation*}
\begin{split}
P = ET + R + \Delta S
\end{split}
\end{equation*}
\begin{sphinxShadowBox}
\sphinxstylesidebartitle{Where,}

\sphinxAtStartPar
where, 
\(P\) = \sphinxstyleemphasis{Precipitation,  \(ET\) = Evapotranspiration,  \(R\) = Runoff,}  and  \(\Delta S\) = \sphinxstyleemphasis{Change in Storage}
\end{sphinxShadowBox}

\sphinxAtStartPar
The \sphinxstyleemphasis{change in storage} can be interpreted in the following way:
\begin{itemize}
\item {} 
\sphinxAtStartPar
change in storage \(\Delta S > 0\) : Water volume is increasing with time in the investigation area.

\item {} 
\sphinxAtStartPar
change in storage \(\Delta S < 0\):Water volume is decreasing with time in the investigation area.

\item {} 
\sphinxAtStartPar
change in storage \(\Delta S = 0\): Water volume does not change with time in the investigation area (steady\sphinxhyphen{}state or stationary situation, i.e. inflow equals outflow).

\end{itemize}


\section{Water Volume}
\label{\detokenize{content/background/03_basic_hydrogeology:water-volume}}
\sphinxAtStartPar
\sphinxstylestrong{So how much water do we have?}
It is estimated* that the total volume of water on Earth amounts to ca. 1 358 710 150 km\(^3\) (\(\approx\) 1018 m\(^3\)).

\noindent{\hspace*{\fill}\sphinxincludegraphics[height=200\sphinxpxdimen]{{L01_f_6}.png}\hspace*{\fill}}

\sphinxAtStartPar
The total volume of fresh water on Earth amounts to ca. \(38\times 10^6\) km\(^3\) (\(\approx\) 1016 m\(^3\)).

\sphinxAtStartPar
\sphinxstyleemphasis{*Gleick P. (1996): Water re\sphinxhyphen{}sources, in: Schneider S. H. (ed.), Encyclopedia of climate and weather 2, Oxford Univ. Press.}


\section{Volume of Available Fresh Water}
\label{\detokenize{content/background/03_basic_hydrogeology:volume-of-available-fresh-water}}
\sphinxAtStartPar
\sphinxstylestrong{Fresh water} are water with low concentrations of dissolved salts and other total dissolved solids, i.e., sea/ocean water or brackish water are not fresh water. Human activities (drinking water) are directly dependent on fresh activities.

\sphinxAtStartPar
\sphinxstylestrong{So how much \sphinxstyleemphasis{fresh water} do we have?}

\sphinxAtStartPar
It is estimated* that the total volume of available fresh water (liquid) on Earth amounts to ca. 8 831 600 km\(^3\) (\(\approx\) 1016 m\(^3\)).

\noindent{\hspace*{\fill}\sphinxincludegraphics[height=200\sphinxpxdimen]{{L01_f_7}.png}\hspace*{\fill}}

\sphinxAtStartPar
\sphinxstyleemphasis{*Gleick P. (1996): Water re\sphinxhyphen{}sources, in: Schneider S. H. (ed.), Encyclopedia of climate and weather 2, Oxford Univ. Press.}


\section{Continental distribution of fresh water components}
\label{\detokenize{content/background/03_basic_hydrogeology:continental-distribution-of-fresh-water-components}}
\noindent{\hspace*{\fill}\sphinxincludegraphics[height=200\sphinxpxdimen]{{L01_f_8}.png}\hspace*{\fill}}


\section{Volume and Mass Budget}
\label{\detokenize{content/background/03_basic_hydrogeology:volume-and-mass-budget}}
\sphinxAtStartPar
Very basics of volume and mass budget \sphinxhyphen{} let us start with \sphinxstyleemphasis{budget.}

\sphinxAtStartPar
\sphinxstylestrong{Budget} = quantitative comparison of \sphinxstyleemphasis{growth} (or \sphinxstyleemphasis{production}) and \sphinxstyleemphasis{loss} in a system

\sphinxAtStartPar
Budgets can be put together for various quantities:
\begin{itemize}
\item {} 
\sphinxAtStartPar
energy

\item {} 
\sphinxAtStartPar
mass \(\leftarrow\) needed to quantify transport of solutes in groundwater

\item {} 
\sphinxAtStartPar
volume \(\leftarrow\)  needed to quantify groundwater flow

\item {} 
\sphinxAtStartPar
momentum

\item {} 
\sphinxAtStartPar
electric charge

\item {} 
\sphinxAtStartPar
number of inhabitants

\item {} 
\sphinxAtStartPar
animal population

\item {} 
\sphinxAtStartPar
money (bank account!)

\item {} 
\sphinxAtStartPar
and many others

\end{itemize}

\sphinxAtStartPar
In this course we focus on \sphinxstyleemphasis{Mass Budget} and \sphinxstyleemphasis{Volume Budget.}


\section{Volume Budget}
\label{\detokenize{content/background/03_basic_hydrogeology:volume-budget}}
\sphinxAtStartPar
As discussed in the last topic a \sphinxstyleemphasis{budget} represents the change (e.g., growth and loss). Thus, it is more suitable to quantify the \sphinxstylestrong{volume budget} in terms of a change, representing two different states (e.g., time \((t)\)). More formally, the \sphinxstylestrong{volume budget} (\(\Delta V\)) can be obtained from:
\begin{equation*}
\begin{split}
\Delta V = Q_{in} \cdot \Delta t - Q{out} \cdot \Delta t 
\end{split}
\end{equation*}


\sphinxAtStartPar
\(\Delta t\) = time interval {[}T{]} 
\(\Delta V\) = change of volume in the system {[}L\(^3\){]} 
\(Q_{in}\) = volumetric rate of flow into the system {[}L\(^3\)/T{]} 
\(Q_{out} =\) volumetric rate of flow out of the system {[}L\(^3\)/T{]} 

\sphinxAtStartPar
The following points have to be considered when using the above equation:
\begin{itemize}
\item {} 
\sphinxAtStartPar
Inflow and outflow may each consist of several individual components. 

\item {} 
\sphinxAtStartPar
\(\Delta V = 0\) (no change in  volume) is tantamount to steady\sphinxhyphen{}state or stationary (= time\sphinxhyphen{}independent) conditions. 

\item {} 
\sphinxAtStartPar
For steady\sphinxhyphen{}state conditions we have: \(Q_{in} = Q_{out}\)

\end{itemize}


\subsection{Water Budget for a Catchment}
\label{\detokenize{content/background/03_basic_hydrogeology:water-budget-for-a-catchment}}
\sphinxAtStartPar
The equation provided for \sphinxstyleemphasis{volume budget} looks simple but in practice it is very complicated as several \sphinxstyleemphasis{inflow} and \sphinxstyleemphasis{outflow} components must be considered. Quantifying these components can be a challenging task.

\sphinxAtStartPar
For quantifying water budget of a catchment, one has to consider the following components:

\sphinxAtStartPar
\sphinxstylestrong{To be considered:}
\begin{itemize}
\item {} 
\sphinxAtStartPar
precipitation

\item {} 
\sphinxAtStartPar
evapotranspiration

\item {} 
\sphinxAtStartPar
surface runoff

\item {} 
\sphinxAtStartPar
subsurface runoff

\end{itemize}

\sphinxAtStartPar
Among the above components, quantification of evapotranspiration and subsurface runoff have very high level of uncertainties.

\noindent{\hspace*{\fill}\sphinxincludegraphics[height=500\sphinxpxdimen]{{L01_f_9}.png}\hspace*{\fill}}


\section{Example: Estimation of Subsurface Runoff}
\label{\detokenize{content/background/03_basic_hydrogeology:example-estimation-of-subsurface-runoff}}
\sphinxAtStartPar
Most numbers used in the example do not refer to the catchment shown before!

\sphinxAtStartPar
To calculate the following four\sphinxhyphen{}steps are to be followed:
\begin{itemize}
\item {} 
\sphinxAtStartPar
Step 1: determine rate of inflow in m³/a

\item {} 
\sphinxAtStartPar
step 2: determine rate of outflow due to evapotranspiration (ET.A) in m³/a

\item {} 
\sphinxAtStartPar
Step 3: express rate of outflow due to surface runoff in m³/a

\item {} 
\sphinxAtStartPar
step 4: determine rate of outflow due to subsurface runoff

\end{itemize}

\sphinxAtStartPar
An Example:
For given data, determine the rate of outflow Qout,sub due to subsurface runoff for steady\sphinxhyphen{}state conditions

\begin{sphinxuseclass}{cell}\begin{sphinxVerbatimInput}

\begin{sphinxuseclass}{cell_input}
\begin{sphinxVerbatim}[commandchars=\\\{\}]
\PYG{n}{A} \PYG{o}{=} \PYG{l+m+mi}{4500} \PYG{c+c1}{\PYGZsh{} km², catchment area}
\PYG{n}{P} \PYG{o}{=} \PYG{l+m+mi}{550} \PYG{c+c1}{\PYGZsh{} mm/a, precipitation}
\PYG{n}{ET} \PYG{o}{=} \PYG{l+m+mi}{200} \PYG{c+c1}{\PYGZsh{} mm/a, evapotranspiration}
\PYG{n}{Qout\PYGZus{}surf} \PYG{o}{=} \PYG{l+m+mi}{40} \PYG{c+c1}{\PYGZsh{} m³/s, surface runoff}
\PYG{n}{Delta\PYGZus{}V} \PYG{o}{=} \PYG{l+m+mi}{0} \PYG{c+c1}{\PYGZsh{} m³, change in volume = 0 Steady\PYGZhy{}state conditions}

\PYG{c+c1}{\PYGZsh{}Volume budget in this example: P·A = ET·A + Qout,surf + Qout,sub}

\PYG{c+c1}{\PYGZsh{}Step 1    }
\PYG{n}{Qin} \PYG{o}{=} \PYG{n}{P}\PYG{o}{*}\PYG{n}{A}\PYG{o}{*}\PYG{l+m+mi}{10}\PYG{o}{*}\PYG{o}{*}\PYG{l+m+mi}{3}  \PYG{c+c1}{\PYGZsh{}m³/a, 10\PYGZca{}3 for unit conversion}

\PYG{c+c1}{\PYGZsh{}step 2: }
\PYG{n}{ET\PYGZus{}A} \PYG{o}{=} \PYG{n}{ET}\PYG{o}{*}\PYG{n}{A}\PYG{o}{*}\PYG{l+m+mi}{10}\PYG{o}{*}\PYG{o}{*}\PYG{l+m+mi}{3} \PYG{c+c1}{\PYGZsh{}m³/a, 10\PYGZca{}3 for unit conversion}

\PYG{c+c1}{\PYGZsh{}Step 3: }
\PYG{n}{Qout\PYGZus{}surf} \PYG{o}{=} \PYG{n}{Qout\PYGZus{}surf} \PYG{o}{*}\PYG{l+m+mi}{365}\PYG{o}{*}\PYG{l+m+mi}{24}\PYG{o}{*}\PYG{l+m+mi}{3600} \PYG{c+c1}{\PYGZsh{}  m³/a}

\PYG{c+c1}{\PYGZsh{} step 4:}
\PYG{n}{Qout\PYGZus{}sub} \PYG{o}{=} \PYG{n}{Qin} \PYG{o}{\PYGZhy{}} \PYG{n}{ET\PYGZus{}A} \PYG{o}{\PYGZhy{}} \PYG{n}{Qout\PYGZus{}surf} \PYG{c+c1}{\PYGZsh{} m³/a }



\PYG{n+nb}{print}\PYG{p}{(}\PYG{l+s+s2}{\PYGZdq{}}\PYG{l+s+s2}{The rate of inflow, Qin is }\PYG{l+s+si}{\PYGZob{}0:1.1E\PYGZcb{}}\PYG{l+s+s2}{\PYGZdq{}}\PYG{o}{.}\PYG{n}{format}\PYG{p}{(}\PYG{n}{Qin}\PYG{p}{)}\PYG{p}{,}\PYG{l+s+s2}{\PYGZdq{}}\PYG{l+s+s2}{m}\PYG{l+s+se}{\PYGZbs{}u00b3}\PYG{l+s+s2}{/a }\PYG{l+s+se}{\PYGZbs{}n}\PYG{l+s+s2}{\PYGZdq{}}\PYG{p}{)}\PYG{p}{;} \PYG{n+nb}{print}\PYG{p}{(}\PYG{l+s+s2}{\PYGZdq{}}\PYG{l+s+s2}{The outflow rate due to Evapotranspiration is }\PYG{l+s+si}{\PYGZob{}0:1.1E\PYGZcb{}}\PYG{l+s+s2}{\PYGZdq{}}\PYG{o}{.}\PYG{n}{format}\PYG{p}{(}\PYG{n}{ET\PYGZus{}A}\PYG{p}{)}\PYG{p}{,}\PYG{l+s+s2}{\PYGZdq{}}\PYG{l+s+s2}{m}\PYG{l+s+se}{\PYGZbs{}u00b3}\PYG{l+s+s2}{/a }\PYG{l+s+se}{\PYGZbs{}n}\PYG{l+s+s2}{\PYGZdq{}}\PYG{p}{)}
\PYG{n+nb}{print}\PYG{p}{(}\PYG{l+s+s2}{\PYGZdq{}}\PYG{l+s+s2}{The surface outflow rate, Q\PYGZus{}out\PYGZus{}surf in m}\PYG{l+s+se}{\PYGZbs{}u00b3}\PYG{l+s+s2}{/a is }\PYG{l+s+si}{\PYGZob{}0:1.1E\PYGZcb{}}\PYG{l+s+s2}{\PYGZdq{}}\PYG{o}{.}\PYG{n}{format}\PYG{p}{(}\PYG{n}{Qout\PYGZus{}surf}\PYG{p}{)}\PYG{p}{,}\PYG{l+s+s2}{\PYGZdq{}}\PYG{l+s+s2}{m}\PYG{l+s+se}{\PYGZbs{}u00b3}\PYG{l+s+s2}{/a }\PYG{l+s+se}{\PYGZbs{}n}\PYG{l+s+s2}{\PYGZdq{}}\PYG{p}{)}\PYG{p}{;}\PYG{n+nb}{print}\PYG{p}{(}\PYG{l+s+s2}{\PYGZdq{}}\PYG{l+s+s2}{The subsurface outflow rate, Qout\PYGZus{}surf in m}\PYG{l+s+se}{\PYGZbs{}u00b3}\PYG{l+s+s2}{/a is }\PYG{l+s+si}{\PYGZob{}0:1.1E\PYGZcb{}}\PYG{l+s+s2}{\PYGZdq{}}\PYG{o}{.}\PYG{n}{format}\PYG{p}{(}\PYG{n}{Qout\PYGZus{}sub}\PYG{p}{)}\PYG{p}{,}\PYG{l+s+s2}{\PYGZdq{}}\PYG{l+s+s2}{m}\PYG{l+s+se}{\PYGZbs{}u00b3}\PYG{l+s+s2}{/a }\PYG{l+s+se}{\PYGZbs{}n}\PYG{l+s+s2}{\PYGZdq{}}\PYG{p}{)}
\end{sphinxVerbatim}

\end{sphinxuseclass}\end{sphinxVerbatimInput}
\begin{sphinxVerbatimOutput}

\begin{sphinxuseclass}{cell_output}
\begin{sphinxVerbatim}[commandchars=\\\{\}]
The rate of inflow, Qin is 2.5E+09 m³/a 

The outflow rate due to Evapotranspiration is 9.0E+08 m³/a 

The surface outflow rate, Q\PYGZus{}out\PYGZus{}surf in m³/a is 1.3E+09 m³/a 

The subsurface outflow rate, Qout\PYGZus{}surf in m³/a is 3.1E+08 m³/a 
\end{sphinxVerbatim}

\end{sphinxuseclass}\end{sphinxVerbatimOutput}

\end{sphinxuseclass}

\section{Mass Budget}
\label{\detokenize{content/background/03_basic_hydrogeology:mass-budget}}
\sphinxAtStartPar
The \sphinxstylestrong{mass budget} is quantified similar to the \sphinxstyleemphasis{volume budget.} Mathematically, the \sphinxstyleemphasis{mass budget} is:
\begin{equation*}
\begin{split}\Delta M = J_{in}\cdot \Delta t - J_{out} \cdot \Delta t\end{split}
\end{equation*}
\sphinxAtStartPar
with 
\(\Delta t\) = time interval {[}T{]}
\(\Delta M\) = change of mass in the system {[}M{]}
\(J_{in}\) = rate of mass flow into the system {[}M/T{]}
\(J_{out}\) = rate of mass flow out of the system {[}M/T{]}

\sphinxAtStartPar
Similar to \sphinxstyleemphasis{volume budget,} the following points have to be considered in quantifying mass budget:
\begin{itemize}
\item {} 
\sphinxAtStartPar
Inflow and outflow may each consist of several individual components.

\item {} 
\sphinxAtStartPar
\(\Delta M\) = 0 (no change in mass) is tantamount to steady\sphinxhyphen{}state or stationary (= time\sphinxhyphen{}independent) conditions.

\item {} 
\sphinxAtStartPar
For steady\sphinxhyphen{}state conditions we have: \(J_{in}\)= \(J_{out}\)

\end{itemize}


\section{Example of Mass Budget: Radioactive Decay}
\label{\detokenize{content/background/03_basic_hydrogeology:example-of-mass-budget-radioactive-decay}}
\sphinxAtStartPar
Consider a decay chain comprising of the three chemicals: \sphinxstylestrong{A}, \sphinxstylestrong{B} and \sphinxstylestrong{C}
\begin{itemize}
\item {} 
\sphinxAtStartPar
decay chain: A \(\rightarrow\) B \(\rightarrow\) C       

\item {} 
\sphinxAtStartPar
30\% of \(\text{A}\) and 20\% of \(\text{B}\)  decay each year.

\item {} 
\sphinxAtStartPar
decay rate of \(\text{A}\)   = production rate of \(\text{B}\)   = \(0.3 \cdot a^{-1}\cdot M_A\) 

\item {} 
\sphinxAtStartPar
decay rate of \(\text{B}\) = production rate of \(\text{C}\) = \(0.2\cdot a^{-1}\cdot M_B\) 

\item {} 
\sphinxAtStartPar
mass budgets for \(\text{A}\), \(\text{B}\) and \(\text{C}\):

\end{itemize}

\sphinxAtStartPar
\textbackslash{}begin\{equation*\}
\textbackslash{}begin\{split\}
\textbackslash{}Delta M\_A \&= 0.3 \textbackslash{}text\{ a \(^{-1}\) \} \textbackslash{}cdot M\_A  \textbackslash{}cdot \textbackslash{}Delta t  \textbackslash{}
\textbackslash{}Delta M\_B  \&= 0.3 \textbackslash{}text\{a\(^{-1}\)\} \textbackslash{}cdot M\_A  \textbackslash{}cdot \textbackslash{}Delta t \sphinxhyphen{} 0.2 \textbackslash{}text\{ a\(^{-1}\)\} \textbackslash{}cdot M\_B  \textbackslash{}cdot \textbackslash{}Delta t \textbackslash{}
\textbackslash{}Delta M\_C \&= 0.2 \textbackslash{}text\{a\(^{-1}\)\} \textbackslash{}cdot M\_B  \textbackslash{}cdot \textbackslash{}Delta t
\textbackslash{}end\{split\}
\textbackslash{}end\{equation*\}
\begin{itemize}
\item {} 
\sphinxAtStartPar
Similar equations hold for quantitative descriptions of some chemical reactions which correspond to the type A \(\rightarrow\) B \(\rightarrow\) C

\end{itemize}

\begin{sphinxuseclass}{cell}
\begin{sphinxuseclass}{tag_hide-input}
\begin{sphinxuseclass}{tag_thebe-init}\begin{sphinxVerbatimOutput}

\begin{sphinxuseclass}{cell_output}
\begin{sphinxVerbatim}[commandchars=\\\{\}]
interactive(children=(BoundedIntText(value=20, description=\PYGZsq{}\PYGZam{}Delta; t (day)\PYGZsq{}), BoundedFloatText(value=100.0, d…
\end{sphinxVerbatim}

\end{sphinxuseclass}\end{sphinxVerbatimOutput}

\end{sphinxuseclass}
\end{sphinxuseclass}
\end{sphinxuseclass}

\section{Comparison of Mass and Volume Budgets}
\label{\detokenize{content/background/03_basic_hydrogeology:comparison-of-mass-and-volume-budgets}}
\sphinxAtStartPar
\sphinxstylestrong{mass budget}:	\(\Delta M = J_{in} \cdot \Delta t - J_{out} \cdot \Delta t\)

\sphinxAtStartPar
\sphinxstylestrong{volume budget}:	\(\Delta V = Q_{in} \cdot \Delta t - Q_{out} \cdot \Delta t \)
\begin{itemize}
\item {} 
\sphinxAtStartPar
Mass and volume budgets are equivalent if there is no change of density \(\rho\) {[}M/L\(^3\){]} with time. In this case the well known relationship \(\Delta M\) = \(\rho \cdot \Delta V\) holds and each equation given above can be directly transformed into the other one.

\item {} 
\sphinxAtStartPar
If density changes have to be considered (e.g. for gas flow), the mass budget equation remains valid but the volume budget equation must be modified because \(\Delta M = \rho \cdot \Delta V + \Delta \rho \cdot V\) with \(\Delta \rho\)= change in density.

\item {} 
\sphinxAtStartPar
Cases with changing density have proven to be more easily tractable if the mass budget equation is used.

\end{itemize}


\section{Chapter Quiz}
\label{\detokenize{content/background/03_basic_hydrogeology:chapter-quiz}}
\begin{sphinxuseclass}{cell}
\begin{sphinxuseclass}{tag_hide-input}
\begin{sphinxuseclass}{tag_remove-input}
\begin{sphinxuseclass}{tag_hide-output}
\end{sphinxuseclass}
\end{sphinxuseclass}
\end{sphinxuseclass}
\end{sphinxuseclass}
\sphinxstepscope


\part{Flow}

\sphinxstepscope

\begin{sphinxuseclass}{cell}
\begin{sphinxuseclass}{tag_hide-input}
\begin{sphinxuseclass}{tag_remove-input}
\end{sphinxuseclass}
\end{sphinxuseclass}
\end{sphinxuseclass}

\chapter{Subsurface Structure}
\label{\detokenize{content/flow/12_subsurface_structure:subsurface-structure}}\label{\detokenize{content/flow/12_subsurface_structure::doc}}
\sphinxAtStartPar
\sphinxstyleemphasis{(The contents presented in this section were re\sphinxhyphen{}developed principally by M.Sc. Hanieh Mehrdad and Dr. P. K. Yadav. The original contents are from Prof. Rudolf Liedl)}


\bigskip\hrule\bigskip



\section{Porous Media}
\label{\detokenize{content/flow/12_subsurface_structure:porous-media}}
\sphinxAtStartPar
The general definition of the porous media is a \sphinxstylestrong{solid which contains voids}. This definition applies to the subsurface contains solid material plus voids which represent storage and transmission of the water. The voids may have various shapes and contain fluids (mostly air and/or water). Moreover, voids may be connected to or disconnected from each other.
Generally voids and their properties are important to determine water storage (how much water is or could be available?) and water transmission (How fast the water can move?).


\section{Types of porous media in the subsurface}
\label{\detokenize{content/flow/12_subsurface_structure:types-of-porous-media-in-the-subsurface}}
\noindent{\hspace*{\fill}\sphinxincludegraphics[height=200\sphinxpxdimen]{{L02_fig1}.png}\hspace*{\fill}}
\begin{enumerate}
\sphinxsetlistlabels{\arabic}{enumi}{enumii}{}{.}%
\item {} 
\sphinxAtStartPar
\sphinxstylestrong{Unconsolidated porous medium (Sediments)}: it is non\sphinxhyphen{}cemented porous media and the grains can be taken away. The formation of such porous media is due to deposition of solid material mostly by water.

\end{enumerate}

\noindent{\hspace*{\fill}\sphinxincludegraphics[width=500\sphinxpxdimen]{{L02_fig3}.png}\hspace*{\fill}}
\begin{enumerate}
\sphinxsetlistlabels{\arabic}{enumi}{enumii}{}{.}%
\item {} 
\sphinxAtStartPar
\sphinxstylestrong{Consolidated porous medium (Rocks)}: the formation is due to increased pressure acting together with thermal and chemical processes. It has two types:

\end{enumerate}
\begin{quote}

\sphinxAtStartPar
Fractured porous media
\end{quote}
\begin{quote}

\sphinxAtStartPar
Karstified porous media
\end{quote}

\noindent{\hspace*{\fill}\sphinxincludegraphics[width=600\sphinxpxdimen]{{L02_fig4}.png}\hspace*{\fill}}


\section{Porosity (Total porosity):}
\label{\detokenize{content/flow/12_subsurface_structure:porosity-total-porosity}}
\sphinxAtStartPar
Is defined as the volumetric share of voids in a porous media. It is a number between 0 and 1 and can be expressed as percentage (0\%: no voids, 100\%:no solid)
\begin{equation*}
\begin{split}{n}=\frac{V_{v}}{V_{T}}\end{split}
\end{equation*}\begin{itemize}
\item {} 
\sphinxAtStartPar
\(n\)= total porosity

\item {} 
\sphinxAtStartPar
\({V_{v}}\)= voids volume

\item {} 
\sphinxAtStartPar
\({V_{T}}\)= total volume

\end{itemize}


\subsection{Example Problem}
\label{\detokenize{content/flow/12_subsurface_structure:example-problem}}
\sphinxAtStartPar
If the total volume of a media is 254 cubic meters, and the volume of the void is 27 cubic meters, what is the porosity (give as a percent)?

\begin{sphinxuseclass}{cell}\begin{sphinxVerbatimInput}

\begin{sphinxuseclass}{cell_input}
\begin{sphinxVerbatim}[commandchars=\\\{\}]
\PYG{c+c1}{\PYGZsh{} input data}
\PYG{n}{V\PYGZus{}T}\PYG{o}{=} \PYG{l+m+mi}{254} \PYG{c+c1}{\PYGZsh{}m\PYGZca{}3 total volume}
\PYG{n}{V\PYGZus{}v}\PYG{o}{=}\PYG{l+m+mi}{27} \PYG{c+c1}{\PYGZsh{}m\PYGZca{}3 voids volume}

\PYG{c+c1}{\PYGZsh{}calculations}
\PYG{n}{n}\PYG{o}{=}\PYG{p}{(}\PYG{n}{V\PYGZus{}v}\PYG{o}{/}\PYG{n}{V\PYGZus{}T}\PYG{p}{)}\PYG{o}{*}\PYG{l+m+mi}{100}

\PYG{c+c1}{\PYGZsh{} Output}
\PYG{n+nb}{print}\PYG{p}{(}\PYG{l+s+s2}{\PYGZdq{}}\PYG{l+s+s2}{ Total porosity is: }\PYG{l+s+si}{\PYGZob{}0:0.2f\PYGZcb{}}\PYG{l+s+s2}{\PYGZpc{}}\PYG{l+s+s2}{\PYGZdq{}}\PYG{o}{.}\PYG{n}{format}\PYG{p}{(}\PYG{n}{n}\PYG{p}{)} \PYG{p}{)}
\end{sphinxVerbatim}

\end{sphinxuseclass}\end{sphinxVerbatimInput}
\begin{sphinxVerbatimOutput}

\begin{sphinxuseclass}{cell_output}
\begin{sphinxVerbatim}[commandchars=\\\{\}]
 Total porosity is: 10.63\PYGZpc{}
\end{sphinxVerbatim}

\end{sphinxuseclass}\end{sphinxVerbatimOutput}

\end{sphinxuseclass}

\section{Total porosity of artificial porous media:}
\label{\detokenize{content/flow/12_subsurface_structure:total-porosity-of-artificial-porous-media}}
\sphinxAtStartPar
If “grains” have identical shape and are regularly arranged, it is possible to exactly compute total porosity, the pores should have the same size.

\noindent{\hspace*{\fill}\sphinxincludegraphics[width=300\sphinxpxdimen]{{L02_fig5}.png}\hspace*{\fill}}
\begin{itemize}
\item {} 
\sphinxAtStartPar
Loose packing(first picture): each hole placed on top of the hole underneath

\item {} 
\sphinxAtStartPar
Dense packing (second picture): each hole is placed at the deepest position possible

\end{itemize}

\sphinxAtStartPar
These schematics provides a practical range of porosity in the subsurface. The general range is between 25\% to about 50\%. In more extreme cases porosity higher than 60\% is possible, e.g., cobbles, gravel. The other extreme, subsurface with no porosity (0\%) is also encountered in the subsurface, e.g., in consolidated rocks.


\section{Total porosity of natural (unconsolidated) porous media:}
\label{\detokenize{content/flow/12_subsurface_structure:total-porosity-of-natural-unconsolidated-porous-media}}
\sphinxAtStartPar
Natural unconsolidated porous media consist of grains of different size. Total porosity depends on the grain size distribution.



\noindent{\hspace*{\fill}\sphinxincludegraphics[width=600\sphinxpxdimen]{{L02_fig5b}.png}\hspace*{\fill}}



\sphinxAtStartPar
In general, well sorted unconsolidated porous media exhibit larger total poerosities than poorly sorted unconsolidated porous media. In the figure above on the left in which grain diameters cover a small range, i.e., \sphinxstylestrong{well\sphinxhyphen{}sorted}, the porosity can be approximated in the range 32\%. Similary, in the (above) figure on the right in which the grain diameters cover a large range, i.e., \sphinxstylestrong{poorly sorted}, the porosity can be approximated in the range 17\%.


\section{Typical porosity values:}
\label{\detokenize{content/flow/12_subsurface_structure:typical-porosity-values}}
\sphinxAtStartPar
Table below provide the total porosity of unconsolidated and consolidated media.

\sphinxAtStartPar
Total porosity of consolidated porous media (rocks) is usually smaller than total porosity of unconsolidated porous media. However, weathering effect may lead to increase the value of porosity.
only for unconsolidated porous media, total porosity tends to increase with decreasing grain size.


\section{Grain size distribution of unconsolidated porous media}
\label{\detokenize{content/flow/12_subsurface_structure:grain-size-distribution-of-unconsolidated-porous-media}}
\sphinxAtStartPar
Unconsolidated porous media are able to store and transmit water that can be influenced by grain size distribution. Therefore, the grain size distribution is frequently determined in laboratory experiments in order to deduce important flow properties.
There are five major grain size classes (observed by increasing diameter): clay, silt, sand, and gravel (or debris). The classes for silt, clay and gavel are usually subdivided by “fine”, “medium”, and “coarse” (or “very fine”, “fine”, “medium”, “coarse”, and “very coarse”). Different ranges for individual grain size classes have been defined by different authorities or regulations. However, the standard method to determine the grain size distribution of a sample is sieve analysis.


\section{Classification schemes:}
\label{\detokenize{content/flow/12_subsurface_structure:classification-schemes}}
\sphinxAtStartPar
The diagrams below include a couple of classification schemes to define ranges of grain diameter for clay, silt, sand, and gravel:

\noindent{\hspace*{\fill}\sphinxincludegraphics[width=600\sphinxpxdimen]{{L02_fig7}.png}\hspace*{\fill}}

\sphinxAtStartPar
As can be observed that there exist several standards. These are often based on local requirements e.g., based on countries. In Germany the DIN standards are used.

\begin{sphinxadmonition}{note}{Click for the abbreviation}

\sphinxAtStartPar
\sphinxstyleemphasis{USDA: United States Department of Agriculture},
\sphinxstyleemphasis{ISSS: International Soil Science Society (ISSS)},
\sphinxstyleemphasis{MIT: Massachusetts Institute of Technology},
\sphinxstyleemphasis{ASTM: American Society for Testing and Materials},
\sphinxstyleemphasis{AASHTO: American Association of State Highway and Transportation Officials},
\sphinxstyleemphasis{FAA: Federal Aviation Administration}
\end{sphinxadmonition}


\section{Sieve analysis:}
\label{\detokenize{content/flow/12_subsurface_structure:sieve-analysis}}
\sphinxAtStartPar
The results from a sample consist of different grain size fractions should be transferred on granulometric curve. This curve provides cumulative information; vertical axis shows the mass fraction, and horizontal axis shows the grain diameter. For example, if 1mm grain diameter has 80\% of cumulative mass fraction it means that 80\% of this sample contains 1mm grain diameter or less than 1 mm (see the picture below).

\sphinxAtStartPar
\sphinxstylestrong{How to get granulometric curve?}
In order to perform sieve analysis we can use sieve machine. Sieve machine consist of sets of sieves from coarse sieve on top to fine sieve and a cup at the bottom. The mechanism is to shake the set. Finally, each sieve consists of grain sizes which are bigger than the sieve.

\noindent{\hspace*{\fill}\sphinxincludegraphics[width=600\sphinxpxdimen]{{L02_fig15}.png}\hspace*{\fill}}

\noindent{\hspace*{\fill}\sphinxincludegraphics[width=600\sphinxpxdimen]{{L02_fig8}.png}\hspace*{\fill}}

\sphinxAtStartPar
\sphinxstylestrong{dx and U}: From the granulometric curve, several parameters can be determined in order to characterize the sample. \({d_{x}}\)   denotes the grain diameter for which x\% (in mass or weight, not volume) of the sieve material is smaller than this diameter.

\noindent{\hspace*{\fill}\sphinxincludegraphics[width=400\sphinxpxdimen]{{L02_fig9}.png}\hspace*{\fill}}

\sphinxAtStartPar
Grain diameters \({d_{10}}, {d_{60}}, {d_{75}}\) are of practical importance with regard to groundwater flow properties. The     ratio of d60 and d10 is called \sphinxstylestrong{coefficient of uniformity, U}:
\begin{equation*}
\begin{split}{U}=\frac{d_{60}}{d_{10}}\end{split}
\end{equation*}
\sphinxAtStartPar
\({d_{75}}\) is specifically used for well construction purpose (not covered by this lecture)


\section{Subterranean water}
\label{\detokenize{content/flow/12_subsurface_structure:subterranean-water}}
\sphinxAtStartPar
The subsurface can be regarded as a three\sphinxhyphen{}phase system consisting of a solid phase (soil particles), a water phase, and a gas phase. a schematic illustration for voids or pores in an unconsolidated porous medium is given in the figure below. Each phase has similar density and other properties. Sometimes it is possible for the fourth phase which is contamination.Voids are filled with water and gas. The volumetric ratio of water in voids can be calculated by water content.

\noindent{\hspace*{\fill}\sphinxincludegraphics[width=400\sphinxpxdimen]{{L02_fig10}.png}\hspace*{\fill}}


\section{Water content:}
\label{\detokenize{content/flow/12_subsurface_structure:water-content}}
\sphinxAtStartPar
Water content is defined as the share of water in the porous medium:
\begin{equation*}
\begin{split} {\theta}=\frac{V_{w}}{V_{T}} \end{split}
\end{equation*}\begin{itemize}
\item {} 
\sphinxAtStartPar
\( {\theta}\) = water content

\item {} 
\sphinxAtStartPar
\( {V_{w}}\) = water volume

\item {} 
\sphinxAtStartPar
\( {V_{T}} \) = total volume

\end{itemize}

\sphinxAtStartPar
Water content cannot exceed the total porosity. i.e. θ≤n ( total porosity is independent of the fluid content of porous medium).


\section{Degree of saturation:}
\label{\detokenize{content/flow/12_subsurface_structure:degree-of-saturation}}
\sphinxAtStartPar
Another way to express the ratio of water in the porous medium is the degree of saturation, i.e. the ratio of water volume to void volume:
\begin{equation*}
\begin{split}{S}=\frac{V_{w}}{V_{v}}\end{split}
\end{equation*}\begin{itemize}
\item {} 
\sphinxAtStartPar
\({S}\)= degree of saturation

\item {} 
\sphinxAtStartPar
\({V_{w}}\)= water volume

\item {} 
\sphinxAtStartPar
\({V_{v}}\)= voids volume

\end{itemize}

\sphinxAtStartPar
The degree of saturation is equal to \(\frac{\theta}{n}\) . \(S\) can vary between 0 to 1 (or between 0\% to 100\%), in which \(S=0\) means no water in the voids, whereas \(S=100\) means voids are completely filled with water.


\subsection{Example Problem}
\label{\detokenize{content/flow/12_subsurface_structure:id7}}
\sphinxAtStartPar
The voids volume and the total air in the subsurface sample (0.2 m\(^3\)) was found to be 0.02 m\(^3\) and 0.001 m\(^3\), respectively. How much water does the sample contain and what is the degree of saturation?

\begin{sphinxuseclass}{cell}\begin{sphinxVerbatimInput}

\begin{sphinxuseclass}{cell_input}
\begin{sphinxVerbatim}[commandchars=\\\{\}]
\PYG{c+c1}{\PYGZsh{} solution}

\PYG{n}{V\PYGZus{}T} \PYG{o}{=} \PYG{l+m+mf}{0.2} \PYG{c+c1}{\PYGZsh{} m\PYGZca{}3, Total sample.}
\PYG{n}{V\PYGZus{}v} \PYG{o}{=} \PYG{l+m+mf}{0.06} \PYG{c+c1}{\PYGZsh{} m\PYGZca{}3, volume of voids}
\PYG{n}{V\PYGZus{}a} \PYG{o}{=} \PYG{l+m+mf}{0.004} \PYG{c+c1}{\PYGZsh{} m\PYGZca{}3, volume of air}

\PYG{c+c1}{\PYGZsh{} interim calculation}
\PYG{n}{V\PYGZus{}w} \PYG{o}{=} \PYG{n}{V\PYGZus{}v} \PYG{o}{\PYGZhy{}} \PYG{n}{V\PYGZus{}a} \PYG{c+c1}{\PYGZsh{} m\PYGZca{}3, Vol. water,  the remaining volume}

\PYG{c+c1}{\PYGZsh{} calculation}

\PYG{n}{Theta} \PYG{o}{=} \PYG{n}{V\PYGZus{}w}\PYG{o}{/}\PYG{n}{V\PYGZus{}T} \PYG{c+c1}{\PYGZsh{} \PYGZhy{}, water content}
\PYG{n}{S} \PYG{o}{=} \PYG{n}{V\PYGZus{}w}\PYG{o}{/}\PYG{n}{V\PYGZus{}v} \PYG{c+c1}{\PYGZsh{} \PYGZsh{} \PYGZhy{}, degree of saturation}

\PYG{c+c1}{\PYGZsh{} print}
\PYG{n+nb}{print}\PYG{p}{(}\PYG{l+s+s2}{\PYGZdq{}}\PYG{l+s+s2}{The water content of the sample is: }\PYG{l+s+si}{\PYGZob{}0:.0\PYGZpc{}\PYGZcb{}}\PYG{l+s+s2}{\PYGZdq{}}\PYG{o}{.}\PYG{n}{format}\PYG{p}{(}\PYG{n}{Theta}\PYG{p}{)}\PYG{p}{,}\PYG{l+s+s2}{\PYGZdq{}}\PYG{l+s+se}{\PYGZbs{}n}\PYG{l+s+s2}{\PYGZdq{}}\PYG{p}{)}
\PYG{n+nb}{print}\PYG{p}{(}\PYG{l+s+s2}{\PYGZdq{}}\PYG{l+s+s2}{The degree of saturation of the sample is: }\PYG{l+s+si}{\PYGZob{}0:.0\PYGZpc{}\PYGZcb{}}\PYG{l+s+s2}{\PYGZdq{}}\PYG{o}{.}\PYG{n}{format}\PYG{p}{(}\PYG{n}{S}\PYG{p}{)}\PYG{p}{)}
\end{sphinxVerbatim}

\end{sphinxuseclass}\end{sphinxVerbatimInput}
\begin{sphinxVerbatimOutput}

\begin{sphinxuseclass}{cell_output}
\begin{sphinxVerbatim}[commandchars=\\\{\}]
The water content of the sample is: 28\PYGZpc{} 

The degree of saturation of the sample is: 93\PYGZpc{}
\end{sphinxVerbatim}

\end{sphinxuseclass}\end{sphinxVerbatimOutput}

\end{sphinxuseclass}

\section{Forces acting on subterranean water:}
\label{\detokenize{content/flow/12_subsurface_structure:forces-acting-on-subterranean-water}}
\sphinxAtStartPar
Subterranean water is subject to several forces. The most important ones are:
\begin{itemize}
\item {} 
\sphinxAtStartPar
gravity

\item {} 
\sphinxAtStartPar
attractive forces between the water molecules (cohesion)

\item {} 
\sphinxAtStartPar
attractive forces between water and solids (adhesion)

\end{itemize}

\noindent{\hspace*{\fill}\sphinxincludegraphics[width=400\sphinxpxdimen]{{L02_fig11}.png}\hspace*{\fill}}

\sphinxAtStartPar
In the figure above, dotted area represent the solid phase. In the pore channel the dominant force is gravity, shown as G. getting closer to the solid surface, adhesive force become more important. The numbers indicate the required pressure to remove the corresponding layer of water from the solid surface. As an example, in order to remove the last layer of water from the solid surface, 31 bar pressure needs to be applied. Another easy way to remove the water is boiling the sample in the oven.


\section{Surface tension:}
\label{\detokenize{content/flow/12_subsurface_structure:surface-tension}}
\sphinxAtStartPar
Cohesive forces acting on water molecules compensate each other if the molecule is not located near water\sphinxhyphen{}air or water\sphinxhyphen{}solid interface. This is no longer true at an interface: cohesive interaction is reduced on one side. The resulting force tends to minimize the interface area. Macroscopically, this effect is parametrized by the “surface tension”, which is defined as the energy needed to increase the area of the interface by one unit.

\noindent{\hspace*{\fill}\sphinxincludegraphics[width=300\sphinxpxdimen]{{L02_fig12}.png}\hspace*{\fill}}

\sphinxAtStartPar
Common units of the surface tension σ are \(\frac{J}{m^2}\) or \(\frac{N}{m}\) (Its dimension is \(\frac{M}{T^2}\)). The surface tension of water is about 7.5 . 10 \sphinxhyphen{}2 \(\frac{N}{m}\) at 10 ֯C.


\section{Capillary action:}
\label{\detokenize{content/flow/12_subsurface_structure:capillary-action}}
\noindent{\hspace*{\fill}\sphinxincludegraphics[width=200\sphinxpxdimen]{{L02_fig13}.png}\hspace*{\fill}}

\sphinxAtStartPar
Water is subject to capillary action when adhesion is strongr than cohesion. The capillary rise of water in a tube depends on the surface tension and the tube redius. The maximum capillary rise is given by:
\begin{equation*}
\begin{split}{h_{c}}=\frac{2\sigma_{w}}{\rho_{w}{g}{r}}\end{split}
\end{equation*}\begin{itemize}
\item {} 
\sphinxAtStartPar
\({h_{c}}\)= maximum capillary rise

\item {} 
\sphinxAtStartPar
\(\sigma_{w}\)= surface tension

\item {} 
\sphinxAtStartPar
\(\rho_{w}\)= water density

\item {} 
\sphinxAtStartPar
\({g}\)= acceleration of gravity

\item {} 
\sphinxAtStartPar
\({r}\)= radius of the tube

\end{itemize}


\section{Capillary action in the subsurface:}
\label{\detokenize{content/flow/12_subsurface_structure:capillary-action-in-the-subsurface}}
\sphinxAtStartPar
Capillary actions play a dominant role in the subsurface. The capillaries are given by individual pore channels. Poor channels in poorly sorted material may strongly differ in diameter, such that a certain variability in capillary rise is observed.

\noindent{\hspace*{\fill}\sphinxincludegraphics[width=200\sphinxpxdimen]{{L02_fig14}.png}\hspace*{\fill}}

\sphinxAtStartPar
Left sketch shows the capillary rise in a perfectly sorted material which all the pores have the same size. So capillary rise is similar in every single pores. The right sketch, shows a real situation of subsurface. There are different grain size and then different pore channels, which \sphinxstylestrong{results} in various capillary rise.


\subsection{Example Problem}
\label{\detokenize{content/flow/12_subsurface_structure:id10}}
\sphinxAtStartPar
For water at a tube with a radius \sphinxstyleemphasis{R}, the surface tension is 73 \(\frac{g}{s^2}\), the density is 0.999 \(\frac{g}{cm^3}\). Compute the rise of water in the capillary tube

\begin{sphinxuseclass}{cell}\begin{sphinxVerbatimInput}

\begin{sphinxuseclass}{cell_input}
\begin{sphinxVerbatim}[commandchars=\\\{\}]
\PYG{c+c1}{\PYGZsh{} input data}
\PYG{n}{sigma}\PYG{o}{=} \PYG{l+m+mi}{73} \PYG{c+c1}{\PYGZsh{}g/s\PYGZca{}2 surface tension}
\PYG{n}{rho}\PYG{o}{=} \PYG{l+m+mf}{0.999} \PYG{c+c1}{\PYGZsh{} g/cm\PYGZca{}3 water density}
\PYG{n}{g}\PYG{o}{=}\PYG{l+m+mi}{980} \PYG{c+c1}{\PYGZsh{}cm/s\PYGZca{}2 acceleration of gravity}

\PYG{c+c1}{\PYGZsh{}calculation}
\PYG{n}{h\PYGZus{}c}\PYG{o}{=}\PYG{p}{(}\PYG{l+m+mi}{2}\PYG{o}{*}\PYG{n}{sigma}\PYG{p}{)}\PYG{o}{/}\PYG{p}{(}\PYG{n}{rho}\PYG{o}{*}\PYG{n}{g}\PYG{p}{)}

\PYG{c+c1}{\PYGZsh{}output}
\PYG{n+nb}{print}\PYG{p}{(}\PYG{l+s+s2}{\PYGZdq{}}\PYG{l+s+s2}{The maximum water rise in this tube is: }\PYG{l+s+si}{\PYGZob{}0:0.2f\PYGZcb{}}\PYG{l+s+s2}{ 1/R cm}\PYG{l+s+s2}{\PYGZdq{}}\PYG{o}{.}\PYG{n}{format}\PYG{p}{(}\PYG{n}{h\PYGZus{}c}\PYG{p}{)}\PYG{p}{)}
\end{sphinxVerbatim}

\end{sphinxuseclass}\end{sphinxVerbatimInput}
\begin{sphinxVerbatimOutput}

\begin{sphinxuseclass}{cell_output}
\begin{sphinxVerbatim}[commandchars=\\\{\}]
The maximum water rise in this tube is: 0.15 1/R cm
\end{sphinxVerbatim}

\end{sphinxuseclass}\end{sphinxVerbatimOutput}

\end{sphinxuseclass}

\section{Chapter Quiz}
\label{\detokenize{content/flow/12_subsurface_structure:chapter-quiz}}
\begin{sphinxuseclass}{cell}
\begin{sphinxuseclass}{tag_remove-input}
\begin{sphinxuseclass}{tag_hide-output}
\end{sphinxuseclass}
\end{sphinxuseclass}
\end{sphinxuseclass}
\sphinxstepscope

\begin{sphinxuseclass}{cell}
\begin{sphinxuseclass}{tag_remove-input}
\begin{sphinxuseclass}{tag_hide-output}
\end{sphinxuseclass}
\end{sphinxuseclass}
\end{sphinxuseclass}

\chapter{Groundwater as a reservoir}
\label{\detokenize{content/flow/L3/13_gw_storage:groundwater-as-a-reservoir}}\label{\detokenize{content/flow/L3/13_gw_storage::doc}}
\sphinxAtStartPar
\sphinxstyleemphasis{(The contents presented in this section were re\sphinxhyphen{}developed principally by \sphinxhref{https://www.ufz.de/index.php?de=37303}{Prof. Peter Dietrich} and Dr. P. K. Yadav. The original contents are from Prof. Rudolf Liedl)}


\bigskip\hrule\bigskip


\sphinxAtStartPar
The content of the previous section was dedicated to very fundamental properties, such as aquifer and its types, solid and liquid (water) volumes in an aquifer, a of subsurface.

\sphinxAtStartPar
In this lecture, the subsurface will be considered from the perspective as a groundwater reservoir and some key definition and parameters will be introduced.


\section{Groundwater and Aquifers}
\label{\detokenize{content/flow/L3/13_gw_storage:groundwater-and-aquifers}}
\sphinxAtStartPar
If the subterranean water completely fills the pore space we call it groundwater. In Germany a slightly different definition is in use. There, groundwater is only the subterranean water which is not subject to other forces than gravity (see Fig. below). That means, that the water adhesively bound to the grains is not part of the groundwater. Applicable forces in groundwater are sometime locally defined. For, e.g., In \sphinxstylestrong{Germany} the \sphinxstyleemphasis{gravity} is the only force acting on groundwater, whereas forces such adhesion, cohesion along with gravity are considered internationally.

\noindent{\hspace*{\fill}\sphinxincludegraphics[width=600\sphinxpxdimen]{{L03_f_1}.png}\hspace*{\fill}}

\sphinxAtStartPar
The difference between the international and the German definition of groundwater is the consideration of the adhesive water. Adhesive water does not participate in water movement. The same is true for water in isolated pores or in dead\sphinxhyphen{}end pores. All subterranean water not participating in water movement is summarized as immobile water. In contrast, the mobile water is the subterranean water participating in water movement.

\begin{sphinxuseclass}{cell}
\begin{sphinxuseclass}{tag_remove-input}
\begin{sphinxuseclass}{tag_full-width}\begin{sphinxVerbatimOutput}

\begin{sphinxuseclass}{cell_output}
\begin{sphinxVerbatim}[commandchars=\\\{\}]
Row
    [0] Column
        [0] Markdown(str, style=\PYGZob{}\PYGZsq{}text\PYGZhy{}align\PYGZsq{}: \PYGZsq{}justify\PYGZsq{}\PYGZcb{}, width=400)
        [1] LaTeX(str)
        [2] Markdown(str)
    [1] Spacer(width=50)
    [2] Video(str, height=400, sizing\PYGZus{}mode=\PYGZsq{}fixed\PYGZsq{}, width=600)
\end{sphinxVerbatim}

\end{sphinxuseclass}\end{sphinxVerbatimOutput}

\end{sphinxuseclass}
\end{sphinxuseclass}
\end{sphinxuseclass}
\sphinxAtStartPar
The volumetric share of pore, which can be occupied by mobile water, is termed effective porosity \(n_e\) or flow\sphinxhyphen{}through porosity
\begin{equation*}
\begin{split}
n_e = \frac{V_{p, m}}{V_T}
\end{split}
\end{equation*}
\sphinxAtStartPar
The effective porosity is dimensionless and the pore volume \(V_{p,m}\) available for mobile water as well as the total volume \(V_T\) has the dimension \(L^3\). Effective porosity cannot exceed total porosity, i.e. \(n_e \leq n\). The difference \((n – n_e)\) is termed \sphinxstylestrong{specific retention} or \sphinxstylestrong{field capacity}. Specific retention is the volumetric share of water which is retained in the porous medium after drainage due to gravitation.  The reason for retention is the adhesive force which bounds water at the grain surfaces. Because the available grain surface in a medium depends on the grain size, the effective porosity is also different for various materials. As shown in Figure (below) clay has a high total porosity but only a low effective porosity whereas the total porosity of cobbles is not so significant different from the effective porosity of this material.

\noindent{\hspace*{\fill}\sphinxincludegraphics[height=400\sphinxpxdimen]{{L03_f_5}.png}\hspace*{\fill}}


\subsection{Example Problem}
\label{\detokenize{content/flow/L3/13_gw_storage:example-problem}}
\sphinxAtStartPar
Moist sand specimen = 72.5 cm\(^3\) and its weight = 152 g
Oven dried sample = 71.2 cm\(^3\) and its weight = 145 g

\sphinxAtStartPar
Other available information:
Specific weight of particles \((\gamma_s)\) = 2.65 g/cm\(^3\)
Specific weight of water \((\gamma_w)\) = 1 g/cm\(^3\)

\sphinxAtStartPar
Find, total porosity, void ratio, water content, degree of saturation and effective porosity.

\begin{sphinxuseclass}{cell}\begin{sphinxVerbatimInput}

\begin{sphinxuseclass}{cell_input}
\begin{sphinxVerbatim}[commandchars=\\\{\}]
\PYG{c+c1}{\PYGZsh{} solution}

\PYG{n}{V\PYGZus{}ms} \PYG{o}{=} \PYG{l+m+mf}{72.5} \PYG{c+c1}{\PYGZsh{} cm\PYGZca{}3, volume moist sand}
\PYG{n}{W\PYGZus{}ms} \PYG{o}{=} \PYG{l+m+mi}{152} \PYG{c+c1}{\PYGZsh{} g, weight moist sand, also total volume}
\PYG{n}{V\PYGZus{}ds} \PYG{o}{=} \PYG{l+m+mf}{71.2} \PYG{c+c1}{\PYGZsh{} cm\PYGZca{}3, volume dry sand}
\PYG{n}{W\PYGZus{}ds} \PYG{o}{=} \PYG{l+m+mi}{145} \PYG{c+c1}{\PYGZsh{} g, weight dey sand}

\PYG{c+c1}{\PYGZsh{} Other info}
\PYG{n}{g\PYGZus{}s} \PYG{o}{=} \PYG{l+m+mf}{2.65} \PYG{c+c1}{\PYGZsh{} g/cm\PYGZca{}3, sp. weight, particles}
\PYG{n}{g\PYGZus{}w} \PYG{o}{=} \PYG{l+m+mi}{1} \PYG{c+c1}{\PYGZsh{} g/cm\PYGZca{}3, sp. wt. water}

\PYG{c+c1}{\PYGZsh{} intermediate calculation}
\PYG{n}{V\PYGZus{}w} \PYG{o}{=} \PYG{p}{(}\PYG{n}{W\PYGZus{}ms}\PYG{o}{\PYGZhy{}}\PYG{n}{W\PYGZus{}ds}\PYG{p}{)}\PYG{o}{/}\PYG{n}{g\PYGZus{}w} \PYG{c+c1}{\PYGZsh{} cm\PYGZca{}3,  W\PYGZus{}w/g\PYGZus{}w; density = M/V}
\PYG{n}{V\PYGZus{}s} \PYG{o}{=} \PYG{n}{W\PYGZus{}ds}\PYG{o}{/}\PYG{n}{g\PYGZus{}s} \PYG{c+c1}{\PYGZsh{} cm\PYGZca{}3,}
\PYG{n}{V\PYGZus{}v} \PYG{o}{=} \PYG{n}{V\PYGZus{}ms} \PYG{o}{\PYGZhy{}} \PYG{n}{V\PYGZus{}s} \PYG{c+c1}{\PYGZsh{} cm\PYGZca{}2, volume of voids}
\PYG{n}{W\PYGZus{}w} \PYG{o}{=} \PYG{n}{W\PYGZus{}ms} \PYG{o}{\PYGZhy{}} \PYG{n}{W\PYGZus{}ds} \PYG{c+c1}{\PYGZsh{} g, weight of water}

\PYG{c+c1}{\PYGZsh{} results calculation}

\PYG{n}{n} \PYG{o}{=} \PYG{n}{V\PYGZus{}v}\PYG{o}{/}\PYG{n}{V\PYGZus{}ms}\PYG{o}{*}\PYG{l+m+mi}{100} \PYG{c+c1}{\PYGZsh{} \PYGZpc{}, Total porosity}
\PYG{n}{e} \PYG{o}{=} \PYG{n}{V\PYGZus{}v}\PYG{o}{/}\PYG{n}{V\PYGZus{}s} \PYG{o}{*}\PYG{l+m+mi}{100} \PYG{c+c1}{\PYGZsh{} \PYGZpc{},  void ratio}
\PYG{n}{w} \PYG{o}{=} \PYG{n}{W\PYGZus{}w}\PYG{o}{/}\PYG{n}{W\PYGZus{}ds} \PYG{o}{*}\PYG{l+m+mi}{100} \PYG{c+c1}{\PYGZsh{} \PYGZpc{}, moisture content }
\PYG{n}{S} \PYG{o}{=} \PYG{n}{V\PYGZus{}w}\PYG{o}{/}\PYG{n}{V\PYGZus{}v} \PYG{o}{*} \PYG{l+m+mi}{100} \PYG{c+c1}{\PYGZsh{} \PYGZpc{}, degree of saturation  }

\PYG{n+nb}{print}\PYG{p}{(}\PYG{l+s+s2}{\PYGZdq{}}\PYG{l+s+s2}{Total porosity is }\PYG{l+s+si}{\PYGZob{}0:0.2f\PYGZcb{}}\PYG{l+s+s2}{\PYGZpc{}}\PYG{l+s+s2}{\PYGZdq{}}\PYG{o}{.}\PYG{n}{format}\PYG{p}{(}\PYG{n}{n}\PYG{p}{)}\PYG{p}{)} 
\PYG{n+nb}{print}\PYG{p}{(}\PYG{l+s+s2}{\PYGZdq{}}\PYG{l+s+s2}{Void ratio is }\PYG{l+s+si}{\PYGZob{}0:0.2f\PYGZcb{}}\PYG{l+s+s2}{\PYGZpc{}}\PYG{l+s+s2}{\PYGZdq{}}\PYG{o}{.}\PYG{n}{format}\PYG{p}{(}\PYG{n}{e}\PYG{p}{)}\PYG{p}{)} 
\PYG{n+nb}{print}\PYG{p}{(}\PYG{l+s+s2}{\PYGZdq{}}\PYG{l+s+s2}{Moisture content is }\PYG{l+s+si}{\PYGZob{}0:0.2f\PYGZcb{}}\PYG{l+s+s2}{\PYGZpc{}}\PYG{l+s+s2}{\PYGZdq{}}\PYG{o}{.}\PYG{n}{format}\PYG{p}{(}\PYG{n}{w}\PYG{p}{)}\PYG{p}{)} 
\PYG{n+nb}{print}\PYG{p}{(}\PYG{l+s+s2}{\PYGZdq{}}\PYG{l+s+s2}{Degree of saturation is }\PYG{l+s+si}{\PYGZob{}0:0.2f\PYGZcb{}}\PYG{l+s+s2}{\PYGZpc{}}\PYG{l+s+s2}{\PYGZdq{}}\PYG{o}{.}\PYG{n}{format}\PYG{p}{(}\PYG{n}{S}\PYG{p}{)}\PYG{p}{)}  
\end{sphinxVerbatim}

\end{sphinxuseclass}\end{sphinxVerbatimInput}
\begin{sphinxVerbatimOutput}

\begin{sphinxuseclass}{cell_output}
\begin{sphinxVerbatim}[commandchars=\\\{\}]
Total porosity is 24.53\PYGZpc{}
Void ratio is 32.50\PYGZpc{}
Moisture content is 4.83\PYGZpc{}
Degree of saturation is 39.36\PYGZpc{}
\end{sphinxVerbatim}

\end{sphinxuseclass}\end{sphinxVerbatimOutput}

\end{sphinxuseclass}
\begin{sphinxuseclass}{cell}
\begin{sphinxuseclass}{tag_remove-input}
\begin{sphinxuseclass}{tag_full-width}\begin{sphinxVerbatimOutput}

\begin{sphinxuseclass}{cell_output}
\begin{sphinxVerbatim}[commandchars=\\\{\}]
Row
    [0] Column
        [0] Markdown(str)
        [1] Markdown(str, style=\PYGZob{}\PYGZsq{}text\PYGZhy{}align\PYGZsq{}: \PYGZsq{}justify\PYGZsq{}\PYGZcb{}, width=400)
    [1] Spacer(width=25)
    [2] Video(str, height=200, sizing\PYGZus{}mode=\PYGZsq{}fixed\PYGZsq{}, width=600)
\end{sphinxVerbatim}

\end{sphinxuseclass}\end{sphinxVerbatimOutput}

\end{sphinxuseclass}
\end{sphinxuseclass}
\end{sphinxuseclass}
\begin{sphinxuseclass}{cell}
\begin{sphinxuseclass}{tag_remove-input}
\begin{sphinxuseclass}{tag_full-width}\begin{sphinxVerbatimOutput}

\begin{sphinxuseclass}{cell_output}
\begin{sphinxVerbatim}[commandchars=\\\{\}]
Column
    [0] Markdown(str)
    [1] Markdown(str, style=\PYGZob{}\PYGZsq{}text\PYGZhy{}align\PYGZsq{}: \PYGZsq{}justify\PYGZsq{}\PYGZcb{}, width=600)
    [2] Video(str, height=200, sizing\PYGZus{}mode=\PYGZsq{}fixed\PYGZsq{}, width=400)
    [3] Markdown(str)
    [4] Markdown(str, style=\PYGZob{}\PYGZsq{}text\PYGZhy{}align\PYGZsq{}: \PYGZsq{}justify\PYGZsq{}\PYGZcb{}, width=600)
    [5] Video(str, height=200, sizing\PYGZus{}mode=\PYGZsq{}fixed\PYGZsq{}, width=400)
    [6] Markdown(str)
    [7] Markdown(str, style=\PYGZob{}\PYGZsq{}text\PYGZhy{}align\PYGZsq{}: \PYGZsq{}justify\PYGZsq{}\PYGZcb{}, width=600)
    [8] Video(str, height=200, sizing\PYGZus{}mode=\PYGZsq{}fixed\PYGZsq{}, width=400)
\end{sphinxVerbatim}

\end{sphinxuseclass}\end{sphinxVerbatimOutput}

\end{sphinxuseclass}
\end{sphinxuseclass}
\end{sphinxuseclass}

\subsection{Example}
\label{\detokenize{content/flow/L3/13_gw_storage:example}}

\begin{savenotes}\sphinxattablestart
\centering
\begin{tabulary}{\linewidth}[t]{|T|T|T|T|T|}
\hline
\sphinxstyletheadfamily 
\sphinxAtStartPar
Aquifer
&\sphinxstyletheadfamily 
\sphinxAtStartPar
Obs. Point 1
&\sphinxstyletheadfamily 
\sphinxAtStartPar
Obs. Point 2
&\sphinxstyletheadfamily 
\sphinxAtStartPar
Obs. Point 3
&\sphinxstyletheadfamily 
\sphinxAtStartPar
Obs. Point 4
\\
\hline
\sphinxAtStartPar
A
&
\sphinxAtStartPar
—
&
\sphinxAtStartPar
—
&
\sphinxAtStartPar
Unconfined
&
\sphinxAtStartPar
Unconfined
\\
\hline
\sphinxAtStartPar
B
&
\sphinxAtStartPar
—
&
\sphinxAtStartPar
Unconfined
&
\sphinxAtStartPar
Unconfined
&
\sphinxAtStartPar
Confined
\\
\hline
\sphinxAtStartPar
C
&
\sphinxAtStartPar
Unconfined
&
\sphinxAtStartPar
Unconfined
&
\sphinxAtStartPar
Confined
&
\sphinxAtStartPar
Confined
\\
\hline
\sphinxAtStartPar
D
&
\sphinxAtStartPar
Conf./Artesian
&
\sphinxAtStartPar
Confined
&
\sphinxAtStartPar
Confined
&
\sphinxAtStartPar
Confined
\\
\hline
\end{tabulary}
\par
\sphinxattableend\end{savenotes}

\noindent{\hspace*{\fill}\sphinxincludegraphics[width=600\sphinxpxdimen]{{L03_f_10}.png}\hspace*{\fill}}

\sphinxAtStartPar
\sphinxstyleemphasis{*perched aquifer}: Unconfined aquifer on top of another unconfined aquifer, separated from each other by a shallow aquitard


\section{Pressure and pressure head}
\label{\detokenize{content/flow/L3/13_gw_storage:pressure-and-pressure-head}}
\sphinxAtStartPar
A reason for the movement of groundwater can be (hydrostatic) pressure difference. Let us consider a vertical column containing a porous medium and water filling the voids completely (Fig. below). We can assign to the top an arbitrary reference value pL for the pressure.  Due to the load of the overlaying water column, the pressure increase if we go deeper into the column. This is the same as what we observe if we are diving in a lake. The increase of the pressure depends on the density of the fluid and the depth below the water surface. In the setup given in Fig., we have the (hydrostatic) pressure \(p\) {[}M/L/T\(^2\){]} as a function of the height \(z\) {[}L{]} above the reference point
\begin{equation*}
\begin{split}
p(z) = p_L + \rho \cdot g \cdot (L - z)
\end{split}
\end{equation*}
\begin{sphinxShadowBox}
\sphinxstylesidebartitle{Where, }

\sphinxAtStartPar
\(\rho\) {[}M/L\(^2\){]} as the fluid density, 
\(g\) the acceleration of gravity {[}L/T\(^2\){]}, and  
L {[}L{]} the length of the water column
\end{sphinxShadowBox}

\noindent{\hspace*{\fill}\sphinxincludegraphics[width=600\sphinxpxdimen]{{L03_f_11}.png}\hspace*{\fill}}

\sphinxAtStartPar
As shown in Fig., we can add two observation points, one at the bottom (\(z = 0\)) and the other at the top of the column \((z=L)\). The pressure difference \(\Delta p\) between the observation points is
\begin{equation*}
\begin{split}
\Delta p = p(L) - p(0) = p_L - (p_L + \rho\cdot g \cdot L) = - \rho\cdot g \cdot L
\end{split}
\end{equation*}
\sphinxAtStartPar
\sphinxstylestrong{Example:} Compare the pressure difference for an experimental setup (pipe length 50 cm) in which water and diesel are the two liquids.

\begin{sphinxuseclass}{cell}\begin{sphinxVerbatimInput}

\begin{sphinxuseclass}{cell_input}
\begin{sphinxVerbatim}[commandchars=\\\{\}]
\PYG{c+c1}{\PYGZsh{} solution}
\PYG{n}{L\PYGZus{}p} \PYG{o}{=} \PYG{l+m+mi}{50} \PYG{c+c1}{\PYGZsh{} cm length of pipe}
\PYG{n}{g} \PYG{o}{=} \PYG{l+m+mi}{981} \PYG{c+c1}{\PYGZsh{} cm/s\PYGZca{}2, accl. due to gravity}


\PYG{c+c1}{\PYGZsh{} Assume densities}
\PYG{n}{rho\PYGZus{}w} \PYG{o}{=} \PYG{l+m+mf}{1.0} \PYG{c+c1}{\PYGZsh{} g/cm\PYGZca{}3, density of water}
\PYG{n}{rho\PYGZus{}d} \PYG{o}{=} \PYG{l+m+mf}{0.830} \PYG{c+c1}{\PYGZsh{} g/cm\PYGZca{}3, density of diesel}

\PYG{c+c1}{\PYGZsh{} calculate}
\PYG{n}{Dp\PYGZus{}w} \PYG{o}{=} \PYG{n}{rho\PYGZus{}w}\PYG{o}{*}\PYG{n}{g}\PYG{o}{*}\PYG{n}{L\PYGZus{}p} \PYG{c+c1}{\PYGZsh{} g/cm.s\PYGZca{}2, pressure difference due to water}
\PYG{n}{Dp\PYGZus{}d} \PYG{o}{=} \PYG{n}{rho\PYGZus{}d}\PYG{o}{*}\PYG{n}{g}\PYG{o}{*}\PYG{n}{L\PYGZus{}p}\PYG{c+c1}{\PYGZsh{} g/cm.s\PYGZca{}2, pressure difference due to water}
\PYG{n}{Dp\PYGZus{}w}
\PYG{n+nb}{print}\PYG{p}{(}\PYG{l+s+s2}{\PYGZdq{}}\PYG{l+s+s2}{The pressure difference due to water is }\PYG{l+s+si}{\PYGZob{}0:2.2f\PYGZcb{}}\PYG{l+s+s2}{ g/cm.s}\PYG{l+s+se}{\PYGZbs{}u00b2}\PYG{l+s+s2}{,}\PYG{l+s+s2}{\PYGZdq{}}\PYG{o}{.}\PYG{n}{format}\PYG{p}{(}\PYG{n}{Dp\PYGZus{}w}\PYG{p}{)}\PYG{p}{,} 
      \PYG{l+s+s2}{\PYGZdq{}}\PYG{l+s+s2}{and that due to diesel is }\PYG{l+s+si}{\PYGZob{}0:0.2f\PYGZcb{}}\PYG{l+s+s2}{ g/cm.s}\PYG{l+s+se}{\PYGZbs{}u00b2}\PYG{l+s+s2}{.}\PYG{l+s+s2}{\PYGZdq{}}\PYG{o}{.}\PYG{n}{format}\PYG{p}{(}\PYG{n}{Dp\PYGZus{}d}\PYG{p}{)}\PYG{p}{,} \PYG{p}{)} 
\end{sphinxVerbatim}

\end{sphinxuseclass}\end{sphinxVerbatimInput}
\begin{sphinxVerbatimOutput}

\begin{sphinxuseclass}{cell_output}
\begin{sphinxVerbatim}[commandchars=\\\{\}]
The pressure difference due to water is 49050.00 g/cm.s², and that due to diesel is 40711.50 g/cm.s².
\end{sphinxVerbatim}

\end{sphinxuseclass}\end{sphinxVerbatimOutput}

\end{sphinxuseclass}

\section{Hydrostatic pressure}
\label{\detokenize{content/flow/L3/13_gw_storage:hydrostatic-pressure}}
\sphinxAtStartPar
Mostly, pressure head is used instead of pressure when dealing with hydraulic properties or phenomena of the subsurface. The reason is that the pressure head can be easily measured with a tape whereas for a pressure measurement a more expensive manometer is necessary. The (hydrostatic) \sphinxstylestrong{pressure head} \(\psi\) {[}L{]} is defined as
\begin{equation*}
\begin{split}
\psi(z) = \psi_L + L - z
\end{split}
\end{equation*}
\sphinxAtStartPar
This expression makes it clear why we can measure the pressure head with a tape. Similar to pressure head, the hydrostatic pressure can be schematically represented as

\noindent{\hspace*{\fill}\sphinxincludegraphics[width=600\sphinxpxdimen]{{L03_f_12}.png}\hspace*{\fill}}


\section{Pressure in a Confined Aquifer}
\label{\detokenize{content/flow/L3/13_gw_storage:pressure-in-a-confined-aquifer}}
\sphinxAtStartPar
Let us now consider a \sphinxstyleemphasis{confined aquifer} (see Fig. below). The confining bed exerts a certain pressure \(p_{cb}\) on the aquifer. This pressure is compensated partly by the porous medium and partly by the groundwater (pressures \(p_{pm}\) and \(p_w\), respectively). Therefore, we can write the following equation
\begin{equation*}
\begin{split}
p_{cb} + p_{pm} + p_w = 0
\end{split}
\end{equation*}
\noindent{\hspace*{\fill}\sphinxincludegraphics[width=600\sphinxpxdimen]{{L03_f_13}.png}\hspace*{\fill}}

\sphinxAtStartPar
We can induce a \sphinxstyleemphasis{change in the hydrostatic pressure} \(\Delta p_w\) with an injection or release of groundwater. According to the above equation we can formulate
\begin{equation*}
\begin{split}
\Delta p_{cb} + \Delta p_{pm} + \Delta p_w = 0.
\end{split}
\end{equation*}
\sphinxAtStartPar
The hydrostatic pressure changes do not affect the weight of the confining bed and the exerted downward pressure remains unchanged. Therefore we have
\begin{equation*}
\begin{split}
\Delta p_{cb} = 0 \:\:\: \text{and}
\end{split}
\end{equation*}\begin{equation*}
\begin{split}
\Delta p_{pm} = -\Delta p_w
\end{split}
\end{equation*}
\sphinxAtStartPar
This implies that an increase of hydrostatic pressure automatically results in a decrease in the pressure exerted by the porous medium. In the case of a decrease of hydrostatic the pressure exerted by the porous medium would increase.
The change in hydraulic pressure will have two effects with regard to water volume. First, the hydraulic pressure change \(\Delta p_w\) directly leads to expansion/compression of the water and the water volume is accordingly increased/decreased. Secondly, the opposite change \(\Delta p_{pm} = -\Delta p_w\) leads to compression/expansion of the porous medium as a whole (not the individual grains!). This, in turn, results in a reduced/an enlarged pore space such that the stored water volume is decreased/ increased. Both effects contribute to aquifer storage properties (see next section).


\section{Aquifer storage properties}
\label{\detokenize{content/flow/L3/13_gw_storage:aquifer-storage-properties}}
\sphinxAtStartPar
\sphinxstyleemphasis{Storage properties} of the aquifer and associated parameters can be understood by considering pressure changes. For this purpose, we consider the effect of a \sphinxstyleemphasis{change in water volume} \(\Delta V_w'\) due to a \sphinxstyleemphasis{change in hydrostatic pressure}. The \sphinxstyleemphasis{relative} changes in water volume \(\Delta V_w'/\Delta w\) {[}\sphinxhyphen{}{]} are proportional to change of pressure in groundwater \(\Delta p_w\):
\begin{equation*}
\begin{split}
\frac{\Delta V_w'}{V_w} = \alpha_w \cdot \Delta p_w
\end{split}
\end{equation*}
\sphinxAtStartPar
with \(\alpha_w\) as the \sphinxstylestrong{compressibility of water} {[}LT\(^2\)/M{]}. The compressibility of water is roughly \(4.4 \cdot 10^{-10}\) m\(^2\)/N. Taking into account an incompressible behavior of the water, that means an increase in hydrostatic pressure results in an inflow of water. A decrease in hydrostatic pressure results would cause an outflow of water. The above equation can be rearranged to yield
\begin{equation*}
\begin{split}
\Delta V_w' = \alpha_w V_w \Delta p_w = \eta \alpha_w V_T \Delta p_w = \eta \alpha_w V_T \rho_w g \Delta \psi
\end{split}
\end{equation*}
\sphinxAtStartPar
With \(\eta\) {[}\sphinxhyphen{}{]} as the \sphinxstyleemphasis{total porosity} and \(\Delta \psi\) {[}L{]} as the \sphinxstyleemphasis{change in pressure head.}


\section{Change in total volume}
\label{\detokenize{content/flow/L3/13_gw_storage:change-in-total-volume}}
\sphinxAtStartPar
The preceding considerations dealt with a change in storage by inflow or outflow. The change was invoked by a change of pressure in groundwater \(\Delta p_w\). But a change could be also invoked by the change of the pressure exerted by the porous medium on the confining layer \(\Delta p_{pm}\). A change \(\Delta p_{pm}\) in the pressure results in a decrease or an increase \(\Delta V_T\) in total aquifer volume. Both quantities are proportional to each other via
\begin{equation*}
\begin{split}
\frac{\Delta V_T}{V_T} = - \alpha_{pm} \Delta p_{pm}
\end{split}
\end{equation*}
\sphinxAtStartPar
whereby the ratio is the relative change of the total volume and \(\alpha_{pm}\) the compressibility of the porous medium {[}LT\(^2\)/M{]}. The compressibility of the porous medium is roughly \(10^{-10} - 10^{-8}\) m\(^2\)/N for gravel,
\(10^{-9} -  10^{-7}\) m\(^2\)/N for sand, and \(10^{-8} - 10^{-6}\) m\(^2\)/N for clay. Taking into account the relation between pressure and pressure head, the above equation can be rearranged to yield
\begin{equation*}
\begin{split}
\Delta V_T = -\alpha_{pm}V_T\Delta p_{pm} = \alpha_{pm}V_T\Delta p_{pw} = \alpha_{pm}V_T\rho_w g \Delta \psi
\end{split}
\end{equation*}
\sphinxAtStartPar
\(\Delta V_T\) represents a \sphinxstyleemphasis{change in volume of the porous medium} as a whole. It is composed of a change in volume \(\Delta V_s\) of the solids and another change \(\Delta V_w''\) in water volume. Because the change in volume of the solid is negligible, we can write
\begin{equation*}
\begin{split}
\Delta V_T = \Delta V_s + \Delta V_w'' \approx \Delta V_w''
\end{split}
\end{equation*}
\sphinxAtStartPar
If we compare the last two equations we can immediately derive
\begin{equation*}
\begin{split}
\Delta V_w'' = \alpha_{pm} V_T\rho_w g \Delta\psi
\end{split}
\end{equation*}
\sphinxAtStartPar
With words that means a \sphinxstyleemphasis{decrease of pressure} in the porous medium leads to \sphinxstyleemphasis{an expansion of the porous medium} and an associated \sphinxstyleemphasis{increase in water volume} and enlarged pore space. An \sphinxstyleemphasis{increase in pressure} in the porous medium would lead to a \sphinxstyleemphasis{compression} of the porous medium and an associated \sphinxstyleemphasis{decrease} in water volume and \sphinxstyleemphasis{reduced} pore space.

\sphinxAtStartPar
The \sphinxstyleemphasis{total} change \(\Delta V_w\) in water volume consists of both effects caused by pressure changes \(\Delta p_{pm}\) and \(\Delta p_w\). Therefore we have
\begin{equation*}
\begin{split}
\Delta V_w = \Delta V_w' + \Delta V_w''
\end{split}
\end{equation*}
\sphinxAtStartPar
Using the results derived before, we can express how \(\Delta V_w\) depends on changes \(\Delta \psi\) in pressure head
\begin{equation*}
\begin{split}
\Delta V_w = \Delta V_w' + \Delta V_w'' = \eta \alpha_w V_T \rho_w g \Delta \psi + \alpha_{pm} V_T \rho_w g \Delta \psi
\end{split}
\end{equation*}
\sphinxAtStartPar
The \sphinxstyleemphasis{first term} of the sum is related to changes in hydrostatic pressure \((\Delta p_w)\) and the \sphinxstyleemphasis{second term} to pressure changes in the porous medium \((\Delta p_{pm})\).

\sphinxAtStartPar
\sphinxstylestrong{Example:} The 45 m thick aquifer under the change of pressure 245 KPa compacts 0.20 m. What is the compressibility of porous media.

\begin{sphinxuseclass}{cell}\begin{sphinxVerbatimInput}

\begin{sphinxuseclass}{cell_input}
\begin{sphinxVerbatim}[commandchars=\\\{\}]
\PYG{c+c1}{\PYGZsh{} Available information}

\PYG{n}{dP} \PYG{o}{=} \PYG{l+m+mi}{245} \PYG{c+c1}{\PYGZsh{} KPa= KN/m\PYGZca{}2, Change pressure}
\PYG{n}{m} \PYG{o}{=} \PYG{l+m+mi}{45} \PYG{c+c1}{\PYGZsh{} m, aquifer thickness}
\PYG{n}{dm} \PYG{o}{=} \PYG{l+m+mf}{0.20} \PYG{c+c1}{\PYGZsh{} m, change in aquifer thickness}

\PYG{n+nb}{print}\PYG{p}{(}\PYG{l+s+s2}{\PYGZdq{}}\PYG{l+s+s2}{V = A*m and dV = A * dm }\PYG{l+s+se}{\PYGZbs{}n}\PYG{l+s+s2}{\PYGZdq{}}\PYG{p}{)}

\PYG{c+c1}{\PYGZsh{} Calculate}
\PYG{n}{al\PYGZus{}pm} \PYG{o}{=} \PYG{p}{(}\PYG{n}{dm}\PYG{o}{/}\PYG{n}{m}\PYG{p}{)}\PYG{o}{/}\PYG{n}{dP} \PYG{c+c1}{\PYGZsh{} m\PYGZca{}2/KN, compressibility of porous media.}

\PYG{n+nb}{print}\PYG{p}{(}\PYG{l+s+s2}{\PYGZdq{}}\PYG{l+s+s2}{The compressibility of porous media in aquifer }\PYG{l+s+si}{\PYGZob{}0:0.2e\PYGZcb{}}\PYG{l+s+s2}{ m}\PYG{l+s+se}{\PYGZbs{}u00b2}\PYG{l+s+s2}{/KN.}\PYG{l+s+s2}{\PYGZdq{}}\PYG{o}{.}\PYG{n}{format}\PYG{p}{(}\PYG{n}{al\PYGZus{}pm}\PYG{p}{)}\PYG{p}{)}
\end{sphinxVerbatim}

\end{sphinxuseclass}\end{sphinxVerbatimInput}
\begin{sphinxVerbatimOutput}

\begin{sphinxuseclass}{cell_output}
\begin{sphinxVerbatim}[commandchars=\\\{\}]
V = A*m and dV = A * dm 

The compressibility of porous media in aquifer 1.81e\PYGZhy{}05 m²/KN.
\end{sphinxVerbatim}

\end{sphinxuseclass}\end{sphinxVerbatimOutput}

\end{sphinxuseclass}

\section{Specific storage}
\label{\detokenize{content/flow/L3/13_gw_storage:specific-storage}}
\sphinxAtStartPar
For the characterization of the storage properties of an aquifer, we use the term \sphinxstylestrong{specific storage} \(S_s\). It is defined as the volume of water that is released from a unit aquifer volume if hydrostatic pressure head is reduced by one unit
\begin{equation*}
\begin{split}
S_s = \frac{\Delta V_w}{V_T \cdot \Delta \psi}
\end{split}
\end{equation*}
\sphinxAtStartPar
The dimension of \sphinxstyleemphasis{specific storage} is 1/L. Both impacts on water volume discussed before have to be considered in order to quantify \(\Delta V_w\) in the above equation
\begin{equation*}
\begin{split}
\Delta V_w = \eta \alpha_w V_T \rho_w g \Delta \psi + \alpha_{pm} V_T\rho_w g \Delta \psi
\end{split}
\end{equation*}
\sphinxAtStartPar
The specific storage can therefore also be expressed as
\begin{equation*}
\begin{split}
S_s = (\eta \alpha_w + \alpha_{pm})\rho_w g
\end{split}
\end{equation*}
\sphinxAtStartPar
Typical values for specific storage range from \(10^{-6}\) 1/m (e.g. gravel) to \(10^{-2}\) 1/m (e.g. clay).


\section{Storativity}
\label{\detokenize{content/flow/L3/13_gw_storage:storativity}}
\sphinxAtStartPar
Due to their relatively large lateral extent, aquifers are mostly considered as spatially two\sphinxhyphen{}dimensional (2D) systems. In this case, specific storage \(S_s\) is replaced by the \sphinxstylestrong{storativity} or \sphinxstylestrong{storage coefficient} \(S\) (Fig. below). Reference volume in a confined aquifer for defining specific storage \(S_s\) is a unit cube (e.g. \(V_T = 1\) m\(^3\)), and for defining storativity \(S\) is a cuboid extending from the aquifer bottom to the aquifer top over a unit area (e.g. \(A = 1\) m\(^2\) and \(V_T = A\cdot m\)).

\sphinxAtStartPar
For confined aquifers \(S\) is simply obtained by multiplying \(S_s\) by the aquifer thickness \(m\)
\begin{equation*}
\begin{split}
S = S_s \cdot m
\end{split}
\end{equation*}
\noindent{\hspace*{\fill}\sphinxincludegraphics[width=600\sphinxpxdimen]{{L03_f_14}.png}\hspace*{\fill}}

\sphinxAtStartPar
\sphinxstyleemphasis{Storativity} can be interpreted as the volume of water released from an aquifer volume extending from the aquifer bottom up to the aquifer top over a unit area if the hydrostatic pressure is reduced by one unit. \sphinxstyleemphasis{Storativity} is dimensionless.

\sphinxAtStartPar
Actually, \sphinxstylestrong{unconfined} aquifers are always treated as 2D systems. As a consequence, storativity is used to quantify water storage properties. The definition of storativity remains unchanged in principle but the considered aquifer volume now extends from the aquifer bottom up to the water table. For \sphinxstyleemphasis{unconfined aquifers}, storativity values correspond to \sphinxstyleemphasis{effective porosities}. This is explained by the free groundwater table. In this case a pressure changes simply lead to filling or emptying of voids. This is fundamentally different from the storage properties of confined aquifers discussed before.

\sphinxAtStartPar
In \sphinxstylestrong{confined} aquifers all voids remain filled with groundwater during pressure changes and storage properties depend on the compressibilities of water and the porous medium.


\section{Chapter Quiz}
\label{\detokenize{content/flow/L3/13_gw_storage:chapter-quiz}}
\begin{sphinxuseclass}{cell}
\begin{sphinxuseclass}{tag_remove-input}
\begin{sphinxuseclass}{tag_hide-output}
\end{sphinxuseclass}
\end{sphinxuseclass}
\end{sphinxuseclass}
\sphinxstepscope

\begin{sphinxuseclass}{cell}
\begin{sphinxuseclass}{tag_hide-input}
\begin{sphinxuseclass}{tag_remove-output}
\begin{sphinxuseclass}{tag_remove-input}
\end{sphinxuseclass}
\end{sphinxuseclass}
\end{sphinxuseclass}
\end{sphinxuseclass}

\chapter{Darcy’s Law and Conductivity}
\label{\detokenize{content/flow/L4/14_darcy_law_K:darcy-s-law-and-conductivity}}\label{\detokenize{content/flow/L4/14_darcy_law_K::doc}}
\sphinxAtStartPar
\sphinxstyleemphasis{(The contents presented in this section were re\sphinxhyphen{}developed principally by \sphinxhref{http://web.iitd.ac.in/~chahar/}{Prof. B. R. Chahar} and Dr. P. K. Yadav. The original contents are from Prof. Rudolf Liedl)}


\section{Energy and hydraulic head}
\label{\detokenize{content/flow/L4/14_darcy_law_K:energy-and-hydraulic-head}}
\sphinxAtStartPar
In the last section we learned that \sphinxstyleemphasis{hydrostatic pressure difference} \(p(z)\) will not allow the fully quantify water flow. In fact in addition to \(p(z)\) other form of energy must also be considered.

\sphinxAtStartPar
The energy available for groundwater flow is given the name \sphinxstyleemphasis{hydraulic head} \((h)\) or also called \sphinxstyleemphasis{piezometric head}:. It consists of three components, related to
\begin{quote}

\sphinxAtStartPar
\sphinxstylestrong{elevation,
pressure and
velocity}.
\end{quote}

\sphinxAtStartPar
The total energy head is expressed by the equation
\begin{equation*}
\begin{split}
h = z + \frac{p}{\rho g} + \frac{v^2}{2g}
\end{split}
\end{equation*}
\sphinxAtStartPar
where, 
\(z\) is the \sphinxstyleemphasis{elevation} or \sphinxstyleemphasis{datum head} {[}L{]}, 
\(p\) is the \sphinxstyleemphasis{pressure}
exerted by water column {[}M L\(^{-1}\) T\(^{-2}\){]},
\(\rho\) is the density of fluid {[}M L\(^{-3}\){]},
\(g\) is the acceleration due to gravity {[}LT\(^{-2}\){]}, and 
\(v\) is velocity of flow {[}LT\(^{-1}\){]}.

\sphinxAtStartPar
Note that the \(h\) has the dimension of length {[}L{]}.  In groundwater flow, the velocity is so low
that the energy contained in velocity can be neglected when computing the total energy.
Thus, the hydraulic head (see figure below) is written as
\begin{equation*}
\begin{split}
h = z + \frac{p}{\rho g} 
\end{split}
\end{equation*}
\sphinxAtStartPar
The above equation says that Water flow is governed by differences in hydraulic head and not by differences in pressure head alone.

\sphinxAtStartPar
It is to be noted that \(z\) depends on the orientation. In the above equation \(+z\) is considered oriented upward (based on conventional sign convention). If \(z-\)axis is oriented downwards, we have
\begin{equation*}
\begin{split}
h = \frac{p}{\rho g}-z 
\end{split}
\end{equation*}

\subsection{Hydraulic head and discharge \sphinxhyphen{} when there is no discharge}
\label{\detokenize{content/flow/L4/14_darcy_law_K:hydraulic-head-and-discharge-when-there-is-no-discharge}}
\sphinxAtStartPar
Consider the figure below:

\noindent{\hspace*{\fill}\sphinxincludegraphics[height=400\sphinxpxdimen]{{L4_f1}.png}\hspace*{\fill}}

\sphinxAtStartPar
The pressure head: \(p(z) = p_L + \rho \cdot g \cdot (L-z)\) 
The hydraulic head (piezometric head): \(h(z) = \frac{p(z)}{\rho \cdot g } + z = \frac{p_L + \rho \cdot g \cdot (L-z) }{\rho \cdot g } + z = \frac{p_L}{\rho \cdot g}+ L = \text{Const}\)

\sphinxAtStartPar
In the figure above the hydraulic head difference between two points (\(z=0\) and \(z=L\)) are exactly equal, i.e., \(\Delta h = 0\). This refers to the system with no energy gradient and hence a \sphinxstyleemphasis{no flow} system.


\subsection{Hydraulic head and discharge \sphinxhyphen{} when there will be a discharge}
\label{\detokenize{content/flow/L4/14_darcy_law_K:hydraulic-head-and-discharge-when-there-will-be-a-discharge}}
\sphinxAtStartPar
Now consider the figure below

\noindent{\hspace*{\fill}\sphinxincludegraphics[height=400\sphinxpxdimen]{{L4_f2}.png}\hspace*{\fill}}

\sphinxAtStartPar
Here there is clear difference between the elevation head (\(z_1\) and \(z_2\)), which is taken from a reference level (\(z=0\) in this case. Average Sea Level (ASL), is often use for this reference). Also differing are pressure heads (\(p_1\) and \(p_2\)). Therefore, the \(h(z)\) in this case are:
\begin{equation*}
\begin{split}
\begin{align}
h_1 &= \frac{p_1}{\rho \cdot g} + z_1 \\
h_2 &= \frac{p_2}{\rho \cdot g} + z_2
\end{align}
\end{split}
\end{equation*}
\sphinxAtStartPar
As can be observed from the figure, in this case \(h_1<h_2\). Thus, analogus to flow of energy \sphinxhyphen{} i.e., from higher energy level to the lower energy level, the flow in this example case is from cross\sphinxhyphen{}section 2 towards cross section 1.

\sphinxAtStartPar
It is to be noted that the \sphinxstyleemphasis{magnitude} of the hydraulic head (\(h\)) rather than the \sphinxstyleemphasis{orientation} of the set\sphinxhyphen{}up determines the direction of discharge.

\sphinxAtStartPar
Also, important to note is that the \sphinxstyleemphasis{differences of hydraulic head} \((\Delta h)\) is independent of of the position of the origin of the reference axis  (\(z\)\sphinxhyphen{}axis in the above case).


\subsection{Example problem}
\label{\detokenize{content/flow/L4/14_darcy_law_K:example-problem}}
\begin{sphinxadmonition}{note}{Energy and hydraulic head}

\sphinxAtStartPar
At a place where the fluid pressure is \(1500 \frac{N}{m^2}\), the distance above the reference elevation is \(0,8 m\) and the fluid density is \(1000 \frac{kg}{m^3}\). The fluid moves at a speed of \(1 \cdot 10^{-6} \frac{m}{s}\). Find the total energy head.
\end{sphinxadmonition}

\begin{sphinxuseclass}{cell}\begin{sphinxVerbatimInput}

\begin{sphinxuseclass}{cell_input}
\begin{sphinxVerbatim}[commandchars=\\\{\}]
\PYG{n+nb}{print}\PYG{p}{(}\PYG{l+s+s2}{\PYGZdq{}}\PYG{l+s+s2}{Let us find the total energy head.}\PYG{l+s+se}{\PYGZbs{}n}\PYG{l+s+se}{\PYGZbs{}n}\PYG{l+s+s2}{Provided are:}\PYG{l+s+s2}{\PYGZdq{}}\PYG{p}{)}

\PYG{n}{z} \PYG{o}{=} \PYG{l+m+mf}{0.8} \PYG{c+c1}{\PYGZsh{}elevation}
\PYG{n}{g} \PYG{o}{=} \PYG{l+m+mf}{9.81} \PYG{c+c1}{\PYGZsh{} [m/s\PYGZca{}2]}
\PYG{n}{p} \PYG{o}{=} \PYG{l+m+mi}{1500} \PYG{c+c1}{\PYGZsh{} fluid pressure [N/m\PYGZca{}2]}
\PYG{n}{rho} \PYG{o}{=} \PYG{l+m+mi}{1000} \PYG{c+c1}{\PYGZsh{} fluid density [kg/m\PYGZca{}3]}
\PYG{n}{v} \PYG{o}{=} \PYG{l+m+mf}{1e\PYGZhy{}6} \PYG{c+c1}{\PYGZsh{}velocity [m/s]}

\PYG{c+c1}{\PYGZsh{}solution}
\PYG{n}{h} \PYG{o}{=} \PYG{n}{z} \PYG{o}{+} \PYG{n}{p}\PYG{o}{/}\PYG{p}{(}\PYG{n}{rho}\PYG{o}{*}\PYG{n}{g}\PYG{p}{)}\PYG{o}{+} \PYG{n}{v}\PYG{o}{*}\PYG{o}{*}\PYG{l+m+mi}{2}\PYG{o}{/}\PYG{p}{(}\PYG{l+m+mi}{2}\PYG{o}{*}\PYG{n}{g}\PYG{p}{)} \PYG{c+c1}{\PYGZsh{} m, head}

\PYG{n+nb}{print}\PYG{p}{(}\PYG{l+s+s2}{\PYGZdq{}}\PYG{l+s+s2}{elevation = }\PYG{l+s+si}{\PYGZob{}\PYGZcb{}}\PYG{l+s+s2}{ m}\PYG{l+s+se}{\PYGZbs{}n}\PYG{l+s+s2}{acceleration due to gravity = }\PYG{l+s+si}{\PYGZob{}\PYGZcb{}}\PYG{l+s+s2}{ m}\PYG{l+s+se}{\PYGZbs{}n}\PYG{l+s+s2}{fluid pressure = }\PYG{l+s+si}{\PYGZob{}\PYGZcb{}}\PYG{l+s+s2}{ m}\PYG{l+s+se}{\PYGZbs{}n}\PYG{l+s+s2}{fluid density = }\PYG{l+s+si}{\PYGZob{}\PYGZcb{}}\PYG{l+s+s2}{ m, and}\PYG{l+s+se}{\PYGZbs{}n}\PYG{l+s+s2}{velocity = }\PYG{l+s+si}{\PYGZob{}\PYGZcb{}}\PYG{l+s+s2}{ m/s}\PYG{l+s+s2}{\PYGZdq{}}\PYG{o}{.}\PYG{n}{format}\PYG{p}{(}\PYG{n}{z}\PYG{p}{,}\PYG{n}{g}\PYG{p}{,}\PYG{n}{p}\PYG{p}{,}\PYG{n}{rho}\PYG{p}{,}\PYG{n}{v}\PYG{p}{)}\PYG{p}{,}\PYG{l+s+s2}{\PYGZdq{}}\PYG{l+s+se}{\PYGZbs{}n}\PYG{l+s+s2}{\PYGZdq{}}\PYG{p}{)}
\PYG{n+nb}{print}\PYG{p}{(}\PYG{l+s+s2}{\PYGZdq{}}\PYG{l+s+s2}{The resulting total energy head is }\PYG{l+s+si}{\PYGZob{}0:0.2f\PYGZcb{}}\PYG{l+s+s2}{ m}\PYG{l+s+s2}{\PYGZdq{}}\PYG{o}{.}\PYG{n}{format}\PYG{p}{(}\PYG{n}{h}\PYG{p}{)}\PYG{p}{)}
\end{sphinxVerbatim}

\end{sphinxuseclass}\end{sphinxVerbatimInput}
\begin{sphinxVerbatimOutput}

\begin{sphinxuseclass}{cell_output}
\begin{sphinxVerbatim}[commandchars=\\\{\}]
Let us find the total energy head.

Provided are:
elevation = 0.8 m
acceleration due to gravity = 9.81 m
fluid pressure = 1500 m
fluid density = 1000 m, and
velocity = 1e\PYGZhy{}06 m/s 

The resulting total energy head is 0.95 m
\end{sphinxVerbatim}

\end{sphinxuseclass}\end{sphinxVerbatimOutput}

\end{sphinxuseclass}

\section{Darcy’s Law}
\label{\detokenize{content/flow/L4/14_darcy_law_K:darcy-s-law}}
\sphinxAtStartPar
Groundwater in its natural state is invariably moving. This movement is governed by
established hydraulic principles. Darcy’s Law is a phenomenologically derived
constitutive equation that describes the flow of a fluid through a porous medium.
Darcy’s Law along with the equation of conservation of mass is equivalent to the
groundwater flow equation, one of the basic relationships of hydrogeology. It is also
used to describe oil, water, and gas flows through petroleum reservoirs.

\sphinxAtStartPar
The law was formulated by Henry Darcy (Darcy, 1856) based on the results of experiments on the flow
of water through beds of sand (Figure below shows a schematic setup). In the experiments, \sphinxstyleemphasis{area of cross section}  (\(A\) {[}L\(^2\){]} \(= \text{Const}\)) was kept constant. Constant discharge (\(Q= \text{Const}\)) was applied and the sand medium was fully saturated, i.e., voids between sand grains were completely filled with water.

\noindent{\hspace*{\fill}\sphinxincludegraphics[height=400\sphinxpxdimen]{{L4_f3}.png}\hspace*{\fill}}

\sphinxAtStartPar
Darcy’s law is a simple mathematical statement which neatly summarizes several
familiar properties that groundwater flowing in aquifers exhibits, including:
\begin{quote}

\sphinxAtStartPar
if there
is no hydraulic gradient (difference in hydraulic head over a distance), no flow occurs
(this is hydrostatic conditions),
\end{quote}
\begin{quote}

\sphinxAtStartPar
if there is a hydraulic gradient, flow will occur from
a high head towards a low head (opposite the direction of increasing gradient, hence the
negative sign in Darcy’s law),
\end{quote}
\begin{quote}

\sphinxAtStartPar
the greater the hydraulic gradient (through the same
aquifer material), the greater the discharge, and
\end{quote}
\begin{quote}

\sphinxAtStartPar
the discharge may be different
through different aquifer materials (or even through the same material, in a different
direction) even if the same hydraulic gradient exists.
\end{quote}

\sphinxAtStartPar
From the experiments Darcy observed that the \sphinxstyleemphasis{volume of water per unit time} passing through a porous medium
\begin{itemize}
\item {} 
\sphinxAtStartPar
is \sphinxstyleemphasis{directly proportional} to the  \(A\) {[}L\(^2\){]} and the head difference between inlet and outlet \((h_1 – h_2)\) {[}L{]}, and

\item {} 
\sphinxAtStartPar
is inversely proportional to the \sphinxstyleemphasis{length of the medium} \(L\) {[}L{]}

\end{itemize}

\sphinxAtStartPar
i.e.,
\begin{equation*}
\begin{split}
\frac{\text{Vol}}{t}= Q \propto A (h_1 - h_2)\frac{1}{L}
\end{split}
\end{equation*}
\sphinxAtStartPar
which in terms of specific discharge \(q\), or discharge velocity or Darcy velocity \(v\) {[}LT\(^{-1}\){]} is
\begin{equation*}
\begin{split}
q = v = \frac{Q}{A}= - K\frac{\partial h}{\partial L} = - K\,i
\end{split}
\end{equation*}
\sphinxAtStartPar
where constant of proportionality \(K =\) \sphinxstyleemphasis{hydraulic conductivity} {[}LT\(^{-1}\){]}; and \(i = \partial h/ \partial L =\)
\sphinxstyleemphasis{hydraulic gradient} = rate of head loss per unit length of medium {[} {]}. The \sphinxstyleemphasis{negative sign}
indicates that the total head is decreasing in the direction of flow because of friction or
resistance


\subsection{Example problem}
\label{\detokenize{content/flow/L4/14_darcy_law_K:id1}}
\begin{sphinxadmonition}{note}{Darcy’s Law}

\sphinxAtStartPar
Calculate the specific discharge and the flow rate passing through the surface with the given parameters.
\end{sphinxadmonition}

\begin{sphinxuseclass}{cell}
\begin{sphinxuseclass}{tag_remove-input}\begin{sphinxVerbatimOutput}

\begin{sphinxuseclass}{cell_output}
\begin{sphinxVerbatim}[commandchars=\\\{\}]
Provided are:

Conductivity = 0.0005 m/s
Surface = 10 m²
Hydraulic head at the inlet = 10 m
Hydraulic head at the outlet = 2 m
Lenght of the column = 5 m 
\end{sphinxVerbatim}

\end{sphinxuseclass}\end{sphinxVerbatimOutput}

\end{sphinxuseclass}
\end{sphinxuseclass}
\begin{sphinxuseclass}{cell}\begin{sphinxVerbatimInput}

\begin{sphinxuseclass}{cell_input}
\begin{sphinxVerbatim}[commandchars=\\\{\}]
\PYG{n}{K} \PYG{o}{=} \PYG{l+m+mf}{5e\PYGZhy{}4} \PYG{c+c1}{\PYGZsh{} m/s, conductivity}
\PYG{n}{A} \PYG{o}{=} \PYG{l+m+mi}{10} \PYG{c+c1}{\PYGZsh{} m², surface}
\PYG{n}{h\PYGZus{}in} \PYG{o}{=} \PYG{l+m+mi}{10} \PYG{c+c1}{\PYGZsh{} m, hydraulic head at the inlet}
\PYG{n}{h\PYGZus{}out} \PYG{o}{=} \PYG{l+m+mi}{2} \PYG{c+c1}{\PYGZsh{} m, hydraulic head at the outlet}
\PYG{n}{L} \PYG{o}{=} \PYG{l+m+mi}{5} \PYG{c+c1}{\PYGZsh{}m, lenght of the column}

\PYG{c+c1}{\PYGZsh{}intermediate calculation}
\PYG{n}{I} \PYG{o}{=} \PYG{p}{(}\PYG{n}{h\PYGZus{}in}\PYG{o}{\PYGZhy{}}\PYG{n}{h\PYGZus{}out}\PYG{p}{)}\PYG{o}{/}\PYG{n}{L}

\PYG{c+c1}{\PYGZsh{}solution}
\PYG{n}{q} \PYG{o}{=} \PYG{n}{K}\PYG{o}{*}\PYG{n}{I}
\PYG{n}{Q} \PYG{o}{=} \PYG{n}{K}\PYG{o}{*}\PYG{n}{I}\PYG{o}{*}\PYG{n}{A}

\PYG{n+nb}{print}\PYG{p}{(}\PYG{l+s+s2}{\PYGZdq{}}\PYG{l+s+s2}{Conductivity = }\PYG{l+s+si}{\PYGZob{}\PYGZcb{}}\PYG{l+s+s2}{ m/s}\PYG{l+s+se}{\PYGZbs{}n}\PYG{l+s+s2}{Surface = }\PYG{l+s+si}{\PYGZob{}\PYGZcb{}}\PYG{l+s+s2}{ m²}\PYG{l+s+se}{\PYGZbs{}n}\PYG{l+s+s2}{Hydraulic head at the inlet = }\PYG{l+s+si}{\PYGZob{}\PYGZcb{}}\PYG{l+s+s2}{ m}\PYG{l+s+se}{\PYGZbs{}n}\PYG{l+s+s2}{Hydraulic head at the outlet = }\PYG{l+s+si}{\PYGZob{}\PYGZcb{}}\PYG{l+s+s2}{ m}\PYG{l+s+se}{\PYGZbs{}n}\PYG{l+s+s2}{Lenght of the column = }\PYG{l+s+si}{\PYGZob{}\PYGZcb{}}\PYG{l+s+s2}{ m}\PYG{l+s+s2}{\PYGZdq{}}\PYG{o}{.}\PYG{n}{format}\PYG{p}{(}\PYG{n}{K}\PYG{p}{,} \PYG{n}{A}\PYG{p}{,} \PYG{n}{h\PYGZus{}in}\PYG{p}{,} \PYG{n}{h\PYGZus{}out}\PYG{p}{,} \PYG{n}{L}\PYG{p}{)}\PYG{p}{,} \PYG{l+s+s2}{\PYGZdq{}}\PYG{l+s+se}{\PYGZbs{}n}\PYG{l+s+s2}{\PYGZdq{}}\PYG{p}{)}
\PYG{n+nb}{print}\PYG{p}{(}\PYG{l+s+s2}{\PYGZdq{}}\PYG{l+s+s2}{Solution:}\PYG{l+s+se}{\PYGZbs{}n}\PYG{l+s+s2}{The resulting specific discharge is }\PYG{l+s+si}{\PYGZob{}0:0.0e\PYGZcb{}}\PYG{l+s+s2}{ m/s}\PYG{l+s+s2}{\PYGZdq{}}\PYG{o}{.}\PYG{n}{format}\PYG{p}{(}\PYG{n}{q}\PYG{p}{)}\PYG{p}{,} \PYG{l+s+s2}{\PYGZdq{}}\PYG{l+s+se}{\PYGZbs{}n}\PYG{l+s+s2}{and the flow rate is }\PYG{l+s+si}{\PYGZob{}0:0.0e\PYGZcb{}}\PYG{l+s+s2}{ m³/s}\PYG{l+s+s2}{\PYGZdq{}}\PYG{o}{.}\PYG{n}{format}\PYG{p}{(}\PYG{n}{Q}\PYG{p}{)}\PYG{p}{)}
\end{sphinxVerbatim}

\end{sphinxuseclass}\end{sphinxVerbatimInput}
\begin{sphinxVerbatimOutput}

\begin{sphinxuseclass}{cell_output}
\begin{sphinxVerbatim}[commandchars=\\\{\}]
Conductivity = 0.0005 m/s
Surface = 10 m²
Hydraulic head at the inlet = 10 m
Hydraulic head at the outlet = 2 m
Lenght of the column = 5 m 

Solution:
The resulting specific discharge is 8e\PYGZhy{}04 m/s 
and the flow rate is 8e\PYGZhy{}03 m³/s
\end{sphinxVerbatim}

\end{sphinxuseclass}\end{sphinxVerbatimOutput}

\end{sphinxuseclass}

\subsection{Darcy’s law and analogous physical systems}
\label{\detokenize{content/flow/L4/14_darcy_law_K:darcy-s-law-and-analogous-physical-systems}}
\sphinxAtStartPar
Darcy’s law is analogous to pipe flow in which energy is dissipated over the distance to overcome frictional loss resulting
from fluid viscosity.

\sphinxAtStartPar
It also forms the scientific basis of permeability used in the earth
sciences.

\sphinxAtStartPar
It may be noted Darcy’s law is analogous to Fourier’s law in the field of heat conduction, Ohm’s law in the field of electrical networks, or Fick’s law in diffusion theory.


\section{Hydraulic Conductivity and Intrinsic Permeability}
\label{\detokenize{content/flow/L4/14_darcy_law_K:hydraulic-conductivity-and-intrinsic-permeability}}
\sphinxAtStartPar
\sphinxstyleemphasis{Hydraulic Conductivity (\(K\))} appeared in the Darcy’s law as a constant of proportionality, i.e., it is the fundamental quantity that is required to describe groundwater flow. Therefore we illustrate this further,

\sphinxAtStartPar
A medium has a \sphinxstyleemphasis{unit hydraulic conductivity} if it will transmit in \sphinxstyleemphasis{unit time} a \sphinxstyleemphasis{unit volume of groundwater} at the prevailing kinematic viscosity through a cross section of \sphinxstyleemphasis{unit area} measured at \sphinxstyleemphasis{right angles} to the direction of flow, under a \sphinxstyleemphasis{unit hydraulic gradient}.

\sphinxAtStartPar
The hydraulic conductivity of a soil or rock depends on a variety of
physical factors, important ones are:
\begin{itemize}
\item {} 
\sphinxAtStartPar
porosity,

\item {} 
\sphinxAtStartPar
particle size and distribution,

\item {} 
\sphinxAtStartPar
shape of particles,

\item {} 
\sphinxAtStartPar
arrangement of particles.

\end{itemize}

\sphinxAtStartPar
Also, hydraulic conductivity is depends on the property of the fluid e.g., density, viscosity

\sphinxAtStartPar
In general for unconsolidated porous media,\(K\) varies with \sphinxstyleemphasis{square} of particle size; clayey materials exhibit low values of \(K\), whereas
sands and gravels display high values

\sphinxAtStartPar
Typical values for hydraulic conductivity (see figure XX for more comprehensive listing):


\begin{savenotes}\sphinxattablestart
\centering
\begin{tabulary}{\linewidth}[t]{|T|T|}
\hline
\sphinxstyletheadfamily 
\sphinxAtStartPar
Media Type
&\sphinxstyletheadfamily 
\sphinxAtStartPar
hydraulic conductivity (m/s)
\\
\hline
\sphinxAtStartPar
Gravel
&
\sphinxAtStartPar
\(10^{-2} – 10^{-1}\)
\\
\hline
\sphinxAtStartPar
coarse sand
&
\sphinxAtStartPar
\(\approx 10^{-3}\)
\\
\hline
\sphinxAtStartPar
medium sand
&
\sphinxAtStartPar
\(10 ^{-4} – 10^{-3}\)
\\
\hline
\sphinxAtStartPar
fine sand
&
\sphinxAtStartPar
\(10^{-5} – 10^{-4}\)
\\
\hline
\sphinxAtStartPar
Silt
&
\sphinxAtStartPar
\(10^{-9}  – 10^{-6} \)
\\
\hline
\sphinxAtStartPar
Clay
&
\sphinxAtStartPar
\(< 10^{-9} \)
\\
\hline
\end{tabulary}
\par
\sphinxattableend\end{savenotes}


\subsection{Obtaining Hydraulic Conductivities}
\label{\detokenize{content/flow/L4/14_darcy_law_K:obtaining-hydraulic-conductivities}}
\sphinxAtStartPar
The hydraulic conductivity depends on properties of the fluid (density, viscosity,
temperature) and on properties of the porous medium (effective porosity, grain size
distribution). It can be obtained by calculation from formulas, laboratory methods, or
field tests

\sphinxAtStartPar
The laboratory method can be indirect method or direct method. For example,
determination of the hydraulic conductivity based on the evaluation of sieve analysis
data is the indirect laboratory method. On the other hand, the direct method of
determination of the hydraulic conductivity (e.g. permeameter) is based on some version of Darcy‘s experiment. Advantages of laboratory methods include controlled
conditions, small sample size (easy handling), lower costs, larger number of
experiments, etc. While, the disadvantages of laboratory methods are disturbed
samples, additional pathways at column walls, small sample size (randomly high or low
K), flushing of fine material, etc.


\subsubsection{Hydraulic Conductivities estimations from Sieve analysis}
\label{\detokenize{content/flow/L4/14_darcy_law_K:hydraulic-conductivities-estimations-from-sieve-analysis}}
\sphinxAtStartPar
Sieve analysis data can be evaluated to estimate hydraulic conductivity of
unconsolidated media. There are several \sphinxstyleemphasis{empirical methods}. The simplest one dates back to Hazen (1892):
\begin{equation*}
\begin{split}
K = 0.0116 \cdot d_{10}^2
\end{split}
\end{equation*}
\sphinxAtStartPar
where, \(d_{10}\) = grain diameter (mm) corresponding to \(10
\%\) of cumulative mass fraction. It can be generalized by including temperature (\(\theta\) in \(^\circ\)C). Thus the Hazen formula becomes
\begin{equation*}
\begin{split}
K = 0.0116 \cdot d_{10}^2 \cdot (0.7 + 0.03\cdot\theta)
\end{split}
\end{equation*}
\sphinxAtStartPar
Hazen’s formula is only valid for the indicated units, i.e., conversion of the \sphinxstyleemphasis{unit} may be required before using the formula.


\subsection{Example problem}
\label{\detokenize{content/flow/L4/14_darcy_law_K:id2}}
\begin{sphinxadmonition}{note}{Hydraulic Conductivity from sieve data}

\sphinxAtStartPar
An Aquifer with fine to medium sand was investigated with an sieve analysis. At a temperature of \(20°C\) a \(d_{10}\) of \(0.13\) mm was measured. Determine the hydraulic conductivity (using Hazen’s formula) and how it changes when the temperature rises by \(5°C\).
\end{sphinxadmonition}

\begin{sphinxuseclass}{cell}\begin{sphinxVerbatimInput}

\begin{sphinxuseclass}{cell_input}
\begin{sphinxVerbatim}[commandchars=\\\{\}]
\PYG{n+nb}{print}\PYG{p}{(}\PYG{l+s+s2}{\PYGZdq{}}\PYG{l+s+s2}{Let us find the hydraulic conductivities.}\PYG{l+s+se}{\PYGZbs{}n}\PYG{l+s+se}{\PYGZbs{}n}\PYG{l+s+s2}{Provided are:}\PYG{l+s+s2}{\PYGZdq{}}\PYG{p}{)}

\PYG{n}{d10} \PYG{o}{=} \PYG{l+m+mf}{0.13} \PYG{c+c1}{\PYGZsh{} mm,  grain diameter corresponding to 10\PYGZpc{} of cumulative mass fraction}
\PYG{n}{T1} \PYG{o}{=} \PYG{l+m+mi}{10} \PYG{c+c1}{\PYGZsh{} °C, Temperature}
\PYG{n}{deltaT} \PYG{o}{=} \PYG{l+m+mi}{5} \PYG{c+c1}{\PYGZsh{} °C, temperature change}

\PYG{c+c1}{\PYGZsh{}intermediate calculation}
\PYG{n}{T2} \PYG{o}{=} \PYG{n}{T1} \PYG{o}{+} \PYG{n}{deltaT}

\PYG{c+c1}{\PYGZsh{}solution based on Hazen\PYGZsq{}s formula}
\PYG{n}{K1} \PYG{o}{=} \PYG{l+m+mf}{0.0116} \PYG{o}{*} \PYG{n}{d10}\PYG{o}{*}\PYG{o}{*}\PYG{l+m+mi}{2} \PYG{o}{*} \PYG{p}{(}\PYG{l+m+mf}{0.7} \PYG{o}{+} \PYG{l+m+mf}{0.03}\PYG{o}{*}\PYG{n}{T1}\PYG{p}{)}
\PYG{n}{K2} \PYG{o}{=} \PYG{l+m+mf}{0.0116} \PYG{o}{*} \PYG{n}{d10}\PYG{o}{*}\PYG{o}{*}\PYG{l+m+mi}{2} \PYG{o}{*} \PYG{p}{(}\PYG{l+m+mf}{0.7} \PYG{o}{+} \PYG{l+m+mf}{0.03}\PYG{o}{*}\PYG{n}{T2}\PYG{p}{)}

\PYG{n+nb}{print}\PYG{p}{(}\PYG{l+s+s2}{\PYGZdq{}}\PYG{l+s+s2}{grain diameter corresponding to 10}\PYG{l+s+si}{\PYGZpc{} o}\PYG{l+s+s2}{f cumulative mass fraction = }\PYG{l+s+si}{\PYGZob{}\PYGZcb{}}\PYG{l+s+s2}{ mm}\PYG{l+s+se}{\PYGZbs{}n}\PYG{l+s+s2}{Temperature = }\PYG{l+s+si}{\PYGZob{}\PYGZcb{}}\PYG{l+s+s2}{ °C}\PYG{l+s+se}{\PYGZbs{}n}\PYG{l+s+s2}{temperature change = }\PYG{l+s+si}{\PYGZob{}\PYGZcb{}}\PYG{l+s+s2}{ K}\PYG{l+s+s2}{\PYGZdq{}}\PYG{o}{.}\PYG{n}{format}\PYG{p}{(}\PYG{n}{d10}\PYG{p}{,} \PYG{n}{T1}\PYG{p}{,} \PYG{n}{deltaT}\PYG{p}{)}\PYG{p}{,}\PYG{l+s+s2}{\PYGZdq{}}\PYG{l+s+se}{\PYGZbs{}n}\PYG{l+s+s2}{\PYGZdq{}}\PYG{p}{)}
\PYG{n+nb}{print}\PYG{p}{(}\PYG{l+s+s2}{\PYGZdq{}}\PYG{l+s+s2}{The resulting hydraulic Conductivity at 20°C is }\PYG{l+s+si}{\PYGZob{}0:0.2e\PYGZcb{}}\PYG{l+s+s2}{ m/s}\PYG{l+s+s2}{\PYGZdq{}}\PYG{o}{.}\PYG{n}{format}\PYG{p}{(}\PYG{n}{K1}\PYG{p}{)}\PYG{p}{,}
      \PYG{l+s+s2}{\PYGZdq{}}\PYG{l+s+se}{\PYGZbs{}n}\PYG{l+s+s2}{and the resulting hydraulic conductivity at 25°C is }\PYG{l+s+si}{\PYGZob{}0:0.2e\PYGZcb{}}\PYG{l+s+s2}{ m/s}\PYG{l+s+s2}{\PYGZdq{}}\PYG{o}{.}\PYG{n}{format}\PYG{p}{(}\PYG{n}{K2}\PYG{p}{)}\PYG{p}{)}
\end{sphinxVerbatim}

\end{sphinxuseclass}\end{sphinxVerbatimInput}
\begin{sphinxVerbatimOutput}

\begin{sphinxuseclass}{cell_output}
\begin{sphinxVerbatim}[commandchars=\\\{\}]
Let us find the hydraulic conductivities.

Provided are:
grain diameter corresponding to 10\PYGZpc{} of cumulative mass fraction = 0.13 mm
Temperature = 10 °C
temperature change = 5 K 

The resulting hydraulic Conductivity at 20°C is 1.96e\PYGZhy{}04 m/s 
and the resulting hydraulic conductivity at 25°C is 2.25e\PYGZhy{}04 m/s
\end{sphinxVerbatim}

\end{sphinxuseclass}\end{sphinxVerbatimOutput}

\end{sphinxuseclass}

\subsubsection{Hydraulic Conductivities estimations from Darcy’s Law}
\label{\detokenize{content/flow/L4/14_darcy_law_K:hydraulic-conductivities-estimations-from-darcy-s-law}}
\sphinxAtStartPar
\sphinxstylestrong{Permeameter} is an instrument used to determine hydraulic conductivity of soil samples
in the laboratory as \sphinxstyleemphasis{direct method}. The design of permeameters is based on Darcy‘s
experiment.

\noindent{\hspace*{\fill}\sphinxincludegraphics[width=200\sphinxpxdimen]{{L4_f4}.png}\hspace*{\fill}}

\sphinxAtStartPar
There are mainly two types of permeameters
\begin{enumerate}
\sphinxsetlistlabels{\arabic}{enumi}{enumii}{}{.}%
\item {} 
\sphinxAtStartPar
Constant\sphinxhyphen{}head permeameter, and

\item {} 
\sphinxAtStartPar
Falling\sphinxhyphen{}head permeameter.

\end{enumerate}

\sphinxAtStartPar
In \sphinxstylestrong{constant\sphinxhyphen{}head permeameters} as
shown in Figure the hydraulic heads at inflow and outflow of the Darcy column are
constant in time. As a consequence, the discharge is not changing with time.

\noindent{\hspace*{\fill}\sphinxincludegraphics[height=200\sphinxpxdimen]{{L4_f5}.png}\hspace*{\fill}}

\sphinxAtStartPar
The hydraulic conductivity can be obtained by observing discharge and heads and then
substituting in the below formula that is rearranged form of Darcy’s law from:
\begin{equation*}
\begin{split}
K = \frac{QL}{A(h_{in}- h_{out}}
\end{split}
\end{equation*}
\sphinxAtStartPar
where \(Q\) =discharge{[}L\(^3\)T\(^{-1}\){]};
\(L\) = length of sample {[}L{]};  \(A\) = cross\sphinxhyphen{}sectional area of sample {[}L\(^2\){]}; 
\(h_{in}\) = hydraulic head at column inlet {[}L{]};
\(h_{out}\) = hydraulic head at column outlet {[}L{]};
\(\Delta h\) = \(h_{in} - h_{out}\)

\sphinxAtStartPar
\(h_{out}\) can be set equal to zero as only head differences are important.


\subsection{Example problem}
\label{\detokenize{content/flow/L4/14_darcy_law_K:id3}}
\begin{sphinxadmonition}{note}{Hydraulic Conductivity from Constant head\sphinxhyphen{}permeameter}

\sphinxAtStartPar
A constant\sphinxhyphen{}head permeameter has a length of 15 cm and a cross\sphinxhyphen{}sectional area of \(25\) cm\(^2\). With a head of 5 cm, a total Volume of 100 mL of water is collected in 12 min. Determine the hydraulic conductivity.
\end{sphinxadmonition}

\begin{sphinxuseclass}{cell}\begin{sphinxVerbatimInput}

\begin{sphinxuseclass}{cell_input}
\begin{sphinxVerbatim}[commandchars=\\\{\}]
\PYG{n+nb}{print}\PYG{p}{(}\PYG{l+s+s2}{\PYGZdq{}}\PYG{l+s+s2}{Let us find the hydraulic conductivity with a constant\PYGZhy{}head permeameter.}\PYG{l+s+se}{\PYGZbs{}n}\PYG{l+s+se}{\PYGZbs{}n}\PYG{l+s+s2}{Provided are:}\PYG{l+s+s2}{\PYGZdq{}}\PYG{p}{)}

\PYG{n}{L} \PYG{o}{=} \PYG{l+m+mi}{15} \PYG{c+c1}{\PYGZsh{} Length of the permeameter [cm]}
\PYG{n}{A} \PYG{o}{=} \PYG{l+m+mi}{25} \PYG{c+c1}{\PYGZsh{} cross\PYGZhy{}sectional area [cm\PYGZca{}2]}
\PYG{n}{h} \PYG{o}{=} \PYG{l+m+mi}{5} \PYG{c+c1}{\PYGZsh{} hydraulic head [cm]}
\PYG{n}{V} \PYG{o}{=} \PYG{l+m+mi}{100} \PYG{c+c1}{\PYGZsh{} Volume of collected water [mL = cm\PYGZca{}3]}
\PYG{n}{t} \PYG{o}{=} \PYG{l+m+mi}{12} \PYG{c+c1}{\PYGZsh{} time [min]}

\PYG{c+c1}{\PYGZsh{}solution}
\PYG{n}{K1} \PYG{o}{=} \PYG{p}{(}\PYG{n}{V} \PYG{o}{*} \PYG{n}{L}\PYG{p}{)}\PYG{o}{/}\PYG{p}{(}\PYG{n}{A} \PYG{o}{*} \PYG{n}{t} \PYG{o}{*} \PYG{n}{h}\PYG{p}{)}
\PYG{n}{K2} \PYG{o}{=} \PYG{n}{K1}\PYG{o}{/}\PYG{p}{(}\PYG{l+m+mi}{60}\PYG{o}{*}\PYG{l+m+mi}{100}\PYG{p}{)}

\PYG{n+nb}{print}\PYG{p}{(}\PYG{l+s+s2}{\PYGZdq{}}\PYG{l+s+s2}{Length of the permeameter = }\PYG{l+s+si}{\PYGZob{}\PYGZcb{}}\PYG{l+s+s2}{ cm }\PYG{l+s+se}{\PYGZbs{}n}\PYG{l+s+s2}{cross\PYGZhy{}sectional area = }\PYG{l+s+si}{\PYGZob{}\PYGZcb{}}\PYG{l+s+s2}{ cm}\PYG{l+s+se}{\PYGZbs{}u00b2}\PYG{l+s+se}{\PYGZbs{}n}\PYG{l+s+s2}{hydraulic head = }\PYG{l+s+si}{\PYGZob{}\PYGZcb{}}\PYG{l+s+s2}{ cm }\PYG{l+s+se}{\PYGZbs{}n}\PYG{l+s+s2}{Volume = }\PYG{l+s+si}{\PYGZob{}\PYGZcb{}}\PYG{l+s+s2}{ mL }\PYG{l+s+se}{\PYGZbs{}n}\PYG{l+s+s2}{time = }\PYG{l+s+si}{\PYGZob{}\PYGZcb{}}\PYG{l+s+s2}{ min}\PYG{l+s+s2}{\PYGZdq{}}\PYG{o}{.}\PYG{n}{format}\PYG{p}{(}\PYG{n}{L}\PYG{p}{,}\PYG{n}{A}\PYG{p}{,}\PYG{n}{h}\PYG{p}{,}\PYG{n}{V}\PYG{p}{,}\PYG{n}{t}\PYG{p}{)}\PYG{p}{,}\PYG{l+s+s2}{\PYGZdq{}}\PYG{l+s+se}{\PYGZbs{}n}\PYG{l+s+s2}{\PYGZdq{}}\PYG{p}{)}
\PYG{n+nb}{print}\PYG{p}{(}\PYG{l+s+s2}{\PYGZdq{}}\PYG{l+s+s2}{The resulting hydraulic Conductivity is }\PYG{l+s+si}{\PYGZob{}0:2.0e\PYGZcb{}}\PYG{l+s+s2}{ cm/min}\PYG{l+s+s2}{\PYGZdq{}}\PYG{o}{.}\PYG{n}{format}\PYG{p}{(}\PYG{n}{K1}\PYG{p}{)}\PYG{p}{,}
      \PYG{l+s+s2}{\PYGZdq{}}\PYG{l+s+se}{\PYGZbs{}n}\PYG{l+s+s2}{and which is }\PYG{l+s+si}{\PYGZob{}:02.0e\PYGZcb{}}\PYG{l+s+s2}{ m/s}\PYG{l+s+s2}{\PYGZdq{}}\PYG{o}{.}\PYG{n}{format}\PYG{p}{(}\PYG{n}{K2}\PYG{p}{)}\PYG{p}{)}
\end{sphinxVerbatim}

\end{sphinxuseclass}\end{sphinxVerbatimInput}
\begin{sphinxVerbatimOutput}

\begin{sphinxuseclass}{cell_output}
\begin{sphinxVerbatim}[commandchars=\\\{\}]
Let us find the hydraulic conductivity with a constant\PYGZhy{}head permeameter.

Provided are:
Length of the permeameter = 15 cm 
cross\PYGZhy{}sectional area = 25 cm²
hydraulic head = 5 cm 
Volume = 100 mL 
time = 12 min 

The resulting hydraulic Conductivity is 1e+00 cm/min 
and which is 2e\PYGZhy{}04 m/s
\end{sphinxVerbatim}

\end{sphinxuseclass}\end{sphinxVerbatimOutput}

\end{sphinxuseclass}
\sphinxAtStartPar
In falling\sphinxhyphen{}head permeameter as shown in Figure the hydraulic head at the outflow of
the Darcy column is not changing, but the hydraulic head at the inflow is decreasing
with time.

\noindent{\hspace*{\fill}\sphinxincludegraphics[height=255\sphinxpxdimen]{{L4_f6}.png}\hspace*{\fill}}

\sphinxAtStartPar
As a result, the discharge also decreases with time. Rewriting Darcy’s law for a small time interval
\begin{equation*}
\begin{split}
K \frac{\pi d_c^2}{4}\big(h_{in} - h_{out}\big)\frac{1}{L}\text{d}t = \frac{\pi d_t^2}{4}\text{d}h 
\end{split}
\end{equation*}
\sphinxAtStartPar
\(K\) can be obtained from the above equation by separating the variables and then integrating. The resulting expression for \(K\) will be
\begin{equation*}
\begin{split}
K = \frac{d_t^2 L}{d_c^2}\ln \Bigg(\frac{h_{in}(0) - h_{out}}{h_{in}(t)-h_{out}}\Bigg)
\end{split}
\end{equation*}
\sphinxAtStartPar
where, 
\(L\) = length of sample {[}L{]};
\(d_c\) = diameter of sample cylinder {[}L{]};
\(d_t\) = diameter of piezometer tube {[}L{]}; 
\(h_{in}(0)\) = initial hydraulic head at column inlet {[}L{]}; 
\(h_{in}(t)\) = final hydraulic head at column inlet {[}L{]}; 
\(h_{out}\) = hydraulic head at column outlet {[}L{]}; 
\(h_0 = h_{in}(0) - h_{out}\); 
\(h = h_{in}(t) - h_{out}\)

\sphinxAtStartPar
and \(h_{out}\) can be set equal to zero as only head differences are
important. The hydraulic conductivity can be obtained from the above equation using observed time and corresponding heads. Larger experimental time periods are needed for the falling\sphinxhyphen{}head permeameter, in particular if hydraulic conductivity is low. On the other hand, no measurement of discharge or water volume is required.

\sphinxAtStartPar
\sphinxstylestrong{Field methods} of determination of the hydraulic conductivity include tracer tests, auger hole tests, pumping tests of wells, etc. Field experiments are much more complicated and expensive than laboratory tests. Resulting hydraulic conductivities represent averages over an aquifer volume, which is covered by the experiment. The size of this
volume depends on subsurface properties and on the experimental method used.


\subsection{Intrinsic Permeability}
\label{\detokenize{content/flow/L4/14_darcy_law_K:intrinsic-permeability}}
\sphinxAtStartPar
A convenient alternative is to write Darcy’s equation in a form of \sphinxstylestrong{intrinsic permeability}
where the properties of the medium and the fluid are represented explicitly
\begin{equation*}
\begin{split}
v = \frac{-k \cdot \rho\cdot g}{\mu}\frac{\partial h}{\partial L}
\end{split}
\end{equation*}
\sphinxAtStartPar
where, \(k\) is the intrinsic permeability {[}L\(^2\){]}, and \(\eta\) is the dynamic viscosity of fluid {[}ML\(^{-1}\)T\(^{-1}\){]} e.g., (Pa\sphinxhyphen{}S). The relation between \sphinxstyleemphasis{hydraulic conductivity} \sphinxstyleemphasis{intrinsic permeability}, therefore, is
\begin{equation*}
\begin{split}
K = k\cdot \frac{ \rho \cdot g}{\eta}
\end{split}
\end{equation*}
\sphinxAtStartPar
In terms of Kinematic viscosity {[}L\(^2\)T\(^{-1}\)\}{]}, \(\eta = \rho \cdot \nu\), the above relation becomes
\begin{equation*}
\begin{split}
K = k\cdot \frac{ g}{\nu}
\end{split}
\end{equation*}
\sphinxAtStartPar
Both density and viscosity are temperature dependent quantites. Their values in field conditions \(\approx 10 ^\circ\)C and in the laboratory conditions \(\approx 20 ^\circ\)C are:


\begin{savenotes}\sphinxattablestart
\centering
\begin{tabulary}{\linewidth}[t]{|T|T|T|}
\hline

\sphinxAtStartPar

&\sphinxstyletheadfamily 
\sphinxAtStartPar
10°C
&\sphinxstyletheadfamily 
\sphinxAtStartPar
20°C
\\
\hline
\sphinxAtStartPar
density (kg/m\(^3\))
&
\sphinxAtStartPar
999.7
&
\sphinxAtStartPar
999.7
\\
\hline
\sphinxAtStartPar
kinematic viscosity (m\(^2\)/s)
&
\sphinxAtStartPar
1.3101·10\(^{-6}\)
&
\sphinxAtStartPar
1.0105·100\(^{-6}\)
\\
\hline
\sphinxAtStartPar
ynamic viscosity (Pa\(\cdot\)s)
&
\sphinxAtStartPar
1.3097·10\(^{-3}\)
&
\sphinxAtStartPar
1.3097·100\(^{-3}\)
\\
\hline
\sphinxAtStartPar

&
\sphinxAtStartPar

&
\sphinxAtStartPar

\\
\hline
\end{tabulary}
\par
\sphinxattableend\end{savenotes}

\sphinxAtStartPar
The \sphinxstyleemphasis{intrinsic permeability} can be written in terms of specific weight or weight density {[}ML\(^{-2}\)T\(^{-2}\){]} (or in metric unit\sphinxhyphen{} N/m\(^3\)), \(\gamma = \rho\cdot g\) as
\begin{equation*}
\begin{split}
k = \frac{\eta}{\gamma}K
\end{split}
\end{equation*}

\subsection{Example problem}
\label{\detokenize{content/flow/L4/14_darcy_law_K:id5}}
\begin{sphinxadmonition}{note}{Instrinsic Permeability}

\sphinxAtStartPar
The intrinsic permeability of a consolidated rock is \(2,7 \cdot 10^{-11} cm^2\). What is the hydraulic conductivity for water at 20°C
\end{sphinxadmonition}

\begin{sphinxuseclass}{cell}\begin{sphinxVerbatimInput}

\begin{sphinxuseclass}{cell_input}
\begin{sphinxVerbatim}[commandchars=\\\{\}]
\PYG{n+nb}{print}\PYG{p}{(}\PYG{l+s+s2}{\PYGZdq{}}\PYG{l+s+s2}{Let us find the hydraulic conductivity.}\PYG{l+s+s2}{\PYGZdq{}}\PYG{p}{)}

\PYG{n}{k} \PYG{o}{=} \PYG{l+m+mf}{2.7e\PYGZhy{}15} \PYG{c+c1}{\PYGZsh{} intrinsic permeability [m\PYGZca{}2]}
\PYG{n}{rho} \PYG{o}{=} \PYG{l+m+mf}{999.7} \PYG{c+c1}{\PYGZsh{} density at 20°C [kg/m\PYGZca{}3]}
\PYG{n}{eta} \PYG{o}{=} \PYG{l+m+mf}{0.013097} \PYG{c+c1}{\PYGZsh{} dynamic viscosity at 20°C [Pa*s]}
\PYG{n}{g} \PYG{o}{=} \PYG{l+m+mf}{9.81} \PYG{c+c1}{\PYGZsh{} [m/s\PYGZca{}2]}

\PYG{c+c1}{\PYGZsh{} solution}
\PYG{n}{K} \PYG{o}{=} \PYG{n}{k} \PYG{o}{*} \PYG{p}{(}\PYG{p}{(}\PYG{n}{rho} \PYG{o}{*} \PYG{n}{g}\PYG{p}{)}\PYG{o}{/}\PYG{n}{eta}\PYG{p}{)}

\PYG{n+nb}{print}\PYG{p}{(}\PYG{l+s+s2}{\PYGZdq{}}\PYG{l+s+s2}{intrinsic permeability = }\PYG{l+s+si}{\PYGZob{}\PYGZcb{}}\PYG{l+s+s2}{ m}\PYG{l+s+se}{\PYGZbs{}u00b2}\PYG{l+s+se}{\PYGZbs{}n}\PYG{l+s+s2}{density = }\PYG{l+s+si}{\PYGZob{}\PYGZcb{}}\PYG{l+s+s2}{ kg/m}\PYG{l+s+se}{\PYGZbs{}u00b3}\PYG{l+s+se}{\PYGZbs{}n}\PYG{l+s+s2}{dynamic viscosity = }\PYG{l+s+si}{\PYGZob{}\PYGZcb{}}\PYG{l+s+s2}{ Pa*s}\PYG{l+s+s2}{\PYGZdq{}}\PYG{o}{.}\PYG{n}{format}\PYG{p}{(}\PYG{n}{k}\PYG{p}{,} \PYG{n}{rho}\PYG{p}{,} \PYG{n}{eta}\PYG{p}{)}\PYG{p}{,}\PYG{l+s+s2}{\PYGZdq{}}\PYG{l+s+se}{\PYGZbs{}n}\PYG{l+s+s2}{\PYGZdq{}}\PYG{p}{)}
\PYG{n+nb}{print}\PYG{p}{(}\PYG{l+s+s2}{\PYGZdq{}}\PYG{l+s+s2}{The resulting hydraulic conductivity at 20°C is }\PYG{l+s+si}{\PYGZob{}0:0.1e\PYGZcb{}}\PYG{l+s+s2}{ m/s}\PYG{l+s+s2}{\PYGZdq{}}\PYG{o}{.}\PYG{n}{format}\PYG{p}{(}\PYG{n}{K}\PYG{p}{)}\PYG{p}{)}
\end{sphinxVerbatim}

\end{sphinxuseclass}\end{sphinxVerbatimInput}
\begin{sphinxVerbatimOutput}

\begin{sphinxuseclass}{cell_output}
\begin{sphinxVerbatim}[commandchars=\\\{\}]
Let us find the hydraulic conductivity.
intrinsic permeability = 2.7e\PYGZhy{}15 m²
density = 999.7 kg/m³
dynamic viscosity = 0.013097 Pa*s 

The resulting hydraulic conductivity at 20°C is 2.0e\PYGZhy{}09 m/s
\end{sphinxVerbatim}

\end{sphinxuseclass}\end{sphinxVerbatimOutput}

\end{sphinxuseclass}

\subsection{Properties of Intrinsic Permeability}
\label{\detokenize{content/flow/L4/14_darcy_law_K:properties-of-intrinsic-permeability}}\begin{itemize}
\item {} 
\sphinxAtStartPar
The value of intrinsic permeability of a porous medium
equals 1 m\(^2\) if a fluid with dynamic viscosity of 1 Pa\(\cdot\)s can pass through the porous
medium at a Darcy velocity of 1 m/s under a hydrostatic pressure gradient of 1 Pa/m
(horizontal flow).

\item {} 
\sphinxAtStartPar
The intrinsic permeability is \sphinxstyleemphasis{independent} of the fluid moving through the medium and depends only upon the medium properties. Intrinsic
permeability of unconsolidated porous media is roughly proportional to the square of
the pore diameter.

\item {} 
\sphinxAtStartPar
The intrinsic permeability is used primarily when the density or the viscosity of the
fluid varies with position.

\item {} 
\sphinxAtStartPar
The dimension of \(k\) is {[}L\(^2\){]}, but when expressed in m\(^2\) is so small that square
micrometers (\(\mu\) m)\(^2  = 10^{-12}\) m\(^2\) is used. In the petroleum industry it is expressed in \sphinxstylestrong{Darcy}
(symbol: D) with conversion factor to SI units is:  1 D \(= 0.987\cdot 10^{-12}\) m\(^2\).

\item {} 
\sphinxAtStartPar
Intrinsic permeability for a weakly , well and
highly permeable aquifers vary in the range 10\(^{-4}\) to 10\(^{-1}\) D, 10\(^{-1}\) to 10\(^2\) D, and \(> 10^2\) D
respectively

\item {} 
\sphinxAtStartPar
Typical value of conductivity and intrinsic permeability is  provided in the fig (from Todd and Mays, 2004) below.

\end{itemize}

\noindent{\hspace*{\fill}\sphinxincludegraphics[height=500\sphinxpxdimen]{{L4_f7}.png}\hspace*{\fill}}


\section{Darcy velocity and Interstitial velocity}
\label{\detokenize{content/flow/L4/14_darcy_law_K:darcy-velocity-and-interstitial-velocity}}
\sphinxAtStartPar
Darcy velocity \(v\) is the \sphinxstyleemphasis{apparent velocity} or \sphinxstyleemphasis{fictitious velocity} or \sphinxstyleemphasis{Darcy flux} (discharge per
unit area). This value of velocity, often referred to as the \sphinxstyleemphasis{apparent velocity}, is not the
velocity which the water traveling through the pores is experiencing. The velocity \(v\) is
referred to as the Darcy velocity because it assumes that flow occurs through the entire cross section of the material without regard to solids and pores. Actually water can flow
though pores only and the pore spaces vary continuously with location within the
medium. Therefore the actual velocity is nonuniform, involving endless accelerations,
deceleration, and changes in direction. To define the \sphinxstyleemphasis{actual flow velocity} or \sphinxstylestrong{interstitial
velocity}, one must consider the microstructure of the rock material.

\sphinxAtStartPar
For naturally occurring geologic materials, the microstructure cannot be specified three\sphinxhyphen{}
dimensionally; hence, actual velocities can only be quantified statistically.

\noindent{\hspace*{\fill}\sphinxincludegraphics[height=255\sphinxpxdimen]{{L4_f8}.png}\hspace*{\fill}}

\sphinxAtStartPar
Actually, the flow is limited to the pores (white spaces in Figure) only so that the
\sphinxstyleemphasis{average interstitial velocity} or \sphinxstyleemphasis{actual velocity} or \sphinxstyleemphasis{seepage velocity} \((v_s)\) through pore space
can be determined by applying continuity equation
\begin{equation*}
\begin{split}
Q = A_s\cdot v_s = Av
\end{split}
\end{equation*}
\sphinxAtStartPar
Leading to
\begin{equation*}
\begin{split}
v_s = v\frac{A}{A_s} = \frac{v}{\nu_e}
\end{split}
\end{equation*}
\sphinxAtStartPar
where \(A\) = total area of soil specimen, and \(A_s\) = area of pores only (see Figure). The velocity
is divided by effective porosity (\(\nu_e\)) to account for the fact that only a fraction of the total aquifer
volume is available for flow. This indicates that for a sand with a porosity of 33\% the \(v_s
= 3 v\). Thus the average interstitial velocity or seepage velocity or linear velocity through
pore space is never smaller than Darcy velocity. Sometimes, the average flow velocity of
water in the pore space is termed \sphinxstyleemphasis{linear velocity}.


\subsection{Example problem}
\label{\detokenize{content/flow/L4/14_darcy_law_K:id6}}
\begin{sphinxadmonition}{note}{Interstitial velocity}

\sphinxAtStartPar
From the data below obtain the average interstitial velocity in the Darcy’s column.
\end{sphinxadmonition}

\begin{sphinxuseclass}{cell}
\begin{sphinxuseclass}{tag_remove-input}\begin{sphinxVerbatimOutput}

\begin{sphinxuseclass}{cell_output}
\begin{sphinxVerbatim}[commandchars=\\\{\}]
Provided data are:

Flow rate = 0.005 m³/s
total area = 1000 m²
effective porosity = 0.4  
\end{sphinxVerbatim}

\end{sphinxuseclass}\end{sphinxVerbatimOutput}

\end{sphinxuseclass}
\end{sphinxuseclass}
\begin{sphinxuseclass}{cell}\begin{sphinxVerbatimInput}

\begin{sphinxuseclass}{cell_input}
\begin{sphinxVerbatim}[commandchars=\\\{\}]
\PYG{n}{Q} \PYG{o}{=} \PYG{l+m+mf}{0.005} \PYG{c+c1}{\PYGZsh{} Flow rate [m\PYGZca{}3/s]}
\PYG{n}{A} \PYG{o}{=} \PYG{l+m+mi}{1000} \PYG{c+c1}{\PYGZsh{} total area of soil specimen [m\PYGZca{}2]}
\PYG{n}{ne} \PYG{o}{=} \PYG{l+m+mf}{0.4} \PYG{c+c1}{\PYGZsh{} effective porosity [\PYGZhy{}] = }


\PYG{c+c1}{\PYGZsh{}solution}
\PYG{n}{vs} \PYG{o}{=} \PYG{n}{Q} \PYG{o}{/} \PYG{p}{(}\PYG{n}{ne} \PYG{o}{*} \PYG{n}{A}\PYG{p}{)}\PYG{o}{*}\PYG{l+m+mi}{3600}\PYG{o}{*}\PYG{l+m+mi}{24} 
\PYG{n+nb}{print}\PYG{p}{(}\PYG{l+s+s2}{\PYGZdq{}}\PYG{l+s+s2}{The resulting average interstitial velocity is }\PYG{l+s+si}{\PYGZob{}\PYGZcb{}}\PYG{l+s+s2}{ m/d}\PYG{l+s+s2}{\PYGZdq{}}\PYG{o}{.}\PYG{n}{format}\PYG{p}{(}\PYG{n}{vs}\PYG{p}{)}\PYG{p}{)}
\end{sphinxVerbatim}

\end{sphinxuseclass}\end{sphinxVerbatimInput}
\begin{sphinxVerbatimOutput}

\begin{sphinxuseclass}{cell_output}
\begin{sphinxVerbatim}[commandchars=\\\{\}]
The resulting average interstitial velocity is 1.08 m/d
\end{sphinxVerbatim}

\end{sphinxuseclass}\end{sphinxVerbatimOutput}

\end{sphinxuseclass}

\subsection{Typical values of linear velocities}
\label{\detokenize{content/flow/L4/14_darcy_law_K:typical-values-of-linear-velocities}}
\sphinxAtStartPar
Typical values for average interstitial velocities or linear velocities in unconsolidated aquifers are 0.5 m/d to 1 m/d and 30
m/d to 300 m/d in sand and gravel respectively.

\sphinxAtStartPar
Linear velocities in fractured or
karstified aquifers can be rather high along fractures or conduits e.g. 200 m/d to 1.2
km/d along fractures and 3 km/d to 14 km/d in karst conduits.

\sphinxAtStartPar
On the contrary, the
linear velocities are very low in the rock matrix of consolidated aquifers (1 cm/d or
even less).


\subsection{Travel time and Pore volume}
\label{\detokenize{content/flow/L4/14_darcy_law_K:travel-time-and-pore-volume}}
\sphinxAtStartPar
The average linear/pore velocity is the velocity a conservative tracer/dye experiences if
carried by water through the aquifer.

\sphinxAtStartPar
The travel time of water through a column of
length \(L\) is given by
\begin{equation*}
\begin{split}
t = L/v_s
\end{split}
\end{equation*}
\sphinxAtStartPar
It is to be noted that the linear/seepage velocity \(v_s\) has to be used in travel time computation, not the Darcy velocity. The term \sphinxstyleemphasis{residence time} is can also be found to be used for referring to \sphinxstyleemphasis{travel time} .

\sphinxAtStartPar
Yet another important term that is often found in standard texts is \sphinxstyleemphasis{pore volume}. The \_pore volume is the travel time through a column. It can be understood as the \sphinxstyleemphasis{time} needed to replace the water in the column. In this sense, the pore volume is not a \sphinxstyleemphasis{volume} but a \sphinxstyleemphasis{time} (i.e., 1 PV corresponds to the ratio \(L/v_s\) ). The pore volume (PV) is frequently used for \sphinxstyleemphasis{normalisation} purposes in order to better compare column
experiments conducted under different flow velocities. This is mostly done for studying the transport behaviour of chemicals dissolved in water and their arrivals at the column outlets


\subsection{Example problem}
\label{\detokenize{content/flow/L4/14_darcy_law_K:id7}}
\begin{sphinxadmonition}{note}{Travel time and pore volume}

\sphinxAtStartPar
In a tracer test, the breaktrought was measured after 100 h at a distance of 200 m. Determine the linear velocity and the pore volume. what is the darcy velocity, if there is an effective porosity of 0.25.
\end{sphinxadmonition}

\begin{sphinxuseclass}{cell}\begin{sphinxVerbatimInput}

\begin{sphinxuseclass}{cell_input}
\begin{sphinxVerbatim}[commandchars=\\\{\}]
\PYG{n+nb}{print}\PYG{p}{(}\PYG{l+s+s2}{\PYGZdq{}}\PYG{l+s+s2}{Provided are:}\PYG{l+s+se}{\PYGZbs{}n}\PYG{l+s+s2}{\PYGZdq{}}\PYG{p}{)}

\PYG{n}{t} \PYG{o}{=} \PYG{l+m+mi}{100} \PYG{c+c1}{\PYGZsh{} travel time of water [h]}
\PYG{n}{L} \PYG{o}{=} \PYG{l+m+mi}{200} \PYG{c+c1}{\PYGZsh{} Distance from injection to measurement [m]}
\PYG{n}{ne} \PYG{o}{=} \PYG{l+m+mf}{0.25} \PYG{c+c1}{\PYGZsh{} effective porosity [\PYGZhy{}] }


\PYG{c+c1}{\PYGZsh{}solution}
\PYG{n}{vs} \PYG{o}{=} \PYG{n}{L} \PYG{o}{/} \PYG{n}{t}
\PYG{n}{PV} \PYG{o}{=} \PYG{n}{L} \PYG{o}{/} \PYG{n}{vs}
\PYG{n}{v} \PYG{o}{=} \PYG{n}{vs} \PYG{o}{*} \PYG{n}{ne}

\PYG{n+nb}{print}\PYG{p}{(}\PYG{l+s+s2}{\PYGZdq{}}\PYG{l+s+s2}{travel time of water = }\PYG{l+s+si}{\PYGZob{}\PYGZcb{}}\PYG{l+s+s2}{ s}\PYG{l+s+se}{\PYGZbs{}n}\PYG{l+s+s2}{Length = }\PYG{l+s+si}{\PYGZob{}\PYGZcb{}}\PYG{l+s+s2}{ m}\PYG{l+s+se}{\PYGZbs{}u00b2}\PYG{l+s+se}{\PYGZbs{}n}\PYG{l+s+s2}{effective porosity = }\PYG{l+s+si}{\PYGZob{}\PYGZcb{}}\PYG{l+s+s2}{ }\PYG{l+s+s2}{\PYGZdq{}}\PYG{o}{.}\PYG{n}{format}\PYG{p}{(}\PYG{n}{t}\PYG{p}{,} \PYG{n}{L}\PYG{p}{,} \PYG{n}{ne}\PYG{p}{)}\PYG{p}{,}\PYG{l+s+s2}{\PYGZdq{}}\PYG{l+s+se}{\PYGZbs{}n}\PYG{l+s+s2}{\PYGZdq{}}\PYG{p}{)}
\PYG{n+nb}{print}\PYG{p}{(}\PYG{l+s+s2}{\PYGZdq{}}\PYG{l+s+s2}{The linear velocity is }\PYG{l+s+si}{\PYGZob{}\PYGZcb{}}\PYG{l+s+s2}{ m/h }\PYG{l+s+se}{\PYGZbs{}n}\PYG{l+s+s2}{the pore volume is }\PYG{l+s+si}{\PYGZob{}\PYGZcb{}}\PYG{l+s+s2}{ s, and }\PYG{l+s+se}{\PYGZbs{}n}\PYG{l+s+s2}{the darcy velocity is }\PYG{l+s+si}{\PYGZob{}\PYGZcb{}}\PYG{l+s+s2}{ m/h}\PYG{l+s+s2}{\PYGZdq{}}\PYG{o}{.}\PYG{n}{format}\PYG{p}{(}\PYG{n}{vs}\PYG{p}{,} \PYG{n}{PV}\PYG{p}{,} \PYG{n}{v}\PYG{p}{)}\PYG{p}{)}
\end{sphinxVerbatim}

\end{sphinxuseclass}\end{sphinxVerbatimInput}
\begin{sphinxVerbatimOutput}

\begin{sphinxuseclass}{cell_output}
\begin{sphinxVerbatim}[commandchars=\\\{\}]
Provided are:

travel time of water = 100 s
Length = 200 m²
effective porosity = 0.25  

The linear velocity is 2.0 m/h 
the pore volume is 100.0 s, and 
the darcy velocity is 0.5 m/h
\end{sphinxVerbatim}

\end{sphinxuseclass}\end{sphinxVerbatimOutput}

\end{sphinxuseclass}

\section{Chapter Quiz}
\label{\detokenize{content/flow/L4/14_darcy_law_K:chapter-quiz}}
\begin{sphinxuseclass}{cell}
\begin{sphinxuseclass}{tag_remove-input}
\begin{sphinxuseclass}{tag_hide-output}
\end{sphinxuseclass}
\end{sphinxuseclass}
\end{sphinxuseclass}
\sphinxstepscope


\chapter{Aquifer Heterogeneity and Anisotropy}
\label{\detokenize{content/flow/L5/15_het_iso:aquifer-heterogeneity-and-anisotropy}}\label{\detokenize{content/flow/L5/15_het_iso::doc}}
\sphinxAtStartPar
\sphinxstyleemphasis{(The contents presented in this section were re\sphinxhyphen{}developed principally by \sphinxhref{http://web.iitd.ac.in/~chahar/}{Prof. B. R. Chahar} and Dr. P. K. Yadav. The original contents are from Prof. Rudolf Liedl)}


\bigskip\hrule\bigskip



\section{Motivation}
\label{\detokenize{content/flow/L5/15_het_iso:motivation}}
\sphinxAtStartPar
The last lecture introduces aquifer properties such hydraulic conductivity, storativity, porosity. The key assumption in that lecture was that the aquifer is an 1D unit, e.g., the Darcy column, and that its properties do not vary in space. In contrast to these, an aquifer is more accurately represented by a 3D system and its properties vary both in space and directions. In fact variations of hydraulic conductivity (\(K\)), a most critical aquifer quantity, are dominant in most cases. Variations of \(K\) can be observed at very small spatial scales and directions. Thus, aquifer properties that depends on \(K\) also varies. Consequently, aquifer properties such as \(K\) takes a tensor form\sphinxhyphen{} a quantity whose magnitude is space and direction dependent.

\sphinxAtStartPar
In the picture below (\hyperref[\detokenize{content/flow/L5/15_het_iso:aquifer-het}]{Fig.\@ \ref{\detokenize{content/flow/L5/15_het_iso:aquifer-het}}}) the 3\sphinxhyphen{}D spatial variability of aquifer properties is illustrated by a 2\sphinxhyphen{}D vertical cross\sphinxhyphen{}section. The picture of the outcrop show some form of a layered system, with each layer likely possessing individual subsurface properties.

\begin{figure}[htbp]
\centering
\capstart

\noindent\sphinxincludegraphics[scale=0.7]{{L5_f1}.png}
\caption{Aquifer Heterogeneity}\label{\detokenize{content/flow/L5/15_het_iso:aquifer-het}}\end{figure}


\section{Heterogeneity}
\label{\detokenize{content/flow/L5/15_het_iso:heterogeneity}}
\sphinxAtStartPar
A solid or a porous medium is \sphinxstylestrong{homogeneous} if its property do not vary in space. In contrast, the porous medium is \sphinxstylestrong{heterogeneous}, or also sometime termed inhomogeneous, if at least one of its properties varies in space. In groundwater studies, heterogeneity or homogeneity is generally associated with hydraulic conductivity \((K)\) of the aquifer. In many practical applications, properties such as strativity and porosity are treated as spatially constant or homogeneous. This is usually done for two reasons:
\begin{itemize}
\item {} 
\sphinxAtStartPar
the spatial variability of hydraulic conductivity is much more pronounced than that of storativity and porosity, and

\item {} 
\sphinxAtStartPar
very limited data are available of the spatial variability of storativity and porosity.

\end{itemize}

\sphinxAtStartPar
Thus, an aquifer is:
\begin{quote}

\sphinxAtStartPar
homogeneous if \(K (x, y, z) = K\), and
\end{quote}
\begin{quote}

\sphinxAtStartPar
heterogeneous if \(K (x, y, z) \neq K\)
\end{quote}

\sphinxAtStartPar
For heterogeneous aquifer it is not required that \(K\) is varying in all directions, i.e., varying of \(K\) can be restricted to one or two spatial coordinates.

\sphinxAtStartPar
Heterogeneity in aquifer can exist in many configurations. But they are broadly categorized to the following three classes:
\begin{quote}

\sphinxAtStartPar
\sphinxstylestrong{layered heterogeneity}: common in sedimentary systems and unconsolidated deposits.
\end{quote}
\begin{quote}

\sphinxAtStartPar
\sphinxstylestrong{discontinuous heterogeneity}: due to presence of faults, e.g., at the contact between overburden\sphinxhyphen{}bedrock.
\end{quote}
\begin{quote}

\sphinxAtStartPar
\sphinxstylestrong{trending heterogeneity}: Common in the aquifers with active sedimentation processes
\end{quote}

\sphinxAtStartPar
Aquifer properties such as storativity and porosity are affected by heterogeneity but they are often considered spatially constant in practical applications. This is because
\begin{itemize}
\item {} 
\sphinxAtStartPar
The spatial variability of hydraulic conductivity is much more pronounced than any other aquifer property, and

\item {} 
\sphinxAtStartPar
Only limited spatial variability data are available for storativity or porosity.

\end{itemize}

\sphinxAtStartPar
Heterogeneity play a significant role in controlling the flow of groundwater, but its quantification is more relevant to the transport of chemicals in groundwater.


\subsection{Effective Hydraulic Conductivity}
\label{\detokenize{content/flow/L5/15_het_iso:effective-hydraulic-conductivity}}
\sphinxAtStartPar
It is possible to represent the spatial distribution of
hydraulic conductivity in a heterogeneous aquifer by an \sphinxstyleemphasis{average}
value such that steady\sphinxhyphen{}state groundwater discharge remains the
same as in the heterogeneous case. Such obtained \sphinxstyleemphasis{average} \(K\) value is termed as \sphinxstylestrong{effective hydraulic conductivity}. In other words, it can be mentioned that the effective hydraulic conductivity characterizes a
fictitious homogeneous aquifer with the same discharge and the
same overall head difference under steady\sphinxhyphen{}state conditions as some
heterogeneous aquifer to be investigated.

\sphinxAtStartPar
The quantification of \sphinxstyleemphasis{effective hydraulic conductivity} is straight\sphinxhyphen{}forward for perfectly layered systems, which can be seen as an idealized representation of natural layering. In more natural systems with complex heterogeneous configurations, cumbersome mathematical derivations are required to obtain effective \(K\).

\sphinxAtStartPar
In the next few sections, effective hydraulic conductivity for ideal layered system is derived.


\section{Layered Systems}
\label{\detokenize{content/flow/L5/15_het_iso:layered-systems}}

\subsection{Case I \sphinxhyphen{} Flow Parallel to Layering}
\label{\detokenize{content/flow/L5/15_het_iso:case-i-flow-parallel-to-layering}}
\sphinxAtStartPar
A confined aquifer consisting of \(n\) layers is considered. In the set\sphinxhyphen{}up (see \hyperref[\detokenize{content/flow/L5/15_het_iso:flow-parallel}]{Fig.\@ \ref{\detokenize{content/flow/L5/15_het_iso:flow-parallel}}}), the layer boundaries are parallel to each other and groundwater flow is parallel to the layering.

\begin{figure}[htbp]
\centering
\capstart

\noindent\sphinxincludegraphics[scale=0.5]{{L5_f2}.png}
\caption{Flow parallel to layering}\label{\detokenize{content/flow/L5/15_het_iso:flow-parallel}}\end{figure}

\sphinxAtStartPar
The analysis is required to consider the aquifer as a \sphinxstyleemphasis{single entire aquifer unit} and a set of \sphinxstyleemphasis{layered aquifer unit}. Data of both units are to be considered.

\sphinxAtStartPar
With these informations available, using volumetric budget and Darcy’s law the \sphinxstyleemphasis{effective hydraulic conductivity} \(K\) can be quantified from the following steps:

\sphinxAtStartPar
I. Total thickness: \(m = \sum\limits_{i=1}^n m_i\)

\sphinxAtStartPar
II. Volumetric budget: \(Q =  \sum\limits_{i=1}^n Q_i\)

\sphinxAtStartPar
III. Head loss in \(i^{th}\) layer: \(\Delta h_i = \Delta h\)

\sphinxAtStartPar
IV. Darcy’s law for \(i^{th}\) layer: \(Q_i = - wm_iK_i\frac{\Delta h}{L} \)

\sphinxAtStartPar
V. Darcy’s law for the homogeneous aquifer to replace the layered system: \(Q = -wmK\frac{\Delta h}{L}\)

\begin{sphinxadmonition}{note}{Note:}
\sphinxAtStartPar
As flow is parallel to the layering, the head loss in individual layer equals the total head loss (step. 3)
\end{sphinxadmonition}

\sphinxAtStartPar
Summing the discharge in each layer (step 4) will result to the discharge of the homogeneous aquifer (step 5), i.e., we can equate step 4 and step 5 as:
\begin{equation*}
\begin{split}
- w \cdot m \cdot K \cdot \frac{\Delta h}{L} = \sum\limits_{i=1}^n\bigg(- w \cdot m_i \cdot K_i\cdot\frac{\Delta h}{L}\bigg)
\end{split}
\end{equation*}
\sphinxAtStartPar
Constant quantities from the right\sphinxhyphen{}hand side can be taken out of summation and equated with the left side. This leads to:
\begin{equation*}
\begin{split}
{- w}\cdot m \cdot K \cdot \frac{\Delta h}{L} = {- w} \cdot \frac{\Delta h}{L} \cdot \sum\limits_{i=1}^n( m_iK_i)
\end{split}
\end{equation*}
\sphinxAtStartPar
providing
\begin{equation*}
\begin{split}
m\cdot K = \sum\limits_{i=1}^n( m_iK_i)
\end{split}
\end{equation*}
\sphinxAtStartPar
As a result, the effective hydraulic conductivity of a layered system when the flow is parallel to the layering equals
\begin{equation*}
\begin{split}
K = \frac{\sum\limits_{i=1}^n(m_i K_i)}{m}
\end{split}
\end{equation*}
\sphinxAtStartPar
In the above equation, effective hydraulic conductivity \(K\) is obtained as the \sphinxstyleemphasis{weighted arithmetic average} of layer conductivities \(K_i\). Weights correspond to relative thickness \(m_i/m\). Thus, the \sphinxstyleemphasis{largest term} in the sum contributes most to the arithmetic average. Thus, the effective hydraulic conductivity \(K\) can be approximated from
\begin{equation*}
\begin{split}
K \approx \frac{\max (m_i\cdot K_i)}{m}
\end{split}
\end{equation*}

\subsection{Example problem}
\label{\detokenize{content/flow/L5/15_het_iso:example-problem}}
\begin{sphinxadmonition}{note}{Flow parallel to layering}

\sphinxAtStartPar
Calculate the effective hydraulic conductivity of the layer system consisting of 3 layers if the flow is parallel to the stratification.
\end{sphinxadmonition}

\begin{sphinxuseclass}{cell}
\begin{sphinxuseclass}{tag_remove-input}\begin{sphinxVerbatimOutput}

\begin{sphinxuseclass}{cell_output}
\begin{sphinxVerbatim}[commandchars=\\\{\}]
\PYG{Color+ColorBold}{ Provided are: } 

thickness of layer 1 = 3 m
thickness of layer 2 = 2.5 m
thickness of layer 3 = 1.75 m

conductivity of layer 1 = 3.5e\PYGZhy{}03 m/s
conductivity of layer 2 = 2.0e\PYGZhy{}02 m/s
conductivity of layer 3 5.0e\PYGZhy{}04 m/s
\end{sphinxVerbatim}

\end{sphinxuseclass}\end{sphinxVerbatimOutput}

\end{sphinxuseclass}
\end{sphinxuseclass}
\begin{sphinxuseclass}{cell}\begin{sphinxVerbatimInput}

\begin{sphinxuseclass}{cell_input}
\begin{sphinxVerbatim}[commandchars=\\\{\}]
\PYG{k+kn}{import} \PYG{n+nn}{numpy} \PYG{k}{as} \PYG{n+nn}{np}

\PYG{c+c1}{\PYGZsh{}Thickness of i\PYGZhy{}th layer [m]}
\PYG{n}{m1} \PYG{o}{=} \PYG{l+m+mi}{3} 
\PYG{n}{m2} \PYG{o}{=} \PYG{l+m+mf}{2.5}
\PYG{n}{m3} \PYG{o}{=} \PYG{l+m+mf}{1.75}

\PYG{c+c1}{\PYGZsh{}conductivity of i\PYGZhy{}th layer [m/s]}
\PYG{n}{K1} \PYG{o}{=} \PYG{l+m+mf}{3.5e\PYGZhy{}3}
\PYG{n}{K2} \PYG{o}{=} \PYG{l+m+mf}{2e\PYGZhy{}2}
\PYG{n}{K3} \PYG{o}{=} \PYG{l+m+mf}{5e\PYGZhy{}4}

\PYG{c+c1}{\PYGZsh{}intermediate calculation}
\PYG{n}{m} \PYG{o}{=} \PYG{n}{m1}\PYG{o}{+}\PYG{n}{m2}\PYG{o}{+}\PYG{n}{m3}

\PYG{c+c1}{\PYGZsh{}solution}
\PYG{n}{K} \PYG{o}{=} \PYG{p}{(}\PYG{n}{m1}\PYG{o}{*}\PYG{n}{K1}\PYG{o}{+}\PYG{n}{m2}\PYG{o}{*}\PYG{n}{K2}\PYG{o}{+}\PYG{n}{m3}\PYG{o}{*}\PYG{n}{K3}\PYG{p}{)}\PYG{o}{/}\PYG{n}{m}

\PYG{n+nb}{print}\PYG{p}{(}\PYG{l+s+s2}{\PYGZdq{}}\PYG{l+s+se}{\PYGZbs{}n}\PYG{l+s+se}{\PYGZbs{}033}\PYG{l+s+s2}{[1mSolution:}\PYG{l+s+se}{\PYGZbs{}033}\PYG{l+s+s2}{[0m}\PYG{l+s+se}{\PYGZbs{}n}\PYG{l+s+s2}{The resulting hydraulic conductivity of the layer system is }\PYG{l+s+se}{\PYGZbs{}033}\PYG{l+s+s2}{[1m}\PYG{l+s+si}{\PYGZob{}:02.1e\PYGZcb{}}\PYG{l+s+s2}{ m/s}\PYG{l+s+se}{\PYGZbs{}033}\PYG{l+s+s2}{[0m.}\PYG{l+s+s2}{\PYGZdq{}}\PYG{o}{.}\PYG{n}{format}\PYG{p}{(}\PYG{n}{K}\PYG{p}{)}\PYG{p}{)}
\end{sphinxVerbatim}

\end{sphinxuseclass}\end{sphinxVerbatimInput}
\begin{sphinxVerbatimOutput}

\begin{sphinxuseclass}{cell_output}
\begin{sphinxVerbatim}[commandchars=\\\{\}]
\PYG{Color+ColorBold}{Solution:}
The resulting hydraulic conductivity of the layer system is \PYG{Color+ColorBold}{8.5e\PYGZhy{}03 m/s}.
\end{sphinxVerbatim}

\end{sphinxuseclass}\end{sphinxVerbatimOutput}

\end{sphinxuseclass}

\subsection{Case 2 \sphinxhyphen{} Flow Perpendicular to Layering}
\label{\detokenize{content/flow/L5/15_het_iso:case-2-flow-perpendicular-to-layering}}
\sphinxAtStartPar
In the second case the the flow \sphinxstyleemphasis{perpendicular} to the layering is considered. Just as in the \sphinxstyleemphasis{parallel} flow case, the aquifer in this case also is a confined aquifer with \(n\) layers (see \hyperref[\detokenize{content/flow/L5/15_het_iso:flow-perpendicular}]{Fig.\@ \ref{\detokenize{content/flow/L5/15_het_iso:flow-perpendicular}}}).

\begin{figure}[htbp]
\centering
\capstart

\noindent\sphinxincludegraphics[scale=0.5]{{L5_f3}.png}
\caption{Flow perpendicular to layering}\label{\detokenize{content/flow/L5/15_het_iso:flow-perpendicular}}\end{figure}

\sphinxAtStartPar
These data combined with \sphinxstyleemphasis{equation of continuity} and \sphinxstyleemphasis{Darcy’s Law} can be used to obtain the effective hydraulic conductivity of the system in which flow is perpendicular to the layers. The following steps are required:

\sphinxAtStartPar
I. Total thickness: \(m = \sum\limits_{i=1}^n m_i \)

\sphinxAtStartPar
II.   Equation of continuity: \(Q_i = Q\)

\sphinxAtStartPar
II.  Cross\sphinxhyphen{}sectional area for layer: \(A_i = A \)

\sphinxAtStartPar
IV. Decomposition of head loss: \(\Delta h = \sum\limits_{i=1}^n \Delta h_i \)

\sphinxAtStartPar
V. Darcy’s law for layer \(i\):
\begin{equation*}
\begin{split}Q_i = - A_i K_i \frac{\Delta h_i}{m_i}\end{split}
\end{equation*}\begin{equation*}
\begin{split}\Delta h_i = -\frac{Q_i m_i}{A_i K_i} = - \frac{Q m_i}{A K_i}\end{split}
\end{equation*}
\sphinxAtStartPar
VI. Similarly the Darcy’s law for the homogeneous aquifer to replace the layered system:
\begin{equation*}
\begin{split}Q = - A K \frac{\Delta h}{m}\end{split}
\end{equation*}\begin{equation*}
\begin{split}\Delta h = - \frac{Q m}{A K }\end{split}
\end{equation*}
\sphinxAtStartPar
VII. The head loss from step 4 can be equated with the sum of head loss of each layered unit (from step 5), i.e.,
\begin{equation*}
\begin{split}
\frac{Q\cdot m}{A\cdot K}
=\sum\limits_{i=1}^n\frac{Q\cdot m}{A\cdot K}
\end{split}
\end{equation*}
\sphinxAtStartPar
The constants \(Q\) and \(A\) can be taken out of the summation, this leads to
\begin{equation*}
\begin{split}
\frac{ {-Q}\cdot m}{ A\cdot K}
=\frac{ {-Q}}{A}\sum\limits_{i=1}^n\frac{m_i}{K_i}
\end{split}
\end{equation*}
\sphinxAtStartPar
As a result, the effective hydraulic conductivity of a layered system for a flow perpendicular to the layering equals
\begin{equation*}
\begin{split}
K = \frac{m}{\sum\limits_{i=1}^n\frac{m_i}{K_i}}
\end{split}
\end{equation*}
\sphinxAtStartPar
In the above equation, effective hydraulic conductivity \(K\) is obtained as the \sphinxstyleemphasis{weighted harmonic average} of layer conductivities \(K_i\). Weights
correspond to relative thicknesses \(m_i/m\). Thus, the largest term in the sum contributes most to
the harmonic average and therefore, the effective hydraulic conductivity can be approximated from
\begin{equation*}
\begin{split}
K \approx \frac{m}{{\max\big(\frac{m_i}{K_i}}\big)}
\end{split}
\end{equation*}

\subsection{Example problem}
\label{\detokenize{content/flow/L5/15_het_iso:id1}}
\begin{sphinxadmonition}{note}{Flow perpendicular to layering}

\sphinxAtStartPar
Calculate the effective hydraulic conductivity of the layer system consisting of 3 layers if the flow is perpendicular to the layering.
\end{sphinxadmonition}

\begin{sphinxuseclass}{cell}
\begin{sphinxuseclass}{tag_remove-input}\begin{sphinxVerbatimOutput}

\begin{sphinxuseclass}{cell_output}
\begin{sphinxVerbatim}[commandchars=\\\{\}]
\PYG{Color+ColorBold}{ Provided are:}
thickness of layer 1 = 3 m
thickness of layer 2 = 2.5 m
thickness of layer 3 = 1.75 m

conductivity of layer 1 = 3.5e\PYGZhy{}03 m/s
conductivity of layer 2 = 2.0e\PYGZhy{}02 m/s
conductivity of layer 3 5.0e\PYGZhy{}04 m/s
\end{sphinxVerbatim}

\end{sphinxuseclass}\end{sphinxVerbatimOutput}

\end{sphinxuseclass}
\end{sphinxuseclass}
\begin{sphinxuseclass}{cell}\begin{sphinxVerbatimInput}

\begin{sphinxuseclass}{cell_input}
\begin{sphinxVerbatim}[commandchars=\\\{\}]
\PYG{c+c1}{\PYGZsh{}Thickness of i\PYGZhy{}th layer [m]}
\PYG{n}{m1} \PYG{o}{=} \PYG{l+m+mi}{3} 
\PYG{n}{m2} \PYG{o}{=} \PYG{l+m+mf}{2.5}
\PYG{n}{m3} \PYG{o}{=} \PYG{l+m+mf}{1.75}

\PYG{c+c1}{\PYGZsh{}conductivity of i\PYGZhy{}th layer [m/s]}
\PYG{n}{K1} \PYG{o}{=} \PYG{l+m+mf}{3.5e\PYGZhy{}3}
\PYG{n}{K2} \PYG{o}{=} \PYG{l+m+mf}{2e\PYGZhy{}2}
\PYG{n}{K3} \PYG{o}{=} \PYG{l+m+mf}{5e\PYGZhy{}4}

\PYG{c+c1}{\PYGZsh{}intermediate calculation}
\PYG{n}{m} \PYG{o}{=} \PYG{n}{m1}\PYG{o}{+}\PYG{n}{m2}\PYG{o}{+}\PYG{n}{m3}

\PYG{c+c1}{\PYGZsh{}solution}
\PYG{n}{K} \PYG{o}{=} \PYG{n}{m}\PYG{o}{/}\PYG{p}{(}\PYG{n}{m1}\PYG{o}{/}\PYG{n}{K1}\PYG{o}{+}\PYG{n}{m2}\PYG{o}{/}\PYG{n}{K2}\PYG{o}{+}\PYG{n}{m3}\PYG{o}{/}\PYG{n}{K3}\PYG{p}{)}


\PYG{n+nb}{print}\PYG{p}{(}\PYG{l+s+s2}{\PYGZdq{}}\PYG{l+s+se}{\PYGZbs{}n}\PYG{l+s+se}{\PYGZbs{}033}\PYG{l+s+s2}{[1mSolution:}\PYG{l+s+se}{\PYGZbs{}033}\PYG{l+s+s2}{[0m}\PYG{l+s+se}{\PYGZbs{}n}\PYG{l+s+s2}{The resulting hydraulic conductivity of the layer system is }\PYG{l+s+se}{\PYGZbs{}033}\PYG{l+s+s2}{[1m}\PYG{l+s+si}{\PYGZob{}:02.1e\PYGZcb{}}\PYG{l+s+s2}{ m/s}\PYG{l+s+se}{\PYGZbs{}033}\PYG{l+s+s2}{[0m.}\PYG{l+s+s2}{\PYGZdq{}}\PYG{o}{.}\PYG{n}{format}\PYG{p}{(}\PYG{n}{K}\PYG{p}{)}\PYG{p}{)}
\end{sphinxVerbatim}

\end{sphinxuseclass}\end{sphinxVerbatimInput}
\begin{sphinxVerbatimOutput}

\begin{sphinxuseclass}{cell_output}
\begin{sphinxVerbatim}[commandchars=\\\{\}]
\PYG{Color+ColorBold}{Solution:}
The resulting hydraulic conductivity of the layer system is \PYG{Color+ColorBold}{1.6e\PYGZhy{}03 m/s}.
\end{sphinxVerbatim}

\end{sphinxuseclass}\end{sphinxVerbatimOutput}

\end{sphinxuseclass}

\subsection{Summary: Effective Conductivity of Layered Aquifers}
\label{\detokenize{content/flow/L5/15_het_iso:summary-effective-conductivity-of-layered-aquifers}}\begin{itemize}
\item {} 
\sphinxAtStartPar
\sphinxstylestrong{For flow parallel to layering:} 
Effective hydraulic conductivity equals the \sphinxstyleemphasis{weighted arithmetic mean} of layer conductivities.

\item {} 
\sphinxAtStartPar
\sphinxstylestrong{Flow perpendicular to layering:} 
Effective hydraulic conductivity equals the \sphinxstyleemphasis{weighted harmonic mean} of layer conductivities.

\item {} 
\sphinxAtStartPar
\sphinxstylestrong{Weights} in both cases are given by relative layer thicknesses \(m_i/m\)

\item {} 
\sphinxAtStartPar
It can be mathematically shown that the harmonic mean of a set of
positive numbers cannot exceed the arithmetic mean of the same set. Both means are identical only if all numbers in the set are identical.
Apart from this very special case, we have

\end{itemize}

\begin{sphinxadmonition}{note}{Note:}\begin{quote}

\sphinxAtStartPar
\sphinxstyleemphasis{\sphinxstylestrong{\sphinxstyleemphasis{harmonic mean < arithmetic mean.}}}
\end{quote}
\end{sphinxadmonition}

\sphinxAtStartPar
This implies that the flow direction perpendicular to the layering is associated with a smaller effective hydraulic conductivity than the flow direction parallel to the layering.


\section{Hydraulic Resistance}
\label{\detokenize{content/flow/L5/15_het_iso:hydraulic-resistance}}
\sphinxAtStartPar
The \sphinxstyleemphasis{Hydraulic Resistance} \((R)\) provide an alternative approach to express conductivity. It is defined as a reciprocal to hydraulic conductivity \((K)\), i.e.,
\begin{equation*}
\begin{split}
R = \frac{1}{K}
\end{split}
\end{equation*}
\sphinxAtStartPar
This implies that large \(K\) corresponds to small \(R\) and vice\sphinxhyphen{}versa. Considerations about effective hydraulic conductivities of layered
aquifers can be transferred to hydraulic resistances by recalling
that the arithmetic mean of positive numbers coincides with the
harmonic mean of reciprocal numbers and vice versa. This can be used in the following manner:
\begin{itemize}
\item {} 
\sphinxAtStartPar
\sphinxstylestrong{For \sphinxstyleemphasis{flow parallel to layering}:}

\end{itemize}

\sphinxAtStartPar
As Effective hydraulic conductivity equals the weighted arithmetic mean of layer conductivities, the reciprocal of this, i.e., the effective hydraulic resistance, will equal the weighted harmonic mean of layer resistances.

\sphinxAtStartPar
Furthermore, if all layer thicknesses are identical \((m_i =\) const.) and flow is parallel to layering, the largest discharge passes through the layer with
highest hydraulic conductivity (smallest hydraulic resistance). In this case, the discharge through each layer is proportional to layer conductivity or inversely proportional to layer resistance.
\begin{itemize}
\item {} 
\sphinxAtStartPar
\sphinxstylestrong{For \sphinxstyleemphasis{flow perpendicular to layering}:}

\end{itemize}

\sphinxAtStartPar
In this case the effective hydraulic conductivity equals the weighted harmonic
mean of layer conductivities. This leads to Effective hydraulic resistance equaling the weighted arithmetic mean of layer resistances.

\sphinxAtStartPar
Similar to the \sphinxstyleemphasis{flow parallel to the layering,} if all layer thicknesses are identical (\(m_i =\) const.) and flow is perpendicular to layering, the largest hydraulic gradient is across the layer with lowest hydraulic conductivity (highest hydraulic resistance). In this case, the \sphinxstyleemphasis{head gradient} across each layer is proportional to layer resistance or inversely proportional to layer conductivity.


\subsection{Example problem}
\label{\detokenize{content/flow/L5/15_het_iso:id2}}
\begin{sphinxadmonition}{note}{Hydraulic Resistance}

\sphinxAtStartPar
Find the hydraulic resistance with the given hydraulic counductivity
\end{sphinxadmonition}

\begin{sphinxuseclass}{cell}
\begin{sphinxuseclass}{tag_remove-input}\begin{sphinxVerbatimOutput}

\begin{sphinxuseclass}{cell_output}
\begin{sphinxVerbatim}[commandchars=\\\{\}]
\PYG{Color+ColorBold}{ Provided are:} 

Hydraulic conductivity = 5.0e\PYGZhy{}04 m/s
\end{sphinxVerbatim}

\end{sphinxuseclass}\end{sphinxVerbatimOutput}

\end{sphinxuseclass}
\end{sphinxuseclass}
\begin{sphinxuseclass}{cell}\begin{sphinxVerbatimInput}

\begin{sphinxuseclass}{cell_input}
\begin{sphinxVerbatim}[commandchars=\\\{\}]
\PYG{n}{K} \PYG{o}{=} \PYG{l+m+mf}{5e\PYGZhy{}4} \PYG{c+c1}{\PYGZsh{} m/s, hydraulic conductivity}
\PYG{c+c1}{\PYGZsh{}solution}
\PYG{n}{R} \PYG{o}{=} \PYG{l+m+mi}{1}\PYG{o}{/}\PYG{n}{K}
\PYG{n+nb}{print}\PYG{p}{(}\PYG{l+s+s2}{\PYGZdq{}}\PYG{l+s+se}{\PYGZbs{}n}\PYG{l+s+se}{\PYGZbs{}033}\PYG{l+s+s2}{[1mSolution:}\PYG{l+s+se}{\PYGZbs{}033}\PYG{l+s+s2}{[0m}\PYG{l+s+se}{\PYGZbs{}n}\PYG{l+s+se}{\PYGZbs{}n}\PYG{l+s+s2}{The resulting hydraulic resistance is }\PYG{l+s+se}{\PYGZbs{}033}\PYG{l+s+s2}{[1m}\PYG{l+s+si}{\PYGZob{}:02.1e\PYGZcb{}}\PYG{l+s+s2}{ s/m}\PYG{l+s+se}{\PYGZbs{}033}\PYG{l+s+s2}{[0m.}\PYG{l+s+s2}{\PYGZdq{}}\PYG{o}{.}\PYG{n}{format}\PYG{p}{(}\PYG{n}{R}\PYG{p}{)}\PYG{p}{)}
\end{sphinxVerbatim}

\end{sphinxuseclass}\end{sphinxVerbatimInput}
\begin{sphinxVerbatimOutput}

\begin{sphinxuseclass}{cell_output}
\begin{sphinxVerbatim}[commandchars=\\\{\}]
\PYG{Color+ColorBold}{Solution:}

The resulting hydraulic resistance is \PYG{Color+ColorBold}{2.0e+03 s/m}.
\end{sphinxVerbatim}

\end{sphinxuseclass}\end{sphinxVerbatimOutput}

\end{sphinxuseclass}

\section{Aquifer Anisotropy}
\label{\detokenize{content/flow/L5/15_het_iso:aquifer-anisotropy}}
\sphinxAtStartPar
A solid or a porous medium is \sphinxstylestrong{isotropic} if its (all) properties are independent of \sphinxstyleemphasis{direction.} Conversely, a solid or a porous medium is \sphinxstylestrong{anisotropic} if at least one of its property is dependent on direction. Thus, isotropic (or anisotropic) property of porous media refines the concept of homogeneity (or heterogeneity). The key difference being that anisotropy of aquifer is associated only with hydraulic conductivity as other aquifer properties like storativity or porosity cannot depend on direction. Groundwater flow (and more so solute transport) is affected by anisotropy. However, in unconsolidated aquifers the impact of heterogeneity is usually more important.

\sphinxAtStartPar
Figure below (\hyperref[\detokenize{content/flow/L5/15_het_iso:iso-anisotropy}]{Fig.\@ \ref{\detokenize{content/flow/L5/15_het_iso:iso-anisotropy}}}) explains the concept of isotropy more clearly. The direction dependency of \(K\) is represented by the arrows diagram.

\begin{figure}[htbp]
\centering
\capstart

\noindent\sphinxincludegraphics[scale=0.4]{{L5_f4}.png}
\caption{Isotropy and Anisotropy in aquifers}\label{\detokenize{content/flow/L5/15_het_iso:iso-anisotropy}}\end{figure}


\subsection{Anisotropy and scale effects}
\label{\detokenize{content/flow/L5/15_het_iso:anisotropy-and-scale-effects}}
\sphinxAtStartPar
The effective hydraulic conductivity of layered aquifers was shown earlier to depend on the orientation of the flow direction relative to layering, i.e., parallel versus perpendicular. On a larger scale, it may not be possible to identify or resolve heterogeneities associated with example thin layers, small lenses, shape and orientation of grains (see figure above). Nevertheless, \(K\) is found to be direction\sphinxhyphen{}dependent when groundwater flow is quantified at the larger scale. In these case, small\sphinxhyphen{}scale heterogeneity (e.g., due to layering) expresses itself as anisotropy of hydraulic conductivity at the larger scale.


\subsection{Indices and functional arguments to describe anisotropy}
\label{\detokenize{content/flow/L5/15_het_iso:indices-and-functional-arguments-to-describe-anisotropy}}
\sphinxAtStartPar
One could consider heterogeneity as a property of entirety. While function arguments point towards heterogeneity, \sphinxstyleemphasis{indices} are
used to express that hydraulic conductivity depends on direction. Considering a 3\sphinxhyphen{}D Cartesian coordinate system, the directional dependent hydraulic conductivity can be represented by \(K_x\), \(K_y\) and \(K_z\) in parallel with the \(x-\), \(y-\) and \(z-\)axis, respectively. Within these, one could distinguish between horizontal conductivities (\(K_x\) and \(K_y\)) with the vertical conductivity \(K_z\). More often, anisotropy is observed only between horizontal and vertical directions (i.e., along \(x\) or \(y\) and \(z\) directions), while isotropy is observed along horizontal directions (\(x\) and \(y\) direction). Thus, symbols \(K_h\) representing \(K_x = K_y\) and \(K_v\) representing \(K_z\) can be used to denote horizontal and vertical hydraulic conductivity, respectively.


\subsection{Concept of the hydraulic conductivity ellipse}
\label{\detokenize{content/flow/L5/15_het_iso:concept-of-the-hydraulic-conductivity-ellipse}}
\sphinxAtStartPar
Let us consider an aquifer with horizontal conductivity \(K_h\) and vertical hydraulic conductivity \(K_v\). Let us assume that the flow in the aquifer is in some arbitrary direction characterized by the angle \(\theta\) between the flow direction and horizontal plane (see \hyperref[\detokenize{content/flow/L5/15_het_iso:k-ellipse}]{Fig.\@ \ref{\detokenize{content/flow/L5/15_het_iso:k-ellipse}}}).

\begin{figure}[htbp]
\centering
\capstart

\noindent\sphinxincludegraphics[scale=0.55]{{L5_f5}.png}
\caption{hydraulic conductivity ellipse}\label{\detokenize{content/flow/L5/15_het_iso:k-ellipse}}\end{figure}

\sphinxAtStartPar
In this case, it can be shown that the effective hydraulic conductivity \(K\) is
\begin{equation*}
\begin{split}
K = \frac{1}{\frac{\cos^2\theta}{K_h}+\frac{\sin^2\theta}{K_v}} 
\end{split}
\end{equation*}
\sphinxAtStartPar
If the angle \(\theta\) is varied, the above equation defines an ellipse (also called “hydraulic conductivity ellipse”) with semi\sphinxhyphen{}axes equal to \(\surd{K_h}\) and \(\surd{K_v}\), respectively. The square root of \(K\) can be visualised by the length of a line segment parallel to the direction of flow. This line segment extends from the centre to the perimeter of the ellipse.


\subsection{Example problem}
\label{\detokenize{content/flow/L5/15_het_iso:id3}}
\begin{sphinxadmonition}{note}{Hydraulic Resistance}

\sphinxAtStartPar
Find resulting hydraulic conductivity from provided horizontal and vertical conductivities.
\end{sphinxadmonition}

\begin{sphinxuseclass}{cell}
\begin{sphinxuseclass}{tag_remove-input}\begin{sphinxVerbatimOutput}

\begin{sphinxuseclass}{cell_output}
\begin{sphinxVerbatim}[commandchars=\\\{\}]
\PYG{Color+ColorBold}{Provided are:}

horizontal hydraulic conductivity = 1.0e\PYGZhy{}03 m/s
vertical hydraulic conductivity = 1.0e\PYGZhy{}04 m/s
angle = 50°
\end{sphinxVerbatim}

\end{sphinxuseclass}\end{sphinxVerbatimOutput}

\end{sphinxuseclass}
\end{sphinxuseclass}
\begin{sphinxuseclass}{cell}\begin{sphinxVerbatimInput}

\begin{sphinxuseclass}{cell_input}
\begin{sphinxVerbatim}[commandchars=\\\{\}]
\PYG{c+c1}{\PYGZsh{}solution}
\PYG{n}{K} \PYG{o}{=} \PYG{l+m+mi}{1} \PYG{o}{/}\PYG{p}{(}\PYG{p}{(}\PYG{n}{np}\PYG{o}{.}\PYG{n}{cos}\PYG{p}{(}\PYG{n}{theta}\PYG{p}{)}\PYG{o}{*}\PYG{o}{*}\PYG{l+m+mi}{2}\PYG{o}{/}\PYG{n}{Kh}\PYG{p}{)}\PYG{o}{+}\PYG{p}{(}\PYG{n}{np}\PYG{o}{.}\PYG{n}{sin}\PYG{p}{(}\PYG{n}{theta}\PYG{p}{)}\PYG{o}{*}\PYG{o}{*}\PYG{l+m+mi}{2}\PYG{o}{/}\PYG{n}{Kv}\PYG{p}{)}\PYG{p}{)}


\PYG{n+nb}{print}\PYG{p}{(}\PYG{l+s+s2}{\PYGZdq{}}\PYG{l+s+se}{\PYGZbs{}033}\PYG{l+s+s2}{[1mSolution:}\PYG{l+s+se}{\PYGZbs{}033}\PYG{l+s+s2}{[0m}\PYG{l+s+se}{\PYGZbs{}n}\PYG{l+s+se}{\PYGZbs{}n}\PYG{l+s+s2}{The resulting hydraulic conductivity is }\PYG{l+s+se}{\PYGZbs{}033}\PYG{l+s+s2}{[1m}\PYG{l+s+si}{\PYGZob{}:02.1e\PYGZcb{}}\PYG{l+s+s2}{ m/s}\PYG{l+s+se}{\PYGZbs{}033}\PYG{l+s+s2}{[0m.}\PYG{l+s+s2}{\PYGZdq{}}\PYG{o}{.}\PYG{n}{format}\PYG{p}{(}\PYG{n}{K}\PYG{p}{)}\PYG{p}{)}
\end{sphinxVerbatim}

\end{sphinxuseclass}\end{sphinxVerbatimInput}
\begin{sphinxVerbatimOutput}

\begin{sphinxuseclass}{cell_output}
\begin{sphinxVerbatim}[commandchars=\\\{\}]
\PYG{Color+ColorBold}{Solution:}

The resulting hydraulic conductivity is \PYG{Color+ColorBold}{6.2e\PYGZhy{}04 m/s}.
\end{sphinxVerbatim}

\end{sphinxuseclass}\end{sphinxVerbatimOutput}

\end{sphinxuseclass}

\section{Combining Heterogeneity and Anisotropy}
\label{\detokenize{content/flow/L5/15_het_iso:combining-heterogeneity-and-anisotropy}}
\sphinxAtStartPar
The distinction between\_Homogeneous, Heterogeneous, isotropic and anisotropic aquifer or any combination of them can be more clearly understood from the figure (for vertical \(x-z\) plane).

\begin{figure}[htbp]
\centering
\capstart

\noindent\sphinxincludegraphics[scale=0.55]{{L5_f6}.png}
\caption{Heterogeneity and Anisotropy in aquifer}\label{\detokenize{content/flow/L5/15_het_iso:het-aniso}}\end{figure}


\section{Chapter Quiz}
\label{\detokenize{content/flow/L5/15_het_iso:chapter-quiz}}
\begin{sphinxuseclass}{cell}
\begin{sphinxuseclass}{tag_remove-input}
\begin{sphinxuseclass}{tag_hide-output}
\end{sphinxuseclass}
\end{sphinxuseclass}
\end{sphinxuseclass}
\sphinxstepscope


\chapter{Steady\sphinxhyphen{}State Groundwater flow in 3D}
\label{\detokenize{content/flow/L6/16_darcy_law_3D:steady-state-groundwater-flow-in-3d}}\label{\detokenize{content/flow/L6/16_darcy_law_3D::doc}}
\sphinxAtStartPar
\sphinxstyleemphasis{(The contents presented in this section were re\sphinxhyphen{}developed principally by Dr. P. K. Yadav. The original contents are from Prof. Rudolf Liedl)}


\bigskip\hrule\bigskip



\section{Motivation}
\label{\detokenize{content/flow/L6/16_darcy_law_3D:motivation}}
\sphinxAtStartPar
This lecture shows how Darcy’s law can be used for variety of groundwater problems, e.g., in isotropic and anisotropic aquifers, multiple dimension problems \sphinxhyphen{} 2D and 3D. The lecture will also introduce the basic concepts on visualizing groundwater flow, e.g., using streamlines, flow\sphinxhyphen{}nets, and present their uses for example using isochrones and delineating protection zones.


\section{Darcy’s Law in Isotropic Aquifers}
\label{\detokenize{content/flow/L6/16_darcy_law_3D:darcy-s-law-in-isotropic-aquifers}}

\subsection{Summarizing Hydraulic Head}
\label{\detokenize{content/flow/L6/16_darcy_law_3D:summarizing-hydraulic-head}}
\sphinxAtStartPar
The Darcy’s column experiment is associated with one\sphinxhyphen{}dimensional (1D) steady\sphinxhyphen{}state or time independent flow system. In addition, the experiments as we discussed in lecture 4, dealt with homogeneous (space dependency) and isotropic (direction dependency)  porous media. The consequence of these is that the hydraulic head \(h\) in a Darcy column therefore depends on a single space coordinate, say \(x\) that is oriented along the column length. Therefore \(h = h(x)\) in this case.

\sphinxAtStartPar
Now considering the real/natural aquifer, which is inherently three dimensional (3D), the hydraulic head is then a function of three space coordinates, i.e., \(h = h(x,y,z)\). But the 3D natural problem can be simplified to 2D problems based on the fact that vertical flow components (e.g., the aquifer depth) is very much smaller than horizontal flow components (the aquifer width or its length), i.e., \(h = h(x,y)\) considering \(z\) as vertical components. The 2D approach, instead of 3D, has been found to appropriately quantify groundwater flow systems. The application of 2D or 1D for quantifying in groundwater flow system have be justified for each case.  For example the layered aquifers, that was discussed in lecture 5, can be easily treated as a 2D system.

\begin{sphinxadmonition}{note}{Note:}
\sphinxAtStartPar
The hydraulic head (\(h\)) in any case (1D, 2D or 3D), is the sum of hydrostatic pressure head (\(\psi\)) and the elevation head (\(z\)) and has a dimension of length {[}L{]}, i.e.,
\begin{equation*}
\begin{split}
h = \psi + z
\end{split}
\end{equation*}\end{sphinxadmonition}


\section{Isolines and Isosurfaces}
\label{\detokenize{content/flow/L6/16_darcy_law_3D:isolines-and-isosurfaces}}
\sphinxAtStartPar
We make a slight detour from the Darcy’s law and quantifying groundwater flow system to \sphinxstylestrong{visualizing} the groundwater flow. As can be expected, visualizing groundwater is a challenging task. This is largely because sub\sphinxhyphen{}surface, compared to surface, cannot be mapped with very high resolution. However, it is possible to connect aquifer properties such as \(h\) hydraulic head to visualize groundwater flow system. Particularly used to visualize groundwater are:
\begin{quote}

\sphinxAtStartPar
\sphinxstylestrong{Isolines}: The curves (in 2D space) where a physical quantity (e.g., \(h\)) assumes a certain value. Isolines are thus suitable for visualizing 2D problems.
\end{quote}
\begin{quote}

\sphinxAtStartPar
\sphinxstylestrong{Isosurface}: The surfaces (in 3D space) where a physical quantity assumes a certain value. Isosurface addresses the 3D groundwater problems.
\end{quote}

\sphinxAtStartPar
The representation of \sphinxstyleemphasis{isolines} yields a \sphinxstyleemphasis{map.} Isolines are therefore frequently superimposed on an already existing geographical map, e.g., the topographical map (see \hyperref[\detokenize{content/flow/L6/16_darcy_law_3D:isolines}]{Fig.\@ \ref{\detokenize{content/flow/L6/16_darcy_law_3D:isolines}}}). The simultaneous representation of several isolines/isosurfaces can be confusing. Thus intervals between values represented by these representations are constant.

\sphinxAtStartPar
Isosurfaces are in general confusing to interpret due to presence of the third dimensions. As such often 2D cross\sphinxhyphen{}section through isosurfaces are depicted.

\begin{figure}[htbp]
\centering
\capstart

\noindent\sphinxincludegraphics[scale=0.3]{{L06_f1}.png}
\caption{Head Isolines}\label{\detokenize{content/flow/L6/16_darcy_law_3D:isolines}}\end{figure}


\section{Hydrologic Triangle}
\label{\detokenize{content/flow/L6/16_darcy_law_3D:hydrologic-triangle}}
\sphinxAtStartPar
The \sphinxstylestrong{hydrologic triangle} is a method that is used to approximate head isolines when head values are available at some distinct locations. As the name suggests, at least three head data are required to be known for approximating isolines. The process used in approximating is as follows:
\begin{itemize}
\item {} 
\sphinxAtStartPar
First the known head data points are connected with straight lines (dashed lines in the figure \hyperref[\detokenize{content/flow/L6/16_darcy_law_3D:hydro-tri}]{Fig.\@ \ref{\detokenize{content/flow/L6/16_darcy_law_3D:hydro-tri}}}).

\item {} 
\sphinxAtStartPar
Linear interpolation is then performed with pre\sphinxhyphen{}defined head intervals (\( \Delta h = 1 m \) in the figure).

\item {} 
\sphinxAtStartPar
Finally, points with identical head values are connected to obtain the isolines (solid lines in the figure).

\end{itemize}

\begin{figure}[htbp]
\centering
\capstart

\noindent\sphinxincludegraphics[scale=0.5]{{L06_f2}.png}
\caption{Hydrologic Triangle}\label{\detokenize{content/flow/L6/16_darcy_law_3D:hydro-tri}}\end{figure}


\subsection{Applying the Hydrologic Triangle}
\label{\detokenize{content/flow/L6/16_darcy_law_3D:applying-the-hydrologic-triangle}}
\sphinxAtStartPar
In practical cases more than the just three observation points are available. This implies a triangulation of the investigation area as shown in the figure (\hyperref[\detokenize{content/flow/L6/16_darcy_law_3D:hydro-tri2}]{Fig.\@ \ref{\detokenize{content/flow/L6/16_darcy_law_3D:hydro-tri2}}}). Linear interpolation of heads is again performed along the dashed lines. Points with identical head values are connected to obtain the isolines. The connection lines do not have to be strictly a straight line as was the case in \hyperref[\detokenize{content/flow/L6/16_darcy_law_3D:hydro-tri}]{Fig.\@ \ref{\detokenize{content/flow/L6/16_darcy_law_3D:hydro-tri}}}. In addition to straight line segments, the connected line can be \sphinxstyleemphasis{smooth} curves as shown in the \hyperref[\detokenize{content/flow/L6/16_darcy_law_3D:hydro-tri2}]{Fig.\@ \ref{\detokenize{content/flow/L6/16_darcy_law_3D:hydro-tri2}}}.

\begin{figure}[htbp]
\centering
\capstart

\noindent\sphinxincludegraphics[scale=0.35]{{L06_f3}.png}
\caption{Applying the Hydrologic Triangle}\label{\detokenize{content/flow/L6/16_darcy_law_3D:hydro-tri2}}\end{figure}


\subsection{Flow Direction}
\label{\detokenize{content/flow/L6/16_darcy_law_3D:flow-direction}}
\sphinxAtStartPar
Based on Darcy’s law it is known that the flow is directed from the higher head towards the lower head. Though Darcy’s law is based on 1D, the same concept can be extended to the 2D and 3D flow. Thus in the figure, (\hyperref[\detokenize{content/flow/L6/16_darcy_law_3D:hydro-tri3}]{Fig.\@ \ref{\detokenize{content/flow/L6/16_darcy_law_3D:hydro-tri3}}}), the flow direction can be assumed to be directed from the top towards the base of the triangle (red arrow in the figure). But this remains assumptions so far.

\begin{figure}[htbp]
\centering
\capstart

\noindent\sphinxincludegraphics[scale=0.4]{{L06_f4}.png}
\caption{Flow Direction in Hydrologic Triangle}\label{\detokenize{content/flow/L6/16_darcy_law_3D:hydro-tri3}}\end{figure}

\begin{sphinxadmonition}{attention}{Attention:}
\sphinxAtStartPar
What is the correct extension of Darcy’s law\sphinxhyphen{} in 2D and 3D systems?
\end{sphinxadmonition}


\section{Hydraulic Gradient in 2D and 3D systems}
\label{\detokenize{content/flow/L6/16_darcy_law_3D:hydraulic-gradient-in-2d-and-3d-systems}}
\sphinxAtStartPar
The hydraulic gradient appears as a major constituent of Darcy’s law for 1D flow. In the 1D case the hydraulic gradient is given as the ratio of difference of heads at two points in the space and the distance between those points. This distance measures also the flow length. Since head is direction oriented, the hydraulic gradient in the 2D or the 3D flow must be a vector quantity. This combines both magnitude and direction that fits the definition of hydraulic gradient. Therefore hydraulic gradient in 2D or 3D is achieved by employing partial derivatives of hydraulic head with respect to Cartesian coordinates. Mathematically, this can be represented as:
\begin{equation*}
\begin{split}
\text{grad}h=\nabla h = \begin{bmatrix}
\frac{\partial h}{\partial x}
\\ 
\frac{\partial h}{\partial y}
\\ 
\frac{\partial h}{\partial z}
\end{bmatrix}   
\end{split}
\end{equation*}
\sphinxAtStartPar
\(\text{grad}h\) in the equation above is a vector pointing in the direction of the steepest \sphinxstylestrong{increase} of \(h\). In groundwater/hydrogeology study, in contrast to the mathematical definition, \(\text{grad}h\) refer to the direction of steepest \sphinxstylestrong{decrease} of hydraulic head. This can be observed in \hyperref[\detokenize{content/flow/L6/16_darcy_law_3D:isolines}]{Fig.\@ \ref{\detokenize{content/flow/L6/16_darcy_law_3D:isolines}}}


\subsection{Darcy’s Law (Isotropic Aquifer)}
\label{\detokenize{content/flow/L6/16_darcy_law_3D:darcy-s-law-isotropic-aquifer}}
\sphinxAtStartPar
The concept used for hydraulic gradient for 2D/3D discussed above can be extended to specific discharge or Darcy velocity \(v_f\). Similar to hydraulic gradient, \(v_f\) is a vector with two or three components for 2D and 3D systems, respectively. Thus the 3D Darcy velocity vector is
\begin{equation*}
\begin{split}
v_f = \begin{bmatrix}
v_{fx}
\\ 
v_{fy}
\\ 
v_{fz}
\end{bmatrix}   
\end{split}
\end{equation*}
\sphinxAtStartPar
Therefore in higher dimension systems, the Darcy law can be stated as
\begin{equation*}
\begin{split}
v_f = - K \cdot \text{grad}h
\end{split}
\end{equation*}
\sphinxAtStartPar
Based on this definition, the hydraulic conductivity (\(K\)) is
\begin{quote}

\sphinxAtStartPar
\sphinxstylestrong{Heterogeneous aquifer}: \(K = K(x,y,z\), and
\end{quote}
\begin{quote}

\sphinxAtStartPar
\sphinxstylestrong{Homogeneous aquifer}: \(K = \text{constant}\)
\end{quote}

\sphinxAtStartPar
For \sphinxstylestrong{isotropic} aquifers the Darcy velocity is oriented opposite to the hydraulic gradient vector (due to minus sign in Darcy’s law). This is explained further below.


\subsection{Example problem}
\label{\detokenize{content/flow/L6/16_darcy_law_3D:example-problem}}
\begin{sphinxadmonition}{note}{Hydraulic gradient in 2D}

\sphinxAtStartPar
Obtain the hydraulic gradients of the following 2D system shown in the figure below
\end{sphinxadmonition}

\begin{figure}[htbp]
\centering
\capstart

\noindent\sphinxincludegraphics[scale=0.4]{{L06_f5}.png}
\caption{Example 1}\label{\detokenize{content/flow/L6/16_darcy_law_3D:ex-1}}\end{figure}

\sphinxAtStartPar
\sphinxstylestrong{Solution}

\sphinxAtStartPar
Known for the 2D system is:
\begin{equation*}
\begin{split}
\nabla h = \begin{bmatrix} \frac{\partial h}{\partial x}
\\
\frac{\partial h}{\partial y} \end{bmatrix}
\end{split}
\end{equation*}
\sphinxAtStartPar
In which for fixed \(y\),
\begin{equation*}
\begin{split}
\frac{\partial h}{\partial x}\approx \frac{h(x+\Delta x, y)- h(x,y)}{\Delta x}
\end{split}
\end{equation*}
\sphinxAtStartPar
and for fixed \(x\),
\begin{equation*}
\begin{split}
\frac{\partial h}{\partial y}\approx \frac{h(x, y + \Delta y)- h(x,y)}{\Delta y}
\end{split}
\end{equation*}
\sphinxAtStartPar
Then from the given figure:

\begin{sphinxuseclass}{cell}\begin{sphinxVerbatimInput}

\begin{sphinxuseclass}{cell_input}
\begin{sphinxVerbatim}[commandchars=\\\{\}]
\PYG{c+c1}{\PYGZsh{} solution}

\PYG{k+kn}{import} \PYG{n+nn}{numpy} \PYG{k}{as} \PYG{n+nn}{np}

\PYG{n}{h\PYGZus{}xdelx} \PYG{o}{=} \PYG{l+m+mf}{40.8} \PYG{c+c1}{\PYGZsh{} m, head at W2}
\PYG{n}{h\PYGZus{}x} \PYG{o}{=} \PYG{l+m+mf}{41.2} \PYG{c+c1}{\PYGZsh{} m, head at W1}
\PYG{n}{Lx} \PYG{o}{=} \PYG{l+m+mi}{2} \PYG{c+c1}{\PYGZsh{} m, length between W1 and W2}

\PYG{n}{h\PYGZus{}xdely} \PYG{o}{=} \PYG{l+m+mf}{40.6} \PYG{c+c1}{\PYGZsh{} m, head at W2}
\PYG{n}{h\PYGZus{}y} \PYG{o}{=} \PYG{l+m+mf}{40.2} \PYG{c+c1}{\PYGZsh{} m, head at W1}
\PYG{n}{Ly} \PYG{o}{=} \PYG{l+m+mi}{2} \PYG{c+c1}{\PYGZsh{} m, length between W1 and W2}

\PYG{c+c1}{\PYGZsh{} calculate}
\PYG{n}{Delh\PYGZus{}x} \PYG{o}{=} \PYG{p}{(}\PYG{n}{h\PYGZus{}xdelx} \PYG{o}{\PYGZhy{}} \PYG{n}{h\PYGZus{}x}\PYG{p}{)}\PYG{o}{/}\PYG{n}{Lx} \PYG{c+c1}{\PYGZsh{} (\PYGZhy{}), hydraulic head along x\PYGZhy{}axis}
\PYG{n}{Delh\PYGZus{}y} \PYG{o}{=} \PYG{p}{(}\PYG{n}{h\PYGZus{}xdely} \PYG{o}{\PYGZhy{}} \PYG{n}{h\PYGZus{}y}\PYG{p}{)}\PYG{o}{/}\PYG{n}{Ly} \PYG{c+c1}{\PYGZsh{} (\PYGZhy{}), hydraulic head along y\PYGZhy{}axis}

\PYG{c+c1}{\PYGZsh{}print}
\PYG{n+nb}{print}\PYG{p}{(}\PYG{l+s+s2}{\PYGZdq{}}\PYG{l+s+s2}{Hydraulic gradient along x\PYGZhy{}axis is }\PYG{l+s+si}{\PYGZob{}0:0.3f\PYGZcb{}}\PYG{l+s+se}{\PYGZbs{}n}\PYG{l+s+s2}{\PYGZdq{}}\PYG{o}{.}\PYG{n}{format}\PYG{p}{(}\PYG{n}{Delh\PYGZus{}x}\PYG{p}{)}\PYG{p}{)}
\PYG{n+nb}{print}\PYG{p}{(}\PYG{l+s+s2}{\PYGZdq{}}\PYG{l+s+s2}{Hydraulic gradient along y\PYGZhy{}axis is }\PYG{l+s+si}{\PYGZob{}0:0.3f\PYGZcb{}}\PYG{l+s+se}{\PYGZbs{}n}\PYG{l+s+s2}{\PYGZdq{}}\PYG{o}{.}\PYG{n}{format}\PYG{p}{(}\PYG{n}{Delh\PYGZus{}y}\PYG{p}{)}\PYG{p}{)}
\end{sphinxVerbatim}

\end{sphinxuseclass}\end{sphinxVerbatimInput}
\begin{sphinxVerbatimOutput}

\begin{sphinxuseclass}{cell_output}
\begin{sphinxVerbatim}[commandchars=\\\{\}]
Hydraulic gradient along x\PYGZhy{}axis is \PYGZhy{}0.200

Hydraulic gradient along y\PYGZhy{}axis is 0.200
\end{sphinxVerbatim}

\end{sphinxuseclass}\end{sphinxVerbatimOutput}

\end{sphinxuseclass}
\begin{sphinxadmonition}{attention}{Attention:}
\sphinxAtStartPar
In the above example the orientation of co\sphinxhyphen{}ordinate axis is important. The change in orientation of axis, e.g., \(y\)\sphinxhyphen{}axis pointing downward will make also hydraulic gradient along \(y\) negative.
\end{sphinxadmonition}


\section{Streamlines and Flow Nets}
\label{\detokenize{content/flow/L6/16_darcy_law_3D:streamlines-and-flow-nets}}

\subsection{Basics}
\label{\detokenize{content/flow/L6/16_darcy_law_3D:basics}}
\sphinxAtStartPar
\sphinxstylestrong{Streamlines} or \sphinxstylestrong{flowlines} are curves which are in each point tangential to the flow direction. There is no flow component perpendicular to the streamlines. As a consequence, streamlines are perpendicular to head iso\sphinxhyphen{}surface of head isolines if the aquifer is \sphinxstylestrong{isotropic}

\begin{figure}[htbp]
\centering
\capstart

\noindent\sphinxincludegraphics[scale=0.3]{{L06_f6}.png}
\caption{Streamlines and equipotential lines}\label{\detokenize{content/flow/L6/16_darcy_law_3D:streamlines}}\end{figure}

\sphinxAtStartPar
\sphinxstylestrong{Flow nets} consist of a set of isolines (or isosurfaces) and a set of streamlines. Flow nets are ususally employed for 2D flow scenarios only. The flow behaviour is illustrated by covering the investigation area with a  mesh of isolines and streamlines.

\begin{figure}[htbp]
\centering
\capstart

\noindent\sphinxincludegraphics[scale=0.45]{{L06_f7}.png}
\caption{Flow nets}\label{\detokenize{content/flow/L6/16_darcy_law_3D:flownets}}\end{figure}


\subsection{Radial Flow Near a Well}
\label{\detokenize{content/flow/L6/16_darcy_law_3D:radial-flow-near-a-well}}
\sphinxAtStartPar
In this example steady\sphinxhyphen{}state water abstraction from a pumping well, e.g., for drinking water supply, is considered. The hydraulic conductivity is assumed to be spatially constant, i.e., the aquifer is \sphinxstyleemphasis{homogeneous}. Now assuming that there are no other hydraulic impacts in the system, the pumping of water from the groundwater will lead to a radially symmetric \sphinxstylestrong{drawdown} of \sphinxstyleemphasis{hydraulic heads} (this will be further discussed in future lecture 8). In this case the \sphinxstyleemphasis{streamlines} radially approaches the pumping well, and the \sphinxstyleemphasis{head isolines} are circles with the well in the centre.

\begin{figure}[htbp]
\centering
\capstart

\noindent\sphinxincludegraphics[scale=0.45]{{L06_f8}.png}
\caption{Radial flow near a well}\label{\detokenize{content/flow/L6/16_darcy_law_3D:id1}}\end{figure}


\subsection{Superposition of Uniform and Radial Flow}
\label{\detokenize{content/flow/L6/16_darcy_law_3D:superposition-of-uniform-and-radial-flow}}
\sphinxAtStartPar
The \sphinxstylestrong{dividing streamline} (solid black line) represents the boundary of the \sphinxstylestrong{well capture zone}. The flow velocities of the uniform flow and the radial flow exactly compensate each other at the \sphinxstylestrong{stagnation point.} The resulting flow velocity equal zero.

\begin{figure}[htbp]
\centering
\capstart

\noindent\sphinxincludegraphics[scale=0.25]{{L06_f9}.png}
\caption{Superposition of Radial flow near a well}\label{\detokenize{content/flow/L6/16_darcy_law_3D:radial-flow}}\end{figure}


\subsection{Some Rules for Drawing Flow Nets}
\label{\detokenize{content/flow/L6/16_darcy_law_3D:some-rules-for-drawing-flow-nets}}
\sphinxAtStartPar
Drawing a flow net for a certain domain requires information about domain boundaries.
\begin{quote}

\sphinxAtStartPar
\sphinxstylestrong{Impermeable} boundaries represents a \sphinxstyleemphasis{streamline.}
\end{quote}

\begin{figure}[htbp]
\centering
\capstart

\noindent\sphinxincludegraphics[scale=0.25]{{L06_f10a}.png}
\caption{No flow boundary}\label{\detokenize{content/flow/L6/16_darcy_law_3D:flow-net}}\end{figure}
\begin{quote}

\sphinxAtStartPar
\sphinxstylestrong{Constant head} boundaries with given head values represents \sphinxstyleemphasis{isolines.}
\end{quote}

\begin{figure}[htbp]
\centering
\capstart

\noindent\sphinxincludegraphics[scale=0.45]{{L06_f10b}.png}
\caption{Constant head boundary}\label{\detokenize{content/flow/L6/16_darcy_law_3D:id2}}\end{figure}
\begin{quote}

\sphinxAtStartPar
\sphinxstylestrong{Water table} with and without evapotranspiration and recharge: Isolines and streamlines are drawn only for the saturated zone, i.e., they do not cross the water table.
\end{quote}

\begin{figure}[htbp]
\centering
\capstart

\noindent\sphinxincludegraphics[scale=0.35]{{L06_f10c}.png}
\caption{Water table as boundary}\label{\detokenize{content/flow/L6/16_darcy_law_3D:id3}}\end{figure}
\begin{quote}

\sphinxAtStartPar
Isolines do not intersect each other. Streamlines do not intersect each other.
\end{quote}
\begin{quote}

\sphinxAtStartPar
Streamlines are never closed (\sphinxstylestrong{circular}). They start at an inflow boundary and end at an outflow boundary.
\end{quote}
\begin{quote}

\sphinxAtStartPar
Adjacent isolines and streamlines should form \sphinxstylestrong{curvilinear squares}.
\end{quote}


\subsection{Example problem}
\label{\detokenize{content/flow/L6/16_darcy_law_3D:id4}}
\begin{sphinxadmonition}{note}{Flow parallel to layering}

\sphinxAtStartPar
Considering \(K = 6\) cm/s and \(w = 50 \)cm, the width normal to plane, find the total discharge from the flow net provided in figure below: 
(source: \sphinxurl{https://gw-project.org/})

\begin{figure}[H]
\centering
\capstart

\noindent\sphinxincludegraphics[scale=0.4]{{L06_f17}.png}
\caption{Example problem 2\sphinxcode{\sphinxupquote{ }}}\label{\detokenize{content/flow/L6/16_darcy_law_3D:ex2}}\end{figure}
\end{sphinxadmonition}

\begin{sphinxuseclass}{cell}\begin{sphinxVerbatimInput}

\begin{sphinxuseclass}{cell_input}
\begin{sphinxVerbatim}[commandchars=\\\{\}]
\PYG{c+c1}{\PYGZsh{}\PYGZsh{}\PYGZsh{} solution}

\PYG{n}{n\PYGZus{}f} \PYG{o}{=} \PYG{l+m+mi}{3} \PYG{c+c1}{\PYGZsh{} numer of flow tubes}
\PYG{n}{n\PYGZus{}d} \PYG{o}{=} \PYG{l+m+mi}{6} \PYG{c+c1}{\PYGZsh{} number of head drops in the flow net}
\PYG{n}{w} \PYG{o}{=} \PYG{l+m+mi}{50} \PYG{c+c1}{\PYGZsh{} cm, width lateral to the plane}
\PYG{n}{K} \PYG{o}{=} \PYG{l+m+mf}{0.4} \PYG{c+c1}{\PYGZsh{} cm/s, conductivity}
\PYG{n}{h\PYGZus{}in} \PYG{o}{=} \PYG{l+m+mi}{50} \PYG{c+c1}{\PYGZsh{} cm, head inlet}
\PYG{n}{h\PYGZus{}out} \PYG{o}{=} \PYG{l+m+mi}{44} \PYG{c+c1}{\PYGZsh{} cm, head outlet}

\PYG{c+c1}{\PYGZsh{}interim calculation}
\PYG{n}{H} \PYG{o}{=} \PYG{n}{h\PYGZus{}in}\PYG{o}{\PYGZhy{}}\PYG{n}{h\PYGZus{}out}

\PYG{n}{Q\PYGZus{}t} \PYG{o}{=} \PYG{n}{K}\PYG{o}{*}\PYG{n}{H}\PYG{o}{*}\PYG{n}{n\PYGZus{}f}\PYG{o}{/}\PYG{n}{n\PYGZus{}d}\PYG{o}{*}\PYG{n}{w} \PYG{c+c1}{\PYGZsh{} cm\PYGZca{}3/s, the total discharge}

\PYG{n+nb}{print}\PYG{p}{(}\PYG{l+s+s2}{\PYGZdq{}}\PYG{l+s+s2}{The total discharge out of domain is }\PYG{l+s+si}{\PYGZob{}0:0.0f\PYGZcb{}}\PYG{l+s+s2}{\PYGZdq{}}\PYG{o}{.}\PYG{n}{format}\PYG{p}{(}\PYG{n}{Q\PYGZus{}t}\PYG{p}{)}\PYG{p}{,} \PYG{l+s+s2}{\PYGZdq{}}\PYG{l+s+s2}{cm}\PYG{l+s+se}{\PYGZbs{}u00b3}\PYG{l+s+s2}{/s}\PYG{l+s+s2}{\PYGZdq{}}\PYG{p}{)}
\end{sphinxVerbatim}

\end{sphinxuseclass}\end{sphinxVerbatimInput}
\begin{sphinxVerbatimOutput}

\begin{sphinxuseclass}{cell_output}
\begin{sphinxVerbatim}[commandchars=\\\{\}]
The total discharge out of domain is 60 cm³/s
\end{sphinxVerbatim}

\end{sphinxuseclass}\end{sphinxVerbatimOutput}

\end{sphinxuseclass}

\section{Isochrones and Protection Zones.}
\label{\detokenize{content/flow/L6/16_darcy_law_3D:isochrones-and-protection-zones}}

\subsection{Ischrones}
\label{\detokenize{content/flow/L6/16_darcy_law_3D:ischrones}}
\sphinxAtStartPar
The \sphinxstylestrong{isochrones} are curves of identical travel times. This is analogous to head isolines, in which the curves provides point of identical head. Similar to isolines, isochrones and streamlines do not \sphinxstyleemphasis{necessarily} intersect each other at an angle of 90\(^\circ\), i.e., curvilinear intersection between these curves are also possible.Isochrones are mostly used for delineating water protection zones. These zones are mostly part of the local water\sphinxhyphen{}use regulations.

\begin{sphinxuseclass}{cell}
\begin{sphinxuseclass}{tag_hide-input}\begin{sphinxVerbatimOutput}

\begin{sphinxuseclass}{cell_output}
\begin{sphinxVerbatim}[commandchars=\\\{\}]
interactive(children=(FloatLogSlider(value=0.0003, description=\PYGZsq{}hydraulic conductivity [m/s]:\PYGZsq{}, max=0.0, min=\PYGZhy{}…
\end{sphinxVerbatim}

\begin{sphinxVerbatim}[commandchars=\\\{\}]
\PYGZlt{}function \PYGZus{}\PYGZus{}main\PYGZus{}\PYGZus{}.uniform\PYGZus{}flow(K, ne, m, v, Q, t1, t2, t3)\PYGZgt{}
\end{sphinxVerbatim}

\end{sphinxuseclass}\end{sphinxVerbatimOutput}

\end{sphinxuseclass}
\end{sphinxuseclass}

\subsection{Example of Protection Zones}
\label{\detokenize{content/flow/L6/16_darcy_law_3D:example-of-protection-zones}}
\sphinxAtStartPar
The figure below show a schematic of a well protection zone. The main goals of the protection zones are:
\begin{quote}

\sphinxAtStartPar
To protect aquifers against pollution
\end{quote}
\begin{quote}

\sphinxAtStartPar
Simple and robust zones based on aquifer pollution vulnerability and source protection perimeters
\end{quote}
\begin{quote}

\sphinxAtStartPar
A sensible balance between the protection of groundwater resources (aquifer as a whole) and specific sources (e.g., wells)
\end{quote}

\begin{figure}[htbp]
\centering
\capstart

\noindent\sphinxincludegraphics[scale=0.65]{{L06_f13}.png}
\caption{A schematic of the groundwater protection zone.}\label{\detokenize{content/flow/L6/16_darcy_law_3D:pzones}}\end{figure}

\sphinxAtStartPar
As can be observed in the figure, the zones are based on the travel time, and each travel time is divided into protection of groundwater resources against different activities.

\sphinxAtStartPar
In Germany law \sphinxhref{https://www.lbeg.niedersachsen.de/download/51154}{DVGW W\sphinxhyphen{}101} exitst for groundwater protection zones. As per the German rule, the groundwater protection zones is divided into three different zones:
\begin{quote}

\sphinxAtStartPar
\sphinxstylestrong{Zone I} : The \sphinxstyleemphasis{immediate} protection zone \sphinxhyphen{} at least 10 m from the water well, not less than 20 m in the upstream direction of a spring.
\end{quote}
\begin{quote}

\sphinxAtStartPar
\sphinxstylestrong{Zone II} : The \sphinxstyleemphasis{inner} protection zone \sphinxhyphen{} 50\sphinxhyphen{}day travel time but not less than 100 m from the well or spring.
\end{quote}
\begin{quote}

\sphinxAtStartPar
\sphinxstylestrong{Zone III} : The \sphinxstyleemphasis{outer} protection zone \sphinxhyphen{} entire contribution zone of the groundwater catchment area (maybe sub\sphinxhyphen{}divided see figure)
\end{quote}

\begin{figure}[htbp]
\centering
\capstart

\noindent\sphinxincludegraphics[scale=0.35]{{L06_f14}.png}
\caption{Groundwater protection zone in Germany}\label{\detokenize{content/flow/L6/16_darcy_law_3D:pzones-de}}\end{figure}


\subsection{Well Capture Zones in Natural Aquifers}
\label{\detokenize{content/flow/L6/16_darcy_law_3D:well-capture-zones-in-natural-aquifers}}
\begin{figure}[htbp]
\centering
\capstart

\noindent\sphinxincludegraphics[scale=0.55]{{L06_f15}.png}
\caption{Groundwater protection zone natural aquifers}\label{\detokenize{content/flow/L6/16_darcy_law_3D:pzones-nat}}\end{figure}


\section{Darcy’s Law in Anisotropic}
\label{\detokenize{content/flow/L6/16_darcy_law_3D:darcy-s-law-in-anisotropic}}

\subsection{Principal Axes of Hydraulic Conductivity}
\label{\detokenize{content/flow/L6/16_darcy_law_3D:principal-axes-of-hydraulic-conductivity}}
\sphinxAtStartPar
In anisotropic aquifer refers to those aquifers whose properties vary with direction. This means that properties such as hydraulic conductivity will have maximum value along a direction and the minimum value along the other direction. It has been found that (also depicted in fig {\color{red}\bfseries{}:refnum:`axes\_an`}) these two directions are perpendicular to each other. This is true in general, not only for the layered systems that was covered in the previous lecture

\begin{figure}[htbp]
\centering
\capstart

\noindent\sphinxincludegraphics[scale=0.3]{{L06_f16}.png}
\caption{Hydraulic conductivity in anisotropic aquifers}\label{\detokenize{content/flow/L6/16_darcy_law_3D:axes-an}}\end{figure}

\sphinxAtStartPar
The three \sphinxstyleemphasis{principal axes} of hydraulic conductivity are oriented
\begin{itemize}
\item {} 
\sphinxAtStartPar
along the direction with maximum hydraulic conductivity

\item {} 
\sphinxAtStartPar
along the direction with hydraulic conductivity

\item {} 
\sphinxAtStartPar
perpendicular to both

\end{itemize}


\subsection{Formulation of Darcy’s Law in Anisotropic Aquifers}
\label{\detokenize{content/flow/L6/16_darcy_law_3D:formulation-of-darcy-s-law-in-anisotropic-aquifers}}
\sphinxAtStartPar
If the \sphinxstyleemphasis{principal axes} of conductivity coincide with the axes of a Cartesian coordinate system, Darcy’s law for anisotropic aquifer is given as
\begin{equation*}
\begin{split}
\begin{bmatrix}
v_{fx}\\ v_{fy}\\ v_{fz}
\end{bmatrix}
= -\begin{bmatrix}
K_x, 0,  0 \\
0, K_y, 0\\
0, 0, K_z
\end{bmatrix}
\cdot
\begin{bmatrix}
\frac{\partial h}{\partial x}\\ \frac{\partial h}{\partial y} \\ \frac{\partial h}{\partial z}
\end{bmatrix}
\end{split}
\end{equation*}
\sphinxAtStartPar
In this equation, hydraulic conductivity has to be written as a matrix. To be precise, \(K\) represents a \sphinxstylestrong{tensor}, i.e., a quantity which is subject to certain rules under coordinate transforms (from Cartesian to cylinder coordinates, for example). These rules guarantee that Darcy’s law can also be applied in coordinate systems other than Cartesian.

\sphinxAtStartPar
If the aquifer is isotropic in the horizontally oriented \(xy\)\sphinxhyphen{}plane, we get,
\begin{equation*}
\begin{split}
\begin{bmatrix}
v_{fx}\\ v_{fy}\\ v_{fz}
\end{bmatrix}
= -\begin{bmatrix}
K_h, 0,  0 \\
0, K_h, 0\\
0, 0, K_v
\end{bmatrix}
\cdot
\begin{bmatrix}
\frac{\partial h}{\partial x}\\ \frac{\partial h}{\partial y} \\ \frac{\partial h}{\partial z}
\end{bmatrix}
\end{split}
\end{equation*}
\sphinxAtStartPar
The most general case is encountered when all \sphinxstyleemphasis{principal axes} are associated with different hydraulic conductivities and none of them coincides with a coordinate axis:
\begin{equation*}
\begin{split}
\begin{bmatrix}
v_{fx}\\ v_{fy}\\ v_{fz}
\end{bmatrix}
= -\begin{bmatrix}
K_{xx}, K_{xy},  K_{xz} \\
K_{xy}, K_{yy}, K_{yz}\\
K_{xz}, K_{yz}, K_{zz}
\end{bmatrix}
\cdot
\begin{bmatrix}
\frac{\partial h}{\partial x}\\ \frac{\partial h}{\partial y} \\ \frac{\partial h}{\partial z}
\end{bmatrix}
\end{split}
\end{equation*}
\sphinxAtStartPar
From this, one may easily conclude that it is advantageous to arrange coordinate axes in parallel with \sphinxstyleemphasis{principal axes} of \(K\). Unfortunately, this is not always possible, e.g., when layers are folded.


\subsection{Direction of flow}
\label{\detokenize{content/flow/L6/16_darcy_law_3D:direction-of-flow}}
\sphinxAtStartPar
With directional dependence on conductivity also affects the direction of the flow. This is summarized in the figure below. Basically we have
\begin{quote}

\sphinxAtStartPar
\sphinxstylestrong{Isotropic aquifer}: \(K\) can be represented by a scalar and the direction of flow is opposite to the direction of the gradient.
\end{quote}
\begin{quote}

\sphinxAtStartPar
\sphinxstylestrong{Anisotropic aquifer}: \(K\) has to be represented by a tensor (matrix) and the angle between the gradient vector and the flow direction is between 90° and 180°.
\end{quote}

\begin{figure}[htbp]
\centering
\capstart

\noindent\sphinxincludegraphics[scale=0.4]{{L06_f12}.png}
\caption{Direction of flow in anisotropic aquifer}\label{\detokenize{content/flow/L6/16_darcy_law_3D:aniso-dir}}\end{figure}


\subsection{Examples for Flow Nets in Anisotropic Aquifers}
\label{\detokenize{content/flow/L6/16_darcy_law_3D:examples-for-flow-nets-in-anisotropic-aquifers}}
\sphinxAtStartPar
The figure provides flow nets for different ratios of anisotropy but for the identical conditions at domain boundaries (i.e., inflow from the top, outflow to a pipe on the left). As can be observed isolines and streamlines intersect each other at right angles only if the aquifer is isotropic.

\begin{figure}[htbp]
\centering
\capstart

\noindent\sphinxincludegraphics[scale=0.4]{{L06_f11}.png}
\caption{Flow nets in anisotropic aquifer}\label{\detokenize{content/flow/L6/16_darcy_law_3D:flow-net-aniso}}\end{figure}

\sphinxAtStartPar
The interactive simulation on effect of anisotropy on flux and gradient can be found in:
Tools Section as Simulating Anisotropy and Flow Direction

\begin{sphinxuseclass}{cell}
\begin{sphinxuseclass}{tag_hide-input}\begin{sphinxVerbatimOutput}

\begin{sphinxuseclass}{cell_output}
\begin{sphinxVerbatim}[commandchars=\\\{\}]
interactive(children=(BoundedFloatText(value=45.0, description=\PYGZsq{}angle (°)\PYGZsq{}, max=360.0, step=0.5), BoundedIntTe…
\end{sphinxVerbatim}

\end{sphinxuseclass}\end{sphinxVerbatimOutput}

\end{sphinxuseclass}
\end{sphinxuseclass}

\section{Chapter Quiz}
\label{\detokenize{content/flow/L6/16_darcy_law_3D:chapter-quiz}}
\begin{sphinxuseclass}{cell}
\begin{sphinxuseclass}{tag_remove-input}
\begin{sphinxuseclass}{tag_hide-output}
\end{sphinxuseclass}
\end{sphinxuseclass}
\end{sphinxuseclass}
\sphinxstepscope


\chapter{Quantifying 3D Groundwater Flow}
\label{\detokenize{content/flow/L7/17_quantify_flow:quantifying-3d-groundwater-flow}}\label{\detokenize{content/flow/L7/17_quantify_flow::doc}}
\sphinxAtStartPar
\sphinxstyleemphasis{(The contents are based on the class lecture materials of Prof. R. Lied. Modifications mostly to fit this specific were done by Dr. P. K. Yadav, and numerical examples were contributed by  Ms. Anne Pförtner and Sophie Pförtner.)}


\bigskip\hrule\bigskip



\section{Motivation}
\label{\detokenize{content/flow/L7/17_quantify_flow:motivation}}
\sphinxAtStartPar
In the previous lectures we attempted to understand and quantify the different sub\sphinxhyphen{}surface properties, e.g., hydraulic head, conductivity, and attempted to characterize them in space (homogeneous versus heterogeneous) and direction (isotropic versus anisotropic). Finally, introducing the Darcy’s law in all dimensions.

\sphinxAtStartPar
In this lecture we will attempt to quantify groundwater flow by developing system equation. This we will begin with a confined aquifers, following that with unconfined aquifer and eventually, we will formulate an approach to quantify general groundwater flow problems.


\section{Quantification of Three\sphinxhyphen{}Dimension (3D) Groundwater Flow}
\label{\detokenize{content/flow/L7/17_quantify_flow:quantification-of-three-dimension-3d-groundwater-flow}}

\subsection{Control Volume}
\label{\detokenize{content/flow/L7/17_quantify_flow:control-volume}}
\sphinxAtStartPar
A concept of fictitious or mathematical volume is important to discuss before attempting to formulate system equation. The fictitious volume, called \sphinxstylestrong{control volume} is a portion of an aquifer which is
\begin{quote}

\sphinxAtStartPar
much smaller than the regions of investigation
\end{quote}
\begin{quote}

\sphinxAtStartPar
much bigger than individual grains or pores
\end{quote}
\begin{quote}

\sphinxAtStartPar
Darcy’s law is fully applicable in the volume.
\end{quote}

\sphinxAtStartPar
The most important advantage is that the \sphinxstyleemphasis{control volume} can be advantageously adjusted to the coordinated system used (e.g., rectangular for Cartesian coordinates). The picture below is an attempt to depict a \sphinxstyleemphasis{control volume} in a natural aquifer.

\begin{figure}[htbp]
\centering
\capstart

\noindent\sphinxincludegraphics[scale=0.7]{{L07_f1}.png}
\caption{A control volume}\label{\detokenize{content/flow/L7/17_quantify_flow:cv}}\end{figure}


\subsection{Classification of Groundwater Flow Regimes}
\label{\detokenize{content/flow/L7/17_quantify_flow:classification-of-groundwater-flow-regimes}}
\sphinxAtStartPar
Since we intend to formulate system equations for groundwater flow, it is important that we first classify the groundwater problems as per the flow regimes. In general such classification can contain:


\begin{savenotes}\sphinxattablestart
\centering
\begin{tabulary}{\linewidth}[t]{|T|T|}
\hline
\sphinxstyletheadfamily 
\sphinxAtStartPar
\sphinxstylestrong{Property}
&\sphinxstyletheadfamily 
\sphinxAtStartPar
\sphinxstylestrong{Problem\sphinxhyphen{}Type}
\\
\hline
\sphinxAtStartPar
\sphinxstylestrong{Dimension\sphinxhyphen{}wise}
&
\sphinxAtStartPar
1D, 2D or 3D
\\
\hline
\sphinxAtStartPar
\sphinxstylestrong{Time\sphinxhyphen{}function}
&
\sphinxAtStartPar
Stead\sphinxhyphen{}State or transient
\\
\hline
\sphinxAtStartPar
\sphinxstylestrong{Space\sphinxhyphen{}dependent}
&
\sphinxAtStartPar
Homogeneous or heterogeneous
\\
\hline
\sphinxAtStartPar
\sphinxstylestrong{Direction\sphinxhyphen{}dependent}
&
\sphinxAtStartPar
Isotropic or anisotropic
\\
\hline
\sphinxAtStartPar
\sphinxstylestrong{Aquifer type}
&
\sphinxAtStartPar
Confined or unconfined
\\
\hline
\sphinxAtStartPar
\sphinxstylestrong{Source function}
&
\sphinxAtStartPar
With or without source/sinks
\\
\hline
\end{tabulary}
\par
\sphinxattableend\end{savenotes}

\sphinxAtStartPar
The combination presented in the table above amounts to 96 possible combinations and only very exceptional scenarios are possible. Each combination resulting from the table corresponds to a certain equation governing groundwater flow. Thus many similarities in formulation exist but only differences are with respect to details.

\begin{sphinxadmonition}{attention}{Attention:}
\sphinxAtStartPar
All the formulations of groundwater system equation are based on \sphinxstylestrong{just two} principles. They are:
\begin{enumerate}
\sphinxsetlistlabels{\arabic}{enumi}{enumii}{}{.}%
\item {} 
\sphinxAtStartPar
The conservation of volume

\item {} 
\sphinxAtStartPar
Darcy’s law

\end{enumerate}
\end{sphinxadmonition}

\sphinxAtStartPar
The above two principles are sufficient for formulation of system equations for consolidated aquifers. The system equations for consolidated aquifers can be bit more complicated. This course only very limitedly deals with such aquifers.


\subsection{Conservation of Volume}
\label{\detokenize{content/flow/L7/17_quantify_flow:conservation-of-volume}}
\sphinxAtStartPar
Let us consider a control volume with inlet and outlet as shown in the figure below.

\begin{figure}[htbp]
\centering
\capstart

\noindent\sphinxincludegraphics[scale=0.4]{{L07_f2}.png}
\caption{Conservation of volume}\label{\detokenize{content/flow/L7/17_quantify_flow:vol-bud}}\end{figure}

\sphinxAtStartPar
The volume budget in this control volume can be expressed as:

\sphinxAtStartPar
\textbackslash{}begin\{equation\}
\textbackslash{}frac\{\textbackslash{}Delta V\_w\}\{\textbackslash{}Delta t\} = Q\_\{in\} \sphinxhyphen{} Q\_\{out\}
\textbackslash{}end\{equation\}

\sphinxAtStartPar
The scenario that can be considered with the conservation of volume are:
\begin{itemize}
\item {} 
\sphinxAtStartPar
3D

\item {} 
\sphinxAtStartPar
transient

\item {} 
\sphinxAtStartPar
isotropic

\item {} 
\sphinxAtStartPar
heterogeneous

\item {} 
\sphinxAtStartPar
confined

\item {} 
\sphinxAtStartPar
without sources/sinks

\end{itemize}


\subsection{Darcy’s Law}
\label{\detokenize{content/flow/L7/17_quantify_flow:darcy-s-law}}
\sphinxAtStartPar
The generalized Darcy’s law is
\begin{equation*}
\begin{split}
v_f = -K \cdot \text{grad}h
\end{split}
\end{equation*}
\sphinxAtStartPar
where \(K\) when the principal axes not aligned to the domain coordinates is gives as (see {\hyperref[\detokenize{content/flow/L6/16_darcy_law_3D::doc}]{\sphinxcrossref{\DUrole{doc}{Steady\sphinxhyphen{}State Groundwater flow in 3D}}}})
\begin{equation*}
\begin{split}
\begin{bmatrix}
K_{xx}, K_{xy},  K_{xz} \\
K_{xy}, K_{yy}, K_{yz}\\
K_{xz}, K_{yz}, K_{zz}
\end{bmatrix}
\end{split}
\end{equation*}
\sphinxAtStartPar
It is to be noted that \(K\) is a scalar for isotropic aquifers and tensor (as above) for anisotropic aquifers. Further, in the Darcy’s law, the gradient vector {[}\sphinxhyphen{}{]} is oriented in the direction of the steepest \sphinxstyleemphasis{increase} in hydraulic head. The minus sign indicates that groundwater flow is directed from \sphinxstylestrong{large} to \sphinxstylestrong{small} head values.


\subsection{Inflow and Outflow}
\label{\detokenize{content/flow/L7/17_quantify_flow:inflow-and-outflow}}
\sphinxAtStartPar
The importance of control volume can be now observed as we derive the system equations for groundwater flow. The control volume as stated earlier has a very small size compared to the region of investigation. This entails that the changes of Darcy velocity components across this volume can therefore considered to be linear, i.e., the flow components do not meander. With this assumption, the linear changes are obtained by multiplying first\sphinxhyphen{}order derivatives with corresponding distance (see figure below). Mathematically put we have

\begin{figure}[htbp]
\centering
\capstart

\noindent\sphinxincludegraphics[scale=0.4]{{L07_f2}.png}
\caption{Conservation of volume}\label{\detokenize{content/flow/L7/17_quantify_flow:id1}}\end{figure}

\sphinxAtStartPar
\sphinxstylestrong{The sum of all discharge at the inlets:}
\begin{equation*}
\begin{split}
Q_{in} = v_{fx}\Delta y \Delta z + v_{fy}\Delta x \Delta z + v_{fz}\Delta x \Delta y 
\end{split}
\end{equation*}
\sphinxAtStartPar
\sphinxstylestrong{The sum of all discharge at the outlets:}
\begin{equation*}
\begin{split}
Q_{out} = \bigg(v_{fx}+\frac{\partial v_{fx}}{\partial x} \Delta x\bigg)\Delta y \Delta z + 
\bigg(v_{fy}+\frac{\partial v_{fy}}{\partial y} \Delta y\bigg)\Delta x \Delta z +
\bigg(v_{fz}+\frac{\partial v_{fz}}{\partial z} \Delta z\bigg)\Delta x \Delta y
\end{split}
\end{equation*}
\sphinxAtStartPar
\sphinxstylestrong{The different in discharge between inlet and outlet}:

\sphinxAtStartPar
\textbackslash{}begin\{equation\}
Q\_\{in\}\sphinxhyphen{}Q\_\{out\}=\sphinxhyphen{}\textbackslash{}frac\{\textbackslash{}partial v\_\{fx\}\}\{\textbackslash{}partial x\}\textbackslash{}Delta x\textbackslash{}Delta y \textbackslash{}Delta z
\sphinxhyphen{}\textbackslash{}frac\{\textbackslash{}partial v\_\{fy\}\}\{\textbackslash{}partial y\}\textbackslash{}Delta x\textbackslash{}Delta y \textbackslash{}Delta z\sphinxhyphen{}\textbackslash{}frac\{\textbackslash{}partial v\_\{fz\}\}\{\textbackslash{}partial z\}\textbackslash{}Delta x\textbackslash{}Delta y \textbackslash{}Delta z
\textbackslash{}end\{equation\}

\sphinxAtStartPar
Now, as we discussed in {\hyperref[\detokenize{content/flow/L3/13_gw_storage::doc}]{\sphinxcrossref{\DUrole{doc}{Groundwater as a reservoir}}}} it is known that change in water volume in an specific volumetric space is
\begin{equation*}
\begin{split}
\frac{\Delta V_w}{\Delta x \Delta y \Delta z} \propto \Delta h
\end{split}
\end{equation*}
\sphinxAtStartPar
which was formulated as

\sphinxAtStartPar
\textbackslash{}begin\{equation\}
\textbackslash{}Delta V\_w = S\_s\textbackslash{}Delta h \textbackslash{}Delta x \textbackslash{}Delta y \textbackslash{}Delta z\textbackslash{}end\{equation\}

\sphinxAtStartPar
That is, the value of the storage coefficient corresponds to the change in water volume within a unit control volume if hydraulic head is increased/decreased by one unit.


\subsection{Budgeting}
\label{\detokenize{content/flow/L7/17_quantify_flow:budgeting}}
\sphinxAtStartPar
Equations (1), (2) and (3) can be combined to obtain the expression for the water budget, i.e., rate of change of water volume, which is
\begin{equation*}
\begin{split}
S_s\frac{\Delta h}{\Delta t} = -\frac{\partial v_{fx}}{\partial x} - \frac{\partial v_{fy}}{\partial y} - \frac{\partial v_{fz}}{\partial z} 
\end{split}
\end{equation*}
\sphinxAtStartPar
letting \sphinxstylestrong{\(\Delta t \to 0\),} the above expression transforms to
\begin{equation*}
\begin{split}
S_s\frac{\partial h}{\partial t} = -\frac{\partial v_{fx}}{\partial x} - \frac{\partial v_{fy}}{\partial y} - \frac{\partial v_{fz}}{\partial z} 
\end{split}
\end{equation*}
\sphinxAtStartPar
Now, replacing Darcy’s velocity vector by components:
\begin{equation*}
\begin{split}
v_{fx}= -K \frac{\partial h}{\partial x},
\end{split}
\end{equation*}\begin{equation*}
\begin{split}
v_{fy}= -K \frac{\partial h}{\partial y},\, \text{ and}
\end{split}
\end{equation*}\begin{equation*}
\begin{split}
v_{fz}= -K \frac{\partial h}{\partial z}
\end{split}
\end{equation*}
\sphinxAtStartPar
we get the a system equation for a groundwater flow as

\begin{sphinxadmonition}{note}{Groundwater system equation for a 3D transient homogeneous and isotropic aquifers }
\begin{equation*}
\begin{split}
S_s\frac{\partial h}{\partial t} = \frac{\partial }{\partial x} \bigg(K \frac{\partial h}{\partial x}\bigg) + \frac{\partial}{\partial y} \bigg(K \frac{\partial h}{\partial y}\bigg) + \frac{\partial }{\partial z}\bigg(K \frac{\partial h}{\partial z} \bigg)
\end{split}
\end{equation*}\end{sphinxadmonition}


\subsection{Further Versions of the 3D Groundwater Flow Equation:}
\label{\detokenize{content/flow/L7/17_quantify_flow:further-versions-of-the-3d-groundwater-flow-equation}}
\sphinxAtStartPar
The groundwater system equation derived above can be modified in several ways to align with the problem in question. Few modifications are presented below.


\subsubsection{Inclusion of sources/sinks (\protect\(q\protect\))}
\label{\detokenize{content/flow/L7/17_quantify_flow:inclusion-of-sources-sinks-q}}
\sphinxAtStartPar
\sphinxstylestrong{Source} in groundwater system can be for example water transferred to groundwater from river or more often encountered case of \sphinxstyleemphasis{water injection} through wells. Similarly, \sphinxstylestrong{sinks} can be \sphinxstyleemphasis{water abstraction} or drainage of groundwater to the river (or any other surface water bodies). The sources/sinks can be represented by term \(q\) {[}1/T{]} in the groundwater system equation, which leads to
\begin{equation*}
\begin{split}
S_s\frac{\partial h}{\partial t} = \frac{\partial }{\partial x} \bigg(K \frac{\partial h}{\partial x}\bigg) + \frac{\partial}{\partial y} \bigg(K \frac{\partial h}{\partial y}\bigg) + \frac{\partial }{\partial z}\bigg(K \frac{\partial h}{\partial z} \bigg) + q
\end{split}
\end{equation*}
\begin{sphinxadmonition}{note}{Note:}
\sphinxAtStartPar
There can be several sources and sinks term and they can be simply added to the equation. A care has to be taken on the dimension {[}1/T{]} of the sources and sinks
\end{sphinxadmonition}


\subsubsection{Inclusion of Anisotropy}
\label{\detokenize{content/flow/L7/17_quantify_flow:inclusion-of-anisotropy}}
\sphinxAtStartPar
For simplicity, we assume that principal axes of anisotropy are in parallel with the coordinate axes. This makes the groundwater equation take the form:
\begin{equation*}
\begin{split}
S_s\frac{\partial h}{\partial t} = \frac{\partial }{\partial x} \bigg(K_x \frac{\partial h}{\partial x}\bigg) + \frac{\partial}{\partial y} \bigg(K_y \frac{\partial h}{\partial y}\bigg) + \frac{\partial }{\partial z}\bigg(K_z \frac{\partial h}{\partial z} \bigg) + q
\end{split}
\end{equation*}
\begin{sphinxadmonition}{note}{Note:}
\sphinxAtStartPar
Every directional \(K\)’s have to be considered (see {\hyperref[\detokenize{content/flow/L6/16_darcy_law_3D::doc}]{\sphinxcrossref{\DUrole{doc}{Steady\sphinxhyphen{}State Groundwater flow in 3D}}}}) if principal axes of anisotropy is not parallel with the considered coordinate axes.
\end{sphinxadmonition}


\subsection{Special Groundwater Problems and Their System Equations}
\label{\detokenize{content/flow/L7/17_quantify_flow:special-groundwater-problems-and-their-system-equations}}

\subsubsection{The Poisson equation}
\label{\detokenize{content/flow/L7/17_quantify_flow:the-poisson-equation}}
\sphinxAtStartPar
The Poisson equation deals with groundwater problem with the following condition:
\begin{itemize}
\item {} 
\sphinxAtStartPar
Steady\sphinxhyphen{}state

\item {} 
\sphinxAtStartPar
Homogeneous

\item {} 
\sphinxAtStartPar
Isotropic

\item {} 
\sphinxAtStartPar
Confined

\item {} 
\sphinxAtStartPar
Sources/sinks

\end{itemize}

\sphinxAtStartPar
The equation is given as
\begin{equation*}
\begin{split}
\frac{\partial^2 h }{\partial x^2} + \frac{\partial^2 h}{\partial y^2} + \frac{\partial^2 h}{\partial z^2} = -\frac{q}{K} 
\end{split}
\end{equation*}

\subsubsection{The Laplace equation}
\label{\detokenize{content/flow/L7/17_quantify_flow:the-laplace-equation}}
\sphinxAtStartPar
The Laplace equation deals with groundwater problem with the following condition:
\begin{itemize}
\item {} 
\sphinxAtStartPar
Steady\sphinxhyphen{}state

\item {} 
\sphinxAtStartPar
Homogeneous

\item {} 
\sphinxAtStartPar
Isotropic

\item {} 
\sphinxAtStartPar
Confined

\item {} 
\sphinxAtStartPar
Without sources/sinks

\end{itemize}

\sphinxAtStartPar
The equation is given as
\begin{equation*}
\begin{split}
\frac{\partial^2 h }{\partial x^2} + \frac{\partial^2 h}{\partial y^2} + \frac{\partial^2 h}{\partial z^2} = 0 
\end{split}
\end{equation*}

\section{Two\sphinxhyphen{}dimensional Groundwater Flow in Confined Aquifers}
\label{\detokenize{content/flow/L7/17_quantify_flow:two-dimensional-groundwater-flow-in-confined-aquifers}}
\sphinxAtStartPar
For most unconsolidated aquifers it is observed that groundwater flow components perpendicular to layering are negligible (see figure below). Therefore, groundwater flow problems are frequently treated as two\sphinxhyphen{}dimensional, i.e., employing only two spaces coordinates.

\begin{figure}[htbp]
\centering
\capstart

\noindent\sphinxincludegraphics[scale=0.2]{{L07_f4}.png}
\caption{2D\sphinxhyphen{} flow in confined aquifers}\label{\detokenize{content/flow/L7/17_quantify_flow:d-confined}}\end{figure}

\sphinxAtStartPar
The control volume in the confined aquifer is extended over the entire layer thickness of \(m\). If an aquifer consists of several distinct major layers, the two\sphinxhyphen{}dimensional approach is used for each layer separately and, in addition, water transfer between layers is handled by appropriate source/sink terms (not shown in figure).

\sphinxAtStartPar
The step from three to two dimensions requires to \sphinxstylestrong{sum up} hydraulic conductivity values over the entire layer thickness in order to correctly quantity two\sphinxhyphen{}dimensional groundwater flow. This is done by introducing \sphinxstylestrong{transmissivities} \(T_x\) and \(T_y\) {[}L2/T{]} as follows:
\begin{equation*}
\begin{split}
T_x = K_x\cdot m, \text{and}
\end{split}
\end{equation*}\begin{equation*}
\begin{split}
T_y = K_y\cdot m
\end{split}
\end{equation*}
\sphinxAtStartPar
where \(K_x\) and \(K_y\) denote vertically averaged hydraulic conductivities {[}L/T{]} along the \(x-\) and \(y-\) coordinate, respectively. For a confined aquifer which is horizontally isotropic we simply have
\begin{equation*}
\begin{split}
T = K\cdot m
\end{split}
\end{equation*}

\subsection{Example problem}
\label{\detokenize{content/flow/L7/17_quantify_flow:example-problem}}
\begin{sphinxadmonition}{note}{Hydraulic gradient in 2D}

\sphinxAtStartPar
Calculate the transmissivity for an isotropic \sphinxstylestrong{confined} aquifer.
\end{sphinxadmonition}

\begin{sphinxuseclass}{cell}\begin{sphinxVerbatimInput}

\begin{sphinxuseclass}{cell_input}
\begin{sphinxVerbatim}[commandchars=\\\{\}]
\PYG{n+nb}{print}\PYG{p}{(}\PYG{l+s+s2}{\PYGZdq{}}\PYG{l+s+se}{\PYGZbs{}n}\PYG{l+s+se}{\PYGZbs{}033}\PYG{l+s+s2}{[1mProvided are:}\PYG{l+s+se}{\PYGZbs{}033}\PYG{l+s+s2}{[0m}\PYG{l+s+se}{\PYGZbs{}n}\PYG{l+s+s2}{\PYGZdq{}}\PYG{p}{)}

\PYG{n}{K} \PYG{o}{=} \PYG{l+m+mf}{5e\PYGZhy{}4} \PYG{c+c1}{\PYGZsh{} m/s, hydraulic conductivity}
\PYG{n}{m} \PYG{o}{=} \PYG{l+m+mi}{45} \PYG{c+c1}{\PYGZsh{} m, aquifer thickness}
\PYG{n+nb}{print}\PYG{p}{(}\PYG{l+s+s2}{\PYGZdq{}}\PYG{l+s+s2}{hydraulic conductivity = }\PYG{l+s+si}{\PYGZob{}\PYGZcb{}}\PYG{l+s+s2}{ m/s}\PYG{l+s+se}{\PYGZbs{}n}\PYG{l+s+s2}{aquifer thickness = }\PYG{l+s+si}{\PYGZob{}\PYGZcb{}}\PYG{l+s+s2}{ m}\PYG{l+s+se}{\PYGZbs{}n}\PYG{l+s+s2}{\PYGZdq{}}\PYG{o}{.}\PYG{n}{format}\PYG{p}{(}\PYG{n}{K}\PYG{p}{,}\PYG{n}{m}\PYG{p}{)}\PYG{p}{,}\PYG{p}{)}
\end{sphinxVerbatim}

\end{sphinxuseclass}\end{sphinxVerbatimInput}
\begin{sphinxVerbatimOutput}

\begin{sphinxuseclass}{cell_output}
\begin{sphinxVerbatim}[commandchars=\\\{\}]
\PYG{Color+ColorBold}{Provided are:}

hydraulic conductivity = 0.0005 m/s
aquifer thickness = 45 m
\end{sphinxVerbatim}

\end{sphinxuseclass}\end{sphinxVerbatimOutput}

\end{sphinxuseclass}
\begin{sphinxuseclass}{cell}\begin{sphinxVerbatimInput}

\begin{sphinxuseclass}{cell_input}
\begin{sphinxVerbatim}[commandchars=\\\{\}]
\PYG{n}{K} \PYG{o}{=} \PYG{l+m+mf}{5e\PYGZhy{}4} \PYG{c+c1}{\PYGZsh{} m/s, hydraulic conductivity}
\PYG{n}{m} \PYG{o}{=} \PYG{l+m+mi}{45} \PYG{c+c1}{\PYGZsh{} m, aquifer thickness}

\PYG{c+c1}{\PYGZsh{}solution}
\PYG{n}{T} \PYG{o}{=} \PYG{n}{K}\PYG{o}{*}\PYG{n}{m}
\PYG{n+nb}{print}\PYG{p}{(}\PYG{l+s+s2}{\PYGZdq{}}\PYG{l+s+se}{\PYGZbs{}033}\PYG{l+s+s2}{[1mSolution:}\PYG{l+s+se}{\PYGZbs{}033}\PYG{l+s+s2}{[0m}\PYG{l+s+se}{\PYGZbs{}n}\PYG{l+s+s2}{The resulting transmissivity is }\PYG{l+s+se}{\PYGZbs{}033}\PYG{l+s+s2}{[1m}\PYG{l+s+si}{\PYGZob{}:02.4\PYGZcb{}}\PYG{l+s+s2}{ m²/s}\PYG{l+s+se}{\PYGZbs{}033}\PYG{l+s+s2}{[0m.}\PYG{l+s+s2}{\PYGZdq{}}\PYG{o}{.}\PYG{n}{format}\PYG{p}{(}\PYG{n}{T}\PYG{p}{)}\PYG{p}{)}
\end{sphinxVerbatim}

\end{sphinxuseclass}\end{sphinxVerbatimInput}
\begin{sphinxVerbatimOutput}

\begin{sphinxuseclass}{cell_output}
\begin{sphinxVerbatim}[commandchars=\\\{\}]
\PYG{Color+ColorBold}{Solution:}
The resulting transmissivity is \PYG{Color+ColorBold}{0.0225 m²/s}.
\end{sphinxVerbatim}

\end{sphinxuseclass}\end{sphinxVerbatimOutput}

\end{sphinxuseclass}

\subsection{Example problem}
\label{\detokenize{content/flow/L7/17_quantify_flow:id2}}
\begin{sphinxadmonition}{note}{Hydraulic gradient in 2D}

\sphinxAtStartPar
Calculate the transmissivity for an isotropic \sphinxstylestrong{unconfined aquifer}.
\end{sphinxadmonition}

\begin{sphinxuseclass}{cell}
\begin{sphinxuseclass}{tag_remove-input}\begin{sphinxVerbatimOutput}

\begin{sphinxuseclass}{cell_output}
\begin{sphinxVerbatim}[commandchars=\\\{\}]
\PYG{Color+ColorBold}{Provided are:}

hydraulic conductivity = 0.0005 m/s
hydraulic head = 130 m
aquifer bottom elevation = 110 m
\end{sphinxVerbatim}

\end{sphinxuseclass}\end{sphinxVerbatimOutput}

\end{sphinxuseclass}
\end{sphinxuseclass}
\begin{sphinxuseclass}{cell}\begin{sphinxVerbatimInput}

\begin{sphinxuseclass}{cell_input}
\begin{sphinxVerbatim}[commandchars=\\\{\}]
\PYG{n}{K} \PYG{o}{=} \PYG{l+m+mf}{5e\PYGZhy{}4} \PYG{c+c1}{\PYGZsh{} m/s, hydraulich conducticity}
\PYG{n}{h} \PYG{o}{=} \PYG{l+m+mi}{130} \PYG{c+c1}{\PYGZsh{} m, hydraulic head}
\PYG{n}{z\PYGZus{}bot} \PYG{o}{=} \PYG{l+m+mi}{110} \PYG{c+c1}{\PYGZsh{} m, aquifer bottom elevation}

\PYG{c+c1}{\PYGZsh{}solution}
\PYG{n}{T} \PYG{o}{=} \PYG{n}{K}\PYG{o}{*}\PYG{p}{(}\PYG{n}{h}\PYG{o}{\PYGZhy{}}\PYG{n}{z\PYGZus{}bot}\PYG{p}{)}
\PYG{n+nb}{print}\PYG{p}{(}\PYG{l+s+s2}{\PYGZdq{}}\PYG{l+s+se}{\PYGZbs{}033}\PYG{l+s+s2}{[1mSolution:}\PYG{l+s+se}{\PYGZbs{}033}\PYG{l+s+s2}{[0m}\PYG{l+s+se}{\PYGZbs{}n}\PYG{l+s+s2}{The resulting transmissivity is }\PYG{l+s+se}{\PYGZbs{}033}\PYG{l+s+s2}{[1m}\PYG{l+s+si}{\PYGZob{}:02.4\PYGZcb{}}\PYG{l+s+s2}{ m²/s}\PYG{l+s+se}{\PYGZbs{}033}\PYG{l+s+s2}{[0m.}\PYG{l+s+s2}{\PYGZdq{}}\PYG{o}{.}\PYG{n}{format}\PYG{p}{(}\PYG{n}{T}\PYG{p}{)}\PYG{p}{)}
\end{sphinxVerbatim}

\end{sphinxuseclass}\end{sphinxVerbatimInput}
\begin{sphinxVerbatimOutput}

\begin{sphinxuseclass}{cell_output}
\begin{sphinxVerbatim}[commandchars=\\\{\}]
\PYG{Color+ColorBold}{Solution:}
The resulting transmissivity is \PYG{Color+ColorBold}{0.01 m²/s}.
\end{sphinxVerbatim}

\end{sphinxuseclass}\end{sphinxVerbatimOutput}

\end{sphinxuseclass}

\subsection{2D Groundwater Flow Equations for Confined Conditions}
\label{\detokenize{content/flow/L7/17_quantify_flow:d-groundwater-flow-equations-for-confined-conditions}}
\sphinxAtStartPar
As was mentioned in {\hyperref[\detokenize{content/flow/L3/13_gw_storage::doc}]{\sphinxcrossref{\DUrole{doc}{Groundwater as a reservoir}}}}, the storage coefficient \(S\) {[}\sphinxhyphen{}{]} is to be used instead of \(S_s\) {[}1/L{]} if vertical flow components are neglected. This results in the following 2D groundwater flow equations without sources/sinks:
\begin{equation*}
\begin{split}
S\frac{\partial h}{\partial t} = \frac{\partial }{\partial x}\bigg(T_x\frac{\partial h}{\partial x}\bigg) + \frac{\partial }{\partial y}\bigg(T_y\frac{\partial h}{\partial y}\bigg)
\end{split}
\end{equation*}
\sphinxAtStartPar
The 2D groundwater flow equation with sources/sinks:
\begin{equation*}
\begin{split}
S\frac{\partial h}{\partial t} = \frac{\partial }{\partial x}\bigg(T_x\frac{\partial h}{\partial x}\bigg) + \frac{\partial }{\partial y}\bigg(T_y\frac{\partial h}{\partial y}\bigg) + N
\end{split}
\end{equation*}
\sphinxAtStartPar
where \(N\) denotes the volumetric flux due to source/sinks per unit surface area {[}L/T{]}.


\section{Two\sphinxhyphen{}dimensional Groundwater Flow in Unconfined Aquifers}
\label{\detokenize{content/flow/L7/17_quantify_flow:two-dimensional-groundwater-flow-in-unconfined-aquifers}}
\sphinxAtStartPar
For unconfined layers, the \sphinxstyleemphasis{control volume} extends from the aquifer bottom to the groundwater table (see figure below). This implies that the height of the control volume depends on the flow behaviour.

\begin{figure}[htbp]
\centering
\capstart

\noindent\sphinxincludegraphics[scale=0.4]{{L07_f5}.png}
\caption{2D\sphinxhyphen{} flow in unconfined aquifers}\label{\detokenize{content/flow/L7/17_quantify_flow:d-unconfined}}\end{figure}

\sphinxAtStartPar
Since the height of the control volume depends on the flow behaviour, the transmissivities are defined by using hydraulic head \(h\) to account for the saturated thickness:
\begin{equation*}
\begin{split}
T_x = K_x \cdot (h-z_{bot})
\end{split}
\end{equation*}\begin{equation*}
\begin{split}
T_y = K_y \cdot (h-z_{bot})
\end{split}
\end{equation*}
\sphinxAtStartPar
where \(z_{bot}\) represents the elevation of the aquifer bottom {[}L{]}. Formally, the 2D groundwater flow equation for unconfined aquifers is the same as for the confined conditions, i.e.,
\begin{equation*}
\begin{split}
S\frac{\partial h}{\partial t} = \frac{\partial }{\partial x}\bigg(T_x\frac{\partial h}{\partial x}\bigg) + \frac{\partial }{\partial y}\bigg(T_y\frac{\partial h}{\partial y}\bigg) + N
\end{split}
\end{equation*}
\sphinxAtStartPar
However, transmissivities are computed in a different way. The above equation is also termed \sphinxstylestrong{Boussinesq equation}.


\subsection{Dupuit Assumptions}
\label{\detokenize{content/flow/L7/17_quantify_flow:dupuit-assumptions}}
\sphinxAtStartPar
The Boussinesq equation for 2D groundwater flow in unconfined aquifer is based on the following assumptions which are due to Dupuit (1863)%
\begin{footnote}[1]\sphinxAtStartFootnote
Dupuit, J. (1863), \sphinxstyleemphasis{Etudes Théoriques et Pratiques sur le mouvement des Eaux dans les canaux découverts et à travers les terrains perméables} (Second ed.). Paris: Dunod.
%
\end{footnote}. The Dupuit’s work can be expressed in the figure below.

\begin{figure}[htbp]
\centering
\capstart

\noindent\sphinxincludegraphics[scale=0.4]{{L07_f7}.png}
\caption{Illustration of Dupuit’s assumptions}\label{\detokenize{content/flow/L7/17_quantify_flow:id4}}\end{figure}

\sphinxAtStartPar
\sphinxstylestrong{The Dupuit’s assumptions are}:
\begin{itemize}
\item {} 
\sphinxAtStartPar
Groundwater flow is horizontal (no vertical flow components).

\item {} 
\sphinxAtStartPar
The flow velocity does not vary with depth.

\item {} 
\sphinxAtStartPar
Darcy’s law also holds at the water table.

\end{itemize}


\section{Complete Formulation of Groundwater Flow Problems}
\label{\detokenize{content/flow/L7/17_quantify_flow:complete-formulation-of-groundwater-flow-problems}}
\sphinxAtStartPar
The complete formulation of groundwater flow requires following the specific steps. These are:

\sphinxAtStartPar
\sphinxstylestrong{Step I} : Specify the geometric properties of the region of interest (dimensionality, shape).  

\sphinxAtStartPar
\sphinxstylestrong{Step II} : Specify values of aquifer parameters (hydraulic conductivity, storage coefficient) by considering spatial variability and anisotropy, if necessary.  

\sphinxAtStartPar
\sphinxstylestrong{Step III} : Select the appropriate flow equation.  

\sphinxAtStartPar
\sphinxstylestrong{Step IV} : Specify the initial condition. (IC):

\sphinxAtStartPar
    \sphinxstylestrong{A.} head value at time. \(t=0\) 
    \sphinxstylestrong{B.} This step is not required for steady\sphinxhyphen{}state problems.  

\sphinxAtStartPar
\sphinxstylestrong{Step IV} : Specify boundary conditions (BC):

\sphinxAtStartPar
    \sphinxstylestrong{A.} BCs have to be given along the complete boundary (also at infinity if regions are assumed to be unbounded). 
    \sphinxstylestrong{B.} BCs may be time\sphinxhyphen{}dependent. 
    \sphinxstylestrong{C.} There are three major types of BCs . 


\subsection{Boundary Conditions}
\label{\detokenize{content/flow/L7/17_quantify_flow:boundary-conditions}}
\sphinxAtStartPar
The number of boundary conditions required corresponds to the highest space derivative for each coordinate in the flow equation. The three boundary conditions are:

\begin{figure}[htbp]
\centering

\noindent\sphinxincludegraphics[scale=0.8]{{L07_f10}.png}
\end{figure}
\begin{enumerate}
\sphinxsetlistlabels{\arabic}{enumi}{enumii}{}{.}%
\item {} 
\sphinxAtStartPar
\sphinxstylestrong{Boundary condition of the first kind} or \sphinxstylestrong{Dirichlet Boundary condition:} \sphinxhyphen{} The head value is given.

\item {} 
\sphinxAtStartPar
\sphinxstylestrong{Boundary condition of the second kind} or \sphinxstylestrong{Neumann Boundary condition:} \sphinxhyphen{} The component of the head gradient, which is perpendicular to the boundary, is given.

\item {} 
\sphinxAtStartPar
\sphinxstylestrong{Boundary condition of the third kind} or \sphinxstylestrong{Cauchy boundary condition} or \sphinxstylestrong{Robin boundary condition} (for completeness only): \sphinxhyphen{} A relationship between the head value and the component of the head gradient, which is perpendicular to the boundary, is given.

\end{enumerate}


\subsection{Relationships Aquifer/Flow Property and Mathematical Formulation}
\label{\detokenize{content/flow/L7/17_quantify_flow:relationships-aquifer-flow-property-and-mathematical-formulation}}
\sphinxAtStartPar
The following table summarizes the relationship between aquifer flow property and the corresponding mathematical formulation.


\begin{savenotes}\sphinxattablestart
\centering
\begin{tabulary}{\linewidth}[t]{|T|T|}
\hline
\sphinxstyletheadfamily 
\sphinxAtStartPar
\sphinxstylestrong{Aquifer/Flow property}
&\sphinxstyletheadfamily 
\sphinxAtStartPar
\sphinxstylestrong{Mathematical Formulation}
\\
\hline
\sphinxAtStartPar
Transient
&
\sphinxAtStartPar
with time derivative
\\
\hline
\sphinxAtStartPar
Confined
&
\sphinxAtStartPar
linear partial differential equation
\\
\hline
\sphinxAtStartPar
Anisotropic
&
\sphinxAtStartPar
\(T_x \neq T_y\) or \(K_x \neq K_y\), resp.  (tensor)
\\
\hline
\sphinxAtStartPar
Heterogeneous (inhomogeneous)
&
\sphinxAtStartPar
coefficients depends on space coordinate(s)
\\
\hline
\sphinxAtStartPar
With sources/sinks
&
\sphinxAtStartPar
inhomogeneous differential equation  (contains a term without \(h\))
\\
\hline
\sphinxAtStartPar
Fixed\sphinxhyphen{}head boundary condition
&
\sphinxAtStartPar
boundary condition of the first kind  (Dirichlet)
\\
\hline
\sphinxAtStartPar
Flux boundary condition  (in particular: \sphinxstyleemphasis{no flow})
&
\sphinxAtStartPar
boundary condition of the second kind (Neumann)
\\
\hline
\end{tabulary}
\par
\sphinxattableend\end{savenotes}


\subsection{Implementing Groundwater Flow System Equation}
\label{\detokenize{content/flow/L7/17_quantify_flow:implementing-groundwater-flow-system-equation}}
\sphinxAtStartPar
The figure below presents a groundwater problem with associated boundary conditions. The figure shows when a particular boundary condition is suitable. A zero flux or no flow boundary is streamline along boundaries 2\sphinxhyphen{}3 and 1\sphinxhyphen{}5. Likewise, the lake, whose water level can be constant is considered as prescribed head. From this example, it is evident that boundary conditions should be set based on geographical setting.

\begin{figure}[htbp]
\centering
\capstart

\noindent\sphinxincludegraphics[scale=0.6]{{L07_f11}.png}
\caption{Setting a groundwater problem (from Kinzelbach, 1986\sphinxfootnotemark[2])}\label{\detokenize{content/flow/L7/17_quantify_flow:concept-gw-problem}}\end{figure}
%
\begin{footnotetext}[2]\phantomsection\label{\thesphinxscope.2}%
\sphinxAtStartFootnote
Kinzelbach, W. (1986), \sphinxstyleemphasis{Groundwater Modelling: An Introduction with Sample Programs in BASIC,} Elsevier.
%
\end{footnotetext}\ignorespaces 
\sphinxAtStartPar
The system equation including the boundary conditions makes a \sphinxstylestrong{complete} groundwater flow problem. The solution with inclusion of specific boundary conditions provides a unique or specific solution to the problem.

\sphinxAtStartPar
\sphinxstylestrong{Numerical methods} (discussed in lectures 11\sphinxhyphen{}13) are generally required to solve groundwater flow problems. In exceptional cases (e.g., single well, steady\sphinxhyphen{}state, isotropic and homogeneous), \sphinxstylestrong{analytical solution} are also possible (discussed in the next lecture).


\section{Chapter Quiz}
\label{\detokenize{content/flow/L7/17_quantify_flow:chapter-quiz}}
\begin{sphinxuseclass}{cell}
\begin{sphinxuseclass}{tag_remove-input}
\begin{sphinxuseclass}{tag_hide-output}
\end{sphinxuseclass}
\end{sphinxuseclass}
\end{sphinxuseclass}

\bigskip\hrule\bigskip


\sphinxstepscope

\begin{sphinxuseclass}{cell}
\begin{sphinxuseclass}{tag_hide-input}
\begin{sphinxuseclass}{tag_remove-output}
\begin{sphinxuseclass}{tag_remove-input}
\end{sphinxuseclass}
\end{sphinxuseclass}
\end{sphinxuseclass}
\end{sphinxuseclass}

\chapter{Wells*}
\label{\detokenize{content/flow/L8/18_wells:wells}}\label{\detokenize{content/flow/L8/18_wells::doc}}
\sphinxAtStartPar
(\sphinxstyleemphasis{The contents are based on the class lecture materials of Prof. R. Liedl. Modifications mostly to fit this specific format were done by Prof. Liedl and Dr. P. K. Yadav.})


\bigskip\hrule\bigskip



\section{Motivation}
\label{\detokenize{content/flow/L8/18_wells:motivation}}
\sphinxAtStartPar
In the last lecture ({\hyperref[\detokenize{content/flow/L7/17_quantify_flow::doc}]{\sphinxcrossref{\DUrole{doc}{Quantifying 3D Groundwater Flow}}}} we derived system equations for different groundwater flow problems. We realized of the difficulties associated with solving flow problems specially at higher dimensions (2D/3D). Numerical methods are mostly used for solving groundwater problems but direct (analytical) solutions are also possible for some problems.

\sphinxAtStartPar
\sphinxstylestrong{Wells} are the most common and also most extensively used method of utilizing (or affecting) groundwater. Thus, \sphinxstyleemphasis{wells} represent a very common groundwater problem. We can now use our understanding of aquifer properties and groundwater system equations to analyze effect of \sphinxstyleemphasis{wells} on the natural groundwater flow. We will however restrict our extent, in this lecture, to problems that can be directly solved. After learning numerical methods (in last part of this course), we will apply it to evaluate more complex groundwater problems also associated with wells.

\sphinxAtStartPar
We begin this lecture recalling groundwater storage property \sphinxstyleemphasis{transmissivity.} Introducing \sphinxstyleemphasis{wells} we then derive few relations that can help us understand the effect of wells in groundwater flow. To conclude, we will use the \sphinxstyleemphasis{wells} to characterize aquifers, i.e., using so called pumping test.

\sphinxAtStartPar
(\sphinxstylestrong{\sphinxstyleemphasis{The contents are based on the class lecture materials of Prof. R. Liedl. Modifications mostly to fit this specific format were done by Prof. Liedl and Dr. P. K. Yadav.}})


\section{Transmissivity}
\label{\detokenize{content/flow/L8/18_wells:transmissivity}}
\sphinxAtStartPar
When discussing storage properties in {\hyperref[\detokenize{content/flow/L3/13_gw_storage::doc}]{\sphinxcrossref{\DUrole{doc}{Groundwater as a reservoir}}}}, we saw that aquifers or single layers may frequently be treated as two\sphinxhyphen{}dimensional systems. This is justified because the lateral extension of aquifers is usually much larger than the vertical extension. Thus, vertical variations of storage properties can be replaced by some average value without adversely affecting the quantification of groundwater storage.

\sphinxAtStartPar
Similar things can be done with regard to conductivity properties and this brings us to the geohydraulic parameter of transmissivity (\(T\),  L\(^2\)T\(^{-1}\)). The idea is to neglect vertical variations of hydraulic conductivity and to use vertically averaged values instead. This procedure does not eliminate horizontal variability, so transmissivity may still depend on horizontal coordinates  \((x, y)\).

\sphinxAtStartPar
The vertically averaged \(K\) value is then multiplied by the water\sphinxhyphen{}saturated thickness to obtain transmissivity. The concept of water\sphinxhyphen{}saturated thickness (or water\sphinxhyphen{}saturated depth) requires to distinguish whether \sphinxstyleemphasis{confined} or \sphinxstyleemphasis{unconfined flow} conditions prevail.

\sphinxAtStartPar
In general, water\sphinxhyphen{}saturated thickness is the distance from the aquifer bottom to a level up to which all pores are filled with water. For \sphinxstyleemphasis{confined aquifers,} this level is equal to aquifer top and water\sphinxhyphen{}saturated thickness is tantamount to aquifer thickness. For \sphinxstyleemphasis{unconfined aquifers,} however, water\sphinxhyphen{}saturated thickness corresponds to the distance between aquifer bottom and groundwater level. We will see some illustrations below when we try to quantify transmissivity.

\sphinxAtStartPar
Let us have a closer look at the confined case first. The black cuboid in \hyperref[\detokenize{content/transport/L9/21_conservative_transport:trans-c-2d}]{Fig.\@ \ref{\detokenize{content/transport/L9/21_conservative_transport:trans-c-2d}}}  illustrates that water\sphinxhyphen{}saturated thickness extends from aquifer bottom to aquifer top. So, it is equal to aquifer thickness \(m\). Transmissivity is calculated by \(T_x = K_x \cdot m\) and \(T_y = K_y \cdot m\). Here we allow for horizontal aquifer anisotropy with different hydraulic conductivities in \(x-\) and \(y-\) direction \((K_x\neq K_y)\). For horizontally isotropic aquifers \((K_x = K_y = K)\), transmissivity is given by \(T = K \cdot m\).

\begin{figure}[htbp]
\centering
\capstart

\noindent\sphinxincludegraphics[scale=0.4]{{L08_f1}.png}
\caption{The transmissivity cuboid in confined aquifer.}\label{\detokenize{content/flow/L8/18_wells:trans-c-2d}}\end{figure}

\sphinxAtStartPar
Things are a bit more complicated for unconfined aquifers.  \hyperref[\detokenize{content/flow/L8/18_wells:trans-u-2d}]{Fig.\@ \ref{\detokenize{content/flow/L8/18_wells:trans-u-2d}}} illustrates that water\sphinxhyphen{}saturated thickness extends from the aquifer bottom to the groundwater table. It is important to note that transmissivity of unconfined aquifers depends on the vertical position of the groundwater table. For instance, if the groundwater table is lowered during to a draught period, transmissivity is decreasing. This is fundamentally different from the confined case where the water\sphinxhyphen{}saturated thickness is given by aquifer geometry only and is not affected by hydraulic head changes.

\begin{figure}[htbp]
\centering
\capstart

\noindent\sphinxincludegraphics[scale=0.4]{{L08_f2}.png}
\caption{The transmissivity cuboid in unconfined aquifer.}\label{\detokenize{content/flow/L8/18_wells:trans-u-2d}}\end{figure}

\sphinxAtStartPar
Computing transmissivity of unconfined aquifers requires to determine the difference of hydraulic head h and the elevation of aquifer bottom \(z_{bot}\). Based on this, transmissivity is given by \(T_x = K_x\cdot(h - z_{bot})\) and \(T_y = K_y\cdot(h - z_{bot})\). As above, we are allowing for horizontal aquifer anisotropy. For an isotropic unconfined aquifer we get \(T = K\cdot(h - z_{bot})\).

\sphinxAtStartPar
Two more remarks appear to be appropriate: First, transmissivity may be computed by the given equations even if the aquifer bottom is not horizontal. This case is not covered by the \hyperref[\detokenize{content/flow/L8/18_wells:trans-u-2d}]{Fig.\@ \ref{\detokenize{content/flow/L8/18_wells:trans-u-2d}}}. Second, textbooks frequently present the equation \sphinxstylestrong{\(T = K\cdot h\)} for transmissivity of unconfined aquifers. It is to be noted that this equation only holds if two conditions are fulfilled:

\begin{sphinxadmonition}{note}{Required conditions when \protect\(T = K\cdot h\protect\) is valid}
\begin{enumerate}
\sphinxsetlistlabels{\arabic}{enumi}{enumii}{}{.}%
\item {} 
\sphinxAtStartPar
The aquifer bottom must be horizontal, and

\item {} 
\sphinxAtStartPar
hydraulic head values are expressed with respect to the elevation of aquifer bottom (= reference datum).

\end{enumerate}
\end{sphinxadmonition}

\sphinxAtStartPar
Finally, we can try to compute transmissivity for isotropic aquifers and check how the result depends on several quantities like aquifer bottom, aquifer top, and hydraulic head.


\subsection{Example problem}
\label{\detokenize{content/flow/L8/18_wells:example-problem}}
\begin{sphinxadmonition}{note}{Transmissivity}

\sphinxAtStartPar
Find if the aquifer is confined or unconfined, and then calculate transmissivity of the aquifer.
\end{sphinxadmonition}

\begin{sphinxuseclass}{cell}\begin{sphinxVerbatimInput}

\begin{sphinxuseclass}{cell_input}
\begin{sphinxVerbatim}[commandchars=\\\{\}]
\PYG{n}{K\PYGZus{}a} \PYG{o}{=} \PYG{l+m+mf}{8.5e\PYGZhy{}05} \PYG{c+c1}{\PYGZsh{} m/s, Hydraulic conductivity }
\PYG{n}{Z\PYGZus{}bot} \PYG{o}{=} \PYG{l+m+mi}{120} \PYG{c+c1}{\PYGZsh{} m, aquifer bottom }
\PYG{n}{Z\PYGZus{}top} \PYG{o}{=} \PYG{l+m+mi}{150} \PYG{c+c1}{\PYGZsh{} m, aquifer top}
\PYG{n}{h\PYGZus{}a}   \PYG{o}{=}  \PYG{l+m+mi}{139} \PYG{c+c1}{\PYGZsh{} m, hydraulic head in aquifer}

\PYG{c+c1}{\PYGZsh{} interim calculation}
\PYG{n}{A\PYGZus{}t} \PYG{o}{=} \PYG{n}{Z\PYGZus{}top}\PYG{o}{\PYGZhy{}}\PYG{n}{Z\PYGZus{}bot} \PYG{c+c1}{\PYGZsh{} m, Aquifer thickness}
\PYG{n}{A\PYGZus{}wt} \PYG{o}{=} \PYG{n}{h\PYGZus{}a} \PYG{o}{\PYGZhy{}} \PYG{n}{Z\PYGZus{}bot} \PYG{c+c1}{\PYGZsh{} m, water\PYGZus{}table level}
\PYG{n}{S\PYGZus{}t} \PYG{o}{=} \PYG{n+nb}{min}\PYG{p}{(}\PYG{n}{A\PYGZus{}t}\PYG{p}{,} \PYG{n}{A\PYGZus{}wt}\PYG{p}{)} \PYG{c+c1}{\PYGZsh{} m, saturated thickness}

\PYG{c+c1}{\PYGZsh{} result}
\PYG{k}{if} \PYG{n}{h\PYGZus{}a}\PYG{o}{\PYGZlt{}}\PYG{n}{Z\PYGZus{}top}\PYG{p}{:}
    \PYG{n+nb}{print}\PYG{p}{(}\PYG{l+s+s2}{\PYGZdq{}}\PYG{l+s+s2}{It is Unconfined Aquifer }\PYG{l+s+se}{\PYGZbs{}n}\PYG{l+s+s2}{\PYGZdq{}}\PYG{p}{)} 
\PYG{k}{else}\PYG{p}{:} 
    \PYG{n+nb}{print}\PYG{p}{(}\PYG{l+s+s2}{\PYGZdq{}}\PYG{l+s+s2}{It is Confined Aquifer }\PYG{l+s+se}{\PYGZbs{}n}\PYG{l+s+s2}{\PYGZdq{}}\PYG{p}{)} 

\PYG{n}{T\PYGZus{}a} \PYG{o}{=} \PYG{n}{K\PYGZus{}a}\PYG{o}{*}\PYG{n}{S\PYGZus{}t} \PYG{c+c1}{\PYGZsh{} m\PYGZca{}2/s, transmissivity}

\PYG{n+nb}{print}\PYG{p}{(}\PYG{l+s+s2}{\PYGZdq{}}\PYG{l+s+s2}{The required transmissivity is }\PYG{l+s+si}{\PYGZob{}0:1.2e\PYGZcb{}}\PYG{l+s+s2}{\PYGZdq{}}\PYG{o}{.}\PYG{n}{format}\PYG{p}{(}\PYG{n}{T\PYGZus{}a}\PYG{p}{)}\PYG{p}{,} \PYG{l+s+s2}{\PYGZdq{}}\PYG{l+s+s2}{m}\PYG{l+s+se}{\PYGZbs{}u00b2}\PYG{l+s+s2}{/s}\PYG{l+s+s2}{\PYGZdq{}}\PYG{p}{)}
\end{sphinxVerbatim}

\end{sphinxuseclass}\end{sphinxVerbatimInput}
\begin{sphinxVerbatimOutput}

\begin{sphinxuseclass}{cell_output}
\begin{sphinxVerbatim}[commandchars=\\\{\}]
It is Unconfined Aquifer 

The required transmissivity is 1.62e\PYGZhy{}03 m²/s
\end{sphinxVerbatim}

\end{sphinxuseclass}\end{sphinxVerbatimOutput}

\end{sphinxuseclass}

\section{Wells \sphinxhyphen{} Overview}
\label{\detokenize{content/flow/L8/18_wells:wells-overview}}

\subsection{What is a Well?}
\label{\detokenize{content/flow/L8/18_wells:what-is-a-well}}
\sphinxAtStartPar
A \sphinxstylestrong{well} is a shaft or a hole that has been sunk, dug or drilled into the earth to extract water (source: \sphinxhref{https://hydrologie.org/glu/HINDEN.HTM}{Glossary of Hydrology}).

\begin{figure}[htbp]
\centering
\capstart

\noindent\sphinxincludegraphics[scale=0.5]{{L08_f3}.png}
\caption{Well and its components.}\label{\detokenize{content/flow/L8/18_wells:well}}\end{figure}


\subsection{Using wells}
\label{\detokenize{content/flow/L8/18_wells:using-wells}}
\sphinxAtStartPar
Wells are very extensively used around the globe. The contents below only highlights few of the use of wells.
\begin{quote}

\sphinxAtStartPar
Water supply: e.g., for households, agriculture, industry
\end{quote}
\begin{quote}

\sphinxAtStartPar
Lowering the groundwater level: e.g., for excavations, open\sphinxhyphen{}pit mining
\end{quote}
\begin{quote}

\sphinxAtStartPar
Remediation of aquifer contamination: e.g., applying pump and treat method.
\end{quote}
\begin{quote}

\sphinxAtStartPar
Aquifer characterisation: e.g., using pumping test (this lecture)
\end{quote}

\sphinxAtStartPar
Apart from aquifer characterisation, wells are usually operated at steady\sphinxhyphen{}state i.e., at constant pumping rate. The figure below presents different uses of wells. \hyperref[\detokenize{content/flow/L8/18_wells:well-ex}]{Fig.\@ \ref{\detokenize{content/flow/L8/18_wells:well-ex}}} shows the case of lowering the groundwater level at the excavation site. The lowering is very often observed at the works that requires sub\sphinxhyphen{}surface construction works, e.g., high\sphinxhyphen{}rise building, tunnels. For this wells are placed close to excavation works and water is pumped out at higher discharge rates compare to the groundwater replenishing rate. This leads to decline of water level at the excavation site.

\begin{figure}[htbp]
\centering
\capstart

\noindent\sphinxincludegraphics[scale=0.4]{{L08_f5}.png}
\caption{Well at the excavation site}\label{\detokenize{content/flow/L8/18_wells:well-ex}}\end{figure}

\sphinxAtStartPar
\hyperref[\detokenize{content/flow/L8/18_wells:well-refuse}]{Fig.\@ \ref{\detokenize{content/flow/L8/18_wells:well-refuse}}} presents the case of using wells to delineate contamination site from the groundwater. These are part of the development of sanitary landfills sites or industries that pose threat to groundwater quality. Wells are used to lower groundwater table such that seepage of refuse is contained in the limited region.

\begin{figure}[htbp]
\centering
\capstart

\noindent\sphinxincludegraphics[width=30cm,height=8cm]{{L08_f6z}.png}
\caption{Well at the refuse site}\label{\detokenize{content/flow/L8/18_wells:well-refuse}}\end{figure}


\subsection{Fully versus Partially Penetrating Wells}
\label{\detokenize{content/flow/L8/18_wells:fully-versus-partially-penetrating-wells}}
\sphinxAtStartPar
Fully penetrating wells:
: The fully penetrating wells are those which extends through the whole saturated depth of an aquifer and are constructed in such a manner that water is permitted to the well screen over its length \sphinxhref{https://hydrologie.org/glu/HINDEN.HTM}{Glossary of Hydrology}).

\sphinxAtStartPar
\hyperref[\detokenize{content/flow/L8/18_wells:full-pen}]{Fig.\@ \ref{\detokenize{content/flow/L8/18_wells:full-pen}}} shows a schematic of water flow in a fully penetrating wells.

\begin{figure}[htbp]
\centering
\capstart

\noindent\sphinxincludegraphics[scale=0.4]{{L08_f7x}.png}
\caption{Flownet in a fully penetrating wells}\label{\detokenize{content/flow/L8/18_wells:full-pen}}\end{figure}

\sphinxAtStartPar
Partially penetrating Wells:
: The partially penetrating Wells are those in which the length of water entry is less than the thickness of the saturated aquifer which it penetrates (\sphinxhref{https://hydrologie.org/glu/HINDEN.HTM}{Glossary of Hydrology}).

\sphinxAtStartPar
Partially penetrating wells are constructed when aquifer depths are very high; in which case a fully penetrating well may not also be economical. \hyperref[\detokenize{content/flow/L8/18_wells:par-pen}]{Fig.\@ \ref{\detokenize{content/flow/L8/18_wells:par-pen}}} presents a schematic of a flownet that is likely to be observed in the partially penetrating well.

\begin{figure}[htbp]
\centering
\capstart

\noindent\sphinxincludegraphics[scale=0.4]{{L08_f7y}.png}
\caption{Flownet in a partially penetrating wells}\label{\detokenize{content/flow/L8/18_wells:par-pen}}\end{figure}

\sphinxAtStartPar
Comparing the Flownets in \hyperref[\detokenize{content/flow/L8/18_wells:full-pen}]{Fig.\@ \ref{\detokenize{content/flow/L8/18_wells:full-pen}}} and \hyperref[\detokenize{content/flow/L8/18_wells:par-pen}]{Fig.\@ \ref{\detokenize{content/flow/L8/18_wells:par-pen}}}, it can be observed that the vertical flow components can be significant in the partially penetrating case. This then lead to a 3D groundwater problem compared to the fully penetrating case, which can be treated as 2D groundwater problem as vertical flow component has limited effect.

\sphinxAtStartPar
This course we will deal only with the fully penetrating steady\sphinxhyphen{}state cases.


\section{Groundwater Flow Near Wells Operated at Steady State}
\label{\detokenize{content/flow/L8/18_wells:groundwater-flow-near-wells-operated-at-steady-state}}
\sphinxAtStartPar
As stated earlier wells are mostly operated under steady\sphinxhyphen{}state condition. Here will attempt to quantify the case. We will first attempt to identify the most relevant problems associated with the steadily pumping wells and then define our approach to solve them. The \sphinxstylestrong{two} most important points that need to be addressed here  are:
\begin{quote}

\sphinxAtStartPar
Which relevant quantities are needed to describe steady\sphinxhyphen{}state flow
towards a well?
\end{quote}
\begin{quote}

\sphinxAtStartPar
What is the quantitative relationship between the hydraulic
parameters under steady\sphinxhyphen{}state conditions?
\end{quote}

\sphinxAtStartPar
To answer the above questions, we follow the following approach:
\begin{quote}

\sphinxAtStartPar
Find an appropriate way to apply the \sphinxstyleemphasis{law of continuity}
(conservation of volume) and \sphinxstyleemphasis{Darcy‘s law.}
\end{quote}
\begin{quote}

\sphinxAtStartPar
We delineate the problem, e.g., study the \sphinxstyleemphasis{confined} and the \sphinxstyleemphasis{unconfined} cases separately.
\end{quote}


\subsection{Cone of Depression in a Confined Aquifer}
\label{\detokenize{content/flow/L8/18_wells:cone-of-depression-in-a-confined-aquifer}}
\sphinxAtStartPar
A \sphinxstylestrong{cone of depression} (or drawdown cone) will result when the well pumps groundwater groundwater is pumped from the well. \hyperref[\detokenize{content/flow/L8/18_wells:cone-con-un}]{Fig.\@ \ref{\detokenize{content/flow/L8/18_wells:cone-con-un}}} presents the schematic with relevant quantities of a well pumping case in both confined and unconfined aquifers. In both cases the well is fully penetrating. As can be understood from the figure that in the unconfined case the drawdown (\(s\)) magnitude depends on the water table level. This in the case of confined aquifer is dependent on the hydraulic head.

\begin{figure}[htbp]
\centering
\capstart

\noindent\sphinxincludegraphics[scale=0.25]{{L08_f8-9}.png}
\caption{Cone of depression in a (a) confined Aquifer and (b) unconfined aquifer (right)}\label{\detokenize{content/flow/L8/18_wells:cone-con-un}}\end{figure}

\sphinxAtStartPar
Relevant quantities in the figure:
\begin{itemize}
\item {} 
\sphinxAtStartPar
pumping rate \(Q\) {[}L\(^3\)T\(^{-1}\){]}

\end{itemize}
\begin{itemize}
\item {} 
\sphinxAtStartPar
aquifer thickness \(m\) {[}L{]}

\end{itemize}
\begin{itemize}
\item {} 
\sphinxAtStartPar
hydraulic conductivity \(K\) {[}LT\(^{-1}\){]}

\end{itemize}
\begin{itemize}
\item {} 
\sphinxAtStartPar
water level at rest \(H\) {[}L{]}

\end{itemize}
\begin{itemize}
\item {} 
\sphinxAtStartPar
water\sphinxhyphen{}level in the well \(h\) {[}L{]}

\end{itemize}
\begin{itemize}
\item {} 
\sphinxAtStartPar
Radius of influence \(R\) {[}L{]}

\end{itemize}
\begin{itemize}
\item {} 
\sphinxAtStartPar
well radius \(r_w\) (incl. gravel pack!){[}L{]}

\end{itemize}
\begin{itemize}
\item {} 
\sphinxAtStartPar
drawdown \(s = H - h\) {[}L{]}

\end{itemize}

\sphinxAtStartPar
The Radius of influence (\(R\)) in the figure is the distance from the well at which the drawdown becomes negligible or unobservable. Thus \(R\) delineates the influence of the well on the normal groundwater flow.


\subsubsection{Evolution of Cone of Depression with Time}
\label{\detokenize{content/flow/L8/18_wells:evolution-of-cone-of-depression-with-time}}
\sphinxAtStartPar
Transient models have to be used to observe the evolution of cone of depression. These are mostly only possible through use of numerical models (to be discussed in the later end of this course). The animation provides the computer simulation of evolving cone of depression as a function of time. From the figure it is easier to obtain the radius of influence (\(R\)). It is to be noted that \(R\) will be maximum at the steady\sphinxhyphen{}state condition.

\begin{sphinxuseclass}{cell}
\begin{sphinxuseclass}{tag_hide-input}\begin{sphinxVerbatimOutput}

\begin{sphinxuseclass}{cell_output}
\begin{sphinxVerbatim}[commandchars=\\\{\}]
Row
    [0] Video(str, height=150, sizing\PYGZus{}mode=\PYGZsq{}fixed\PYGZsq{}, width=400)
    [1] Spacer(width=50)
    [2] PNG(str, width=200)
\end{sphinxVerbatim}

\end{sphinxuseclass}\end{sphinxVerbatimOutput}

\end{sphinxuseclass}
\end{sphinxuseclass}

\subsection{Law of Continuity}
\label{\detokenize{content/flow/L8/18_wells:law-of-continuity}}
\sphinxAtStartPar
Next we attempt to quantify the cone of depression. For this, and it is common in any hydraulics study, we begin with the \sphinxstyleemphasis{Law of Continuity.} Overall law of continuity implies:

\begin{sphinxadmonition}{note}{Law of Continuity}

\sphinxAtStartPar
Discharge \(Q_w = \) constant
\end{sphinxadmonition}

\begin{figure}[htbp]
\centering
\capstart

\noindent\sphinxincludegraphics[scale=0.4]{{L08_f10X}.png}
\caption{Graphic visualization of Law of quantity.}\label{\detokenize{content/flow/L8/18_wells:law-cont}}\end{figure}

\sphinxAtStartPar
This, also seen in \hyperref[\detokenize{content/flow/L8/18_wells:law-cont}]{Fig.\@ \ref{\detokenize{content/flow/L8/18_wells:law-cont}}}, implies that the pumping rate \(Q\) corresponds to the discharge \(Q_w\) near the well. Further, under steady\sphinxhyphen{}state conditions the law of continuity
implies that there is the same discharge at all cross sections which
completely surround the well (“mantle of a cylinder”).


\subsection{Darcy’s Law for Flow Towards a Well in a Confined Aquifer}
\label{\detokenize{content/flow/L8/18_wells:darcy-s-law-for-flow-towards-a-well-in-a-confined-aquifer}}
\sphinxAtStartPar
The Darcy’s law, as we have known so far,  relates the Darcy’s velocity \(v_f\) and the hydraulic gradient \(i\). If the hydraulic gradient was constant in space, the hydraulic gradient would be (also see \hyperref[\detokenize{content/flow/L8/18_wells:law-cont}]{Fig.\@ \ref{\detokenize{content/flow/L8/18_wells:law-cont}}} )
\begin{equation*}
\begin{split}
i = \frac{\Delta h}{\Delta r}
\end{split}
\end{equation*}
\sphinxAtStartPar
where, \(r\) represents the radial distance from the well axis. In this case the hydraulic gradient \(i\) depends on the distance \(r\) within a cone of depression \hyperref[\detokenize{content/flow/L8/18_wells:law-cont}]{Fig.\@ \ref{\detokenize{content/flow/L8/18_wells:law-cont}}}). Qualitatively, it can be understood that \(i\) is decreasing with increasing distance from the well. Since the hydraulic gradient is not constant in space, the ratio \(\frac{\Delta h}{\Delta r}\) has to be replaced by
\begin{equation*}
\begin{split}
i(r) = \frac{\textrm{d} h (r)}{\textrm{d} r} \tag{C1}
\end{split}
\end{equation*}
\sphinxAtStartPar
With this defined, the Darcy’s law should be expressed as function of \(r\) as
\begin{equation*}
\begin{split}
v_f (r) = - K\cdot i(r) = - K \cdot \frac{\textrm{d} h (r)}{\textrm{d} r} \tag{C2}
\end{split}
\end{equation*}
\sphinxAtStartPar
And, now introducing the continuity equation we obtain:
\begin{equation*}
\begin{split}
Q_w = A\cdot v_f (r) = - K\cdot i(r) = - K \cdot A\cdot\frac{\textrm{d} h (r)}{\textrm{d} r} \tag{C3}
\end{split}
\end{equation*}
\sphinxAtStartPar
with \(A\) is the area of a cross section at distance \(r\) from the well axis. Therefore \(A\) is
\begin{equation*}
\begin{split}
A = 2\cdot \pi \cdot r \cdot m \tag{C4}
\end{split}
\end{equation*}
\sphinxAtStartPar
\(A\) is now inserted to eq. (C3), This results to
\begin{equation*}
\begin{split}
Q_w =  - 2\cdot \pi \cdot r \cdot m \cdot K \cdot \frac{\textrm{d} h (r)}{\textrm{d} r} \tag{C5}
\end{split}
\end{equation*}
\noindent{\hspace*{\fill}\sphinxincludegraphics[width=300\sphinxpxdimen]{{L08_f8}.png}\hspace*{\fill}}

\sphinxAtStartPar
Eq (C5) is a first\sphinxhyphen{}order differential equation for hydraulic head \(h(r)\). It can be solved by separation of variable. Doing that eq. (C5) becomes
\$\(
\textrm{d}h(r) = \frac{Q_w}{2\cdot \pi \cdot m \cdot K }\cdot \frac{\textrm{d}r}{r} \tag{C6}
\)\$

\sphinxAtStartPar
We now integrate eq. (C6). The limit of integration (see figure) along the vertical direction (\(h\)) will be from \(h\) to \(H\), and that along the radial axis will be from \(r_w\) and \(R\), i.e., we get
\begin{equation*}
\begin{split}
\int\limits_h^H\textrm{d}h(r) = \frac{Q_w}{2\cdot \pi \cdot m \cdot K }\cdot \int\limits_{r_w}^R\frac{\textrm{d}r}{r} \tag{C7}
\end{split}
\end{equation*}
\sphinxAtStartPar
The integrals in eq. (C7) are direct integrals and can be obtained from the table of integrals. The eq. (C7) with the indefinite integrals is thus
\begin{equation*}
\begin{split}
\big[h(r)\big]_h^H = - \frac{Q_w}{2\cdot\pi\cdot m \cdot K}\cdot \big[\ln r\big]_{r_w}^R 
\end{split}
\end{equation*}
\sphinxAtStartPar
after inserting the limits of integration, we get
\begin{equation*}
\begin{split}
H - h = - \frac{Q_w}{2\cdot\pi\cdot m \cdot K}\cdot (\ln R - \ln r_w) \tag{C8}
\end{split}
\end{equation*}
\sphinxAtStartPar
Finally, we can solve for the discharge \(Q_w\) and obtain
\begin{equation*}
\begin{split}
Q_w = - \frac{2\cdot\pi\cdot m \cdot K \cdot (H-h)}{(\ln R - \ln r_w)}\tag{C9}
\end{split}
\end{equation*}
\sphinxAtStartPar
The \sphinxstylestrong{negative} sign on the right\sphinxhyphen{}hand side of eq. (C9) indicates that flow is anti\sphinxhyphen{}parallel to the direction of the coordinate axis. Frequently, the pumping rate \(Q\) is used instead of \(Q_w\) and the negative sign is omitted. Thus we get a well known solution called Theim equation after Thiem (1906)%
\begin{footnote}[1]\sphinxAtStartFootnote
Thiem, G. (1906), \sphinxstyleemphasis{Hydrologische Methoden}, 56 pp., Gephardt, Leipzig, Germany.
%
\end{footnote}

\sphinxAtStartPar
with decadic logarithm, we get,
\begin{equation*}
\begin{split}
Q_w = \frac{2\cdot\pi\cdot m \cdot K \cdot (H-h)}{2.3\cdot(\log R - \log r_w)}  =  \frac{2\cdot\pi\cdot m \cdot K \cdot (H-h)}{2.3 \cdot\log (R/r_w)}
\end{split}
\end{equation*}

\subsection{Example problem}
\label{\detokenize{content/flow/L8/18_wells:id2}}
\begin{sphinxadmonition}{note}{Well discharge from confined aquifer}

\sphinxAtStartPar
From the provided data, calculate the transmissivity of the aquifer.
\end{sphinxadmonition}

\begin{sphinxuseclass}{cell}
\begin{sphinxuseclass}{tag_remove-input}\begin{sphinxVerbatimOutput}

\begin{sphinxuseclass}{cell_output}
\begin{sphinxVerbatim}[commandchars=\\\{\}]
\PYG{Color+ColorBold}{ Provided are:}

 The given dscharge is: 9 m³/min 

 The distance to Well 1 and well 2 are: 8m and 22m  

 The head at Well 1 and well 2 are: 9m and 10m
\end{sphinxVerbatim}

\end{sphinxuseclass}\end{sphinxVerbatimOutput}

\end{sphinxuseclass}
\end{sphinxuseclass}

\subsubsection{Solution}
\label{\detokenize{content/flow/L8/18_wells:solution}}
\sphinxAtStartPar
For confined aquifer \(T = K\cdot m\), the Thiem equation can be modified as:
\begin{equation*}
\begin{split}
Q = \frac{2\cdot\pi\cdot T \cdot (H-h)}{\ln (R/r_w)}
\end{split}
\end{equation*}
\begin{sphinxuseclass}{cell}\begin{sphinxVerbatimInput}

\begin{sphinxuseclass}{cell_input}
\begin{sphinxVerbatim}[commandchars=\\\{\}]
\PYG{n}{Q} \PYG{o}{=} \PYG{l+m+mi}{9} \PYG{c+c1}{\PYGZsh{} m\PYGZca{}3/min, Given discharge}
\PYG{n}{r1} \PYG{o}{=} \PYG{l+m+mi}{8} \PYG{c+c1}{\PYGZsh{} m, distance from well to point 1}
\PYG{n}{h1} \PYG{o}{=} \PYG{l+m+mi}{9} \PYG{c+c1}{\PYGZsh{} m, head at well 1}
\PYG{n}{R2} \PYG{o}{=} \PYG{l+m+mi}{22} \PYG{c+c1}{\PYGZsh{} m, distance from well to point 2}
\PYG{n}{H2} \PYG{o}{=} \PYG{l+m+mi}{10} \PYG{c+c1}{\PYGZsh{} m, head at well 2}



\PYG{c+c1}{\PYGZsh{}interim calculation }
\PYG{n}{Q\PYGZus{}min} \PYG{o}{=} \PYG{n}{Q} \PYG{o}{*} \PYG{l+m+mi}{1440} \PYG{c+c1}{\PYGZsh{} m\PYGZca{}3/d}

\PYG{c+c1}{\PYGZsh{}Calculation}

\PYG{n}{T} \PYG{o}{=} \PYG{n}{Q}\PYG{o}{/}\PYG{p}{(}\PYG{l+m+mi}{2}\PYG{o}{*}\PYG{n}{np}\PYG{o}{.}\PYG{n}{pi}\PYG{o}{*}\PYG{p}{(}\PYG{n}{H2}\PYG{o}{\PYGZhy{}}\PYG{n}{h1}\PYG{p}{)}\PYG{p}{)}\PYG{o}{*}\PYG{n}{np}\PYG{o}{.}\PYG{n}{log}\PYG{p}{(}\PYG{n}{R2}\PYG{o}{/}\PYG{n}{r1}\PYG{p}{)} \PYG{c+c1}{\PYGZsh{} m\PYGZca{}2/d, Transmissivity \PYGZhy{} inverting Theim equation}

\PYG{n+nb}{print}\PYG{p}{(}\PYG{l+s+s2}{\PYGZdq{}}\PYG{l+s+se}{\PYGZbs{}n}\PYG{l+s+se}{\PYGZbs{}033}\PYG{l+s+s2}{[1m Result:}\PYG{l+s+se}{\PYGZbs{}033}\PYG{l+s+s2}{[0m}\PYG{l+s+se}{\PYGZbs{}n}\PYG{l+s+s2}{\PYGZdq{}}\PYG{p}{)}
\PYG{n+nb}{print}\PYG{p}{(}\PYG{l+s+s2}{\PYGZdq{}}\PYG{l+s+s2}{The transmissivity in the aquifer is }\PYG{l+s+si}{\PYGZob{}0:0.2f\PYGZcb{}}\PYG{l+s+s2}{ m}\PYG{l+s+se}{\PYGZbs{}u00b2}\PYG{l+s+s2}{/d}\PYG{l+s+s2}{\PYGZdq{}}\PYG{o}{.}\PYG{n}{format}\PYG{p}{(}\PYG{n}{T}\PYG{p}{)}\PYG{p}{)} 
\end{sphinxVerbatim}

\end{sphinxuseclass}\end{sphinxVerbatimInput}
\begin{sphinxVerbatimOutput}

\begin{sphinxuseclass}{cell_output}
\begin{sphinxVerbatim}[commandchars=\\\{\}]
\PYG{Color+ColorBold}{ Result:}

The transmissivity in the aquifer is 1.45 m²/d
\end{sphinxVerbatim}

\end{sphinxuseclass}\end{sphinxVerbatimOutput}

\end{sphinxuseclass}

\section{Flow Towards a Well in an Unconfined Aquifer}
\label{\detokenize{content/flow/L8/18_wells:flow-towards-a-well-in-an-unconfined-aquifer}}
\sphinxAtStartPar
In the unconfined aquifer, the hydraulic head is function of \(r\). Thus the discharge in the aquifer:
\begin{equation*}
\begin{split}
Q_w = A \cdot v_f = - A \cdot K \frac{\textrm{d}h}{\textrm{d}r}(r) \tag{U1}
\end{split}
\end{equation*}
\noindent{\hspace*{\fill}\sphinxincludegraphics[width=300\sphinxpxdimen]{{L08_f9}.png}\hspace*{\fill}}

\sphinxAtStartPar
Also in this case, the cross section area \(A\) also is dependent on \(r\), i.e.,
\begin{equation*}
\begin{split}
A = 2 \cdot \pi \cdot r \cdot h(r) \tag{U2}
\end{split}
\end{equation*}
\sphinxAtStartPar
Next, we insert eq (U2) in eq. (U1) and get
\begin{equation*}
\begin{split}
Q_w = A \cdot v_f = - 2 \cdot \pi \cdot r \cdot h(r)  \cdot K \frac{\textrm{d}h}{\textrm{d}r}(r) \tag{U3}
\end{split}
\end{equation*}
\sphinxAtStartPar
Eq. (U3) is a first\sphinxhyphen{}order differential equation for hydraulic head \(h(r)\), and it can be solved by separation of variables. Separating the variables of eq. (U3) leads to
\begin{equation*}
\begin{split}
h(r) \cdot \textrm{d}h(r) = - \frac{Q_w}{2 \cdot \pi \cdot K }\cdot \frac{\textrm{d}r}{r} \tag{U4}
\end{split}
\end{equation*}
\sphinxAtStartPar
Next we integrate eq. (U4) from limits \(r_w\) to \(R\) in the right\sphinxhyphen{}hand side, and from \(h\) to \(H\) (left\sphinxhyphen{}hand side), i.e.,
\begin{equation*}
\begin{split}
\int\limits_h^H h(r) \cdot \textrm{d}h(r) = - \frac{Q_w}{2 \cdot \pi \cdot K }\cdot \int\limits_{r_w}^{R}\frac{\textrm{d}r}{r} \tag{U5}
\end{split}
\end{equation*}
\sphinxAtStartPar
The integral in eq. (U5) can be obtained from the standard table of integrals. With that we get
\begin{equation*}
\begin{split}
\frac{1}{2}(H^2 - h^2) = - \frac{Q_w}{2 \cdot \pi \cdot K }\cdot (\ln R - \ln r_w) \tag{U6}
\end{split}
\end{equation*}
\sphinxAtStartPar
The discharge in aquifer \(Q_w\) can be obtained from eq. (U6) and the expression is
\begin{equation*}
\begin{split}
Q_w = -\frac{\pi\cdot K \cdot (H^2-h^2)}{\ln R - \ln r_w}\tag{U6}
\end{split}
\end{equation*}
\sphinxAtStartPar
The negative sign on the right\sphinxhyphen{}hand side of the eq. (U6) indicates that flow is anti\sphinxhyphen{}parallel to the orientation of the coordinate axis. Frequently, the pumping rate \(Q\) is used instead of \(Q_w\) and the negative sign can thus be
omitted, and we get

\sphinxAtStartPar
In the decadic logarithm, the discharge from the well is:
\begin{equation*}
\begin{split}
Q_= \frac{\pi\cdot K \cdot (H^2-h^2)}{2.3(\log R - \log r_w)} = \frac{\pi\cdot K \cdot (H^2-h^2)}{2.3\log (R/r_w)}
\end{split}
\end{equation*}

\subsection{Example problem}
\label{\detokenize{content/flow/L8/18_wells:id3}}
\begin{sphinxadmonition}{note}{Well discharge from unconfined aquifer}

\sphinxAtStartPar
From the provided data, calculate discharge of the aquifer.
\end{sphinxadmonition}

\begin{sphinxuseclass}{cell}\begin{sphinxVerbatimInput}

\begin{sphinxuseclass}{cell_input}
\begin{sphinxVerbatim}[commandchars=\\\{\}]
\PYG{n+nb}{print}\PYG{p}{(}\PYG{l+s+s2}{\PYGZdq{}}\PYG{l+s+se}{\PYGZbs{}n}\PYG{l+s+se}{\PYGZbs{}033}\PYG{l+s+s2}{[1m Provided are:}\PYG{l+s+se}{\PYGZbs{}033}\PYG{l+s+s2}{[0m}\PYG{l+s+se}{\PYGZbs{}n}\PYG{l+s+s2}{\PYGZdq{}}\PYG{p}{)}

\PYG{n}{K} \PYG{o}{=} \PYG{l+m+mf}{24.50} \PYG{c+c1}{\PYGZsh{} m/d, conductivity}
\PYG{n}{r\PYGZus{}1} \PYG{o}{=} \PYG{l+m+mf}{0.23} \PYG{c+c1}{\PYGZsh{} m, distance from well to point 1}
\PYG{n}{h\PYGZus{}1} \PYG{o}{=} \PYG{l+m+mi}{12} \PYG{c+c1}{\PYGZsh{} m, head at well 1}
\PYG{n}{R\PYGZus{}2} \PYG{o}{=} \PYG{l+m+mi}{275} \PYG{c+c1}{\PYGZsh{} m, distance from well to point 2}
\PYG{n}{H\PYGZus{}2} \PYG{o}{=} \PYG{l+m+mi}{18} \PYG{c+c1}{\PYGZsh{} m, head at well 2}

\PYG{n+nb}{print}\PYG{p}{(}\PYG{l+s+s2}{\PYGZdq{}}\PYG{l+s+s2}{ The given conductivity is: }\PYG{l+s+si}{\PYGZob{}\PYGZcb{}}\PYG{l+s+s2}{\PYGZdq{}}\PYG{o}{.}\PYG{n}{format}\PYG{p}{(}\PYG{n}{K}\PYG{p}{)}\PYG{p}{,} \PYG{l+s+s2}{\PYGZdq{}}\PYG{l+s+s2}{m/d }\PYG{l+s+se}{\PYGZbs{}n}\PYG{l+s+s2}{\PYGZdq{}}\PYG{p}{)}
\PYG{n+nb}{print}\PYG{p}{(}\PYG{l+s+s2}{\PYGZdq{}}\PYG{l+s+s2}{ The distance to Well 1 and well 2 are: }\PYG{l+s+si}{\PYGZob{}\PYGZcb{}}\PYG{l+s+s2}{ m and }\PYG{l+s+si}{\PYGZob{}\PYGZcb{}}\PYG{l+s+s2}{ m are }\PYG{l+s+se}{\PYGZbs{}n}\PYG{l+s+s2}{\PYGZdq{}}\PYG{o}{.}\PYG{n}{format}\PYG{p}{(}\PYG{n}{r\PYGZus{}1}\PYG{p}{,} \PYG{n}{R\PYGZus{}2}\PYG{p}{)}\PYG{p}{)}
\PYG{n+nb}{print}\PYG{p}{(}\PYG{l+s+s2}{\PYGZdq{}}\PYG{l+s+s2}{ The head at Well 1 and well 2 are: }\PYG{l+s+si}{\PYGZob{}\PYGZcb{}}\PYG{l+s+s2}{ m and }\PYG{l+s+si}{\PYGZob{}\PYGZcb{}}\PYG{l+s+s2}{ m}\PYG{l+s+s2}{\PYGZdq{}}\PYG{o}{.}\PYG{n}{format}\PYG{p}{(}\PYG{n}{h\PYGZus{}1}\PYG{p}{,} \PYG{n}{H\PYGZus{}2}\PYG{p}{)}\PYG{p}{)}

\PYG{c+c1}{\PYGZsh{}Calculation}

\PYG{n}{Q\PYGZus{}1} \PYG{o}{=} \PYG{p}{(}\PYG{n}{np}\PYG{o}{.}\PYG{n}{pi}\PYG{o}{*}\PYG{n}{K}\PYG{o}{*}\PYG{p}{(}\PYG{n}{H\PYGZus{}2}\PYG{o}{*}\PYG{o}{*}\PYG{l+m+mi}{2}\PYG{o}{\PYGZhy{}}\PYG{n}{h\PYGZus{}1}\PYG{o}{*}\PYG{o}{*}\PYG{l+m+mi}{2}\PYG{p}{)}\PYG{p}{)}\PYG{o}{/}\PYG{p}{(}\PYG{n}{np}\PYG{o}{.}\PYG{n}{log}\PYG{p}{(}\PYG{n}{R\PYGZus{}2}\PYG{o}{/}\PYG{n}{r\PYGZus{}1}\PYG{p}{)}\PYG{p}{)} \PYG{c+c1}{\PYGZsh{} m\PYGZca{}2/d, Transmissivity \PYGZhy{} inverting Theim equation}

\PYG{n+nb}{print}\PYG{p}{(}\PYG{l+s+s2}{\PYGZdq{}}\PYG{l+s+se}{\PYGZbs{}n}\PYG{l+s+se}{\PYGZbs{}033}\PYG{l+s+s2}{[1m Result:}\PYG{l+s+se}{\PYGZbs{}033}\PYG{l+s+s2}{[0m}\PYG{l+s+se}{\PYGZbs{}n}\PYG{l+s+s2}{\PYGZdq{}}\PYG{p}{)}
\PYG{n+nb}{print}\PYG{p}{(}\PYG{l+s+s2}{\PYGZdq{}}\PYG{l+s+s2}{Discharge from the well is }\PYG{l+s+si}{\PYGZob{}0:0.2f\PYGZcb{}}\PYG{l+s+s2}{ m}\PYG{l+s+se}{\PYGZbs{}u00b3}\PYG{l+s+s2}{/d}\PYG{l+s+s2}{\PYGZdq{}}\PYG{o}{.}\PYG{n}{format}\PYG{p}{(}\PYG{n}{Q\PYGZus{}1}\PYG{p}{)}\PYG{p}{)} 
\end{sphinxVerbatim}

\end{sphinxuseclass}\end{sphinxVerbatimInput}
\begin{sphinxVerbatimOutput}

\begin{sphinxuseclass}{cell_output}
\begin{sphinxVerbatim}[commandchars=\\\{\}]
\PYG{Color+ColorBold}{ Provided are:}

 The given conductivity is: 24.5 m/d 

 The distance to Well 1 and well 2 are: 0.23 m and 275 m are 

 The head at Well 1 and well 2 are: 12 m and 18 m

\PYG{Color+ColorBold}{ Result:}

Discharge from the well is 1955.06 m³/d
\end{sphinxVerbatim}

\end{sphinxuseclass}\end{sphinxVerbatimOutput}

\end{sphinxuseclass}

\subsection{Radius of Influence}
\label{\detokenize{content/flow/L8/18_wells:radius-of-influence}}
\sphinxAtStartPar
The radius of influence (\(R\)) can also (instead of using numerical simulations) be obtained from empirical equations. The table below provides a list of few equations.

\begin{sphinxShadowBox}
\sphinxstylesidebartitle{The symbols in equations are:}
\begin{itemize}
\item {} 
\sphinxAtStartPar
\(s\) = drawdown in pumping well,

\item {} 
\sphinxAtStartPar
\(t\) = pumping time,

\item {} 
\sphinxAtStartPar
\(N\) = groundwater recharge,

\item {} 
\sphinxAtStartPar
\(S\) = storage coefficient,

\item {} 
\sphinxAtStartPar
\(K\) = hydraulic conductivity,

\item {} 
\sphinxAtStartPar
\(H\) = water level at rest (unconfined aquifer).

\end{itemize}

\sphinxAtStartPar
In confined aquifers, \(H\) has to be replaced by the aquifer thickness \(m\).
\end{sphinxShadowBox}


\begin{savenotes}\sphinxattablestart
\centering
\begin{tabulary}{\linewidth}[t]{|T|T|T|}
\hline
\sphinxstyletheadfamily 
\sphinxAtStartPar
Source
&
\sphinxAtStartPar

&\sphinxstyletheadfamily 
\sphinxAtStartPar
Equation
\\
\hline
\sphinxAtStartPar
Lembke (1886, 1887)
&
\sphinxAtStartPar

&
\sphinxAtStartPar
\(R = H (K/2\cdot N)^{1/2}\)
\\
\hline
\sphinxAtStartPar
Weber (Schultze, 1924)
&
\sphinxAtStartPar

&
\sphinxAtStartPar
\(R = 2.45 (H\cdot K\cdot t/S)^{1/2}\)
\\
\hline
\sphinxAtStartPar
Kusakin (Aravin and Numerov, 1953)
&
\sphinxAtStartPar

&
\sphinxAtStartPar
\(R = 1.9 (H \cdot K \cdot t/s)^{1/2}\)
\\
\hline
\sphinxAtStartPar
Siechardt (Certousov, 1962)
&
\sphinxAtStartPar

&
\sphinxAtStartPar
\(R = 3000\cdot s \cdot K^{1/2}\)
\\
\hline
\sphinxAtStartPar
Kusakin (Certousov, 1949)
&
\sphinxAtStartPar

&
\sphinxAtStartPar
\(R = 575\cdot s \cdot (H \cdot K)^{1/2} \)
\\
\hline
\end{tabulary}
\par
\sphinxattableend\end{savenotes}

\sphinxAtStartPar
\sphinxstylestrong{Siechardt} and \sphinxstylestrong{Kusakin} equation are among the preferred equations by practitioners. In both equations, \(K\) has to be expressed in \sphinxstyleemphasis{m/s} and all other quantities must be expressed in \sphinxstyleemphasis{m.} \sphinxstyleemphasis{R} depends on drawdown \(s= H-h\) in both equations. Trial and error or iterative strategies have to be used to determine \(R\) and \(h\).


\section{Aquifer Characterisation by Pumping Tests}
\label{\detokenize{content/flow/L8/18_wells:aquifer-characterisation-by-pumping-tests}}
\sphinxAtStartPar
Pumping tests are used to estimate aquifer properties such as hydraulic conductivity \((K)\), transmissivity \((T)\) or storativity \((S)\). Pumping results in an evolving cone of depression as was discussed earlier (see \hyperref[\detokenize{content/flow/L8/18_wells:cone-con-un}]{Fig.\@ \ref{\detokenize{content/flow/L8/18_wells:cone-con-un}}}). The decrease in hydraulic head (or increase in drawdown) with time is recorded in one or more observation wells (and sometimes also in the pumping well itself).

\sphinxAtStartPar
A variety of different schemes exist to evaluate pumping test data. The appropriate method has to be selected according to the specific setting (confined or unconfined, layered system, horizontal or inclined aquifer bottom etc.). A well known approach to derive \(T\) and \(S\) from pumping test data was developed by Theis (1935)%
\begin{footnote}[2]\sphinxAtStartFootnote
Theis, C.V., 1935. The relation between the lowering of the piezometric surface and the rate and duration of discharge of a well using groundwater storage, Am. Geophys. Union Trans., vol. 16, pp. 519\sphinxhyphen{}524.
%
\end{footnote}


\subsection{Applicability of the Theis Method}
\label{\detokenize{content/flow/L8/18_wells:applicability-of-the-theis-method}}
\sphinxAtStartPar
Pumping test data can be evaluated according to Theis (1935) if the following assumptions are (approximately) justified:
\begin{itemize}
\item {} 
\sphinxAtStartPar
The aquifer is confined, homogeneous and isotropic.

\end{itemize}
\begin{itemize}
\item {} 
\sphinxAtStartPar
The aquifer thickness is uniform.

\end{itemize}
\begin{itemize}
\item {} 
\sphinxAtStartPar
The aquifer bottom is horizontal.

\end{itemize}
\begin{itemize}
\item {} 
\sphinxAtStartPar
The well is fully penetrating.

\end{itemize}
\begin{itemize}
\item {} 
\sphinxAtStartPar
The well radius is very small as compared to the radius of influence.

\end{itemize}
\begin{itemize}
\item {} 
\sphinxAtStartPar
The pumping rate is constant within the measurement period.

\end{itemize}
\begin{itemize}
\item {} 
\sphinxAtStartPar
There is no vertical flow component.

\end{itemize}
\begin{itemize}
\item {} 
\sphinxAtStartPar
The evolution of the cone of depression is not influenced by other hydraulic factors (surface water, impermeable boundaries etc.).

\end{itemize}


\subsection{Drawdown According to Theis (1935)}
\label{\detokenize{content/flow/L8/18_wells:drawdown-according-to-theis-1935}}
\sphinxAtStartPar
Theis (1935) deals with the transient flow (as opposed to steady\sphinxhyphen{}state methods discussed above) of water to a pumping well. The time\sphinxhyphen{}dependent drawdown \(s\) in an observation well, which is a distance \(r\) apart from the pumping well, is given by

\begin{sphinxShadowBox}
\sphinxstylesidebartitle{}
\begin{itemize}
\item {} 
\sphinxAtStartPar
\(s\) is drawdown {[}L{]}

\end{itemize}
\begin{itemize}
\item {} 
\sphinxAtStartPar
\(Q\) is pumping rate {[}L\(^3\)T\(^{-1}\){]}

\end{itemize}

\sphinxAtStartPar
\(T\) is transmissivity {[}L\(^2\)T\(^{-1}\){]}

\sphinxAtStartPar
\(W(u)\) is the well function
\end{sphinxShadowBox}
\begin{equation*}
\begin{split}
s(r,t) = \frac{Q}{4\pi T}\cdot W(u) \tag{T1}
\end{split}
\end{equation*}
\sphinxAtStartPar
where, \(W(u)\) is the well function, which is given as
\begin{equation*}
\begin{split}
W(u) = \int\limits_u^\infty \frac{\textrm{e}^{-\widetilde{u}}}{\widetilde{u}}\textrm{d}\widetilde{u}
\end{split}
\end{equation*}
\sphinxAtStartPar
in which \(u\) is defined by
\begin{equation*}
\begin{split}
u = \frac{Sr^2}{4Tt} \tag{T2}
\end{split}
\end{equation*}
\sphinxAtStartPar
where \(S\) {[}\sphinxhyphen{}{]} is storage coefficient. For application Eq (T1) and eq. (T2), these equation are log\sphinxhyphen{}transformed. Eq (T1) then becomes
\begin{equation*}
\begin{split}
\log s(r,t) = \log \frac{Q}{4\pi T}\cdot \log W(u) \tag{T3}
\end{split}
\end{equation*}
\sphinxAtStartPar
and upon log\sphinxhyphen{}transformation and rearrangement of eq. (T2) results to
\begin{equation*}
\begin{split}
\log \frac{t}{r^2} = \log\frac{S}{4T} + \log\frac{1}{u} \tag{T4}
\end{split}
\end{equation*}
\sphinxAtStartPar
These two equations are used to derive \(T\) and \(S\) from drawdown data. This can either be done by applying special computer software or manually by a graphical method, which is discussed next.


\subsection{Manual Comparison of Data and Type Curve}
\label{\detokenize{content/flow/L8/18_wells:manual-comparison-of-data-and-type-curve}}
\sphinxAtStartPar
Theis (1935) provides a graphical approach to use the  Theis equation. The following steps are to be followed for using the graphical approach:
\begin{itemize}
\item {} 
\sphinxAtStartPar
The logarithm of drawdown (\(\log s\)) is plotted against the \(\log(t/r^2)\) in the data sheet

\item {} 
\sphinxAtStartPar
The logarithm of the well function \((\log W(u))\) is plotted against \(\log (1/u)\) in a type curve sheet.

\end{itemize}

\noindent{\hspace*{\fill}\sphinxincludegraphics[width=600\sphinxpxdimen]{{L08_f11}.png}\hspace*{\fill}}
\begin{itemize}
\item {} 
\sphinxAtStartPar
Both sheets are put on top of each other such that the data coincide with some part of the type curve.

\item {} 
\sphinxAtStartPar
The shifts along the vertical and the horizontal axes correspond to the constant terms in the equations

\end{itemize}
\begin{equation*}
\begin{split}
\log s = \log\frac{Q}{4\cdot \pi \cdot T} + \log W(u)
\end{split}
\end{equation*}\begin{equation*}
\begin{split}
\log \frac{t}{r^2} = \log\frac{S}{4 T} + \log \frac{1}{u} 
\end{split}
\end{equation*}\begin{itemize}
\item {} 
\sphinxAtStartPar
The constant term in the upper equation (\(\log\frac{Q}{4\cdot \pi \cdot T}\)) can then be solved for \(T\).

\item {} 
\sphinxAtStartPar
Finally, the constant term \((\log\frac{S}{4 T})\) in the lower equation can be used to solve for \(S\).

\end{itemize}

\begin{sphinxadmonition}{important}{Important:}
\sphinxAtStartPar
The Type curve is independent from aquifer properties.
\end{sphinxadmonition}


\subsection{Example}
\label{\detokenize{content/flow/L8/18_wells:example}}
\begin{sphinxadmonition}{note}{Using Type curve}

\sphinxAtStartPar
For the provided pumping data, find the Transmissivity and Storage coefficient when the steady\sphinxhyphen{}discharge was 26.7 L/s.
\end{sphinxadmonition}


\subsubsection{Solution}
\label{\detokenize{content/flow/L8/18_wells:id6}}
\sphinxAtStartPar
The practical application of the Theis method is facilitated by selecting a match point in the range of the data such that corresponding values \(W_A\)
and \(1/u_A\) are “simple”.

\noindent{\hspace*{\fill}\sphinxincludegraphics[width=600\sphinxpxdimen]{{L08_f14}.png}\hspace*{\fill}}

\sphinxAtStartPar
In the example:
\begin{equation*}
\begin{split}
W_A = 10^0 = 1
\end{split}
\end{equation*}\begin{equation*}
\begin{split}
1/u_A = 10^2 = 100
\end{split}
\end{equation*}
\sphinxAtStartPar
Next, values for \(s\) and \(t/r^2\) at the match points are determined. In this example:
\begin{equation*}
\begin{split}
S = 0.2
\end{split}
\end{equation*}\begin{equation*}
\begin{split}
t/r^2 = 0.57
\end{split}
\end{equation*}
\sphinxAtStartPar
Next, we obtain \(T\) from \(T = \frac{Q}{4\cdot\pi\cdot s }\cdot W_A\). For \(Q = 26.7\) L/s, we obtain \(T = 1.06 \cdot 10^{-2} \) m2/s. Finally, \(S\) is obtained from \(S = \frac{4 T\cdot t/r^2}{1/u_A}\). In this example:
\(S = 2.42 \cdot 10^{-4}\)


\subsection{Computer\sphinxhyphen{}Based Comparison of Data and Type Curve}
\label{\detokenize{content/flow/L8/18_wells:computer-based-comparison-of-data-and-type-curve}}
\sphinxAtStartPar
The data in the type\sphinxhyphen{}curve can now easily be fitted using computer simulations. The \hyperref[\detokenize{content/flow/L8/18_wells:type-cur-com}]{Fig.\@ \ref{\detokenize{content/flow/L8/18_wells:type-cur-com}}} is from the {\hyperref[\detokenize{content/tutorials/T7/tutorial_07::doc}]{\sphinxcrossref{\DUrole{doc}{Tutorial 7 \sphinxhyphen{} Wells}}}}. The simulation tool {\hyperref[\detokenize{content/tools/type_curve_fit::doc}]{\sphinxcrossref{\DUrole{doc}{Type curve and fitting pumping data tool}}}} provided can be used to fit user data to the type curve.

\begin{figure}[htbp]
\centering
\capstart

\noindent\sphinxincludegraphics[scale=0.2]{{L08_f15}.png}
\caption{Computationally fitted data to the Type curve}\label{\detokenize{content/flow/L8/18_wells:type-cur-com}}\end{figure}


\section{Chapter Quiz}
\label{\detokenize{content/flow/L8/18_wells:chapter-quiz}}
\begin{sphinxuseclass}{cell}
\begin{sphinxuseclass}{tag_remove-input}
\begin{sphinxuseclass}{tag_hide-output}
\end{sphinxuseclass}
\end{sphinxuseclass}
\end{sphinxuseclass}

\bigskip\hrule\bigskip


\sphinxstepscope


\part{Transport}

\sphinxstepscope

\begin{sphinxuseclass}{cell}
\begin{sphinxuseclass}{tag_hide-input}
\begin{sphinxuseclass}{tag_remove-input}
\end{sphinxuseclass}
\end{sphinxuseclass}
\end{sphinxuseclass}

\chapter{Conservative Transport}
\label{\detokenize{content/transport/L9/21_conservative_transport:conservative-transport}}\label{\detokenize{content/transport/L9/21_conservative_transport::doc}}

\bigskip\hrule\bigskip



\section{Motivation}
\label{\detokenize{content/transport/L9/21_conservative_transport:motivation}}
\sphinxAtStartPar
Groundwater is used by almost half of the world population for drinking purposes. Therefore maintaining it’s quality is very important, not only for human consumption but also for the maintenance of the ecosystem that we live. Groundwater contamination artificial (e.g., from human intervention) or natural (e.g., flood) can impact groundwater quality. Of these two, artificial contamination, e.g., industrial development, improper waste management is more alarming one as quite few contamination (e.g, fuels, radioactive waste) can remain in groundwater for very long time. In this lecture we will learn about processes affecting groundwater quality issues. It has to be mentioned that when we talk about groundwater quality, we mean the change in the original state of groundwater. At many instances this can be groundwater quality can be defined in terms of regulatory standards.


\section{Groundwater flow and transport}
\label{\detokenize{content/transport/L9/21_conservative_transport:groundwater-flow-and-transport}}
\sphinxAtStartPar
Before we begin, let us distinguish between \sphinxstyleemphasis{groundwater flow} and \sphinxstyleemphasis{transport} problems. With \sphinxstyleemphasis{flow} problems we deal with groundwater only, whereas with \sphinxstyleemphasis{transport} problem we deal with particles flowing along with groundwater. It is not required that particle be a contaminant. The particle in this sense may be defines as anything other than water. We begin with a groundwater contamination scenario.

\begin{figure}[htbp]
\centering
\capstart

\noindent\sphinxincludegraphics[scale=0.4]{{T9_f1a}.png}
\caption{Contamination scenario}\label{\detokenize{content/transport/L9/21_conservative_transport:trans-c-2d}}\end{figure}

\sphinxAtStartPar
In the figure, the chemicals (contaminants) first enters the \sphinxstyleemphasis{vadose zone}, an unstatured part of the subsurface from the ground surface. The entry is faciliated with phenomenan such as rainfall. As only limited water are available and there are almost no continuous horizontal flow, the contaminants, mostly as a pure phase,  in this zone often travels vertically until it reaches the groundwater table \sphinxhyphen{} a boundary between an unstaturated and saturated subsurface. Beneath the water\sphinxhyphen{}table, the contaminants mixes with the flowing groundwater and spreads both vertically and horizontally along the saturated zone resulting to a development of a contaminant plume.

\sphinxAtStartPar
\sphinxstylestrong{In this course we learn only of contamination in the saturated zone.}

\sphinxAtStartPar
Before we approach the transport processes, it is important that we learn about the category of groundwater problem. At the top\sphinxhyphen{}most level groundwater transport problem is categorized as
\begin{enumerate}
\sphinxsetlistlabels{\arabic}{enumi}{enumii}{}{.}%
\item {} 
\sphinxAtStartPar
\sphinxstylestrong{Conservative transport problem}: The transport problem which involves only movement of particles along with the water flow. We can call this as non\sphinxhyphen{}reactive transport problems, e.g., dye or (inert) salt transport. In this lecture we deal with this transport type.

\item {} 
\sphinxAtStartPar
\sphinxstylestrong{Reactive transport problem}: The transport problem that deals with reaction along with the flow of particles. This is compound problem type in which flow properties, the conservative part of the transport problem, as well as the chemical properties (and reactions) is considered. Several types of reactions are possible and they need to be considered. We will deal with this type of transport problem in the next lecture.

\end{enumerate}


\section{Conservative transport processes}
\label{\detokenize{content/transport/L9/21_conservative_transport:conservative-transport-processes}}
\sphinxAtStartPar
As defined earlier and also shown in the figure, the three main processes active in conservative transport in groundwater are:
\begin{enumerate}
\sphinxsetlistlabels{\arabic}{enumi}{enumii}{}{.}%
\item {} 
\sphinxAtStartPar
Advection

\item {} 
\sphinxAtStartPar
Dispersion

\item {} 
\sphinxAtStartPar
Diffusion

\end{enumerate}

\sphinxAtStartPar
We discuss each of these process, learn to quantify them and then combine them to set\sphinxhyphen{}up of our system equation for groundwater transport problem.


\subsection{Advection}
\label{\detokenize{content/transport/L9/21_conservative_transport:advection}}
\sphinxAtStartPar
Advection, also called \sphinxstylestrong{convection}, is the transport of components (matter or also energy) due to moving medium. The moving medium, i.e., \sphinxstyleemphasis{a carrier} is, usually groundwater. Thus, quantities that characterizes flow also characterizes advective transport.

\sphinxAtStartPar
Advective transport is not limited to groundwater system. Heating circuit, sailplaning are examples of convection of heat. Similarly, movement of suspended particles in the surface water or by the wind are few examples of advective transport in non\sphinxhyphen{}groundwater systems.

\sphinxAtStartPar
The term \sphinxstylestrong{convection} is more often associated with energy transport, whereas \sphinxstylestrong{advection} is frequently used for transport of the matter.


\subsection{1D Advection and advective mass rate}
\label{\detokenize{content/transport/L9/21_conservative_transport:d-advection-and-advective-mass-rate}}
\sphinxAtStartPar
2D studies are more often the case used in exploring groundwater and its processes. 1D studies can also be sufficient to understand the \sphinxstylestrong{flow} dominated processes such as advection. Also, 1D study is more suitable in lab column studies, as often length of pipes are many times larger that its diameter. For advection, considering 1D pipe flow, to be considered are:
\begin{itemize}
\item {} 
\sphinxAtStartPar
Steady\sphinxhyphen{}state flow with discharge \(Q\) {[}L\(^3\)T\(^{-1}\){]}

\item {} 
\sphinxAtStartPar
Cross\sphinxhyphen{}sectional area, a constant quantity,  \(A\) {[}L\(^2\){]}

\item {} 
\sphinxAtStartPar
Fluid continuity equation: \(Q= n_e\cdot A\cdot v\) = constant, with \(n_e\){[}\sphinxhyphen{}{]} is the \sphinxstylestrong{effective} porosity and \(v\) {[}LT\(^{-1}\){]} the \sphinxstylestrong{linear} groundwater velocity

\end{itemize}

\sphinxAtStartPar
With these information, the advective mass rate \(J_{adv}\) {[}MT\(^{-1}\){]} is obtained from:
\begin{equation*}
\begin{split}
J_{adv} = Q\cdot C
\end{split}
\end{equation*}
\sphinxAtStartPar
with concentration \(C\) {[}ML\(^{-3}\){]}.

\sphinxAtStartPar
Since \(A\) is constant and so is \(Q\), then \(v\) has to be constant too. The consequence of qunatities being constant is that the transport of matter, in  resembles a horizontal pushing.

\begin{sphinxuseclass}{cell}\begin{sphinxVerbatimInput}

\begin{sphinxuseclass}{cell_input}
\begin{sphinxVerbatim}[commandchars=\\\{\}]
\PYG{n+nb}{print}\PYG{p}{(}\PYG{l+s+s2}{\PYGZdq{}}\PYG{l+s+s2}{A quick example: You can change the provided value}\PYG{l+s+se}{\PYGZbs{}n}\PYG{l+s+s2}{\PYGZdq{}}\PYG{p}{)}

\PYG{n+nb}{print}\PYG{p}{(}\PYG{l+s+s2}{\PYGZdq{}}\PYG{l+s+s2}{Let us find advective mass rate exiting a column.}\PYG{l+s+se}{\PYGZbs{}n}\PYG{l+s+se}{\PYGZbs{}n}\PYG{l+s+s2}{Provided are:}\PYG{l+s+s2}{\PYGZdq{}}\PYG{p}{)}

\PYG{n}{R\PYGZus{}1} \PYG{o}{=} \PYG{l+m+mf}{0.25} \PYG{c+c1}{\PYGZsh{} cm, radius of the column}
\PYG{n}{ne\PYGZus{}1} \PYG{o}{=} \PYG{l+m+mf}{0.3} \PYG{c+c1}{\PYGZsh{} (), effective porosity}
\PYG{n}{v\PYGZus{}1} \PYG{o}{=} \PYG{l+m+mf}{0.02} \PYG{c+c1}{\PYGZsh{} cm/s, velocity}
\PYG{n}{C\PYGZus{}1}   \PYG{o}{=} \PYG{l+m+mi}{2} \PYG{c+c1}{\PYGZsh{} mg/L, concentration}

\PYG{c+c1}{\PYGZsh{} intermediate calculation}
\PYG{n}{A\PYGZus{}1} \PYG{o}{=} \PYG{n}{np}\PYG{o}{.}\PYG{n}{pi}\PYG{o}{*}\PYG{n}{R\PYGZus{}1}\PYG{o}{*}\PYG{o}{*}\PYG{l+m+mi}{2} \PYG{c+c1}{\PYGZsh{} Column surface area}
\PYG{n}{Q\PYGZus{}1} \PYG{o}{=} \PYG{n}{ne\PYGZus{}1}\PYG{o}{*}\PYG{n}{A\PYGZus{}1}\PYG{o}{*}\PYG{n}{v\PYGZus{}1} \PYG{c+c1}{\PYGZsh{} cm\PYGZca{}3/s, discharge}

\PYG{c+c1}{\PYGZsh{}solution}
\PYG{n}{J\PYGZus{}adv} \PYG{o}{=} \PYG{n}{Q\PYGZus{}1}\PYG{o}{*}\PYG{n}{C\PYGZus{}1}

\PYG{n+nb}{print}\PYG{p}{(}\PYG{l+s+s2}{\PYGZdq{}}\PYG{l+s+s2}{Radius = }\PYG{l+s+si}{\PYGZob{}\PYGZcb{}}\PYG{l+s+s2}{ cm}\PYG{l+s+se}{\PYGZbs{}n}\PYG{l+s+s2}{Water flow rate = }\PYG{l+s+si}{\PYGZob{}\PYGZcb{}}\PYG{l+s+s2}{ cm/s}\PYG{l+s+se}{\PYGZbs{}n}\PYG{l+s+s2}{Input concentration = }\PYG{l+s+si}{\PYGZob{}\PYGZcb{}}\PYG{l+s+s2}{ mg/L and}\PYG{l+s+se}{\PYGZbs{}n}\PYG{l+s+s2}{Effective porosity = }\PYG{l+s+si}{\PYGZob{}:02.2f\PYGZcb{}}\PYG{l+s+s2}{\PYGZdq{}}\PYG{o}{.}\PYG{n}{format}\PYG{p}{(}\PYG{n}{R\PYGZus{}1}\PYG{p}{,} \PYG{n}{v\PYGZus{}1}\PYG{p}{,} \PYG{n}{C\PYGZus{}1}\PYG{p}{,} \PYG{n}{ne\PYGZus{}1}\PYG{p}{)}\PYG{p}{,} \PYG{l+s+s2}{\PYGZdq{}}\PYG{l+s+se}{\PYGZbs{}n}\PYG{l+s+s2}{\PYGZdq{}}\PYG{p}{)}
\PYG{n+nb}{print}\PYG{p}{(}\PYG{l+s+s2}{\PYGZdq{}}\PYG{l+s+s2}{The resulting advective mass flow rate is }\PYG{l+s+si}{\PYGZob{}:02.4f\PYGZcb{}}\PYG{l+s+s2}{ mg/s}\PYG{l+s+s2}{\PYGZdq{}}\PYG{o}{.}\PYG{n}{format}\PYG{p}{(}\PYG{n}{J\PYGZus{}adv}\PYG{p}{)}\PYG{p}{)}
\end{sphinxVerbatim}

\end{sphinxuseclass}\end{sphinxVerbatimInput}
\begin{sphinxVerbatimOutput}

\begin{sphinxuseclass}{cell_output}
\begin{sphinxVerbatim}[commandchars=\\\{\}]
A quick example: You can change the provided value

Let us find advective mass rate exiting a column.

Provided are:
Radius = 0.25 cm
Water flow rate = 0.02 cm/s
Input concentration = 2 mg/L and
Effective porosity = 0.30 

The resulting advective mass flow rate is 0.0024 mg/s
\end{sphinxVerbatim}

\end{sphinxuseclass}\end{sphinxVerbatimOutput}

\end{sphinxuseclass}
\begin{sphinxuseclass}{cell}\begin{sphinxVerbatimInput}

\begin{sphinxuseclass}{cell_input}
\begin{sphinxVerbatim}[commandchars=\\\{\}]
\PYG{c+c1}{\PYGZsh{} simulating advection 1D}

\PYG{n+nb}{print}\PYG{p}{(}\PYG{l+s+s2}{\PYGZdq{}}\PYG{l+s+se}{\PYGZbs{}n}\PYG{l+s+s2}{Simulating advective flow}\PYG{l+s+se}{\PYGZbs{}n}\PYG{l+s+s2}{\PYGZdq{}}\PYG{p}{)}

\PYG{k+kn}{import} \PYG{n+nn}{numpy} \PYG{k}{as} \PYG{n+nn}{np}
\PYG{k+kn}{import} \PYG{n+nn}{matplotlib}\PYG{n+nn}{.}\PYG{n+nn}{pyplot} \PYG{k}{as} \PYG{n+nn}{plt}

\PYG{n}{N}\PYG{o}{=} \PYG{l+m+mi}{100} \PYG{c+c1}{\PYGZsh{} number of cell \PYGZhy{} i.e. compartment a column is divided into}
\PYG{n}{c0} \PYG{o}{=} \PYG{l+m+mi}{0} \PYG{c+c1}{\PYGZsh{} initial concentration}
\PYG{n}{cin} \PYG{o}{=} \PYG{l+m+mi}{1} \PYG{c+c1}{\PYGZsh{} input concentration}
\PYG{n}{c1} \PYG{o}{=} \PYG{n}{c0}\PYG{o}{*}\PYG{n}{np}\PYG{o}{.}\PYG{n}{ones}\PYG{p}{(}\PYG{p}{(}\PYG{l+m+mi}{1}\PYG{p}{,}\PYG{n}{N}\PYG{p}{)}\PYG{p}{)} \PYG{c+c1}{\PYGZsh{} initializing \PYGZhy{} i.e., all compartment has zero conc.}

\PYG{k}{for} \PYG{n}{i} \PYG{o+ow}{in} \PYG{n+nb}{range}\PYG{p}{(}\PYG{l+m+mi}{1}\PYG{p}{,} \PYG{n}{N}\PYG{o}{+}\PYG{l+m+mi}{1}\PYG{p}{)}\PYG{p}{:} \PYG{c+c1}{\PYGZsh{} number of time steps\PYGZhy{} we make it equivalent to cell number.}
    \PYG{n}{c1} \PYG{o}{=} \PYG{n}{np}\PYG{o}{.}\PYG{n}{roll}\PYG{p}{(}\PYG{n}{c1}\PYG{p}{,} \PYG{n}{N}\PYG{o}{\PYGZhy{}}\PYG{l+m+mi}{1}\PYG{p}{)} \PYG{c+c1}{\PYGZsh{} \PYGZdq{}roll\PYGZdq{} is numpy function for shifting}
    \PYG{n}{c1}\PYG{p}{[}\PYG{l+m+mi}{0}\PYG{p}{,}\PYG{l+m+mi}{0}\PYG{p}{]} \PYG{o}{=} \PYG{n}{cin} \PYG{c+c1}{\PYGZsh{} make the first cell = cin again. This is because np.roll uses the last value as the first one.}
    
    \PYG{n}{plt}\PYG{o}{.}\PYG{n}{plot}\PYG{p}{(}\PYG{n}{c1}\PYG{o}{.}\PYG{n}{T}\PYG{p}{)}
    \PYG{n}{plt}\PYG{o}{.}\PYG{n}{xlabel}\PYG{p}{(}\PYG{l+s+s2}{\PYGZdq{}}\PYG{l+s+s2}{space (m)}\PYG{l+s+s2}{\PYGZdq{}}\PYG{p}{)}\PYG{p}{;} \PYG{n}{plt}\PYG{o}{.}\PYG{n}{ylabel}\PYG{p}{(}\PYG{l+s+s2}{\PYGZdq{}}\PYG{l+s+s2}{Concentration (mg/L)}\PYG{l+s+s2}{\PYGZdq{}}\PYG{p}{)}
    \PYG{n}{plt}\PYG{o}{.}\PYG{n}{xlim}\PYG{p}{(}\PYG{p}{[}\PYG{l+m+mi}{0}\PYG{p}{,}\PYG{n}{N}\PYG{p}{]}\PYG{p}{)}\PYG{p}{;} \PYG{n}{plt}\PYG{o}{.}\PYG{n}{ylim}\PYG{p}{(}\PYG{p}{[}\PYG{l+m+mi}{0}\PYG{p}{,}\PYG{n}{cin}\PYG{p}{]}\PYG{p}{)}
\end{sphinxVerbatim}

\end{sphinxuseclass}\end{sphinxVerbatimInput}
\begin{sphinxVerbatimOutput}

\begin{sphinxuseclass}{cell_output}
\begin{sphinxVerbatim}[commandchars=\\\{\}]
Simulating advective flow
\end{sphinxVerbatim}

\noindent\sphinxincludegraphics{{C:/Users/vibhu/GWtextbook/_build/jupyter_execute/21_conservative_transport_5_1}.png}

\end{sphinxuseclass}\end{sphinxVerbatimOutput}

\end{sphinxuseclass}

\subsection{1D advection in variable cross\sphinxhyphen{}section}
\label{\detokenize{content/transport/L9/21_conservative_transport:d-advection-in-variable-cross-section}}
\sphinxAtStartPar
In the just completed case, the column was of uniform diameter. So, what happens when:

\sphinxAtStartPar
In this case the equation of continuity has to be explored. The continuity equation for variable cross section is:

\sphinxAtStartPar
\textbackslash{}begin\{align*\}
Q\_1 \&=Q\_2 \textbackslash{}
A\_1 v\_1 \&= A\_2 v\_2
\textbackslash{}end\{align*\}

\sphinxAtStartPar
As \(Q_1=Q_2= Q\), and when \(A\) changes then, \(v\) has to change as well. In this case considering that \(n_e\) is spatially constant, \(v\) will be inversely proportional to the cross\sphinxhyphen{}section. i.e., increase in \(A\) will result to decrease in \(v\) and vice\sphinxhyphen{}versa.

\sphinxAtStartPar
Based on this, the mass (patch in the figure), is actually equal in both figure and they are onyl spread out laterally. In other words, as cross\sphinxhyphen{}section increased and velocity decreased, the mass extension is reduced along the flow direction. The effect of \sphinxstyleemphasis{advection} is that the \sphinxstyleemphasis{concentration}, which is the total mass per litre of groundwater, remains unchanged in the patched mass area.


\subsection{Is advection enough?}
\label{\detokenize{content/transport/L9/21_conservative_transport:is-advection-enough}}
\sphinxAtStartPar
Probably not. Advection may only be possible in uniformly packed porous medium with also perfectly uniform particle size distribution of medium particles. This is rather exception. In normal cases, the the particle transport will spread out leading some matter exiting earlier then remaining others. This is explained by process called \sphinxstylestrong{mechanical dispersion}.

\begin{figure}[htbp]
\centering
\capstart

\noindent\sphinxincludegraphics[scale=0.7]{{T9_f3}.png}
\caption{Advection\sphinxhyphen{}dispersion processes}\label{\detokenize{content/transport/L9/21_conservative_transport:ad-dis}}\end{figure}
\begin{quote}

\sphinxAtStartPar
Note: 1D only have longitudinal dispersivity. Transverse dispersivity is 2D/3D property
This is a footnote reference.
\end{quote}


\section{Mechanical dispersion}
\label{\detokenize{content/transport/L9/21_conservative_transport:mechanical-dispersion}}
\sphinxAtStartPar
\sphinxstylestrong{Mechanical dispersion} can result due to alone or combination of several porous medium and flow hydraulics properties. In general the following three reasons are considered when explaining the mechanical dispersion process:

\begin{figure}[htbp]
\centering
\capstart

\noindent\sphinxincludegraphics[scale=0.4]{{T9_f4}.png}
\caption{Mechanical dispersion mechanisms}\label{\detokenize{content/transport/L9/21_conservative_transport:mech-dis}}\end{figure}
\begin{quote}

\sphinxAtStartPar
a.  The varying flow velocities across each individual pore. A \sphinxstylestrong{parabolic velocity profile} results as close to the surface there is resistance to flow and the maximum velocity is at the centreline.
\end{quote}
\begin{quote}

\sphinxAtStartPar
b. Different flow velocities in different pores. Pore sizes can vary due to different particle sizes and also due to non\sphinxhyphen{}uniform compaction. Larger pore will then have higher velocity compared to smaller pores.
\end{quote}
\begin{quote}

\sphinxAtStartPar
c. Varying flow paths of individual flow streams. The water streams can take its own path around the particles. This can lead some taking shorter exit path compared to other streams.
\end{quote}

\sphinxAtStartPar
These effects, either individually or any combination, result in different transport distances and different transport times, respectively. Thus, matter carried by streamline will also have different transport distances and different times\sphinxhyphen{} or the solute are \sphinxstylestrong{dispersed}.


\subsection{Dispersive Mass flow rate}
\label{\detokenize{content/transport/L9/21_conservative_transport:dispersive-mass-flow-rate}}
\sphinxAtStartPar
The dispersive mass flow rate \(J_{disp, m}\) {[}MT\(^{-1}\){]} due to mechanical dispersion in porous media is:
\begin{itemize}
\item {} 
\sphinxAtStartPar
propertional to the cross\sphinxhyphen{}sectional area \(A\)

\item {} 
\sphinxAtStartPar
proportional to the linear velocity, \(v\)

\item {} 
\sphinxAtStartPar
proportional to the difference in concentration \(\Delta C\) between transport points

\item {} 
\sphinxAtStartPar
inversely propertional to the transport distance \(L\)

\end{itemize}

\sphinxAtStartPar
These jointly results to \(J_{disp, m}\):
\textbackslash{}begin\{align*\}
J\_\{disp, m\} \&\textbackslash{}propto \sphinxhyphen{}n\_e \textbackslash{}cdot A \textbackslash{}cdot v \textbackslash{}cdot \textbackslash{}frac\{\textbackslash{}Delta C\}\{L\}\textbackslash{}
J\_\{disp, m\} \&= \sphinxhyphen{} \textbackslash{}alpha \textbackslash{}cdot n\_e \textbackslash{}cdot A \textbackslash{}cdot v \textbackslash{}cdot \textbackslash{}frac\{\textbackslash{}Delta C\}\{L\}
\textbackslash{}end\{align*\}

\sphinxAtStartPar
The ratio \(\frac{\Delta C}{L}\) {[}ML\(^{-4}\){]} is called \sphinxstylestrong{concentration gradient}. The \sphinxstylestrong{negative} sign is to indicate that dispersive mass flow is from region with higher concentrations to regions with lower concentrations.

\sphinxAtStartPar
\(\alpha\){[}L{]} is proportionality constant called \sphinxstylestrong{dispersivity}. This quantity equals the value of the dispersive mass flow through a unit cross\sphinxhyphen{}sectional area for a unit concentration gradient and a unit linear velocity.


\subsection{Dispersivity and Mechanical Dispersion coefficient}
\label{\detokenize{content/transport/L9/21_conservative_transport:dispersivity-and-mechanical-dispersion-coefficient}}
\sphinxAtStartPar
It is more convinient to use dispersivity together with groundwater velocity. Thus, a qunatity called \sphinxstylestrong{mechnical dispersion coefficient} \(D_{mech}\) {[}L\(^2\)T\(^{-1}\){]} is defined as:
\begin{equation*}
\begin{split}
D_{mech} = \alpha \cdot v
\end{split}
\end{equation*}
\sphinxAtStartPar
Following the equation, the \(D_{mech}\) is a qunatity that is dependent on both the properties of porous medium, which is characterized by dispersivity (\(\alpha\)) and the property of flow hydrauclics that is characterized by flow velocity (\(v\)).

\sphinxAtStartPar
Using mechanical dispersion coefficient, the dispersive mass flow can be redefine as
\begin{equation*}
\begin{split}
J_{disp, m} = -  n_e \cdot A \cdot D_{mech} \cdot \frac{\Delta C}{L}
\end{split}
\end{equation*}
\begin{sphinxuseclass}{cell}\begin{sphinxVerbatimInput}

\begin{sphinxuseclass}{cell_input}
\begin{sphinxVerbatim}[commandchars=\\\{\}]
\PYG{n+nb}{print}\PYG{p}{(}\PYG{l+s+s2}{\PYGZdq{}}\PYG{l+s+s2}{A quick example: You can change the provided values}\PYG{l+s+se}{\PYGZbs{}n}\PYG{l+s+s2}{\PYGZdq{}}\PYG{p}{)}

\PYG{n+nb}{print}\PYG{p}{(}\PYG{l+s+s2}{\PYGZdq{}}\PYG{l+s+s2}{Let us find dispersive mass rate exiting a column.}\PYG{l+s+se}{\PYGZbs{}n}\PYG{l+s+se}{\PYGZbs{}n}\PYG{l+s+s2}{Provided are:}\PYG{l+s+s2}{\PYGZdq{}}\PYG{p}{)}

\PYG{n}{L\PYGZus{}2}  \PYG{o}{=} \PYG{l+m+mi}{50} \PYG{c+c1}{\PYGZsh{} cm, length of the pipe}
\PYG{n}{R\PYGZus{}2}  \PYG{o}{=} \PYG{l+m+mf}{0.25} \PYG{c+c1}{\PYGZsh{} cm, radius of the column}
\PYG{n}{ne\PYGZus{}2} \PYG{o}{=} \PYG{l+m+mf}{0.3} \PYG{c+c1}{\PYGZsh{} (), effective porosity}
\PYG{n}{v\PYGZus{}2}  \PYG{o}{=} \PYG{l+m+mf}{0.02} \PYG{c+c1}{\PYGZsh{} cm/s, velocity}
\PYG{n}{Ci\PYGZus{}2} \PYG{o}{=} \PYG{l+m+mi}{10} \PYG{c+c1}{\PYGZsh{} mg/L, inlet concentration}
\PYG{n}{Co\PYGZus{}2} \PYG{o}{=} \PYG{l+m+mi}{2} \PYG{c+c1}{\PYGZsh{} mg/L, outlet concentration}
\PYG{n}{a\PYGZus{}2}  \PYG{o}{=} \PYG{l+m+mi}{1} \PYG{c+c1}{\PYGZsh{} cm, dispersivity}

\PYG{c+c1}{\PYGZsh{} intermediate calculation}
\PYG{n}{A\PYGZus{}2} \PYG{o}{=} \PYG{n}{np}\PYG{o}{.}\PYG{n}{pi}\PYG{o}{*}\PYG{n}{R\PYGZus{}2}\PYG{o}{*}\PYG{o}{*}\PYG{l+m+mi}{2} \PYG{c+c1}{\PYGZsh{} Column surface area}
\PYG{n}{Cg\PYGZus{}2} \PYG{o}{=} \PYG{p}{(}\PYG{n}{Ci\PYGZus{}2}\PYG{o}{\PYGZhy{}}\PYG{n}{Co\PYGZus{}2}\PYG{p}{)}\PYG{o}{/}\PYG{n}{L\PYGZus{}2} \PYG{c+c1}{\PYGZsh{} mg/L\PYGZhy{}cm, concentration gradient}

\PYG{c+c1}{\PYGZsh{}solution}
\PYG{n}{Jm\PYGZus{}dis} \PYG{o}{=} \PYG{n}{ne\PYGZus{}2}\PYG{o}{*}\PYG{n}{A\PYGZus{}2}\PYG{o}{*}\PYG{n}{a\PYGZus{}2}\PYG{o}{*}\PYG{n}{v\PYGZus{}2}\PYG{o}{*}\PYG{n}{Cg\PYGZus{}2}



\PYG{n+nb}{print}\PYG{p}{(}\PYG{l+s+s2}{\PYGZdq{}}\PYG{l+s+s2}{Length of column = }\PYG{l+s+si}{\PYGZob{}\PYGZcb{}}\PYG{l+s+s2}{ cm}\PYG{l+s+se}{\PYGZbs{}n}\PYG{l+s+s2}{Radius of column = }\PYG{l+s+si}{\PYGZob{}\PYGZcb{}}\PYG{l+s+s2}{ cm}\PYG{l+s+se}{\PYGZbs{}n}\PYG{l+s+s2}{Water flow rate = }\PYG{l+s+si}{\PYGZob{}\PYGZcb{}}\PYG{l+s+s2}{ cm/s}\PYG{l+s+se}{\PYGZbs{}n}\PYG{l+s+s2}{Inlet concentration = }\PYG{l+s+si}{\PYGZob{}\PYGZcb{}}\PYG{l+s+s2}{ mg/L}\PYG{l+s+se}{\PYGZbs{}n}\PYG{l+s+s2}{Outlet concentration = }\PYG{l+s+si}{\PYGZob{}\PYGZcb{}}\PYG{l+s+s2}{ mg/L }\PYG{l+s+se}{\PYGZbs{}}
\PYG{l+s+se}{\PYGZbs{}n}\PYG{l+s+s2}{Effective porosity = }\PYG{l+s+si}{\PYGZob{}:02.2f\PYGZcb{}}\PYG{l+s+se}{\PYGZbs{}n}\PYG{l+s+s2}{Dispersivity = }\PYG{l+s+si}{\PYGZob{}\PYGZcb{}}\PYG{l+s+s2}{\PYGZdq{}}\PYG{o}{.}\PYG{n}{format}\PYG{p}{(}\PYG{n}{L\PYGZus{}2}\PYG{p}{,}\PYG{n}{R\PYGZus{}2}\PYG{p}{,} \PYG{n}{v\PYGZus{}2}\PYG{p}{,} \PYG{n}{Ci\PYGZus{}2}\PYG{p}{,} \PYG{n}{Co\PYGZus{}2}\PYG{p}{,}\PYG{n}{ne\PYGZus{}2}\PYG{p}{,} \PYG{n}{a\PYGZus{}2}\PYG{p}{)}\PYG{p}{,} \PYG{l+s+s2}{\PYGZdq{}}\PYG{l+s+se}{\PYGZbs{}n}\PYG{l+s+s2}{\PYGZdq{}}\PYG{p}{)}

\PYG{n+nb}{print}\PYG{p}{(}\PYG{l+s+s2}{\PYGZdq{}}\PYG{l+s+s2}{The resulting dispersive mass flow is }\PYG{l+s+si}{\PYGZob{}:02.4f\PYGZcb{}}\PYG{l+s+s2}{ mg/s}\PYG{l+s+s2}{\PYGZdq{}}\PYG{o}{.}\PYG{n}{format}\PYG{p}{(}\PYG{n}{Jm\PYGZus{}dis}\PYG{p}{)}\PYG{p}{)}
\end{sphinxVerbatim}

\end{sphinxuseclass}\end{sphinxVerbatimInput}
\begin{sphinxVerbatimOutput}

\begin{sphinxuseclass}{cell_output}
\begin{sphinxVerbatim}[commandchars=\\\{\}]
A quick example: You can change the provided values

Let us find dispersive mass rate exiting a column.

Provided are:
Length of column = 50 cm
Radius of column = 0.25 cm
Water flow rate = 0.02 cm/s
Inlet concentration = 10 mg/L
Outlet concentration = 2 mg/L 
Effective porosity = 0.30
Dispersivity = 1 

The resulting dispersive mass flow is 0.0002 mg/s
\end{sphinxVerbatim}

\end{sphinxuseclass}\end{sphinxVerbatimOutput}

\end{sphinxuseclass}
\begin{sphinxuseclass}{cell}\begin{sphinxVerbatimInput}

\begin{sphinxuseclass}{cell_input}
\begin{sphinxVerbatim}[commandchars=\\\{\}]
\PYG{c+c1}{\PYGZsh{}Simulating mechanical dispersion}

\PYG{n+nb}{print}\PYG{p}{(}\PYG{l+s+s2}{\PYGZdq{}}\PYG{l+s+se}{\PYGZbs{}n}\PYG{l+s+s2}{Simulating mechanical dispersion}\PYG{l+s+se}{\PYGZbs{}n}\PYG{l+s+s2}{You can change input values to see the effect}\PYG{l+s+s2}{\PYGZdq{}} \PYG{p}{)}

\PYG{n}{N} \PYG{o}{=} \PYG{l+m+mi}{100}
\PYG{n}{c0} \PYG{o}{=} \PYG{l+m+mi}{0}
\PYG{n}{cin} \PYG{o}{=} \PYG{l+m+mi}{1}
\PYG{n}{Neumann} \PYG{o}{=} \PYG{l+m+mf}{0.5}\PYG{p}{;} \PYG{c+c1}{\PYGZsh{} Neumann number ensures that transport is combination of adv and disp.}
\PYG{n}{c1} \PYG{o}{=} \PYG{n}{c0}\PYG{o}{*}\PYG{n}{np}\PYG{o}{.}\PYG{n}{ones}\PYG{p}{(}\PYG{p}{(}\PYG{l+m+mi}{1}\PYG{p}{,}\PYG{n}{N}\PYG{p}{)}\PYG{p}{)}
\PYG{n}{c2} \PYG{o}{=} \PYG{n}{c0}\PYG{o}{*}\PYG{n}{np}\PYG{o}{.}\PYG{n}{zeros}\PYG{p}{(}\PYG{p}{(}\PYG{l+m+mi}{1}\PYG{p}{,}\PYG{n}{N}\PYG{p}{)}\PYG{p}{)}
\PYG{n}{c}\PYG{o}{=}\PYG{n}{c1}

\PYG{k}{for} \PYG{n}{i} \PYG{o+ow}{in} \PYG{n+nb}{range}\PYG{p}{(}\PYG{l+m+mi}{1}\PYG{p}{,} \PYG{n}{N}\PYG{o}{+}\PYG{l+m+mi}{1}\PYG{p}{)}\PYG{p}{:}
    
    \PYG{c+c1}{\PYGZsh{}dispersion component using Neumann number (D. Dt/Dx² \PYGZhy{}D= Delta )}
    \PYG{k}{for} \PYG{n}{i} \PYG{o+ow}{in} \PYG{n+nb}{range}\PYG{p}{(}\PYG{l+m+mi}{2}\PYG{p}{,}\PYG{n}{N}\PYG{p}{)}\PYG{p}{:}
        \PYG{n}{c2}\PYG{p}{[}\PYG{l+m+mi}{0}\PYG{p}{,}\PYG{n}{i}\PYG{o}{\PYGZhy{}}\PYG{l+m+mi}{1}\PYG{p}{]} \PYG{o}{=} \PYG{n}{c1}\PYG{p}{[}\PYG{l+m+mi}{0}\PYG{p}{,}\PYG{n}{i}\PYG{o}{\PYGZhy{}}\PYG{l+m+mi}{1}\PYG{p}{]} \PYG{o}{+} \PYG{n}{Neumann}\PYG{o}{*}\PYG{p}{(}\PYG{n}{c1}\PYG{p}{[}\PYG{l+m+mi}{0}\PYG{p}{,}\PYG{n}{i}\PYG{o}{\PYGZhy{}}\PYG{l+m+mi}{2}\PYG{p}{]}\PYG{o}{\PYGZhy{}}\PYG{l+m+mi}{2}\PYG{o}{*}\PYG{n}{c1}\PYG{p}{[}\PYG{l+m+mi}{0}\PYG{p}{,}\PYG{n}{i}\PYG{o}{\PYGZhy{}}\PYG{l+m+mi}{1}\PYG{p}{]}\PYG{o}{+}\PYG{n}{c1}\PYG{p}{[}\PYG{l+m+mi}{0}\PYG{p}{,}\PYG{n}{i}\PYG{p}{]}\PYG{p}{)}\PYG{p}{;} \PYG{c+c1}{\PYGZsh{} FD stensil for d²C/dx²}
           
    \PYG{n}{c2}\PYG{p}{[}\PYG{l+m+mi}{0}\PYG{p}{,}\PYG{l+m+mi}{0}\PYG{p}{]} \PYG{o}{=} \PYG{n}{c1}\PYG{p}{[}\PYG{l+m+mi}{0}\PYG{p}{,}\PYG{l+m+mi}{0}\PYG{p}{]} \PYG{o}{+} \PYG{n}{Neumann}\PYG{o}{*}\PYG{p}{(}\PYG{n}{cin} \PYG{o}{\PYGZhy{}} \PYG{l+m+mi}{2}\PYG{o}{*}\PYG{n}{c1}\PYG{p}{[}\PYG{l+m+mi}{0}\PYG{p}{,}\PYG{l+m+mi}{0}\PYG{p}{]}\PYG{o}{+}\PYG{n}{c1}\PYG{p}{[}\PYG{l+m+mi}{0}\PYG{p}{,}\PYG{l+m+mi}{1}\PYG{p}{]}\PYG{p}{)}\PYG{p}{;} \PYG{c+c1}{\PYGZsh{} the first cell value}
    \PYG{n}{c2}\PYG{p}{[}\PYG{l+m+mi}{0}\PYG{p}{,}\PYG{n}{N}\PYG{o}{\PYGZhy{}}\PYG{l+m+mi}{1}\PYG{p}{]} \PYG{o}{=} \PYG{n}{c1}\PYG{p}{[}\PYG{l+m+mi}{0}\PYG{p}{,}\PYG{n}{N}\PYG{o}{\PYGZhy{}}\PYG{l+m+mi}{1}\PYG{p}{]} \PYG{o}{+} \PYG{n}{Neumann}\PYG{o}{*}\PYG{p}{(}\PYG{n}{c1}\PYG{p}{[}\PYG{l+m+mi}{0}\PYG{p}{,}\PYG{n}{N}\PYG{o}{\PYGZhy{}}\PYG{l+m+mi}{2}\PYG{p}{]} \PYG{o}{\PYGZhy{}} \PYG{n}{c1}\PYG{p}{[}\PYG{l+m+mi}{0}\PYG{p}{,}\PYG{n}{N}\PYG{o}{\PYGZhy{}}\PYG{l+m+mi}{1}\PYG{p}{]}\PYG{p}{)}\PYG{p}{;} \PYG{c+c1}{\PYGZsh{} the lass cell value}
    \PYG{n}{c1} \PYG{o}{=}\PYG{n}{c2}\PYG{p}{;}
    
    \PYG{c+c1}{\PYGZsh{}shifting cell \PYGZhy{} advection component}
    \PYG{n}{c1} \PYG{o}{=} \PYG{n}{np}\PYG{o}{.}\PYG{n}{roll}\PYG{p}{(}\PYG{n}{c1}\PYG{p}{,} \PYG{n}{N}\PYG{o}{+}\PYG{l+m+mi}{1}\PYG{p}{)}\PYG{p}{;}  
    \PYG{n}{c1}\PYG{p}{[}\PYG{l+m+mi}{0}\PYG{p}{,}\PYG{l+m+mi}{0}\PYG{p}{]} \PYG{o}{=} \PYG{n}{cin}\PYG{p}{;}
    
    \PYG{n}{plt}\PYG{o}{.}\PYG{n}{plot}\PYG{p}{(}\PYG{n}{c1}\PYG{o}{.}\PYG{n}{T}\PYG{p}{)}\PYG{p}{;}
    \PYG{n}{plt}\PYG{o}{.}\PYG{n}{xlabel}\PYG{p}{(}\PYG{l+s+s2}{\PYGZdq{}}\PYG{l+s+s2}{space (m)}\PYG{l+s+s2}{\PYGZdq{}}\PYG{p}{)}\PYG{p}{;} \PYG{n}{plt}\PYG{o}{.}\PYG{n}{ylabel}\PYG{p}{(}\PYG{l+s+s2}{\PYGZdq{}}\PYG{l+s+s2}{Concentration (mg/L)}\PYG{l+s+s2}{\PYGZdq{}}\PYG{p}{)}
    \PYG{n}{plt}\PYG{o}{.}\PYG{n}{xlim}\PYG{p}{(}\PYG{p}{[}\PYG{l+m+mi}{0}\PYG{p}{,}\PYG{n}{N}\PYG{p}{]}\PYG{p}{)}\PYG{p}{;} \PYG{n}{plt}\PYG{o}{.}\PYG{n}{ylim}\PYG{p}{(}\PYG{p}{[}\PYG{l+m+mi}{0}\PYG{p}{,}\PYG{n}{cin}\PYG{p}{]}\PYG{p}{)}
\end{sphinxVerbatim}

\end{sphinxuseclass}\end{sphinxVerbatimInput}
\begin{sphinxVerbatimOutput}

\begin{sphinxuseclass}{cell_output}
\begin{sphinxVerbatim}[commandchars=\\\{\}]
Simulating mechanical dispersion
You can change input values to see the effect
\end{sphinxVerbatim}

\noindent\sphinxincludegraphics{{C:/Users/vibhu/GWtextbook/_build/jupyter_execute/21_conservative_transport_12_1}.png}

\end{sphinxuseclass}\end{sphinxVerbatimOutput}

\end{sphinxuseclass}

\subsection{Hydrodynamic Dispersion}
\label{\detokenize{content/transport/L9/21_conservative_transport:hydrodynamic-dispersion}}
\sphinxAtStartPar
In groundwater transport study mechanical dispersion and diffusion coefficient are summed up and a common term \sphinxstylestrong{hydrodynamic dispersion} \(D_{hyd}\) {[}L\(^2\)T\(^{-1}\){]}. However, the diffusion coefficient \(D\) used in diffusion mass flow equation is valid for liquid water. Groundwater exist with solid porous medium. Thus, pore diffusion coefficient \(D_p\) {[}L\(^2\)T\(^{-1}\){]} is considered. In general:
\begin{equation*}
\begin{split}
D_p < D
\end{split}
\end{equation*}
\sphinxAtStartPar
and it is frequently assumed that
\begin{equation*}
\begin{split}
D_p = n_e \cdot D
\end{split}
\end{equation*}
\sphinxAtStartPar
where \(n_e\) is effectively porosity. Several other emperical formulae are found in literature relating \(D\) and \(D_p\). With \(D_p\), defined the \sphinxstylestrong{hydrodynamic dispersion} is defined as:
\begin{equation*}
\begin{split}
D_{hyd} = D_{mech} + D_p = \alpha \cdot v + D_p = \alpha \cdot v + n_e \cdot D
\end{split}
\end{equation*}

\section{1D Diffusion}
\label{\detokenize{content/transport/L9/21_conservative_transport:d-diffusion}}
\sphinxAtStartPar
Contrasting with the advection and mechanical dispersion \sphinxhyphen{} mostly a flow influenced transport processes, the \sphinxstylestrong{diffusion} led transport depends on \sphinxstylestrong{concentration gradient}, i.e., flow is not required for transport.

\sphinxAtStartPar
The 1D diffusion transpor is found be:
\begin{itemize}
\item {} 
\sphinxAtStartPar
proportional to the cross\sphinxhyphen{}sectional area \(A\).

\item {} 
\sphinxAtStartPar
proportional to the concentration gradient \(\frac{\Delta C}{L}\)

\end{itemize}

\sphinxAtStartPar
The diffusive mass transport \(J_{diff}\) {[}MT\(^{-1}\){]} is thus:

\sphinxAtStartPar
\textbackslash{}begin\{align*\}
J\_\{diff\} \&\textbackslash{}propto \sphinxhyphen{} A\textbackslash{}cdot\textbackslash{}frac\{\textbackslash{}Delta C\}\{L\}\textbackslash{}
J\_\{diff\} \&= \sphinxhyphen{} D \textbackslash{}cdot A\textbackslash{}cdot\textbackslash{}frac\{\textbackslash{}Delta C\}\{L\}\textbackslash{}
\textbackslash{}end\{align*\}

\sphinxAtStartPar
\(D\) {[}L\(^2\)T\(^{-1}\){]} is the proportional coefficient called diffusion coefficient. And, the negative sign is for indicating that the flow is from the region with higher concentration to the region with lower concentration. As can be observed from the equation, \(D\) will equal \(J_{diff}\) for unit cross\sphinxhyphen{}section for a unit concentration gradient. The value of \(D\) of dissolved chemicals in liquid water (not exactly in groundwater, which is water in porous media) are mostly in the range from \(10^{-10}\) m\(^2\)/s \sphinxhyphen{}  \(10^{-9}\) m\(^2\)/s. The value of \(D\) of chemicals in gases are larger by about four orders of magnitude, i.e., the range in this case from from \(10^{-6}\) m\(^2\)/s \sphinxhyphen{}  \(10^{-5}\) m\(^2\)/s.

\sphinxAtStartPar
The diffusion phenomena can be demonstrated from the figure below. The fig shows spread of solute in the absence of flow (\(v=0\) and consequently \(Q=0\)). The spread continues with time \(t\) until solute concentration in the entire medium is leveled.

\begin{figure}[htbp]
\centering
\capstart

\noindent\sphinxincludegraphics[scale=0.6]{{T9_f5}.png}
\caption{The diffusion phenomena}\label{\detokenize{content/transport/L9/21_conservative_transport:dif}}\end{figure}

\begin{sphinxuseclass}{cell}\begin{sphinxVerbatimInput}

\begin{sphinxuseclass}{cell_input}
\begin{sphinxVerbatim}[commandchars=\\\{\}]
\PYG{n+nb}{print}\PYG{p}{(}\PYG{l+s+s2}{\PYGZdq{}}\PYG{l+s+se}{\PYGZbs{}n}\PYG{l+s+s2}{Simulating diffusion }\PYG{l+s+se}{\PYGZbs{}n}\PYG{l+s+s2}{You can change input values to see the effect}\PYG{l+s+se}{\PYGZbs{}n}\PYG{l+s+s2}{\PYGZdq{}} \PYG{p}{)} 
\PYG{c+c1}{\PYGZsh{} Initial Conditions}
\PYG{n}{nx} \PYG{o}{=} \PYG{l+m+mi}{20}
\PYG{n}{dx} \PYG{o}{=} \PYG{l+m+mi}{2} \PYG{o}{/} \PYG{p}{(}\PYG{n}{nx} \PYG{o}{\PYGZhy{}} \PYG{l+m+mi}{1}\PYG{p}{)} \PYG{c+c1}{\PYGZsh{} number of space compartment}
\PYG{n}{nt} \PYG{o}{=} \PYG{l+m+mi}{20}    \PYG{c+c1}{\PYGZsh{}the number of timesteps we want to calculate}
\PYG{n}{Neumann} \PYG{o}{=} \PYG{l+m+mf}{0.5} \PYG{c+c1}{\PYGZsh{} [ ], Neumann number = D. Dt/Dx² \PYGZhy{}D= Delta}

\PYG{n}{u} \PYG{o}{=} \PYG{n}{np}\PYG{o}{.}\PYG{n}{ones}\PYG{p}{(}\PYG{n}{nx}\PYG{p}{)}      \PYG{c+c1}{\PYGZsh{}a numpy array with nx elements all equal to 1.}
\PYG{n}{u}\PYG{p}{[}\PYG{n+nb}{int}\PYG{p}{(}\PYG{l+m+mf}{.5} \PYG{o}{/} \PYG{n}{dx}\PYG{p}{)}\PYG{p}{:}\PYG{n+nb}{int}\PYG{p}{(}\PYG{l+m+mi}{1} \PYG{o}{/} \PYG{n}{dx} \PYG{o}{+} \PYG{l+m+mi}{1}\PYG{p}{)}\PYG{p}{]} \PYG{o}{=} \PYG{l+m+mi}{2}  \PYG{c+c1}{\PYGZsh{}setting u = 2 between 0.5 and 1 as per our I.C.s}

\PYG{c+c1}{\PYGZsh{} Calculation}
\PYG{n}{un} \PYG{o}{=} \PYG{n}{np}\PYG{o}{.}\PYG{n}{ones}\PYG{p}{(}\PYG{n}{nx}\PYG{p}{)} \PYG{c+c1}{\PYGZsh{}our placeholder array, un, to advance the solution in time}
\PYG{k}{for} \PYG{n}{n} \PYG{o+ow}{in} \PYG{n+nb}{range}\PYG{p}{(}\PYG{n}{nt}\PYG{p}{)}\PYG{p}{:}  \PYG{c+c1}{\PYGZsh{}iterate through time}
    \PYG{n}{un} \PYG{o}{=} \PYG{n}{u}\PYG{o}{.}\PYG{n}{copy}\PYG{p}{(}\PYG{p}{)} \PYG{c+c1}{\PYGZsh{}\PYGZsh{}copy the existing values of u into un}
    \PYG{k}{for} \PYG{n}{i} \PYG{o+ow}{in} \PYG{n+nb}{range}\PYG{p}{(}\PYG{l+m+mi}{1}\PYG{p}{,} \PYG{n}{nx} \PYG{o}{\PYGZhy{}} \PYG{l+m+mi}{1}\PYG{p}{)}\PYG{p}{:}
        \PYG{n}{u}\PYG{p}{[}\PYG{n}{i}\PYG{p}{]} \PYG{o}{=} \PYG{n}{un}\PYG{p}{[}\PYG{n}{i}\PYG{p}{]} \PYG{o}{+} \PYG{n}{Neumann} \PYG{o}{*} \PYG{p}{(}\PYG{n}{un}\PYG{p}{[}\PYG{n}{i}\PYG{o}{+}\PYG{l+m+mi}{1}\PYG{p}{]} \PYG{o}{\PYGZhy{}} \PYG{l+m+mi}{2} \PYG{o}{*} \PYG{n}{un}\PYG{p}{[}\PYG{n}{i}\PYG{p}{]} \PYG{o}{+} \PYG{n}{un}\PYG{p}{[}\PYG{n}{i}\PYG{o}{\PYGZhy{}}\PYG{l+m+mi}{1}\PYG{p}{]}\PYG{p}{)}
        
    \PYG{n}{plt}\PYG{o}{.}\PYG{n}{plot}\PYG{p}{(}\PYG{n}{np}\PYG{o}{.}\PYG{n}{linspace}\PYG{p}{(}\PYG{l+m+mi}{0}\PYG{p}{,} \PYG{l+m+mi}{2}\PYG{p}{,} \PYG{n}{nx}\PYG{p}{)}\PYG{p}{,} \PYG{n}{u}\PYG{p}{)}\PYG{p}{;}
    \PYG{n}{plt}\PYG{o}{.}\PYG{n}{xlabel}\PYG{p}{(}\PYG{l+s+s2}{\PYGZdq{}}\PYG{l+s+s2}{space (m)}\PYG{l+s+s2}{\PYGZdq{}}\PYG{p}{)}\PYG{p}{;} \PYG{n}{plt}\PYG{o}{.}\PYG{n}{ylabel}\PYG{p}{(}\PYG{l+s+s2}{\PYGZdq{}}\PYG{l+s+s2}{Concentration (mg/L)}\PYG{l+s+s2}{\PYGZdq{}}\PYG{p}{)}\PYG{p}{;}
\end{sphinxVerbatim}

\end{sphinxuseclass}\end{sphinxVerbatimInput}
\begin{sphinxVerbatimOutput}

\begin{sphinxuseclass}{cell_output}
\begin{sphinxVerbatim}[commandchars=\\\{\}]
Simulating diffusion 
You can change input values to see the effect
\end{sphinxVerbatim}

\noindent\sphinxincludegraphics{{C:/Users/vibhu/GWtextbook/_build/jupyter_execute/21_conservative_transport_15_1}.png}

\end{sphinxuseclass}\end{sphinxVerbatimOutput}

\end{sphinxuseclass}
\begin{sphinxuseclass}{cell}\begin{sphinxVerbatimInput}

\begin{sphinxuseclass}{cell_input}
\begin{sphinxVerbatim}[commandchars=\\\{\}]
\PYG{n+nb}{print}\PYG{p}{(}\PYG{l+s+s2}{\PYGZdq{}}\PYG{l+s+s2}{A quick example: You can change the provided values}\PYG{l+s+se}{\PYGZbs{}n}\PYG{l+s+s2}{\PYGZdq{}}\PYG{p}{)}

\PYG{n+nb}{print}\PYG{p}{(}\PYG{l+s+s2}{\PYGZdq{}}\PYG{l+s+s2}{Let us find diffusive mass rate exiting a column.}\PYG{l+s+se}{\PYGZbs{}n}\PYG{l+s+se}{\PYGZbs{}n}\PYG{l+s+s2}{Provided are:}\PYG{l+s+s2}{\PYGZdq{}}\PYG{p}{)}

\PYG{n}{L\PYGZus{}3}  \PYG{o}{=} \PYG{l+m+mi}{50} \PYG{c+c1}{\PYGZsh{} cm, length of the columns}
\PYG{n}{R\PYGZus{}3}  \PYG{o}{=} \PYG{l+m+mf}{0.25} \PYG{c+c1}{\PYGZsh{} cm, radius of the column}
\PYG{n}{Ci\PYGZus{}3} \PYG{o}{=} \PYG{l+m+mi}{10} \PYG{c+c1}{\PYGZsh{} mg/L, inlet concentration}
\PYG{n}{Co\PYGZus{}3} \PYG{o}{=} \PYG{l+m+mi}{2} \PYG{c+c1}{\PYGZsh{} mg/L, outlet concentration}
\PYG{n}{D\PYGZus{}3}  \PYG{o}{=} \PYG{l+m+mi}{10}\PYG{o}{*}\PYG{o}{*}\PYG{o}{\PYGZhy{}}\PYG{l+m+mi}{5} \PYG{c+c1}{\PYGZsh{} cm\PYGZca{}2/s, dispersivity}

\PYG{c+c1}{\PYGZsh{} intermediate calculation}
\PYG{n}{A\PYGZus{}3} \PYG{o}{=} \PYG{n}{np}\PYG{o}{.}\PYG{n}{pi}\PYG{o}{*}\PYG{n}{R\PYGZus{}3}\PYG{o}{*}\PYG{o}{*}\PYG{l+m+mi}{2} \PYG{c+c1}{\PYGZsh{} Column surface area}
\PYG{n}{Cg\PYGZus{}3} \PYG{o}{=} \PYG{p}{(}\PYG{n}{Ci\PYGZus{}3}\PYG{o}{\PYGZhy{}}\PYG{n}{Co\PYGZus{}3}\PYG{p}{)}\PYG{o}{/}\PYG{n}{L\PYGZus{}3} \PYG{c+c1}{\PYGZsh{} mg/L\PYGZhy{}cm, concentration gradient}

\PYG{c+c1}{\PYGZsh{}solution}
\PYG{n}{Jm\PYGZus{}diff} \PYG{o}{=} \PYG{n}{A\PYGZus{}3}\PYG{o}{*}\PYG{n}{D\PYGZus{}3}\PYG{o}{*}\PYG{n}{Cg\PYGZus{}3}


\PYG{n+nb}{print}\PYG{p}{(}\PYG{l+s+s2}{\PYGZdq{}}\PYG{l+s+s2}{Radius = }\PYG{l+s+si}{\PYGZob{}\PYGZcb{}}\PYG{l+s+s2}{ cm}\PYG{l+s+se}{\PYGZbs{}n}\PYG{l+s+s2}{Length of the column = }\PYG{l+s+si}{\PYGZob{}\PYGZcb{}}\PYG{l+s+s2}{ cm/s}\PYG{l+s+se}{\PYGZbs{}n}\PYG{l+s+s2}{Inlet concentration = }\PYG{l+s+si}{\PYGZob{}\PYGZcb{}}\PYG{l+s+s2}{ mg/L}\PYG{l+s+se}{\PYGZbs{}n}\PYG{l+s+s2}{Outlet concentration = }\PYG{l+s+si}{\PYGZob{}\PYGZcb{}}\PYG{l+s+s2}{ mg/L }\PYG{l+s+se}{\PYGZbs{}}
\PYG{l+s+se}{\PYGZbs{}n}\PYG{l+s+s2}{Diffusion Coefficient = }\PYG{l+s+si}{\PYGZob{}\PYGZcb{}}\PYG{l+s+s2}{\PYGZdq{}}\PYG{o}{.}\PYG{n}{format}\PYG{p}{(}\PYG{n}{R\PYGZus{}3}\PYG{p}{,} \PYG{n}{L\PYGZus{}3}\PYG{p}{,} \PYG{n}{Ci\PYGZus{}3}\PYG{p}{,} \PYG{n}{Co\PYGZus{}3}\PYG{p}{,} \PYG{n}{D\PYGZus{}3}\PYG{p}{)}\PYG{p}{,} \PYG{l+s+s2}{\PYGZdq{}}\PYG{l+s+se}{\PYGZbs{}n}\PYG{l+s+s2}{\PYGZdq{}}\PYG{p}{)}

\PYG{n+nb}{print}\PYG{p}{(}\PYG{l+s+s2}{\PYGZdq{}}\PYG{l+s+s2}{The resulting dispersive mass flux is }\PYG{l+s+si}{\PYGZob{}:02.7f\PYGZcb{}}\PYG{l+s+s2}{ mg/s}\PYG{l+s+s2}{\PYGZdq{}}\PYG{o}{.}\PYG{n}{format}\PYG{p}{(}\PYG{n}{Jm\PYGZus{}diff}\PYG{p}{)}\PYG{p}{)}
\end{sphinxVerbatim}

\end{sphinxuseclass}\end{sphinxVerbatimInput}
\begin{sphinxVerbatimOutput}

\begin{sphinxuseclass}{cell_output}
\begin{sphinxVerbatim}[commandchars=\\\{\}]
A quick example: You can change the provided values

Let us find diffusive mass rate exiting a column.

Provided are:
Radius = 0.25 cm
Length of the column = 50 cm/s
Inlet concentration = 10 mg/L
Outlet concentration = 2 mg/L 
Diffusion Coefficient = 1e\PYGZhy{}05 

The resulting dispersive mass flux is 0.0000003 mg/s
\end{sphinxVerbatim}

\end{sphinxuseclass}\end{sphinxVerbatimOutput}

\end{sphinxuseclass}

\subsection{Hydrodynamic Dispersion}
\label{\detokenize{content/transport/L9/21_conservative_transport:id1}}
\sphinxAtStartPar
In groundwater transport study mechanical dispersion and diffusion coefficient are summed up and a common term \sphinxstylestrong{hydrodynamic dispersion} \(D_{hyd}\) {[}L\(^2\)T\(^{-1}\){]}. However, the diffusion coefficient \(D\) used in diffusion mass flow equation is valid for liquid water. Groundwater exist with solid porous medium. Thus, pore diffusion coefficient \(D_p\) {[}L\(^2\)T\(^{-1}\){]} is considered. In general:
\begin{equation*}
\begin{split}
D_p < D
\end{split}
\end{equation*}
\sphinxAtStartPar
and it is frequently assumed that
\begin{equation*}
\begin{split}
D_p = n_e \cdot D
\end{split}
\end{equation*}
\sphinxAtStartPar
where \(n_e\) is effectively porosity. Several other emperical formulae are found in literature relating \(D\) and \(D_p\). With \(D_p\), defined the \sphinxstylestrong{hydrodynamic dispersion} is defined as:
\begin{equation*}
\begin{split}
D_{hyd} = D_{mech} + D_p = \alpha \cdot v + D_p = \alpha \cdot v + n_e \cdot D
\end{split}
\end{equation*}

\section{Joint Action of Transport Processes}
\label{\detokenize{content/transport/L9/21_conservative_transport:joint-action-of-transport-processes}}
\sphinxAtStartPar
The spread of conservative solute in an unconsolidated aquifer can be considered as a combined effect or superposition of advection, mechanical dispersion and pore diffusion processes. Thus, for the complete description of the transport process is then:
\begin{equation*}
\begin{split}
J = J_{adv} + J_{disp, m} + J_{diff} = J_adv + J_{disp, h}
\end{split}
\end{equation*}
\sphinxAtStartPar
i.e,
\begin{equation*}
\begin{split}
J = n_e \cdot A \cdot A \cdot v \cdot C - n_e \cdot A\cdot\alpha \cdot v 
\frac{\Delta C}{L} - n_e \cdot A\cdot D_p \cdot \frac{\Delta C}{L}
\end{split}
\end{equation*}
\sphinxAtStartPar
which simplifies to
\begin{equation*}
\begin{split}
J = n_e \cdot A\cdot v \cdot C - n_e \cdot A \cdot D_hyd \cdot\frac{\Delta C}{L}
\end{split}
\end{equation*}
\sphinxAtStartPar
Thus the spread of solutes due to transport processes can be quantified by combining a mass budget and the corresponding laws of motion. The combination results in a \sphinxstylestrong{Transport Equation}, which is more commonly known as \sphinxstyleemphasis{advection\sphinxhyphen{}dispersion equation} or also, when energy transport is generally the case,  as \sphinxstyleemphasis{convection\sphinxhyphen{}dispersion equation}. The derivation of the transport equation can be found in standard hydrogeology texts. For the sake of completeness, the 1D (conservative) transport equation is:
\begin{equation*}
\begin{split}
\frac{\partial C}{\partial x} = - v\frac{\partial C}{\partial x} + D_{hyd}\frac{\partial^2 C}{\partial x^2}
\end{split}
\end{equation*}
\sphinxAtStartPar
The solution of transport equation is thus Concentration varying in space and time, i.e., \(C(x,t)\).


\section{Analysis of conservative transport problem}
\label{\detokenize{content/transport/L9/21_conservative_transport:analysis-of-conservative-transport-problem}}
\sphinxAtStartPar
Several analytical solutions (discussed in modeling chapter) are available that very often let us understand which type \sphinxhyphen{} e.g. advection dominant or dispersion dominant, of the problem exist. Very useful solution is provided by Ogata and Banks (1961). The solution for a particular initial and boundary conditions (see reference text) is given as:
\begin{equation*}
\begin{split}
C(x, t) = \frac{1}{2}C_0 \Bigg[\text{erfc}\Bigg(\frac{x-v\cdot t}{2\sqrt{D_x \cdot t}}\Bigg)+ \exp\Bigg(\frac{v\cdot x}{D_x}\Bigg)\cdot \text{erfc}\Bigg(\frac{x+v\cdot t}{2\sqrt{D_x \cdot t}}\Bigg)\Bigg]
\end{split}
\end{equation*}
\sphinxAtStartPar
where, \(C_0\) is continuous input concentration. More often in groundwater studies, the term \(\text{erfc}\Big(\frac{x+v\cdot t}{2\sqrt{D_x \cdot t}}\Big)\to 0 \). This simplifies the Ogata and Banks (1961) solution to
\begin{equation*}
\begin{split}
C(x, t) = \frac{1}{2}C_0 \Bigg[\text{erfc}\Bigg(\frac{x-v\cdot t}{2\sqrt{D_x \cdot t}}\Bigg)\Bigg]
\end{split}
\end{equation*}
\sphinxAtStartPar
This solution can be used to evaluate the dominant front between advection and dispersion for example.

\begin{sphinxShadowBox}
\sphinxstylesidebartitle{Ogata and Banks (1961) scenario}


\begin{wrapfigure}{r}{0pt}
\centering
\noindent\sphinxincludegraphics[scale=0.4]{{T9_f6}.png}
\captionof{figure}{Joint action of advection and dispersion
\sphinxcode{\sphinxupquote{  }}}\label{\detokenize{content/transport/L9/21_conservative_transport:j-ad-dis}}\end{wrapfigure}
\end{sphinxShadowBox}

\begin{sphinxuseclass}{cell}\begin{sphinxVerbatimInput}

\begin{sphinxuseclass}{cell_input}
\begin{sphinxVerbatim}[commandchars=\\\{\}]
\PYG{c+c1}{\PYGZsh{} simulating Ogata and Banks (1961) for evaluation dominant front.}

\PYG{n}{x} \PYG{o}{=} \PYG{n}{np}\PYG{o}{.}\PYG{n}{linspace}\PYG{p}{(}\PYG{o}{\PYGZhy{}}\PYG{l+m+mi}{100}\PYG{p}{,} \PYG{l+m+mi}{100}\PYG{p}{,} \PYG{l+m+mi}{1000}\PYG{p}{)}
\PYG{n}{C0} \PYG{o}{=} \PYG{l+m+mi}{10}
\PYG{n}{Dx} \PYG{o}{=} \PYG{l+m+mi}{4}
\PYG{n}{v} \PYG{o}{=} \PYG{l+m+mi}{5}\PYG{c+c1}{\PYGZsh{} change v and t together  to observe the dominant front.}
\PYG{n}{t} \PYG{o}{=} \PYG{l+m+mi}{10} \PYG{c+c1}{\PYGZsh{} }


\PYG{n}{C} \PYG{o}{=} \PYG{l+m+mf}{0.5}\PYG{o}{*}\PYG{n}{C0}\PYG{o}{*}\PYG{n}{erfc}\PYG{p}{(}\PYG{p}{(}\PYG{n}{x}\PYG{o}{\PYGZhy{}}\PYG{n}{v}\PYG{o}{*}\PYG{n}{t}\PYG{p}{)}\PYG{o}{/}\PYG{p}{(}\PYG{l+m+mi}{2}\PYG{o}{*}\PYG{n}{np}\PYG{o}{.}\PYG{n}{sqrt}\PYG{p}{(}\PYG{n}{Dx}\PYG{o}{*}\PYG{n}{t}\PYG{p}{)}\PYG{p}{)}\PYG{p}{)} 
\PYG{n}{plt}\PYG{o}{.}\PYG{n}{plot}\PYG{p}{(}\PYG{n}{x}\PYG{p}{,} \PYG{n}{C}\PYG{o}{/}\PYG{n}{C0}\PYG{p}{)}

\PYG{n}{t2} \PYG{o}{=} \PYG{n}{t}\PYG{o}{*}\PYG{l+m+mi}{5}
\PYG{n}{x1} \PYG{o}{=} \PYG{n}{x} \PYG{o}{+} \PYG{n}{v}\PYG{o}{*}\PYG{n}{t2}
\PYG{n}{C1} \PYG{o}{=} \PYG{l+m+mf}{0.5}\PYG{o}{*}\PYG{n}{C0}\PYG{o}{*}\PYG{n}{erfc}\PYG{p}{(}\PYG{p}{(}\PYG{n}{x1}\PYG{o}{\PYGZhy{}}\PYG{n}{v}\PYG{o}{*}\PYG{n}{t2}\PYG{p}{)}\PYG{o}{/}\PYG{p}{(}\PYG{l+m+mi}{2}\PYG{o}{*}\PYG{n}{np}\PYG{o}{.}\PYG{n}{sqrt}\PYG{p}{(}\PYG{n}{Dx}\PYG{o}{*}\PYG{n}{t2}\PYG{p}{)}\PYG{p}{)}\PYG{p}{)} 
\PYG{n}{plt}\PYG{o}{.}\PYG{n}{plot}\PYG{p}{(}\PYG{n}{x1}\PYG{p}{,} \PYG{n}{C1}\PYG{o}{/}\PYG{n}{C0}\PYG{p}{)}
\PYG{n}{plt}\PYG{o}{.}\PYG{n}{ylim}\PYG{p}{(}\PYG{p}{(}\PYG{l+m+mi}{0}\PYG{p}{,} \PYG{l+m+mi}{1}\PYG{p}{)}\PYG{p}{)}
\PYG{n}{plt}\PYG{o}{.}\PYG{n}{ylabel}\PYG{p}{(}\PYG{l+s+sa}{r}\PYG{l+s+s2}{\PYGZdq{}}\PYG{l+s+s2}{Normalized concentration \PYGZdl{}C/C\PYGZus{}0\PYGZdl{} (\PYGZhy{})}\PYG{l+s+s2}{\PYGZdq{}} \PYG{p}{)}
\PYG{n}{plt}\PYG{o}{.}\PYG{n}{xlabel}\PYG{p}{(}\PYG{l+s+s2}{\PYGZdq{}}\PYG{l+s+s2}{Distance (m)}\PYG{l+s+s2}{\PYGZdq{}}\PYG{p}{)}\PYG{p}{;}
\PYG{n}{plt}\PYG{o}{.}\PYG{n}{legend}\PYG{p}{(}\PYG{p}{[}\PYG{l+s+s2}{\PYGZdq{}}\PYG{l+s+s2}{time 1}\PYG{l+s+s2}{\PYGZdq{}}\PYG{p}{,} \PYG{l+s+s2}{\PYGZdq{}}\PYG{l+s+s2}{time 2}\PYG{l+s+s2}{\PYGZdq{}}\PYG{p}{]}\PYG{p}{)}\PYG{p}{;}
\end{sphinxVerbatim}

\end{sphinxuseclass}\end{sphinxVerbatimInput}
\begin{sphinxVerbatimOutput}

\begin{sphinxuseclass}{cell_output}
\noindent\sphinxincludegraphics{{C:/Users/vibhu/GWtextbook/_build/jupyter_execute/21_conservative_transport_21_0}.png}

\end{sphinxuseclass}\end{sphinxVerbatimOutput}

\end{sphinxuseclass}
\begin{sphinxuseclass}{cell}\begin{sphinxVerbatimInput}

\begin{sphinxuseclass}{cell_input}
\begin{sphinxVerbatim}[commandchars=\\\{\}]
\PYG{c+c1}{\PYGZsh{}from IPython.core.interactiveshell import InteractiveShell}
\PYG{c+c1}{\PYGZsh{}InteractiveShell.ast\PYGZus{}node\PYGZus{}interactivity = \PYGZdq{}all\PYGZdq{} }

\PYG{n}{Latex}\PYG{p}{(}\PYG{l+s+s2}{\PYGZdq{}}\PYG{l+s+s2}{A quick example to present the relative importance of different mass flow in the transport of solute }\PYG{l+s+se}{\PYGZbs{}n}\PYG{l+s+s2}{\PYGZdq{}}\PYG{p}{)}

\PYG{n}{A}  \PYG{o}{=}  \PYG{l+m+mi}{1} \PYG{c+c1}{\PYGZsh{} m², =ne.A,  cross\PYGZhy{}sectional area, with ne= eff. porosity}
\PYG{n}{v}  \PYG{o}{=}  \PYG{l+m+mi}{1} \PYG{c+c1}{\PYGZsh{} m/d, linear velocity}
\PYG{n}{C}  \PYG{o}{=}  \PYG{l+m+mi}{1} \PYG{c+c1}{\PYGZsh{} mg/L = 1 g/m\PYGZca{}3, concentration}
\PYG{n}{D}  \PYG{o}{=}  \PYG{l+m+mi}{1} \PYG{c+c1}{\PYGZsh{} m, transport disntance}
\PYG{n}{gr} \PYG{o}{=}  \PYG{l+m+mi}{1} \PYG{c+c1}{\PYGZsh{} g/m⁴, Delta C/L, conc. gradient}
\PYG{n}{Al} \PYG{o}{=}  \PYG{l+m+mf}{0.1} \PYG{c+c1}{\PYGZsh{} m, dispersivity}
\PYG{n}{Dp} \PYG{o}{=}  \PYG{l+m+mi}{10}\PYG{o}{*}\PYG{o}{*}\PYG{o}{\PYGZhy{}}\PYG{l+m+mi}{5} \PYG{c+c1}{\PYGZsh{} m²/d, pore diff. coefficient}

\PYG{c+c1}{\PYGZsh{} computation}
\PYG{n}{Jadv} \PYG{o}{=} \PYG{n}{A}\PYG{o}{*}\PYG{n}{v}\PYG{o}{*}\PYG{n}{C}
\PYG{n}{Jdisp} \PYG{o}{=} \PYG{n}{A}\PYG{o}{*}\PYG{n}{Al}\PYG{o}{*}\PYG{n}{v}\PYG{o}{*}\PYG{n}{gr}
\PYG{n}{Jdiff} \PYG{o}{=} \PYG{n}{A}\PYG{o}{*}\PYG{n}{Dp}\PYG{o}{*}\PYG{n}{gr}


\PYG{k+kn}{from} \PYG{n+nn}{IPython}\PYG{n+nn}{.}\PYG{n+nn}{display} \PYG{k+kn}{import} \PYG{n}{Latex}
\PYG{n}{Latex}\PYG{p}{(}\PYG{l+s+sa}{r}\PYG{l+s+s2}{\PYGZdq{}\PYGZdq{}\PYGZdq{}}\PYG{l+s+s2}{\PYGZbs{}}\PYG{l+s+s2}{begin}\PYG{l+s+si}{\PYGZob{}eqnarray\PYGZcb{}}
\PYG{l+s+s2}{J\PYGZus{}}\PYG{l+s+si}{\PYGZob{}adv\PYGZcb{}}\PYG{l+s+s2}{ \PYGZam{} = n\PYGZus{}e}\PYG{l+s+s2}{\PYGZbs{}}\PYG{l+s+s2}{cdot A }\PYG{l+s+s2}{\PYGZbs{}}\PYG{l+s+s2}{cdot v }\PYG{l+s+s2}{\PYGZbs{}}\PYG{l+s+s2}{cdot C }\PYG{l+s+s2}{\PYGZbs{}}\PYG{l+s+s2}{\PYGZbs{}}
\PYG{l+s+s2}{J\PYGZus{}}\PYG{l+s+si}{\PYGZob{}dis\PYGZcb{}}\PYG{l+s+s2}{ \PYGZam{} = n\PYGZus{}e}\PYG{l+s+s2}{\PYGZbs{}}\PYG{l+s+s2}{cdot A }\PYG{l+s+s2}{\PYGZbs{}}\PYG{l+s+s2}{cdot }\PYG{l+s+s2}{\PYGZbs{}}\PYG{l+s+s2}{alpha  }\PYG{l+s+s2}{\PYGZbs{}}\PYG{l+s+s2}{cdot  v }\PYG{l+s+s2}{\PYGZbs{}}\PYG{l+s+s2}{cdot }\PYG{l+s+s2}{\PYGZbs{}}\PYG{l+s+s2}{frac}\PYG{l+s+s2}{\PYGZob{}}\PYG{l+s+s2}{\PYGZbs{}}\PYG{l+s+s2}{Delta C\PYGZcb{}}\PYG{l+s+si}{\PYGZob{}L\PYGZcb{}}\PYG{l+s+s2}{  }\PYG{l+s+s2}{\PYGZbs{}}\PYG{l+s+s2}{\PYGZbs{}}
\PYG{l+s+s2}{J\PYGZus{}}\PYG{l+s+si}{\PYGZob{}diff\PYGZcb{}}\PYG{l+s+s2}{ \PYGZam{} = n\PYGZus{}e}\PYG{l+s+s2}{\PYGZbs{}}\PYG{l+s+s2}{cdot A }\PYG{l+s+s2}{\PYGZbs{}}\PYG{l+s+s2}{cdot D\PYGZus{}p  }\PYG{l+s+s2}{\PYGZbs{}}\PYG{l+s+s2}{cdot }\PYG{l+s+s2}{\PYGZbs{}}\PYG{l+s+s2}{frac}\PYG{l+s+s2}{\PYGZob{}}\PYG{l+s+s2}{\PYGZbs{}}\PYG{l+s+s2}{Delta C\PYGZcb{}}\PYG{l+s+si}{\PYGZob{}L\PYGZcb{}}\PYG{l+s+s2}{ }
\PYG{l+s+s2}{\PYGZbs{}}\PYG{l+s+s2}{end}\PYG{l+s+si}{\PYGZob{}eqnarray\PYGZcb{}}\PYG{l+s+s2}{\PYGZdq{}\PYGZdq{}\PYGZdq{}}\PYG{p}{)}


\PYG{c+c1}{\PYGZsh{}output}
\PYG{n+nb}{print}\PYG{p}{(}\PYG{l+s+s2}{\PYGZdq{}}\PYG{l+s+se}{\PYGZbs{}n}\PYG{l+s+s2}{ The contribution of advective flow is: }\PYG{l+s+si}{\PYGZob{}0:0.2f\PYGZcb{}}\PYG{l+s+s2}{\PYGZdq{}}\PYG{o}{.}\PYG{n}{format}\PYG{p}{(}\PYG{n}{Jadv}\PYG{p}{)}\PYG{p}{,} \PYG{l+s+s2}{\PYGZdq{}}\PYG{l+s+s2}{g/d }\PYG{l+s+se}{\PYGZbs{}n}\PYG{l+s+s2}{\PYGZdq{}}\PYG{p}{)}
\PYG{n+nb}{print}\PYG{p}{(}\PYG{l+s+s2}{\PYGZdq{}}\PYG{l+s+s2}{The contribution of dispersive flow is: }\PYG{l+s+si}{\PYGZob{}0:0.2f\PYGZcb{}}\PYG{l+s+s2}{\PYGZdq{}}\PYG{o}{.}\PYG{n}{format}\PYG{p}{(}\PYG{n}{Jdisp}\PYG{p}{)}\PYG{p}{,} \PYG{l+s+s2}{\PYGZdq{}}\PYG{l+s+s2}{g/d }\PYG{l+s+se}{\PYGZbs{}n}\PYG{l+s+s2}{\PYGZdq{}}\PYG{p}{)}
\PYG{n+nb}{print}\PYG{p}{(}\PYG{l+s+s2}{\PYGZdq{}}\PYG{l+s+s2}{The contribution of diffusive flow is: }\PYG{l+s+si}{\PYGZob{}0:0.1e\PYGZcb{}}\PYG{l+s+s2}{\PYGZdq{}}\PYG{o}{.}\PYG{n}{format}\PYG{p}{(}\PYG{n}{Jdiff}\PYG{p}{)}\PYG{p}{,} \PYG{l+s+s2}{\PYGZdq{}}\PYG{l+s+s2}{g/d }\PYG{l+s+se}{\PYGZbs{}n}\PYG{l+s+s2}{\PYGZdq{}}\PYG{p}{)}
\end{sphinxVerbatim}

\end{sphinxuseclass}\end{sphinxVerbatimInput}
\begin{sphinxVerbatimOutput}

\begin{sphinxuseclass}{cell_output}
\begin{sphinxVerbatim}[commandchars=\\\{\}]
 The contribution of advective flow is: 1.00 g/d 

The contribution of dispersive flow is: 0.10 g/d 

The contribution of diffusive flow is: 1.0e\PYGZhy{}05 g/d 
\end{sphinxVerbatim}

\end{sphinxuseclass}\end{sphinxVerbatimOutput}

\end{sphinxuseclass}
\sphinxAtStartPar
In general, \(J_{adv} > J_{dis} \) and  \(J_{disp} >> J_{diff} \)


\section{Concnetration Profiles and Breakthrough Curves}
\label{\detokenize{content/transport/L9/21_conservative_transport:concnetration-profiles-and-breakthrough-curves}}
\sphinxAtStartPar
\sphinxstyleemphasis{Concentration profiles} and \sphinxstyleemphasis{Breakthrough Curves} are very generally used to visualize and analyze the solute transport results.

\sphinxAtStartPar
\sphinxstylestrong{Conentration profiles} represents  the solute concentration as a function of a space coordinate at fixed time level. Normalized concentration \(C/C_0\), with \(C_0\) is geneally used in \(y\)\sphinxhyphen{}axis against distance \(x\) along \(x-\)axis.

\sphinxAtStartPar
The \sphinxstyleemphasis{concentration profile} can be visualize as in the figure below:

\begin{figure}[htbp]
\centering
\capstart

\noindent\sphinxincludegraphics[scale=0.5]{{T9_f7}.png}
\caption{Concentration profile}\label{\detokenize{content/transport/L9/21_conservative_transport:c-profile}}\end{figure}

\begin{sphinxuseclass}{cell}\begin{sphinxVerbatimInput}

\begin{sphinxuseclass}{cell_input}
\begin{sphinxVerbatim}[commandchars=\\\{\}]
\PYG{n+nb}{print}\PYG{p}{(}\PYG{l+s+s2}{\PYGZdq{}}\PYG{l+s+s2}{An example code for obtaining concentration profile }\PYG{l+s+se}{\PYGZbs{}n}\PYG{l+s+s2}{\PYGZdq{}}\PYG{p}{)}

\PYG{c+c1}{\PYGZsh{} Input        }
\PYG{n}{Cb} \PYG{o}{=} \PYG{l+m+mi}{100} \PYG{c+c1}{\PYGZsh{} mg/L  \PYGZsh{} }
\PYG{n}{Ca} \PYG{o}{=} \PYG{l+m+mi}{2} \PYG{c+c1}{\PYGZsh{} mg/L  }
\PYG{n}{D} \PYG{o}{=} \PYG{l+m+mi}{1} \PYG{c+c1}{\PYGZsh{} m²/d  }
\PYG{n}{x} \PYG{o}{=} \PYG{n}{np}\PYG{o}{.}\PYG{n}{linspace}\PYG{p}{(}\PYG{o}{\PYGZhy{}}\PYG{l+m+mi}{50}\PYG{p}{,} \PYG{l+m+mi}{50}\PYG{p}{,} \PYG{l+m+mi}{300}\PYG{p}{)}
\PYG{n}{t} \PYG{o}{=} \PYG{p}{[}\PYG{l+m+mi}{0}\PYG{p}{,} \PYG{l+m+mi}{10}\PYG{p}{,} \PYG{l+m+mi}{100}\PYG{p}{,} \PYG{l+m+mi}{1000}\PYG{p}{]} \PYG{c+c1}{\PYGZsh{} change time as you like \PYGZhy{} limit to four}

\PYG{c+c1}{\PYGZsh{} dont change anything from here}

\PYG{k}{def} \PYG{n+nf}{f1}\PYG{p}{(}\PYG{n}{t}\PYG{p}{)}\PYG{p}{:} 
    \PYG{k}{for} \PYG{n}{i} \PYG{o+ow}{in} \PYG{n}{t}\PYG{p}{:}
        \PYG{n}{A} \PYG{o}{=} \PYG{l+m+mi}{2}\PYG{o}{*}\PYG{n}{np}\PYG{o}{.}\PYG{n}{sqrt}\PYG{p}{(}\PYG{n}{D}\PYG{o}{*}\PYG{n}{i}\PYG{p}{)}
        \PYG{n}{C} \PYG{o}{=} \PYG{n}{Ca} \PYG{o}{+} \PYG{p}{(}\PYG{n}{Cb}\PYG{o}{\PYGZhy{}}\PYG{n}{Ca}\PYG{p}{)}\PYG{o}{/}\PYG{l+m+mi}{2} \PYG{o}{*} \PYG{n}{erfc}\PYG{p}{(}\PYG{n}{x}\PYG{o}{/}\PYG{n}{A}\PYG{p}{)}
        
        \PYG{n}{plt}\PYG{o}{.}\PYG{n}{plot}\PYG{p}{(}\PYG{n}{x}\PYG{p}{,} \PYG{n}{C}\PYG{p}{)}
        \PYG{n}{label} \PYG{o}{=} \PYG{p}{[}\PYG{l+s+s2}{\PYGZdq{}}\PYG{l+s+s2}{0 day}\PYG{l+s+s2}{\PYGZdq{}}\PYG{p}{,} \PYG{l+s+s2}{\PYGZdq{}}\PYG{l+s+s2}{First time}\PYG{l+s+s2}{\PYGZdq{}}\PYG{p}{,} \PYG{l+s+s2}{\PYGZdq{}}\PYG{l+s+s2}{Second time}\PYG{l+s+s2}{\PYGZdq{}}\PYG{p}{,} \PYG{l+s+s2}{\PYGZdq{}}\PYG{l+s+s2}{third time}\PYG{l+s+s2}{\PYGZdq{}}\PYG{p}{]}
        \PYG{n}{plt}\PYG{o}{.}\PYG{n}{legend}\PYG{p}{(}\PYG{n}{label}\PYG{p}{)}\PYG{p}{;} 
        \PYG{n}{plt}\PYG{o}{.}\PYG{n}{ylabel}\PYG{p}{(}\PYG{l+s+s2}{\PYGZdq{}}\PYG{l+s+s2}{Concentration (mg/L)}\PYG{l+s+s2}{\PYGZdq{}}\PYG{p}{)}\PYG{p}{;} \PYG{n}{plt}\PYG{o}{.}\PYG{n}{xlabel}\PYG{p}{(}\PYG{l+s+s2}{\PYGZdq{}}\PYG{l+s+s2}{Distance (m)}\PYG{l+s+s2}{\PYGZdq{}}\PYG{p}{)}

\PYG{n}{f1}\PYG{p}{(}\PYG{n}{t}\PYG{p}{)} \PYG{c+c1}{\PYGZsh{} evaluate}
\end{sphinxVerbatim}

\end{sphinxuseclass}\end{sphinxVerbatimInput}
\begin{sphinxVerbatimOutput}

\begin{sphinxuseclass}{cell_output}
\begin{sphinxVerbatim}[commandchars=\\\{\}]
An example code for obtaining concentration profile 
\end{sphinxVerbatim}

\noindent\sphinxincludegraphics{{C:/Users/vibhu/GWtextbook/_build/jupyter_execute/21_conservative_transport_25_1}.png}

\end{sphinxuseclass}\end{sphinxVerbatimOutput}

\end{sphinxuseclass}
\sphinxAtStartPar
\sphinxstylestrong{Breakthrough Curves}, an alternative to \sphinxstyleemphasis{concentration profile} provides the solute concentration as a function of time at specified observation locations, i.e., along \(y-\)axis \(C\) or normalized \(C/C_0\) is used and along the \(x-\)axis time \(t\) information is provided. The graphics below schematically illustrates th \sphinxstyleemphasis{breakthrough curve} visualization

\begin{figure}[htbp]
\centering
\capstart

\noindent\sphinxincludegraphics[scale=0.5]{{T9_f7}.png}
\caption{The break\sphinxhyphen{}through curve}\label{\detokenize{content/transport/L9/21_conservative_transport:break}}\end{figure}

\begin{sphinxuseclass}{cell}\begin{sphinxVerbatimInput}

\begin{sphinxuseclass}{cell_input}
\begin{sphinxVerbatim}[commandchars=\\\{\}]
\PYG{n+nb}{print}\PYG{p}{(}\PYG{l+s+s2}{\PYGZdq{}}\PYG{l+s+s2}{An example code for obtaining Breakthrough curve }\PYG{l+s+se}{\PYGZbs{}n}\PYG{l+s+s2}{\PYGZdq{}}\PYG{p}{)}

\PYG{k}{def} \PYG{n+nf}{f1}\PYG{p}{(}\PYG{n}{x}\PYG{p}{)}\PYG{p}{:} \PYG{c+c1}{\PYGZsh{} dont change anything from here}
    \PYG{k}{for} \PYG{n}{i} \PYG{o+ow}{in} \PYG{n}{X}\PYG{p}{:}
        \PYG{n}{A} \PYG{o}{=} \PYG{l+m+mi}{2}\PYG{o}{*}\PYG{n}{np}\PYG{o}{.}\PYG{n}{sqrt}\PYG{p}{(}\PYG{n}{D}\PYG{o}{*}\PYG{n}{t}\PYG{p}{)}
        \PYG{n}{C} \PYG{o}{=} \PYG{n}{Ca} \PYG{o}{+} \PYG{p}{(}\PYG{n}{Cb}\PYG{o}{\PYGZhy{}}\PYG{n}{Ca}\PYG{p}{)}\PYG{o}{/}\PYG{l+m+mi}{2} \PYG{o}{*} \PYG{n}{erfc}\PYG{p}{(}\PYG{n}{i}\PYG{o}{/}\PYG{n}{A}\PYG{p}{)} \PYG{c+c1}{\PYGZsh{} Crank (1975) modified}
        
        \PYG{n}{plt}\PYG{o}{.}\PYG{n}{plot}\PYG{p}{(}\PYG{n}{t}\PYG{p}{,} \PYG{n}{C}\PYG{p}{)}
        \PYG{n}{label} \PYG{o}{=} \PYG{p}{[}\PYG{l+s+s2}{\PYGZdq{}}\PYG{l+s+s2}{distance 1}\PYG{l+s+s2}{\PYGZdq{}}\PYG{p}{,} \PYG{l+s+s2}{\PYGZdq{}}\PYG{l+s+s2}{distance 2}\PYG{l+s+s2}{\PYGZdq{}}\PYG{p}{,} \PYG{l+s+s2}{\PYGZdq{}}\PYG{l+s+s2}{distance 3}\PYG{l+s+s2}{\PYGZdq{}}\PYG{p}{,} \PYG{l+s+s2}{\PYGZdq{}}\PYG{l+s+s2}{distance 4}\PYG{l+s+s2}{\PYGZdq{}}\PYG{p}{]}
        \PYG{n}{plt}\PYG{o}{.}\PYG{n}{legend}\PYG{p}{(}\PYG{n}{label}\PYG{p}{)}\PYG{p}{;} 
        \PYG{n}{plt}\PYG{o}{.}\PYG{n}{ylabel}\PYG{p}{(}\PYG{l+s+s2}{\PYGZdq{}}\PYG{l+s+s2}{Concentration (mg/L)}\PYG{l+s+s2}{\PYGZdq{}}\PYG{p}{)}\PYG{p}{;} \PYG{n}{plt}\PYG{o}{.}\PYG{n}{xlabel}\PYG{p}{(}\PYG{l+s+s2}{\PYGZdq{}}\PYG{l+s+s2}{time (s)}\PYG{l+s+s2}{\PYGZdq{}}\PYG{p}{)}


\PYG{c+c1}{\PYGZsh{} Input        }
\PYG{n}{Cb} \PYG{o}{=} \PYG{l+m+mi}{100} \PYG{c+c1}{\PYGZsh{} mg/L  \PYGZsh{} }
\PYG{n}{Ca} \PYG{o}{=} \PYG{l+m+mi}{2}\PYG{c+c1}{\PYGZsh{} mg/L  }
\PYG{n}{D} \PYG{o}{=} \PYG{l+m+mi}{1} \PYG{c+c1}{\PYGZsh{} m²/d  }
\PYG{n}{t} \PYG{o}{=} \PYG{n}{np}\PYG{o}{.}\PYG{n}{linspace}\PYG{p}{(}\PYG{l+m+mi}{0}\PYG{p}{,} \PYG{l+m+mi}{500}\PYG{p}{,} \PYG{l+m+mi}{5000}\PYG{p}{)}
\PYG{n}{X} \PYG{o}{=} \PYG{p}{[}\PYG{l+m+mi}{0}\PYG{p}{,} \PYG{l+m+mi}{5}\PYG{p}{,} \PYG{l+m+mi}{20}\PYG{p}{,} \PYG{l+m+mi}{50}\PYG{p}{]} \PYG{c+c1}{\PYGZsh{} 0 = advective, change other numbers}

\PYG{n}{f1}\PYG{p}{(}\PYG{n}{X}\PYG{p}{)} \PYG{c+c1}{\PYGZsh{} evaluate}
\end{sphinxVerbatim}

\end{sphinxuseclass}\end{sphinxVerbatimInput}
\begin{sphinxVerbatimOutput}

\begin{sphinxuseclass}{cell_output}
\begin{sphinxVerbatim}[commandchars=\\\{\}]
An example code for obtaining Breakthrough curve 
\end{sphinxVerbatim}

\noindent\sphinxincludegraphics{{C:/Users/vibhu/GWtextbook/_build/jupyter_execute/21_conservative_transport_27_1}.png}

\end{sphinxuseclass}\end{sphinxVerbatimOutput}

\end{sphinxuseclass}

\section{Additional Tool}
\label{\detokenize{content/transport/L9/21_conservative_transport:additional-tool}}
\sphinxAtStartPar
The additional tool: \DUrole{xref,myst}{1D\sphinxhyphen{}Advection\sphinxhyphen{}Dispersion Simulation Tool} simulates all the concepts that are provided above. The tool simulates:
\begin{itemize}
\item {} 
\sphinxAtStartPar
1D solute transport in porous media (e.g., laboratory column)

\item {} 
\sphinxAtStartPar
uses unifrom cross\sphinxhyphen{}section

\item {} 
\sphinxAtStartPar
steady\sphinxhyphen{}state water flow

\item {} 
\sphinxAtStartPar
input of tracer

\end{itemize}

\sphinxAtStartPar
The output are then:
\begin{itemize}
\item {} 
\sphinxAtStartPar
spreading of tracer due to advection and mechanical dispersion

\item {} 
\sphinxAtStartPar
computation and graphical representation of a breakthrough curve

\item {} 
\sphinxAtStartPar
comparison with measured data.

\end{itemize}


\section{Chapter Quiz}
\label{\detokenize{content/transport/L9/21_conservative_transport:chapter-quiz}}
\begin{sphinxuseclass}{cell}
\begin{sphinxuseclass}{tag_remove-input}
\begin{sphinxuseclass}{tag_hide-output}
\end{sphinxuseclass}
\end{sphinxuseclass}
\end{sphinxuseclass}
\sphinxstepscope

\begin{sphinxuseclass}{cell}
\begin{sphinxuseclass}{tag_hide-input}
\begin{sphinxuseclass}{tag_remove-input}
\end{sphinxuseclass}
\end{sphinxuseclass}
\end{sphinxuseclass}

\chapter{Reactive Mass Transport}
\label{\detokenize{content/transport/L10/22_reactive_transport:reactive-mass-transport}}\label{\detokenize{content/transport/L10/22_reactive_transport::doc}}
\sphinxAtStartPar
\sphinxstyleemphasis{(The contents presented in this section were re\sphinxhyphen{}developed principally by Dr. P. K. Yadav. The original contents are from Prof. Rudolf Liedl)}


\bigskip\hrule\bigskip



\section{Motivation}
\label{\detokenize{content/transport/L10/22_reactive_transport:motivation}}
\sphinxAtStartPar
The last lecture dealt with the  conservative transport processes and  quantified the mass flow and flux emanating from those processes. The effects of these processes were evaluated as an isolated processes and as joint transport process.

\sphinxAtStartPar
The last lecture dealt with the  conservative transport processes and  quantified the mass flow and flux emanating from those processes. The effects of these processes were evaluated as an isolated processes and as joint transport process.

\begin{sphinxadmonition}{note}{Important conclusions from last lecture}
\begin{quote}

\sphinxAtStartPar
\(J_{adv}>J_{dis}>>J_{diff}\) is observed in normal aquifers.
\end{quote}
\begin{quote}

\sphinxAtStartPar
\(J_{diff}\) may only be useful as an individual processes in special aquifers, e.g., clayey aquifers.
\end{quote}
\begin{quote}

\sphinxAtStartPar
In general aquifers, hydrodynamic dispersion \(J_{hyd} = J_{dis} + J_{diff}\) is used in the analysis of solute transport process.
\end{quote}
\end{sphinxadmonition}

\sphinxAtStartPar
Finally, the last chapter introduced \sphinxstyleemphasis{Concentration Profile} \((C-t)\) and \sphinxstyleemphasis{Breakthrough Curve} \((C-x)\) to visually evaluate solute transport in aquifers using \sphinxstyleemphasis{Concentration} \((C)\), a process output, as a function of \sphinxstyleemphasis{time} \((t)\) and \sphinxstyleemphasis{space} \((x)\).

\sphinxAtStartPar
This \sphinxstylestrong{Lecture} focuses on the \sphinxstyleemphasis{reactive transport processes,} which as already discussed involves the transport of solute with \sphinxstyleemphasis{reaction} processes. This course being an introductory groundwater course, \sphinxstyleemphasis{sorption} and \sphinxstyleemphasis{degradation} are the only two reaction types introduced and they are combined with the conservative transport processes\sphinxhyphen{} \sphinxstyleemphasis{advection} and \sphinxstyleemphasis{dispersion}. Eventully, the section evaluates the joint action of conservative transport and reactive processes limiting to 1\sphinxhyphen{}D scenario.

\sphinxAtStartPar
The lecture will, however, first deal with 3\sphinxhyphen{}D effects of dispersive process, which is more important to quantify the reactive processes.


\section{Dispersive Mass Flow in 3\sphinxhyphen{}D}
\label{\detokenize{content/transport/L10/22_reactive_transport:dispersive-mass-flow-in-3-d}}
\sphinxAtStartPar
In the last lecture we saw that concentration gradient \(\frac{\Delta C}{\Delta L}\) and flow velocity \(v\) drives dispersive and diffusive solute transport processes.

\sphinxAtStartPar
However, in the natural aquifers \(\frac{\Delta C}{\Delta L}\) is normally varying with space \((x,y,z)\) and time \((t)\). Therefore, a differential operator \(\big(\frac{\mathrm{d}}{\mathrm{d} x}\big)\) is more suitable representation of gradient than the difference operator \(\frac{\Delta C}{\Delta x}\). The differential operator also generalizes the gradient case.

\sphinxAtStartPar
Considering the differential operator, the diffusive mass flow and diffusive mass flux (= mass flow per unit area) in 1D is then expressed as:
\begin{equation*}
\begin{split}
J_{diff} = - n_e \cdot A \cdot D_p \cdot \frac{\mathrm{d} C}{\mathrm{d}x} 
\end{split}
\end{equation*}
\sphinxAtStartPar
and
\$\(
j_{diff} = - n_e \cdot D_p \cdot \frac{\mathrm{d} C}{\mathrm{d}x} 
\)\$

\sphinxAtStartPar
Likewise, the dispersive mass flow and dispersive mass flux (1\sphinxhyphen{}D) is:
\begin{equation*}
\begin{split}
J_{disp, h} = - n_e \cdot A \cdot D_{disp} \cdot \frac{\mathrm{d} C}{\mathrm{d}x} 
\end{split}
\end{equation*}\begin{equation*}
\begin{split}
j_{disp, h} = - n_e \cdot D_{disp} \cdot \frac{\mathrm{d} C}{\mathrm{d}x} 
\end{split}
\end{equation*}
\sphinxAtStartPar
The examples of these relations are presented in the last lecture \DUrole{xref,myst}{Conservative Transport}


\section{3\sphinxhyphen{}D concentration Gradient}
\label{\detokenize{content/transport/L10/22_reactive_transport:d-concentration-gradient}}
\sphinxAtStartPar
The concentration gradient  \(\frac{\mathrm{d}C}{\mathrm{d}x}\)  for 1\sphinxhyphen{}D solute transport problems is uni\sphinxhyphen{}directional, i.e, direction is fixed, and thus only the magnitude of the gradient is the important factor. However in higher dimensions, 2\sphinxhyphen{}D or 3\sphinxhyphen{}D, solute transport problems, the \sphinxstyleemphasis{direction} of gradient along with it’s \sphinxstyleemphasis{magnitude} in that direction has to be specified. Thus, for higher dimension solute transport problems, the \sphinxstyleemphasis{concentration gradient} becomes \sphinxstylestrong{concentration vector}, i.e., a quantity providing both magnitude and direction.

\sphinxAtStartPar
Thus, the representation of concentration gradient in Cartesian coordinate in 2\sphinxhyphen{}D and 3\sphinxhyphen{}D is:
\begin{equation*}
\begin{split}
\mathrm{grad}\,C = \nabla C=  \begin{pmatrix}
\frac{\partial C}{\partial x}\\
\frac{\partial C}{\partial y}
\end{pmatrix}
\end{split}
\end{equation*}
\sphinxAtStartPar
and
\begin{equation*}
\begin{split}
\mathrm{grad} \,C = \nabla C=  \begin{pmatrix}
\frac{\partial C}{\partial x}\\
\frac{\partial C}{\partial y}\\
\frac{\partial C}{\partial z}
\end{pmatrix}
\end{split}
\end{equation*}
\sphinxAtStartPar
The \(\nabla\), the inverted Delta symbol, is called the \sphinxstylestrong{del} or \sphinxstylestrong{nabla} operator. The vector \sphinxstylestrong{grad \(C\)} in the above relations points in the direction of the \sphinxstyleemphasis{steepest increase} of \(C\). However, for the \sphinxstylestrong{Hydrogeologists}, the concentration gradients as well the grad \(C\) points to the \sphinxstyleemphasis{steepest decrease} of \(C\).


\section{Isotropic and Anisotropic  Dispersion}
\label{\detokenize{content/transport/L10/22_reactive_transport:isotropic-and-anisotropic-dispersion}}
\sphinxAtStartPar
Corresponding the expression for the concentration gradient at higher dimensions, the expression for mass flow and flux becomes:
\begin{equation*}
\begin{split}
J_{disp,\, h} = \begin{pmatrix}
J_{disp,\, hx}\\
J_{disp,\, hy}\\
J_{disp,\, hz}
\end{pmatrix}
\end{split}
\end{equation*}
\sphinxAtStartPar
and the 3\sphinxhyphen{}D mass flux is:
\begin{equation*}
\begin{split}
j_{disp,\, h} = \begin{pmatrix}
j_{disp,\, hx}\\
j_{disp,\, hy}\\
j_{disp,\, hz}
\end{pmatrix}
\end{split}
\end{equation*}
\sphinxAtStartPar
The subscript in \({{disp, h}}\) refers to \sphinxstyleemphasis{hydrodynamic dispersion} which is sum of \sphinxstyleemphasis{mechanical dispersion} and \sphinxstyleemphasis{diffusion}. Likewise, the subscript \({{disp,\, hx}}\), \({{disp,\, hy}}\) and \({{disp,\, hz}}\) refers to dispersion components along the Cartesian coordinates. The corresponding mass flow and mass flux in the higher dimension is then:
\begin{equation*}
\begin{split}
J_{disp,\, h} = - n_e \cdot A \cdot D_{hyd} \cdot \text{grad}C
\end{split}
\end{equation*}
\sphinxAtStartPar
and


\section{Isotropic and Anisotropic  Dispersion}
\label{\detokenize{content/transport/L10/22_reactive_transport:id1}}
\sphinxAtStartPar
Corresponding the expression for the concentration gradient at higher dimensions, the expression for mass flow and flux becomes:
\begin{equation*}
\begin{split}
J_{disp,\, h} = \begin{pmatrix}
J_{disp,\, hx}\\
J_{disp,\, hy}\\
J_{disp,\, hz}
\end{pmatrix}
\end{split}
\end{equation*}
\sphinxAtStartPar
and the 3\sphinxhyphen{}D mass flux is:
\begin{equation*}
\begin{split}
j_{disp,\, h} = \begin{pmatrix}
j_{disp,\, hx}\\
j_{disp,\, hy}\\
j_{disp,\, hz}
\end{pmatrix}
\end{split}
\end{equation*}
\sphinxAtStartPar
The subscript in \({{disp,\, h}}\) refers to \sphinxstyleemphasis{hydrodynamic dispersion} which is sum of \sphinxstyleemphasis{mechanical dispersion} and \sphinxstyleemphasis{diffusion}. Likewise, the subscript \({{disp,\, hx}}\), \({{disp,\, hy}}\) and \({{disp,\, hz}}\) refers to dispersion components along the Cartesian coordinates. The corresponding mass flow and mass flux in the higher dimension is then:
\begin{equation*}
\begin{split}
J_{disp,\, h} = - n_e \cdot A \cdot D_{hyd} \cdot \text{grad}C
\end{split}
\end{equation*}
\sphinxAtStartPar
and
\begin{equation*}
\begin{split}
j_{disp,\, h} = - n_e \cdot A \cdot D_{hyd} \cdot \text{grad}C
\end{split}
\end{equation*}
\sphinxAtStartPar
The \sphinxstylestrong{isotropic dispersion}, rather an \sphinxstyleemphasis{exceptional,} the \(D_{hyd}\) in this case is:
\begin{equation*}
\begin{split}
D_{hyd} = \alpha \cdot |v| + n_e \cdot D 
\end{split}
\end{equation*}
\sphinxAtStartPar
where, \(D\) is direction independent dispersion coefficient and \(\alpha\) \([L]\) is dispersivity, which in:

\sphinxAtStartPar
\sphinxstylestrong{heterogeneous aquifer}: \(\alpha = \alpha(x,y,z)\) and in

\sphinxAtStartPar
\sphinxstylestrong{homogeneous aquifer}: \(\alpha = \text{constant}\)

\sphinxAtStartPar
For more practical cases and in normal aquifers, the 2\sphinxhyphen{}D and 3\sphinxhyphen{}D the dispersion for solute transport is \sphinxstyleemphasis{direction dependent,} i.e. \sphinxstylestrong{anisotropic}. Hence the \(D_{hyd}\) is not an scalar quantity but a \sphinxstyleemphasis{matrix (tensor),} which relates the concentration gradient (vector) to the dispersive mass flow (vector). However, if the princopal axes of the dispersion tensor \(D_{hyd}\) is made to coincide with the axes of a Cartesian coordinate system \sphinxstyleemphasis{and} the groundwater flow is considered uniform along the \(x-\)axis, the dispersive mass flux can be obtained from
\begin{equation*}
\begin{split}
\begin{pmatrix} J_x \\ j_y \\ j_z \end{pmatrix} =
\begin{pmatrix} \alpha_L \cdot v_x + n_e \cdot D & 0 & 0 \\
0 & \alpha_{Th} \cdot v_x + n_e \cdot D & 0\\
0 & 0 & \alpha_{Tv} \cdot v_x + n_e \cdot D
\end{pmatrix}
\cdot
\begin{pmatrix} \frac{\partial C}{\partial x} \\ \frac{\partial C}{\partial y} \\ \frac{\partial C}{\partial z} \end{pmatrix} 
\end{split}
\end{equation*}
\sphinxAtStartPar
with \(\alpha_L\), \(\alpha_{Th}\) and \(\alpha_{Tv}\) are longitudinal dispersivity, horizontal transverse dispversity and vertical transverse dispersivity, respectively. The statistical analysis of dispersivity data shows that \(\alpha_{L}>\alpha_{Th}>\alpha_{Tv}\) and the values differ roughly by an order of magnitude. This, however, is just a rule of thumb.

\begin{sphinxadmonition}{note}{A quick example}

\sphinxAtStartPar
Discuss the role of 2D dispersitvity in  column when discharge is limited at 10 m\(^3\)/d of a chemical with concentration 1 mg/L. The flow velocity can be assumed to be 0.05 m/d.
\end{sphinxadmonition}

\begin{sphinxuseclass}{cell}\begin{sphinxVerbatimInput}

\begin{sphinxuseclass}{cell_input}
\begin{sphinxVerbatim}[commandchars=\\\{\}]
\PYG{c+c1}{\PYGZsh{} Analytical solution from Bear (1976) \PYGZhy{} Line source, 1st\PYGZhy{}type input and infinte plane}

\PYG{c+c1}{\PYGZsh{} Input (values can be changed)}
\PYG{n}{Co} \PYG{o}{=} \PYG{l+m+mi}{1} \PYG{c+c1}{\PYGZsh{} mg/L, input concentration }
\PYG{n}{Dx} \PYG{o}{=} \PYG{l+m+mi}{3} \PYG{c+c1}{\PYGZsh{} m, Dispersion in x direction }
\PYG{n}{Dy} \PYG{o}{=} \PYG{n}{Dx}\PYG{o}{/}\PYG{l+m+mi}{10} \PYG{c+c1}{\PYGZsh{} m}
\PYG{n}{v} \PYG{o}{=} \PYG{l+m+mf}{0.05} \PYG{c+c1}{\PYGZsh{} m/d}
\PYG{n}{Q} \PYG{o}{=} \PYG{l+m+mi}{10} \PYG{c+c1}{\PYGZsh{} m\PYGZca{}3/d}

\PYG{c+c1}{\PYGZsh{}\PYGZsh{} domain dimension and descritization (values can be changed)}
\PYG{n}{xmin} \PYG{o}{=} \PYG{o}{\PYGZhy{}}\PYG{l+m+mi}{100}\PYG{p}{;} \PYG{n}{xmax}\PYG{o}{=} \PYG{l+m+mi}{101}  
\PYG{n}{ymin} \PYG{o}{=} \PYG{l+m+mf}{0.1}\PYG{p}{;} \PYG{n}{ymax} \PYG{o}{=} \PYG{l+m+mi}{11}
\PYG{p}{[}\PYG{n}{x}\PYG{p}{,} \PYG{n}{y}\PYG{p}{]} \PYG{o}{=} \PYG{n}{np}\PYG{o}{.}\PYG{n}{meshgrid}\PYG{p}{(}\PYG{n}{np}\PYG{o}{.}\PYG{n}{linspace}\PYG{p}{(}\PYG{n}{xmin}\PYG{p}{,} \PYG{n}{xmax}\PYG{p}{,} \PYG{l+m+mi}{1000}\PYG{p}{)}\PYG{p}{,} \PYG{n}{np}\PYG{o}{.}\PYG{n}{linspace}\PYG{p}{(}\PYG{n}{ymin}\PYG{p}{,} \PYG{n}{ymax}\PYG{p}{,} \PYG{l+m+mi}{100}\PYG{p}{)}\PYG{p}{)} \PYG{c+c1}{\PYGZsh{} mesh}

\PYG{c+c1}{\PYGZsh{} Bear (1976) solution Implementation }
\PYG{c+c1}{\PYGZsh{}\PYGZdq{}k0: Modified Bessel function of second type and zero Order\PYGZdq{}}

\PYG{n}{term1} \PYG{o}{=} \PYG{p}{(}\PYG{n}{Co}\PYG{o}{*}\PYG{n}{Q}\PYG{p}{)}\PYG{o}{/}\PYG{p}{(}\PYG{l+m+mi}{2}\PYG{o}{*}\PYG{n}{np}\PYG{o}{.}\PYG{n}{pi}\PYG{o}{*} \PYG{n}{np}\PYG{o}{.}\PYG{n}{sqrt}\PYG{p}{(}\PYG{n}{Dx}\PYG{o}{*}\PYG{n}{Dy}\PYG{p}{)}\PYG{p}{)}
\PYG{n}{term2} \PYG{o}{=} \PYG{p}{(}\PYG{n}{x}\PYG{o}{*}\PYG{n}{v}\PYG{p}{)}\PYG{o}{/}\PYG{p}{(}\PYG{l+m+mi}{2}\PYG{o}{*}\PYG{n}{Dx}\PYG{p}{)}
\PYG{n}{args} \PYG{o}{=} \PYG{p}{(}\PYG{n}{v}\PYG{o}{*}\PYG{o}{*}\PYG{l+m+mi}{2}\PYG{o}{*}\PYG{n}{x}\PYG{o}{*}\PYG{o}{*}\PYG{l+m+mi}{2}\PYG{p}{)}\PYG{o}{/}\PYG{p}{(}\PYG{l+m+mi}{4}\PYG{o}{*}\PYG{n}{Dx}\PYG{o}{*}\PYG{o}{*}\PYG{l+m+mi}{2}\PYG{p}{)} \PYG{o}{+} \PYG{p}{(}\PYG{n}{v}\PYG{o}{*}\PYG{o}{*}\PYG{l+m+mi}{2}\PYG{o}{*}\PYG{n}{y}\PYG{o}{*}\PYG{o}{*}\PYG{l+m+mi}{2}\PYG{p}{)}\PYG{o}{/}\PYG{p}{(}\PYG{l+m+mi}{4}\PYG{o}{*}\PYG{n}{Dx}\PYG{o}{*}\PYG{n}{Dy}\PYG{p}{)}
\PYG{n}{sol} \PYG{o}{=} \PYG{n}{term1}\PYG{o}{*}\PYG{n}{np}\PYG{o}{.}\PYG{n}{exp}\PYG{p}{(}\PYG{n}{term2}\PYG{p}{)}\PYG{o}{*}\PYG{n}{sci}\PYG{o}{.}\PYG{n}{k0}\PYG{p}{(}\PYG{n}{args}\PYG{p}{)}

\PYG{c+c1}{\PYGZsh{} plots}
\PYG{n}{fig}\PYG{p}{,} \PYG{n}{ax} \PYG{o}{=} \PYG{n}{plt}\PYG{o}{.}\PYG{n}{subplots}\PYG{p}{(}\PYG{p}{)}
\PYG{n}{CS} \PYG{o}{=} \PYG{n}{ax}\PYG{o}{.}\PYG{n}{contour}\PYG{p}{(}\PYG{n}{x}\PYG{p}{,}\PYG{n}{y}\PYG{p}{,}\PYG{n}{sol}\PYG{p}{,} \PYG{n}{cmap}\PYG{o}{=}\PYG{l+s+s1}{\PYGZsq{}}\PYG{l+s+s1}{flag}\PYG{l+s+s1}{\PYGZsq{}}\PYG{p}{)}
\PYG{n}{ax}\PYG{o}{.}\PYG{n}{clabel}\PYG{p}{(}\PYG{n}{CS}\PYG{p}{,} \PYG{n}{inline}\PYG{o}{=}\PYG{l+m+mi}{1}\PYG{p}{,} \PYG{n}{fontsize}\PYG{o}{=} \PYG{l+m+mi}{10}\PYG{p}{)}
\PYG{n}{CB} \PYG{o}{=} \PYG{n}{fig}\PYG{o}{.}\PYG{n}{colorbar}\PYG{p}{(}\PYG{n}{CS}\PYG{p}{,} \PYG{n}{shrink}\PYG{o}{=}\PYG{l+m+mf}{0.8}\PYG{p}{,} \PYG{n}{extend}\PYG{o}{=}\PYG{l+s+s1}{\PYGZsq{}}\PYG{l+s+s1}{both}\PYG{l+s+s1}{\PYGZsq{}}\PYG{p}{)}\PYG{p}{;}
\end{sphinxVerbatim}

\end{sphinxuseclass}\end{sphinxVerbatimInput}
\begin{sphinxVerbatimOutput}

\begin{sphinxuseclass}{cell_output}
\noindent\sphinxincludegraphics{{C:/Users/vibhu/GWtextbook/_build/jupyter_execute/22_reactive_transport_8_0}.png}

\end{sphinxuseclass}\end{sphinxVerbatimOutput}

\end{sphinxuseclass}

\section{Equilibrium Sorption}
\label{\detokenize{content/transport/L10/22_reactive_transport:equilibrium-sorption}}
\sphinxAtStartPar
A reactive transport system can include a single reactive process, e.g., degradation, or combination of several multiple reactive processes, e.g. degradation and sorption. The inclusion of the reactive process(es) in the transport studies are site specific. The important to note is that an inclusion of a reactive process increases the complexity of transport problem.

\sphinxAtStartPar
In this course we limit ourselves with the following two types of reaction processes:

\sphinxAtStartPar
\sphinxstylestrong{1. Sorption}

\sphinxAtStartPar
\sphinxstylestrong{2. Degradation}

\sphinxAtStartPar
Acid\sphinxhyphen{}base reaction, precipitation\sphinxhyphen{}dissolution reaction, organic combustion etc. are among the reactions type that can be part of the reactive process individually or in any combination.

\sphinxAtStartPar
Also an important distinction is the rate or speed of the reaction. One distinguishes between time\sphinxhyphen{}dependent reaction (kinetics) or time\sphinxhyphen{}independent reactions (steady\sphinxhyphen{}state or equilibrium). Special reaction rates such as instantaneous reaction (extremely fast reaction) can also be part of the reaction process in the transport system.


\section{Sorption Basics}
\label{\detokenize{content/transport/L10/22_reactive_transport:sorption-basics}}
\sphinxAtStartPar
\sphinxstylestrong{Sorption} is a rather a general term used to indicate both \sphinxstylestrong{adsorption} and \sphinxstylestrong{absorption}. But in this course the \sphinxstyleemphasis{sorption} refers to only \sphinxstyleemphasis{adsorption.}

\sphinxAtStartPar
\sphinxstylestrong{Adsorption} can be more formally defined as the process of accumulation of dissolved chemicals on the surface of a solid, e.g., accumulation of a chemicals dissolved in groundwater on the surface of the aquifer material.

\sphinxAtStartPar
The figure below clarifies the \sphinxstyleemphasis{adsorption} process.

\begin{figure}[htbp]
\centering
\capstart

\noindent\sphinxincludegraphics[scale=0.3]{{T10_f1}.png}
\caption{Sorption terminology}\label{\detokenize{content/transport/L10/22_reactive_transport:sorption}}\end{figure}

\sphinxAtStartPar
In the figure \sphinxstyleemphasis{chemical in solution} (the circular objects) more often called \sphinxstylestrong{solute} in the water is found to attach is the solid surface. The figure presents the following two important terms part of the adsorption process:
\begin{quote}

\sphinxAtStartPar
\sphinxstylestrong{Adsorbent}: The solid onto which the chemicals are attached. More formally, \sphinxstyleemphasis{adsorbents} provide adsorption sites for solutes.
\end{quote}
\begin{quote}

\sphinxAtStartPar
\sphinxstylestrong{Adsorbate}: These are solutes that are attached on the \sphinxstyleemphasis{adsorbent.}
\end{quote}

\sphinxAtStartPar
Based on the figure, \sphinxstyleemphasis{adsorption} can be considered as a partition process that divides the chemical originally present in water between adsorbent and water.

\sphinxAtStartPar
Quite often adsorption is a reversible process, i.e., adsorbed chemicals can get back to water phase. This process is called \sphinxstylestrong{desorption}.

\sphinxAtStartPar
Speaking about \sphinxstyleemphasis{equilibrium,} this is reached when
\begin{quote}

\sphinxAtStartPar
\sphinxstyleemphasis{adsorption rate}  \(\rightleftharpoons \) \sphinxstyleemphasis{desorption rate}
\end{quote}

\sphinxAtStartPar
Adsorption in groundwater is often a rapid process. Although sorption kinetics can be important, the description in this introductory level course is limited to equilibrium sorption. Thus, we learn next to quantify equilibrium sorption.


\section{Adsorption Isotherms}
\label{\detokenize{content/transport/L10/22_reactive_transport:adsorption-isotherms}}
\sphinxAtStartPar
The adsorption process that has reached equilibrium can be relatively easily quantified with the use of empirical models called \sphinxstylestrong{isotherms}. These models are often simple algebraic equation that relates solute concentrations partitioned between the adsorbate and adsorbent at constant temperature. More than 15 different \sphinxstyleemphasis{isotherm} models can be found in the literature. However, in groundwater reactive transport studies the following three are the two most commonly used isotherms:
\begin{enumerate}
\sphinxsetlistlabels{\arabic}{enumi}{enumii}{}{.}%
\item {} 
\sphinxAtStartPar
\sphinxstylestrong{Henry or Linear isotherm} 

\item {} 
\sphinxAtStartPar
\sphinxstylestrong{Freundlich isotherm}

\end{enumerate}

\sphinxAtStartPar
For quantification, laboratory based experiments are performed using solids from subsurface and chemicals of interest. The laboratory observations are then graphically fitted with empirical isotherm models to quantify adsorption properties. Figure below shows isotherms that are particularly observed in groundwater transport studies. As can be observed in the figure sorption coefficient (\(K\)) is the common quantities obtained from isotherm models.

\begin{figure}[htbp]
\centering
\capstart

\noindent\sphinxincludegraphics[scale=0.4]{{T10_f2}.png}
\caption{Different types of sorption isotherms}\label{\detokenize{content/transport/L10/22_reactive_transport:sorption-type}}\end{figure}


\section{Henry Isotherm}
\label{\detokenize{content/transport/L10/22_reactive_transport:henry-isotherm}}
\sphinxAtStartPar
The \sphinxstylestrong{Henry isotherm} (Henry, 1803) is based on the idea of a \sphinxstyleemphasis{linear} relationship between the solute concentration \sphinxstylestrong{\(C\)} and the \sphinxstyleemphasis{adsorbate:adsorbent} mass ratio \(C_a\). Henry isotherm is quite often also called \sphinxstyleemphasis{linear isotherm} or the \(K_d\) model. Mathematically, the Henry isotherm is:
\begin{equation*}
\begin{split}
C_a = K_d \cdot C
\end{split}
\end{equation*}
\sphinxAtStartPar
with

\sphinxAtStartPar
\(C\) = solute concentration {[}ML\(^{-3}\){]}
\(C_a\) = mass ratio adsorbate:adsorbent {[}M:M{]}
\(K_d\) = distribution or partitioning coefficient {[}L\(^3\)M\(^{-1}\){]}.

\sphinxAtStartPar
Often symbols \(C_s\) or \(s\) are used instead of \(C_a\).

\sphinxAtStartPar
The Henry model has been most widely used in groundwater transport studies. This is largely because of the simplicity (see equation) of the model and it’s applicability in representing adsorption process more generally observed in groundwater studies. \(K_d\), the partitioning coefficient, is particularly used in groundwater transport studies. It is equal to the slope of the Henry isotherm.

\begin{sphinxadmonition}{note}{A quick example}

\sphinxAtStartPar
From the experimental data provided below, obtain the Henry distribution coefficient.
\end{sphinxadmonition}

\begin{sphinxuseclass}{cell}\begin{sphinxVerbatimInput}

\begin{sphinxuseclass}{cell_input}
\begin{sphinxVerbatim}[commandchars=\\\{\}]
\PYG{c+c1}{\PYGZsh{} Example of Henry isotherm (Source: Fetter et al. 2018)}

\PYG{c+c1}{\PYGZsh{} Following sorption data are available:}

\PYG{n}{C} \PYG{o}{=} \PYG{n}{np}\PYG{o}{.}\PYG{n}{array}\PYG{p}{(}\PYG{p}{[}\PYG{l+m+mi}{7}\PYG{p}{,} \PYG{l+m+mi}{15}\PYG{p}{,} \PYG{l+m+mi}{174}\PYG{p}{,} \PYG{l+m+mi}{249}\PYG{p}{,} \PYG{l+m+mi}{362}\PYG{p}{]}\PYG{p}{)} \PYG{c+c1}{\PYGZsh{} ug/L, Eq. concentration }
\PYG{n}{Ca} \PYG{o}{=} \PYG{n}{np}\PYG{o}{.}\PYG{n}{array}\PYG{p}{(}\PYG{p}{[}\PYG{l+m+mi}{2}\PYG{p}{,} \PYG{l+m+mi}{4}\PYG{p}{,} \PYG{l+m+mi}{33}\PYG{p}{,} \PYG{l+m+mi}{50}\PYG{p}{,} \PYG{l+m+mi}{70}\PYG{p}{]}\PYG{p}{)} \PYG{c+c1}{\PYGZsh{} ug/g, Eq. sorbed mass }

\PYG{c+c1}{\PYGZsh{} linear\PYGZhy{} fit y = m*x+c}

\PYG{n}{slope}\PYG{p}{,} \PYG{n}{intercept}\PYG{p}{,} \PYG{n}{r\PYGZus{}value}\PYG{p}{,} \PYG{n}{p\PYGZus{}value}\PYG{p}{,} \PYG{n}{std\PYGZus{}err} \PYG{o}{=} \PYG{n}{stats}\PYG{o}{.}\PYG{n}{linregress}\PYG{p}{(}\PYG{n}{C}\PYG{p}{,} \PYG{n}{Ca}\PYG{p}{)}
\PYG{n+nb}{print}\PYG{p}{(}\PYG{l+s+s2}{\PYGZdq{}}\PYG{l+s+s2}{slope = }\PYG{l+s+si}{\PYGZpc{}0.3f}\PYG{l+s+s2}{    intercept= }\PYG{l+s+si}{\PYGZpc{}0.3f}\PYG{l+s+s2}{  R\PYGZhy{}squared=}\PYG{l+s+si}{\PYGZpc{}0.4f}\PYG{l+s+s2}{\PYGZdq{}} \PYG{o}{\PYGZpc{}} \PYG{p}{(}\PYG{n}{slope}\PYG{p}{,} \PYG{n}{intercept}\PYG{p}{,} \PYG{n}{r\PYGZus{}value}\PYG{o}{*}\PYG{o}{*}\PYG{l+m+mi}{2}\PYG{p}{)}\PYG{p}{,}\PYG{l+s+s1}{\PYGZsq{}}\PYG{l+s+se}{\PYGZbs{}n}\PYG{l+s+s1}{\PYGZsq{}}\PYG{p}{)}
\PYG{n}{fit\PYGZus{}line} \PYG{o}{=} \PYG{n}{slope}\PYG{o}{*}\PYG{n}{C} \PYG{o}{+} \PYG{n}{intercept}

\PYG{c+c1}{\PYGZsh{}plot}

\PYG{n}{plt}\PYG{o}{.}\PYG{n}{scatter}\PYG{p}{(}\PYG{n}{C}\PYG{p}{,} \PYG{n}{Ca}\PYG{p}{,} \PYG{n}{label}\PYG{o}{=} \PYG{l+s+s2}{\PYGZdq{}}\PYG{l+s+s2}{Original data}\PYG{l+s+s2}{\PYGZdq{}}\PYG{p}{)} \PYG{c+c1}{\PYGZsh{} data plot}
\PYG{n}{plt}\PYG{o}{.}\PYG{n}{plot}\PYG{p}{(}\PYG{n}{C}\PYG{p}{,} \PYG{n}{fit\PYGZus{}line}\PYG{p}{,} \PYG{n}{color} \PYG{o}{=} \PYG{l+s+s2}{\PYGZdq{}}\PYG{l+s+s2}{red}\PYG{l+s+s2}{\PYGZdq{}}\PYG{p}{,} \PYG{n}{label} \PYG{o}{=} \PYG{l+s+s2}{\PYGZdq{}}\PYG{l+s+s2}{fit\PYGZhy{}line}\PYG{l+s+s2}{\PYGZdq{}}\PYG{p}{)}
\PYG{n}{plt}\PYG{o}{.}\PYG{n}{legend}\PYG{p}{(}\PYG{p}{)}\PYG{p}{;} \PYG{n}{plt}\PYG{o}{.}\PYG{n}{xlabel}\PYG{p}{(}\PYG{l+s+sa}{r}\PYG{l+s+s2}{\PYGZdq{}}\PYG{l+s+s2}{Equilibrium Aqueous Concentration, \PYGZdl{}C\PYGZdl{} (\PYGZdl{}}\PYG{l+s+s2}{\PYGZbs{}}\PYG{l+s+s2}{mu\PYGZdl{}g/L) }\PYG{l+s+s2}{\PYGZdq{}}\PYG{p}{)}
\PYG{n}{plt}\PYG{o}{.}\PYG{n}{ylabel}\PYG{p}{(}\PYG{l+s+sa}{r}\PYG{l+s+s2}{\PYGZdq{}}\PYG{l+s+s2}{Mass sorbed per unit absorbent weight, \PYGZdl{}C\PYGZus{}a\PYGZdl{} (\PYGZdl{}}\PYG{l+s+s2}{\PYGZbs{}}\PYG{l+s+s2}{mu\PYGZdl{}g/g) }\PYG{l+s+s2}{\PYGZdq{}}\PYG{p}{)}\PYG{p}{;}  
\PYG{n}{plt}\PYG{o}{.}\PYG{n}{text}\PYG{p}{(}\PYG{l+m+mi}{0}\PYG{p}{,} \PYG{l+m+mi}{50}\PYG{p}{,} \PYG{l+s+s1}{\PYGZsq{}}\PYG{l+s+s1}{\PYGZdl{}C\PYGZus{}a=}\PYG{l+s+si}{\PYGZpc{}0.5s}\PYG{l+s+s1}{ C + }\PYG{l+s+si}{\PYGZpc{}0.5s}\PYG{l+s+s1}{\PYGZdl{}}\PYG{l+s+s1}{\PYGZsq{}}\PYG{o}{\PYGZpc{}}\PYG{p}{(}\PYG{n}{slope}\PYG{p}{,} \PYG{n}{intercept}\PYG{p}{)}\PYG{p}{,} \PYG{n}{fontsize}\PYG{o}{=}\PYG{l+m+mi}{10}\PYG{p}{)}
\PYG{n}{plt}\PYG{o}{.}\PYG{n}{text}\PYG{p}{(}\PYG{l+m+mi}{0}\PYG{p}{,} \PYG{l+m+mi}{40}\PYG{p}{,} \PYG{l+s+s1}{\PYGZsq{}}\PYG{l+s+s1}{\PYGZdl{}R\PYGZca{}2=}\PYG{l+s+si}{\PYGZpc{}0.5s}\PYG{l+s+s1}{ \PYGZdl{}}\PYG{l+s+s1}{\PYGZsq{}}\PYG{o}{\PYGZpc{}}\PYG{p}{(}\PYG{n}{r\PYGZus{}value}\PYG{o}{*}\PYG{o}{*}\PYG{l+m+mi}{2}\PYG{p}{)}\PYG{p}{,} \PYG{n}{fontsize}\PYG{o}{=}\PYG{l+m+mi}{10}\PYG{p}{)}

\PYG{c+c1}{\PYGZsh{} Output}

\PYG{n+nb}{print}\PYG{p}{(}\PYG{l+s+s2}{\PYGZdq{}}\PYG{l+s+s2}{The required partition coefficient = slope,= }\PYG{l+s+si}{\PYGZpc{}0.5s}\PYG{l+s+s2}{ L/g }\PYG{l+s+s2}{\PYGZdq{}} \PYG{o}{\PYGZpc{}} \PYG{n}{slope}\PYG{p}{)}
\end{sphinxVerbatim}

\end{sphinxuseclass}\end{sphinxVerbatimInput}
\begin{sphinxVerbatimOutput}

\begin{sphinxuseclass}{cell_output}
\begin{sphinxVerbatim}[commandchars=\\\{\}]
slope = 0.192    intercept= 0.775  R\PYGZhy{}squared=0.9990 

The required partition coefficient = slope,= 0.192 L/g 
\end{sphinxVerbatim}

\noindent\sphinxincludegraphics{{C:/Users/vibhu/GWtextbook/_build/jupyter_execute/22_reactive_transport_14_1}.png}

\end{sphinxuseclass}\end{sphinxVerbatimOutput}

\end{sphinxuseclass}

\section{Freundlich Isotherm}
\label{\detokenize{content/transport/L10/22_reactive_transport:freundlich-isotherm}}
\sphinxAtStartPar
\sphinxstylestrong{Freundlich isotherm} (Freundlich, 1907) is a more general isotherm. It is based on the idea of a power law, i.e., includes also the non\sphinxhyphen{}linear behaviour, relating the solute concentration \(C\) to the adsorbate:adsorbent mass ration \(C_a\). The isotherm is mathematically given as
\begin{equation*}
\begin{split}
C_a = K_{Fr} \cdot C^N
\end{split}
\end{equation*}
\sphinxAtStartPar
with

\sphinxAtStartPar
\(C\) = solute concentration {[}ML\(^{-3}\){]}
\(C_a\) = mass ratio adsorbate:adsorbent {[}M:M{]}
\(n\) = Freundlich exponent {[}\sphinxhyphen{}{]}
\(K_{Fr}\) = Freundlich partitioning coefficient {[}(M:M)/(M/L\(^3)^n\){]}.

\sphinxAtStartPar
The Freundlich isotherm equation can be easily linearized by applying logarithmic transformation of the equation, which gives

\sphinxAtStartPar
\textbackslash{}begin\{eqnarray*\}
\textbackslash{}log C\_a = \textbackslash{}log K\_\{Fr\} + n\textbackslash{}cdot \textbackslash{}log C
\textbackslash{}end\{eqnarray*\}

\sphinxAtStartPar
The above equation resembles the straight line equation \(y = b + a \cdot x b\), in which \(b\equiv \log K_{Fr}\) is the intercept and \(a\equiv n\) the slope. Thus, from fitting the adsoprtion experimental results with the above equation, both \(n\) and \(K_{Fr}\) can be obtained.

\begin{sphinxadmonition}{note}{A quick example}

\sphinxAtStartPar
From the experimental data provided below, obtain the Freundlich partitioning coefficient and Freundlich exponent.
\end{sphinxadmonition}

\begin{sphinxuseclass}{cell}\begin{sphinxVerbatimInput}

\begin{sphinxuseclass}{cell_input}
\begin{sphinxVerbatim}[commandchars=\\\{\}]
\PYG{c+c1}{\PYGZsh{} Example of Freundlich isotherm}

\PYG{c+c1}{\PYGZsh{} Following sorption data are available:}

\PYG{n}{Cf}\PYG{o}{=} \PYG{n}{np}\PYG{o}{.}\PYG{n}{array}\PYG{p}{(}\PYG{p}{[}\PYG{l+m+mf}{23.6}\PYG{p}{,} \PYG{l+m+mf}{6.67}\PYG{p}{,} \PYG{l+m+mf}{3.26}\PYG{p}{,} \PYG{l+m+mf}{0.322}\PYG{p}{,} \PYG{l+m+mf}{0.169}\PYG{p}{,} \PYG{l+m+mf}{0.114}\PYG{p}{]}\PYG{p}{)} \PYG{c+c1}{\PYGZsh{} mg/L, Eq. concentration }
\PYG{n}{Caf} \PYG{o}{=} \PYG{n}{np}\PYG{o}{.}\PYG{n}{array}\PYG{p}{(}\PYG{p}{[}\PYG{l+m+mi}{737}\PYG{p}{,} \PYG{l+m+mi}{450}\PYG{p}{,} \PYG{l+m+mi}{318}\PYG{p}{,} \PYG{l+m+mi}{121}\PYG{p}{,} \PYG{l+m+mf}{85.2}\PYG{p}{,} \PYG{l+m+mf}{75.8}\PYG{p}{]}\PYG{p}{)} \PYG{c+c1}{\PYGZsh{} mg/g, Eq. sorbed mass }

\PYG{n}{logCf} \PYG{o}{=} \PYG{n}{np}\PYG{o}{.}\PYG{n}{log10}\PYG{p}{(}\PYG{n}{Cf}\PYG{p}{)} \PYG{c+c1}{\PYGZsh{} log10 transformation of data}
\PYG{n}{logCaf} \PYG{o}{=} \PYG{n}{np}\PYG{o}{.}\PYG{n}{log10}\PYG{p}{(}\PYG{n}{Caf}\PYG{p}{)}

\PYG{c+c1}{\PYGZsh{} fitting: y = mx +c}
\PYG{n}{slope}\PYG{p}{,} \PYG{n}{intercept}\PYG{p}{,} \PYG{n}{r\PYGZus{}value}\PYG{p}{,} \PYG{n}{p\PYGZus{}value}\PYG{p}{,} \PYG{n}{std\PYGZus{}err} \PYG{o}{=} \PYG{n}{stats}\PYG{o}{.}\PYG{n}{linregress}\PYG{p}{(}\PYG{n}{logCf}\PYG{p}{,} \PYG{n}{logCaf}\PYG{p}{)}
\PYG{n+nb}{print}\PYG{p}{(}\PYG{l+s+s2}{\PYGZdq{}}\PYG{l+s+s2}{slope: }\PYG{l+s+si}{\PYGZpc{}0.3f}\PYG{l+s+s2}{    intercept: }\PYG{l+s+si}{\PYGZpc{}0.3f}\PYG{l+s+s2}{  R\PYGZhy{}squared: }\PYG{l+s+si}{\PYGZpc{}0.3f}\PYG{l+s+s2}{\PYGZdq{}} \PYG{o}{\PYGZpc{}} \PYG{p}{(}\PYG{n}{slope}\PYG{p}{,} \PYG{n}{intercept}\PYG{p}{,} \PYG{n}{r\PYGZus{}value}\PYG{o}{*}\PYG{o}{*}\PYG{l+m+mi}{2}\PYG{p}{)}\PYG{p}{)}
\PYG{n}{fit\PYGZus{}line} \PYG{o}{=} \PYG{n}{slope}\PYG{o}{*}\PYG{n}{logCf} \PYG{o}{+} \PYG{n}{intercept}

\PYG{c+c1}{\PYGZsh{} plots}
\PYG{n}{plt}\PYG{o}{.}\PYG{n}{figure}\PYG{p}{(}\PYG{n}{figsize}\PYG{o}{=}\PYG{p}{(}\PYG{l+m+mi}{10}\PYG{p}{,}\PYG{l+m+mi}{4}\PYG{p}{)}\PYG{p}{)}

\PYG{n}{plt}\PYG{o}{.}\PYG{n}{subplot}\PYG{p}{(}\PYG{l+m+mi}{121}\PYG{p}{)}
\PYG{n}{plt}\PYG{o}{.}\PYG{n}{plot}\PYG{p}{(}\PYG{n}{Cf}\PYG{p}{,} \PYG{n}{Caf}\PYG{p}{,} \PYG{l+s+s2}{\PYGZdq{}}\PYG{l+s+s2}{*\PYGZhy{}\PYGZhy{}}\PYG{l+s+s2}{\PYGZdq{}}\PYG{p}{,} \PYG{n}{label}\PYG{o}{=} \PYG{l+s+s2}{\PYGZdq{}}\PYG{l+s+s2}{Original data}\PYG{l+s+s2}{\PYGZdq{}}\PYG{p}{)}
\PYG{n}{plt}\PYG{o}{.}\PYG{n}{legend}\PYG{p}{(}\PYG{p}{)}\PYG{p}{;} \PYG{n}{plt}\PYG{o}{.}\PYG{n}{xlabel}\PYG{p}{(}\PYG{l+s+sa}{r}\PYG{l+s+s2}{\PYGZdq{}}\PYG{l+s+s2}{Eq. Aq. Conc., \PYGZdl{}C\PYGZdl{} (\PYGZdl{}mg\PYGZdl{}g/L) }\PYG{l+s+s2}{\PYGZdq{}}\PYG{p}{)}\PYG{p}{;} 
\PYG{n}{plt}\PYG{o}{.}\PYG{n}{ylabel}\PYG{p}{(}\PYG{l+s+sa}{r}\PYG{l+s+s2}{\PYGZdq{}}\PYG{l+s+s2}{Mass sorbed/absorbent weight, \PYGZdl{}C\PYGZus{}a\PYGZdl{} (\PYGZdl{}mg\PYGZdl{}g/g) }\PYG{l+s+s2}{\PYGZdq{}}\PYG{p}{)}\PYG{p}{;} 

\PYG{n}{plt}\PYG{o}{.}\PYG{n}{subplot}\PYG{p}{(}\PYG{l+m+mi}{122}\PYG{p}{)}
\PYG{n}{plt}\PYG{o}{.}\PYG{n}{scatter}\PYG{p}{(}\PYG{n}{logCf}\PYG{p}{,} \PYG{n}{logCaf}\PYG{p}{,} \PYG{n}{label}\PYG{o}{=}\PYG{l+s+s2}{\PYGZdq{}}\PYG{l+s+s2}{Log transformed data}\PYG{l+s+s2}{\PYGZdq{}}\PYG{p}{)} 
\PYG{n}{plt}\PYG{o}{.}\PYG{n}{plot}\PYG{p}{(}\PYG{n}{logCf}\PYG{p}{,} \PYG{n}{fit\PYGZus{}line}\PYG{p}{,} \PYG{n}{color}\PYG{o}{=}\PYG{l+s+s2}{\PYGZdq{}}\PYG{l+s+s2}{red}\PYG{l+s+s2}{\PYGZdq{}}\PYG{p}{,} \PYG{n}{label}\PYG{o}{=} \PYG{l+s+s2}{\PYGZdq{}}\PYG{l+s+s2}{linear fit line}\PYG{l+s+s2}{\PYGZdq{}}\PYG{p}{)}
\PYG{n}{plt}\PYG{o}{.}\PYG{n}{legend}\PYG{p}{(}\PYG{p}{)}\PYG{p}{;} \PYG{n}{plt}\PYG{o}{.}\PYG{n}{xlabel}\PYG{p}{(}\PYG{l+s+sa}{r}\PYG{l+s+s2}{\PYGZdq{}}\PYG{l+s+s2}{Eq. Aq. Conc., \PYGZdl{}}\PYG{l+s+s2}{\PYGZbs{}}\PYG{l+s+s2}{log C\PYGZdl{} (\PYGZdl{}mg\PYGZdl{}g/L) }\PYG{l+s+s2}{\PYGZdq{}}\PYG{p}{)}\PYG{p}{;} 
\PYG{n}{plt}\PYG{o}{.}\PYG{n}{ylabel}\PYG{p}{(}\PYG{l+s+sa}{r}\PYG{l+s+s2}{\PYGZdq{}}\PYG{l+s+s2}{Mass sorbed/absorbent weight, \PYGZdl{}}\PYG{l+s+s2}{\PYGZbs{}}\PYG{l+s+s2}{log C\PYGZus{}a\PYGZdl{} (\PYGZdl{}mg\PYGZdl{}/g) }\PYG{l+s+s2}{\PYGZdq{}}\PYG{p}{)}\PYG{p}{;} 
\PYG{n}{plt}\PYG{o}{.}\PYG{n}{text}\PYG{p}{(}\PYG{o}{\PYGZhy{}}\PYG{l+m+mi}{1}\PYG{p}{,} \PYG{l+m+mf}{2.6}\PYG{p}{,} \PYG{l+s+s1}{\PYGZsq{}}\PYG{l+s+s1}{\PYGZdl{}C\PYGZus{}a=}\PYG{l+s+si}{\PYGZpc{}0.5s}\PYG{l+s+s1}{ C + }\PYG{l+s+si}{\PYGZpc{}0.5s}\PYG{l+s+s1}{\PYGZdl{}}\PYG{l+s+s1}{\PYGZsq{}}\PYG{o}{\PYGZpc{}}\PYG{p}{(}\PYG{n}{slope}\PYG{p}{,} \PYG{n}{intercept}\PYG{p}{)}\PYG{p}{,} \PYG{n}{fontsize}\PYG{o}{=}\PYG{l+m+mi}{10}\PYG{p}{)}
\PYG{n}{plt}\PYG{o}{.}\PYG{n}{text}\PYG{p}{(}\PYG{o}{\PYGZhy{}}\PYG{l+m+mi}{1}\PYG{p}{,} \PYG{l+m+mf}{2.5}\PYG{p}{,} \PYG{l+s+s1}{\PYGZsq{}}\PYG{l+s+s1}{\PYGZdl{}R\PYGZca{}2=}\PYG{l+s+si}{\PYGZpc{}0.5s}\PYG{l+s+s1}{ \PYGZdl{}}\PYG{l+s+s1}{\PYGZsq{}}\PYG{o}{\PYGZpc{}}\PYG{p}{(}\PYG{n}{r\PYGZus{}value}\PYG{o}{*}\PYG{o}{*}\PYG{l+m+mi}{2}\PYG{p}{)}\PYG{p}{,} \PYG{n}{fontsize}\PYG{o}{=}\PYG{l+m+mi}{10}\PYG{p}{)}
\PYG{n}{plt}\PYG{o}{.}\PYG{n}{subplots\PYGZus{}adjust}\PYG{p}{(}\PYG{n}{wspace}\PYG{o}{=}\PYG{l+m+mf}{0.35}\PYG{p}{)}

\PYG{n+nb}{print}\PYG{p}{(}\PYG{l+s+s2}{\PYGZdq{}}\PYG{l+s+s2}{Freundlich partitioning coefficient =  }\PYG{l+s+si}{\PYGZpc{}0.5s}\PYG{l+s+s2}{ (mg/g)1/n (mg/L) and  Freundlich exponent = }\PYG{l+s+si}{\PYGZpc{}0.4s}\PYG{l+s+s2}{\PYGZdq{}}  \PYG{o}{\PYGZpc{}} \PYG{p}{(}\PYG{l+m+mi}{10}\PYG{o}{*}\PYG{o}{*}\PYG{n}{intercept}\PYG{p}{,} \PYG{n}{slope}\PYG{p}{)}\PYG{p}{)}
\end{sphinxVerbatim}

\end{sphinxuseclass}\end{sphinxVerbatimInput}
\begin{sphinxVerbatimOutput}

\begin{sphinxuseclass}{cell_output}
\begin{sphinxVerbatim}[commandchars=\\\{\}]
slope: 0.433    intercept: 2.283  R\PYGZhy{}squared: 0.999
Freundlich partitioning coefficient =  191.8 (mg/g)1/n (mg/L) and  Freundlich exponent = 0.43
\end{sphinxVerbatim}

\noindent\sphinxincludegraphics{{C:/Users/vibhu/GWtextbook/_build/jupyter_execute/22_reactive_transport_17_1}.png}

\end{sphinxuseclass}\end{sphinxVerbatimOutput}

\end{sphinxuseclass}

\section{Retardation Factor (for Henry Isotherm)}
\label{\detokenize{content/transport/L10/22_reactive_transport:retardation-factor-for-henry-isotherm}}
\sphinxAtStartPar
The net effect of adsorption is the retarded movement of solute in comparison to the average flow of the groundwater. The term \sphinxstylestrong{Retardation Factor} \((R)\) is defined that quantifies the retarded movement of solute. The formulation of \(R\) is based on the type of isotherm. For Henry isotherm \(R\) can be straightforwardly calculated with the help of a mass budget.

\sphinxAtStartPar
For this purpose, an aquifer volume \(V\) with the effective porosity \(n_e\) is considered (see fig. below)

\begin{figure}[htbp]
\centering
\capstart

\noindent\sphinxincludegraphics[scale=0.7]{{T10_f3}.png}
\caption{Retardation factor}\label{\detokenize{content/transport/L10/22_reactive_transport:retardation}}\end{figure}

\sphinxAtStartPar
The steps involved are:
\begin{itemize}
\item {} 
\sphinxAtStartPar
Total volume: \(V\)

\item {} 
\sphinxAtStartPar
Water volume: \(n_e \cdot V\)

\item {} 
\sphinxAtStartPar
Mass of dissolved chemical: \(n_e \cdot V \cdot C\)

\item {} 
\sphinxAtStartPar
Volume of solid: \((1-n_e)\cdot V\)

\item {} 
\sphinxAtStartPar
Density of solid material: \(\rho\)

\item {} 
\sphinxAtStartPar
Mass of solid: \(\rho \cdot(1-n_e)\cdot V\)

\item {} 
\sphinxAtStartPar
Mass of adsorbate: \(\rho \cdot(1-n_e)\cdot V\cdot C_a\) = \((1-n_e)\cdot\rho \cdot V \cdot K_d \cdot C\)

\item {} 
\sphinxAtStartPar
Total mass: \(n_e\cdot V \cdot C + (1-n_e)\cdot\rho \cdot V\cdot K_d \cdot C = n_e \cdot R \cdot V \cdot C\) 
with \sphinxstyleemphasis{Retardation factor} \$\(R = 1 + \frac{1-n_e}{n_e}\cdot \rho \cdot K_d\)\$

\end{itemize}

\sphinxAtStartPar
The expression for \(R\) can be further modified  by using bulk density \(\rho_b\) \(= (1-n_e)\cdot \rho\) = mass of solid/total volume. This leads to
\begin{equation*}
\begin{split}
R = 1+\frac{\rho_b}{n_e} \cdot K_d
\end{split}
\end{equation*}
\sphinxAtStartPar
As can be observed from the equation, \(R = 1\) when there is no adsorption, i.e., when \(K_d= 0\).

\begin{sphinxadmonition}{note}{A quick example}

\sphinxAtStartPar
Calculate the retardation factor from the provided data.
\end{sphinxadmonition}

\begin{sphinxuseclass}{cell}\begin{sphinxVerbatimInput}

\begin{sphinxuseclass}{cell_input}
\begin{sphinxVerbatim}[commandchars=\\\{\}]
\PYG{n+nb}{print}\PYG{p}{(}\PYG{l+s+s2}{\PYGZdq{}}\PYG{l+s+se}{\PYGZbs{}033}\PYG{l+s+s2}{[0m You can change the provided values.}\PYG{l+s+se}{\PYGZbs{}n}\PYG{l+s+s2}{\PYGZdq{}}\PYG{p}{)}

\PYG{n}{ne} \PYG{o}{=} \PYG{l+m+mf}{0.4} \PYG{c+c1}{\PYGZsh{}effective porosity [\PYGZhy{}]}
\PYG{n}{rho} \PYG{o}{=} \PYG{l+m+mf}{1.25} \PYG{c+c1}{\PYGZsh{} density of solid material [kg/m³]}
\PYG{n}{Kd} \PYG{o}{=} \PYG{l+m+mf}{0.2} \PYG{c+c1}{\PYGZsh{} distribution or partition coefficient [m³/kg]}

\PYG{c+c1}{\PYGZsh{}intermediate calculation}
\PYG{n}{rho\PYGZus{}b} \PYG{o}{=} \PYG{p}{(}\PYG{l+m+mi}{1}\PYG{o}{\PYGZhy{}}\PYG{n}{ne}\PYG{p}{)}\PYG{o}{*}\PYG{n}{rho}

\PYG{c+c1}{\PYGZsh{}solution}
\PYG{n}{R}\PYG{o}{=}\PYG{l+m+mi}{1}\PYG{o}{+}\PYG{p}{(}\PYG{n}{rho\PYGZus{}b}\PYG{o}{/}\PYG{n}{ne}\PYG{p}{)}\PYG{o}{*}\PYG{n}{Kd}

\PYG{n+nb}{print}\PYG{p}{(}\PYG{l+s+s2}{\PYGZdq{}}\PYG{l+s+s2}{effective porosity = }\PYG{l+s+si}{\PYGZob{}\PYGZcb{}}\PYG{l+s+se}{\PYGZbs{}n}\PYG{l+s+s2}{density of solid material = }\PYG{l+s+si}{\PYGZob{}\PYGZcb{}}\PYG{l+s+s2}{ kg/m³}\PYG{l+s+se}{\PYGZbs{}n}\PYG{l+s+s2}{Distribution or partition coefficient = }\PYG{l+s+si}{\PYGZob{}\PYGZcb{}}\PYG{l+s+s2}{ m³/kg}\PYG{l+s+se}{\PYGZbs{}n}\PYG{l+s+s2}{\PYGZdq{}}\PYG{o}{.}\PYG{n}{format}\PYG{p}{(}\PYG{n}{ne}\PYG{p}{,} \PYG{n}{rho}\PYG{p}{,} \PYG{n}{Kd}\PYG{p}{)}\PYG{p}{)}
\PYG{n+nb}{print}\PYG{p}{(}\PYG{l+s+s2}{\PYGZdq{}}\PYG{l+s+se}{\PYGZbs{}033}\PYG{l+s+s2}{[1mSolution:}\PYG{l+s+se}{\PYGZbs{}033}\PYG{l+s+s2}{[0m}\PYG{l+s+se}{\PYGZbs{}n}\PYG{l+s+s2}{The resulting retardation factor is }\PYG{l+s+se}{\PYGZbs{}033}\PYG{l+s+s2}{[1m}\PYG{l+s+si}{\PYGZob{}:02.4\PYGZcb{}}\PYG{l+s+se}{\PYGZbs{}033}\PYG{l+s+s2}{[0m.}\PYG{l+s+s2}{\PYGZdq{}}\PYG{o}{.}\PYG{n}{format}\PYG{p}{(}\PYG{n}{R}\PYG{p}{)}\PYG{p}{)}
\end{sphinxVerbatim}

\end{sphinxuseclass}\end{sphinxVerbatimInput}
\begin{sphinxVerbatimOutput}

\begin{sphinxuseclass}{cell_output}
\begin{sphinxVerbatim}[commandchars=\\\{\}]
 You can change the provided values.

effective porosity = 0.4
density of solid material = 1.25 kg/m³
Distribution or partition coefficient = 0.2 m³/kg

\PYG{Color+ColorBold}{Solution:}
The resulting retardation factor is \PYG{Color+ColorBold}{1.375}.
\end{sphinxVerbatim}

\end{sphinxuseclass}\end{sphinxVerbatimOutput}

\end{sphinxuseclass}

\section{Degradation}
\label{\detokenize{content/transport/L10/22_reactive_transport:degradation}}
\sphinxAtStartPar
\sphinxstylestrong{Degradation} leads to alteration or transformation of chemical structure of chemicals. This contrasts to adsorption in which chemical structure is not altered. In adsorption (or desorption) the original chemical is partitioned between the solid particles and water. It is \sphinxstyleemphasis{degradation} that eventually lead to removal of the \sphinxstyleemphasis{original} chemical from the groundwater. The transformation of original chemical, due to degradation, results to so\sphinxhyphen{}called \sphinxstyleemphasis{daughter products (metabolites).} The new chemical(s) can make groundwater more suitable (decrease contamination) or further contaminate it.

\sphinxAtStartPar
In groundwater studies, degradation can appear as:
\begin{itemize}
\item {} 
\sphinxAtStartPar
\sphinxstylestrong{Radioactive decay}

\item {} 
\sphinxAtStartPar
\sphinxstylestrong{Microbial degradation (bio\sphinxhyphen{}degradation)}

\item {} 
\sphinxAtStartPar
\sphinxstylestrong{Chemical degradation}

\end{itemize}

\sphinxAtStartPar
There are several approaches to quantify degradation process.
A common aspect to most of them is the assumption of \sphinxstyleemphasis{time\sphinxhyphen{}dependency} (or \sphinxstyleemphasis{Kinetics} ).


\section{\protect\(n^{th}\protect\) \sphinxhyphen{} Order Degradation Kinetics}
\label{\detokenize{content/transport/L10/22_reactive_transport:n-th-order-degradation-kinetics}}
\sphinxAtStartPar
The general equation for the degradation kinetics is:
\begin{equation*}
\begin{split}
\frac{\text{d}C}{\text{d} t} = - \lambda \cdot C^n
\end{split}
\end{equation*}
\sphinxAtStartPar
with \(t\) = time {[}t{]} 
\(C\) = solute concentration {[}ML\(^{-3}\){]} 
\(n\) = order of the degradation kinetics {[} \sphinxhyphen{} {]} (\(n\geq 0)\) 
\(\lambda\) = degradation rate constant {[}(ML\(^{-3})^{(1-n)}\)T\(^{-1}\){]}.

\sphinxAtStartPar
Considering the initial concentration (or input concentration) \(C_0\), the solutions of the kinetics equation are:
\begin{equation*}
\begin{split}
C(t) = C_0\cdot e^{-\lambda \cdot t} \: \: \: \text{if }\: n = 1  
\end{split}
\end{equation*}
\sphinxAtStartPar
and
\begin{equation*}
\begin{split}
C(t) = [C_0^{1-n} - (1-n)\cdot \lambda t]^{\frac{1}{1-n}} \:\:\: \text{if }\: n\neq 1 
\end{split}
\end{equation*}
\sphinxAtStartPar
The \sphinxstylestrong{half life} \((T_{1/2})\), which is the time span elapsing until the initial concentration \(C_0\) is reduced by half, is an important time\sphinxhyphen{}scale in the degradation analysis. \(T_{1/2}\) is \(C_0\) dependent in nearly all cases with an exception for 1\(^\text{st}\)\sphinxhyphen{} order degradation kinetics. 0\(^{th}\)\sphinxhyphen{}order and the 1\(^\text{st}\)\sphinxhyphen{} order degradation kinetics are most commonly observed in groundwater studies. The \((T_{1/2})\) of these orders are:
\begin{equation*}
\begin{split}
T_{1/2} = \frac{C_0}{2\cdot \lambda} \:\:\: \text{for } \:0^{\text{th}}\text{-order}   
\end{split}
\end{equation*}\begin{equation*}
\begin{split}
T_{1/2} = \frac{\ln 2}{\lambda} \:\:\: \text{for } \:1^{\text{st}}\text{-order}   
\end{split}
\end{equation*}
\sphinxAtStartPar
As can be observed above \(T_{1/2}\) is independent of concentration for the 1\(^{\text{st}}\)\sphinxhyphen{}order degradation kinetics.

\sphinxAtStartPar
Another important properties of the degradation kinetics is that for \(n\geq 1\) the solute concentration \sphinxstyleemphasis{asymptotically} approaches zero, whereas for \(n<1\), the solute concentration actually reaches zero

\begin{sphinxuseclass}{cell}
\begin{sphinxuseclass}{tag_hide-input}\begin{sphinxVerbatimOutput}

\begin{sphinxuseclass}{cell_output}
\noindent\sphinxincludegraphics{{C:/Users/vibhu/GWtextbook/_build/jupyter_execute/22_reactive_transport_23_0}.png}

\end{sphinxuseclass}\end{sphinxVerbatimOutput}

\end{sphinxuseclass}
\end{sphinxuseclass}

\section{Radioactive decay}
\label{\detokenize{content/transport/L10/22_reactive_transport:radioactive-decay}}
\sphinxAtStartPar
Radioactive decay is degradation of a chemical due to radiation. The radioactive decay is limited to radioactive chemicals such as Cobalt, Cesium, Iodine. This decay obeys the 1\(^\text{st}\)\sphinxhyphen{} order degradation kinetics and therefore the half\sphinxhyphen{}life is \(T_{1/2} = \frac{\ln 2}{\lambda}\). \(T_{1/2}\) is characteristic property of radioactive chemicals and it can be used to compute degradation rate (\(\lambda\)).

\begin{sphinxuseclass}{cell}
\begin{sphinxuseclass}{tag_hide-input}\begin{sphinxVerbatimOutput}

\begin{sphinxuseclass}{cell_output}
\begin{sphinxVerbatim}[commandchars=\\\{\}]
   time (a)  Cobalt 60 (mg/L)  Strontium 90 (mg/L)
0         0             10.00                10.00
1         1              8.76                 9.76
2         2              7.68                 9.52
3         5              5.17                 8.84
4        10              2.68                 7.81
5        20              0.72                 6.10
6        28              0.25                 5.00

 The degradation rate (λ) for Cobalt 60 = 0.024 1/y and for Strontium 90 = 0.131 1/y 
\end{sphinxVerbatim}

\noindent\sphinxincludegraphics{{C:/Users/vibhu/GWtextbook/_build/jupyter_execute/22_reactive_transport_25_1}.png}

\end{sphinxuseclass}\end{sphinxVerbatimOutput}

\end{sphinxuseclass}
\end{sphinxuseclass}

\section{Joint Action of Conservative and Reactive Transport (1D)}
\label{\detokenize{content/transport/L10/22_reactive_transport:joint-action-of-conservative-and-reactive-transport-1d}}

\section{Concentration Profile}
\label{\detokenize{content/transport/L10/22_reactive_transport:concentration-profile}}
\sphinxAtStartPar
Figure below presents the joint action of conservative transport with equilibrium sorption (linear isotherm) and degradation. The figure shows the solute concentration \(C\) (in water) at the same time\sphinxhyphen{}levels for various combinations of acting processes.

\begin{figure}[htbp]
\centering
\capstart

\noindent\sphinxincludegraphics[scale=0.6]{{T10_f4}.png}
\caption{1D Conservative and Reactive Transport}\label{\detokenize{content/transport/L10/22_reactive_transport:d-cons-react}}\end{figure}

\sphinxAtStartPar
The figure can be explained in the following way:

\sphinxAtStartPar
(A): The solute is initially present at constant concentration in a limited area.

\sphinxAtStartPar
(B): Solute spreads only due to advection. Due to absence of dispersion there is no (1D) spreading effect.

\sphinxAtStartPar
(C): Inclusion of dispersion process causes spread of concentration. As retardation is absence the front centreline remains unchanged

\sphinxAtStartPar
(D): The inclusion of retardation (\(R\)) with advection and dispersion leads to removal of chemicals from water and as well the retarded movement of the chemical front.

\sphinxAtStartPar
(E): The inclusion of retardation along with degradation and conservative transport process leads to high removal of chemical from water.


\section{Breakthrough Curve}
\label{\detokenize{content/transport/L10/22_reactive_transport:breakthrough-curve}}
\sphinxAtStartPar
Breakthrough curves provide a \sphinxstyleemphasis{time\sphinxhyphen{}dependent} spread of chemicals in the groundwater. The inclusion of multiple processes are normally solved using numerical models. Analytical models are available for limited processes and simplified problems. A 1\sphinxhyphen{}D analytical solution by Kinzelbach (1987) provide a transient (time\sphinxhyphen{}dependent) solution of reactive transport problem with inclusion of equilibrium linear sorption represented by retardation \((R)\), first\sphinxhyphen{}order degradation rate \((\lambda)\) and the conservative transport quantities \sphinxhyphen{} dispersion \((D)\) and advection. The solution is given as:
\begin{equation*}
\begin{split}
C(x,t) = C_0 \cdot \exp(-\lambda\cdot t)\bigg(1- \frac{1}{2}\text{erfc}\bigg(\frac{R\cdot x - v\cdot t}{2\cdot\sqrt{D\cdot R \cdot t}}\bigg) - \frac{1}{2}\exp\bigg(\frac{v\cdot x}{D}\bigg)\text{erfc}\bigg(\frac{R\cdot x + v\cdot t}{2\cdot\sqrt{D\cdot R \cdot t}}\bigg) 
\end{split}
\end{equation*}
\sphinxAtStartPar
with \(C_0\) = input/source concentration {[}ML\(^{-3}\){]} 
\(t\) = time {[}T{]}
\(v\) = groundwater flow velocity {[}LT\(^{-1}\){]}
erfc() = represents the complementary error function \sphinxhref{https://en.wikipedia.org/wiki/Error\_function}{See here for details}. erfc() can be easily computed using Python Scipy special function library.

\begin{sphinxuseclass}{cell}
\begin{sphinxuseclass}{tag_hide-input}
\end{sphinxuseclass}
\end{sphinxuseclass}
\begin{sphinxuseclass}{cell}
\begin{sphinxuseclass}{tag_hide-input}\begin{sphinxVerbatimOutput}

\begin{sphinxuseclass}{cell_output}
\begin{sphinxVerbatim}[commandchars=\\\{\}]
Text(5, 0.2, \PYGZsq{}\PYGZdl{}x= 20\PYGZdl{} m\PYGZsq{})
\end{sphinxVerbatim}

\noindent\sphinxincludegraphics{{C:/Users/vibhu/GWtextbook/_build/jupyter_execute/22_reactive_transport_29_1}.png}

\end{sphinxuseclass}\end{sphinxVerbatimOutput}

\end{sphinxuseclass}
\end{sphinxuseclass}

\section{Mass (Re\sphinxhyphen{})Distribution During Injection / Extraction}
\label{\detokenize{content/transport/L10/22_reactive_transport:mass-re-distribution-during-injection-extraction}}
\sphinxAtStartPar
\sphinxstylestrong{Consider a scenario}:

\sphinxAtStartPar
Water is \sphinxstyleemphasis{injected} into a certain portion of an aquifer with total volume \(V\), bulk density \(\rho_b\) and effective porosity \(n_e\). Assume that the injected water contains a chemical of total mass \(M\), which is adsorbed by the aquifer materials under equilibrium conditions according to Henry isotherm (quantified by \(K_d\)\$).

\sphinxAtStartPar
Bases on the assumption of sorption equilibrium, the total mass \(M\) of the chemical is instantaneously(!) split up into a dissolved and a sorbed part. In such case, the mass distribution can be computed as follows (with \(R\) = retardation factor, \(\rho_b\)
= bulk density):

\sphinxAtStartPar
\textbackslash{}begin\{align\}
M \&= n\_e \textbackslash{}cdot V \textbackslash{}cdot C + V\textbackslash{}cdot\textbackslash{}rho\_b \textbackslash{}cdot C\_a \textbackslash{}
\&= n\_e \textbackslash{}cdot V \textbackslash{}cdot C + V \textbackslash{}cdot \textbackslash{}rho\_b \textbackslash{}cdot K\_d \textbackslash{}cdot C\textbackslash{}
\&= n\_e \textbackslash{}cdot (1 + \textbackslash{}rho\_b \textbackslash{}cdot K\_d/n\_e) \textbackslash{}cdot V \textbackslash{}cdot C\textbackslash{}
\&= n\_e \textbackslash{}cdot R \textbackslash{}cdot V \textbackslash{}cdot C
\textbackslash{}end\{align\}

\sphinxAtStartPar
In which,

\sphinxAtStartPar
\(n_e \cdot V \cdot C\) = dissolved mass 

\sphinxAtStartPar
\(V\cdot\rho_b \cdot C_a\) = mass of adsorbate

\sphinxAtStartPar
For the dissolved mass we thus have \(n_e \cdot V \cdot C = M/R\) and consequently the mass of adsorbate is:
\(V\cdot\rho_b \cdot C_a = M- M/R = (1-1/R)\cdot M\)

\sphinxAtStartPar
The same approach can be adopted for the \sphinxstylestrong{extraction} scenarios, i.e. equilibrium desorption.


\section{Additional Tool}
\label{\detokenize{content/transport/L10/22_reactive_transport:additional-tool}}
\sphinxAtStartPar
The additional tool: \DUrole{xref,myst}{1D\sphinxhyphen{}Advection\sphinxhyphen{}Dispersion Simulation Tool} simulates all the concepts that are provided above. The tool simulates:
\begin{itemize}
\item {} 
\sphinxAtStartPar
1D solute transport in porous media (e.g., laboratory column)

\item {} 
\sphinxAtStartPar
uses unifrom cross\sphinxhyphen{}section

\item {} 
\sphinxAtStartPar
steady\sphinxhyphen{}state water flow

\item {} 
\sphinxAtStartPar
input of tracer

\end{itemize}

\sphinxAtStartPar
The output are then:
\begin{itemize}
\item {} 
\sphinxAtStartPar
spreading of tracer due to advection and mechanical dispersion

\item {} 
\sphinxAtStartPar
computation and graphical representation of a breakthrough curve

\item {} 
\sphinxAtStartPar
comparison with measured data.

\end{itemize}


\section{Chapter Quiz}
\label{\detokenize{content/transport/L10/22_reactive_transport:chapter-quiz}}
\begin{sphinxuseclass}{cell}
\begin{sphinxuseclass}{tag_remove-input}
\begin{sphinxuseclass}{tag_hide-output}
\end{sphinxuseclass}
\end{sphinxuseclass}
\end{sphinxuseclass}
\sphinxstepscope


\part{Modeling}

\sphinxstepscope

\begin{sphinxuseclass}{cell}
\begin{sphinxuseclass}{tag_hide-input}
\begin{sphinxuseclass}{tag_remove-input}
\end{sphinxuseclass}
\end{sphinxuseclass}
\end{sphinxuseclass}

\chapter{Lecture 11:  Background of Groundwater Modeling}
\label{\detokenize{content/modeling/31_intro_modeling:lecture-11-background-of-groundwater-modeling}}\label{\detokenize{content/modeling/31_intro_modeling::doc}}
\sphinxAtStartPar
\sphinxstyleemphasis{(The contents presented in this section were re\sphinxhyphen{}developed principally by Dr. P. K. Yadav. The original contents are from Prof. Rudolf Liedl)}


\bigskip\hrule\bigskip



\section{Motivation}
\label{\detokenize{content/modeling/31_intro_modeling:motivation}}
\sphinxAtStartPar
This lecture introduces the realm of mathematical modeling realm in groundwater studies. In the previous lectures the fundamental quantities, their properties and approach to quantify them were discussed. Those information were then used to develop system equations for varieties of groundwater problems. It was discussed then that these system equations require mathematical approaches the theory for which have to be systematically discussed and understood. Groundwater modeling can then be described as the systematic use of mathematical approaches leading to solution of the groundwater problem.

\sphinxAtStartPar
Groundwater modeling is often the first step towards understanding and solving groundwater problems/issues. Groundwater modeling is a very broad topic, this and the following lectures only introduces fundamental part of groundwater modeling. In this course we focus on groundwater flow problems.


\section{Introduction}
\label{\detokenize{content/modeling/31_intro_modeling:introduction}}

\subsection{What is a Model?}
\label{\detokenize{content/modeling/31_intro_modeling:what-is-a-model}}
\sphinxAtStartPar
Very succinctly a  \sphinxstylestrong{model} is a representation, which may be an image or a description of a real system. The description can be of different form (e.g., scales), of different level of detail (e.g., conceptual versus mathematical). The \sphinxstylestrong{system} to be modeled can be a \sphinxstylestrong{real} or also \sphinxstylestrong{conceptual}. A very relevant example of a real system for this course is the Darcy’s experiment (see figure below), in which water is made  to flow through the porous media.

\begin{figure}[htbp]
\centering
\capstart

\noindent\sphinxincludegraphics[scale=0.6]{{M11_f1}.png}
\caption{Darcy’s experimental setup\sphinxfootnotemark[1]}\label{\detokenize{content/modeling/31_intro_modeling:darcy}}\end{figure}
%
\begin{footnotetext}[1]\phantomsection\label{\thesphinxscope.1}%
\sphinxAtStartFootnote
Darcy, H., Les Fontaines Publiques de la Ville de Dijon, Dalmont, Paris, 1856.
%
\end{footnotetext}\ignorespaces 
\sphinxAtStartPar
With Darcy’s experiment, one could set\sphinxhyphen{}up a mathematical model to relate flow rate and hydraulic gradient (\(h\)). As model is \sphinxstyleemphasis{only an image} of the real system, several assumptions have to be made in it’s development. At many instances the model can not be set without these assumptions. In other cases solution of the model may not be possible without these assumptions being part of the model development. Darcy‘s Law, for instance, does not provide an exact representation of flow through individual pore channels. Rather, \sphinxstylestrong{average} flow behaviour through \sphinxstylestrong{many} pore channels is represented.


\subsection{Model Types: Process\sphinxhyphen{}Based and Empirical Models}
\label{\detokenize{content/modeling/31_intro_modeling:model-types-process-based-and-empirical-models}}
\sphinxAtStartPar
Model can be classified in many ways. The following two types of classification are a more general way to classify models:
\begin{enumerate}
\sphinxsetlistlabels{\arabic}{enumi}{enumii}{}{.}%
\item {} 
\sphinxAtStartPar
\sphinxstylestrong{Conceptual models}

\end{enumerate}
\begin{enumerate}
\sphinxsetlistlabels{\arabic}{enumi}{enumii}{}{.}%
\item {} 
\sphinxAtStartPar
\sphinxstylestrong{Process\sphinxhyphen{}Based and Empirical Models}

\end{enumerate}
\begin{enumerate}
\sphinxsetlistlabels{\arabic}{enumi}{enumii}{}{.}%
\item {} 
\sphinxAtStartPar
\sphinxstylestrong{Mathematical models}

\end{enumerate}


\subsubsection{Conceptual models}
\label{\detokenize{content/modeling/31_intro_modeling:conceptual-models}}
\sphinxAtStartPar
The \sphinxstylestrong{conceptual models} is classification of models that distinguishes the qualitative from the quantitative description of a real system. A \sphinxstylestrong{conceptual model} provides a qualitative representation of the relevant system components, processes, and impacts in the area of investigation. This representation is usually shown graphically, e.g., as block models (see figure {\hyperref[\detokenize{content/modeling/31_intro_modeling:cmodel}]{\sphinxcrossref{\DUrole{std,std-ref}{A conceptual model showing different components of a hydrological system that can impact groundwater.}}}}). It will be shown later that conceptual model in fact is the first block in the development of a mathematical model.

\begin{figure}[htbp]
\centering
\capstart

\noindent\sphinxincludegraphics[scale=0.2]{{M11_f3}.png}
\caption{A conceptual model showing different components of a hydrological system that can impact groundwater.}\label{\detokenize{content/modeling/31_intro_modeling:cmodel}}\end{figure}


\subsubsection{Physically based models and Empirical models}
\label{\detokenize{content/modeling/31_intro_modeling:physically-based-models-and-empirical-models}}
\sphinxAtStartPar
The \sphinxstylestrong{physically based models} also referred to as \sphinxstylestrong{process\sphinxhyphen{}based} models are models that exclusively relies on the fundamental physical laws \sphinxhyphen{} e.g., law of conservation of mass, energy, volume. \sphinxstyleemphasis{Compartment}  based models are example of physically based models.

\sphinxAtStartPar
Contrary to use of fundamental physical laws, \sphinxstylestrong{empirical models} are developed on the basis on experimental/collected data. Sorption isotherms that were developed in lecture \DUrole{xref,myst}{(10)} are types of emperical models. As was with the isotherms, these types of models are often based on regression analysis. Figure below show a prediction (blue) line as a predictor of data.

\begin{sphinxuseclass}{cell}
\begin{sphinxuseclass}{tag_hide-input}\begin{sphinxVerbatimOutput}

\begin{sphinxuseclass}{cell_output}
\end{sphinxuseclass}\end{sphinxVerbatimOutput}

\end{sphinxuseclass}
\end{sphinxuseclass}
\begin{figure}[htbp]
\centering
\capstart
\begin{sphinxVerbatimOutput}

\begin{sphinxuseclass}{cell_output}
\noindent\sphinxincludegraphics{{C:/Users/vibhu/GWtextbook/_build/jupyter_execute/31_intro_modeling_4_0}.png}

\end{sphinxuseclass}\end{sphinxVerbatimOutput}
\caption{Empirical relation between hydraulic head and discharge.}\label{\detokenize{content/modeling/31_intro_modeling:empirical}}\end{figure}

\sphinxAtStartPar
In certain cases \sphinxstylestrong{hybrid models}, e.g. \sphinxstylestrong{semi\sphinxhyphen{}empirical} models can also occur. These models combines the components of empirical/numerical and analytical models. Darcy’s law in fact is a semi\sphinxhyphen{}empirical model. On the one hand, it is based on the momentum conservation.  On the other hand, it is not possible to \sphinxstyleemphasis{strictly} deduce the direct proportionality between flow rate and hydraulic gradient by averaging the flow behaviour over all pores.


\subsubsection{Mathematical models}
\label{\detokenize{content/modeling/31_intro_modeling:mathematical-models}}
\sphinxAtStartPar
A \sphinxstylestrong{mathematical model} provides a quantitative representation of the relevant system components (described by for e.g., conceptual model),processes (e.g., described by physical model) and impacts in the area of investigation. The quantitative representation is based on mathematical equations. The system equations, that were developed and discussed in Lecture \DUrole{xref,myst}{(7)} are mathematical models.

\sphinxAtStartPar
System equation or mathematical models can in certain cases be solved directly resulting to an \sphinxstyleemphasis{exact solution} called \sphinxstylestrong{analytical solution}. Theis equation that was developed in Lecture \DUrole{xref,myst}{(8)} to quantify aquifer drawdown resulting from pumping of groundwater is an example of an analytical solution.

\sphinxAtStartPar
For more complex problems, often a more natural groundwater conditions, only \sphinxstyleemphasis{approximate} (or non\sphinxhyphen{}exact) solution called \sphinxstylestrong{numerical solution} can be  obtained. Numerical solutions are obtained after converting the system equation to so called \sphinxstylestrong{numerical models}.

\sphinxAtStartPar
Our focus in this introductory modeling lecture is to understand the development and solution of \sphinxstyleemphasis{numerical model} of simple groundwater flow problems.


\subsection{Example of an Analytical Solution}
\label{\detokenize{content/modeling/31_intro_modeling:example-of-an-analytical-solution}}
\sphinxAtStartPar
Consider a conceptual model presented in the Figure {\hyperref[\detokenize{content/tools/1D_ditchflow:ditch}]{\sphinxcrossref{\DUrole{std,std-ref}{Conceptual model of a flow between two water bodies separated by unconfined aquifer}}}}. The unconfined aquifer separates two surface exposed water bodies. The water body in the left has a higher hydraulic head (\(h_0\), {[}L{]}) compared to that on the right (\(h_u\)). Thus the flow of water is from left water body to the right one along the separating aquifer. In this scenario one of the problem to address will be to understand how the aquifer reacts to change in heads of water bodies. Additionally, how additional water, e.g., from precipitation/recharge (\(N\), {[}L/T{]}), will effect the aquifer water.

\begin{figure}[htbp]
\centering
\capstart

\noindent\sphinxincludegraphics[scale=0.2]{{M11_f4}.png}
\caption{Conceptual model of a flow between two water bodies separated by unconfined aquifer}\label{\detokenize{content/modeling/31_intro_modeling:ditch}}\end{figure}

\sphinxAtStartPar
The conceptual problem can be addressed when assumptions such as steady condition prevails, Darcy’s law in aquifer is valid, recharge rate are relatively low. Based on the these assumptions, one of the \sphinxstylestrong{mathematical model} of this conceptual problem is:
\begin{equation*}
\begin{split}
\frac{\textrm{d}}{\textrm{d}x}\bigg(-h \cdot K\cdot \frac{\textrm{d}h}{\textrm{d}x}  \bigg) = N
\end{split}
\end{equation*}
\sphinxAtStartPar
and the 2 boundary conditions:
\begin{equation*}
\begin{split}
h(0) = h_0 \:\: \text{and} \:\: h(L) = h_L
\end{split}
\end{equation*}
\sphinxAtStartPar
Note that a complete formulation of mathematical model requires accompanying boundary conditions. The boundary conditions are used to uniquely define the problem. An \sphinxstyleemphasis{analytical solution} for this mathematical model and accompanying boundary condition is:
\begin{equation*}
\begin{split}
h(x) = \sqrt{h_o^2 - (h_o^2 - h_L^2)\cdot \frac{x}{L} + \frac{N}{K}\cdot x \cdot (L-x) }
\end{split}
\end{equation*}
\sphinxAtStartPar
The solution can be used to quantify change in aquifer head \(h\) with different system quantities, e.g., conductivity \(K\) {[}L/T{]}, recharge \(N\) {[}L/T{]} water body heights \(h_0,\, h_L\) {[}L{]} along the flow direction.

\sphinxAtStartPar
The additional tool: \sphinxstyleemphasis{Conservative Transport} (\DUrole{xref,myst}{TOOLS}) interactively simulates the ditch flow concept in more details.


\subsection{Example problem}
\label{\detokenize{content/modeling/31_intro_modeling:example-problem}}
\begin{sphinxadmonition}{note}{Ditch flow}

\sphinxAtStartPar
Explore the effect of recharge (\(N= 0\) and \(N= 0.1\) mm/d) on the aquifer level for the conceptual problem provided above. Other required data are provided below.
The effect are to be explored at mid of the aquifer
\end{sphinxadmonition}

\begin{sphinxuseclass}{cell}\begin{sphinxVerbatimInput}

\begin{sphinxuseclass}{cell_input}
\begin{sphinxVerbatim}[commandchars=\\\{\}]
 \PYG{n+nb}{print}\PYG{p}{(}\PYG{l+s+s2}{\PYGZdq{}}\PYG{l+s+s2}{Provided are:}\PYG{l+s+se}{\PYGZbs{}n}\PYG{l+s+s2}{\PYGZdq{}}\PYG{p}{)}

\PYG{n}{K} \PYG{o}{=} \PYG{l+m+mf}{2E\PYGZhy{}4} \PYG{c+c1}{\PYGZsh{} hydraulic conductivity [m/s]}
\PYG{n}{Ho} \PYG{o}{=} \PYG{l+m+mi}{10} \PYG{c+c1}{\PYGZsh{} head at the origin [m]}
\PYG{n}{Hu} \PYG{o}{=} \PYG{l+m+mf}{7.5} \PYG{c+c1}{\PYGZsh{} head at L [m]}
\PYG{n}{L} \PYG{o}{=} \PYG{l+m+mi}{175} \PYG{c+c1}{\PYGZsh{}flow length [m]}
\PYG{n}{N1} \PYG{o}{=} \PYG{l+m+mi}{0} \PYG{c+c1}{\PYGZsh{} no recharge [m/s]}
\PYG{n}{N2} \PYG{o}{=} \PYG{l+m+mi}{1000} \PYG{c+c1}{\PYGZsh{} recharge [mm/a]}

\PYG{c+c1}{\PYGZsh{} intermediate calculation }
\PYG{n}{x} \PYG{o}{=} \PYG{n}{L}\PYG{o}{/}\PYG{l+m+mi}{2} \PYG{c+c1}{\PYGZsh{} mid of the aquifer [m]}
\PYG{n}{N\PYGZus{}} \PYG{o}{=} \PYG{n}{N2}\PYG{o}{/}\PYG{l+m+mi}{1000}\PYG{o}{/}\PYG{l+m+mi}{365}\PYG{o}{/}\PYG{l+m+mi}{86400} \PYG{c+c1}{\PYGZsh{} recharge, [m/s]}
 
\PYG{c+c1}{\PYGZsh{}solution}
\PYG{n}{h1}\PYG{o}{=}\PYG{p}{(}\PYG{n}{Ho}\PYG{o}{*}\PYG{o}{*}\PYG{l+m+mi}{2}\PYG{o}{\PYGZhy{}}\PYG{p}{(}\PYG{n}{Ho}\PYG{o}{*}\PYG{o}{*}\PYG{l+m+mi}{2}\PYG{o}{\PYGZhy{}}\PYG{n}{Hu}\PYG{o}{*}\PYG{o}{*}\PYG{l+m+mi}{2}\PYG{p}{)}\PYG{o}{/}\PYG{n}{L}\PYG{o}{*}\PYG{n}{x}\PYG{o}{+}\PYG{p}{(}\PYG{n}{N1}\PYG{o}{/}\PYG{n}{K}\PYG{o}{*}\PYG{n}{x}\PYG{o}{*}\PYG{p}{(}\PYG{n}{L}\PYG{o}{\PYGZhy{}}\PYG{n}{x}\PYG{p}{)}\PYG{p}{)}\PYG{p}{)}\PYG{o}{*}\PYG{o}{*}\PYG{l+m+mf}{0.5}
\PYG{n}{h2}\PYG{o}{=}\PYG{p}{(}\PYG{n}{Ho}\PYG{o}{*}\PYG{o}{*}\PYG{l+m+mi}{2}\PYG{o}{\PYGZhy{}}\PYG{p}{(}\PYG{n}{Ho}\PYG{o}{*}\PYG{o}{*}\PYG{l+m+mi}{2}\PYG{o}{\PYGZhy{}}\PYG{n}{Hu}\PYG{o}{*}\PYG{o}{*}\PYG{l+m+mi}{2}\PYG{p}{)}\PYG{o}{/}\PYG{n}{L}\PYG{o}{*}\PYG{n}{x}\PYG{o}{+}\PYG{p}{(}\PYG{n}{N\PYGZus{}}\PYG{o}{/}\PYG{n}{K}\PYG{o}{*}\PYG{n}{x}\PYG{o}{*}\PYG{p}{(}\PYG{n}{L}\PYG{o}{\PYGZhy{}}\PYG{n}{x}\PYG{p}{)}\PYG{p}{)}\PYG{p}{)}\PYG{o}{*}\PYG{o}{*}\PYG{l+m+mf}{0.5}

\PYG{n+nb}{print}\PYG{p}{(}\PYG{l+s+s2}{\PYGZdq{}}\PYG{l+s+s2}{hydraulic conductivity = }\PYG{l+s+si}{\PYGZob{}\PYGZcb{}}\PYG{l+s+s2}{ m}\PYG{l+s+se}{\PYGZbs{}n}\PYG{l+s+s2}{head at origin = }\PYG{l+s+si}{\PYGZob{}\PYGZcb{}}\PYG{l+s+s2}{ m}\PYG{l+s+se}{\PYGZbs{}n}\PYG{l+s+s2}{head at L = }\PYG{l+s+si}{\PYGZob{}\PYGZcb{}}\PYG{l+s+s2}{ m}\PYG{l+s+se}{\PYGZbs{}n}\PYG{l+s+s2}{flow length = }\PYG{l+s+si}{\PYGZob{}\PYGZcb{}}\PYG{l+s+s2}{ m}\PYG{l+s+se}{\PYGZbs{}n}\PYG{l+s+s2}{Recharge = }\PYG{l+s+si}{\PYGZob{}\PYGZcb{}}\PYG{l+s+s2}{ mm/a}\PYG{l+s+s2}{\PYGZdq{}}\PYG{o}{.}\PYG{n}{format}\PYG{p}{(}\PYG{n}{K}\PYG{p}{,} \PYG{n}{Ho}\PYG{p}{,} \PYG{n}{Hu}\PYG{p}{,} \PYG{n}{L}\PYG{p}{,} \PYG{n}{N2} \PYG{p}{)}\PYG{p}{,}\PYG{l+s+s2}{\PYGZdq{}}\PYG{l+s+se}{\PYGZbs{}n}\PYG{l+s+s2}{\PYGZdq{}}\PYG{p}{)}
\PYG{n+nb}{print}\PYG{p}{(}\PYG{l+s+s2}{\PYGZdq{}}\PYG{l+s+s2}{The resulting head without head is }\PYG{l+s+si}{\PYGZob{}:0.2f\PYGZcb{}}\PYG{l+s+s2}{ m }\PYG{l+s+se}{\PYGZbs{}n}\PYG{l+s+s2}{\PYGZdq{}}\PYG{o}{.}\PYG{n}{format}\PYG{p}{(}\PYG{n}{h1}\PYG{p}{)}\PYG{p}{)}
\PYG{n+nb}{print}\PYG{p}{(}\PYG{l+s+s2}{\PYGZdq{}}\PYG{l+s+s2}{The resulting head with head is }\PYG{l+s+si}{\PYGZob{}:0.2f\PYGZcb{}}\PYG{l+s+s2}{ m }\PYG{l+s+se}{\PYGZbs{}n}\PYG{l+s+s2}{\PYGZdq{}}\PYG{o}{.}\PYG{n}{format}\PYG{p}{(}\PYG{n}{h2}\PYG{p}{)}\PYG{p}{)}
\end{sphinxVerbatim}

\end{sphinxuseclass}\end{sphinxVerbatimInput}
\begin{sphinxVerbatimOutput}

\begin{sphinxuseclass}{cell_output}
\begin{sphinxVerbatim}[commandchars=\\\{\}]
Provided are:

hydraulic conductivity = 0.0002 m
head at origin = 10 m
head at L = 7.5 m
flow length = 175 m
Recharge = 1000 mm/a 

The resulting head without head is 8.84 m 

The resulting head with head is 8.91 m 
\end{sphinxVerbatim}

\end{sphinxuseclass}\end{sphinxVerbatimOutput}

\end{sphinxuseclass}

\subsection{Example for a Model without Analytical Solution}
\label{\detokenize{content/modeling/31_intro_modeling:example-for-a-model-without-analytical-solution}}
\sphinxAtStartPar
Analytical solutions are rather an exception. Natural aquifer or groundwater system are more complex (see figure below) and therefore analytical solution are not possible. The complexity in natural system are due to parameter heterogeneity and irregular model domain boundaries, and these must be included in underlying model equation.

\begin{figure}[htbp]
\centering
\capstart

\noindent\sphinxincludegraphics[scale=0.6]{{M11_f5}.png}
\caption{The numerical model of a natural aquifer}\label{\detokenize{content/modeling/31_intro_modeling:nummodel}}\end{figure}


\section{Conceptual Model to Numerical Approach}
\label{\detokenize{content/modeling/31_intro_modeling:conceptual-model-to-numerical-approach}}
\sphinxAtStartPar
The first step in modeling is to establish the purpose of the model. With that established, the development of conceptual model begins the set\sphinxhyphen{}up of the numerical model. This is a step\sphinxhyphen{}wise process that includes:
\begin{enumerate}
\sphinxsetlistlabels{\arabic}{enumi}{enumii}{}{.}%
\item {} 
\sphinxAtStartPar
\sphinxstylestrong{Conceptual model} \sphinxhyphen{} providing hydrogeological units within the model domain

\end{enumerate}
\begin{enumerate}
\sphinxsetlistlabels{\arabic}{enumi}{enumii}{}{.}%
\item {} 
\sphinxAtStartPar
\sphinxstylestrong{Water budgeting} \sphinxhyphen{} identifying water containing units and characterizing it

\end{enumerate}
\begin{enumerate}
\sphinxsetlistlabels{\arabic}{enumi}{enumii}{}{.}%
\item {} 
\sphinxAtStartPar
\sphinxstylestrong{Numerical model} \sphinxhyphen{} Combining the units of conceptual model, water budgeting components and imposing \sphinxstyleemphasis{boundary conditions.}

\end{enumerate}

\begin{figure}[htbp]
\centering
\capstart

\noindent\sphinxincludegraphics[scale=0.6]{{M11_f6}.png}
\caption{The numerical model of a natural aquifer}\label{\detokenize{content/modeling/31_intro_modeling:id2}}\end{figure}

\sphinxAtStartPar
The water budgeting part should include the sources of water inflowing to the system including the expected flow directions and exiting water. Estimate of Groundwater recharge, overland flow etc, are few examples of inflows. Likewise, estimate of baseflow to streams, evapotranspiration, abstraction wells etc. are outflows. Field data are required to prepare water budget.

\sphinxAtStartPar
The type of numerical model, depending on the modeling goal and required/available data, is then decided. The model type is mostly time\sphinxhyphen{}related (transient, steady) and dimension\sphinxhyphen{}related (1D,2D, 3D).


\subsection{Data Requirements}
\label{\detokenize{content/modeling/31_intro_modeling:data-requirements}}
\sphinxAtStartPar
Numerical model are often very data intensive. Several types of data of different origin are required in the development of a numerical model. These data come from site data records/maps, field works, lab works and the ancillary mathematical analysis of the field and lab data. Overall, the required data can be:
\begin{itemize}
\item {} 
\sphinxAtStartPar
topographical maps (with surface waters and water divides)

\item {} 
\sphinxAtStartPar
geological maps, geological profiles (see figure \DUrole{xref,std,std-ref}{nummodel} example)

\item {} 
\sphinxAtStartPar
maps with isolines of aquifer bottoms / thicknesses , aquitard bottoms / thicknesses

\item {} 
\sphinxAtStartPar
maps indicating vertical extensions of sediments under rivers and lakes

\item {} 
\sphinxAtStartPar
hydrogeological maps (hydraulic head isolines)

\item {} 
\sphinxAtStartPar
water level time series in observation wells and rivers

\item {} 
\sphinxAtStartPar
time series of spring discharges

\item {} 
\sphinxAtStartPar
maps and profiles of hydraulic conductivity or transmissivity (also for river / lake sediments mentioned above)

\item {} 
\sphinxAtStartPar
maps and profiles of storage coefficients

\item {} 
\sphinxAtStartPar
information on spatial and temporal variability of inflow / outflow due to
\begin{itemize}
\item {} 
\sphinxAtStartPar
groundwater recharge

\item {} 
\sphinxAtStartPar
evapotranspiration

\item {} 
\sphinxAtStartPar
interaction between groundwater and surface water

\item {} 
\sphinxAtStartPar
groundwater abstraction

\item {} 
\sphinxAtStartPar
natural groundwater flow

\end{itemize}

\end{itemize}


\section{Example of a Groundwater Model}
\label{\detokenize{content/modeling/31_intro_modeling:example-of-a-groundwater-model}}
\sphinxAtStartPar
Let us now develop an example flow model. To begin with we set\sphinxhyphen{}up an overly simplified model that still will require a numerical solution.

\sphinxAtStartPar
\sphinxstylestrong{Step 1} \sphinxhyphen{} Spatial Extension
\begin{itemize}
\item {} 
\sphinxAtStartPar
Horizontal extension along \(x-\)direction: 4000 m

\item {} 
\sphinxAtStartPar
Horizontal extension along \(y-\)direction: 2500 m

\item {} 
\sphinxAtStartPar
vertical extension along \(z\)\sphinxhyphen{}direction: from \(z = 250\) m a.s.l. at the aquifer bottom to \(z= 265\) m at the aquifer top.

\item {} 
\sphinxAtStartPar
aquifer thickness is uniform = 15 m.

\end{itemize}

\sphinxAtStartPar
Putting these information graphically, we get the following schematic:

\begin{figure}[htbp]
\centering
\capstart

\noindent\sphinxincludegraphics[scale=0.4]{{M11_f7}.png}
\caption{Spatial extension of an example numerical model}\label{\detokenize{content/modeling/31_intro_modeling:nummodel-ex}}\end{figure}

\sphinxAtStartPar
The example model is 2D and we \sphinxstyleemphasis{assume} that vertical flow components can be neglected despite the recharge vertically entering the groundwater. These kinds of simplification are quite common in the development of the numerical model. These simplifications have to be justified when presenting model results.

\sphinxAtStartPar
\sphinxstylestrong{Step 2} \sphinxhyphen{} Hydraulic Properties

\sphinxAtStartPar
For our example model we consider the following:
\begin{itemize}
\item {} 
\sphinxAtStartPar
effective porosity (\(\eta_e\)) in the model domain: 0.2 or 20\%

\item {} 
\sphinxAtStartPar
two zones with different hydraulic conductivities (\(K\))

\item {} 
\sphinxAtStartPar
two zones with different groundwater recharge

\item {} 
\sphinxAtStartPar
A section of a river (\sphinxstyleemphasis{river reach}) is in hydraulic contact with the aquifer. i.e. there may be water transfer from the river to the aquifer (\sphinxstyleemphasis{influent conditions}) or vice versa (\sphinxstyleemphasis{effluent conditions}).

\item {} 
\sphinxAtStartPar
inflow boundary with prescribed hydraulic heads

\item {} 
\sphinxAtStartPar
outflow boundary with prescribed hydraulic heads

\item {} 
\sphinxAtStartPar
two impermeable boundaries

\end{itemize}

\sphinxAtStartPar
\sphinxstylestrong{Step 3} \sphinxhyphen{} The model purpose and conceptual model

\sphinxAtStartPar
From our example model we intend to:
\begin{itemize}
\item {} 
\sphinxAtStartPar
Abstraction of groundwater through wells is planned in the area with an overall pumping rate of 7000 m\(^3\)/d.

\item {} 
\sphinxAtStartPar
Water extraction is to be distributed between two wells located at \((x,y) =\) (3050 m, 1550 m) and \((x,y)\) = (3050 m, 1450 m), resp.

\item {} 
\sphinxAtStartPar
The model purpose is to outline the 50\sphinxhyphen{}days isochrone for both wells.

\end{itemize}

\sphinxAtStartPar
With these information available, our conceptual model takes the following form:

\begin{figure}[htbp]
\centering
\capstart

\noindent\sphinxincludegraphics[scale=0.25]{{M11_f8}.png}
\caption{The conceptual model of the example model}\label{\detokenize{content/modeling/31_intro_modeling:concept-model}}\end{figure}

\sphinxAtStartPar
\sphinxstylestrong{Step 4} \sphinxhyphen{} The numerical approach

\sphinxAtStartPar
This is discussed in the next lecture.


\bigskip\hrule\bigskip


\sphinxstepscope


\part{Tutorials}

\sphinxstepscope


\chapter{Python Programming language}
\label{\detokenize{content/background/01_python:python-programming-language}}\label{\detokenize{content/background/01_python::doc}}

\section{What is it?}
\label{\detokenize{content/background/01_python:what-is-it}}
\sphinxAtStartPar
Python is an open\sphinxhyphen{}source interpreted, high\sphinxhyphen{}level, general\sphinxhyphen{}purpose programming language. This means:
\begin{itemize}
\item {} 
\sphinxAtStartPar
Interpreted/high\sphinxhyphen{}level language: This makes we avoid the nuances of fundamental coding as done by computer programmers/engineers.

\item {} 
\sphinxAtStartPar
General purpose programming language and \sphinxstyleemphasis{Open\sphinxhyphen{}source} ecosystem: This means it is extensible. Already over 200,000 Python packages are available (\sphinxhref{https://pypi.org/}{Check Here}). Also, it means being \sphinxstylestrong{free} of cost.

\item {} 
\sphinxAtStartPar
For Groundwater: We can use Python packages such as \sphinxstylestrong{Numpy} (for numerical computing), \sphinxstylestrong{Scipy} (for scientific computing), \sphinxstylestrong{Sympy} (for symbolic computing), \sphinxstylestrong{Matplotlib} (for plotting) etc. for our computing and modelling in the course.

\end{itemize}


\section{A bit of history of Python Programming language}
\label{\detokenize{content/background/01_python:a-bit-of-history-of-python-programming-language}}
\sphinxAtStartPar
Python is now over 15 years old programming language. Its development can be traced from:
\begin{itemize}
\item {} 
\sphinxAtStartPar
\sphinxstylestrong{Guido van Rossum} began developing Python in 1980 at Centrum Wiskunde \& Informatica (CWI), the Netherlands. Its implementation (\sphinxstyleemphasis{Python v.1}) was released in 1994.

\item {} 
\sphinxAtStartPar
\sphinxstyleemphasis{Python 2.0}, released in 2000 became one of the most used general purpose programming language. Python 2.0 is now being replaced by \sphinxstyleemphasis{Python 3.0} (from 2020).

\item {} 
\sphinxAtStartPar
\sphinxstyleemphasis{Python 3.0} will be used in our class. It is \sphinxstyleemphasis{not} 100\% compatible with earlier versions of \sphinxstyleemphasis{Python.}

\item {} 
\sphinxAtStartPar
\sphinxstyleemphasis{Python} name comes from the British comedy group Monty Python (Van Rossum enjoyed their show). The official \sphinxstyleemphasis{Python} documentation \sphinxhref{https://www.python.org/doc/}{(Check here)} also contains various references to Monty Python routines.

\end{itemize}


\section{Why use Python Programming Language}
\label{\detokenize{content/background/01_python:why-use-python-programming-language}}
\sphinxAtStartPar
Many reasons but to put a few points here:
\begin{itemize}
\item {} 
\sphinxAtStartPar
\sphinxstyleemphasis{Python} is a common tool among engineers, experts and researchers at universities and industry.

\item {} 
\sphinxAtStartPar
\sphinxstyleemphasis{Python} is system independent, therefore it is highly portable. Beside, it is a versatile (multi\sphinxhyphen{}purpose) language.

\item {} 
\sphinxAtStartPar
\sphinxstyleemphasis{Python} is incredibly flexible and can be adapted to specific local needs using enormous number of \sphinxstylestrong{PACKAGES}. Beside, it can easily interface with other languages e.g., C++, Java.

\item {} 
\sphinxAtStartPar
\sphinxstyleemphasis{Python} is under incredibly active development, improving greatly, and supported wildly by both professional and academic developers.

\end{itemize}

\begin{sphinxuseclass}{cell}
\begin{sphinxuseclass}{tag_hide-input}\begin{sphinxVerbatimOutput}

\begin{sphinxuseclass}{cell_output}
\begin{sphinxVerbatim}[commandchars=\\\{\}]
Row
    [0] Column
        [0] Markdown(str)
        [1] PNG(str, width=500)
    [1] Markdown(str)
\end{sphinxVerbatim}

\end{sphinxuseclass}\end{sphinxVerbatimOutput}

\end{sphinxuseclass}
\end{sphinxuseclass}

\section{Very basics of Python Programming}
\label{\detokenize{content/background/01_python:very-basics-of-python-programming}}
\sphinxAtStartPar
\sphinxstyleemphasis{Python} is a very extensive language. To get started we learn the very fundamentals of the language.

\sphinxAtStartPar
\sphinxstylestrong{Fundamentals of Python Programming Language}

\noindent{\hspace*{\fill}\sphinxincludegraphics[width=600\sphinxpxdimen]{{bg1_f2}.png}\hspace*{\fill}}

\sphinxAtStartPar
\sphinxstylestrong{Data Types in Python}

\noindent{\hspace*{\fill}\sphinxincludegraphics[width=600\sphinxpxdimen]{{bg1_f3}.png}\hspace*{\fill}}

\sphinxAtStartPar
\sphinxstylestrong{Basic operators in Python}

\sphinxAtStartPar
Refer to \sphinxstyleemphasis{Python} documentation for complete description. Python documentation is very extensive and can be obtained from \sphinxhref{https://www.python.org/doc/}{here}

\noindent{\hspace*{\fill}\sphinxincludegraphics[width=600\sphinxpxdimen]{{bg1_f4}.png}\hspace*{\fill}}

\sphinxAtStartPar
\sphinxstylestrong{A FUNCTION in Python}

\sphinxAtStartPar
A \sphinxstyleemphasis{function} in a programming provide an ability to develop a reusable code\sphinxhyphen{}block with an option of several operations. This means, a function have input (or a set of input) and provide an (or a set of) output.

\noindent{\hspace*{\fill}\sphinxincludegraphics[width=600\sphinxpxdimen]{{bg1_f5}.png}\hspace*{\fill}}

\sphinxAtStartPar
Semicolon (:) in line 1 and \sphinxstyleemphasis{Indentation} after line 1 are required. \sphinxstylestrong{def}, \sphinxstylestrong{return} are \sphinxstyleemphasis{Python} keywords. There are quite few of them.


\section{How much Python Programming should we know?}
\label{\detokenize{content/background/01_python:how-much-python-programming-should-we-know}}
\sphinxAtStartPar
This is probably the most important question. The \sphinxstylestrong{clear} answer at least for this course is that \sphinxstylestrong{no} programming has to be learned.
This course do not expect any pre\sphinxhyphen{}coding skills. This course is intended for Basic Groundwater teaching, and that is the focus.
But, how about learning groundwater by coding?

\sphinxAtStartPar
Eventually, the depth of programming to learn is an individual choice. This course considers programming as a tool to learn better.

\sphinxAtStartPar
In this course the \sphinxstyleemphasis{codes} can are written in a way so that it can be easily read. In addition, this interactive book will allow quite many of the \sphinxstyleemphasis{code} to be edited and executed in the book itself. For more advanced learning the popular notebook interface \sphinxstylestrong{JUPYTER} is to be used.

\sphinxAtStartPar
\sphinxstylestrong{JUPYTER} interface, on which interface this book is developed, is very briefly explained in the next section.

\sphinxstepscope


\chapter{JUPYTER Notebook Interface for Python}
\label{\detokenize{content/background/02_jupyter:jupyter-notebook-interface-for-python}}\label{\detokenize{content/background/02_jupyter::doc}}
\sphinxAtStartPar
As we mentioned earlier, this interactive book/course can be used without any programming experience. Code based calculations that is part of this book can be mostly executed in the book itself. The codes provided in the book, still with only limited skill programming skill, can be adopted for more illustrative use. This can be done in two ways. First approach, and a quick one, will be to use the web\sphinxhyphen{}based tool called \sphinxhref{https://mybinder.org/}{Binder Project}. The second approach will be to use the codes in the off\sphinxhyphen{}line systems. Common to both approach is very useful computing interface called \sphinxstylestrong{JUPYTER}.

\sphinxAtStartPar
Here we briefly learn about \sphinxstylestrong{JUPYTER} interface.

\sphinxAtStartPar
\sphinxstylestrong{JUPYTER} is a computing interface and has been in development since 2015. \sphinxstyleemphasis{JUPYTER} provide computing interface for several programming language, and thus its name is derived from:
\begin{itemize}
\item {} 
\sphinxAtStartPar
\sphinxstylestrong{JU} : Julia programming language

\item {} 
\sphinxAtStartPar
\sphinxstylestrong{PY} : Python programming language

\item {} 
\sphinxAtStartPar
\sphinxstylestrong{R}  : R programming language

\end{itemize}

\sphinxAtStartPar
More important aspects of \sphinxstylestrong{JUPYTER} computing interface:
\begin{itemize}
\item {} 
\sphinxAtStartPar
Browser\sphinxhyphen{}based tool: \sphinxhyphen{} i.e., should also run in smartphone/tabs in a

\item {} 
\sphinxAtStartPar
OPen\sphinxhyphen{}source: i.e., community based development and personalization

\item {} 
\sphinxAtStartPar
Active development: Very actively under development, especially from academic/research sector.

\end{itemize}


\section{How to use JUPYTER}
\label{\detokenize{content/background/02_jupyter:how-to-use-jupyter}}
\sphinxAtStartPar
JUPYTER has a block\sphinxhyphen{}based interface (called \sphinxstylestrong{CELL}). Each \sphinxstyleemphasis{Cell} is either an \sphinxstylestrong{input (In{[}1{]})} cell or corresponding an \sphinxstylestrong{output (Out{[}1{]})}  cell. The \sphinxstyleemphasis{input} and \sphinxstyleemphasis{output} cells can be interactively operated.

\noindent{\hspace*{\fill}\sphinxincludegraphics[width=600\sphinxpxdimen]{{bg2_f1}.png}\hspace*{\fill}}

\sphinxAtStartPar
The \sphinxstyleemphasis{cell\sphinxhyphen{}based} interface is very intuitive specially for learning as it can show combine for example the code or mathematical concept with a visualization, i.e., both the cause and effect can be observed immediately and dynamically.

\noindent{\hspace*{\fill}\sphinxincludegraphics[width=600\sphinxpxdimen]{{bg2_f2}.png}\hspace*{\fill}}

\sphinxAtStartPar
For more advanced use, JUPYTER interface can be used for developing codes in different programming languages other than \sphinxstyleemphasis{Python.}

\sphinxAtStartPar
The JUPYTER cheat\sphinxhyphen{}sheet (from \sphinxhref{https://datacamp-community-prod.s3.amazonaws.com/48093c40-5303-45f4-bbf9-0c96c0133c40}{here}{]} can be helpful to get quickly started.

\sphinxAtStartPar
Also, for better computing and learning, installing \sphinxstylestrong{JUPYTER} locally in personal system is encouraged. This can be done using the instructions provided \sphinxhref{https://jupyter.readthedocs.io/en/latest/install.html}{here}.

\sphinxstepscope

\begin{sphinxuseclass}{cell}
\begin{sphinxuseclass}{tag_remove-output}\begin{sphinxVerbatimInput}

\begin{sphinxuseclass}{cell_input}
\begin{sphinxVerbatim}[commandchars=\\\{\}]
\PYG{k+kn}{import} \PYG{n+nn}{numpy} \PYG{k}{as} \PYG{n+nn}{np}
\PYG{k+kn}{import} \PYG{n+nn}{matplotlib}\PYG{n+nn}{.}\PYG{n+nn}{pyplot} \PYG{k}{as} \PYG{n+nn}{plt}
\PYG{k+kn}{import} \PYG{n+nn}{pandas} \PYG{k}{as} \PYG{n+nn}{pd} 
\PYG{k+kn}{import} \PYG{n+nn}{panel} \PYG{k}{as} \PYG{n+nn}{pn}
\PYG{n}{pn}\PYG{o}{.}\PYG{n}{extension}\PYG{p}{(}\PYG{l+s+s2}{\PYGZdq{}}\PYG{l+s+s2}{katex}\PYG{l+s+s2}{\PYGZdq{}}\PYG{p}{)} 
\end{sphinxVerbatim}

\end{sphinxuseclass}\end{sphinxVerbatimInput}

\end{sphinxuseclass}
\end{sphinxuseclass}

\chapter{Tutorial 3 \sphinxhyphen{} Darcy Law and Conductivity}
\label{\detokenize{content/tutorials/T3/tutorial_03:tutorial-3-darcy-law-and-conductivity}}\label{\detokenize{content/tutorials/T3/tutorial_03::doc}}
\sphinxAtStartPar
\sphinxstyleemphasis{(The contents presented in this section were re\sphinxhyphen{}developed principally by Dr. P. K. Yadav. The original contents are from Prof. Rudolf Liedl)}
\begin{itemize}
\item {} 
\sphinxAtStartPar
\sphinxstylestrong{tutorial problems on Darcy’s law and intrinsic permeability}

\item {} 
\sphinxAtStartPar
\sphinxstylestrong{homework problems on Darcy’s law and intrinsic permeability}

\end{itemize}


\section{Tutorial Problems on}
\label{\detokenize{content/tutorials/T3/tutorial_03:tutorial-problems-on}}\begin{itemize}
\item {} 
\sphinxAtStartPar
Darcy’s Law and

\item {} 
\sphinxAtStartPar
Intrinsic Permeability

\end{itemize}


\section{Tutorial Problem 7: Flow Direction and Hydraulic Gradient}
\label{\detokenize{content/tutorials/T3/tutorial_03:tutorial-problem-7-flow-direction-and-hydraulic-gradient}}
\sphinxAtStartPar
Indicate the direction of flow shown in the figure in next slides, and determine the hydraulic gradient for a Darcy column with length L = 50 cm! (Figures not to scale.)


\subsection{Tutorial Problem 7 – Solution}
\label{\detokenize{content/tutorials/T3/tutorial_03:tutorial-problem-7-solution}}
\sphinxAtStartPar
\sphinxstylestrong{The relevant topic is covered in Lecture 04, slide 8}

\begin{sphinxuseclass}{cell}\begin{sphinxVerbatimInput}

\begin{sphinxuseclass}{cell_input}
\begin{sphinxVerbatim}[commandchars=\\\{\}]
\PYG{n}{png\PYGZus{}pane} \PYG{o}{=} \PYG{n}{pn}\PYG{o}{.}\PYG{n}{pane}\PYG{o}{.}\PYG{n}{PNG}\PYG{p}{(}\PYG{l+s+s2}{\PYGZdq{}}\PYG{l+s+s2}{images/T03\PYGZus{}TP7\PYGZus{}a1.png}\PYG{l+s+s2}{\PYGZdq{}}\PYG{p}{,} \PYG{n}{width}\PYG{o}{=}\PYG{l+m+mi}{600}\PYG{p}{)}
\PYG{n}{png\PYGZus{}pane}
\end{sphinxVerbatim}

\end{sphinxuseclass}\end{sphinxVerbatimInput}
\begin{sphinxVerbatimOutput}

\begin{sphinxuseclass}{cell_output}
\begin{sphinxVerbatim}[commandchars=\\\{\}]
PNG(str, width=600)
\end{sphinxVerbatim}

\end{sphinxuseclass}\end{sphinxVerbatimOutput}

\end{sphinxuseclass}
\begin{sphinxuseclass}{cell}\begin{sphinxVerbatimInput}

\begin{sphinxuseclass}{cell_input}
\begin{sphinxVerbatim}[commandchars=\\\{\}]
\PYG{c+c1}{\PYGZsh{} Image (a)}
\PYG{n}{L} \PYG{o}{=} \PYG{l+m+mf}{0.5} \PYG{c+c1}{\PYGZsh{} m, length of the column}
\PYG{n}{Ea\PYGZus{}hl} \PYG{o}{=} \PYG{l+m+mf}{0.2} \PYG{c+c1}{\PYGZsh{}, m, elevation head, left }
\PYG{n}{Pa\PYGZus{}hl} \PYG{o}{=} \PYG{l+m+mf}{0.1} \PYG{c+c1}{\PYGZsh{}, m  pressure head, left}
\PYG{n}{Ea\PYGZus{}hr} \PYG{o}{=} \PYG{l+m+mf}{0.2} \PYG{c+c1}{\PYGZsh{}, m, elevation head, right}
\PYG{n}{Pa\PYGZus{}hr} \PYG{o}{=} \PYG{l+m+mf}{0.3} \PYG{c+c1}{\PYGZsh{}, m  pressure head, right}
\PYG{n}{Ha\PYGZus{}hl} \PYG{o}{=} \PYG{n}{Ea\PYGZus{}hl} \PYG{o}{+} \PYG{n}{Pa\PYGZus{}hl} \PYG{c+c1}{\PYGZsh{} m, hydraulic head, left}
\PYG{n}{Ha\PYGZus{}hr} \PYG{o}{=} \PYG{n}{Ea\PYGZus{}hr} \PYG{o}{+} \PYG{n}{Pa\PYGZus{}hr} \PYG{c+c1}{\PYGZsh{} m, hydraulic head, right}
\PYG{n}{DH\PYGZus{}a} \PYG{o}{=} \PYG{n}{Ha\PYGZus{}hr} \PYG{o}{\PYGZhy{}} \PYG{n}{Ha\PYGZus{}hl}
\PYG{n}{H\PYGZus{}ga} \PYG{o}{=} \PYG{n}{DH\PYGZus{}a}\PYG{o}{/}\PYG{n}{L}\PYG{c+c1}{\PYGZsh{}, no unit, hydraulic gradient }

\PYG{n+nb}{print}\PYG{p}{(}\PYG{l+s+s2}{\PYGZdq{}}\PYG{l+s+s2}{Hydraulic head LEFT: }\PYG{l+s+si}{\PYGZob{}0:1.1f\PYGZcb{}}\PYG{l+s+s2}{\PYGZdq{}}\PYG{o}{.}\PYG{n}{format}\PYG{p}{(}\PYG{n}{Ha\PYGZus{}hl}\PYG{p}{)}\PYG{p}{,}\PYG{l+s+s2}{\PYGZdq{}}\PYG{l+s+s2}{m}\PYG{l+s+s2}{\PYGZdq{}}\PYG{p}{)}\PYG{p}{;} \PYG{n+nb}{print}\PYG{p}{(}\PYG{l+s+s2}{\PYGZdq{}}\PYG{l+s+s2}{Hydraulic head RIGHT:: }\PYG{l+s+si}{\PYGZob{}0:1.1f\PYGZcb{}}\PYG{l+s+s2}{\PYGZdq{}}\PYG{o}{.}\PYG{n}{format}\PYG{p}{(}\PYG{n}{Ha\PYGZus{}hr}\PYG{p}{)}\PYG{p}{,}\PYG{l+s+s2}{\PYGZdq{}}\PYG{l+s+s2}{m}\PYG{l+s+s2}{\PYGZdq{}}\PYG{p}{)} 
\PYG{n+nb}{print}\PYG{p}{(}\PYG{l+s+s2}{\PYGZdq{}}\PYG{l+s+s2}{Hydraulic Head Difference: }\PYG{l+s+si}{\PYGZob{}0:1.1f\PYGZcb{}}\PYG{l+s+s2}{\PYGZdq{}}\PYG{o}{.}\PYG{n}{format}\PYG{p}{(}\PYG{n}{DH\PYGZus{}a}\PYG{p}{)}\PYG{p}{,}\PYG{l+s+s2}{\PYGZdq{}}\PYG{l+s+s2}{m}\PYG{l+s+s2}{\PYGZdq{}}\PYG{p}{)}\PYG{p}{;}\PYG{n+nb}{print}\PYG{p}{(}\PYG{l+s+s2}{\PYGZdq{}}\PYG{l+s+s2}{Hydraulic gradient: }\PYG{l+s+si}{\PYGZob{}0:1.1f\PYGZcb{}}\PYG{l+s+s2}{\PYGZdq{}}\PYG{o}{.}\PYG{n}{format}\PYG{p}{(}\PYG{n}{H\PYGZus{}ga}\PYG{p}{)}\PYG{p}{)} 
\PYG{n}{png\PYGZus{}pane}\PYG{o}{.}\PYG{n}{object} \PYG{o}{=} \PYG{l+s+s2}{\PYGZdq{}}\PYG{l+s+s2}{images/T03\PYGZus{}TP7\PYGZus{}a2.png}\PYG{l+s+s2}{\PYGZdq{}} 
\end{sphinxVerbatim}

\end{sphinxuseclass}\end{sphinxVerbatimInput}
\begin{sphinxVerbatimOutput}

\begin{sphinxuseclass}{cell_output}
\begin{sphinxVerbatim}[commandchars=\\\{\}]
Hydraulic head LEFT: 0.3 m
Hydraulic head RIGHT:: 0.5 m
Hydraulic Head Difference: 0.2 m
Hydraulic gradient: 0.4
\end{sphinxVerbatim}

\end{sphinxuseclass}\end{sphinxVerbatimOutput}

\end{sphinxuseclass}
\begin{sphinxuseclass}{cell}\begin{sphinxVerbatimInput}

\begin{sphinxuseclass}{cell_input}
\begin{sphinxVerbatim}[commandchars=\\\{\}]
\PYG{n}{png\PYGZus{}pane2} \PYG{o}{=} \PYG{n}{pn}\PYG{o}{.}\PYG{n}{pane}\PYG{o}{.}\PYG{n}{PNG}\PYG{p}{(}\PYG{l+s+s2}{\PYGZdq{}}\PYG{l+s+s2}{images/T03\PYGZus{}TP7\PYGZus{}b1.png}\PYG{l+s+s2}{\PYGZdq{}}\PYG{p}{,} \PYG{n}{width}\PYG{o}{=}\PYG{l+m+mi}{500}\PYG{p}{)}
\PYG{n}{png\PYGZus{}pane2} 
\end{sphinxVerbatim}

\end{sphinxuseclass}\end{sphinxVerbatimInput}
\begin{sphinxVerbatimOutput}

\begin{sphinxuseclass}{cell_output}
\begin{sphinxVerbatim}[commandchars=\\\{\}]
PNG(str, width=500)
\end{sphinxVerbatim}

\end{sphinxuseclass}\end{sphinxVerbatimOutput}

\end{sphinxuseclass}
\begin{sphinxuseclass}{cell}\begin{sphinxVerbatimInput}

\begin{sphinxuseclass}{cell_input}
\begin{sphinxVerbatim}[commandchars=\\\{\}]
\PYG{c+c1}{\PYGZsh{} Image (b)}
\PYG{n}{L} \PYG{o}{=} \PYG{l+m+mf}{0.5} \PYG{c+c1}{\PYGZsh{} m, length of the column}
\PYG{n}{Eb\PYGZus{}hl} \PYG{o}{=} \PYG{l+m+mf}{0.2} \PYG{c+c1}{\PYGZsh{}, m, elevation head, left }
\PYG{n}{Pb\PYGZus{}hl} \PYG{o}{=} \PYG{l+m+mf}{0.1} \PYG{c+c1}{\PYGZsh{}, m  pressure head, left}
\PYG{n}{Eb\PYGZus{}hr} \PYG{o}{=} \PYG{l+m+mf}{0.05} \PYG{c+c1}{\PYGZsh{}, m, elevation head, right}
\PYG{n}{Pb\PYGZus{}hr} \PYG{o}{=} \PYG{l+m+mf}{1.3} \PYG{c+c1}{\PYGZsh{}, m  pressure head, right}
\PYG{n}{Hb\PYGZus{}hl} \PYG{o}{=} \PYG{n}{Eb\PYGZus{}hl} \PYG{o}{+} \PYG{n}{Pb\PYGZus{}hl} \PYG{c+c1}{\PYGZsh{} m, hydraulic head, left}
\PYG{n}{Hb\PYGZus{}hr} \PYG{o}{=} \PYG{n}{Eb\PYGZus{}hr} \PYG{o}{+} \PYG{n}{Pb\PYGZus{}hr} \PYG{c+c1}{\PYGZsh{} m, hydraulic head, right}
\PYG{n}{DH\PYGZus{}b} \PYG{o}{=} \PYG{n}{Hb\PYGZus{}hr} \PYG{o}{\PYGZhy{}} \PYG{n}{Hb\PYGZus{}hl}
\PYG{n}{H\PYGZus{}gb} \PYG{o}{=} \PYG{n}{DH\PYGZus{}b}\PYG{o}{/}\PYG{n}{L}\PYG{c+c1}{\PYGZsh{}, no unit, hydraulic gradient }
\PYG{n+nb}{print}\PYG{p}{(}\PYG{l+s+s2}{\PYGZdq{}}\PYG{l+s+s2}{Hydraulic head LEFT: }\PYG{l+s+si}{\PYGZob{}0:1.1f\PYGZcb{}}\PYG{l+s+s2}{\PYGZdq{}}\PYG{o}{.}\PYG{n}{format}\PYG{p}{(}\PYG{n}{Hb\PYGZus{}hl}\PYG{p}{)}\PYG{p}{,}\PYG{l+s+s2}{\PYGZdq{}}\PYG{l+s+s2}{m}\PYG{l+s+s2}{\PYGZdq{}}\PYG{p}{)}\PYG{p}{;}\PYG{n+nb}{print}\PYG{p}{(}\PYG{l+s+s2}{\PYGZdq{}}\PYG{l+s+s2}{Hydraulic head RIGHT:: }\PYG{l+s+si}{\PYGZob{}0:1.1f\PYGZcb{}}\PYG{l+s+s2}{\PYGZdq{}}\PYG{o}{.}\PYG{n}{format}\PYG{p}{(}\PYG{n}{Hb\PYGZus{}hr}\PYG{p}{)}\PYG{p}{,}\PYG{l+s+s2}{\PYGZdq{}}\PYG{l+s+s2}{m}\PYG{l+s+s2}{\PYGZdq{}}\PYG{p}{)} 
\PYG{n+nb}{print}\PYG{p}{(}\PYG{l+s+s2}{\PYGZdq{}}\PYG{l+s+s2}{Hydraulic Head Difference: }\PYG{l+s+si}{\PYGZob{}0:1.1f\PYGZcb{}}\PYG{l+s+s2}{\PYGZdq{}}\PYG{o}{.}\PYG{n}{format}\PYG{p}{(}\PYG{n}{DH\PYGZus{}b}\PYG{p}{)}\PYG{p}{,}\PYG{l+s+s2}{\PYGZdq{}}\PYG{l+s+s2}{m}\PYG{l+s+s2}{\PYGZdq{}}\PYG{p}{)}\PYG{p}{;}\PYG{n+nb}{print}\PYG{p}{(}\PYG{l+s+s2}{\PYGZdq{}}\PYG{l+s+s2}{Hydraulic gradient: }\PYG{l+s+si}{\PYGZob{}0:1.1f\PYGZcb{}}\PYG{l+s+s2}{\PYGZdq{}}\PYG{o}{.}\PYG{n}{format}\PYG{p}{(}\PYG{n}{H\PYGZus{}gb}\PYG{p}{)}\PYG{p}{)} 
\PYG{n}{png\PYGZus{}pane2}\PYG{o}{.}\PYG{n}{object} \PYG{o}{=} \PYG{l+s+s2}{\PYGZdq{}}\PYG{l+s+s2}{images/T03\PYGZus{}TP7\PYGZus{}b2.png}\PYG{l+s+s2}{\PYGZdq{}} 
\end{sphinxVerbatim}

\end{sphinxuseclass}\end{sphinxVerbatimInput}
\begin{sphinxVerbatimOutput}

\begin{sphinxuseclass}{cell_output}
\begin{sphinxVerbatim}[commandchars=\\\{\}]
Hydraulic head LEFT: 0.3 m
Hydraulic head RIGHT:: 1.4 m
Hydraulic Head Difference: 1.1 m
Hydraulic gradient: 2.1
\end{sphinxVerbatim}

\end{sphinxuseclass}\end{sphinxVerbatimOutput}

\end{sphinxuseclass}
\begin{sphinxuseclass}{cell}\begin{sphinxVerbatimInput}

\begin{sphinxuseclass}{cell_input}
\begin{sphinxVerbatim}[commandchars=\\\{\}]
\PYG{n}{png\PYGZus{}pane3} \PYG{o}{=} \PYG{n}{pn}\PYG{o}{.}\PYG{n}{pane}\PYG{o}{.}\PYG{n}{PNG}\PYG{p}{(}\PYG{l+s+s2}{\PYGZdq{}}\PYG{l+s+s2}{images/T03\PYGZus{}TP7\PYGZus{}c1.png}\PYG{l+s+s2}{\PYGZdq{}}\PYG{p}{,} \PYG{n}{width}\PYG{o}{=}\PYG{l+m+mi}{400}\PYG{p}{)}
\PYG{n}{png\PYGZus{}pane3} 
\end{sphinxVerbatim}

\end{sphinxuseclass}\end{sphinxVerbatimInput}
\begin{sphinxVerbatimOutput}

\begin{sphinxuseclass}{cell_output}
\begin{sphinxVerbatim}[commandchars=\\\{\}]
PNG(str, width=400)
\end{sphinxVerbatim}

\end{sphinxuseclass}\end{sphinxVerbatimOutput}

\end{sphinxuseclass}
\begin{sphinxuseclass}{cell}\begin{sphinxVerbatimInput}

\begin{sphinxuseclass}{cell_input}
\begin{sphinxVerbatim}[commandchars=\\\{\}]
\PYG{c+c1}{\PYGZsh{} Image (c)}
\PYG{n}{L} \PYG{o}{=} \PYG{l+m+mf}{0.5} \PYG{c+c1}{\PYGZsh{} m, length of the column}
\PYG{n}{Ec\PYGZus{}hl} \PYG{o}{=} \PYG{l+m+mf}{0.3} \PYG{c+c1}{\PYGZsh{}, m, elevation head, left }
\PYG{n}{Pc\PYGZus{}hl} \PYG{o}{=} \PYG{l+m+mf}{0.1} \PYG{c+c1}{\PYGZsh{}, m  pressure head, left}
\PYG{n}{Ec\PYGZus{}hr} \PYG{o}{=} \PYG{l+m+mf}{0.2} \PYG{c+c1}{\PYGZsh{}, m, elevation head, right}
\PYG{n}{Pc\PYGZus{}hr} \PYG{o}{=} \PYG{l+m+mf}{0.2} \PYG{c+c1}{\PYGZsh{}, m  pressure head, right}
\PYG{n}{Hc\PYGZus{}hl} \PYG{o}{=} \PYG{n}{Ec\PYGZus{}hl} \PYG{o}{+} \PYG{n}{Pc\PYGZus{}hl} \PYG{c+c1}{\PYGZsh{} m, hydraulic head, left}
\PYG{n}{Hc\PYGZus{}hr} \PYG{o}{=} \PYG{n}{Ec\PYGZus{}hr} \PYG{o}{+} \PYG{n}{Pc\PYGZus{}hr} \PYG{c+c1}{\PYGZsh{} m, hydraulic head, right}
\PYG{n}{DH\PYGZus{}c} \PYG{o}{=} \PYG{n}{Hc\PYGZus{}hr} \PYG{o}{\PYGZhy{}} \PYG{n}{Hc\PYGZus{}hl}
\PYG{n}{H\PYGZus{}gc} \PYG{o}{=} \PYG{n}{DH\PYGZus{}c}\PYG{o}{/}\PYG{n}{L}\PYG{c+c1}{\PYGZsh{}, no unit, hydraulic gradient }
\PYG{n+nb}{print}\PYG{p}{(}\PYG{l+s+s2}{\PYGZdq{}}\PYG{l+s+s2}{Hydraulic head LEFT: }\PYG{l+s+si}{\PYGZob{}0:1.1f\PYGZcb{}}\PYG{l+s+s2}{\PYGZdq{}}\PYG{o}{.}\PYG{n}{format}\PYG{p}{(}\PYG{n}{Hc\PYGZus{}hl}\PYG{p}{)}\PYG{p}{,}\PYG{l+s+s2}{\PYGZdq{}}\PYG{l+s+s2}{m}\PYG{l+s+s2}{\PYGZdq{}}\PYG{p}{)}\PYG{p}{;}\PYG{n+nb}{print}\PYG{p}{(}\PYG{l+s+s2}{\PYGZdq{}}\PYG{l+s+s2}{Hydraulic head RIGHT:: }\PYG{l+s+si}{\PYGZob{}0:1.1f\PYGZcb{}}\PYG{l+s+s2}{\PYGZdq{}}\PYG{o}{.}\PYG{n}{format}\PYG{p}{(}\PYG{n}{Hc\PYGZus{}hr}\PYG{p}{)}\PYG{p}{,}\PYG{l+s+s2}{\PYGZdq{}}\PYG{l+s+s2}{m}\PYG{l+s+s2}{\PYGZdq{}}\PYG{p}{)} 
\PYG{n+nb}{print}\PYG{p}{(}\PYG{l+s+s2}{\PYGZdq{}}\PYG{l+s+s2}{Hydraulic Head Difference: }\PYG{l+s+si}{\PYGZob{}0:1.1f\PYGZcb{}}\PYG{l+s+s2}{\PYGZdq{}}\PYG{o}{.}\PYG{n}{format}\PYG{p}{(}\PYG{n}{DH\PYGZus{}c}\PYG{p}{)}\PYG{p}{,}\PYG{l+s+s2}{\PYGZdq{}}\PYG{l+s+s2}{m}\PYG{l+s+s2}{\PYGZdq{}}\PYG{p}{)}\PYG{p}{;}\PYG{n+nb}{print}\PYG{p}{(}\PYG{l+s+s2}{\PYGZdq{}}\PYG{l+s+s2}{Hydraulic gradient: }\PYG{l+s+si}{\PYGZob{}0:1.1f\PYGZcb{}}\PYG{l+s+s2}{\PYGZdq{}}\PYG{o}{.}\PYG{n}{format}\PYG{p}{(}\PYG{n}{H\PYGZus{}gc}\PYG{p}{)}\PYG{p}{)} 
\PYG{n}{png\PYGZus{}pane3}\PYG{o}{.}\PYG{n}{object} \PYG{o}{=} \PYG{l+s+s2}{\PYGZdq{}}\PYG{l+s+s2}{images/T03\PYGZus{}TP7\PYGZus{}c2.png}\PYG{l+s+s2}{\PYGZdq{}} 
\end{sphinxVerbatim}

\end{sphinxuseclass}\end{sphinxVerbatimInput}
\begin{sphinxVerbatimOutput}

\begin{sphinxuseclass}{cell_output}
\begin{sphinxVerbatim}[commandchars=\\\{\}]
Hydraulic head LEFT: 0.4 m
Hydraulic head RIGHT:: 0.4 m
Hydraulic Head Difference: 0.0 m
Hydraulic gradient: 0.0
\end{sphinxVerbatim}

\end{sphinxuseclass}\end{sphinxVerbatimOutput}

\end{sphinxuseclass}
\begin{sphinxuseclass}{cell}\begin{sphinxVerbatimInput}

\begin{sphinxuseclass}{cell_input}
\begin{sphinxVerbatim}[commandchars=\\\{\}]
\PYG{n}{png\PYGZus{}pane4} \PYG{o}{=} \PYG{n}{pn}\PYG{o}{.}\PYG{n}{pane}\PYG{o}{.}\PYG{n}{PNG}\PYG{p}{(}\PYG{l+s+s2}{\PYGZdq{}}\PYG{l+s+s2}{images/T03\PYGZus{}TP7\PYGZus{}d1.png}\PYG{l+s+s2}{\PYGZdq{}}\PYG{p}{,} \PYG{n}{width}\PYG{o}{=}\PYG{l+m+mi}{400}\PYG{p}{)}
\PYG{n}{png\PYGZus{}pane4} 
\end{sphinxVerbatim}

\end{sphinxuseclass}\end{sphinxVerbatimInput}
\begin{sphinxVerbatimOutput}

\begin{sphinxuseclass}{cell_output}
\begin{sphinxVerbatim}[commandchars=\\\{\}]
PNG(str, width=400)
\end{sphinxVerbatim}

\end{sphinxuseclass}\end{sphinxVerbatimOutput}

\end{sphinxuseclass}
\begin{sphinxuseclass}{cell}\begin{sphinxVerbatimInput}

\begin{sphinxuseclass}{cell_input}
\begin{sphinxVerbatim}[commandchars=\\\{\}]
\PYG{c+c1}{\PYGZsh{} Image (d)}
\PYG{n}{L} \PYG{o}{=} \PYG{l+m+mf}{0.5} \PYG{c+c1}{\PYGZsh{} m, length of the column}
\PYG{n}{Ed\PYGZus{}hl} \PYG{o}{=} \PYG{l+m+mf}{0.3} \PYG{c+c1}{\PYGZsh{}, m, elevation head, left }
\PYG{n}{Pd\PYGZus{}hl} \PYG{o}{=} \PYG{l+m+mf}{0.1} \PYG{c+c1}{\PYGZsh{}, m  pressure head, left}
\PYG{n}{Ed\PYGZus{}hr} \PYG{o}{=} \PYG{l+m+mf}{0.2} \PYG{c+c1}{\PYGZsh{}, m, elevation head, right}
\PYG{n}{Pd\PYGZus{}hr} \PYG{o}{=} \PYG{l+m+mf}{0.1} \PYG{c+c1}{\PYGZsh{}, m  pressure head, right}
\PYG{n}{Hd\PYGZus{}hl} \PYG{o}{=} \PYG{n}{Ed\PYGZus{}hl} \PYG{o}{+} \PYG{n}{Pd\PYGZus{}hl} \PYG{c+c1}{\PYGZsh{} m, hydraulic head, left}
\PYG{n}{Hd\PYGZus{}hr} \PYG{o}{=} \PYG{n}{Ed\PYGZus{}hr} \PYG{o}{+} \PYG{n}{Pd\PYGZus{}hr} \PYG{c+c1}{\PYGZsh{} m, hydraulic head, right}
\PYG{n}{DH\PYGZus{}d} \PYG{o}{=} \PYG{n}{Hd\PYGZus{}hr} \PYG{o}{\PYGZhy{}} \PYG{n}{Hd\PYGZus{}hl}
\PYG{n}{H\PYGZus{}gd} \PYG{o}{=} \PYG{n}{DH\PYGZus{}d}\PYG{o}{/}\PYG{n}{L}\PYG{c+c1}{\PYGZsh{}, no unit, hydraulic gradient }
\PYG{c+c1}{\PYGZsh{}output}
\PYG{n+nb}{print}\PYG{p}{(}\PYG{l+s+s2}{\PYGZdq{}}\PYG{l+s+s2}{Hydraulic head LEFT: }\PYG{l+s+si}{\PYGZob{}0:1.1f\PYGZcb{}}\PYG{l+s+s2}{\PYGZdq{}}\PYG{o}{.}\PYG{n}{format}\PYG{p}{(}\PYG{n}{Hd\PYGZus{}hl}\PYG{p}{)}\PYG{p}{,}\PYG{l+s+s2}{\PYGZdq{}}\PYG{l+s+s2}{m}\PYG{l+s+s2}{\PYGZdq{}}\PYG{p}{)}\PYG{p}{;}\PYG{n+nb}{print}\PYG{p}{(}\PYG{l+s+s2}{\PYGZdq{}}\PYG{l+s+s2}{Hydraulic head Right:: }\PYG{l+s+si}{\PYGZob{}0:1.1f\PYGZcb{}}\PYG{l+s+s2}{\PYGZdq{}}\PYG{o}{.}\PYG{n}{format}\PYG{p}{(}\PYG{n}{Hd\PYGZus{}hr}\PYG{p}{)}\PYG{p}{,}\PYG{l+s+s2}{\PYGZdq{}}\PYG{l+s+s2}{m}\PYG{l+s+s2}{\PYGZdq{}}\PYG{p}{)} 
\PYG{n+nb}{print}\PYG{p}{(}\PYG{l+s+s2}{\PYGZdq{}}\PYG{l+s+s2}{Hydraulic Head Difference: }\PYG{l+s+si}{\PYGZob{}0:1.1f\PYGZcb{}}\PYG{l+s+s2}{\PYGZdq{}}\PYG{o}{.}\PYG{n}{format}\PYG{p}{(}\PYG{n}{DH\PYGZus{}d}\PYG{p}{)}\PYG{p}{,}\PYG{l+s+s2}{\PYGZdq{}}\PYG{l+s+s2}{m}\PYG{l+s+s2}{\PYGZdq{}}\PYG{p}{)}\PYG{p}{;}\PYG{n+nb}{print}\PYG{p}{(}\PYG{l+s+s2}{\PYGZdq{}}\PYG{l+s+s2}{Hydraulic gradient: }\PYG{l+s+si}{\PYGZob{}0:1.1f\PYGZcb{}}\PYG{l+s+s2}{\PYGZdq{}}\PYG{o}{.}\PYG{n}{format}\PYG{p}{(}\PYG{n}{H\PYGZus{}gd}\PYG{p}{)}\PYG{p}{)} 
\PYG{n}{png\PYGZus{}pane4}\PYG{o}{.}\PYG{n}{object} \PYG{o}{=} \PYG{l+s+s2}{\PYGZdq{}}\PYG{l+s+s2}{images/T03\PYGZus{}TP7\PYGZus{}d2.png}\PYG{l+s+s2}{\PYGZdq{}} 
\end{sphinxVerbatim}

\end{sphinxuseclass}\end{sphinxVerbatimInput}
\begin{sphinxVerbatimOutput}

\begin{sphinxuseclass}{cell_output}
\begin{sphinxVerbatim}[commandchars=\\\{\}]
Hydraulic head LEFT: 0.4 m
Hydraulic head Right:: 0.3 m
Hydraulic Head Difference: \PYGZhy{}0.1 m
Hydraulic gradient: \PYGZhy{}0.2
\end{sphinxVerbatim}

\end{sphinxuseclass}\end{sphinxVerbatimOutput}

\end{sphinxuseclass}

\section{Tutorial Problem 8}
\label{\detokenize{content/tutorials/T3/tutorial_03:tutorial-problem-8}}
\sphinxAtStartPar
The hydraulic conductivity of a fine sand sample was found to be equal to \(1.36\times 10^{-5}\) m/s in a Darcy experiment using water at a temperature of \(20^\circ\)C. What is the intrinsic permeability of this sample? Give results in cm\(^2\) and D.
(density of water at \(20^\circ\)C: 998.2 kg/m\(^3\); dynamic viscosity of water at \(20^\circ\)C: \(1.0087 \times 10^{-3}\) Pa\(\cdot\)s;  1 D = \(0.987\times 10^{-12}\) m\(^2\))


\subsection{Tutorial Problem 8  \sphinxhyphen{} Solution}
\label{\detokenize{content/tutorials/T3/tutorial_03:tutorial-problem-8-solution}}
\sphinxAtStartPar
\sphinxstylestrong{Relevant topics are covered in Lecture 04 slides 18\sphinxhyphen{}20.}

\sphinxAtStartPar
Relationship between hydraulic conductivity \(K\) and intrinsic permeability \(k\) from lecture notes:
\$\(
K_{water} = k\cdot \frac{\rho_{water}\cdot g}{\eta_{water}}
\)\$

\sphinxAtStartPar
Solve for , \(k\)
\begin{equation*}
\begin{split}
k = \frac{\eta_{water}\cdot K_{water}}{\rho_{water}\cdot g}{}
\end{split}
\end{equation*}
\begin{sphinxuseclass}{cell}\begin{sphinxVerbatimInput}

\begin{sphinxuseclass}{cell_input}
\begin{sphinxVerbatim}[commandchars=\\\{\}]
\PYG{c+c1}{\PYGZsh{}Given}
\PYG{n}{Darcy} \PYG{o}{=} \PYG{l+m+mf}{0.987} \PYG{o}{*} \PYG{l+m+mi}{10}\PYG{o}{*}\PYG{o}{*}\PYG{o}{\PYGZhy{}}\PYG{l+m+mi}{12} \PYG{c+c1}{\PYGZsh{} m\PYGZca{}2, 1D = 0.987*10\PYGZca{}\PYGZhy{}12 m\PYGZca{}2 }
\PYG{n}{nu\PYGZus{}w} \PYG{o}{=} \PYG{l+m+mf}{1.00087}\PYG{o}{*}\PYG{l+m+mi}{10}\PYG{o}{*}\PYG{o}{*}\PYG{o}{\PYGZhy{}}\PYG{l+m+mi}{3} \PYG{c+c1}{\PYGZsh{} Pa\PYGZhy{}S dynamic viscosity of water}
\PYG{n}{K\PYGZus{}w} \PYG{o}{=} \PYG{l+m+mf}{1.36}\PYG{o}{*}\PYG{l+m+mi}{10}\PYG{o}{*}\PYG{o}{*}\PYG{o}{\PYGZhy{}}\PYG{l+m+mi}{5} \PYG{c+c1}{\PYGZsh{} m/s Conductivity of water}
\PYG{n}{g} \PYG{o}{=} \PYG{l+m+mf}{9.81} \PYG{c+c1}{\PYGZsh{} m/s\PYGZca{}2 accln due to gravity}
\PYG{n}{rho\PYGZus{}w} \PYG{o}{=} \PYG{l+m+mf}{998.2} \PYG{c+c1}{\PYGZsh{} kg/m\PYGZca{}3, density of water}

\PYG{c+c1}{\PYGZsh{} Solution}
\PYG{n}{k} \PYG{o}{=} \PYG{p}{(}\PYG{n}{nu\PYGZus{}w}\PYG{o}{*}\PYG{n}{K\PYGZus{}w}\PYG{p}{)}\PYG{o}{/}\PYG{p}{(}\PYG{n}{rho\PYGZus{}w}\PYG{o}{*}\PYG{n}{g}\PYG{p}{)}\PYG{c+c1}{\PYGZsh{}, m\PYGZca{}2, permeability of water}
\PYG{n}{k\PYGZus{}D} \PYG{o}{=} \PYG{n}{k}\PYG{o}{/}\PYG{n}{Darcy}


\PYG{n+nb}{print}\PYG{p}{(}\PYG{l+s+s2}{\PYGZdq{}}\PYG{l+s+s2}{The permeability is }\PYG{l+s+si}{\PYGZob{}0:1.1E\PYGZcb{}}\PYG{l+s+s2}{\PYGZdq{}}\PYG{o}{.}\PYG{n}{format}\PYG{p}{(}\PYG{n}{k}\PYG{p}{)}\PYG{p}{,}\PYG{l+s+s2}{\PYGZdq{}}\PYG{l+s+s2}{m}\PYG{l+s+se}{\PYGZbs{}u00b2}\PYG{l+s+s2}{\PYGZdq{}}\PYG{p}{)}  
\PYG{n+nb}{print}\PYG{p}{(}\PYG{l+s+s2}{\PYGZdq{}}\PYG{l+s+s2}{The permeability in Darcy unite is: }\PYG{l+s+si}{\PYGZob{}0:1.2f\PYGZcb{}}\PYG{l+s+s2}{\PYGZdq{}}\PYG{o}{.}\PYG{n}{format}\PYG{p}{(}\PYG{n}{k\PYGZus{}D}\PYG{p}{)}\PYG{p}{,}\PYG{l+s+s2}{\PYGZdq{}}\PYG{l+s+s2}{D}\PYG{l+s+s2}{\PYGZdq{}}\PYG{p}{)} 
\end{sphinxVerbatim}

\end{sphinxuseclass}\end{sphinxVerbatimInput}
\begin{sphinxVerbatimOutput}

\begin{sphinxuseclass}{cell_output}
\begin{sphinxVerbatim}[commandchars=\\\{\}]
The permeability is 1.4E\PYGZhy{}12 m²
The permeability in Darcy unite is: 1.41 D
\end{sphinxVerbatim}

\end{sphinxuseclass}\end{sphinxVerbatimOutput}

\end{sphinxuseclass}

\section{Tutorial Problem 9: Constant\sphinxhyphen{}Head Permeameter}
\label{\detokenize{content/tutorials/T3/tutorial_03:tutorial-problem-9-constant-head-permeameter}}
\begin{sphinxuseclass}{cell}
\begin{sphinxuseclass}{tag_hide-input}\begin{sphinxVerbatimOutput}

\begin{sphinxuseclass}{cell_output}
\begin{sphinxVerbatim}[commandchars=\\\{\}]
Row
    [0] LaTeX(str, style=\PYGZob{}\PYGZsq{}font\PYGZhy{}size\PYGZsq{}: \PYGZsq{}12pt\PYGZsq{}\PYGZcb{}, width=450)
    [1] Spacer(width=100)
    [2] PNG(str, width=400)
\end{sphinxVerbatim}

\end{sphinxuseclass}\end{sphinxVerbatimOutput}

\end{sphinxuseclass}
\end{sphinxuseclass}
\begin{sphinxuseclass}{cell}
\begin{sphinxuseclass}{tag_hide-input}\begin{sphinxVerbatimOutput}

\begin{sphinxuseclass}{cell_output}
\begin{sphinxVerbatim}[commandchars=\\\{\}]
Row(width=1000)
    [0] LaTeX(str, style=\PYGZob{}\PYGZsq{}font\PYGZhy{}size\PYGZsq{}: \PYGZsq{}13pt\PYGZsq{}\PYGZcb{}, width=500)
    [1] Spacer(width=100)
    [2] PNG(str, width=300)
\end{sphinxVerbatim}

\end{sphinxuseclass}\end{sphinxVerbatimOutput}

\end{sphinxuseclass}
\end{sphinxuseclass}
\begin{sphinxuseclass}{cell}\begin{sphinxVerbatimInput}

\begin{sphinxuseclass}{cell_input}
\begin{sphinxVerbatim}[commandchars=\\\{\}]
\PYG{c+c1}{\PYGZsh{}Given (solution on 9b)}
\PYG{n}{L} \PYG{o}{=} \PYG{l+m+mi}{15} \PYG{c+c1}{\PYGZsh{}cm, length of column}
\PYG{n}{A} \PYG{o}{=} \PYG{l+m+mi}{25} \PYG{c+c1}{\PYGZsh{} cm\PYGZca{}2, surface area of column}
\PYG{n}{h\PYGZus{}diff} \PYG{o}{=} \PYG{l+m+mi}{5} \PYG{c+c1}{\PYGZsh{} cm, h\PYGZus{}in\PYGZhy{}h\PYGZus{}out}
\PYG{n}{Q} \PYG{o}{=} \PYG{l+m+mi}{100}\PYG{o}{/}\PYG{l+m+mi}{12} \PYG{c+c1}{\PYGZsh{} cm\PYGZca{}3/min discharge per min}

\PYG{c+c1}{\PYGZsh{} Solution using derived equation in first part of the problem}
\PYG{c+c1}{\PYGZsh{} K = QL/A(h\PYGZus{}in\PYGZhy{} h\PYGZus{}out)}

\PYG{n}{K} \PYG{o}{=} \PYG{p}{(}\PYG{n}{Q}\PYG{o}{*}\PYG{n}{L}\PYG{p}{)}\PYG{o}{/}\PYG{p}{(}\PYG{n}{A}\PYG{o}{*}\PYG{n}{h\PYGZus{}diff}\PYG{p}{)}\PYG{c+c1}{\PYGZsh{} cm/min, required conductivity }
\PYG{n}{K\PYGZus{}1} \PYG{o}{=} \PYG{n}{K}\PYG{o}{*}\PYG{l+m+mi}{10}\PYG{o}{*}\PYG{o}{*}\PYG{o}{\PYGZhy{}}\PYG{l+m+mi}{2}\PYG{o}{/}\PYG{l+m+mi}{60} \PYG{c+c1}{\PYGZsh{}, m/s, conductivity in m/s}

\PYG{c+c1}{\PYGZsh{}output}
\PYG{n+nb}{print}\PYG{p}{(}\PYG{l+s+s2}{\PYGZdq{}}\PYG{l+s+s2}{The conductivity in column is }\PYG{l+s+si}{\PYGZob{}0:1.2E\PYGZcb{}}\PYG{l+s+s2}{\PYGZdq{}}\PYG{o}{.}\PYG{n}{format}\PYG{p}{(}\PYG{n}{K}\PYG{p}{)}\PYG{p}{,}\PYG{l+s+s2}{\PYGZdq{}}\PYG{l+s+s2}{cm/min}\PYG{l+s+s2}{\PYGZdq{}}\PYG{p}{)} 
\PYG{n+nb}{print}\PYG{p}{(}\PYG{l+s+s2}{\PYGZdq{}}\PYG{l+s+s2}{The conductivity in column is }\PYG{l+s+si}{\PYGZob{}0:1.2E\PYGZcb{}}\PYG{l+s+s2}{\PYGZdq{}}\PYG{o}{.}\PYG{n}{format}\PYG{p}{(}\PYG{n}{K\PYGZus{}1}\PYG{p}{)}\PYG{p}{,}\PYG{l+s+s2}{\PYGZdq{}}\PYG{l+s+s2}{m/s }\PYG{l+s+se}{\PYGZbs{}n}\PYG{l+s+s2}{\PYGZdq{}}\PYG{p}{)} 

\PYG{k}{if} \PYG{n}{K\PYGZus{}1} \PYG{o}{\PYGZlt{}}\PYG{o}{=} \PYG{l+m+mf}{1.67}\PYG{o}{*}\PYG{l+m+mi}{10}\PYG{o}{*}\PYG{o}{*}\PYG{o}{\PYGZhy{}}\PYG{l+m+mi}{4}\PYG{p}{:}
    \PYG{n+nb}{print}\PYG{p}{(}\PYG{l+s+s2}{\PYGZdq{}}\PYG{l+s+s2}{Fine to medium sand}\PYG{l+s+s2}{\PYGZdq{}}\PYG{p}{)}
\PYG{k}{else}\PYG{p}{:}
    \PYG{n+nb}{print}\PYG{p}{(}\PYG{l+s+s2}{\PYGZdq{}}\PYG{l+s+s2}{to check further}\PYG{l+s+s2}{\PYGZdq{}}\PYG{p}{)} \PYG{c+c1}{\PYGZsh{} to be completed later.}
\end{sphinxVerbatim}

\end{sphinxuseclass}\end{sphinxVerbatimInput}
\begin{sphinxVerbatimOutput}

\begin{sphinxuseclass}{cell_output}
\begin{sphinxVerbatim}[commandchars=\\\{\}]
The conductivity in column is 1.00E+00 cm/min
The conductivity in column is 1.67E\PYGZhy{}04 m/s 

Fine to medium sand
\end{sphinxVerbatim}

\end{sphinxuseclass}\end{sphinxVerbatimOutput}

\end{sphinxuseclass}
\sphinxAtStartPar
Continue solution on 9c

\sphinxAtStartPar
Discharge and Darcy’s law: \(Q_{water} = \frac{V}{t_{water}}=-A\cdot K_{water}\cdot\frac{\Delta h}{L}\)

\sphinxAtStartPar
Solve for \(t_{water}\):   \(t_{water} = \frac{V}{Q_{water}}=-\frac{V}{A\cdot K_{water}\cdot\Delta h/L} = -\frac{V\cdot L}{A\cdot K_{water}\cdot\Delta h}\)

\sphinxAtStartPar
Same step for \(t_{diesel}\):   \(t_{diesel}  = -\frac{V\cdot L}{A\cdot K_{diesel}\cdot\Delta h}\)

\sphinxAtStartPar
time ratio:\(\frac{t_{diesel}}{t_{water}} = \frac{-\frac{V\cdot L}{A\cdot K_{diesel}\cdot\Delta h}}{-\frac{V\cdot L}{A\cdot K_{water}\cdot\Delta h}} = \frac{K_{water}}{K_{diesel}}\)

\sphinxAtStartPar
Use relationship between conductivity \(K\) and permeability \(k\) from lecture notes (slides 18)
\$\(\frac{K_{water}}{K_{diesel}} = \frac{k\cdot \frac{\rho_{water}\cdot g}{\eta_{water}}}{k\cdot \frac{\rho_{diesel}\cdot g}{\eta_{diesel}}} = \frac{\rho_{water}\cdot\eta_{diesel}}{\rho_{diesel}\cdot\eta_{water}}\)\$

\sphinxAtStartPar
Solve for \(t_{diesel}\)

\begin{sphinxuseclass}{cell}\begin{sphinxVerbatimInput}

\begin{sphinxuseclass}{cell_input}
\begin{sphinxVerbatim}[commandchars=\\\{\}]
\PYG{c+c1}{\PYGZsh{} Given data}

\PYG{n}{rho\PYGZus{}w} \PYG{o}{=} \PYG{l+m+mf}{920.2} \PYG{c+c1}{\PYGZsh{} kg/m\PYGZca{}3, density of water at 20°C}
\PYG{n}{eta\PYGZus{}w} \PYG{o}{=} \PYG{l+m+mf}{1.0087}\PYG{o}{*}\PYG{l+m+mi}{10}\PYG{o}{*}\PYG{o}{*}\PYG{o}{\PYGZhy{}}\PYG{l+m+mi}{3}\PYG{c+c1}{\PYGZsh{}, Pa\PYGZhy{}S dynamic viscosity of water}
\PYG{n}{rho\PYGZus{}d} \PYG{o}{=} \PYG{l+m+mf}{0.85} \PYG{c+c1}{\PYGZsh{} g/cm\PYGZca{}3, density of diesel at 20°C}
\PYG{n}{eta\PYGZus{}d} \PYG{o}{=} \PYG{l+m+mf}{3.5}\PYG{o}{*}\PYG{l+m+mi}{10}\PYG{o}{*}\PYG{o}{*}\PYG{o}{\PYGZhy{}}\PYG{l+m+mi}{3}\PYG{c+c1}{\PYGZsh{}, Pa\PYGZhy{}S dynamic viscosity of diesel}
\PYG{n}{V\PYGZus{}d} \PYG{o}{=} \PYG{l+m+mi}{100} \PYG{c+c1}{\PYGZsh{} cm\PYGZca{}3 volume of diesel}
\PYG{n}{t\PYGZus{}w} \PYG{o}{=} \PYG{l+m+mi}{12} \PYG{c+c1}{\PYGZsh{} min, time taken by water}

\PYG{c+c1}{\PYGZsh{} Calculations}

\PYG{n}{t\PYGZus{}d} \PYG{o}{=} \PYG{p}{(}\PYG{n}{rho\PYGZus{}w}\PYG{o}{*}\PYG{n}{eta\PYGZus{}d}\PYG{p}{)}\PYG{o}{/}\PYG{p}{(}\PYG{n}{rho\PYGZus{}d}\PYG{o}{*}\PYG{l+m+mi}{1000}\PYG{o}{*}\PYG{n}{eta\PYGZus{}w}\PYG{p}{)}\PYG{o}{*}\PYG{n}{t\PYGZus{}w} 

\PYG{c+c1}{\PYGZsh{} multiplied by 1000 to convert unit g/cm\PYGZca{}3 to kg/m\PYGZca{}3}

\PYG{n+nb}{print}\PYG{p}{(}\PYG{l+s+s2}{\PYGZdq{}}\PYG{l+s+s2}{The time required for diesel will be: }\PYG{l+s+si}{\PYGZob{}0:0.2f\PYGZcb{}}\PYG{l+s+s2}{\PYGZdq{}}\PYG{o}{.}\PYG{n}{format}\PYG{p}{(}\PYG{n}{t\PYGZus{}d}\PYG{p}{)}\PYG{p}{,} \PYG{l+s+s2}{\PYGZdq{}}\PYG{l+s+s2}{min}\PYG{l+s+s2}{\PYGZdq{}}\PYG{p}{)} 
\end{sphinxVerbatim}

\end{sphinxuseclass}\end{sphinxVerbatimInput}
\begin{sphinxVerbatimOutput}

\begin{sphinxuseclass}{cell_output}
\begin{sphinxVerbatim}[commandchars=\\\{\}]
The time required for diesel will be: 45.08 min
\end{sphinxVerbatim}

\end{sphinxuseclass}\end{sphinxVerbatimOutput}

\end{sphinxuseclass}

\section{Tutorial Problem 10: Falling\sphinxhyphen{}Head Permeameter}
\label{\detokenize{content/tutorials/T3/tutorial_03:tutorial-problem-10-falling-head-permeameter}}
\begin{sphinxuseclass}{cell}
\begin{sphinxuseclass}{tag_hide-input}\begin{sphinxVerbatimOutput}

\begin{sphinxuseclass}{cell_output}
\begin{sphinxVerbatim}[commandchars=\\\{\}]
Row
    [0] Column
        [0] LaTeX(str, style=\PYGZob{}\PYGZsq{}font\PYGZhy{}size\PYGZsq{}: \PYGZsq{}12pt\PYGZsq{}\PYGZcb{}, width=400)
        [1] Markdown(str, style=\PYGZob{}\PYGZsq{}font\PYGZhy{}size\PYGZsq{}: \PYGZsq{}12pt\PYGZsq{}\PYGZcb{}, width=500)
        [2] Markdown(str, style=\PYGZob{}\PYGZsq{}font\PYGZhy{}size\PYGZsq{}: \PYGZsq{}12pt\PYGZsq{}\PYGZcb{}, width=500)
    [1] Spacer(width=50)
    [2] PNG(str, width=350)
\end{sphinxVerbatim}

\end{sphinxuseclass}\end{sphinxVerbatimOutput}

\end{sphinxuseclass}
\end{sphinxuseclass}
\begin{sphinxuseclass}{cell}
\begin{sphinxuseclass}{tag_hide-input}\begin{sphinxVerbatimOutput}

\begin{sphinxuseclass}{cell_output}
\begin{sphinxVerbatim}[commandchars=\\\{\}]
Row
    [0] LaTeX(str, style=\PYGZob{}\PYGZsq{}font\PYGZhy{}size\PYGZsq{}: \PYGZsq{}13pt\PYGZsq{}\PYGZcb{})
    [1] PNG(str, width=300)
\end{sphinxVerbatim}

\end{sphinxuseclass}\end{sphinxVerbatimOutput}

\end{sphinxuseclass}
\end{sphinxuseclass}
\begin{sphinxuseclass}{cell}
\begin{sphinxuseclass}{tag_hide-input}\begin{sphinxVerbatimOutput}

\begin{sphinxuseclass}{cell_output}
\begin{sphinxVerbatim}[commandchars=\\\{\}]
LaTeX(str, style=\PYGZob{}\PYGZsq{}font\PYGZhy{}size\PYGZsq{}: \PYGZsq{}12pt\PYGZsq{}\PYGZcb{}, width=600)
\end{sphinxVerbatim}

\end{sphinxuseclass}\end{sphinxVerbatimOutput}

\end{sphinxuseclass}
\end{sphinxuseclass}
\begin{sphinxuseclass}{cell}\begin{sphinxVerbatimInput}

\begin{sphinxuseclass}{cell_input}
\begin{sphinxVerbatim}[commandchars=\\\{\}]
\PYG{c+c1}{\PYGZsh{} Solution 9(2)}

\PYG{n}{L} \PYG{o}{=} \PYG{l+m+mi}{15} \PYG{c+c1}{\PYGZsh{} cm, length}
\PYG{n}{L\PYGZus{}m} \PYG{o}{=} \PYG{n}{L}\PYG{o}{/}\PYG{l+m+mi}{100} \PYG{c+c1}{\PYGZsh{} m, length}
\PYG{n}{d\PYGZus{}c} \PYG{o}{=} \PYG{l+m+mi}{10} \PYG{c+c1}{\PYGZsh{} cm, diameter column}
\PYG{n}{d\PYGZus{}t} \PYG{o}{=} \PYG{l+m+mi}{2} \PYG{c+c1}{\PYGZsh{} cm, diameter tube}
\PYG{n}{h\PYGZus{}d0} \PYG{o}{=} \PYG{l+m+mi}{5} \PYG{c+c1}{\PYGZsh{} cm, head difference at start}
\PYG{n}{h\PYGZus{}dt} \PYG{o}{=} \PYG{l+m+mf}{0.5} \PYG{c+c1}{\PYGZsh{} cm, head difference at time t}
\PYG{n}{t} \PYG{o}{=} \PYG{l+m+mi}{528} \PYG{c+c1}{\PYGZsh{} min, total time  }
\PYG{n}{t\PYGZus{}s} \PYG{o}{=} \PYG{l+m+mi}{528}\PYG{o}{*}\PYG{l+m+mi}{60} \PYG{c+c1}{\PYGZsh{} sec, total time in seconds}

\PYG{c+c1}{\PYGZsh{}solution using the developed equation }

\PYG{n}{K} \PYG{o}{=} \PYG{p}{(}\PYG{n}{d\PYGZus{}t}\PYG{o}{/}\PYG{n}{d\PYGZus{}c}\PYG{p}{)}\PYG{o}{*}\PYG{o}{*}\PYG{l+m+mi}{2} \PYG{o}{*} \PYG{p}{(}\PYG{p}{(}\PYG{n}{L\PYGZus{}m}\PYG{p}{)}\PYG{o}{/}\PYG{n}{t\PYGZus{}s}\PYG{p}{)}\PYG{o}{*}\PYG{n}{np}\PYG{o}{.}\PYG{n}{log}\PYG{p}{(}\PYG{n}{h\PYGZus{}d0}\PYG{o}{/}\PYG{n}{h\PYGZus{}dt}\PYG{p}{)}

\PYG{c+c1}{\PYGZsh{}Output}
\PYG{n+nb}{print}\PYG{p}{(}\PYG{l+s+s2}{\PYGZdq{}}\PYG{l+s+s2}{The conductivity in column is }\PYG{l+s+si}{\PYGZob{}0:1.2E\PYGZcb{}}\PYG{l+s+s2}{\PYGZdq{}}\PYG{o}{.}\PYG{n}{format}\PYG{p}{(}\PYG{n}{K}\PYG{p}{)}\PYG{p}{,}\PYG{l+s+s2}{\PYGZdq{}}\PYG{l+s+s2}{m/s }\PYG{l+s+se}{\PYGZbs{}n}\PYG{l+s+s2}{\PYGZdq{}}\PYG{p}{)} 

\PYG{k}{if} \PYG{n}{K} \PYG{o}{\PYGZlt{}} \PYG{l+m+mf}{1.67}\PYG{o}{*}\PYG{l+m+mi}{10}\PYG{o}{*}\PYG{o}{*}\PYG{o}{\PYGZhy{}}\PYG{l+m+mi}{5}\PYG{p}{:}
    \PYG{n+nb}{print}\PYG{p}{(}\PYG{l+s+s2}{\PYGZdq{}}\PYG{l+s+s2}{Silt or silty sand}\PYG{l+s+s2}{\PYGZdq{}}\PYG{p}{)}
\PYG{k}{else}\PYG{p}{:}
    \PYG{n+nb}{print}\PYG{p}{(}\PYG{l+s+s2}{\PYGZdq{}}\PYG{l+s+s2}{to check further}\PYG{l+s+s2}{\PYGZdq{}}\PYG{p}{)} \PYG{c+c1}{\PYGZsh{} to be completed later.}
\end{sphinxVerbatim}

\end{sphinxuseclass}\end{sphinxVerbatimInput}
\begin{sphinxVerbatimOutput}

\begin{sphinxuseclass}{cell_output}
\begin{sphinxVerbatim}[commandchars=\\\{\}]
The conductivity in column is 4.36E\PYGZhy{}07 m/s 

Silt or silty sand
\end{sphinxVerbatim}

\end{sphinxuseclass}\end{sphinxVerbatimOutput}

\end{sphinxuseclass}

\section{HOME WORK PROBLEMS}
\label{\detokenize{content/tutorials/T3/tutorial_03:home-work-problems}}
\sphinxAtStartPar
\sphinxstylestrong{Darcy’s Law and Intrinsic Permeability}

\sphinxAtStartPar
\sphinxstyleemphasis{There is no obligation to solve homework problems!}

\sphinxAtStartPar
\sphinxstylestrong{Try to submit within next 2 weeks.}


\section{Homework Problem 3}
\label{\detokenize{content/tutorials/T3/tutorial_03:homework-problem-3}}
\begin{sphinxuseclass}{cell}
\begin{sphinxuseclass}{tag_hide-input}\begin{sphinxVerbatimOutput}

\begin{sphinxuseclass}{cell_output}
\begin{sphinxVerbatim}[commandchars=\\\{\}]
Row
    [0] Markdown(str, style=\PYGZob{}\PYGZsq{}font\PYGZhy{}size\PYGZsq{}: \PYGZsq{}12pt\PYGZsq{}\PYGZcb{}, width=400)
    [1] Spacer(width=50)
    [2] PNG(str, width=500)
\end{sphinxVerbatim}

\end{sphinxuseclass}\end{sphinxVerbatimOutput}

\end{sphinxuseclass}
\end{sphinxuseclass}

\section{Homework Problem 4}
\label{\detokenize{content/tutorials/T3/tutorial_03:homework-problem-4}}
\begin{sphinxuseclass}{cell}
\begin{sphinxuseclass}{tag_hide-input}\begin{sphinxVerbatimOutput}

\begin{sphinxuseclass}{cell_output}
\begin{sphinxVerbatim}[commandchars=\\\{\}]
Row
    [0] Column
        [0] Markdown(str, style=\PYGZob{}\PYGZsq{}font\PYGZhy{}size\PYGZsq{}: \PYGZsq{}12pt\PYGZsq{}\PYGZcb{})
        [1] PNG(str, width=400)
        [2] Markdown(str, style=\PYGZob{}\PYGZsq{}font\PYGZhy{}size\PYGZsq{}: \PYGZsq{}12pt\PYGZsq{}\PYGZcb{})
    [1] Spacer(width=50)
    [2] PNG(str, width=400)
\end{sphinxVerbatim}

\end{sphinxuseclass}\end{sphinxVerbatimOutput}

\end{sphinxuseclass}
\end{sphinxuseclass}
\begin{sphinxuseclass}{cell}\begin{sphinxVerbatimInput}

\begin{sphinxuseclass}{cell_input}
\begin{sphinxVerbatim}[commandchars=\\\{\}]
\PYG{c+c1}{\PYGZsh{}}
\PYG{n}{fig}\PYG{p}{,} \PYG{n}{ax} \PYG{o}{=} \PYG{n}{plt}\PYG{o}{.}\PYG{n}{subplots}\PYG{p}{(}\PYG{n}{figsize}\PYG{o}{=}\PYG{p}{(}\PYG{l+m+mi}{8}\PYG{p}{,} \PYG{l+m+mi}{6}\PYG{p}{)}\PYG{p}{)}  
\PYG{n}{plt}\PYG{o}{.}\PYG{n}{grid}\PYG{p}{(}\PYG{n}{axis}\PYG{o}{=}\PYG{l+s+s1}{\PYGZsq{}}\PYG{l+s+s1}{y}\PYG{l+s+s1}{\PYGZsq{}}\PYG{p}{,} \PYG{n}{linestyle}\PYG{o}{=}\PYG{l+s+s1}{\PYGZsq{}}\PYG{l+s+s1}{\PYGZhy{}\PYGZhy{}}\PYG{l+s+s1}{\PYGZsq{}}\PYG{p}{)}   
\PYG{n}{plt}\PYG{o}{.}\PYG{n}{xlim}\PYG{p}{(}\PYG{p}{(}\PYG{l+m+mi}{0}\PYG{p}{,} \PYG{l+m+mi}{1800}\PYG{p}{)}\PYG{p}{)}\PYG{p}{;} \PYG{n}{plt}\PYG{o}{.}\PYG{n}{ylim}\PYG{p}{(}\PYG{p}{(}\PYG{l+m+mi}{0}\PYG{p}{,}\PYG{l+m+mf}{0.7}\PYG{p}{)}\PYG{p}{)} 
\PYG{n}{plt}\PYG{o}{.}\PYG{n}{xlabel}\PYG{p}{(}\PYG{l+s+s2}{\PYGZdq{}}\PYG{l+s+s2}{t(s)}\PYG{l+s+s2}{\PYGZdq{}}\PYG{p}{,} \PYG{n}{fontsize}\PYG{o}{=}\PYG{l+m+mi}{12} \PYG{p}{)}
\PYG{n}{plt}\PYG{o}{.}\PYG{n}{ylabel}\PYG{p}{(}\PYG{l+s+sa}{r}\PYG{l+s+s2}{\PYGZdq{}}\PYG{l+s+s2}{ln(\PYGZdl{}}\PYG{l+s+s2}{\PYGZbs{}}\PYG{l+s+s2}{Delta h(0)/}\PYG{l+s+s2}{\PYGZbs{}}\PYG{l+s+s2}{Delta h(t)\PYGZdl{})(\PYGZhy{})}\PYG{l+s+s2}{\PYGZdq{}}\PYG{p}{,} \PYG{n}{fontsize}\PYG{o}{=}\PYG{l+m+mi}{12}\PYG{p}{)}\PYG{p}{;} 
\end{sphinxVerbatim}

\end{sphinxuseclass}\end{sphinxVerbatimInput}
\begin{sphinxVerbatimOutput}

\begin{sphinxuseclass}{cell_output}
\noindent\sphinxincludegraphics{{C:/Users/vibhu/GWtextbook/_build/jupyter_execute/tutorial_03_29_0}.png}

\end{sphinxuseclass}\end{sphinxVerbatimOutput}

\end{sphinxuseclass}
\sphinxstepscope

\begin{sphinxuseclass}{cell}
\begin{sphinxuseclass}{tag_remove-output}\begin{sphinxVerbatimInput}

\begin{sphinxuseclass}{cell_input}
\begin{sphinxVerbatim}[commandchars=\\\{\}]
\PYG{k+kn}{import} \PYG{n+nn}{numpy} \PYG{k}{as} \PYG{n+nn}{np}
\PYG{k+kn}{import} \PYG{n+nn}{matplotlib}\PYG{n+nn}{.}\PYG{n+nn}{pyplot} \PYG{k}{as} \PYG{n+nn}{plt}
\PYG{k+kn}{import} \PYG{n+nn}{pandas} \PYG{k}{as} \PYG{n+nn}{pd} 
\PYG{k+kn}{import} \PYG{n+nn}{panel} \PYG{k}{as} \PYG{n+nn}{pn}
\PYG{n}{pn}\PYG{o}{.}\PYG{n}{extension}\PYG{p}{(}\PYG{l+s+s1}{\PYGZsq{}}\PYG{l+s+s1}{katex}\PYG{l+s+s1}{\PYGZsq{}}\PYG{p}{)} 
\end{sphinxVerbatim}

\end{sphinxuseclass}\end{sphinxVerbatimInput}

\end{sphinxuseclass}
\end{sphinxuseclass}

\chapter{Tutorial 4 \sphinxhyphen{} Effective K \& Recitation}
\label{\detokenize{content/tutorials/T4/tutorial_04:tutorial-4-effective-k-recitation}}\label{\detokenize{content/tutorials/T4/tutorial_04::doc}}\begin{itemize}
\item {} 
\sphinxAtStartPar
\sphinxstylestrong{solutions for homework problems 1 – 4}

\item {} 
\sphinxAtStartPar
\sphinxstylestrong{tutorial problems on effective conductivity and flow nets}

\item {} 
\sphinxAtStartPar
\sphinxstylestrong{homework problems on effective conductivity and flow nets}

\end{itemize}


\section{Solutions for Homework Problems 1 – 2}
\label{\detokenize{content/tutorials/T4/tutorial_04:solutions-for-homework-problems-1-2}}

\section{Homework Problems 1}
\label{\detokenize{content/tutorials/T4/tutorial_04:homework-problems-1}}
\begin{sphinxuseclass}{cell}
\begin{sphinxuseclass}{tag_hide-input}\begin{sphinxVerbatimOutput}

\begin{sphinxuseclass}{cell_output}
\begin{sphinxVerbatim}[commandchars=\\\{\}]
Column
    [0] Markdown(str, style=\PYGZob{}\PYGZsq{}font\PYGZhy{}size\PYGZsq{}: \PYGZsq{}12pt\PYGZsq{}\PYGZcb{}, width=800)
    [1] PNG(str, width=350)
    [2] Markdown(str, style=\PYGZob{}\PYGZsq{}font\PYGZhy{}size\PYGZsq{}: \PYGZsq{}12pt\PYGZsq{}\PYGZcb{}, width=800)
    [3] LaTeX(str, style=\PYGZob{}\PYGZsq{}font\PYGZhy{}size\PYGZsq{}: \PYGZsq{}12pt\PYGZsq{}\PYGZcb{}, width=800)
\end{sphinxVerbatim}

\end{sphinxuseclass}\end{sphinxVerbatimOutput}

\end{sphinxuseclass}
\end{sphinxuseclass}
\begin{sphinxuseclass}{cell}\begin{sphinxVerbatimInput}

\begin{sphinxuseclass}{cell_input}
\begin{sphinxVerbatim}[commandchars=\\\{\}]
\PYG{c+c1}{\PYGZsh{} Given }
\PYG{n}{A} \PYG{o}{=} \PYG{l+m+mi}{200} \PYG{c+c1}{\PYGZsh{} km\PYGZca{}2, aquifer area}
\PYG{n}{D\PYGZus{}h} \PYG{o}{=} \PYG{l+m+mf}{1.6}\PYG{o}{*}\PYG{l+m+mf}{0.05} \PYG{c+c1}{\PYGZsh{} m, head decrease}
\PYG{n}{S\PYGZus{}u} \PYG{o}{=} \PYG{l+m+mf}{0.13} \PYG{c+c1}{\PYGZsh{} (\PYGZhy{}), Storativity unconfined aquifer}
\PYG{n}{S\PYGZus{}c} \PYG{o}{=} \PYG{l+m+mf}{0.0005} \PYG{c+c1}{\PYGZsh{} (\PYGZhy{}) Storage coefficient, confined aquifer}

\PYG{c+c1}{\PYGZsh{} interim calculation}
\PYG{n}{A\PYGZus{}m2} \PYG{o}{=} \PYG{n}{A}\PYG{o}{*}\PYG{l+m+mi}{10}\PYG{o}{*}\PYG{o}{*}\PYG{l+m+mi}{6} \PYG{c+c1}{\PYGZsh{} m\PYGZca{}2, aquifer area unit converted}

\PYG{c+c1}{\PYGZsh{} Solution}
\PYG{n}{DV\PYGZus{}wu} \PYG{o}{=} \PYG{n}{A\PYGZus{}m2}\PYG{o}{*}\PYG{n}{S\PYGZus{}u}\PYG{o}{*}\PYG{n}{D\PYGZus{}h} \PYG{c+c1}{\PYGZsh{} m\PYGZca{}3, change in water volume unconfined aquifer}
\PYG{n}{DV\PYGZus{}wc} \PYG{o}{=} \PYG{n}{A\PYGZus{}m2}\PYG{o}{*}\PYG{n}{S\PYGZus{}c}\PYG{o}{*}\PYG{n}{D\PYGZus{}h}  \PYG{c+c1}{\PYGZsh{} m\PYGZca{}3 change in water volume unconfined aquifer}

\PYG{c+c1}{\PYGZsh{} output}

\PYG{n+nb}{print}\PYG{p}{(}\PYG{l+s+s2}{\PYGZdq{}}\PYG{l+s+s2}{Change in water volume in unconfined aquifer is: }\PYG{l+s+si}{\PYGZob{}0:1.1e\PYGZcb{}}\PYG{l+s+s2}{\PYGZdq{}}\PYG{o}{.}\PYG{n}{format}\PYG{p}{(}\PYG{n}{DV\PYGZus{}wu}\PYG{p}{)}\PYG{p}{,}\PYG{l+s+s2}{\PYGZdq{}}\PYG{l+s+s2}{m}\PYG{l+s+se}{\PYGZbs{}u00b3}\PYG{l+s+s2}{ }\PYG{l+s+se}{\PYGZbs{}n}\PYG{l+s+s2}{\PYGZdq{}}\PYG{p}{)}
\PYG{n+nb}{print}\PYG{p}{(}\PYG{l+s+s2}{\PYGZdq{}}\PYG{l+s+s2}{Change in water volume in confined aquifer is: }\PYG{l+s+si}{\PYGZob{}0:1.1e\PYGZcb{}}\PYG{l+s+s2}{\PYGZdq{}}\PYG{o}{.}\PYG{n}{format}\PYG{p}{(}\PYG{n}{DV\PYGZus{}wc}\PYG{p}{)}\PYG{p}{,}\PYG{l+s+s2}{\PYGZdq{}}\PYG{l+s+s2}{m}\PYG{l+s+se}{\PYGZbs{}u00b3}\PYG{l+s+s2}{\PYGZdq{}}\PYG{p}{)}
\end{sphinxVerbatim}

\end{sphinxuseclass}\end{sphinxVerbatimInput}
\begin{sphinxVerbatimOutput}

\begin{sphinxuseclass}{cell_output}
\begin{sphinxVerbatim}[commandchars=\\\{\}]
Change in water volume in unconfined aquifer is: 2.1e+06 m³ 

Change in water volume in confined aquifer is: 8.0e+03 m³
\end{sphinxVerbatim}

\end{sphinxuseclass}\end{sphinxVerbatimOutput}

\end{sphinxuseclass}

\section{Homework Problem 2}
\label{\detokenize{content/tutorials/T4/tutorial_04:homework-problem-2}}
\sphinxAtStartPar
Conduct a sieve analysis for a dried soil sample (see data in the table below)
\begin{enumerate}
\sphinxsetlistlabels{\arabic}{enumi}{enumii}{}{.}%
\item {} 
\sphinxAtStartPar
Draw the granulometric curve (cumulative mass distribution) and briefly characterise the sediment with regard to its major constituent(s).

\item {} 
\sphinxAtStartPar
What is the coefficient of uniformity?

\end{enumerate}

\begin{sphinxuseclass}{cell}\begin{sphinxVerbatimInput}

\begin{sphinxuseclass}{cell_input}
\begin{sphinxVerbatim}[commandchars=\\\{\}]
\PYG{c+c1}{\PYGZsh{}\PYGZsh{}\PYGZsh{} Head = [\PYGZdq{}mesh size [mm]\PYGZdq{}, \PYGZdq{}residue in the sieve [g] \PYGZdq{}, \PYGZdq{}∑ total\PYGZdq{}, \PYGZdq{}∑ / ∑total\PYGZdq{}]}
\PYG{n}{Size} \PYG{o}{=} \PYG{p}{[}\PYG{l+m+mf}{6.3}\PYG{p}{,} \PYG{l+m+mi}{2}\PYG{p}{,} \PYG{l+m+mf}{0.63}\PYG{p}{,} \PYG{l+m+mf}{0.2}\PYG{p}{,} \PYG{l+m+mf}{0.063}\PYG{p}{,} \PYG{l+s+s2}{\PYGZdq{}}\PYG{l+s+s2}{\PYGZlt{} 0.063 /cup}\PYG{l+s+s2}{\PYGZdq{}}\PYG{p}{]}
\PYG{n}{residue} \PYG{o}{=} \PYG{p}{[}\PYG{l+m+mi}{11}\PYG{p}{,} \PYG{l+m+mi}{62}\PYG{p}{,} \PYG{l+m+mi}{288}\PYG{p}{,} \PYG{l+m+mi}{189}\PYG{p}{,} \PYG{l+m+mi}{42}\PYG{p}{,} \PYG{l+m+mi}{8}\PYG{p}{]}


\PYG{n}{data1}\PYG{o}{=} \PYG{p}{\PYGZob{}}\PYG{l+s+s2}{\PYGZdq{}}\PYG{l+s+s2}{mesh size [mm]}\PYG{l+s+s2}{\PYGZdq{}}\PYG{p}{:} \PYG{n}{Size}\PYG{p}{,} \PYG{l+s+s2}{\PYGZdq{}}\PYG{l+s+s2}{residue in the sieve [g] }\PYG{l+s+s2}{\PYGZdq{}}\PYG{p}{:} \PYG{n}{residue}\PYG{p}{\PYGZcb{}}
\PYG{n}{df1}\PYG{o}{=} \PYG{n}{pd}\PYG{o}{.}\PYG{n}{DataFrame}\PYG{p}{(}\PYG{n}{data1}\PYG{p}{)}
\PYG{n}{df1}\PYG{o}{.}\PYG{n}{set\PYGZus{}index}\PYG{p}{(}\PYG{l+s+s2}{\PYGZdq{}}\PYG{l+s+s2}{mesh size [mm]}\PYG{l+s+s2}{\PYGZdq{}}\PYG{p}{)}
\end{sphinxVerbatim}

\end{sphinxuseclass}\end{sphinxVerbatimInput}
\begin{sphinxVerbatimOutput}

\begin{sphinxuseclass}{cell_output}
\begin{sphinxVerbatim}[commandchars=\\\{\}]
                residue in the sieve [g] 
mesh size [mm]                           
6.3                                    11
2                                      62
0.63                                  288
0.2                                   189
0.063                                  42
\PYGZlt{} 0.063 /cup                            8
\end{sphinxVerbatim}

\end{sphinxuseclass}\end{sphinxVerbatimOutput}

\end{sphinxuseclass}

\subsection{Solution of problem 2}
\label{\detokenize{content/tutorials/T4/tutorial_04:solution-of-problem-2}}
\begin{sphinxuseclass}{cell}\begin{sphinxVerbatimInput}

\begin{sphinxuseclass}{cell_input}
\begin{sphinxVerbatim}[commandchars=\\\{\}]
\PYG{c+c1}{\PYGZsh{} solution of problem 2}

\PYG{n}{t\PYGZus{}sample} \PYG{o}{=} \PYG{n}{np}\PYG{o}{.}\PYG{n}{sum}\PYG{p}{(}\PYG{n}{residue}\PYG{p}{)} \PYG{c+c1}{\PYGZsh{} g, add the residue column to get total mass}
\PYG{n}{retain\PYGZus{}per} \PYG{o}{=} \PYG{n}{np}\PYG{o}{.}\PYG{n}{round}\PYG{p}{(}\PYG{n}{residue}\PYG{o}{/}\PYG{n}{t\PYGZus{}sample} \PYG{o}{*}\PYG{l+m+mi}{100}\PYG{p}{,}\PYG{l+m+mi}{2}\PYG{p}{)} \PYG{c+c1}{\PYGZsh{} \PYGZpc{}, \PYGZsh{} retain percentage residue/total mass}
\PYG{n}{retain\PYGZus{}per\PYGZus{}cumsum} \PYG{o}{=}\PYG{n}{np}\PYG{o}{.}\PYG{n}{round}\PYG{p}{(}\PYG{n}{np}\PYG{o}{.}\PYG{n}{cumsum}\PYG{p}{(}\PYG{n}{retain\PYGZus{}per}\PYG{p}{)}\PYG{p}{,}\PYG{l+m+mi}{2}\PYG{p}{)} \PYG{c+c1}{\PYGZsh{} get the cummulative sum of the reatined}
\PYG{n}{passing\PYGZus{}per} \PYG{o}{=} \PYG{n}{np}\PYG{o}{.}\PYG{n}{round}\PYG{p}{(}\PYG{l+m+mi}{100} \PYG{o}{\PYGZhy{}} \PYG{n}{retain\PYGZus{}per\PYGZus{}cumsum}\PYG{p}{,}\PYG{l+m+mi}{3}\PYG{p}{)} \PYG{c+c1}{\PYGZsh{} substract 100\PYGZhy{}cummsum to get passing \PYGZpc{} \PYGZhy{} the last column}

\PYG{n}{data2} \PYG{o}{=} \PYG{p}{\PYGZob{}}\PYG{l+s+s2}{\PYGZdq{}}\PYG{l+s+s2}{mesh size [mm]}\PYG{l+s+s2}{\PYGZdq{}}\PYG{p}{:} \PYG{n}{Size}\PYG{p}{,} \PYG{l+s+s2}{\PYGZdq{}}\PYG{l+s+s2}{residue in the sieve [g]}\PYG{l+s+s2}{\PYGZdq{}}\PYG{p}{:} \PYG{n}{residue}\PYG{p}{,} \PYG{l+s+s2}{\PYGZdq{}}\PYG{l+s+s2}{Σtotal }\PYG{l+s+s2}{\PYGZpc{}}\PYG{l+s+s2}{\PYGZdq{}}\PYG{p}{:} \PYG{n}{retain\PYGZus{}per}\PYG{p}{,} \PYG{l+s+s2}{\PYGZdq{}}\PYG{l+s+s2}{Σ/Σtotal }\PYG{l+s+s2}{\PYGZpc{}}\PYG{l+s+s2}{\PYGZdq{}}\PYG{p}{:} \PYG{n}{passing\PYGZus{}per} \PYG{p}{\PYGZcb{}}

\PYG{n}{df2}\PYG{o}{=} \PYG{n}{pd}\PYG{o}{.}\PYG{n}{DataFrame}\PYG{p}{(}\PYG{n}{data2}\PYG{p}{)}
\PYG{n}{df2} \PYG{o}{=} \PYG{n}{df2}\PYG{o}{.}\PYG{n}{set\PYGZus{}index}\PYG{p}{(}\PYG{l+s+s2}{\PYGZdq{}}\PYG{l+s+s2}{mesh size [mm]}\PYG{l+s+s2}{\PYGZdq{}}\PYG{p}{)}
\PYG{n}{df2}
\end{sphinxVerbatim}

\end{sphinxuseclass}\end{sphinxVerbatimInput}
\begin{sphinxVerbatimOutput}

\begin{sphinxuseclass}{cell_output}
\begin{sphinxVerbatim}[commandchars=\\\{\}]
                residue in the sieve [g]  Σtotal \PYGZpc{}  Σ/Σtotal \PYGZpc{}
mesh size [mm]                                                
6.3                                   11      1.83       98.17
2                                     62     10.33       87.84
0.63                                 288     48.00       39.84
0.2                                  189     31.50        8.34
0.063                                 42      7.00        1.34
\PYGZlt{} 0.063 /cup                           8      1.33        0.01
\end{sphinxVerbatim}

\end{sphinxuseclass}\end{sphinxVerbatimOutput}

\end{sphinxuseclass}
\begin{sphinxuseclass}{cell}\begin{sphinxVerbatimInput}

\begin{sphinxuseclass}{cell_input}
\begin{sphinxVerbatim}[commandchars=\\\{\}]
\PYG{c+c1}{\PYGZsh{} Plotting granulometric curve}

\PYG{n}{plt}\PYG{o}{.}\PYG{n}{rcParams}\PYG{p}{[}\PYG{l+s+s1}{\PYGZsq{}}\PYG{l+s+s1}{axes.linewidth}\PYG{l+s+s1}{\PYGZsq{}}\PYG{p}{]}\PYG{o}{=}\PYG{l+m+mi}{2}
\PYG{n}{plt}\PYG{o}{.}\PYG{n}{rcParams}\PYG{p}{[}\PYG{l+s+s1}{\PYGZsq{}}\PYG{l+s+s1}{grid.linestyle}\PYG{l+s+s1}{\PYGZsq{}}\PYG{p}{]}\PYG{o}{=}\PYG{l+s+s1}{\PYGZsq{}}\PYG{l+s+s1}{\PYGZhy{}\PYGZhy{}}\PYG{l+s+s1}{\PYGZsq{}}
\PYG{n}{plt}\PYG{o}{.}\PYG{n}{rcParams}\PYG{p}{[}\PYG{l+s+s1}{\PYGZsq{}}\PYG{l+s+s1}{grid.linewidth}\PYG{l+s+s1}{\PYGZsq{}}\PYG{p}{]}\PYG{o}{=}\PYG{l+m+mi}{1}

\PYG{n}{x} \PYG{o}{=} \PYG{n}{np}\PYG{o}{.}\PYG{n}{append}\PYG{p}{(}\PYG{p}{[}\PYG{l+m+mi}{20}\PYG{p}{]}\PYG{p}{,} \PYG{n}{Size}\PYG{p}{[}\PYG{p}{:}\PYG{l+m+mi}{5}\PYG{p}{]}\PYG{p}{)} \PYG{c+c1}{\PYGZsh{} adding for all left over.}
\PYG{n}{y} \PYG{o}{=} \PYG{n}{np}\PYG{o}{.}\PYG{n}{append}\PYG{p}{(}\PYG{p}{[}\PYG{l+m+mi}{100}\PYG{p}{]}\PYG{p}{,}\PYG{n}{passing\PYGZus{}per}\PYG{p}{[}\PYG{p}{:}\PYG{l+m+mi}{5}\PYG{p}{]}\PYG{p}{)}

\PYG{n}{fig} \PYG{o}{=} \PYG{n}{plt}\PYG{o}{.}\PYG{n}{figure}\PYG{p}{(}\PYG{n}{figsize}\PYG{o}{=}\PYG{p}{(}\PYG{l+m+mi}{9}\PYG{p}{,}\PYG{l+m+mi}{6}\PYG{p}{)}\PYG{p}{)}\PYG{p}{;}

\PYG{n}{plt}\PYG{o}{.}\PYG{n}{plot}\PYG{p}{(}\PYG{n}{x}\PYG{p}{,} \PYG{n}{y}\PYG{p}{,} \PYG{l+s+s1}{\PYGZsq{}}\PYG{l+s+s1}{x\PYGZhy{}}\PYG{l+s+s1}{\PYGZsq{}}\PYG{p}{,} \PYG{n}{color}\PYG{o}{=}\PYG{l+s+s1}{\PYGZsq{}}\PYG{l+s+s1}{red}\PYG{l+s+s1}{\PYGZsq{}}\PYG{p}{,} \PYG{n}{lw}\PYG{o}{=}\PYG{l+m+mf}{2.5}\PYG{p}{)}\PYG{p}{;} 
\PYG{n}{tics}\PYG{o}{=}\PYG{n}{x}\PYG{o}{.}\PYG{n}{tolist}\PYG{p}{(}\PYG{p}{)}

\PYG{n}{plt}\PYG{o}{.}\PYG{n}{xscale}\PYG{p}{(}\PYG{l+s+s1}{\PYGZsq{}}\PYG{l+s+s1}{log}\PYG{l+s+s1}{\PYGZsq{}}\PYG{p}{)}\PYG{p}{;}\PYG{n}{lw}\PYG{o}{=}\PYG{l+m+mf}{2.5}

\PYG{n}{plt}\PYG{o}{.}\PYG{n}{grid}\PYG{p}{(}\PYG{n}{which}\PYG{o}{=}\PYG{l+s+s1}{\PYGZsq{}}\PYG{l+s+s1}{major}\PYG{l+s+s1}{\PYGZsq{}}\PYG{p}{,} \PYG{n}{color}\PYG{o}{=}\PYG{l+s+s1}{\PYGZsq{}}\PYG{l+s+s1}{k}\PYG{l+s+s1}{\PYGZsq{}}\PYG{p}{,} \PYG{n}{alpha}\PYG{o}{=}\PYG{l+m+mf}{0.7}\PYG{p}{)} 
\PYG{n}{plt}\PYG{o}{.}\PYG{n}{grid}\PYG{p}{(}\PYG{n}{which}\PYG{o}{=}\PYG{l+s+s1}{\PYGZsq{}}\PYG{l+s+s1}{minor}\PYG{l+s+s1}{\PYGZsq{}}\PYG{p}{,} \PYG{n}{color}\PYG{o}{=}\PYG{l+s+s1}{\PYGZsq{}}\PYG{l+s+s1}{k}\PYG{l+s+s1}{\PYGZsq{}}\PYG{p}{,} \PYG{n}{alpha}\PYG{o}{=}\PYG{l+m+mf}{0.3}\PYG{p}{)}
\PYG{n}{plt}\PYG{o}{.}\PYG{n}{xticks}\PYG{p}{(}\PYG{n}{x}\PYG{p}{,} \PYG{n}{tics}\PYG{p}{)}\PYG{p}{;}  
\PYG{n}{plt}\PYG{o}{.}\PYG{n}{yticks}\PYG{p}{(}\PYG{n}{np}\PYG{o}{.}\PYG{n}{arange}\PYG{p}{(}\PYG{l+m+mi}{0}\PYG{p}{,}\PYG{l+m+mi}{110}\PYG{p}{,}\PYG{l+m+mi}{10}\PYG{p}{)}\PYG{p}{)}\PYG{p}{;}

\PYG{c+c1}{\PYGZsh{}plt.title(\PYGZsq{}grain size distribution (combined wet sieving and sedimentation analysis)\PYGZsq{});}
\PYG{n}{plt}\PYG{o}{.}\PYG{n}{xlabel}\PYG{p}{(}\PYG{l+s+s1}{\PYGZsq{}}\PYG{l+s+s1}{grain size d [mm]}\PYG{l+s+s1}{\PYGZsq{}}\PYG{p}{)}\PYG{p}{;}
\PYG{n}{plt}\PYG{o}{.}\PYG{n}{ylabel}\PYG{p}{(}\PYG{l+s+s1}{\PYGZsq{}}\PYG{l+s+s1}{Cummulative Passed fraction }\PYG{l+s+s1}{\PYGZpc{}}\PYG{l+s+s1}{\PYGZsq{}}\PYG{p}{)}\PYG{p}{;}

\PYG{n}{plt}\PYG{o}{.}\PYG{n}{annotate}\PYG{p}{(}\PYG{l+s+s1}{\PYGZsq{}}\PYG{l+s+s1}{\PYGZsq{}}\PYG{p}{,} \PYG{n}{xy}\PYG{o}{=}\PYG{p}{(}\PYG{l+m+mf}{0.20}\PYG{p}{,} \PYG{l+m+mi}{10}\PYG{p}{)}\PYG{p}{,}  \PYG{n}{xycoords}\PYG{o}{=}\PYG{l+s+s1}{\PYGZsq{}}\PYG{l+s+s1}{data}\PYG{l+s+s1}{\PYGZsq{}}\PYG{p}{,} \PYG{n}{xytext}\PYG{o}{=}\PYG{p}{(}\PYG{l+m+mf}{0.045}\PYG{p}{,} \PYG{l+m+mi}{10}\PYG{p}{)}\PYG{p}{,} \PYG{n}{arrowprops}\PYG{o}{=}\PYG{n+nb}{dict}\PYG{p}{(}\PYG{n}{arrowstyle}\PYG{o}{=}\PYG{l+s+s1}{\PYGZsq{}}\PYG{l+s+s1}{\PYGZhy{}\PYGZgt{}}\PYG{l+s+s1}{\PYGZsq{}}\PYG{p}{,} \PYG{n}{color}\PYG{o}{=}\PYG{l+s+s2}{\PYGZdq{}}\PYG{l+s+s2}{b}\PYG{l+s+s2}{\PYGZdq{}}\PYG{p}{,} \PYG{n}{lw}\PYG{o}{=}\PYG{l+m+mf}{2.5}\PYG{p}{)}\PYG{p}{,}\PYG{n}{ha}\PYG{o}{=}\PYG{l+s+s1}{\PYGZsq{}}\PYG{l+s+s1}{right}\PYG{l+s+s1}{\PYGZsq{}}\PYG{p}{,} \PYG{n}{va}\PYG{o}{=}\PYG{l+s+s1}{\PYGZsq{}}\PYG{l+s+s1}{top}\PYG{l+s+s1}{\PYGZsq{}}\PYG{p}{,}\PYG{p}{)}
\PYG{n}{plt}\PYG{o}{.}\PYG{n}{annotate}\PYG{p}{(}\PYG{l+s+s1}{\PYGZsq{}}\PYG{l+s+s1}{\PYGZsq{}}\PYG{p}{,} \PYG{n}{xy}\PYG{o}{=}\PYG{p}{(}\PYG{l+m+mf}{1.1}\PYG{p}{,} \PYG{l+m+mi}{60}\PYG{p}{)}\PYG{p}{,}  \PYG{n}{xycoords}\PYG{o}{=}\PYG{l+s+s1}{\PYGZsq{}}\PYG{l+s+s1}{data}\PYG{l+s+s1}{\PYGZsq{}}\PYG{p}{,} \PYG{n}{xytext}\PYG{o}{=}\PYG{p}{(}\PYG{l+m+mf}{0.045}\PYG{p}{,} \PYG{l+m+mi}{60}\PYG{p}{)}\PYG{p}{,} \PYG{n}{arrowprops}\PYG{o}{=}\PYG{n+nb}{dict}\PYG{p}{(}\PYG{n}{arrowstyle}\PYG{o}{=}\PYG{l+s+s1}{\PYGZsq{}}\PYG{l+s+s1}{\PYGZhy{}\PYGZgt{}}\PYG{l+s+s1}{\PYGZsq{}}\PYG{p}{,} \PYG{n}{color}\PYG{o}{=}\PYG{l+s+s2}{\PYGZdq{}}\PYG{l+s+s2}{b}\PYG{l+s+s2}{\PYGZdq{}}\PYG{p}{,} \PYG{n}{lw}\PYG{o}{=}\PYG{l+m+mf}{2.5}\PYG{p}{)}\PYG{p}{,}\PYG{n}{ha}\PYG{o}{=}\PYG{l+s+s1}{\PYGZsq{}}\PYG{l+s+s1}{right}\PYG{l+s+s1}{\PYGZsq{}}\PYG{p}{,} \PYG{n}{va}\PYG{o}{=}\PYG{l+s+s1}{\PYGZsq{}}\PYG{l+s+s1}{top}\PYG{l+s+s1}{\PYGZsq{}}\PYG{p}{,}\PYG{p}{)}
\PYG{n}{plt}\PYG{o}{.}\PYG{n}{annotate}\PYG{p}{(}\PYG{l+s+sa}{r}\PYG{l+s+s1}{\PYGZsq{}}\PYG{l+s+s1}{\PYGZdl{}d\PYGZus{}}\PYG{l+s+si}{\PYGZob{}60\PYGZcb{}}\PYG{l+s+s1}{\PYGZdl{}}\PYG{l+s+s1}{\PYGZsq{}}\PYG{p}{,} \PYG{n}{xy}\PYG{o}{=}\PYG{p}{(}\PYG{l+m+mi}{1}\PYG{p}{,} \PYG{l+m+mi}{60}\PYG{p}{)}\PYG{p}{,}  \PYG{n}{xycoords}\PYG{o}{=}\PYG{l+s+s2}{\PYGZdq{}}\PYG{l+s+s2}{data}\PYG{l+s+s2}{\PYGZdq{}}\PYG{p}{,} \PYG{n}{xytext}\PYG{o}{=}\PYG{p}{(}\PYG{l+m+mf}{0.85}\PYG{p}{,} \PYG{o}{\PYGZhy{}}\PYG{l+m+mi}{3}\PYG{p}{)}\PYG{p}{,}\PYG{n}{color}\PYG{o}{=}\PYG{l+s+s1}{\PYGZsq{}}\PYG{l+s+s1}{red}\PYG{l+s+s1}{\PYGZsq{}}\PYG{p}{,}\PYG{n}{size}\PYG{o}{=}\PYG{l+m+mi}{12}\PYG{p}{,} \PYG{n}{arrowprops}\PYG{o}{=}\PYG{n+nb}{dict}\PYG{p}{(}\PYG{n}{arrowstyle}\PYG{o}{=}\PYG{l+s+s1}{\PYGZsq{}}\PYG{l+s+s1}{\PYGZlt{}\PYGZhy{}}\PYG{l+s+s1}{\PYGZsq{}}\PYG{p}{,} \PYG{n}{color}\PYG{o}{=}\PYG{l+s+s2}{\PYGZdq{}}\PYG{l+s+s2}{b}\PYG{l+s+s2}{\PYGZdq{}}\PYG{p}{,} \PYG{n}{lw}\PYG{o}{=}\PYG{l+m+mf}{2.5}\PYG{p}{)}\PYG{p}{,}\PYG{n}{ha}\PYG{o}{=}\PYG{l+s+s1}{\PYGZsq{}}\PYG{l+s+s1}{left}\PYG{l+s+s1}{\PYGZsq{}}\PYG{p}{,} \PYG{n}{va}\PYG{o}{=}\PYG{l+s+s1}{\PYGZsq{}}\PYG{l+s+s1}{bottom}\PYG{l+s+s1}{\PYGZsq{}}\PYG{p}{,}\PYG{p}{)}
\PYG{n}{plt}\PYG{o}{.}\PYG{n}{annotate}\PYG{p}{(}\PYG{l+s+sa}{r}\PYG{l+s+s1}{\PYGZsq{}}\PYG{l+s+s1}{\PYGZdl{}d\PYGZus{}}\PYG{l+s+si}{\PYGZob{}10\PYGZcb{}}\PYG{l+s+s1}{\PYGZdl{}}\PYG{l+s+s1}{\PYGZsq{}}\PYG{p}{,} \PYG{n}{xy}\PYG{o}{=}\PYG{p}{(}\PYG{l+m+mf}{0.20}\PYG{p}{,} \PYG{l+m+mi}{10}\PYG{p}{)}\PYG{p}{,}  \PYG{n}{xycoords}\PYG{o}{=}\PYG{l+s+s1}{\PYGZsq{}}\PYG{l+s+s1}{data}\PYG{l+s+s1}{\PYGZsq{}}\PYG{p}{,} \PYG{n}{xytext}\PYG{o}{=}\PYG{p}{(}\PYG{l+m+mf}{0.235}\PYG{p}{,} \PYG{l+m+mf}{1.5}\PYG{p}{)}\PYG{p}{,}\PYG{n}{color}\PYG{o}{=}\PYG{l+s+s1}{\PYGZsq{}}\PYG{l+s+s1}{red}\PYG{l+s+s1}{\PYGZsq{}}\PYG{p}{,}\PYG{n}{size}\PYG{o}{=}\PYG{l+m+mi}{12}\PYG{p}{,} \PYG{n}{arrowprops}\PYG{o}{=}\PYG{n+nb}{dict}\PYG{p}{(}\PYG{n}{arrowstyle}\PYG{o}{=}\PYG{l+s+s1}{\PYGZsq{}}\PYG{l+s+s1}{\PYGZlt{}\PYGZhy{}}\PYG{l+s+s1}{\PYGZsq{}}\PYG{p}{,} \PYG{n}{color}\PYG{o}{=}\PYG{l+s+s2}{\PYGZdq{}}\PYG{l+s+s2}{b}\PYG{l+s+s2}{\PYGZdq{}}\PYG{p}{,} \PYG{n}{lw}\PYG{o}{=}\PYG{l+m+mf}{2.5}\PYG{p}{)}\PYG{p}{,}\PYG{n}{ha}\PYG{o}{=}\PYG{l+s+s1}{\PYGZsq{}}\PYG{l+s+s1}{right}\PYG{l+s+s1}{\PYGZsq{}}\PYG{p}{,} \PYG{n}{va}\PYG{o}{=}\PYG{l+s+s1}{\PYGZsq{}}\PYG{l+s+s1}{top}\PYG{l+s+s1}{\PYGZsq{}}\PYG{p}{,}\PYG{p}{)}
\PYG{n}{plt}\PYG{o}{.}\PYG{n}{rcParams}\PYG{p}{[}\PYG{l+s+s2}{\PYGZdq{}}\PYG{l+s+s2}{font.weight}\PYG{l+s+s2}{\PYGZdq{}}\PYG{p}{]} \PYG{o}{=} \PYG{l+s+s2}{\PYGZdq{}}\PYG{l+s+s2}{bold}\PYG{l+s+s2}{\PYGZdq{}}   

\PYG{n}{plt}\PYG{o}{.}\PYG{n}{savefig}\PYG{p}{(}\PYG{l+s+s2}{\PYGZdq{}}\PYG{l+s+s2}{fig6.png}\PYG{l+s+s2}{\PYGZdq{}}\PYG{p}{,} \PYG{n}{dpi}\PYG{o}{=}\PYG{l+m+mi}{300}\PYG{p}{)}

\PYG{n}{mpl\PYGZus{}pane} \PYG{o}{=} \PYG{n}{pn}\PYG{o}{.}\PYG{n}{pane}\PYG{o}{.}\PYG{n}{Matplotlib}\PYG{p}{(}\PYG{n}{fig}\PYG{p}{)}
\end{sphinxVerbatim}

\end{sphinxuseclass}\end{sphinxVerbatimInput}
\begin{sphinxVerbatimOutput}

\begin{sphinxuseclass}{cell_output}
\noindent\sphinxincludegraphics{{C:/Users/vibhu/GWtextbook/_build/jupyter_execute/tutorial_04_8_0}.png}

\end{sphinxuseclass}\end{sphinxVerbatimOutput}

\end{sphinxuseclass}
\begin{sphinxuseclass}{cell}\begin{sphinxVerbatimInput}

\begin{sphinxuseclass}{cell_input}
\begin{sphinxVerbatim}[commandchars=\\\{\}]
\PYG{c+c1}{\PYGZsh{} From the figure}
\PYG{n}{d\PYGZus{}10} \PYG{o}{=} \PYG{l+m+mf}{0.22} \PYG{c+c1}{\PYGZsh{} mm,approx, diameter 10\PYGZpc{} passing, see the arrow bottom in x\PYGZhy{}axis}
\PYG{n}{d\PYGZus{}60} \PYG{o}{=} \PYG{l+m+mf}{1.0} \PYG{c+c1}{\PYGZsh{} mm, approx diameter 10\PYGZpc{} passing, see the arrow bottom in x\PYGZhy{}axis}

\PYG{n}{c\PYGZus{}u} \PYG{o}{=} \PYG{n}{d\PYGZus{}60}\PYG{o}{/}\PYG{n}{d\PYGZus{}10} \PYG{c+c1}{\PYGZsh{} [], coefficient of uniformity}

\PYG{c+c1}{\PYGZsh{}Output}
\PYG{n+nb}{print}\PYG{p}{(}\PYG{l+s+s2}{\PYGZdq{}}\PYG{l+s+se}{\PYGZbs{}n}\PYG{l+s+s2}{ The coefficient of uniformity is: }\PYG{l+s+si}{\PYGZob{}0:1.1f\PYGZcb{}}\PYG{l+s+s2}{\PYGZdq{}}\PYG{o}{.}\PYG{n}{format}\PYG{p}{(}\PYG{n}{c\PYGZus{}u}\PYG{p}{)}\PYG{p}{,} \PYG{l+s+s2}{\PYGZdq{}}\PYG{l+s+se}{\PYGZbs{}n}\PYG{l+s+s2}{\PYGZdq{}}\PYG{p}{)} 
\PYG{n}{r2\PYGZus{}1} \PYG{o}{=} \PYG{n}{pn}\PYG{o}{.}\PYG{n}{pane}\PYG{o}{.}\PYG{n}{Markdown}\PYG{p}{(}\PYG{l+s+s2}{\PYGZdq{}\PYGZdq{}\PYGZdq{}}
\PYG{l+s+s2}{**Major constituents: coarse sand/medium sand** }\PYG{l+s+s2}{\PYGZdq{}\PYGZdq{}\PYGZdq{}}\PYG{p}{,} \PYG{n}{width}\PYG{o}{=}\PYG{l+m+mi}{600}\PYG{p}{,} \PYG{n}{style}\PYG{o}{=}\PYG{p}{\PYGZob{}}\PYG{l+s+s1}{\PYGZsq{}}\PYG{l+s+s1}{font\PYGZhy{}size}\PYG{l+s+s1}{\PYGZsq{}}\PYG{p}{:} \PYG{l+s+s1}{\PYGZsq{}}\PYG{l+s+s1}{12pt}\PYG{l+s+s1}{\PYGZsq{}}\PYG{p}{,} \PYG{l+s+s1}{\PYGZsq{}}\PYG{l+s+s1}{color}\PYG{l+s+s1}{\PYGZsq{}}\PYG{p}{:} \PYG{l+s+s1}{\PYGZsq{}}\PYG{l+s+s1}{blue}\PYG{l+s+s1}{\PYGZsq{}}\PYG{p}{\PYGZcb{}} \PYG{p}{)}
\PYG{n}{pn}\PYG{o}{.}\PYG{n}{Row}\PYG{p}{(}\PYG{n}{r2\PYGZus{}1}\PYG{p}{)} 
\end{sphinxVerbatim}

\end{sphinxuseclass}\end{sphinxVerbatimInput}
\begin{sphinxVerbatimOutput}

\begin{sphinxuseclass}{cell_output}
\begin{sphinxVerbatim}[commandchars=\\\{\}]
 The coefficient of uniformity is: 4.5 
\end{sphinxVerbatim}

\begin{sphinxVerbatim}[commandchars=\\\{\}]
Row
    [0] Markdown(str, style=\PYGZob{}\PYGZsq{}font\PYGZhy{}size\PYGZsq{}: \PYGZsq{}12pt\PYGZsq{}, ...\PYGZcb{}, width=600)
\end{sphinxVerbatim}

\end{sphinxuseclass}\end{sphinxVerbatimOutput}

\end{sphinxuseclass}

\section{Tutorial Problems on effective conductivity}
\label{\detokenize{content/tutorials/T4/tutorial_04:tutorial-problems-on-effective-conductivity}}

\subsection{Tutorial problem 11}
\label{\detokenize{content/tutorials/T4/tutorial_04:tutorial-problem-11}}
\sphinxAtStartPar
A sandy layer with a thickness of 2.5 m is embedded between two gravel layers. Both gravel layers have a thickness of 1.5 m and a hydraulic conductivity of 3.7·10\sphinxhyphen{}3 m/s.
Steady\sphinxhyphen{}state groundwater flow is in parallel to the layering.
A hydraulic gradient of 0.001 and an overall discharge of 1 m³/d per unit width have been determined.

a. Determine the effective hydraulic conductivity.
b. What is the hydraulic conductivity of the sand layer?
c. Which effective hydraulic conductivity would be obtained if flow was assumed perpendicular to the layering?
d. Calculate effective hydraulic conductivity if the angle between the flow direction and the layering equals 45°.

\begin{sphinxuseclass}{cell}
\begin{sphinxuseclass}{tag_hide-input}\begin{sphinxVerbatimOutput}

\begin{sphinxuseclass}{cell_output}
\begin{sphinxVerbatim}[commandchars=\\\{\}]
Row
    [0] PNG(str, width=400)
    [1] Spacer(width=100)
    [2] LaTeX(str, style=\PYGZob{}\PYGZsq{}font\PYGZhy{}size\PYGZsq{}: \PYGZsq{}12pt\PYGZsq{}\PYGZcb{}, width=500)
\end{sphinxVerbatim}

\end{sphinxuseclass}\end{sphinxVerbatimOutput}

\end{sphinxuseclass}
\end{sphinxuseclass}
\begin{sphinxuseclass}{cell}\begin{sphinxVerbatimInput}

\begin{sphinxuseclass}{cell_input}
\begin{sphinxVerbatim}[commandchars=\\\{\}]
\PYG{c+c1}{\PYGZsh{}Given Solution of 11 a, b}

\PYG{n}{Q} \PYG{o}{=} \PYG{l+m+mi}{1} \PYG{c+c1}{\PYGZsh{} m\PYGZca{}3/d, discharge}
\PYG{n}{W} \PYG{o}{=} \PYG{l+m+mi}{1} \PYG{c+c1}{\PYGZsh{} m, per unit width}
\PYG{n}{K\PYGZus{}g} \PYG{o}{=} \PYG{l+m+mf}{3.7}\PYG{o}{*}\PYG{l+m+mf}{1E\PYGZhy{}3}\PYG{c+c1}{\PYGZsh{} m/s, conductivity of gravel layer }
\PYG{n}{m\PYGZus{}g} \PYG{o}{=} \PYG{l+m+mf}{1.5} \PYG{c+c1}{\PYGZsh{} m, thickness of gravel layer}
\PYG{n}{m\PYGZus{}s} \PYG{o}{=} \PYG{l+m+mf}{2.5} \PYG{c+c1}{\PYGZsh{} m, thickness of sand layer}
\PYG{n}{Dh\PYGZus{}L} \PYG{o}{=} \PYG{l+m+mf}{0.001} \PYG{c+c1}{\PYGZsh{} (\PYGZhy{}), hydraulic gradient}

\PYG{c+c1}{\PYGZsh{}interim calculation}
\PYG{n}{m} \PYG{o}{=} \PYG{l+m+mi}{2}\PYG{o}{*}\PYG{n}{m\PYGZus{}g} \PYG{o}{+} \PYG{n}{m\PYGZus{}s} \PYG{c+c1}{\PYGZsh{} m. total thickness of aquifer}

\PYG{c+c1}{\PYGZsh{}Solution of 11a}
\PYG{n}{Keff\PYGZus{}h} \PYG{o}{=} \PYG{p}{(}\PYG{n}{Q}\PYG{o}{/}\PYG{n}{W}\PYG{p}{)}\PYG{o}{/}\PYG{p}{(}\PYG{n}{m}\PYG{o}{*}\PYG{n}{Dh\PYGZus{}L}\PYG{p}{)} \PYG{c+c1}{\PYGZsh{} m/d, conductivity}
\PYG{n}{Keff\PYGZus{}hs} \PYG{o}{=} \PYG{n}{Keff\PYGZus{}h}\PYG{o}{/}\PYG{p}{(}\PYG{l+m+mi}{24}\PYG{o}{*}\PYG{l+m+mi}{3600}\PYG{p}{)}\PYG{c+c1}{\PYGZsh{} m/s, conductivity unit changed}

\PYG{c+c1}{\PYGZsh{}Solution of 11b}
\PYG{c+c1}{\PYGZsh{} K\PYGZus{}eff = (2*m\PYGZus{}g*K\PYGZus{}g + m\PYGZus{}s*K\PYGZus{}s)/m \PYGZsh{} (Keff*m\PYGZhy{}2*m\PYGZus{}g*K\PYGZus{}g)/m\PYGZus{}s}

\PYG{n}{K\PYGZus{}s} \PYG{o}{=} \PYG{p}{(}\PYG{p}{(}\PYG{n}{m}\PYG{o}{*}\PYG{n}{Keff\PYGZus{}hs} \PYG{o}{\PYGZhy{}} \PYG{l+m+mi}{2}\PYG{o}{*}\PYG{n}{m\PYGZus{}g}\PYG{o}{*}\PYG{n}{K\PYGZus{}g}\PYG{p}{)}\PYG{p}{)}\PYG{o}{/}\PYG{n}{m\PYGZus{}s}  \PYG{c+c1}{\PYGZsh{} m/s cond. of sand layer}

\PYG{n+nb}{print}\PYG{p}{(}\PYG{l+s+s2}{\PYGZdq{}}\PYG{l+s+se}{\PYGZbs{}n}\PYG{l+s+s2}{\PYGZdq{}}\PYG{p}{)}
\PYG{n+nb}{print}\PYG{p}{(}\PYG{l+s+s2}{\PYGZdq{}}\PYG{l+s+s2}{Effective horizontal hydraulic conductivity (Keff\PYGZus{}h) = }\PYG{l+s+si}{\PYGZob{}0:1.2f\PYGZcb{}}\PYG{l+s+s2}{\PYGZdq{}}\PYG{o}{.}\PYG{n}{format}\PYG{p}{(}\PYG{n}{Keff\PYGZus{}h}\PYG{p}{)}\PYG{p}{,} \PYG{l+s+s2}{\PYGZdq{}}\PYG{l+s+s2}{m/d}\PYG{l+s+se}{\PYGZbs{}n}\PYG{l+s+s2}{\PYGZdq{}} \PYG{p}{)} 
\PYG{n+nb}{print}\PYG{p}{(}\PYG{l+s+s2}{\PYGZdq{}}\PYG{l+s+s2}{Effective horizontal hydraulic conductivity (Keff\PYGZus{}hs) = }\PYG{l+s+si}{\PYGZob{}0:1.3E\PYGZcb{}}\PYG{l+s+s2}{\PYGZdq{}}\PYG{o}{.}\PYG{n}{format}\PYG{p}{(}\PYG{n}{Keff\PYGZus{}hs}\PYG{p}{)}\PYG{p}{,} \PYG{l+s+s2}{\PYGZdq{}}\PYG{l+s+s2}{m/s}\PYG{l+s+se}{\PYGZbs{}n}\PYG{l+s+s2}{\PYGZdq{}} \PYG{p}{)}
\PYG{n+nb}{print}\PYG{p}{(}\PYG{l+s+s2}{\PYGZdq{}}\PYG{l+s+s2}{Hydraulic conductivity of sand layer (K\PYGZus{}s) = }\PYG{l+s+si}{\PYGZob{}0:1.1E\PYGZcb{}}\PYG{l+s+s2}{\PYGZdq{}}\PYG{o}{.}\PYG{n}{format}\PYG{p}{(}\PYG{n}{K\PYGZus{}s}\PYG{p}{)}\PYG{p}{,} \PYG{l+s+s2}{\PYGZdq{}}\PYG{l+s+s2}{m/s}\PYG{l+s+se}{\PYGZbs{}n}\PYG{l+s+s2}{\PYGZdq{}} \PYG{p}{)}     
\end{sphinxVerbatim}

\end{sphinxuseclass}\end{sphinxVerbatimInput}
\begin{sphinxVerbatimOutput}

\begin{sphinxuseclass}{cell_output}
\begin{sphinxVerbatim}[commandchars=\\\{\}]
Effective horizontal hydraulic conductivity (Keff\PYGZus{}h) = 181.82 m/d

Effective horizontal hydraulic conductivity (Keff\PYGZus{}hs) = 2.104E\PYGZhy{}03 m/s

Hydraulic conductivity of sand layer (K\PYGZus{}s) = 1.9E\PYGZhy{}04 m/s
\end{sphinxVerbatim}

\end{sphinxuseclass}\end{sphinxVerbatimOutput}

\end{sphinxuseclass}
\begin{sphinxuseclass}{cell}
\begin{sphinxuseclass}{tag_hide-input}\begin{sphinxVerbatimOutput}

\begin{sphinxuseclass}{cell_output}
\begin{sphinxVerbatim}[commandchars=\\\{\}]
Row
    [0] Column
        [0] PNG(str, width=200)
        [1] PNG(str, width=200)
    [1] Spacer(width=100)
    [2] LaTeX(str, style=\PYGZob{}\PYGZsq{}font\PYGZhy{}size\PYGZsq{}: \PYGZsq{}13pt\PYGZsq{}\PYGZcb{})
\end{sphinxVerbatim}

\end{sphinxuseclass}\end{sphinxVerbatimOutput}

\end{sphinxuseclass}
\end{sphinxuseclass}
\begin{sphinxuseclass}{cell}\begin{sphinxVerbatimInput}

\begin{sphinxuseclass}{cell_input}
\begin{sphinxVerbatim}[commandchars=\\\{\}]
\PYG{c+c1}{\PYGZsh{} Solution of 11c}

\PYG{n}{Keff\PYGZus{}v} \PYG{o}{=} \PYG{n}{m}\PYG{o}{/}\PYG{p}{(}\PYG{l+m+mi}{2}\PYG{o}{*}\PYG{p}{(}\PYG{n}{m\PYGZus{}g}\PYG{o}{/}\PYG{n}{K\PYGZus{}g}\PYG{p}{)}\PYG{o}{+} \PYG{p}{(}\PYG{n}{m\PYGZus{}s}\PYG{o}{/}\PYG{n}{K\PYGZus{}s}\PYG{p}{)}\PYG{p}{)} \PYG{c+c1}{\PYGZsh{} m/s, conductivity K\PYGZus{}v}

\PYG{c+c1}{\PYGZsh{}Given }
\PYG{n}{theta} \PYG{o}{=} \PYG{l+m+mi}{45} \PYG{c+c1}{\PYGZsh{} theta }
\PYG{n}{theta\PYGZus{}r} \PYG{o}{=} \PYG{n}{theta}\PYG{o}{*}\PYG{p}{(}\PYG{n}{np}\PYG{o}{.}\PYG{n}{pi}\PYG{p}{)}\PYG{o}{/}\PYG{l+m+mi}{180} \PYG{c+c1}{\PYGZsh{} degree to radian conversion}
\PYG{n}{K\PYGZus{}h} \PYG{o}{=} \PYG{n}{Keff\PYGZus{}hs} \PYG{c+c1}{\PYGZsh{} m/s, solution from 11a}
\PYG{n}{K\PYGZus{}v} \PYG{o}{=} \PYG{n}{Keff\PYGZus{}v} \PYG{c+c1}{\PYGZsh{} m/s, solution from 11c}

\PYG{c+c1}{\PYGZsh{} solution from 11d}
\PYG{n}{Keff\PYGZus{}i} \PYG{o}{=} \PYG{l+m+mi}{1}\PYG{o}{/}\PYG{p}{(}\PYG{p}{(}\PYG{n}{np}\PYG{o}{.}\PYG{n}{cos}\PYG{p}{(}\PYG{n}{theta\PYGZus{}r}\PYG{p}{)}\PYG{o}{*}\PYG{o}{*}\PYG{l+m+mi}{2}\PYG{o}{/}\PYG{n}{K\PYGZus{}h}\PYG{p}{)}\PYG{o}{+}\PYG{p}{(}\PYG{n}{np}\PYG{o}{.}\PYG{n}{sin}\PYG{p}{(}\PYG{n}{theta\PYGZus{}r}\PYG{p}{)}\PYG{o}{*}\PYG{o}{*}\PYG{l+m+mi}{2}\PYG{o}{/}\PYG{n}{K\PYGZus{}v}\PYG{p}{)}\PYG{p}{)}

\PYG{n+nb}{print}\PYG{p}{(}\PYG{l+s+s2}{\PYGZdq{}}\PYG{l+s+se}{\PYGZbs{}n}\PYG{l+s+s2}{\PYGZdq{}}\PYG{p}{)}
\PYG{n+nb}{print}\PYG{p}{(}\PYG{l+s+s2}{\PYGZdq{}}\PYG{l+s+s2}{Effective horizontal hydraulic conductivity (Keff\PYGZus{}hs) = }\PYG{l+s+si}{\PYGZob{}0:1.2E\PYGZcb{}}\PYG{l+s+s2}{\PYGZdq{}}\PYG{o}{.}\PYG{n}{format}\PYG{p}{(}\PYG{n}{Keff\PYGZus{}hs}\PYG{p}{)}\PYG{p}{,} \PYG{l+s+s2}{\PYGZdq{}}\PYG{l+s+s2}{m/s}\PYG{l+s+se}{\PYGZbs{}n}\PYG{l+s+s2}{\PYGZdq{}} \PYG{p}{)} 
\PYG{n+nb}{print}\PYG{p}{(}\PYG{l+s+s2}{\PYGZdq{}}\PYG{l+s+s2}{Effective vertical hydraulic conductivity (Keff\PYGZus{}v) = }\PYG{l+s+si}{\PYGZob{}0:1.2E\PYGZcb{}}\PYG{l+s+s2}{\PYGZdq{}}\PYG{o}{.}\PYG{n}{format}\PYG{p}{(}\PYG{n}{Keff\PYGZus{}v}\PYG{p}{)}\PYG{p}{,} \PYG{l+s+s2}{\PYGZdq{}}\PYG{l+s+s2}{m/s}\PYG{l+s+se}{\PYGZbs{}n}\PYG{l+s+s2}{\PYGZdq{}} \PYG{p}{)} 
\PYG{n+nb}{print}\PYG{p}{(}\PYG{l+s+s2}{\PYGZdq{}}\PYG{l+s+s2}{Effective inclined hydraulic conductivity (Keff\PYGZus{}i) = }\PYG{l+s+si}{\PYGZob{}0:1.2E\PYGZcb{}}\PYG{l+s+s2}{\PYGZdq{}}\PYG{o}{.}\PYG{n}{format}\PYG{p}{(}\PYG{n}{Keff\PYGZus{}i}\PYG{p}{)}\PYG{p}{,} \PYG{l+s+s2}{\PYGZdq{}}\PYG{l+s+s2}{m/s}\PYG{l+s+se}{\PYGZbs{}n}\PYG{l+s+s2}{\PYGZdq{}} \PYG{p}{)} 
\end{sphinxVerbatim}

\end{sphinxuseclass}\end{sphinxVerbatimInput}
\begin{sphinxVerbatimOutput}

\begin{sphinxuseclass}{cell_output}
\begin{sphinxVerbatim}[commandchars=\\\{\}]
Effective horizontal hydraulic conductivity (Keff\PYGZus{}hs) = 2.10E\PYGZhy{}03 m/s

Effective vertical hydraulic conductivity (Keff\PYGZus{}v) = 3.93E\PYGZhy{}04 m/s

Effective inclined hydraulic conductivity (Keff\PYGZus{}i) = 6.62E\PYGZhy{}04 m/s
\end{sphinxVerbatim}

\end{sphinxuseclass}\end{sphinxVerbatimOutput}

\end{sphinxuseclass}
\begin{sphinxuseclass}{cell}
\begin{sphinxuseclass}{tag_hide-input}\begin{sphinxVerbatimOutput}

\begin{sphinxuseclass}{cell_output}
\begin{sphinxVerbatim}[commandchars=\\\{\}]
Column
    [0] Markdown(str, style=\PYGZob{}\PYGZsq{}font\PYGZhy{}size\PYGZsq{}: \PYGZsq{}12pt\PYGZsq{}\PYGZcb{}, width=800)
    [1] Markdown(str, style=\PYGZob{}\PYGZsq{}font\PYGZhy{}size\PYGZsq{}: \PYGZsq{}12pt\PYGZsq{}, ...\PYGZcb{}, width=800)
\end{sphinxVerbatim}

\end{sphinxuseclass}\end{sphinxVerbatimOutput}

\end{sphinxuseclass}
\end{sphinxuseclass}
\begin{sphinxuseclass}{cell}
\begin{sphinxuseclass}{tag_hide-input}\begin{sphinxVerbatimOutput}

\begin{sphinxuseclass}{cell_output}
\begin{sphinxVerbatim}[commandchars=\\\{\}]
Markdown(str, style=\PYGZob{}\PYGZsq{}font\PYGZhy{}size\PYGZsq{}: \PYGZsq{}12pt\PYGZsq{}\PYGZcb{}, width=900)
\end{sphinxVerbatim}

\end{sphinxuseclass}\end{sphinxVerbatimOutput}

\end{sphinxuseclass}
\end{sphinxuseclass}
\sphinxstepscope

\begin{sphinxuseclass}{cell}
\begin{sphinxuseclass}{tag_remove-output}\begin{sphinxVerbatimInput}

\begin{sphinxuseclass}{cell_input}
\begin{sphinxVerbatim}[commandchars=\\\{\}]
\PYG{k+kn}{import} \PYG{n+nn}{numpy} \PYG{k}{as} \PYG{n+nn}{np}
\PYG{k+kn}{import} \PYG{n+nn}{matplotlib}\PYG{n+nn}{.}\PYG{n+nn}{pyplot} \PYG{k}{as} \PYG{n+nn}{plt}
\PYG{k+kn}{import} \PYG{n+nn}{panel} \PYG{k}{as} \PYG{n+nn}{pn}
\PYG{n}{pn}\PYG{o}{.}\PYG{n}{extension}\PYG{p}{(}\PYG{l+s+s1}{\PYGZsq{}}\PYG{l+s+s1}{katex}\PYG{l+s+s1}{\PYGZsq{}}\PYG{p}{,} \PYG{l+s+s1}{\PYGZsq{}}\PYG{l+s+s1}{mathjax}\PYG{l+s+s1}{\PYGZsq{}}\PYG{p}{)} 
\end{sphinxVerbatim}

\end{sphinxuseclass}\end{sphinxVerbatimInput}

\end{sphinxuseclass}
\end{sphinxuseclass}

\chapter{Tutorial 5 \sphinxhyphen{} Tutorial problems aquifer heterogeneity/anisotropy}
\label{\detokenize{content/tutorials/T5/tutorial_05:tutorial-5-tutorial-problems-aquifer-heterogeneity-anisotropy}}\label{\detokenize{content/tutorials/T5/tutorial_05::doc}}

\section{Tutorial Problem 12: Hydrologic Triangle}
\label{\detokenize{content/tutorials/T5/tutorial_05:tutorial-problem-12-hydrologic-triangle}}
\begin{sphinxuseclass}{cell}
\begin{sphinxuseclass}{tag_hide-input}
\begin{sphinxuseclass}{tag_full-width}\begin{sphinxVerbatimOutput}

\begin{sphinxuseclass}{cell_output}
\begin{sphinxVerbatim}[commandchars=\\\{\}]
Row
    [0] Markdown(str, style=\PYGZob{}\PYGZsq{}font\PYGZhy{}size\PYGZsq{}: \PYGZsq{}13pt\PYGZsq{}\PYGZcb{}, width=400)
    [1] Spacer(width=100)
    [2] PNG(str, width=400)
\end{sphinxVerbatim}

\end{sphinxuseclass}\end{sphinxVerbatimOutput}

\end{sphinxuseclass}
\end{sphinxuseclass}
\end{sphinxuseclass}

\section{Solution of Tutotrial Problem 12}
\label{\detokenize{content/tutorials/T5/tutorial_05:solution-of-tutotrial-problem-12}}
\sphinxAtStartPar
The following 4 steps are to be performed:
\begin{itemize}
\item {} 
\sphinxAtStartPar
\sphinxstylestrong{Step I}   : Connects all the points

\item {} 
\sphinxAtStartPar
\sphinxstylestrong{Step II}  : Divide the connected lines at equal head\sphinxhyphen{}level (here = 1 m)

\item {} 
\sphinxAtStartPar
\sphinxstylestrong{Step III} : Join all the equal head lines

\item {} 
\sphinxAtStartPar
\sphinxstylestrong{Step IV}  : Mark the flow direction from higher head towards lower head

\end{itemize}

\begin{sphinxuseclass}{cell}
\begin{sphinxuseclass}{tag_hide-input}\begin{sphinxVerbatimOutput}

\begin{sphinxuseclass}{cell_output}
\begin{sphinxVerbatim}[commandchars=\\\{\}]
Tabs
    [0] PNG(str, width=400)
    [1] PNG(str, width=500)
    [2] PNG(str, width=500)
    [3] PNG(str, width=500)
    [4] PNG(str, width=700)
\end{sphinxVerbatim}

\end{sphinxuseclass}\end{sphinxVerbatimOutput}

\end{sphinxuseclass}
\end{sphinxuseclass}

\section{Tutorial Problem 13: Flow Nets}
\label{\detokenize{content/tutorials/T5/tutorial_05:tutorial-problem-13-flow-nets}}
\begin{sphinxuseclass}{cell}
\begin{sphinxuseclass}{tag_hide-input}
\begin{sphinxuseclass}{tag_full-width}\begin{sphinxVerbatimOutput}

\begin{sphinxuseclass}{cell_output}
\begin{sphinxVerbatim}[commandchars=\\\{\}]
Column
    [0] Markdown(str, style=\PYGZob{}\PYGZsq{}font\PYGZhy{}size\PYGZsq{}: \PYGZsq{}13pt\PYGZsq{}\PYGZcb{}, width=800)
    [1] Row
        [0] Column
            [0] Markdown(str, style=\PYGZob{}\PYGZsq{}font\PYGZhy{}size\PYGZsq{}: \PYGZsq{}13pt\PYGZsq{}\PYGZcb{}, width=400)
            [1] PNG(str, width=200)
        [1] Spacer(width=200)
        [2] Column
            [0] Markdown(str, style=\PYGZob{}\PYGZsq{}font\PYGZhy{}size\PYGZsq{}: \PYGZsq{}13pt\PYGZsq{}\PYGZcb{}, width=400)
            [1] PNG(str, width=200)
\end{sphinxVerbatim}

\end{sphinxuseclass}\end{sphinxVerbatimOutput}

\end{sphinxuseclass}
\end{sphinxuseclass}
\end{sphinxuseclass}

\section{Solution of Tutorial Problem}
\label{\detokenize{content/tutorials/T5/tutorial_05:solution-of-tutorial-problem}}
\sphinxAtStartPar
This is to be sketched and demonstrated.


\section{Tutorial Problem 14 (special topic)}
\label{\detokenize{content/tutorials/T5/tutorial_05:tutorial-problem-14-special-topic}}
\sphinxAtStartPar
From the laboratory test the degree of saturation(\(\theta\)) of the unsaturated core (temperature = 9\(^\circ C\)) sample was
found to be 30\% and relative permeability (\(k_r\)) is assumed to be 0.1.  From the grain analysis the sample was determined to be predominantly
medium sand (intrinsic permeability, \(k = 1.61 \times 10^{-7}\) cm\(^2\)). Provided that density (\(\rho\)) and dynamic
viscosity of water (\(\mu\)) at 9\(^\circ C\) is 999.73 kg/m\(^3\) and 0.0013465 N\(\cdot\)s/m\(^2\) respectively, find the conductivity of the
sample. What will be the conductivity of the same sample when the moisture content is 1\% (\(k_r \approx 0.001\)) and 80\% (\(k_r \approx 0.4\)). Explain the effect of moisture content on the sample.


\section{Solution of Tutorial Problem 14}
\label{\detokenize{content/tutorials/T5/tutorial_05:solution-of-tutorial-problem-14}}
\sphinxAtStartPar
\sphinxstylestrong{Lecture contents on the topic in L02\sphinxhyphen{} slides 02, 22 \& 26}

\sphinxAtStartPar
Hydraulic conductivity of the unsaturated sample (\(\theta < 100\%\)) can be obtained from the following expression:
\begin{equation*}
\begin{split}
K(\theta) = \bigg(\frac{k\rho g}{\mu}\bigg)k_r(\theta)
\end{split}
\end{equation*}
\begin{sphinxuseclass}{cell}\begin{sphinxVerbatimInput}

\begin{sphinxuseclass}{cell_input}
\begin{sphinxVerbatim}[commandchars=\\\{\}]
\PYG{c+c1}{\PYGZsh{} Given }
\PYG{n}{kr\PYGZus{}30}  \PYG{o}{=} \PYG{l+m+mf}{0.1} \PYG{c+c1}{\PYGZsh{} (\PYGZhy{}), relative permeability for moist. cont. 30\PYGZpc{}}
\PYG{n}{i\PYGZus{}p} \PYG{o}{=} \PYG{l+m+mf}{1.61} \PYG{o}{*} \PYG{l+m+mi}{10}\PYG{o}{*}\PYG{o}{*}\PYG{o}{\PYGZhy{}}\PYG{l+m+mi}{7} \PYG{c+c1}{\PYGZsh{} cm\PYGZca{}2, intrinsic permeability}
\PYG{n}{rho} \PYG{o}{=} \PYG{l+m+mf}{999.73} \PYG{c+c1}{\PYGZsh{} kg/m\PYGZca{}3, Sample density}
\PYG{n}{mu} \PYG{o}{=} \PYG{l+m+mf}{0.0013467} \PYG{c+c1}{\PYGZsh{}  N\PYGZhy{}s/m\PYGZca{}2, dynamic visc.}
\PYG{n}{g\PYGZus{}c} \PYG{o}{=} \PYG{l+m+mf}{9.81} \PYG{c+c1}{\PYGZsh{} N/kg, force unit used for gravitational constant}

\PYG{c+c1}{\PYGZsh{} Solutions 1}
\PYG{n}{i\PYGZus{}pm} \PYG{o}{=} \PYG{n}{i\PYGZus{}p}\PYG{o}{/}\PYG{l+m+mi}{10000} \PYG{c+c1}{\PYGZsh{} m\PYGZca{}2 unit conversion for int. permeab.}
\PYG{n}{K\PYGZus{}30} \PYG{o}{=} \PYG{p}{(}\PYG{n}{i\PYGZus{}p}\PYG{o}{*}\PYG{n}{rho}\PYG{o}{*}\PYG{n}{g\PYGZus{}c}\PYG{o}{/}\PYG{n}{mu}\PYG{p}{)}\PYG{o}{*}\PYG{n}{kr\PYGZus{}30}

\PYG{c+c1}{\PYGZsh{} Solution 2  when moisture content is 1\PYGZpc{} and 80\PYGZpc{}}
\PYG{n}{kr\PYGZus{}1} \PYG{o}{=} \PYG{l+m+mf}{0.001} \PYG{c+c1}{\PYGZsh{} (\PYGZhy{}), relative permeability for moist. cont. 1\PYGZpc{}}
\PYG{n}{kr\PYGZus{}80} \PYG{o}{=} \PYG{l+m+mf}{0.4} \PYG{c+c1}{\PYGZsh{} (\PYGZhy{}), relative permeability for moist. cont. 80\PYGZpc{}}
\PYG{n}{K\PYGZus{}1} \PYG{o}{=} \PYG{p}{(}\PYG{n}{i\PYGZus{}p}\PYG{o}{*}\PYG{n}{rho}\PYG{o}{*}\PYG{n}{g\PYGZus{}c}\PYG{o}{/}\PYG{n}{mu}\PYG{p}{)}\PYG{o}{*}\PYG{n}{kr\PYGZus{}1}
\PYG{n}{K\PYGZus{}80} \PYG{o}{=} \PYG{p}{(}\PYG{n}{i\PYGZus{}p}\PYG{o}{*}\PYG{n}{rho}\PYG{o}{*}\PYG{n}{g\PYGZus{}c}\PYG{o}{/}\PYG{n}{mu}\PYG{p}{)}\PYG{o}{*}\PYG{n}{kr\PYGZus{}80}

\PYG{c+c1}{\PYGZsh{} output}

\PYG{n+nb}{print}\PYG{p}{(}\PYG{l+s+s2}{\PYGZdq{}}\PYG{l+s+s2}{The conductivity of water when moisture content is 30}\PYG{l+s+si}{\PYGZpc{}  i}\PYG{l+s+s2}{s: }\PYG{l+s+si}{\PYGZob{}0:1.1e\PYGZcb{}}\PYG{l+s+s2}{\PYGZdq{}}\PYG{o}{.}\PYG{n}{format}\PYG{p}{(}\PYG{n}{K\PYGZus{}30}\PYG{p}{)}\PYG{p}{,}\PYG{l+s+s2}{\PYGZdq{}}\PYG{l+s+s2}{m/s }\PYG{l+s+se}{\PYGZbs{}n}\PYG{l+s+s2}{\PYGZdq{}}\PYG{p}{)}
\PYG{n+nb}{print}\PYG{p}{(}\PYG{l+s+s2}{\PYGZdq{}}\PYG{l+s+s2}{The conductivity of water when moisture content is 1}\PYG{l+s+si}{\PYGZpc{}  i}\PYG{l+s+s2}{s: }\PYG{l+s+si}{\PYGZob{}0:1.1e\PYGZcb{}}\PYG{l+s+s2}{\PYGZdq{}}\PYG{o}{.}\PYG{n}{format}\PYG{p}{(}\PYG{n}{K\PYGZus{}1}\PYG{p}{)}\PYG{p}{,}\PYG{l+s+s2}{\PYGZdq{}}\PYG{l+s+s2}{m/s }\PYG{l+s+se}{\PYGZbs{}n}\PYG{l+s+s2}{\PYGZdq{}}\PYG{p}{)}
\PYG{n+nb}{print}\PYG{p}{(}\PYG{l+s+s2}{\PYGZdq{}}\PYG{l+s+s2}{The conductivity of water when moisture content is 80}\PYG{l+s+si}{\PYGZpc{}  i}\PYG{l+s+s2}{s: }\PYG{l+s+si}{\PYGZob{}0:1.1e\PYGZcb{}}\PYG{l+s+s2}{\PYGZdq{}}\PYG{o}{.}\PYG{n}{format}\PYG{p}{(}\PYG{n}{K\PYGZus{}80}\PYG{p}{)}\PYG{p}{,}\PYG{l+s+s2}{\PYGZdq{}}\PYG{l+s+s2}{m/s }\PYG{l+s+se}{\PYGZbs{}n}\PYG{l+s+s2}{\PYGZdq{}}\PYG{p}{)}
\PYG{n+nb}{print}\PYG{p}{(}\PYG{l+s+s2}{\PYGZdq{}}\PYG{l+s+s2}{The conductivity of media increases very rapidly with increase of moisture content}\PYG{l+s+s2}{\PYGZdq{}}\PYG{p}{)}
\end{sphinxVerbatim}

\end{sphinxuseclass}\end{sphinxVerbatimInput}
\begin{sphinxVerbatimOutput}

\begin{sphinxuseclass}{cell_output}
\begin{sphinxVerbatim}[commandchars=\\\{\}]
The conductivity of water when moisture content is 30\PYGZpc{}  is: 1.2e\PYGZhy{}01 m/s 

The conductivity of water when moisture content is 1\PYGZpc{}  is: 1.2e\PYGZhy{}03 m/s 

The conductivity of water when moisture content is 80\PYGZpc{}  is: 4.7e\PYGZhy{}01 m/s 

The conductivity of media increases very rapidly with increase of moisture content
\end{sphinxVerbatim}

\end{sphinxuseclass}\end{sphinxVerbatimOutput}

\end{sphinxuseclass}

\section{Tutorial Problem 15}
\label{\detokenize{content/tutorials/T5/tutorial_05:tutorial-problem-15}}
\begin{sphinxuseclass}{cell}
\begin{sphinxuseclass}{tag_hide-input}
\begin{sphinxuseclass}{tag_full-width}\begin{sphinxVerbatimOutput}

\begin{sphinxuseclass}{cell_output}
\begin{sphinxVerbatim}[commandchars=\\\{\}]
Column
    [0] LaTeX(str, style=\PYGZob{}\PYGZsq{}font\PYGZhy{}size\PYGZsq{}: \PYGZsq{}13pt\PYGZsq{}\PYGZcb{}, width=900)
    [1] PNG(str, width=400)
\end{sphinxVerbatim}

\end{sphinxuseclass}\end{sphinxVerbatimOutput}

\end{sphinxuseclass}
\end{sphinxuseclass}
\end{sphinxuseclass}

\section{Solution Tutorial Problem 15}
\label{\detokenize{content/tutorials/T5/tutorial_05:solution-tutorial-problem-15}}
\begin{sphinxuseclass}{cell}\begin{sphinxVerbatimInput}

\begin{sphinxuseclass}{cell_input}
\begin{sphinxVerbatim}[commandchars=\\\{\}]
\PYG{c+c1}{\PYGZsh{} Given}

\PYG{n}{K\PYGZus{}s} \PYG{o}{=} \PYG{l+m+mi}{2} \PYG{c+c1}{\PYGZsh{} cm/d \PYGZsh{} saturated conductivity}
\PYG{n}{al\PYGZus{}a} \PYG{o}{=} \PYG{l+m+mf}{0.04} \PYG{c+c1}{\PYGZsh{} 1/cm, fit constant}
\PYG{n}{Ph\PYGZus{}a} \PYG{o}{=} \PYG{o}{\PYGZhy{}}\PYG{l+m+mi}{100} \PYG{c+c1}{\PYGZsh{} cm, pressure head at A}
\PYG{n}{Ph\PYGZus{}b} \PYG{o}{=} \PYG{o}{\PYGZhy{}}\PYG{l+m+mi}{90} \PYG{c+c1}{\PYGZsh{} cm, pressure head at B}
\PYG{n}{Z\PYGZus{}a} \PYG{o}{=} \PYG{l+m+mi}{300} \PYG{c+c1}{\PYGZsh{} cm, elevation head at A from datum}
\PYG{n}{Z\PYGZus{}b} \PYG{o}{=} \PYG{l+m+mi}{200} \PYG{c+c1}{\PYGZsh{} cm, elevation head at B from datum}

\PYG{c+c1}{\PYGZsh{} Solution 1 }
\PYG{n}{Ph\PYGZus{}m} \PYG{o}{=} \PYG{p}{(}\PYG{n}{Ph\PYGZus{}a}\PYG{o}{+}\PYG{n}{Ph\PYGZus{}b}\PYG{p}{)}\PYG{o}{/}\PYG{l+m+mi}{2} \PYG{c+c1}{\PYGZsh{} mean pressure head}
\PYG{n}{K\PYGZus{}psi} \PYG{o}{=} \PYG{n}{K\PYGZus{}s}\PYG{o}{*}\PYG{n}{np}\PYG{o}{.}\PYG{n}{exp}\PYG{p}{(}\PYG{n}{al\PYGZus{}a}\PYG{o}{*}\PYG{n}{Ph\PYGZus{}m}\PYG{p}{)} \PYG{c+c1}{\PYGZsh{} cm/d, from the given model}

\PYG{c+c1}{\PYGZsh{}Solution 2}
\PYG{n}{H\PYGZus{}A} \PYG{o}{=} \PYG{n}{Ph\PYGZus{}a}\PYG{o}{+}\PYG{n}{Z\PYGZus{}a} \PYG{c+c1}{\PYGZsh{} cm, hydraulic head at A}
\PYG{n}{H\PYGZus{}B} \PYG{o}{=} \PYG{n}{Ph\PYGZus{}b}\PYG{o}{+}\PYG{n}{Z\PYGZus{}b} \PYG{c+c1}{\PYGZsh{} cm, hydraulic head at B}
\PYG{n}{dh\PYGZus{}dz} \PYG{o}{=} \PYG{p}{(}\PYG{n}{H\PYGZus{}B} \PYG{o}{\PYGZhy{}} \PYG{n}{H\PYGZus{}A}\PYG{p}{)}\PYG{o}{/}\PYG{p}{(}\PYG{n}{Z\PYGZus{}b} \PYG{o}{\PYGZhy{}} \PYG{n}{Z\PYGZus{}a}\PYG{p}{)} \PYG{c+c1}{\PYGZsh{} (\PYGZhy{}), hydraulic head gradient}
\PYG{n}{q\PYGZus{}z} \PYG{o}{=} \PYG{o}{\PYGZhy{}}\PYG{n}{K\PYGZus{}psi}\PYG{o}{*}\PYG{n}{dh\PYGZus{}dz} \PYG{c+c1}{\PYGZsh{} cm/d, Darcy velocity }

\PYG{n+nb}{print}\PYG{p}{(}\PYG{l+s+s2}{\PYGZdq{}}\PYG{l+s+s2}{The unsaturated conductiviy of the sample is: }\PYG{l+s+si}{\PYGZob{}0:1.3f\PYGZcb{}}\PYG{l+s+s2}{\PYGZdq{}}\PYG{o}{.}\PYG{n}{format}\PYG{p}{(}\PYG{n}{K\PYGZus{}psi}\PYG{p}{)}\PYG{p}{,} \PYG{l+s+s2}{\PYGZdq{}}\PYG{l+s+s2}{cm/d}\PYG{l+s+se}{\PYGZbs{}n}\PYG{l+s+s2}{\PYGZdq{}}\PYG{p}{)}
\PYG{n+nb}{print}\PYG{p}{(}\PYG{l+s+s2}{\PYGZdq{}}\PYG{l+s+s2}{The Darcy velocity is: }\PYG{l+s+si}{\PYGZob{}0:1.3f\PYGZcb{}}\PYG{l+s+s2}{\PYGZdq{}}\PYG{o}{.}\PYG{n}{format}\PYG{p}{(}\PYG{n}{q\PYGZus{}z}\PYG{p}{)}\PYG{p}{,} \PYG{l+s+s2}{\PYGZdq{}}\PYG{l+s+s2}{cm/d}\PYG{l+s+se}{\PYGZbs{}n}\PYG{l+s+s2}{\PYGZdq{}}\PYG{p}{)} 
\PYG{n+nb}{print}\PYG{p}{(}\PYG{l+s+s2}{\PYGZdq{}}\PYG{l+s+s2}{The negative sign indicates the direction opposite to increase in z.}\PYG{l+s+s2}{\PYGZdq{}}\PYG{p}{)} 
\end{sphinxVerbatim}

\end{sphinxuseclass}\end{sphinxVerbatimInput}
\begin{sphinxVerbatimOutput}

\begin{sphinxuseclass}{cell_output}
\begin{sphinxVerbatim}[commandchars=\\\{\}]
The unsaturated conductiviy of the sample is: 0.045 cm/d

The Darcy velocity is: \PYGZhy{}0.040 cm/d

The negative sign indicates the direction opposite to increase in z.
\end{sphinxVerbatim}

\end{sphinxuseclass}\end{sphinxVerbatimOutput}

\end{sphinxuseclass}

\section{Homework Problems}
\label{\detokenize{content/tutorials/T5/tutorial_05:homework-problems}}

\subsection{Homework Problem 6: Hydrologic Triangle}
\label{\detokenize{content/tutorials/T5/tutorial_05:homework-problem-6-hydrologic-triangle}}
\begin{sphinxuseclass}{cell}
\begin{sphinxuseclass}{tag_hide-input}
\begin{sphinxuseclass}{tag_full-width}\begin{sphinxVerbatimOutput}

\begin{sphinxuseclass}{cell_output}
\begin{sphinxVerbatim}[commandchars=\\\{\}]
Row
    [0] Markdown(str, style=\PYGZob{}\PYGZsq{}font\PYGZhy{}size\PYGZsq{}: \PYGZsq{}13pt\PYGZsq{}\PYGZcb{}, width=500)
    [1] PNG(str, width=400)
\end{sphinxVerbatim}

\end{sphinxuseclass}\end{sphinxVerbatimOutput}

\end{sphinxuseclass}
\end{sphinxuseclass}
\end{sphinxuseclass}

\subsection{Homework Problem 7: Flow Nets}
\label{\detokenize{content/tutorials/T5/tutorial_05:homework-problem-7-flow-nets}}
\begin{sphinxuseclass}{cell}
\begin{sphinxuseclass}{tag_hide-input}
\begin{sphinxuseclass}{tag_full-width}\begin{sphinxVerbatimOutput}

\begin{sphinxuseclass}{cell_output}
\begin{sphinxVerbatim}[commandchars=\\\{\}]
Column
    [0] Markdown(str, style=\PYGZob{}\PYGZsq{}font\PYGZhy{}size\PYGZsq{}: \PYGZsq{}13pt\PYGZsq{}\PYGZcb{}, width=900)
    [1] PNG(str, width=600)
    [2] Markdown(str, style=\PYGZob{}\PYGZsq{}font\PYGZhy{}size\PYGZsq{}: \PYGZsq{}13pt\PYGZsq{}\PYGZcb{}, width=900)
\end{sphinxVerbatim}

\end{sphinxuseclass}\end{sphinxVerbatimOutput}

\end{sphinxuseclass}
\end{sphinxuseclass}
\end{sphinxuseclass}
\sphinxstepscope

\begin{sphinxuseclass}{cell}
\begin{sphinxuseclass}{tag_remove-output}\begin{sphinxVerbatimInput}

\begin{sphinxuseclass}{cell_input}
\begin{sphinxVerbatim}[commandchars=\\\{\}]
\PYG{k+kn}{import} \PYG{n+nn}{numpy} \PYG{k}{as} \PYG{n+nn}{np}
\PYG{k+kn}{import} \PYG{n+nn}{matplotlib}\PYG{n+nn}{.}\PYG{n+nn}{pyplot} \PYG{k}{as} \PYG{n+nn}{plt}
\PYG{k+kn}{import} \PYG{n+nn}{pandas} \PYG{k}{as} \PYG{n+nn}{pd} 
\PYG{k+kn}{import} \PYG{n+nn}{panel} \PYG{k}{as} \PYG{n+nn}{pn}
\PYG{k+kn}{from} \PYG{n+nn}{scipy} \PYG{k+kn}{import} \PYG{n}{stats} 
\PYG{n}{pn}\PYG{o}{.}\PYG{n}{extension}\PYG{p}{(}\PYG{l+s+s1}{\PYGZsq{}}\PYG{l+s+s1}{katex}\PYG{l+s+s1}{\PYGZsq{}}\PYG{p}{,} \PYG{l+s+s1}{\PYGZsq{}}\PYG{l+s+s1}{mathjax}\PYG{l+s+s1}{\PYGZsq{}}\PYG{p}{)} 
\end{sphinxVerbatim}

\end{sphinxuseclass}\end{sphinxVerbatimInput}

\end{sphinxuseclass}
\end{sphinxuseclass}

\chapter{Tutorial 6 \sphinxhyphen{} Tutorial Problems on Flow in Confined/Unconfined Aquifer}
\label{\detokenize{content/tutorials/T6/tutorial_06:tutorial-6-tutorial-problems-on-flow-in-confined-unconfined-aquifer}}\label{\detokenize{content/tutorials/T6/tutorial_06::doc}}\begin{itemize}
\item {} 
\sphinxAtStartPar
\sphinxstylestrong{solutions for homework problems 3 – 4}

\item {} 
\sphinxAtStartPar
\sphinxstylestrong{tutorial problems on flow in confined and unconfined aquifers}

\item {} 
\sphinxAtStartPar
\sphinxstylestrong{homework problems on flow in confined and unconfined aquifers}

\end{itemize}


\section{Solutions for Homework Problems 3 – 4}
\label{\detokenize{content/tutorials/T6/tutorial_06:solutions-for-homework-problems-3-4}}

\subsection{Homework Problem 3}
\label{\detokenize{content/tutorials/T6/tutorial_06:homework-problem-3}}
\begin{sphinxuseclass}{cell}
\begin{sphinxuseclass}{tag_hide-input}\begin{sphinxVerbatimOutput}

\begin{sphinxuseclass}{cell_output}
\begin{sphinxVerbatim}[commandchars=\\\{\}]
Row
    [0] Markdown(str, style=\PYGZob{}\PYGZsq{}font\PYGZhy{}size\PYGZsq{}: \PYGZsq{}12pt\PYGZsq{}\PYGZcb{})
    [1] Spacer(width=50)
    [2] PNG(str, width=500)
\end{sphinxVerbatim}

\end{sphinxuseclass}\end{sphinxVerbatimOutput}

\end{sphinxuseclass}
\end{sphinxuseclass}

\subsection{Solution of the Homework Problem 3}
\label{\detokenize{content/tutorials/T6/tutorial_06:solution-of-the-homework-problem-3}}
\begin{sphinxuseclass}{cell}
\begin{sphinxuseclass}{tag_hide-input}\begin{sphinxVerbatimOutput}

\begin{sphinxuseclass}{cell_output}
\begin{sphinxVerbatim}[commandchars=\\\{\}]
Column
    [0] Markdown(str, style=\PYGZob{}\PYGZsq{}font\PYGZhy{}size\PYGZsq{}: \PYGZsq{}13pt\PYGZsq{}\PYGZcb{}, width=700)
    [1] LaTeX(str, style=\PYGZob{}\PYGZsq{}font\PYGZhy{}size\PYGZsq{}: \PYGZsq{}12pt\PYGZsq{}\PYGZcb{})
    [2] Row
        [0] LaTeX(str, style=\PYGZob{}\PYGZsq{}font\PYGZhy{}size\PYGZsq{}: \PYGZsq{}12pt\PYGZsq{}\PYGZcb{}, width=300)
        [1] LaTeX(str, style=\PYGZob{}\PYGZsq{}font\PYGZhy{}size\PYGZsq{}: \PYGZsq{}12pt\PYGZsq{}\PYGZcb{}, width=300)
    [3] LaTeX(str, style=\PYGZob{}\PYGZsq{}font\PYGZhy{}size\PYGZsq{}: \PYGZsq{}12pt\PYGZsq{}\PYGZcb{}, width=800)
\end{sphinxVerbatim}

\end{sphinxuseclass}\end{sphinxVerbatimOutput}

\end{sphinxuseclass}
\end{sphinxuseclass}
\begin{sphinxuseclass}{cell}\begin{sphinxVerbatimInput}

\begin{sphinxuseclass}{cell_input}
\begin{sphinxVerbatim}[commandchars=\\\{\}]
\PYG{c+c1}{\PYGZsh{} Problem 3b, Given are:}
\PYG{n}{L} \PYG{o}{=} \PYG{l+m+mi}{10}\PYG{c+c1}{\PYGZsh{} cm, length of column }
\PYG{n}{a1} \PYG{o}{=} \PYG{l+m+mi}{6}\PYG{c+c1}{\PYGZsh{} cm, pressure head at 1 }
\PYG{n}{a2} \PYG{o}{=} \PYG{l+m+mi}{3}\PYG{c+c1}{\PYGZsh{} cm, pressure head at 2 }
\PYG{n}{d} \PYG{o}{=} \PYG{l+m+mi}{4} \PYG{c+c1}{\PYGZsh{} cm, diameter of the column}
\PYG{n}{V} \PYG{o}{=} \PYG{l+m+mi}{250} \PYG{c+c1}{\PYGZsh{} mL, volume  }
\PYG{n}{A} \PYG{o}{=} \PYG{n}{np}\PYG{o}{.}\PYG{n}{pi}\PYG{o}{*}\PYG{p}{(}\PYG{n}{d}\PYG{o}{/}\PYG{l+m+mi}{2}\PYG{p}{)}\PYG{o}{*}\PYG{o}{*}\PYG{l+m+mi}{2} \PYG{c+c1}{\PYGZsh{} cm\PYGZca{}2 Area of the column}
\PYG{n}{t} \PYG{o}{=} \PYG{l+m+mi}{36} \PYG{c+c1}{\PYGZsh{} s, time}

\PYG{c+c1}{\PYGZsh{} interim calculation}
\PYG{n}{Q} \PYG{o}{=} \PYG{n}{V}\PYG{o}{/}\PYG{n}{t} \PYG{c+c1}{\PYGZsh{} mL/s, discharge}

\PYG{c+c1}{\PYGZsh{}calculation}
\PYG{n}{K} \PYG{o}{=} \PYG{p}{(}\PYG{n}{Q}\PYG{o}{*}\PYG{n}{L}\PYG{p}{)}\PYG{o}{/}\PYG{p}{(}\PYG{n}{A}\PYG{o}{*}\PYG{p}{(}\PYG{n}{a1}\PYG{o}{+}\PYG{n}{L}\PYG{o}{\PYGZhy{}}\PYG{n}{a2}\PYG{p}{)}\PYG{p}{)}\PYG{c+c1}{\PYGZsh{} cm/s, Conductivity}

\PYG{c+c1}{\PYGZsh{}output}
\PYG{n+nb}{print}\PYG{p}{(}\PYG{l+s+s1}{\PYGZsq{}}\PYG{l+s+se}{\PYGZbs{}033}\PYG{l+s+s1}{[1m}\PYG{l+s+s1}{\PYGZsq{}} \PYG{o}{+} \PYG{l+s+s1}{\PYGZsq{}}\PYG{l+s+s1}{Results are:}\PYG{l+s+s1}{\PYGZsq{}} \PYG{o}{+} \PYG{l+s+s1}{\PYGZsq{}}\PYG{l+s+se}{\PYGZbs{}033}\PYG{l+s+s1}{[0m }\PYG{l+s+se}{\PYGZbs{}n}\PYG{l+s+s1}{\PYGZsq{}}\PYG{p}{)}
\PYG{n+nb}{print}\PYG{p}{(}\PYG{l+s+s2}{\PYGZdq{}}\PYG{l+s+s2}{The conductivity of the column is:}\PYG{l+s+si}{\PYGZob{}0:1.3f\PYGZcb{}}\PYG{l+s+s2}{\PYGZdq{}}\PYG{o}{.}\PYG{n}{format}\PYG{p}{(}\PYG{n}{K}\PYG{p}{)}\PYG{p}{,} \PYG{l+s+s2}{\PYGZdq{}}\PYG{l+s+s2}{cm/s }\PYG{l+s+se}{\PYGZbs{}n}\PYG{l+s+s2}{\PYGZdq{}}\PYG{p}{)}
\PYG{n+nb}{print}\PYG{p}{(}\PYG{l+s+s2}{\PYGZdq{}}\PYG{l+s+s2}{The conductivity of the column is:}\PYG{l+s+si}{\PYGZob{}0:1.2e\PYGZcb{}}\PYG{l+s+s2}{\PYGZdq{}}\PYG{o}{.}\PYG{n}{format}\PYG{p}{(}\PYG{n}{K}\PYG{o}{/}\PYG{l+m+mi}{100}\PYG{p}{)}\PYG{p}{,} \PYG{l+s+s2}{\PYGZdq{}}\PYG{l+s+s2}{m/s}\PYG{l+s+se}{\PYGZbs{}n}\PYG{l+s+s2}{\PYGZdq{}}\PYG{p}{)}

\PYG{n}{r3\PYGZus{}8} \PYG{o}{=} \PYG{n}{pn}\PYG{o}{.}\PYG{n}{pane}\PYG{o}{.}\PYG{n}{Markdown}\PYG{p}{(}\PYG{l+s+s2}{\PYGZdq{}\PYGZdq{}\PYGZdq{}}
\PYG{l+s+s2}{The sample in the column is: **Coarse sand \PYGZhy{} Fine gravel**}
\PYG{l+s+s2}{\PYGZdq{}\PYGZdq{}\PYGZdq{}}\PYG{p}{,} \PYG{n}{width}\PYG{o}{=}\PYG{l+m+mi}{400}\PYG{p}{)}  

\PYG{n}{pn}\PYG{o}{.}\PYG{n}{Row}\PYG{p}{(}\PYG{n}{r3\PYGZus{}8}\PYG{p}{)} 
\end{sphinxVerbatim}

\end{sphinxuseclass}\end{sphinxVerbatimInput}
\begin{sphinxVerbatimOutput}

\begin{sphinxuseclass}{cell_output}
\begin{sphinxVerbatim}[commandchars=\\\{\}]
\PYG{Color+ColorBold}{Results are:} 

The conductivity of the column is:0.425 cm/s 

The conductivity of the column is:4.25e\PYGZhy{}03 m/s
\end{sphinxVerbatim}

\begin{sphinxVerbatim}[commandchars=\\\{\}]
Row
    [0] Markdown(str, width=400)
\end{sphinxVerbatim}

\end{sphinxuseclass}\end{sphinxVerbatimOutput}

\end{sphinxuseclass}

\subsection{Homework Problem 4}
\label{\detokenize{content/tutorials/T6/tutorial_06:homework-problem-4}}
\begin{sphinxuseclass}{cell}\begin{sphinxVerbatimInput}

\begin{sphinxuseclass}{cell_input}
\begin{sphinxVerbatim}[commandchars=\\\{\}]
\PYG{c+c1}{\PYGZsh{} given data \PYGZhy{} you may change the number}

\PYG{n}{t} \PYG{o}{=} \PYG{n}{np}\PYG{o}{.}\PYG{n}{array}\PYG{p}{(}\PYG{p}{[}\PYG{l+m+mi}{0}\PYG{p}{,} \PYG{l+m+mi}{5}\PYG{p}{,} \PYG{l+m+mi}{18}\PYG{p}{,} \PYG{l+m+mi}{23}\PYG{p}{,} \PYG{l+m+mi}{27}\PYG{p}{,} \PYG{l+m+mi}{29}\PYG{p}{]}\PYG{p}{)} \PYG{c+c1}{\PYGZsh{} min, given time}
\PYG{n}{Dh} \PYG{o}{=} \PYG{n}{np}\PYG{o}{.}\PYG{n}{array}\PYG{p}{(}\PYG{p}{[}\PYG{l+m+mf}{36.9}\PYG{p}{,} \PYG{l+m+mi}{32}\PYG{p}{,} \PYG{l+m+mf}{22.3}\PYG{p}{,} \PYG{l+m+mf}{20.9}\PYG{p}{,} \PYG{l+m+mf}{16.1}\PYG{p}{,} \PYG{l+m+mf}{12.3}\PYG{p}{]}\PYG{p}{)} \PYG{c+c1}{\PYGZsh{} cm, head difference }

\PYG{c+c1}{\PYGZsh{} creating data table}
\PYG{n}{data} \PYG{o}{=} \PYG{p}{\PYGZob{}}\PYG{l+s+s2}{\PYGZdq{}}\PYG{l+s+s2}{Time (min)}\PYG{l+s+s2}{\PYGZdq{}}\PYG{p}{:} \PYG{n}{t}\PYG{p}{,} \PYG{l+s+s2}{\PYGZdq{}}\PYG{l+s+s2}{Δh (m)}\PYG{l+s+s2}{\PYGZdq{}}\PYG{p}{:} \PYG{n}{Dh}\PYG{p}{\PYGZcb{}}
\PYG{n}{df} \PYG{o}{=} \PYG{n}{pd}\PYG{o}{.}\PYG{n}{DataFrame}\PYG{p}{(}\PYG{n}{data}\PYG{p}{)}
\PYG{n}{df1} \PYG{o}{=} \PYG{n}{df}\PYG{o}{.}\PYG{n}{T}


\PYG{n}{r\PYGZus{}h4} \PYG{o}{=} \PYG{n}{pn}\PYG{o}{.}\PYG{n}{pane}\PYG{o}{.}\PYG{n}{Markdown}\PYG{p}{(}\PYG{l+s+s2}{\PYGZdq{}\PYGZdq{}\PYGZdq{}}
\PYG{l+s+s2}{A Darcy experiment is performed by a falling\PYGZhy{}head permeameter using water at 20°C. }
\PYG{l+s+s2}{Length and diameter of the sample are 20 cm and 6 cm, resp. The inner tube diameter is 4 cm. }
\PYG{l+s+s2}{The following data are available for the time\PYGZhy{}dependent hydraulic head difference : }
\PYG{l+s+s2}{\PYGZdq{}\PYGZdq{}\PYGZdq{}}\PYG{p}{,}\PYG{n}{width} \PYG{o}{=} \PYG{l+m+mi}{600}\PYG{p}{,} \PYG{n}{style}\PYG{o}{=}\PYG{p}{\PYGZob{}}\PYG{l+s+s1}{\PYGZsq{}}\PYG{l+s+s1}{font\PYGZhy{}size}\PYG{l+s+s1}{\PYGZsq{}}\PYG{p}{:} \PYG{l+s+s1}{\PYGZsq{}}\PYG{l+s+s1}{12pt}\PYG{l+s+s1}{\PYGZsq{}}\PYG{p}{\PYGZcb{}}\PYG{p}{)}


\PYG{n}{spacer2}\PYG{o}{=}\PYG{n}{pn}\PYG{o}{.}\PYG{n}{Spacer}\PYG{p}{(}\PYG{n}{width}\PYG{o}{=}\PYG{l+m+mi}{50}\PYG{p}{)}

\PYG{n}{pn}\PYG{o}{.}\PYG{n}{Column}\PYG{p}{(}\PYG{n}{r\PYGZus{}h4}\PYG{p}{,}\PYG{n}{spacer2}\PYG{p}{,} \PYG{n}{df1}\PYG{p}{)}
\end{sphinxVerbatim}

\end{sphinxuseclass}\end{sphinxVerbatimInput}
\begin{sphinxVerbatimOutput}

\begin{sphinxuseclass}{cell_output}
\begin{sphinxVerbatim}[commandchars=\\\{\}]
Column
    [0] Markdown(str, style=\PYGZob{}\PYGZsq{}font\PYGZhy{}size\PYGZsq{}: \PYGZsq{}12pt\PYGZsq{}\PYGZcb{}, width=600)
    [1] Spacer(width=50)
    [2] DataFrame(DataFrame)
\end{sphinxVerbatim}

\end{sphinxuseclass}\end{sphinxVerbatimOutput}

\end{sphinxuseclass}
\begin{sphinxuseclass}{cell}
\begin{sphinxuseclass}{tag_hide-input}\begin{sphinxVerbatimOutput}

\begin{sphinxuseclass}{cell_output}
\begin{sphinxVerbatim}[commandchars=\\\{\}]
Row
    [0] Markdown(str, style=\PYGZob{}\PYGZsq{}font\PYGZhy{}size\PYGZsq{}: \PYGZsq{}12pt\PYGZsq{}\PYGZcb{}, width=500)
    [1] Spacer(width=50)
    [2] PNG(str, width=350)
\end{sphinxVerbatim}

\end{sphinxuseclass}\end{sphinxVerbatimOutput}

\end{sphinxuseclass}
\end{sphinxuseclass}

\subsection{Solution of Homework Problem 4}
\label{\detokenize{content/tutorials/T6/tutorial_06:solution-of-homework-problem-4}}
\begin{sphinxuseclass}{cell}
\begin{sphinxuseclass}{tag_hide-input}\begin{sphinxVerbatimOutput}

\begin{sphinxuseclass}{cell_output}
\begin{sphinxVerbatim}[commandchars=\\\{\}]
LaTeX(str, style=\PYGZob{}\PYGZsq{}font\PYGZhy{}size\PYGZsq{}: \PYGZsq{}12pt\PYGZsq{}\PYGZcb{}, width=800)
\end{sphinxVerbatim}

\end{sphinxuseclass}\end{sphinxVerbatimOutput}

\end{sphinxuseclass}
\end{sphinxuseclass}
\begin{sphinxuseclass}{cell}\begin{sphinxVerbatimInput}

\begin{sphinxuseclass}{cell_input}
\begin{sphinxVerbatim}[commandchars=\\\{\}]
\PYG{c+c1}{\PYGZsh{}Caclulation}

\PYG{n}{t\PYGZus{}s} \PYG{o}{=} \PYG{n}{t}\PYG{o}{*}\PYG{l+m+mi}{60} \PYG{c+c1}{\PYGZsh{} s, time in second}
\PYG{n}{Dh0\PYGZus{}Dht} \PYG{o}{=} \PYG{n}{Dh}\PYG{p}{[}\PYG{l+m+mi}{0}\PYG{p}{]}\PYG{o}{/}\PYG{n}{Dh} \PYG{c+c1}{\PYGZsh{} (\PYGZhy{}), Delta h(0)/Delta h(t)}
\PYG{n}{ln\PYGZus{}Dhodht} \PYG{o}{=} \PYG{n}{np}\PYG{o}{.}\PYG{n}{log}\PYG{p}{(}\PYG{n}{Dh0\PYGZus{}Dht}\PYG{p}{)}\PYG{c+c1}{\PYGZsh{} (\PYGZhy{}), ln(Delta h(0)/Delta h(t))}

\PYG{n}{slope}\PYG{p}{,} \PYG{n}{intercept}\PYG{p}{,} \PYG{n}{r\PYGZus{}value}\PYG{p}{,} \PYG{n}{p\PYGZus{}value}\PYG{p}{,} \PYG{n}{std\PYGZus{}err} \PYG{o}{=} \PYG{n}{stats}\PYG{o}{.}\PYG{n}{linregress}\PYG{p}{(}\PYG{n}{t\PYGZus{}s}\PYG{p}{,} \PYG{n}{ln\PYGZus{}Dhodht}\PYG{p}{)} \PYG{c+c1}{\PYGZsh{} linear regression}


\PYG{c+c1}{\PYGZsh{} result table}
\PYG{n}{data2} \PYG{o}{=} \PYG{p}{\PYGZob{}}\PYG{l+s+s2}{\PYGZdq{}}\PYG{l+s+s2}{Time (s)}\PYG{l+s+s2}{\PYGZdq{}}\PYG{p}{:} \PYG{n}{t\PYGZus{}s}\PYG{p}{,} \PYG{l+s+s2}{\PYGZdq{}}\PYG{l+s+s2}{Δh(0)/Δh(t)}\PYG{l+s+s2}{\PYGZdq{}}\PYG{p}{:}\PYG{n}{Dh0\PYGZus{}Dht}\PYG{p}{,}\PYG{l+s+s2}{\PYGZdq{}}\PYG{l+s+s2}{ln (Δh(0)/Δh(t)}\PYG{l+s+s2}{\PYGZdq{}}\PYG{p}{:} \PYG{n}{ln\PYGZus{}Dhodht}\PYG{p}{\PYGZcb{}}
\PYG{n}{df2} \PYG{o}{=} \PYG{n}{pd}\PYG{o}{.}\PYG{n}{DataFrame}\PYG{p}{(}\PYG{n}{data2}\PYG{p}{)}

\PYG{n}{fig} \PYG{o}{=} \PYG{n}{plt}\PYG{o}{.}\PYG{n}{figure}\PYG{p}{(}\PYG{p}{)}
\PYG{n}{plt}\PYG{o}{.}\PYG{n}{plot}\PYG{p}{(}\PYG{n}{t\PYGZus{}s}\PYG{p}{,} \PYG{n}{ln\PYGZus{}Dhodht}\PYG{p}{,} \PYG{l+s+s1}{\PYGZsq{}}\PYG{l+s+s1}{o}\PYG{l+s+s1}{\PYGZsq{}}\PYG{p}{,} \PYG{n}{label}\PYG{o}{=}\PYG{l+s+s1}{\PYGZsq{}}\PYG{l+s+s1}{ provided data}\PYG{l+s+s1}{\PYGZsq{}}\PYG{p}{)}\PYG{p}{;}
\PYG{n}{pred} \PYG{o}{=} \PYG{n}{intercept} \PYG{o}{+} \PYG{n}{slope}\PYG{o}{*}\PYG{n}{t\PYGZus{}s}
\PYG{n}{plt}\PYG{o}{.}\PYG{n}{plot}\PYG{p}{(}\PYG{n}{t\PYGZus{}s}\PYG{p}{,} \PYG{n}{pred}\PYG{p}{,} \PYG{l+s+s1}{\PYGZsq{}}\PYG{l+s+s1}{r}\PYG{l+s+s1}{\PYGZsq{}}\PYG{p}{,} \PYG{n}{label}\PYG{o}{=}\PYG{l+s+s1}{\PYGZsq{}}\PYG{l+s+s1}{y=}\PYG{l+s+si}{\PYGZob{}:0.2e\PYGZcb{}}\PYG{l+s+s1}{x+}\PYG{l+s+si}{\PYGZob{}:0.2e\PYGZcb{}}\PYG{l+s+s1}{\PYGZsq{}}\PYG{o}{.}\PYG{n}{format}\PYG{p}{(}\PYG{n}{slope}\PYG{p}{,}\PYG{n}{intercept}\PYG{p}{)}\PYG{p}{)} \PYG{p}{;}
\PYG{n}{plt}\PYG{o}{.}\PYG{n}{xlabel}\PYG{p}{(}\PYG{l+s+sa}{r}\PYG{l+s+s2}{\PYGZdq{}}\PYG{l+s+s2}{\PYGZdl{}t (s)\PYGZdl{}}\PYG{l+s+s2}{\PYGZdq{}}\PYG{p}{)}\PYG{p}{;}
\PYG{n}{plt}\PYG{o}{.}\PYG{n}{ylabel}\PYG{p}{(}\PYG{l+s+sa}{r}\PYG{l+s+s2}{\PYGZdq{}}\PYG{l+s+s2}{\PYGZdl{}}\PYG{l+s+s2}{\PYGZbs{}}\PYG{l+s+s2}{ln}\PYG{l+s+s2}{\PYGZbs{}}\PYG{l+s+s2}{frac}\PYG{l+s+s2}{\PYGZob{}}\PYG{l+s+s2}{\PYGZbs{}}\PYG{l+s+s2}{Delta h (0)\PYGZcb{}}\PYG{l+s+s2}{\PYGZob{}}\PYG{l+s+s2}{\PYGZbs{}}\PYG{l+s+s2}{Delta h (t)\PYGZcb{}}\PYG{l+s+s2}{\PYGZbs{}}\PYG{l+s+s2}{;}\PYG{l+s+s2}{\PYGZbs{}}\PYG{l+s+s2}{:(\PYGZhy{})\PYGZdl{}}\PYG{l+s+s2}{\PYGZdq{}}\PYG{p}{)}\PYG{p}{;}
\PYG{n}{plt}\PYG{o}{.}\PYG{n}{grid}\PYG{p}{(}\PYG{p}{)}\PYG{p}{;}
\PYG{n}{plt}\PYG{o}{.}\PYG{n}{legend}\PYG{p}{(}\PYG{n}{fontsize}\PYG{o}{=}\PYG{l+m+mi}{11}\PYG{p}{)} 
\PYG{n}{plt}\PYG{o}{.}\PYG{n}{text}\PYG{p}{(}\PYG{l+m+mi}{150}\PYG{p}{,} \PYG{l+m+mf}{0.42}\PYG{p}{,}\PYG{l+s+s1}{\PYGZsq{}}\PYG{l+s+s1}{\PYGZdl{}R\PYGZca{}2 = }\PYG{l+s+si}{\PYGZob{}:0.2f\PYGZcb{}}\PYG{l+s+s1}{\PYGZdl{}}\PYG{l+s+s1}{\PYGZsq{}}\PYG{o}{.}\PYG{n}{format}\PYG{p}{(}\PYG{n}{r\PYGZus{}value}\PYG{p}{)} \PYG{p}{)}
\PYG{n}{plt}\PYG{o}{.}\PYG{n}{close}\PYG{p}{(}\PYG{p}{)} \PYG{c+c1}{\PYGZsh{} otherwise we have 2 figure}
\PYG{n}{r4\PYGZus{}2} \PYG{o}{=} \PYG{n}{pn}\PYG{o}{.}\PYG{n}{pane}\PYG{o}{.}\PYG{n}{Matplotlib}\PYG{p}{(}\PYG{n}{fig}\PYG{p}{,} \PYG{n}{dpi}\PYG{o}{=}\PYG{l+m+mi}{300}\PYG{p}{)}

\PYG{n}{pn}\PYG{o}{.}\PYG{n}{Row}\PYG{p}{(}\PYG{n}{df2}\PYG{p}{,}\PYG{n}{spacer2}\PYG{p}{,}\PYG{n}{r4\PYGZus{}2}\PYG{p}{)}  
\end{sphinxVerbatim}

\end{sphinxuseclass}\end{sphinxVerbatimInput}
\begin{sphinxVerbatimOutput}

\begin{sphinxuseclass}{cell_output}
\begin{sphinxVerbatim}[commandchars=\\\{\}]
Row
    [0] DataFrame(DataFrame)
    [1] Spacer(width=50)
    [2] Matplotlib(Figure, dpi=300)
\end{sphinxVerbatim}

\end{sphinxuseclass}\end{sphinxVerbatimOutput}

\end{sphinxuseclass}
\begin{sphinxuseclass}{cell}\begin{sphinxVerbatimInput}

\begin{sphinxuseclass}{cell_input}
\begin{sphinxVerbatim}[commandchars=\\\{\}]
\PYG{c+c1}{\PYGZsh{}Solution of 4C}
\PYG{c+c1}{\PYGZsh{} Given }
\PYG{n}{L} \PYG{o}{=} \PYG{l+m+mi}{20} \PYG{c+c1}{\PYGZsh{} cm, Length of the column}
\PYG{n}{d\PYGZus{}t} \PYG{o}{=} \PYG{l+m+mi}{4} \PYG{c+c1}{\PYGZsh{} cm, diameter of the tube}
\PYG{n}{d\PYGZus{}c} \PYG{o}{=} \PYG{l+m+mi}{6}\PYG{c+c1}{\PYGZsh{} cm, diameter of the column}
\PYG{n}{slope} \PYG{o}{=} \PYG{n}{slope} \PYG{c+c1}{\PYGZsh{} 1/s, from fitting see plot}

\PYG{n}{K} \PYG{o}{=} \PYG{n}{L}\PYG{o}{*}\PYG{p}{(}\PYG{n}{d\PYGZus{}t}\PYG{o}{*}\PYG{o}{*}\PYG{l+m+mi}{2}\PYG{o}{/}\PYG{n}{d\PYGZus{}c}\PYG{o}{*}\PYG{o}{*}\PYG{l+m+mi}{2}\PYG{p}{)}\PYG{o}{*}\PYG{n}{slope} \PYG{c+c1}{\PYGZsh{} cm/s, conductivity calculated using eqn from previous slide}
\PYG{n}{K\PYGZus{}m} \PYG{o}{=} \PYG{n}{K}\PYG{o}{/}\PYG{l+m+mi}{100} \PYG{c+c1}{\PYGZsh{} m/s, conductivity }

\PYG{c+c1}{\PYGZsh{}output}
\PYG{n+nb}{print}\PYG{p}{(}\PYG{l+s+s1}{\PYGZsq{}}\PYG{l+s+se}{\PYGZbs{}033}\PYG{l+s+s1}{[1m}\PYG{l+s+s1}{\PYGZsq{}} \PYG{o}{+} \PYG{l+s+s1}{\PYGZsq{}}\PYG{l+s+s1}{Results are:}\PYG{l+s+s1}{\PYGZsq{}} \PYG{o}{+} \PYG{l+s+s1}{\PYGZsq{}}\PYG{l+s+se}{\PYGZbs{}033}\PYG{l+s+s1}{[0m }\PYG{l+s+se}{\PYGZbs{}n}\PYG{l+s+s1}{\PYGZsq{}}\PYG{p}{)}
\PYG{n+nb}{print}\PYG{p}{(}\PYG{l+s+s2}{\PYGZdq{}}\PYG{l+s+s2}{The conductivity in the column is: }\PYG{l+s+si}{\PYGZob{}0:1.2e\PYGZcb{}}\PYG{l+s+s2}{\PYGZdq{}}\PYG{o}{.}\PYG{n}{format}\PYG{p}{(}\PYG{n}{K}\PYG{p}{)}\PYG{p}{,} \PYG{l+s+s2}{\PYGZdq{}}\PYG{l+s+s2}{cm/s}\PYG{l+s+se}{\PYGZbs{}n}\PYG{l+s+s2}{\PYGZdq{}}\PYG{p}{)}
\PYG{n+nb}{print}\PYG{p}{(}\PYG{l+s+s2}{\PYGZdq{}}\PYG{l+s+s2}{The conductivity in the column is: }\PYG{l+s+si}{\PYGZob{}0:1.2e\PYGZcb{}}\PYG{l+s+s2}{\PYGZdq{}}\PYG{o}{.}\PYG{n}{format}\PYG{p}{(}\PYG{n}{K\PYGZus{}m}\PYG{p}{)}\PYG{p}{,} \PYG{l+s+s2}{\PYGZdq{}}\PYG{l+s+s2}{m/s}\PYG{l+s+se}{\PYGZbs{}n}\PYG{l+s+s2}{\PYGZdq{}}\PYG{p}{)}

\PYG{c+c1}{\PYGZsh{}Solution of 4D}
\PYG{c+c1}{\PYGZsh{} Given}
\PYG{n}{rho\PYGZus{}w} \PYG{o}{=} \PYG{l+m+mf}{998.2} \PYG{c+c1}{\PYGZsh{} kg/m\PYGZca{}3, density of water}
\PYG{n}{eta\PYGZus{}w} \PYG{o}{=} \PYG{l+m+mf}{1.0087E\PYGZhy{}3}\PYG{c+c1}{\PYGZsh{} kg/(m\PYGZhy{}s), viscocity of water}
\PYG{n}{g} \PYG{o}{=} \PYG{l+m+mf}{9.81} \PYG{c+c1}{\PYGZsh{} m/s\PYGZca{}2, accl. due to gravity}

\PYG{n}{k} \PYG{o}{=} \PYG{n}{K}\PYG{o}{/}\PYG{l+m+mi}{100}\PYG{o}{*}\PYG{n}{eta\PYGZus{}w}\PYG{o}{/}\PYG{p}{(}\PYG{n}{rho\PYGZus{}w}\PYG{o}{*}\PYG{n}{g}\PYG{p}{)}\PYG{c+c1}{\PYGZsh{} m\PYGZca{}2, K = k*ρ*g/n}
\PYG{n}{k\PYGZus{}D} \PYG{o}{=} \PYG{n}{k}\PYG{o}{/}\PYG{l+m+mf}{0.987E\PYGZhy{}12} \PYG{c+c1}{\PYGZsh{} D, 1D = 0.987E10\PYGZhy{}12 m\PYGZca{}2}

\PYG{n+nb}{print}\PYG{p}{(}\PYG{l+s+s2}{\PYGZdq{}}\PYG{l+s+s2}{The permeability of the media is: }\PYG{l+s+si}{\PYGZob{}0:1.2e\PYGZcb{}}\PYG{l+s+s2}{\PYGZdq{}}\PYG{o}{.}\PYG{n}{format}\PYG{p}{(}\PYG{n}{k}\PYG{p}{)}\PYG{p}{,} \PYG{l+s+s2}{\PYGZdq{}}\PYG{l+s+s2}{m}\PYG{l+s+se}{\PYGZbs{}u00b2}\PYG{l+s+s2}{ }\PYG{l+s+se}{\PYGZbs{}n}\PYG{l+s+s2}{\PYGZdq{}}\PYG{p}{)}  
\PYG{n+nb}{print}\PYG{p}{(}\PYG{l+s+s2}{\PYGZdq{}}\PYG{l+s+s2}{The permeability of the media in Darcy}\PYG{l+s+s2}{\PYGZsq{}}\PYG{l+s+s2}{s unit is: }\PYG{l+s+si}{\PYGZob{}0:1.2f\PYGZcb{}}\PYG{l+s+s2}{\PYGZdq{}}\PYG{o}{.}\PYG{n}{format}\PYG{p}{(}\PYG{n}{k\PYGZus{}D}\PYG{p}{)}\PYG{p}{,} \PYG{l+s+s2}{\PYGZdq{}}\PYG{l+s+s2}{D}\PYG{l+s+s2}{\PYGZdq{}}\PYG{p}{)}  
\end{sphinxVerbatim}

\end{sphinxuseclass}\end{sphinxVerbatimInput}
\begin{sphinxVerbatimOutput}

\begin{sphinxuseclass}{cell_output}
\begin{sphinxVerbatim}[commandchars=\\\{\}]
\PYG{Color+ColorBold}{Results are:} 

The conductivity in the column is: 4.91e\PYGZhy{}03 cm/s

The conductivity in the column is: 4.91e\PYGZhy{}05 m/s

The permeability of the media is: 5.06e\PYGZhy{}12 m² 

The permeability of the media in Darcy\PYGZsq{}s unit is: 5.13 D
\end{sphinxVerbatim}

\end{sphinxuseclass}\end{sphinxVerbatimOutput}

\end{sphinxuseclass}

\section{Tutorial Problems}
\label{\detokenize{content/tutorials/T6/tutorial_06:tutorial-problems}}

\subsection{Tutorial Problem 16\sphinxhyphen{} flow in confined aquifer}
\label{\detokenize{content/tutorials/T6/tutorial_06:tutorial-problem-16-flow-in-confined-aquifer}}
\begin{sphinxuseclass}{cell}
\begin{sphinxuseclass}{tag_hide-input}\begin{sphinxVerbatimOutput}

\begin{sphinxuseclass}{cell_output}
\begin{sphinxVerbatim}[commandchars=\\\{\}]
Row
    [0] LaTeX(str, style=\PYGZob{}\PYGZsq{}font\PYGZhy{}size\PYGZsq{}: \PYGZsq{}12pt\PYGZsq{}\PYGZcb{}, width=600)
    [1] PNG(str, width=350)
\end{sphinxVerbatim}

\end{sphinxuseclass}\end{sphinxVerbatimOutput}

\end{sphinxuseclass}
\end{sphinxuseclass}

\subsection{Solution of Problem 16}
\label{\detokenize{content/tutorials/T6/tutorial_06:solution-of-problem-16}}
\begin{sphinxuseclass}{cell}
\begin{sphinxuseclass}{tag_hide-input}\begin{sphinxVerbatimOutput}

\begin{sphinxuseclass}{cell_output}
\begin{sphinxVerbatim}[commandchars=\\\{\}]
LaTeX(str, style=\PYGZob{}\PYGZsq{}font\PYGZhy{}size\PYGZsq{}: \PYGZsq{}12pt\PYGZsq{}\PYGZcb{}, width=900)
\end{sphinxVerbatim}

\end{sphinxuseclass}\end{sphinxVerbatimOutput}

\end{sphinxuseclass}
\end{sphinxuseclass}
\begin{sphinxuseclass}{cell}\begin{sphinxVerbatimInput}

\begin{sphinxuseclass}{cell_input}
\begin{sphinxVerbatim}[commandchars=\\\{\}]
\PYG{c+c1}{\PYGZsh{} Given are:}
\PYG{n}{m\PYGZus{}1} \PYG{o}{=} \PYG{l+m+mi}{30} \PYG{c+c1}{\PYGZsh{} m, uniform thinckness of aquifer }
\PYG{n}{w\PYGZus{}1} \PYG{o}{=} \PYG{l+m+mi}{5} \PYG{c+c1}{\PYGZsh{} km, width of the aquifer}
\PYG{n}{d\PYGZus{}l} \PYG{o}{=} \PYG{l+m+mf}{1.5} \PYG{c+c1}{\PYGZsh{} km, distance between wells}
\PYG{n}{hy1\PYGZus{}w1} \PYG{o}{=} \PYG{l+m+mi}{90} \PYG{c+c1}{\PYGZsh{} m, head in well 1}
\PYG{n}{hy1\PYGZus{}w2} \PYG{o}{=} \PYG{l+m+mi}{85} \PYG{c+c1}{\PYGZsh{} m, head in well 2}
\PYG{n}{K\PYGZus{}1} \PYG{o}{=} \PYG{l+m+mf}{1.5} \PYG{c+c1}{\PYGZsh{} m/d, conductivity in aquifer}
\PYG{n}{d\PYGZus{}x} \PYG{o}{=} \PYG{l+m+mf}{1.5} \PYG{c+c1}{\PYGZsh{} km, distance from head 1}

\PYG{c+c1}{\PYGZsh{} interim calculation}
\PYG{n}{w\PYGZus{}1m} \PYG{o}{=} \PYG{n}{w\PYGZus{}1}\PYG{o}{*}\PYG{l+m+mi}{1000} \PYG{c+c1}{\PYGZsh{} m, widht of the aquifer}
\PYG{n}{d\PYGZus{}lm} \PYG{o}{=} \PYG{n}{d\PYGZus{}l}\PYG{o}{*}\PYG{l+m+mi}{1000} \PYG{c+c1}{\PYGZsh{} m, distance between wells}
\PYG{n}{d\PYGZus{}xm} \PYG{o}{=} \PYG{n}{d\PYGZus{}x}\PYG{o}{*}\PYG{l+m+mi}{1000} \PYG{c+c1}{\PYGZsh{} m, distance from head 1}

\PYG{c+c1}{\PYGZsh{}Solution 1}
\PYG{n}{dh\PYGZus{}y1} \PYG{o}{=} \PYG{p}{(}\PYG{n}{hy1\PYGZus{}w1} \PYG{o}{\PYGZhy{}} \PYG{n}{hy1\PYGZus{}w2}\PYG{p}{)}\PYG{o}{/}\PYG{n}{d\PYGZus{}lm} \PYG{c+c1}{\PYGZsh{} (\PYGZhy{}), head gradient}
\PYG{n}{Q\PYGZus{}y1} \PYG{o}{=} \PYG{n}{K\PYGZus{}1}\PYG{o}{*}\PYG{n}{m\PYGZus{}1}\PYG{o}{*}\PYG{n}{dh\PYGZus{}y1}\PYG{o}{*}\PYG{n}{w\PYGZus{}1m} \PYG{c+c1}{\PYGZsh{} m\PYGZca{}3/day, discharge using the first eq. above.}

\PYG{c+c1}{\PYGZsh{}Solution 2 }
\PYG{n}{q\PYGZus{}1} \PYG{o}{=} \PYG{n}{Q\PYGZus{}y1}\PYG{o}{/}\PYG{n}{w\PYGZus{}1m} \PYG{c+c1}{\PYGZsh{} m\PYGZca{}2/d, flow per unit width}
\PYG{n}{h\PYGZus{}y1} \PYG{o}{=} \PYG{n}{hy1\PYGZus{}w1}\PYG{o}{\PYGZhy{}}\PYG{p}{(}\PYG{n}{q\PYGZus{}1}\PYG{o}{/}\PYG{p}{(}\PYG{n}{K\PYGZus{}1}\PYG{o}{*}\PYG{n}{m\PYGZus{}1}\PYG{p}{)}\PYG{p}{)}\PYG{o}{*}\PYG{n}{d\PYGZus{}xm} \PYG{c+c1}{\PYGZsh{} head at 0.3 Km from Well 1, using the second equation }


\PYG{c+c1}{\PYGZsh{}output}
\PYG{n+nb}{print}\PYG{p}{(}\PYG{l+s+s1}{\PYGZsq{}}\PYG{l+s+se}{\PYGZbs{}033}\PYG{l+s+s1}{[1m}\PYG{l+s+s1}{\PYGZsq{}} \PYG{o}{+} \PYG{l+s+s1}{\PYGZsq{}}\PYG{l+s+s1}{Results are:}\PYG{l+s+s1}{\PYGZsq{}} \PYG{o}{+} \PYG{l+s+s1}{\PYGZsq{}}\PYG{l+s+se}{\PYGZbs{}033}\PYG{l+s+s1}{[0m }\PYG{l+s+se}{\PYGZbs{}n}\PYG{l+s+s1}{\PYGZsq{}}\PYG{p}{)}
\PYG{n+nb}{print}\PYG{p}{(}\PYG{l+s+s2}{\PYGZdq{}}\PYG{l+s+s2}{The daily discharge from the aquifer is: }\PYG{l+s+si}{\PYGZob{}0:1.2f\PYGZcb{}}\PYG{l+s+s2}{\PYGZdq{}}\PYG{o}{.}\PYG{n}{format}\PYG{p}{(}\PYG{n}{Q\PYGZus{}y1}\PYG{p}{)}\PYG{p}{,} \PYG{l+s+s2}{\PYGZdq{}}\PYG{l+s+s2}{m}\PYG{l+s+se}{\PYGZbs{}u00b3}\PYG{l+s+s2}{/d}\PYG{l+s+se}{\PYGZbs{}n}\PYG{l+s+s2}{\PYGZdq{}}\PYG{p}{)}
\PYG{n+nb}{print}\PYG{p}{(}\PYG{l+s+s2}{\PYGZdq{}}\PYG{l+s+s2}{The head at 0.5 Km from well 1 is : }\PYG{l+s+si}{\PYGZob{}0:1.2f\PYGZcb{}}\PYG{l+s+s2}{\PYGZdq{}}\PYG{o}{.}\PYG{n}{format}\PYG{p}{(}\PYG{n}{h\PYGZus{}y1}\PYG{p}{)}\PYG{p}{,} \PYG{l+s+s2}{\PYGZdq{}}\PYG{l+s+s2}{m}\PYG{l+s+s2}{\PYGZdq{}}\PYG{p}{)}
\end{sphinxVerbatim}

\end{sphinxuseclass}\end{sphinxVerbatimInput}
\begin{sphinxVerbatimOutput}

\begin{sphinxuseclass}{cell_output}
\begin{sphinxVerbatim}[commandchars=\\\{\}]
\PYG{Color+ColorBold}{Results are:} 

The daily discharge from the aquifer is: 750.00 m³/d

The head at 0.5 Km from well 1 is : 85.00 m
\end{sphinxVerbatim}

\end{sphinxuseclass}\end{sphinxVerbatimOutput}

\end{sphinxuseclass}

\subsection{Tutorial Problem 17\sphinxhyphen{} flow in unconfined aquifer}
\label{\detokenize{content/tutorials/T6/tutorial_06:tutorial-problem-17-flow-in-unconfined-aquifer}}
\begin{sphinxuseclass}{cell}
\begin{sphinxuseclass}{tag_hide-input}\begin{sphinxVerbatimOutput}

\begin{sphinxuseclass}{cell_output}
\begin{sphinxVerbatim}[commandchars=\\\{\}]
Row
    [0] LaTeX(str, style=\PYGZob{}\PYGZsq{}font\PYGZhy{}size\PYGZsq{}: \PYGZsq{}12pt\PYGZsq{}\PYGZcb{}, width=500)
    [1] PNG(str, width=400)
\end{sphinxVerbatim}

\end{sphinxuseclass}\end{sphinxVerbatimOutput}

\end{sphinxuseclass}
\end{sphinxuseclass}

\subsection{Solution Tutorial Problem 17}
\label{\detokenize{content/tutorials/T6/tutorial_06:solution-tutorial-problem-17}}
\begin{sphinxuseclass}{cell}
\begin{sphinxuseclass}{tag_hide-input}\begin{sphinxVerbatimOutput}

\begin{sphinxuseclass}{cell_output}
\begin{sphinxVerbatim}[commandchars=\\\{\}]
LaTeX(str, style=\PYGZob{}\PYGZsq{}font\PYGZhy{}size\PYGZsq{}: \PYGZsq{}12pt\PYGZsq{}\PYGZcb{}, width=700)
\end{sphinxVerbatim}

\end{sphinxuseclass}\end{sphinxVerbatimOutput}

\end{sphinxuseclass}
\end{sphinxuseclass}
\begin{sphinxuseclass}{cell}\begin{sphinxVerbatimInput}

\begin{sphinxuseclass}{cell_input}
\begin{sphinxVerbatim}[commandchars=\\\{\}]
\PYG{c+c1}{\PYGZsh{}Solution of Tutorial Problem 22:}
\PYG{c+c1}{\PYGZsh{} Given}

\PYG{n}{h3\PYGZus{}1} \PYG{o}{=} \PYG{l+m+mi}{20} \PYG{c+c1}{\PYGZsh{} m, aquifer head at point 1}
\PYG{n}{h3\PYGZus{}2} \PYG{o}{=} \PYG{l+m+mi}{19} \PYG{c+c1}{\PYGZsh{} m, aquifer head at point 2}
\PYG{n}{K3} \PYG{o}{=} \PYG{l+m+mi}{5} \PYG{o}{*} \PYG{l+m+mi}{10}\PYG{o}{*}\PYG{o}{*}\PYG{o}{\PYGZhy{}}\PYG{l+m+mi}{4} \PYG{c+c1}{\PYGZsh{} m/s uniform conductivity of aquifer}
\PYG{n}{K4} \PYG{o}{=} \PYG{l+m+mf}{5e\PYGZhy{}4} \PYG{c+c1}{\PYGZsh{} m/s uniform conductivity of aquifer}
\PYG{n}{L3} \PYG{o}{=} \PYG{l+m+mi}{50} \PYG{c+c1}{\PYGZsh{} m, length of the aquifer}
\PYG{n}{W3} \PYG{o}{=} \PYG{l+m+mi}{30} \PYG{c+c1}{\PYGZsh{} m, width of the aquifer}

\PYG{c+c1}{\PYGZsh{}Calculation}
\PYG{n}{q3} \PYG{o}{=} \PYG{o}{\PYGZhy{}}\PYG{l+m+mi}{1}\PYG{o}{/}\PYG{l+m+mi}{2}\PYG{o}{*}\PYG{n}{K3}\PYG{o}{*}\PYG{p}{(}\PYG{p}{(}\PYG{n}{h3\PYGZus{}2}\PYG{o}{*}\PYG{o}{*}\PYG{l+m+mi}{2} \PYG{o}{\PYGZhy{}} \PYG{n}{h3\PYGZus{}1}\PYG{o}{*}\PYG{o}{*}\PYG{l+m+mi}{2}\PYG{p}{)}\PYG{o}{/}\PYG{n}{L3}\PYG{p}{)} \PYG{c+c1}{\PYGZsh{} m\PYGZca{}2/s, unit width discharge using eq. 3B}
\PYG{n}{Q3} \PYG{o}{=} \PYG{n}{q3} \PYG{o}{*} \PYG{n}{W3} \PYG{c+c1}{\PYGZsh{} m\PYGZca{}3/s, total dischage from given width}

\PYG{c+c1}{\PYGZsh{}output}
\PYG{c+c1}{\PYGZsh{}print(\PYGZsq{}\PYGZbs{}033[1m\PYGZsq{} + \PYGZsq{}Results are:\PYGZsq{} + \PYGZsq{}\PYGZbs{}033[0m \PYGZbs{}n\PYGZsq{})}
\PYG{c+c1}{\PYGZsh{}print(\PYGZdq{}Discharge per unit width of aquifer is: \PYGZob{}0:1.2e\PYGZcb{}\PYGZdq{}.format(q3), \PYGZdq{}m\PYGZbs{}u00b2/s \PYGZbs{}n\PYGZdq{})}
\PYG{c+c1}{\PYGZsh{}print(\PYGZdq{}Discharge from the given width of aquifer is: \PYGZob{}0:1.2e\PYGZcb{}\PYGZdq{}.format(Q3), \PYGZdq{}m\PYGZbs{}u00b3/s\PYGZdq{}) }

\PYG{n+nb}{print}\PYG{p}{(}\PYG{l+s+s2}{\PYGZdq{}}\PYG{l+s+s2}{Conductivity: }\PYG{l+s+si}{\PYGZob{}0:1.2e\PYGZcb{}}\PYG{l+s+s2}{\PYGZdq{}}\PYG{o}{.}\PYG{n}{format}\PYG{p}{(}\PYG{n}{K3}\PYG{p}{)}\PYG{p}{,} \PYG{l+s+s2}{\PYGZdq{}}\PYG{l+s+s2}{m}\PYG{l+s+se}{\PYGZbs{}u00b3}\PYG{l+s+s2}{/s}\PYG{l+s+s2}{\PYGZdq{}}\PYG{p}{)}
\PYG{n+nb}{print}\PYG{p}{(}\PYG{l+s+s2}{\PYGZdq{}}\PYG{l+s+s2}{Conductivity: }\PYG{l+s+si}{\PYGZob{}0:1.5f\PYGZcb{}}\PYG{l+s+s2}{\PYGZdq{}}\PYG{o}{.}\PYG{n}{format}\PYG{p}{(}\PYG{n}{K3}\PYG{p}{)}\PYG{p}{,} \PYG{l+s+s2}{\PYGZdq{}}\PYG{l+s+s2}{m}\PYG{l+s+se}{\PYGZbs{}u00b3}\PYG{l+s+s2}{/s}\PYG{l+s+s2}{\PYGZdq{}}\PYG{p}{)}
\PYG{n+nb}{print}\PYG{p}{(}\PYG{n}{K4}\PYG{p}{)}
\end{sphinxVerbatim}

\end{sphinxuseclass}\end{sphinxVerbatimInput}
\begin{sphinxVerbatimOutput}

\begin{sphinxuseclass}{cell_output}
\begin{sphinxVerbatim}[commandchars=\\\{\}]
Conductivity: 5.00e\PYGZhy{}04 m³/s
Conductivity: 0.00050 m³/s
0.0005
\end{sphinxVerbatim}

\end{sphinxuseclass}\end{sphinxVerbatimOutput}

\end{sphinxuseclass}

\section{Homework Problems Flow problems}
\label{\detokenize{content/tutorials/T6/tutorial_06:homework-problems-flow-problems}}
\sphinxAtStartPar
These problems will require you to make some research. I will suggest that you check the \sphinxstylestrong{Groundwater} book by \sphinxstylestrong{R. Allan Freeze} and \sphinxstylestrong{John A. Cherry}. The book is now freely available at \sphinxurl{https://gw-project.org/books/groundwater/}

\sphinxAtStartPar
\sphinxstylestrong{There is no\sphinxhyphen{}obligation to submit the homework problems}


\subsection{Homework Problem 8 \sphinxhyphen{} confined aquifer}
\label{\detokenize{content/tutorials/T6/tutorial_06:homework-problem-8-confined-aquifer}}
\begin{sphinxuseclass}{cell}
\begin{sphinxuseclass}{tag_hide-input}\begin{sphinxVerbatimOutput}

\begin{sphinxuseclass}{cell_output}
\begin{sphinxVerbatim}[commandchars=\\\{\}]
Row
    [0] LaTeX(str, style=\PYGZob{}\PYGZsq{}font\PYGZhy{}size\PYGZsq{}: \PYGZsq{}12pt\PYGZsq{}\PYGZcb{}, width=400)
    [1] PNG(str, width=500)
\end{sphinxVerbatim}

\end{sphinxuseclass}\end{sphinxVerbatimOutput}

\end{sphinxuseclass}
\end{sphinxuseclass}

\subsection{Homework Problem 9 \sphinxhyphen{} unconfined aquifer}
\label{\detokenize{content/tutorials/T6/tutorial_06:homework-problem-9-unconfined-aquifer}}
\begin{sphinxuseclass}{cell}
\begin{sphinxuseclass}{tag_hide-input}\begin{sphinxVerbatimOutput}

\begin{sphinxuseclass}{cell_output}
\begin{sphinxVerbatim}[commandchars=\\\{\}]
Column
    [0] LaTeX(str, style=\PYGZob{}\PYGZsq{}font\PYGZhy{}size\PYGZsq{}: \PYGZsq{}12pt\PYGZsq{}\PYGZcb{}, width=900)
    [1] PNG(str, width=450)
\end{sphinxVerbatim}

\end{sphinxuseclass}\end{sphinxVerbatimOutput}

\end{sphinxuseclass}
\end{sphinxuseclass}
\sphinxstepscope

\begin{sphinxuseclass}{cell}
\begin{sphinxuseclass}{tag_remove-output}\begin{sphinxVerbatimInput}

\begin{sphinxuseclass}{cell_input}
\begin{sphinxVerbatim}[commandchars=\\\{\}]
\PYG{k+kn}{import} \PYG{n+nn}{numpy} \PYG{k}{as} \PYG{n+nn}{np}
\PYG{k+kn}{import} \PYG{n+nn}{matplotlib}\PYG{n+nn}{.}\PYG{n+nn}{pyplot} \PYG{k}{as} \PYG{n+nn}{plt}
\PYG{k+kn}{import} \PYG{n+nn}{pandas} \PYG{k}{as} \PYG{n+nn}{pd} 
\PYG{k+kn}{import} \PYG{n+nn}{panel} \PYG{k}{as} \PYG{n+nn}{pn}
\PYG{n}{pn}\PYG{o}{.}\PYG{n}{extension}\PYG{p}{(}\PYG{l+s+s1}{\PYGZsq{}}\PYG{l+s+s1}{katex}\PYG{l+s+s1}{\PYGZsq{}}\PYG{p}{)} 
\end{sphinxVerbatim}

\end{sphinxuseclass}\end{sphinxVerbatimInput}

\end{sphinxuseclass}
\end{sphinxuseclass}

\chapter{Tutorial 7 \sphinxhyphen{} Wells}
\label{\detokenize{content/tutorials/T7/tutorial_07:tutorial-7-wells}}\label{\detokenize{content/tutorials/T7/tutorial_07::doc}}

\begin{enumerate}
\sphinxsetlistlabels{\arabic}{enumi}{enumii}{}{.}%
\item {} 
\sphinxAtStartPar
\sphinxstylestrong{Solution of Homework Problems 5 \sphinxhyphen{}6}


\item {} 
\sphinxAtStartPar
\sphinxstylestrong{Tutorial Problems on Wells}


\item {} 
\sphinxAtStartPar
\sphinxstylestrong{Homework Problems on Wells}

\end{enumerate}




\section{Homework Problem 5: Effective Hydraulic Conductivity}
\label{\detokenize{content/tutorials/T7/tutorial_07:homework-problem-5-effective-hydraulic-conductivity}}
\begin{sphinxuseclass}{cell}
\begin{sphinxuseclass}{tag_hide-input}\begin{sphinxVerbatimOutput}

\begin{sphinxuseclass}{cell_output}
\begin{sphinxVerbatim}[commandchars=\\\{\}]
Markdown(str, style=\PYGZob{}\PYGZsq{}font\PYGZhy{}size\PYGZsq{}: \PYGZsq{}12pt\PYGZsq{}\PYGZcb{}, width=900)
\end{sphinxVerbatim}

\end{sphinxuseclass}\end{sphinxVerbatimOutput}

\end{sphinxuseclass}
\end{sphinxuseclass}

\subsection{Solution}
\label{\detokenize{content/tutorials/T7/tutorial_07:solution}}
\sphinxAtStartPar
The given problem can be visualized as:

\noindent{\hspace*{\fill}\sphinxincludegraphics[width=350\sphinxpxdimen]{{T07_fH5}.png}\hspace*{\fill}}

\sphinxAtStartPar
We will use Darcy’s law for the homogeneous system, which is given as
\begin{equation*}
\begin{split}
Q/A = K_{eff} \cdot \Delta h/m
\end{split}
\end{equation*}
\sphinxAtStartPar
where \(m = 2\times m_1 + m_2\) is the total thickness

\sphinxAtStartPar
For vertical flow, the effective conductivity is \(K_v\)
\begin{equation*}
\begin{split}
K_v = \frac{m}{\sum\limits_{i=1}^n\frac{m_i}{K_i}}
\end{split}
\end{equation*}\begin{equation*}
\begin{split}
K_h = \frac{\sum\limits_{i=1}^n m_i\cdot K_i}{m}
\end{split}
\end{equation*}
\sphinxAtStartPar
And for the inclined flow, the known relation fro \(K_{eff}\) is
\begin{equation*}
\begin{split}
K_\theta = \frac{1}{\frac{\cos^2\theta}{K_h}+ \frac{\sin^2\theta}{K_v}}
\end{split}
\end{equation*}
\begin{sphinxuseclass}{cell}\begin{sphinxVerbatimInput}

\begin{sphinxuseclass}{cell_input}
\begin{sphinxVerbatim}[commandchars=\\\{\}]
\PYG{c+c1}{\PYGZsh{} Given}

\PYG{n}{Q\PYGZus{}A} \PYG{o}{=} \PYG{l+m+mi}{500} \PYG{c+c1}{\PYGZsh{} L/d/m\PYGZca{}2, discharge/m\PYGZca{}2}
\PYG{n}{D\PYGZus{}h} \PYG{o}{=} \PYG{l+m+mf}{5.5} \PYG{c+c1}{\PYGZsh{} cm, head difference}
\PYG{n}{m1}\PYG{o}{=} \PYG{n}{m3} \PYG{o}{=} \PYG{l+m+mf}{1.5} \PYG{c+c1}{\PYGZsh{} m, layer thickness for layer 1 and 3}
\PYG{n}{K1}\PYG{o}{=} \PYG{n}{K3} \PYG{o}{=} \PYG{l+m+mf}{3.7e\PYGZhy{}4} \PYG{c+c1}{\PYGZsh{} m/s, Conductivity of layer 1 and 3}
\PYG{n}{m2} \PYG{o}{=} \PYG{l+m+mf}{2.5} \PYG{c+c1}{\PYGZsh{} m, thickness of layer 2}

\PYG{c+c1}{\PYGZsh{} interim calculation}
\PYG{n}{Q\PYGZus{}Am} \PYG{o}{=} \PYG{n}{Q\PYGZus{}A}\PYG{o}{/}\PYG{l+m+mi}{1000} \PYG{c+c1}{\PYGZsh{} m/d/m\PYGZca{}2, discharge/m\PYGZca{}2 \PYGZhy{} unit change}
\PYG{n}{D\PYGZus{}hm} \PYG{o}{=} \PYG{n}{D\PYGZus{}h}\PYG{o}{/}\PYG{l+m+mi}{100} \PYG{c+c1}{\PYGZsh{} m, unit change, head difference}
\PYG{n}{m} \PYG{o}{=} \PYG{n}{m1}\PYG{o}{+}\PYG{n}{m2}\PYG{o}{+}\PYG{n}{m3} \PYG{c+c1}{\PYGZsh{} m, total thickness}

\PYG{c+c1}{\PYGZsh{} solution part a. (from the first equation above)}

\PYG{n}{K\PYGZus{}eff} \PYG{o}{=} \PYG{n}{Q\PYGZus{}Am}\PYG{o}{*}\PYG{n}{m}\PYG{o}{/}\PYG{n}{D\PYGZus{}hm} \PYG{c+c1}{\PYGZsh{} m/d, Q/A*m/Dh =\PYGZgt{} Q/A is given}
\PYG{n}{K\PYGZus{}effs} \PYG{o}{=} \PYG{n}{K\PYGZus{}eff}\PYG{o}{/}\PYG{p}{(}\PYG{l+m+mi}{24}\PYG{o}{*}\PYG{l+m+mi}{60}\PYG{o}{*}\PYG{l+m+mi}{60}\PYG{p}{)}

\PYG{c+c1}{\PYGZsh{}output}
\PYG{n+nb}{print}\PYG{p}{(}\PYG{l+s+s2}{\PYGZdq{}}\PYG{l+s+se}{\PYGZbs{}033}\PYG{l+s+s2}{[1m Result of HW Problem 5(a) are:}\PYG{l+s+se}{\PYGZbs{}033}\PYG{l+s+s2}{[0m}\PYG{l+s+se}{\PYGZbs{}n}\PYG{l+s+s2}{\PYGZdq{}}\PYG{p}{)}
\PYG{n+nb}{print}\PYG{p}{(}\PYG{l+s+s2}{\PYGZdq{}}\PYG{l+s+s2}{The effective conductivity is }\PYG{l+s+si}{\PYGZob{}0:0.1f\PYGZcb{}}\PYG{l+s+s2}{\PYGZdq{}}\PYG{o}{.}\PYG{n}{format}\PYG{p}{(}\PYG{n}{K\PYGZus{}eff}\PYG{p}{)}\PYG{p}{,} \PYG{l+s+s2}{\PYGZdq{}}\PYG{l+s+s2}{m/d }\PYG{l+s+se}{\PYGZbs{}n}\PYG{l+s+s2}{\PYGZdq{}}\PYG{p}{)}
\PYG{n+nb}{print}\PYG{p}{(}\PYG{l+s+s2}{\PYGZdq{}}\PYG{l+s+s2}{The effective conductivity is }\PYG{l+s+si}{\PYGZob{}0:0.2e\PYGZcb{}}\PYG{l+s+s2}{\PYGZdq{}}\PYG{o}{.}\PYG{n}{format}\PYG{p}{(}\PYG{n}{K\PYGZus{}effs}\PYG{p}{)}\PYG{p}{,} \PYG{l+s+s2}{\PYGZdq{}}\PYG{l+s+s2}{m/s}\PYG{l+s+s2}{\PYGZdq{}}\PYG{p}{)}
\end{sphinxVerbatim}

\end{sphinxuseclass}\end{sphinxVerbatimInput}
\begin{sphinxVerbatimOutput}

\begin{sphinxuseclass}{cell_output}
\begin{sphinxVerbatim}[commandchars=\\\{\}]
\PYG{Color+ColorBold}{ Result of HW Problem 5(a) are:}

The effective conductivity is 50.0 m/d 

The effective conductivity is 5.79e\PYGZhy{}04 m/s
\end{sphinxVerbatim}

\end{sphinxuseclass}\end{sphinxVerbatimOutput}

\end{sphinxuseclass}
\begin{sphinxuseclass}{cell}\begin{sphinxVerbatimInput}

\begin{sphinxuseclass}{cell_input}
\begin{sphinxVerbatim}[commandchars=\\\{\}]
\PYG{c+c1}{\PYGZsh{} solution 5b and 5c}

\PYG{n}{K\PYGZus{}v} \PYG{o}{=} \PYG{n}{K\PYGZus{}effs} \PYG{c+c1}{\PYGZsh{} m/s, Keff = Kv}

\PYG{c+c1}{\PYGZsh{} Re\PYGZhy{}organizing the second equation above}
\PYG{c+c1}{\PYGZsh{} Solution}

\PYG{n}{K2} \PYG{o}{=} \PYG{n}{m2}\PYG{o}{/}\PYG{p}{(}\PYG{p}{(}\PYG{n}{m}\PYG{o}{/}\PYG{n}{K\PYGZus{}v}\PYG{p}{)} \PYG{o}{\PYGZhy{}} \PYG{l+m+mi}{2}\PYG{o}{*}\PYG{p}{(}\PYG{n}{m1}\PYG{o}{/}\PYG{n}{K1}\PYG{p}{)}\PYG{p}{)} \PYG{c+c1}{\PYGZsh{} m/s, conductivity of layer 2}

\PYG{c+c1}{\PYGZsh{}output}
\PYG{n+nb}{print}\PYG{p}{(}\PYG{l+s+s2}{\PYGZdq{}}\PYG{l+s+se}{\PYGZbs{}033}\PYG{l+s+s2}{[1m Result of HW Problem 5(a) is:}\PYG{l+s+se}{\PYGZbs{}033}\PYG{l+s+s2}{[0m}\PYG{l+s+se}{\PYGZbs{}n}\PYG{l+s+s2}{\PYGZdq{}}\PYG{p}{)}
\PYG{n+nb}{print}\PYG{p}{(}\PYG{l+s+s2}{\PYGZdq{}}\PYG{l+s+s2}{The conductivity of layer 2 is }\PYG{l+s+si}{\PYGZob{}0:0.2e\PYGZcb{}}\PYG{l+s+s2}{\PYGZdq{}}\PYG{o}{.}\PYG{n}{format}\PYG{p}{(}\PYG{n}{K2}\PYG{p}{)}\PYG{p}{,} \PYG{l+s+s2}{\PYGZdq{}}\PYG{l+s+s2}{m/s }\PYG{l+s+se}{\PYGZbs{}n}\PYG{l+s+s2}{\PYGZdq{}}\PYG{p}{)}
\end{sphinxVerbatim}

\end{sphinxuseclass}\end{sphinxVerbatimInput}
\begin{sphinxVerbatimOutput}

\begin{sphinxuseclass}{cell_output}
\begin{sphinxVerbatim}[commandchars=\\\{\}]
\PYG{Color+ColorBold}{ Result of HW Problem 5(a) is:}

The conductivity of layer 2 is 1.79e\PYGZhy{}03 m/s 
\end{sphinxVerbatim}

\end{sphinxuseclass}\end{sphinxVerbatimOutput}

\end{sphinxuseclass}
\begin{sphinxuseclass}{cell}\begin{sphinxVerbatimInput}

\begin{sphinxuseclass}{cell_input}
\begin{sphinxVerbatim}[commandchars=\\\{\}]
\PYG{c+c1}{\PYGZsh{} Solution of part 5(c) and 5(d)}

\PYG{c+c1}{\PYGZsh{} 5(c) can be obtained from the third equation above.}

\PYG{n}{K\PYGZus{}h} \PYG{o}{=} \PYG{p}{(}\PYG{n}{m1}\PYG{o}{*}\PYG{n}{K1}\PYG{o}{+}\PYG{n}{m2}\PYG{o}{*}\PYG{n}{K2}\PYG{o}{+}\PYG{n}{m3}\PYG{o}{*}\PYG{n}{K3}\PYG{p}{)}\PYG{o}{/}\PYG{n}{m}

\PYG{c+c1}{\PYGZsh{} 5(d) can be obtained from the last equation above.}

\PYG{n}{theta} \PYG{o}{=} \PYG{l+m+mi}{30} \PYG{c+c1}{\PYGZsh{} degrees}

\PYG{c+c1}{\PYGZsh{} interim calculation }
\PYG{n}{theta\PYGZus{}r} \PYG{o}{=} \PYG{n}{theta}\PYG{o}{*}\PYG{n}{np}\PYG{o}{.}\PYG{n}{pi}\PYG{o}{/}\PYG{l+m+mi}{180} \PYG{c+c1}{\PYGZsh{} radian, degree must be changed into radian}

\PYG{n}{K\PYGZus{}theta} \PYG{o}{=} \PYG{l+m+mi}{1}\PYG{o}{/}\PYG{p}{(}\PYG{p}{(}\PYG{n}{np}\PYG{o}{.}\PYG{n}{cos}\PYG{p}{(}\PYG{n}{theta\PYGZus{}r}\PYG{p}{)}\PYG{o}{*}\PYG{o}{*}\PYG{l+m+mi}{2}\PYG{o}{/}\PYG{n}{K\PYGZus{}h}\PYG{p}{)}\PYG{o}{+} \PYG{p}{(}\PYG{n}{np}\PYG{o}{.}\PYG{n}{sin}\PYG{p}{(}\PYG{n}{theta\PYGZus{}r}\PYG{p}{)}\PYG{o}{*}\PYG{o}{*}\PYG{l+m+mi}{2}\PYG{o}{/}\PYG{n}{K\PYGZus{}v}\PYG{p}{)}\PYG{p}{)}

\PYG{c+c1}{\PYGZsh{} output}

\PYG{c+c1}{\PYGZsh{}output}
\PYG{n+nb}{print}\PYG{p}{(}\PYG{l+s+s2}{\PYGZdq{}}\PYG{l+s+se}{\PYGZbs{}033}\PYG{l+s+s2}{[1m Result of HW Problem 5(c) and 5(d) are:}\PYG{l+s+se}{\PYGZbs{}033}\PYG{l+s+s2}{[0m}\PYG{l+s+se}{\PYGZbs{}n}\PYG{l+s+s2}{\PYGZdq{}}\PYG{p}{)}
\PYG{n+nb}{print}\PYG{p}{(}\PYG{l+s+s2}{\PYGZdq{}}\PYG{l+s+s2}{The effective horizontal conductivity is }\PYG{l+s+si}{\PYGZob{}0:0.2e\PYGZcb{}}\PYG{l+s+s2}{\PYGZdq{}}\PYG{o}{.}\PYG{n}{format}\PYG{p}{(}\PYG{n}{K\PYGZus{}h}\PYG{p}{)}\PYG{p}{,} \PYG{l+s+s2}{\PYGZdq{}}\PYG{l+s+s2}{m/s }\PYG{l+s+se}{\PYGZbs{}n}\PYG{l+s+s2}{\PYGZdq{}}\PYG{p}{)}
\PYG{n+nb}{print}\PYG{p}{(}\PYG{l+s+s2}{\PYGZdq{}}\PYG{l+s+s2}{The effective conductivity when the discharge is 30 degree inclined is }\PYG{l+s+si}{\PYGZob{}0:0.2e\PYGZcb{}}\PYG{l+s+s2}{\PYGZdq{}}\PYG{o}{.}\PYG{n}{format}\PYG{p}{(}\PYG{n}{K\PYGZus{}theta}\PYG{p}{)}\PYG{p}{,} \PYG{l+s+s2}{\PYGZdq{}}\PYG{l+s+s2}{m/s}\PYG{l+s+s2}{\PYGZdq{}}\PYG{p}{)}
\end{sphinxVerbatim}

\end{sphinxuseclass}\end{sphinxVerbatimInput}
\begin{sphinxVerbatimOutput}

\begin{sphinxuseclass}{cell_output}
\begin{sphinxVerbatim}[commandchars=\\\{\}]
\PYG{Color+ColorBold}{ Result of HW Problem 5(c) and 5(d) are:}

The effective horizontal conductivity is 1.02e\PYGZhy{}03 m/s 

The effective conductivity when the discharge is 30 degree inclined is 8.55e\PYGZhy{}04 m/s
\end{sphinxVerbatim}

\end{sphinxuseclass}\end{sphinxVerbatimOutput}

\end{sphinxuseclass}

\section{Homework Problem 6: Flow net using triangulation}
\label{\detokenize{content/tutorials/T7/tutorial_07:homework-problem-6-flow-net-using-triangulation}}
\begin{sphinxuseclass}{cell}
\begin{sphinxuseclass}{tag_full-width}\begin{sphinxVerbatimInput}

\begin{sphinxuseclass}{cell_input}
\begin{sphinxVerbatim}[commandchars=\\\{\}]
\PYG{n}{r10\PYGZus{}1}\PYG{o}{=} \PYG{n}{pn}\PYG{o}{.}\PYG{n}{pane}\PYG{o}{.}\PYG{n}{Markdown}\PYG{p}{(}\PYG{l+s+s2}{\PYGZdq{}\PYGZdq{}\PYGZdq{}}
\PYG{l+s+s2}{The figure below shows the position of five groundwater observation wells with measured hydraulic heads in m a.s.l. }
\PYG{l+s+s2}{ \PYGZlt{}br\PYGZgt{}\PYGZlt{}br\PYGZgt{}}

\PYG{l+s+s2}{**a.** Sketch head isolines for intervals of 1 m by applying the hydrologic triangle method.}
\PYG{l+s+s2}{\PYGZlt{}br\PYGZgt{}\PYGZlt{}br\PYGZgt{}}
\PYG{l+s+s2}{**b.** Indicate the flow direction.\PYGZlt{}br\PYGZgt{}\PYGZlt{}br\PYGZgt{}}
\PYG{l+s+s2}{\PYGZdq{}\PYGZdq{}\PYGZdq{}}\PYG{p}{,} \PYG{n}{width} \PYG{o}{=} \PYG{l+m+mi}{500}\PYG{p}{,} \PYG{n}{style}\PYG{o}{=}\PYG{p}{\PYGZob{}}\PYG{l+s+s1}{\PYGZsq{}}\PYG{l+s+s1}{font\PYGZhy{}size}\PYG{l+s+s1}{\PYGZsq{}}\PYG{p}{:} \PYG{l+s+s1}{\PYGZsq{}}\PYG{l+s+s1}{13pt}\PYG{l+s+s1}{\PYGZsq{}}\PYG{p}{\PYGZcb{}}\PYG{p}{)}
\PYG{n}{r10\PYGZus{}2} \PYG{o}{=} \PYG{n}{pn}\PYG{o}{.}\PYG{n}{pane}\PYG{o}{.}\PYG{n}{PNG}\PYG{p}{(}\PYG{l+s+s2}{\PYGZdq{}}\PYG{l+s+s2}{images/T07\PYGZus{}fH6.png}\PYG{l+s+s2}{\PYGZdq{}}\PYG{p}{,} \PYG{n}{width}\PYG{o}{=}\PYG{l+m+mi}{400}\PYG{p}{)}  

\PYG{n}{pn}\PYG{o}{.}\PYG{n}{Row}\PYG{p}{(}\PYG{n}{r10\PYGZus{}1}\PYG{p}{,} \PYG{n}{r10\PYGZus{}2}\PYG{p}{)}
\end{sphinxVerbatim}

\end{sphinxuseclass}\end{sphinxVerbatimInput}
\begin{sphinxVerbatimOutput}

\begin{sphinxuseclass}{cell_output}
\begin{sphinxVerbatim}[commandchars=\\\{\}]
Row
    [0] Markdown(str, style=\PYGZob{}\PYGZsq{}font\PYGZhy{}size\PYGZsq{}: \PYGZsq{}13pt\PYGZsq{}\PYGZcb{}, width=500)
    [1] PNG(str, width=400)
\end{sphinxVerbatim}

\end{sphinxuseclass}\end{sphinxVerbatimOutput}

\end{sphinxuseclass}
\end{sphinxuseclass}

\subsection{Solution of Homework Problem 6}
\label{\detokenize{content/tutorials/T7/tutorial_07:solution-of-homework-problem-6}}
\begin{sphinxuseclass}{cell}\begin{sphinxVerbatimInput}

\begin{sphinxuseclass}{cell_input}
\begin{sphinxVerbatim}[commandchars=\\\{\}]
\PYG{c+c1}{\PYGZsh{} solution}

\PYG{n}{img\PYGZus{}1} \PYG{o}{=} \PYG{n}{pn}\PYG{o}{.}\PYG{n}{pane}\PYG{o}{.}\PYG{n}{PNG}\PYG{p}{(}\PYG{l+s+s2}{\PYGZdq{}}\PYG{l+s+s2}{images/T07\PYGZus{}fH6a.png}\PYG{l+s+s2}{\PYGZdq{}}\PYG{p}{,} \PYG{n}{width}\PYG{o}{=}\PYG{l+m+mi}{400}\PYG{p}{)}
\PYG{n}{img\PYGZus{}2} \PYG{o}{=} \PYG{n}{pn}\PYG{o}{.}\PYG{n}{pane}\PYG{o}{.}\PYG{n}{PNG}\PYG{p}{(}\PYG{l+s+s2}{\PYGZdq{}}\PYG{l+s+s2}{images/T07\PYGZus{}fH6b.png}\PYG{l+s+s2}{\PYGZdq{}}\PYG{p}{,} \PYG{n}{width}\PYG{o}{=}\PYG{l+m+mi}{400}\PYG{p}{)}
\PYG{n}{img\PYGZus{}3} \PYG{o}{=} \PYG{n}{pn}\PYG{o}{.}\PYG{n}{pane}\PYG{o}{.}\PYG{n}{PNG}\PYG{p}{(}\PYG{l+s+s2}{\PYGZdq{}}\PYG{l+s+s2}{images/T07\PYGZus{}fH6c.png}\PYG{l+s+s2}{\PYGZdq{}}\PYG{p}{,} \PYG{n}{width}\PYG{o}{=}\PYG{l+m+mi}{400}\PYG{p}{)}
\PYG{n}{img\PYGZus{}4} \PYG{o}{=} \PYG{n}{pn}\PYG{o}{.}\PYG{n}{pane}\PYG{o}{.}\PYG{n}{PNG}\PYG{p}{(}\PYG{l+s+s2}{\PYGZdq{}}\PYG{l+s+s2}{images/T07\PYGZus{}fH6d.png}\PYG{l+s+s2}{\PYGZdq{}}\PYG{p}{,} \PYG{n}{width}\PYG{o}{=}\PYG{l+m+mi}{400}\PYG{p}{)}
\PYG{n}{img\PYGZus{}5} \PYG{o}{=} \PYG{n}{pn}\PYG{o}{.}\PYG{n}{pane}\PYG{o}{.}\PYG{n}{PNG}\PYG{p}{(}\PYG{l+s+s2}{\PYGZdq{}}\PYG{l+s+s2}{images/T07\PYGZus{}fH6e.png}\PYG{l+s+s2}{\PYGZdq{}}\PYG{p}{,} \PYG{n}{width}\PYG{o}{=}\PYG{l+m+mi}{700}\PYG{p}{)}

\PYG{n}{tabs} \PYG{o}{=} \PYG{n}{pn}\PYG{o}{.}\PYG{n}{Tabs}\PYG{p}{(}\PYG{p}{(}\PYG{l+s+s1}{\PYGZsq{}}\PYG{l+s+s1}{Step 1}\PYG{l+s+s1}{\PYGZsq{}}\PYG{p}{,} \PYG{n}{img\PYGZus{}1}\PYG{p}{)}\PYG{p}{,} \PYG{p}{(}\PYG{l+s+s2}{\PYGZdq{}}\PYG{l+s+s2}{Step 2}\PYG{l+s+s2}{\PYGZdq{}}\PYG{p}{,} \PYG{n}{img\PYGZus{}2}\PYG{p}{)}\PYG{p}{,} \PYG{p}{(}\PYG{l+s+s2}{\PYGZdq{}}\PYG{l+s+s2}{Step 3}\PYG{l+s+s2}{\PYGZdq{}}\PYG{p}{,} \PYG{n}{img\PYGZus{}3}\PYG{p}{)}\PYG{p}{,} \PYG{p}{(}\PYG{l+s+s2}{\PYGZdq{}}\PYG{l+s+s2}{Step 4}\PYG{l+s+s2}{\PYGZdq{}}\PYG{p}{,} \PYG{n}{img\PYGZus{}4}\PYG{p}{)}\PYG{p}{,} \PYG{p}{(}\PYG{l+s+s2}{\PYGZdq{}}\PYG{l+s+s2}{Step 5}\PYG{l+s+s2}{\PYGZdq{}}\PYG{p}{,} \PYG{n}{img\PYGZus{}5}\PYG{p}{)} \PYG{p}{)}
\PYG{n}{tabs}
\end{sphinxVerbatim}

\end{sphinxuseclass}\end{sphinxVerbatimInput}
\begin{sphinxVerbatimOutput}

\begin{sphinxuseclass}{cell_output}
\begin{sphinxVerbatim}[commandchars=\\\{\}]
Tabs
    [0] PNG(str, width=400)
    [1] PNG(str, width=400)
    [2] PNG(str, width=400)
    [3] PNG(str, width=400)
    [4] PNG(str, width=700)
\end{sphinxVerbatim}

\end{sphinxuseclass}\end{sphinxVerbatimOutput}

\end{sphinxuseclass}

\section{Homework Problem 7: Flow Nets wells}
\label{\detokenize{content/tutorials/T7/tutorial_07:homework-problem-7-flow-nets-wells}}
\begin{sphinxuseclass}{cell}\begin{sphinxVerbatimInput}

\begin{sphinxuseclass}{cell_input}
\begin{sphinxVerbatim}[commandchars=\\\{\}]
\PYG{c+c1}{\PYGZsh{}}
\PYG{n}{r11\PYGZus{}1}\PYG{o}{=} \PYG{n}{pn}\PYG{o}{.}\PYG{n}{pane}\PYG{o}{.}\PYG{n}{Markdown}\PYG{p}{(}\PYG{l+s+s2}{\PYGZdq{}\PYGZdq{}\PYGZdq{}}
\PYG{l+s+s2}{\PYGZsh{}\PYGZsh{}\PYGZsh{}Homework Problem 7: Flow Nets}
\PYG{l+s+s2}{Sketch head isolines and streamlines for the well doublette shown below. }
\PYG{l+s+s2}{In this case, injection and withdrawal of groundwater is superimposed to a uniform flow component.}
\PYG{l+s+s2}{ \PYGZlt{}br\PYGZgt{}\PYGZlt{}br\PYGZgt{}\PYGZlt{}br\PYGZgt{}}
\PYG{l+s+s2}{ }\PYG{l+s+s2}{\PYGZdq{}\PYGZdq{}\PYGZdq{}}\PYG{p}{,} \PYG{n}{width} \PYG{o}{=} \PYG{l+m+mi}{900}\PYG{p}{,} \PYG{n}{style}\PYG{o}{=}\PYG{p}{\PYGZob{}}\PYG{l+s+s1}{\PYGZsq{}}\PYG{l+s+s1}{font\PYGZhy{}size}\PYG{l+s+s1}{\PYGZsq{}}\PYG{p}{:} \PYG{l+s+s1}{\PYGZsq{}}\PYG{l+s+s1}{12pt}\PYG{l+s+s1}{\PYGZsq{}}\PYG{p}{\PYGZcb{}}\PYG{p}{)}

\PYG{n}{r11\PYGZus{}2} \PYG{o}{=} \PYG{n}{pn}\PYG{o}{.}\PYG{n}{pane}\PYG{o}{.}\PYG{n}{PNG}\PYG{p}{(}\PYG{l+s+s2}{\PYGZdq{}}\PYG{l+s+s2}{images/T07\PYGZus{}fH8a.png}\PYG{l+s+s2}{\PYGZdq{}}\PYG{p}{,} \PYG{n}{width}\PYG{o}{=}\PYG{l+m+mi}{500}\PYG{p}{)}  

\PYG{n}{r11\PYGZus{}3}\PYG{o}{=} \PYG{n}{pn}\PYG{o}{.}\PYG{n}{pane}\PYG{o}{.}\PYG{n}{Markdown}\PYG{p}{(}\PYG{l+s+s2}{\PYGZdq{}\PYGZdq{}\PYGZdq{}}
\PYG{l+s+s2}{ \PYGZlt{}br\PYGZgt{}\PYGZlt{}br\PYGZgt{}}
\PYG{l+s+s2}{ }\PYG{l+s+s2}{\PYGZdq{}\PYGZdq{}\PYGZdq{}}\PYG{p}{,} \PYG{n}{width} \PYG{o}{=} \PYG{l+m+mi}{900}\PYG{p}{,} \PYG{n}{style}\PYG{o}{=}\PYG{p}{\PYGZob{}}\PYG{l+s+s1}{\PYGZsq{}}\PYG{l+s+s1}{font\PYGZhy{}size}\PYG{l+s+s1}{\PYGZsq{}}\PYG{p}{:} \PYG{l+s+s1}{\PYGZsq{}}\PYG{l+s+s1}{12pt}\PYG{l+s+s1}{\PYGZsq{}}\PYG{p}{\PYGZcb{}}\PYG{p}{)}
\PYG{n}{pn}\PYG{o}{.}\PYG{n}{Column}\PYG{p}{(}\PYG{n}{r11\PYGZus{}1}\PYG{p}{,} \PYG{n}{r11\PYGZus{}2}\PYG{p}{,} \PYG{n}{r11\PYGZus{}3}\PYG{p}{)}
\end{sphinxVerbatim}

\end{sphinxuseclass}\end{sphinxVerbatimInput}
\begin{sphinxVerbatimOutput}

\begin{sphinxuseclass}{cell_output}
\begin{sphinxVerbatim}[commandchars=\\\{\}]
Column
    [0] Markdown(str, style=\PYGZob{}\PYGZsq{}font\PYGZhy{}size\PYGZsq{}: \PYGZsq{}12pt\PYGZsq{}\PYGZcb{}, width=900)
    [1] PNG(str, width=500)
    [2] Markdown(str, style=\PYGZob{}\PYGZsq{}font\PYGZhy{}size\PYGZsq{}: \PYGZsq{}12pt\PYGZsq{}\PYGZcb{}, width=900)
\end{sphinxVerbatim}

\end{sphinxuseclass}\end{sphinxVerbatimOutput}

\end{sphinxuseclass}

\subsection{Solution of Homework Problem 7}
\label{\detokenize{content/tutorials/T7/tutorial_07:solution-of-homework-problem-7}}
\sphinxAtStartPar
The flow\sphinxhyphen{}net of the above scenario can be as shown in the figure below:

\noindent{\hspace*{\fill}\sphinxincludegraphics[width=600\sphinxpxdimen]{{T07_fH8}.png}\hspace*{\fill}}


\section{Tutorial Problems on Wells}
\label{\detokenize{content/tutorials/T7/tutorial_07:tutorial-problems-on-wells}}

\subsection{Tutorial Problem 18}
\label{\detokenize{content/tutorials/T7/tutorial_07:tutorial-problem-18}}
\begin{sphinxuseclass}{cell}
\begin{sphinxuseclass}{tag_hide-input}
\begin{sphinxuseclass}{tag_full-width}\begin{sphinxVerbatimOutput}

\begin{sphinxuseclass}{cell_output}
\begin{sphinxVerbatim}[commandchars=\\\{\}]
Row
    [0] Markdown(str, style=\PYGZob{}\PYGZsq{}font\PYGZhy{}size\PYGZsq{}: \PYGZsq{}12pt\PYGZsq{}\PYGZcb{}, width=600)
    [1] DataFrame(DataFrame)
\end{sphinxVerbatim}

\end{sphinxuseclass}\end{sphinxVerbatimOutput}

\end{sphinxuseclass}
\end{sphinxuseclass}
\end{sphinxuseclass}

\subsection{Solution of Problem 18}
\label{\detokenize{content/tutorials/T7/tutorial_07:solution-of-problem-18}}
\sphinxAtStartPar
\sphinxstylestrong{See L08 \sphinxhyphen{} slides 29\sphinxhyphen{}33 for more information on this problem}

\begin{sphinxuseclass}{cell}\begin{sphinxVerbatimInput}

\begin{sphinxuseclass}{cell_input}
\begin{sphinxVerbatim}[commandchars=\\\{\}]
\PYG{c+c1}{\PYGZsh{} Problem 18 solution contd.}
\PYG{n}{r3\PYGZus{}2} \PYG{o}{=} \PYG{n}{pn}\PYG{o}{.}\PYG{n}{pane}\PYG{o}{.}\PYG{n}{Markdown}\PYG{p}{(}\PYG{l+s+s2}{\PYGZdq{}\PYGZdq{}\PYGZdq{}}
\PYG{l+s+s2}{As a first approach, the graphical solution of the problem is to be determined, i.e. double\PYGZhy{}logarithmic data and type curve sheets are compared as follows:}
\PYG{l+s+s2}{\PYGZlt{}br\PYGZgt{} \PYGZlt{}br\PYGZgt{} }

\PYG{l+s+s2}{1. Plot data for \PYGZus{}s\PYGZus{} vs. \PYGZus{}t/r\PYGZlt{}sup\PYGZgt{}2\PYGZlt{}/sup\PYGZgt{}\PYGZus{} in the data sheet.\PYGZlt{}br\PYGZgt{}}

\PYG{l+s+s2}{2. Determine coordinates of the match point \PYGZus{}A\PYGZus{} on the type curve sheet, e.g. \PYGZus{}1/u\PYGZlt{}sub\PYGZgt{}A\PYGZlt{}/sub\PYGZgt{}\PYGZus{} = 1 and \PYGZus{}W\PYGZlt{}sub\PYGZgt{}A\PYGZlt{}/sub\PYGZgt{}\PYGZus{} = 1.\PYGZlt{}br\PYGZgt{}}

\PYG{l+s+s2}{3. Put the data sheet on top of the type curve sheet and shift it in parallel to the coordinate axes until data points }
\PYG{l+s+s2}{fall on the type curve as close as possible.\PYGZlt{}br\PYGZgt{}}
\PYG{l+s+s2}{4. Determine coordinates of the match point \PYGZus{}A\PYGZus{} on the data sheet: }

\PYG{l+s+s2}{**\PYGZus{}As our course is online this semester, we will follow the computational approach, see below:\PYGZus{}**}
\PYG{l+s+s2}{\PYGZdq{}\PYGZdq{}\PYGZdq{}}\PYG{p}{,}\PYG{n}{width} \PYG{o}{=} \PYG{l+m+mi}{600}\PYG{p}{,} \PYG{n}{style}\PYG{o}{=}\PYG{p}{\PYGZob{}}\PYG{l+s+s1}{\PYGZsq{}}\PYG{l+s+s1}{font\PYGZhy{}size}\PYG{l+s+s1}{\PYGZsq{}}\PYG{p}{:} \PYG{l+s+s1}{\PYGZsq{}}\PYG{l+s+s1}{12pt}\PYG{l+s+s1}{\PYGZsq{}}\PYG{p}{\PYGZcb{}}\PYG{p}{)}

\PYG{c+c1}{\PYGZsh{}given}
\PYG{n}{r} \PYG{o}{=} \PYG{l+m+mf}{9.85} \PYG{c+c1}{\PYGZsh{} m, observation well distance}
\PYG{n}{t\PYGZus{}s} \PYG{o}{=} \PYG{n}{t\PYGZus{}m}\PYG{o}{*}\PYG{l+m+mi}{60} \PYG{c+c1}{\PYGZsh{} s, converting time in s}
\PYG{n}{t\PYGZus{}r2} \PYG{o}{=} \PYG{n}{t\PYGZus{}s}\PYG{o}{/}\PYG{n}{r}\PYG{o}{*}\PYG{o}{*}\PYG{l+m+mi}{2}\PYG{c+c1}{\PYGZsh{} s/m\PYGZca{}2,  finding t/r\PYGZca{}2 }

\PYG{c+c1}{\PYGZsh{}output}
\PYG{n}{d2}\PYG{o}{=} \PYG{p}{\PYGZob{}}\PYG{l+s+s1}{\PYGZsq{}}\PYG{l+s+s1}{time (s)}\PYG{l+s+s1}{\PYGZsq{}}\PYG{p}{:} \PYG{n}{t\PYGZus{}s}\PYG{p}{,} \PYG{l+s+s1}{\PYGZsq{}}\PYG{l+s+s1}{drawdown (m)}\PYG{l+s+s1}{\PYGZsq{}}\PYG{p}{:} \PYG{n}{s\PYGZus{}m}\PYG{p}{,} \PYG{l+s+s2}{\PYGZdq{}}\PYG{l+s+s2}{t/r}\PYG{l+s+se}{\PYGZbs{}u00b2}\PYG{l+s+s2}{ (s/m}\PYG{l+s+se}{\PYGZbs{}u00b2}\PYG{l+s+s2}{)}\PYG{l+s+s2}{\PYGZdq{}}\PYG{p}{:} \PYG{n}{t\PYGZus{}r2}\PYG{p}{\PYGZcb{}}  
\PYG{n}{df2} \PYG{o}{=} \PYG{n}{pd}\PYG{o}{.}\PYG{n}{DataFrame}\PYG{p}{(}\PYG{n}{data}\PYG{o}{=}\PYG{n}{d2}\PYG{p}{)} 
\PYG{n}{df3} \PYG{o}{=} \PYG{n}{df2}\PYG{o}{.}\PYG{n}{set\PYGZus{}index}\PYG{p}{(}\PYG{l+s+s1}{\PYGZsq{}}\PYG{l+s+s1}{time (s)}\PYG{l+s+s1}{\PYGZsq{}}\PYG{p}{)}
\PYG{n}{pn}\PYG{o}{.}\PYG{n}{Row}\PYG{p}{(}\PYG{n}{r3\PYGZus{}2}\PYG{p}{,} \PYG{n}{df3}\PYG{o}{.}\PYG{n}{head}\PYG{p}{(}\PYG{l+m+mi}{10}\PYG{p}{)}\PYG{p}{)}  \PYG{c+c1}{\PYGZsh{} only top 10 data shown}
\end{sphinxVerbatim}

\end{sphinxuseclass}\end{sphinxVerbatimInput}
\begin{sphinxVerbatimOutput}

\begin{sphinxuseclass}{cell_output}
\begin{sphinxVerbatim}[commandchars=\\\{\}]
Row
    [0] Markdown(str, style=\PYGZob{}\PYGZsq{}font\PYGZhy{}size\PYGZsq{}: \PYGZsq{}12pt\PYGZsq{}\PYGZcb{}, width=600)
    [1] DataFrame(DataFrame)
\end{sphinxVerbatim}

\end{sphinxuseclass}\end{sphinxVerbatimOutput}

\end{sphinxuseclass}
\begin{sphinxuseclass}{cell}\begin{sphinxVerbatimInput}

\begin{sphinxuseclass}{cell_input}
\begin{sphinxVerbatim}[commandchars=\\\{\}]
\PYG{n+nb}{print}\PYG{p}{(}\PYG{l+s+s2}{\PYGZdq{}}\PYG{l+s+se}{\PYGZbs{}033}\PYG{l+s+s2}{[1m The given data:}\PYG{l+s+se}{\PYGZbs{}033}\PYG{l+s+s2}{[0m}\PYG{l+s+se}{\PYGZbs{}n}\PYG{l+s+s2}{\PYGZdq{}}\PYG{p}{)}

\PYG{n}{Q\PYGZus{}h} \PYG{o}{=} \PYG{l+m+mi}{9} \PYG{c+c1}{\PYGZsh{} m\PYGZca{}3/h \PYGZhy{} well discharge}
\PYG{n}{m} \PYG{o}{=} \PYG{l+m+mi}{5} \PYG{c+c1}{\PYGZsh{} m, aquifer thickness}
\PYG{n}{r} \PYG{o}{=} \PYG{l+m+mf}{9.85} \PYG{c+c1}{\PYGZsh{} m, observation\PYGZhy{} well location from pumping well}

\PYG{c+c1}{\PYGZsh{} interim calculation}
\PYG{n}{Q\PYGZus{}s} \PYG{o}{=} \PYG{n}{Q\PYGZus{}h}\PYG{o}{/}\PYG{l+m+mi}{3600} \PYG{c+c1}{\PYGZsh{} m\PYGZca{}3/s, unit\PYGZhy{} change of discharge}

\PYG{n+nb}{print}\PYG{p}{(}\PYG{l+s+s2}{\PYGZdq{}}\PYG{l+s+s2}{The avialable information in addition to data are:}\PYG{l+s+se}{\PYGZbs{}n}\PYG{l+s+s2}{\PYGZdq{}}\PYG{p}{)}
\PYG{n+nb}{print}\PYG{p}{(}\PYG{l+s+s2}{\PYGZdq{}}\PYG{l+s+s2}{The well discharge, Q =  }\PYG{l+s+si}{\PYGZob{}\PYGZcb{}}\PYG{l+s+s2}{ m}\PYG{l+s+se}{\PYGZbs{}u00b3}\PYG{l+s+s2}{/s }\PYG{l+s+se}{\PYGZbs{}n}\PYG{l+s+s2}{\PYGZdq{}}\PYG{o}{.}\PYG{n}{format}\PYG{p}{(}\PYG{n}{Q\PYGZus{}s}\PYG{p}{)}\PYG{p}{)}
\PYG{n+nb}{print}\PYG{p}{(}\PYG{l+s+s2}{\PYGZdq{}}\PYG{l+s+s2}{The aquifer thickness, m =  }\PYG{l+s+si}{\PYGZob{}\PYGZcb{}}\PYG{l+s+s2}{ m }\PYG{l+s+se}{\PYGZbs{}n}\PYG{l+s+s2}{\PYGZdq{}}\PYG{o}{.}\PYG{n}{format}\PYG{p}{(}\PYG{n}{m}\PYG{p}{)}\PYG{p}{)}
\PYG{n+nb}{print}\PYG{p}{(}\PYG{l+s+s2}{\PYGZdq{}}\PYG{l+s+s2}{The Observation well location, r =  }\PYG{l+s+si}{\PYGZob{}\PYGZcb{}}\PYG{l+s+s2}{ m }\PYG{l+s+se}{\PYGZbs{}n}\PYG{l+s+s2}{\PYGZdq{}}\PYG{o}{.}\PYG{n}{format}\PYG{p}{(}\PYG{n}{r}\PYG{p}{)}\PYG{p}{)}
\end{sphinxVerbatim}

\end{sphinxuseclass}\end{sphinxVerbatimInput}
\begin{sphinxVerbatimOutput}

\begin{sphinxuseclass}{cell_output}
\begin{sphinxVerbatim}[commandchars=\\\{\}]
\PYG{Color+ColorBold}{ The given data:}

The avialable information in addition to data are:

The well discharge, Q =  0.0025 m³/s 

The aquifer thickness, m =  5 m 

The Observation well location, r =  9.85 m 
\end{sphinxVerbatim}

\end{sphinxuseclass}\end{sphinxVerbatimOutput}

\end{sphinxuseclass}
\begin{sphinxuseclass}{cell}\begin{sphinxVerbatimInput}

\begin{sphinxuseclass}{cell_input}
\begin{sphinxVerbatim}[commandchars=\\\{\}]
\PYG{k+kn}{from} \PYG{n+nn}{ipywidgets} \PYG{k+kn}{import} \PYG{n}{interact}\PYG{p}{,} \PYG{n}{widgets} \PYG{c+c1}{\PYGZsh{} for interactive plot with slider}
\PYG{k+kn}{from} \PYG{n+nn}{scipy}\PYG{n+nn}{.}\PYG{n+nn}{special} \PYG{k+kn}{import} \PYG{n}{expi} \PYG{c+c1}{\PYGZsh{}}

\PYG{k}{def} \PYG{n+nf}{W}\PYG{p}{(}\PYG{n}{u}\PYG{p}{)}\PYG{p}{:}  
    \PYG{k}{return}  \PYG{o}{\PYGZhy{}}\PYG{n}{expi}\PYG{p}{(}\PYG{o}{\PYGZhy{}}\PYG{n}{u}\PYG{p}{)} \PYG{c+c1}{\PYGZsh{} provides the well function}

\PYG{k}{def} \PYG{n+nf}{well\PYGZus{}f}\PYG{p}{(}\PYG{n}{T}\PYG{p}{,} \PYG{n}{S\PYGZus{}C}\PYG{p}{,} \PYG{n}{r}\PYG{p}{,} \PYG{n}{Q}\PYG{p}{)}\PYG{p}{:} \PYG{c+c1}{\PYGZsh{} provides the fit curve for given r and Q}
    
        
    \PYG{c+c1}{\PYGZsh{} calculated function see L07\PYGZhy{}slide 31}
    \PYG{n}{u\PYGZus{}1d} \PYG{o}{=} \PYG{l+m+mi}{4}\PYG{o}{*}\PYG{n}{T}\PYG{o}{*}\PYG{n}{t\PYGZus{}s}\PYG{o}{/}\PYG{p}{(}\PYG{n}{S\PYGZus{}C}\PYG{o}{*}\PYG{n}{r}\PYG{o}{*}\PYG{o}{*}\PYG{l+m+mi}{2}\PYG{p}{)} \PYG{c+c1}{\PYGZsh{} calculating 1/u}
    \PYG{n}{w\PYGZus{}ud} \PYG{o}{=} \PYG{l+m+mi}{4}\PYG{o}{*}\PYG{n}{np}\PYG{o}{.}\PYG{n}{pi}\PYG{o}{*}\PYG{n}{s\PYGZus{}m}\PYG{o}{*}\PYG{n}{T}\PYG{o}{/}\PYG{n}{Q}   \PYG{c+c1}{\PYGZsh{} well function}

    \PYG{c+c1}{\PYGZsh{} plots}
    \PYG{n}{u\PYGZus{}1} \PYG{o}{=} \PYG{n}{np}\PYG{o}{.}\PYG{n}{logspace}\PYG{p}{(}\PYG{l+m+mi}{10}\PYG{p}{,}\PYG{o}{\PYGZhy{}}\PYG{l+m+mi}{1}\PYG{p}{,}\PYG{l+m+mi}{250}\PYG{p}{,} \PYG{n}{base}\PYG{o}{=}\PYG{l+m+mf}{10.0}\PYG{p}{)}
    \PYG{n}{w\PYGZus{}u} \PYG{o}{=}\PYG{n}{W}\PYG{p}{(}\PYG{l+m+mi}{1}\PYG{o}{/}\PYG{n}{u\PYGZus{}1}\PYG{p}{)} 
    
    \PYG{n}{plt}\PYG{o}{.}\PYG{n}{figure}\PYG{p}{(}\PYG{n}{figsize}\PYG{o}{=}\PYG{p}{(}\PYG{l+m+mi}{9}\PYG{p}{,}\PYG{l+m+mi}{6}\PYG{p}{)}\PYG{p}{)}\PYG{p}{;}
    \PYG{n}{plt}\PYG{o}{.}\PYG{n}{loglog}\PYG{p}{(}\PYG{n}{u\PYGZus{}1}\PYG{p}{,} \PYG{n}{w\PYGZus{}u}\PYG{p}{)}\PYG{p}{;} 
    \PYG{n}{plt}\PYG{o}{.}\PYG{n}{loglog}\PYG{p}{(}\PYG{n}{u\PYGZus{}1d}\PYG{p}{,} \PYG{n}{w\PYGZus{}ud}\PYG{p}{,} \PYG{l+s+s2}{\PYGZdq{}}\PYG{l+s+s2}{o}\PYG{l+s+s2}{\PYGZdq{}}\PYG{p}{,} \PYG{n}{color}\PYG{o}{=}\PYG{l+s+s2}{\PYGZdq{}}\PYG{l+s+s2}{red}\PYG{l+s+s2}{\PYGZdq{}}  \PYG{p}{)}
    \PYG{n}{plt}\PYG{o}{.}\PYG{n}{ylim}\PYG{p}{(}\PYG{p}{(}\PYG{l+m+mf}{0.1}\PYG{p}{,} \PYG{l+m+mi}{10}\PYG{p}{)}\PYG{p}{)}\PYG{p}{;}\PYG{n}{plt}\PYG{o}{.}\PYG{n}{xlim}\PYG{p}{(}\PYG{l+m+mi}{1}\PYG{p}{,} \PYG{l+m+mf}{1e5}\PYG{p}{)}
    \PYG{n}{plt}\PYG{o}{.}\PYG{n}{grid}\PYG{p}{(}\PYG{k+kc}{True}\PYG{p}{,} \PYG{n}{which}\PYG{o}{=}\PYG{l+s+s2}{\PYGZdq{}}\PYG{l+s+s2}{both}\PYG{l+s+s2}{\PYGZdq{}}\PYG{p}{,}\PYG{n}{ls}\PYG{o}{=}\PYG{l+s+s2}{\PYGZdq{}}\PYG{l+s+s2}{\PYGZhy{}}\PYG{l+s+s2}{\PYGZdq{}}\PYG{p}{)} 
    \PYG{n}{plt}\PYG{o}{.}\PYG{n}{ylabel}\PYG{p}{(}\PYG{l+s+sa}{r}\PYG{l+s+s2}{\PYGZdq{}}\PYG{l+s+s2}{W(u)}\PYG{l+s+s2}{\PYGZdq{}}\PYG{p}{)}\PYG{p}{;}\PYG{n}{plt}\PYG{o}{.}\PYG{n}{xlabel}\PYG{p}{(}\PYG{l+s+sa}{r}\PYG{l+s+s2}{\PYGZdq{}}\PYG{l+s+s2}{1/u}\PYG{l+s+s2}{\PYGZdq{}}\PYG{p}{)}
\end{sphinxVerbatim}

\end{sphinxuseclass}\end{sphinxVerbatimInput}

\end{sphinxuseclass}
\begin{sphinxuseclass}{cell}\begin{sphinxVerbatimInput}

\begin{sphinxuseclass}{cell_input}
\begin{sphinxVerbatim}[commandchars=\\\{\}]
\PYG{c+c1}{\PYGZsh{} select the starting value for T and S\PYGZus{}c\PYGZhy{} this has to be iteratively done.}

\PYG{n}{T} \PYG{o}{=} \PYG{l+m+mf}{0.00322} \PYG{c+c1}{\PYGZsh{} m\PYGZca{}2/S \PYGZhy{} check value for Transmissivity}
\PYG{n}{S\PYGZus{}c} \PYG{o}{=} \PYG{l+m+mf}{7.97e\PYGZhy{}03} \PYG{c+c1}{\PYGZsh{} (\PYGZhy{}), the storage coefficient}
\PYG{n}{Q} \PYG{o}{=} \PYG{n}{Q\PYGZus{}s}

\PYG{n}{well\PYGZus{}f}\PYG{p}{(}\PYG{n}{T}\PYG{p}{,} \PYG{n}{S\PYGZus{}c}\PYG{p}{,} \PYG{n}{r}\PYG{p}{,} \PYG{n}{Q}\PYG{p}{)}
\end{sphinxVerbatim}

\end{sphinxuseclass}\end{sphinxVerbatimInput}
\begin{sphinxVerbatimOutput}

\begin{sphinxuseclass}{cell_output}
\noindent\sphinxincludegraphics{{C:/Users/vibhu/GWtextbook/_build/jupyter_execute/tutorial_07_20_0}.png}

\end{sphinxuseclass}\end{sphinxVerbatimOutput}

\end{sphinxuseclass}
\begin{sphinxuseclass}{cell}\begin{sphinxVerbatimInput}

\begin{sphinxuseclass}{cell_input}
\begin{sphinxVerbatim}[commandchars=\\\{\}]
\PYG{c+c1}{\PYGZsh{} Compute condutivity based on the fit}

\PYG{n}{K} \PYG{o}{=} \PYG{n}{T}\PYG{o}{/}\PYG{n}{m} \PYG{c+c1}{\PYGZsh{} m/s, conductivity of the aquifer.}

\PYG{n+nb}{print}\PYG{p}{(}\PYG{l+s+s2}{\PYGZdq{}}\PYG{l+s+s2}{The conductivity of the aquifer, K = }\PYG{l+s+si}{\PYGZob{}0:0.2e\PYGZcb{}}\PYG{l+s+s2}{ m/s}\PYG{l+s+s2}{\PYGZdq{}}\PYG{o}{.}\PYG{n}{format}\PYG{p}{(}\PYG{n}{K}\PYG{p}{)}\PYG{p}{)}
\end{sphinxVerbatim}

\end{sphinxuseclass}\end{sphinxVerbatimInput}
\begin{sphinxVerbatimOutput}

\begin{sphinxuseclass}{cell_output}
\begin{sphinxVerbatim}[commandchars=\\\{\}]
The conductivity of the aquifer, K = 6.44e\PYGZhy{}04 m/s
\end{sphinxVerbatim}

\end{sphinxuseclass}\end{sphinxVerbatimOutput}

\end{sphinxuseclass}

\subsection{Tutorial Problem 19}
\label{\detokenize{content/tutorials/T7/tutorial_07:tutorial-problem-19}}
\sphinxAtStartPar
A pumping test is conducted in a confined aquifer with thickness \sphinxstyleemphasis{m} = 14.65 m.
The pumping rate is kept constant at \sphinxstyleemphasis{Q} = 50 m3/h and the corresponding drawdown
\sphinxstyleemphasis{S} is recorded in an observation well at a distance \sphinxstyleemphasis{r} = 251.32 m from the pumping well (see table).
\begin{enumerate}
\sphinxsetlistlabels{\arabic}{enumi}{enumii}{}{.}%
\item {} 
\sphinxAtStartPar
Determine storage coefficient \sphinxstyleemphasis{S},  transmissivity  \sphinxstyleemphasis{T} and hydraulic conductivity \sphinxstyleemphasis{K} by employing
the Theis method (graphical solution).

\item {} 
\sphinxAtStartPar
What is the drawdown in the pumping well (radius including gravel pack: \sphinxstyleemphasis{rw} = 0.3 m) after 500 min?                               (Hint: Use the approximation          W(u) ≈ –0.5772 – lnu + u which is valid for u << 1)


\item {} 
\sphinxAtStartPar
How big is the radius of influence according to Siechardt’s equation?


\end{enumerate}


\subsection{Solution of Problem 19}
\label{\detokenize{content/tutorials/T7/tutorial_07:solution-of-problem-19}}
\sphinxAtStartPar
\sphinxstylestrong{See L07 \sphinxhyphen{} slides 29\sphinxhyphen{}33 for more information on this problem}

\begin{sphinxuseclass}{cell}\begin{sphinxVerbatimInput}

\begin{sphinxuseclass}{cell_input}
\begin{sphinxVerbatim}[commandchars=\\\{\}]
\PYG{n}{data19} \PYG{o}{=} \PYG{n}{pd}\PYG{o}{.}\PYG{n}{read\PYGZus{}csv}\PYG{p}{(}\PYG{l+s+s2}{\PYGZdq{}}\PYG{l+s+s2}{T07\PYGZus{}TP19\PYGZus{}data.csv}\PYG{l+s+s2}{\PYGZdq{}}\PYG{p}{,} \PYG{n}{sep} \PYG{o}{=} \PYG{l+s+s2}{\PYGZdq{}}\PYG{l+s+s2}{,}\PYG{l+s+s2}{\PYGZdq{}}\PYG{p}{,} \PYG{n}{usecols} \PYG{o}{=}\PYG{p}{[}\PYG{l+s+s2}{\PYGZdq{}}\PYG{l+s+s2}{Time (min)}\PYG{l+s+s2}{\PYGZdq{}}\PYG{p}{,} \PYG{l+s+s2}{\PYGZdq{}}\PYG{l+s+s2}{Drawdown (m)}\PYG{l+s+s2}{\PYGZdq{}}\PYG{p}{]}\PYG{p}{)}

\PYG{n}{df\PYGZus{}t1}\PYG{o}{=} \PYG{n}{data19}\PYG{o}{.}\PYG{n}{values}\PYG{p}{[}\PYG{p}{:}\PYG{p}{,}\PYG{l+m+mi}{0}\PYG{p}{]}
\PYG{n}{df\PYGZus{}s1}\PYG{o}{=} \PYG{n}{data19}\PYG{o}{.}\PYG{n}{values}\PYG{p}{[}\PYG{p}{:}\PYG{p}{,}\PYG{l+m+mi}{1}\PYG{p}{]}

\PYG{n}{d} \PYG{o}{=} \PYG{p}{\PYGZob{}}\PYG{l+s+s1}{\PYGZsq{}}\PYG{l+s+s1}{time (min)}\PYG{l+s+s1}{\PYGZsq{}}\PYG{p}{:} \PYG{n}{df\PYGZus{}t1}\PYG{p}{,} \PYG{l+s+s1}{\PYGZsq{}}\PYG{l+s+s1}{drawdown (m)}\PYG{l+s+s1}{\PYGZsq{}}\PYG{p}{:} \PYG{n}{df\PYGZus{}s1}\PYG{p}{\PYGZcb{}}
\PYG{n}{df} \PYG{o}{=} \PYG{n}{pd}\PYG{o}{.}\PYG{n}{DataFrame}\PYG{p}{(}\PYG{n}{data}\PYG{o}{=}\PYG{n}{d}\PYG{p}{,} \PYG{n}{index}\PYG{o}{=}\PYG{k+kc}{None}\PYG{p}{)} 
\PYG{n}{df}\PYG{o}{.}\PYG{n}{head}\PYG{p}{(}\PYG{l+m+mi}{5}\PYG{p}{)}
\end{sphinxVerbatim}

\end{sphinxuseclass}\end{sphinxVerbatimInput}
\begin{sphinxVerbatimOutput}

\begin{sphinxuseclass}{cell_output}
\begin{sphinxVerbatim}[commandchars=\\\{\}]
   time (min)  drawdown (m)
0         3.0        0.0915
1         5.0        0.2135
2         8.0        0.3965
3        12.0        0.6405
4        20.0        0.9760
\end{sphinxVerbatim}

\end{sphinxuseclass}\end{sphinxVerbatimOutput}

\end{sphinxuseclass}
\begin{sphinxuseclass}{cell}\begin{sphinxVerbatimInput}

\begin{sphinxuseclass}{cell_input}
\begin{sphinxVerbatim}[commandchars=\\\{\}]
\PYG{c+c1}{\PYGZsh{}given}

\PYG{c+c1}{\PYGZsh{} make sure the time data is named \PYGZdq{}t\PYGZus{}s\PYGZdq{} and drawdown is named s\PYGZus{}m}
\PYG{n}{t\PYGZus{}s} \PYG{o}{=} \PYG{n}{df\PYGZus{}t1}\PYG{o}{*}\PYG{l+m+mi}{60} \PYG{c+c1}{\PYGZsh{} s, converting time in sec}
\PYG{n}{s\PYGZus{}m} \PYG{o}{=} \PYG{n}{df\PYGZus{}s1} \PYG{c+c1}{\PYGZsh{} m, drawdown data}

\PYG{n}{m\PYGZus{}19} \PYG{o}{=} \PYG{l+m+mf}{14.65} \PYG{c+c1}{\PYGZsh{} m, aquifer thickness}
\PYG{n}{Q\PYGZus{}19} \PYG{o}{=} \PYG{l+m+mi}{50} \PYG{c+c1}{\PYGZsh{} m\PYGZca{}3/h, pumping rate }
\PYG{n}{r\PYGZus{}19} \PYG{o}{=} \PYG{l+m+mf}{251.32} \PYG{c+c1}{\PYGZsh{} m, distance of observation well}

\PYG{c+c1}{\PYGZsh{}interim calculation}
\PYG{n}{t\PYGZus{}19r2} \PYG{o}{=} \PYG{n}{t\PYGZus{}s}\PYG{o}{/}\PYG{n}{r\PYGZus{}19}\PYG{o}{*}\PYG{o}{*}\PYG{l+m+mi}{2} \PYG{c+c1}{\PYGZsh{} s/m\PYGZca{}2,  finding t/r\PYGZca{}2 }
\PYG{n}{Q\PYGZus{}19s} \PYG{o}{=} \PYG{n}{Q\PYGZus{}19}\PYG{o}{/}\PYG{l+m+mi}{3600} \PYG{c+c1}{\PYGZsh{} m\PYGZca{}3/s, pumping rate in s\PYGZhy{} unit change}

\PYG{c+c1}{\PYGZsh{}output}
\PYG{n}{d2}\PYG{o}{=} \PYG{p}{\PYGZob{}}\PYG{l+s+s1}{\PYGZsq{}}\PYG{l+s+s1}{time [min]}\PYG{l+s+s1}{\PYGZsq{}}\PYG{p}{:} \PYG{n}{df\PYGZus{}t1}\PYG{p}{,} \PYG{l+s+s1}{\PYGZsq{}}\PYG{l+s+s1}{drawdown [m]}\PYG{l+s+s1}{\PYGZsq{}}\PYG{p}{:} \PYG{n}{df\PYGZus{}s1}\PYG{p}{,} \PYG{l+s+s2}{\PYGZdq{}}\PYG{l+s+s2}{t/r}\PYG{l+s+se}{\PYGZbs{}u00b2}\PYG{l+s+s2}{ (s/m}\PYG{l+s+se}{\PYGZbs{}u00b2}\PYG{l+s+s2}{)}\PYG{l+s+s2}{\PYGZdq{}}\PYG{p}{:} \PYG{n}{t\PYGZus{}19r2}\PYG{p}{\PYGZcb{}}  
\PYG{n}{df2} \PYG{o}{=} \PYG{n}{pd}\PYG{o}{.}\PYG{n}{DataFrame}\PYG{p}{(}\PYG{n}{data}\PYG{o}{=}\PYG{n}{d2}\PYG{p}{,} \PYG{n}{index}\PYG{o}{=}\PYG{k+kc}{None}\PYG{p}{)} 
\PYG{n}{df2}\PYG{o}{.}\PYG{n}{head}\PYG{p}{(}\PYG{l+m+mi}{5}\PYG{p}{)}
\end{sphinxVerbatim}

\end{sphinxuseclass}\end{sphinxVerbatimInput}
\begin{sphinxVerbatimOutput}

\begin{sphinxuseclass}{cell_output}
\begin{sphinxVerbatim}[commandchars=\\\{\}]
   time [min]  drawdown [m]  t/r² (s/m²)
0         3.0        0.0915     0.002850
1         5.0        0.2135     0.004750
2         8.0        0.3965     0.007600
3        12.0        0.6405     0.011399
4        20.0        0.9760     0.018999
\end{sphinxVerbatim}

\end{sphinxuseclass}\end{sphinxVerbatimOutput}

\end{sphinxuseclass}
\begin{sphinxuseclass}{cell}\begin{sphinxVerbatimInput}

\begin{sphinxuseclass}{cell_input}
\begin{sphinxVerbatim}[commandchars=\\\{\}]
\PYG{c+c1}{\PYGZsh{} Fitting the typ curve}

\PYG{c+c1}{\PYGZsh{} Trial data \PYGZhy{} chage T and S\PYGZus{}c data to fit the curve}

\PYG{n}{T\PYGZus{}19} \PYG{o}{=} \PYG{l+m+mf}{0.00138} \PYG{c+c1}{\PYGZsh{} m\PYGZca{}2/s, transmissivity}
\PYG{n}{S\PYGZus{}19}\PYG{o}{=} \PYG{l+m+mf}{2.21e\PYGZhy{}05} \PYG{c+c1}{\PYGZsh{} (\PYGZhy{}), storage coefficient}

\PYG{c+c1}{\PYGZsh{} fit \PYGZhy{}  make sure to write the variable name correctly}

\PYG{n}{well\PYGZus{}f}\PYG{p}{(}\PYG{n}{T\PYGZus{}19}\PYG{p}{,}\PYG{n}{S\PYGZus{}19}\PYG{p}{,}\PYG{n}{r\PYGZus{}19}\PYG{p}{,} \PYG{n}{Q\PYGZus{}19s}\PYG{p}{)} 
\end{sphinxVerbatim}

\end{sphinxuseclass}\end{sphinxVerbatimInput}
\begin{sphinxVerbatimOutput}

\begin{sphinxuseclass}{cell_output}
\noindent\sphinxincludegraphics{{C:/Users/vibhu/GWtextbook/_build/jupyter_execute/tutorial_07_25_0}.png}

\end{sphinxuseclass}\end{sphinxVerbatimOutput}

\end{sphinxuseclass}
\begin{sphinxuseclass}{cell}\begin{sphinxVerbatimInput}

\begin{sphinxuseclass}{cell_input}
\begin{sphinxVerbatim}[commandchars=\\\{\}]
\PYG{c+c1}{\PYGZsh{} Solutions and output}

\PYG{n}{K\PYGZus{}19} \PYG{o}{=} \PYG{n}{T\PYGZus{}19}\PYG{o}{/}\PYG{n}{m\PYGZus{}19}

\PYG{c+c1}{\PYGZsh{}output}
\PYG{n+nb}{print}\PYG{p}{(}\PYG{l+s+s2}{\PYGZdq{}}\PYG{l+s+se}{\PYGZbs{}033}\PYG{l+s+s2}{[1m The results are:}\PYG{l+s+se}{\PYGZbs{}033}\PYG{l+s+s2}{[0m}\PYG{l+s+se}{\PYGZbs{}n}\PYG{l+s+s2}{\PYGZdq{}}\PYG{p}{)}
\PYG{n+nb}{print}\PYG{p}{(}\PYG{l+s+s2}{\PYGZdq{}}\PYG{l+s+s2}{The Transmissivity at the site is: }\PYG{l+s+si}{\PYGZob{}0:1.2e\PYGZcb{}}\PYG{l+s+s2}{\PYGZdq{}}\PYG{o}{.}\PYG{n}{format}\PYG{p}{(}\PYG{n}{T\PYGZus{}19}\PYG{p}{)}\PYG{p}{,} \PYG{l+s+s2}{\PYGZdq{}}\PYG{l+s+s2}{m}\PYG{l+s+se}{\PYGZbs{}u00b2}\PYG{l+s+s2}{/s}\PYG{l+s+se}{\PYGZbs{}n}\PYG{l+s+s2}{\PYGZdq{}}\PYG{p}{)}
\PYG{n+nb}{print}\PYG{p}{(}\PYG{l+s+s2}{\PYGZdq{}}\PYG{l+s+s2}{The Storage coefficient at the site is: }\PYG{l+s+si}{\PYGZob{}0:1.3e\PYGZcb{}}\PYG{l+s+s2}{\PYGZdq{}}\PYG{o}{.}\PYG{n}{format}\PYG{p}{(}\PYG{n}{S\PYGZus{}19}\PYG{p}{)}\PYG{p}{,} \PYG{l+s+s2}{\PYGZdq{}}\PYG{l+s+se}{\PYGZbs{}n}\PYG{l+s+s2}{\PYGZdq{}}\PYG{p}{)}  
\PYG{n+nb}{print}\PYG{p}{(}\PYG{l+s+s2}{\PYGZdq{}}\PYG{l+s+s2}{The Conductivity at the site is: }\PYG{l+s+si}{\PYGZob{}0:1.1e\PYGZcb{}}\PYG{l+s+s2}{\PYGZdq{}}\PYG{o}{.}\PYG{n}{format}\PYG{p}{(}\PYG{n}{K\PYGZus{}19}\PYG{p}{)}\PYG{p}{,} \PYG{l+s+s2}{\PYGZdq{}}\PYG{l+s+s2}{m/s}\PYG{l+s+s2}{\PYGZdq{}}\PYG{p}{)} 
\end{sphinxVerbatim}

\end{sphinxuseclass}\end{sphinxVerbatimInput}
\begin{sphinxVerbatimOutput}

\begin{sphinxuseclass}{cell_output}
\begin{sphinxVerbatim}[commandchars=\\\{\}]
\PYG{Color+ColorBold}{ The results are:}

The Transmissivity at the site is: 1.38e\PYGZhy{}03 m²/s

The Storage coefficient at the site is: 2.210e\PYGZhy{}05 

The Conductivity at the site is: 9.4e\PYGZhy{}05 m/s
\end{sphinxVerbatim}

\end{sphinxuseclass}\end{sphinxVerbatimOutput}

\end{sphinxuseclass}
\begin{sphinxuseclass}{cell}\begin{sphinxVerbatimInput}

\begin{sphinxuseclass}{cell_input}
\begin{sphinxVerbatim}[commandchars=\\\{\}]
\PYG{c+c1}{\PYGZsh{} solution 19(b) and 19(c)}

\PYG{c+c1}{\PYGZsh{} Given }
\PYG{n}{r\PYGZus{}19w} \PYG{o}{=} \PYG{l+m+mf}{0.3} \PYG{c+c1}{\PYGZsh{} m, radius of the well}
\PYG{n}{t\PYGZus{}19m} \PYG{o}{=} \PYG{l+m+mi}{500} \PYG{c+c1}{\PYGZsh{} min, given time in min, }

\PYG{c+c1}{\PYGZsh{}interm calculation}
\PYG{n}{t\PYGZus{}19s} \PYG{o}{=} \PYG{n}{t\PYGZus{}19m}\PYG{o}{*}\PYG{l+m+mi}{60} \PYG{c+c1}{\PYGZsh{} s, tive converted to second}

\PYG{c+c1}{\PYGZsh{}Calculations}
\PYG{n}{u\PYGZus{}19} \PYG{o}{=} \PYG{p}{(}\PYG{n}{S\PYGZus{}19}\PYG{o}{*}\PYG{n}{r\PYGZus{}19w}\PYG{o}{*}\PYG{o}{*}\PYG{l+m+mi}{2}\PYG{p}{)}\PYG{o}{/}\PYG{p}{(}\PYG{l+m+mi}{4}\PYG{o}{*}\PYG{n}{T\PYGZus{}19}\PYG{o}{*}\PYG{n}{t\PYGZus{}19s}\PYG{p}{)}
\PYG{n}{W\PYGZus{}19b} \PYG{o}{=} \PYG{o}{\PYGZhy{}}\PYG{l+m+mf}{0.5772} \PYG{o}{\PYGZhy{}} \PYG{n}{np}\PYG{o}{.}\PYG{n}{log}\PYG{p}{(}\PYG{n}{u\PYGZus{}19}\PYG{p}{)}\PYG{o}{+}\PYG{n}{u\PYGZus{}19} \PYG{c+c1}{\PYGZsh{} using the given approximate of W(u)}
\PYG{n}{s\PYGZus{}19b} \PYG{o}{=} \PYG{p}{(}\PYG{n}{Q\PYGZus{}19s}\PYG{o}{*}\PYG{n}{W\PYGZus{}19b}\PYG{p}{)}\PYG{o}{/}\PYG{p}{(}\PYG{l+m+mi}{4}\PYG{o}{*}\PYG{n}{np}\PYG{o}{.}\PYG{n}{pi}\PYG{o}{*}\PYG{n}{T\PYGZus{}19}\PYG{p}{)} \PYG{c+c1}{\PYGZsh{} see L07 \PYGZhy{} slide 32}

\PYG{c+c1}{\PYGZsh{} Solution of 19C: }
\PYG{c+c1}{\PYGZsh{}How big is the radius of influence according to Siechardt‘s equation? }
\PYG{c+c1}{\PYGZsh{} (L07, slide 27)}

\PYG{n}{R\PYGZus{}19} \PYG{o}{=} \PYG{l+m+mi}{3000}\PYG{o}{*}\PYG{n}{s\PYGZus{}19b}\PYG{o}{*}\PYG{n}{np}\PYG{o}{.}\PYG{n}{sqrt}\PYG{p}{(}\PYG{n}{K\PYGZus{}19}\PYG{p}{)} 

\PYG{c+c1}{\PYGZsh{}output}
\PYG{n+nb}{print}\PYG{p}{(}\PYG{l+s+s2}{\PYGZdq{}}\PYG{l+s+se}{\PYGZbs{}033}\PYG{l+s+s2}{[1m The results are:}\PYG{l+s+se}{\PYGZbs{}033}\PYG{l+s+s2}{[0m}\PYG{l+s+se}{\PYGZbs{}n}\PYG{l+s+s2}{\PYGZdq{}}\PYG{p}{)}
\PYG{n+nb}{print}\PYG{p}{(}\PYG{l+s+s2}{\PYGZdq{}}\PYG{l+s+s2}{u = }\PYG{l+s+si}{\PYGZob{}0:1.2e\PYGZcb{}}\PYG{l+s+s2}{\PYGZdq{}}\PYG{o}{.}\PYG{n}{format}\PYG{p}{(}\PYG{n}{u\PYGZus{}19}\PYG{p}{)}\PYG{p}{,} \PYG{l+s+s2}{\PYGZdq{}}\PYG{l+s+se}{\PYGZbs{}n}\PYG{l+s+s2}{\PYGZdq{}}\PYG{p}{)}
\PYG{n+nb}{print}\PYG{p}{(}\PYG{l+s+s2}{\PYGZdq{}}\PYG{l+s+s2}{W(u)= }\PYG{l+s+si}{\PYGZob{}0:1.2f\PYGZcb{}}\PYG{l+s+s2}{\PYGZdq{}}\PYG{o}{.}\PYG{n}{format}\PYG{p}{(}\PYG{n}{W\PYGZus{}19b}\PYG{p}{)}\PYG{p}{,} \PYG{l+s+s2}{\PYGZdq{}}\PYG{l+s+se}{\PYGZbs{}n}\PYG{l+s+s2}{\PYGZdq{}}\PYG{p}{)}  
\PYG{n+nb}{print}\PYG{p}{(}\PYG{l+s+s2}{\PYGZdq{}}\PYG{l+s+s2}{The drawdonw at the site is: }\PYG{l+s+si}{\PYGZob{}0:1.2f\PYGZcb{}}\PYG{l+s+s2}{\PYGZdq{}}\PYG{o}{.}\PYG{n}{format}\PYG{p}{(}\PYG{n}{s\PYGZus{}19b}\PYG{p}{)}\PYG{p}{,} \PYG{l+s+s2}{\PYGZdq{}}\PYG{l+s+s2}{m }\PYG{l+s+se}{\PYGZbs{}n}\PYG{l+s+s2}{\PYGZdq{}}\PYG{p}{)}
\PYG{n+nb}{print}\PYG{p}{(}\PYG{l+s+s2}{\PYGZdq{}}\PYG{l+s+s2}{The radius of influence is is: }\PYG{l+s+si}{\PYGZob{}0:1.2f\PYGZcb{}}\PYG{l+s+s2}{\PYGZdq{}}\PYG{o}{.}\PYG{n}{format}\PYG{p}{(}\PYG{n}{R\PYGZus{}19}\PYG{p}{)}\PYG{p}{,} \PYG{l+s+s2}{\PYGZdq{}}\PYG{l+s+s2}{m}\PYG{l+s+s2}{\PYGZdq{}}\PYG{p}{)}
\end{sphinxVerbatim}

\end{sphinxuseclass}\end{sphinxVerbatimInput}
\begin{sphinxVerbatimOutput}

\begin{sphinxuseclass}{cell_output}
\begin{sphinxVerbatim}[commandchars=\\\{\}]
\PYG{Color+ColorBold}{ The results are:}

u = 1.20e\PYGZhy{}08 

W(u)= 17.66 

The drawdonw at the site is: 14.14 m 

The radius of influence is is: 411.83 m
\end{sphinxVerbatim}

\end{sphinxuseclass}\end{sphinxVerbatimOutput}

\end{sphinxuseclass}

\section{HOME WORK PROBLEMS}
\label{\detokenize{content/tutorials/T7/tutorial_07:home-work-problems}}
\sphinxAtStartPar
\sphinxstylestrong{Effective Conductivity and Wells}

\sphinxAtStartPar


\sphinxAtStartPar
There is no obligation to submit the homework



\sphinxAtStartPar
\sphinxstylestrong{Pls. submit within two weeks if you wish to.}


\subsection{Homework Problem 10}
\label{\detokenize{content/tutorials/T7/tutorial_07:homework-problem-10}}
\sphinxAtStartPar
A pumping test is conducted to determine hydraulic properties (storage coefficient \(S\), the transmissivity \(T\) and the hydraulic conductivity \(K\)) of
the aquifer. of a confined aquifer. For this purpose, a constant
pumping rate of 1219 m3/d is established and drawdown is recorded in an observation well. This problem is to be
solved with the Theis method implemented in the code below.

\sphinxAtStartPar
The code generates the typ curve based on your date of birth (ddmmyyyy). To use the code, you will provide different value of \(T\) and \(S\) and make a match of the data with the typ\sphinxhyphen{}curve.

\sphinxAtStartPar
Code (2 cells below)

\begin{sphinxuseclass}{cell}\begin{sphinxVerbatimInput}

\begin{sphinxuseclass}{cell_input}
\begin{sphinxVerbatim}[commandchars=\\\{\}]
\PYG{c+c1}{\PYGZsh{} Functions to generate well\PYGZhy{}function (this is another method based on scipy library)}

\PYG{k+kn}{from} \PYG{n+nn}{scipy}\PYG{n+nn}{.}\PYG{n+nn}{special} \PYG{k+kn}{import} \PYG{n}{expi}
\PYG{k}{def} \PYG{n+nf}{W}\PYG{p}{(}\PYG{n}{u}\PYG{p}{)}\PYG{p}{:} 
    \PYG{k}{return} \PYG{o}{\PYGZhy{}}\PYG{n}{expi}\PYG{p}{(}\PYG{o}{\PYGZhy{}}\PYG{n}{u}\PYG{p}{)}

\PYG{c+c1}{\PYGZsh{}Generate your data and function required to solve}

\PYG{k}{def} \PYG{n+nf}{data}\PYG{p}{(}\PYG{n}{Q}\PYG{p}{,} \PYG{n}{DOB}\PYG{p}{,} \PYG{n}{S}\PYG{p}{,} \PYG{n}{T}\PYG{p}{)}\PYG{p}{:}

    \PYG{l+s+sd}{\PYGZsq{}\PYGZsq{}\PYGZsq{}}
\PYG{l+s+sd}{    Q = pumping rate in m\PYGZca{}3/s, }
\PYG{l+s+sd}{    DOB\PYGZhy{} date of birth (ddmmyyyy), }
\PYG{l+s+sd}{    S = Storage Coeff. and }
\PYG{l+s+sd}{    T = Transmissivity (m\PYGZca{}2/s)}
\PYG{l+s+sd}{    \PYGZsq{}\PYGZsq{}\PYGZsq{}}
    \PYG{n}{S\PYGZus{}dob} \PYG{o}{=} \PYG{n+nb}{sum}\PYG{p}{(}\PYG{n+nb}{int}\PYG{p}{(}\PYG{n}{DOB}\PYG{p}{)} \PYG{k}{for} \PYG{n}{DOB} \PYG{o+ow}{in} \PYG{n+nb}{str}\PYG{p}{(}\PYG{n}{DOB}\PYG{p}{)}\PYG{p}{)} \PYG{c+c1}{\PYGZsh{} add numbers in your DOB}
    \PYG{n}{d\PYGZus{}t} \PYG{o}{=} \PYG{n}{np}\PYG{o}{.}\PYG{n}{array}\PYG{p}{(}\PYG{p}{[}\PYG{l+m+mf}{3.5}\PYG{p}{,} \PYG{l+m+mi}{5}\PYG{p}{,} \PYG{l+m+mf}{6.2}\PYG{p}{,} \PYG{l+m+mi}{8}\PYG{p}{,} \PYG{l+m+mf}{9.2}\PYG{p}{,} \PYG{l+m+mf}{12.4}\PYG{p}{,} \PYG{l+m+mf}{16.5}\PYG{p}{,} \PYG{l+m+mi}{20}\PYG{p}{,} \PYG{l+m+mi}{30}\PYG{p}{,} \PYG{l+m+mi}{60}\PYG{p}{,} \PYG{l+m+mi}{100}\PYG{p}{,} \PYG{l+m+mi}{200}\PYG{p}{,} \PYG{l+m+mi}{320}\PYG{p}{,} \PYG{l+m+mi}{380}\PYG{p}{,} \PYG{l+m+mi}{500}\PYG{p}{]}\PYG{p}{)}
    \PYG{n}{d\PYGZus{}d} \PYG{o}{=} \PYG{n}{np}\PYG{o}{.}\PYG{n}{array}\PYG{p}{(}\PYG{p}{[}\PYG{l+m+mf}{0.12}\PYG{p}{,} \PYG{l+m+mf}{0.23}\PYG{p}{,} \PYG{l+m+mf}{0.31}\PYG{p}{,} \PYG{l+m+mf}{0.41}\PYG{p}{,} \PYG{l+m+mf}{0.47}\PYG{p}{,} \PYG{l+m+mf}{0.64}\PYG{p}{,} \PYG{l+m+mf}{0.82}\PYG{p}{,} \PYG{l+m+mf}{0.92}\PYG{p}{,} \PYG{l+m+mf}{1.2}\PYG{p}{,} \PYG{l+m+mf}{1.74}\PYG{p}{,} \PYG{l+m+mf}{2.14}\PYG{p}{,} \PYG{l+m+mf}{2.57}\PYG{p}{,} \PYG{l+m+mi}{3}\PYG{p}{,} \PYG{l+m+mf}{3.1}\PYG{p}{,} \PYG{l+m+mf}{3.34}\PYG{p}{]}\PYG{p}{)}
    \PYG{n}{data\PYGZus{}t} \PYG{o}{=} \PYG{n}{d\PYGZus{}t}\PYG{o}{/}\PYG{p}{(}\PYG{n}{S\PYGZus{}dob}\PYG{o}{/}\PYG{l+m+mi}{22}\PYG{p}{)}\PYG{o}{*}\PYG{o}{*}\PYG{l+m+mi}{3} \PYG{c+c1}{\PYGZsh{} min, time based on DOB}
    \PYG{n}{data\PYGZus{}d} \PYG{o}{=} \PYG{n}{d\PYGZus{}d}\PYG{o}{/}\PYG{p}{(}\PYG{n}{S\PYGZus{}dob}\PYG{o}{/}\PYG{l+m+mi}{22}\PYG{p}{)} \PYG{c+c1}{\PYGZsh{} m, drawdown data based on DOB}
    \PYG{n}{dist} \PYG{o}{=} \PYG{l+m+mi}{251}\PYG{o}{/}\PYG{p}{(}\PYG{n}{S\PYGZus{}dob}\PYG{o}{/}\PYG{l+m+mi}{22}\PYG{p}{)} \PYG{c+c1}{\PYGZsh{} m, distance to observation well based on DOB}
    \PYG{n}{Aq\PYGZus{}t} \PYG{o}{=} \PYG{l+m+mi}{15}\PYG{o}{/}\PYG{p}{(}\PYG{n}{S\PYGZus{}dob}\PYG{o}{/}\PYG{l+m+mi}{22}\PYG{p}{)} \PYG{c+c1}{\PYGZsh{} m, aquifer thickness based on DOB}
    
    \PYG{n}{i\PYGZus{}u} \PYG{o}{=} \PYG{p}{(}\PYG{l+m+mi}{4}\PYG{o}{*}\PYG{n}{T}\PYG{o}{*}\PYG{n}{data\PYGZus{}t}\PYG{o}{*}\PYG{l+m+mi}{60}\PYG{p}{)}\PYG{o}{/}\PYG{p}{(}\PYG{n}{S}\PYG{o}{*}\PYG{n}{dist}\PYG{o}{*}\PYG{o}{*}\PYG{l+m+mi}{2}\PYG{p}{)} 
    \PYG{n}{W\PYGZus{}u} \PYG{o}{=} \PYG{p}{(}\PYG{l+m+mi}{4}\PYG{o}{*}\PYG{n}{np}\PYG{o}{.}\PYG{n}{pi}\PYG{o}{*}\PYG{n}{data\PYGZus{}d}\PYG{o}{*}\PYG{n}{T}\PYG{p}{)}\PYG{o}{/}\PYG{p}{(}\PYG{n}{Q}\PYG{p}{)}
    \PYG{k}{return} \PYG{n}{i\PYGZus{}u}\PYG{p}{,} \PYG{n}{W\PYGZus{}u}
\end{sphinxVerbatim}

\end{sphinxuseclass}\end{sphinxVerbatimInput}

\end{sphinxuseclass}
\begin{sphinxuseclass}{cell}
\begin{sphinxuseclass}{tag_full-width}\begin{sphinxVerbatimInput}

\begin{sphinxuseclass}{cell_input}
\begin{sphinxVerbatim}[commandchars=\\\{\}]
\PYG{c+c1}{\PYGZsh{}Solution }
\PYG{c+c1}{\PYGZsh{}Q = pumping rate in m\PYGZca{}3/s, DOB\PYGZhy{} date of birth (ddmmyyyy), S = Storage Coeff. and T = Transmissivity (m\PYGZca{}2/s)}
\PYG{c+c1}{\PYGZsh{} Change the value in the bracket to find the fit}

\PYG{n}{i\PYGZus{}u}\PYG{p}{,} \PYG{n}{W\PYGZus{}u} \PYG{o}{=} \PYG{n}{data}\PYG{p}{(}\PYG{n}{Q}\PYG{o}{=}\PYG{l+m+mf}{2.41E\PYGZhy{}02}\PYG{p}{,} \PYG{n}{DOB}\PYG{o}{=}\PYG{l+m+mi}{11111945}\PYG{p}{,} \PYG{n}{S}\PYG{o}{=}\PYG{l+m+mf}{3.53e\PYGZhy{}05}\PYG{p}{,} \PYG{n}{T} \PYG{o}{=} \PYG{l+m+mf}{2.70e\PYGZhy{}03}\PYG{p}{)}

\PYG{c+c1}{\PYGZsh{}interim calculation to get typ\PYGZhy{}curve}
\PYG{n}{u\PYGZus{}1} \PYG{o}{=} \PYG{n}{np}\PYG{o}{.}\PYG{n}{logspace}\PYG{p}{(}\PYG{l+m+mi}{10}\PYG{p}{,}\PYG{o}{\PYGZhy{}}\PYG{l+m+mi}{1}\PYG{p}{,}\PYG{l+m+mi}{250}\PYG{p}{,} \PYG{n}{base}\PYG{o}{=}\PYG{l+m+mf}{10.0}\PYG{p}{)} \PYG{c+c1}{\PYGZsh{} setting the value of u}
\PYG{n}{w\PYGZus{}u} \PYG{o}{=}\PYG{n}{W}\PYG{p}{(}\PYG{l+m+mi}{1}\PYG{o}{/}\PYG{n}{u\PYGZus{}1}\PYG{p}{)} \PYG{c+c1}{\PYGZsh{} finding W(1/u) : as we use 1/u in the typ curce}

\PYG{c+c1}{\PYGZsh{} Output}
\PYG{n}{dx\PYGZus{}1} \PYG{o}{=} \PYG{p}{\PYGZob{}}\PYG{l+s+s2}{\PYGZdq{}}\PYG{l+s+s2}{1/u}\PYG{l+s+s2}{\PYGZdq{}}\PYG{p}{:}\PYG{n}{i\PYGZus{}u}\PYG{p}{,} \PYG{l+s+s2}{\PYGZdq{}}\PYG{l+s+s2}{W(u)}\PYG{l+s+s2}{\PYGZdq{}}\PYG{p}{:}\PYG{n}{W\PYGZus{}u}\PYG{p}{\PYGZcb{}}\PYG{p}{;} \PYG{n}{dfx\PYGZus{}a} \PYG{o}{=} \PYG{n}{pd}\PYG{o}{.}\PYG{n}{DataFrame}\PYG{p}{(}\PYG{n}{dx\PYGZus{}1}\PYG{p}{)}\PYG{p}{;} \PYG{n}{figs} \PYG{o}{=} \PYG{n}{plt}\PYG{o}{.}\PYG{n}{figure}\PYG{p}{(}\PYG{n}{figsize}\PYG{o}{=}\PYG{p}{(}\PYG{l+m+mi}{9}\PYG{p}{,}\PYG{l+m+mi}{6}\PYG{p}{)}\PYG{p}{)} 
\PYG{n}{plt}\PYG{o}{.}\PYG{n}{loglog}\PYG{p}{(}\PYG{n}{u\PYGZus{}1}\PYG{p}{,} \PYG{n}{w\PYGZus{}u}\PYG{p}{)} \PYG{c+c1}{\PYGZsh{} typ curve}
\PYG{n}{plt}\PYG{o}{.}\PYG{n}{loglog}\PYG{p}{(}\PYG{n}{i\PYGZus{}u}\PYG{p}{,} \PYG{n}{W\PYGZus{}u}\PYG{p}{,} \PYG{l+s+s2}{\PYGZdq{}}\PYG{l+s+s2}{ro}\PYG{l+s+s2}{\PYGZdq{}} \PYG{p}{)} \PYG{c+c1}{\PYGZsh{} your data}
\PYG{n}{plt}\PYG{o}{.}\PYG{n}{title}\PYG{p}{(}\PYG{l+s+s2}{\PYGZdq{}}\PYG{l+s+s2}{The typ curve}\PYG{l+s+s2}{\PYGZdq{}}\PYG{p}{)}\PYG{p}{;} \PYG{n}{plt}\PYG{o}{.}\PYG{n}{ylim}\PYG{p}{(}\PYG{p}{(}\PYG{l+m+mf}{0.1}\PYG{p}{,} \PYG{l+m+mi}{10}\PYG{p}{)}\PYG{p}{)}\PYG{p}{;} \PYG{n}{plt}\PYG{o}{.}\PYG{n}{xlim}\PYG{p}{(}\PYG{l+m+mi}{1}\PYG{p}{,} \PYG{l+m+mf}{1e5}\PYG{p}{)}
\PYG{n}{plt}\PYG{o}{.}\PYG{n}{grid}\PYG{p}{(}\PYG{k+kc}{True}\PYG{p}{,} \PYG{n}{which}\PYG{o}{=}\PYG{l+s+s2}{\PYGZdq{}}\PYG{l+s+s2}{both}\PYG{l+s+s2}{\PYGZdq{}}\PYG{p}{,}\PYG{n}{ls}\PYG{o}{=}\PYG{l+s+s2}{\PYGZdq{}}\PYG{l+s+s2}{\PYGZhy{}}\PYG{l+s+s2}{\PYGZdq{}}\PYG{p}{)}\PYG{p}{;} \PYG{n}{plt}\PYG{o}{.}\PYG{n}{ylabel}\PYG{p}{(}\PYG{l+s+sa}{r}\PYG{l+s+s2}{\PYGZdq{}}\PYG{l+s+s2}{W(u)}\PYG{l+s+s2}{\PYGZdq{}}\PYG{p}{)}\PYG{p}{;}\PYG{n}{plt}\PYG{o}{.}\PYG{n}{xlabel}\PYG{p}{(}\PYG{l+s+sa}{r}\PYG{l+s+s2}{\PYGZdq{}}\PYG{l+s+s2}{1/u}\PYG{l+s+s2}{\PYGZdq{}}\PYG{p}{)}\PYG{p}{;} \PYG{n}{plt}\PYG{o}{.}\PYG{n}{close}\PYG{p}{(}\PYG{p}{)}
\PYG{n}{rx\PYGZus{}2} \PYG{o}{=} \PYG{n}{pn}\PYG{o}{.}\PYG{n}{pane}\PYG{o}{.}\PYG{n}{Matplotlib}\PYG{p}{(}\PYG{n}{figs}\PYG{p}{,} \PYG{n}{dpi}\PYG{o}{=}\PYG{l+m+mi}{150}\PYG{p}{)}\PYG{p}{;} \PYG{n}{pn}\PYG{o}{.}\PYG{n}{Row}\PYG{p}{(}\PYG{n}{dfx\PYGZus{}a}\PYG{p}{,} \PYG{n}{rx\PYGZus{}2}\PYG{p}{)} 
\end{sphinxVerbatim}

\end{sphinxuseclass}\end{sphinxVerbatimInput}
\begin{sphinxVerbatimOutput}

\begin{sphinxuseclass}{cell_output}
\begin{sphinxVerbatim}[commandchars=\\\{\}]
Row
    [0] DataFrame(DataFrame)
    [1] Matplotlib(Figure, dpi=150, height=450, width=675)
\end{sphinxVerbatim}

\end{sphinxuseclass}\end{sphinxVerbatimOutput}

\end{sphinxuseclass}
\end{sphinxuseclass}
\sphinxstepscope

\begin{sphinxuseclass}{cell}
\begin{sphinxuseclass}{tag_remove-output}\begin{sphinxVerbatimInput}

\begin{sphinxuseclass}{cell_input}
\begin{sphinxVerbatim}[commandchars=\\\{\}]
\PYG{k+kn}{import} \PYG{n+nn}{numpy} \PYG{k}{as} \PYG{n+nn}{np}
\PYG{k+kn}{import} \PYG{n+nn}{matplotlib}\PYG{n+nn}{.}\PYG{n+nn}{pyplot} \PYG{k}{as} \PYG{n+nn}{plt}
\PYG{k+kn}{import} \PYG{n+nn}{pandas} \PYG{k}{as} \PYG{n+nn}{pd} 
\PYG{k+kn}{import} \PYG{n+nn}{panel} \PYG{k}{as} \PYG{n+nn}{pn}
\PYG{n}{pn}\PYG{o}{.}\PYG{n}{extension}\PYG{p}{(}\PYG{l+s+s1}{\PYGZsq{}}\PYG{l+s+s1}{katex}\PYG{l+s+s1}{\PYGZsq{}}\PYG{p}{)} 
\end{sphinxVerbatim}

\end{sphinxuseclass}\end{sphinxVerbatimInput}

\end{sphinxuseclass}
\end{sphinxuseclass}

\chapter{Tutorial 8 \sphinxhyphen{} Conservative transport}
\label{\detokenize{content/tutorials/T8/tutorial_08:tutorial-8-conservative-transport}}\label{\detokenize{content/tutorials/T8/tutorial_08::doc}}

\begin{enumerate}
\sphinxsetlistlabels{\arabic}{enumi}{enumii}{}{.}%
\item {} 
\sphinxAtStartPar
\sphinxstylestrong{Solution of Homework Problems 8 \sphinxhyphen{} 9}


\item {} 
\sphinxAtStartPar
\sphinxstylestrong{Tutorial Problems on Conservative Transport}


\item {} 
\sphinxAtStartPar
\sphinxstylestrong{Homework Problems on Conservative Transport}

\end{enumerate}




\section{Homework Problem 8: Flow in confined aquifer}
\label{\detokenize{content/tutorials/T8/tutorial_08:homework-problem-8-flow-in-confined-aquifer}}
\begin{sphinxuseclass}{cell}
\begin{sphinxuseclass}{tag_hide-input}
\begin{sphinxuseclass}{tag_full-width}\begin{sphinxVerbatimOutput}

\begin{sphinxuseclass}{cell_output}
\begin{sphinxVerbatim}[commandchars=\\\{\}]
Row
    [0] LaTeX(str, style=\PYGZob{}\PYGZsq{}font\PYGZhy{}size\PYGZsq{}: \PYGZsq{}12pt\PYGZsq{}\PYGZcb{}, width=400)
    [1] PNG(str, width=500)
\end{sphinxVerbatim}

\end{sphinxuseclass}\end{sphinxVerbatimOutput}

\end{sphinxuseclass}
\end{sphinxuseclass}
\end{sphinxuseclass}

\subsection{Solution Homework Problem 8}
\label{\detokenize{content/tutorials/T8/tutorial_08:solution-homework-problem-8}}
\begin{sphinxuseclass}{cell}
\begin{sphinxuseclass}{tag_hide-input}
\begin{sphinxuseclass}{tag_full-width}\begin{sphinxVerbatimOutput}

\begin{sphinxuseclass}{cell_output}
\begin{sphinxVerbatim}[commandchars=\\\{\}]
LaTeX(str, style=\PYGZob{}\PYGZsq{}font\PYGZhy{}size\PYGZsq{}: \PYGZsq{}13pt\PYGZsq{}\PYGZcb{}, width=900)
\end{sphinxVerbatim}

\end{sphinxuseclass}\end{sphinxVerbatimOutput}

\end{sphinxuseclass}
\end{sphinxuseclass}
\end{sphinxuseclass}
\begin{sphinxuseclass}{cell}
\begin{sphinxuseclass}{tag_hide-input}
\begin{sphinxuseclass}{tag_full-width}\begin{sphinxVerbatimOutput}

\begin{sphinxuseclass}{cell_output}
\begin{sphinxVerbatim}[commandchars=\\\{\}]
LaTeX(str, style=\PYGZob{}\PYGZsq{}font\PYGZhy{}size\PYGZsq{}: \PYGZsq{}13pt\PYGZsq{}\PYGZcb{}, width=900)
\end{sphinxVerbatim}

\end{sphinxuseclass}\end{sphinxVerbatimOutput}

\end{sphinxuseclass}
\end{sphinxuseclass}
\end{sphinxuseclass}
\begin{sphinxuseclass}{cell}
\begin{sphinxuseclass}{tag_full-width}\begin{sphinxVerbatimInput}

\begin{sphinxuseclass}{cell_input}
\begin{sphinxVerbatim}[commandchars=\\\{\}]
\PYG{c+c1}{\PYGZsh{}Homework Problem 8 – Continued}
\PYG{c+c1}{\PYGZsh{} given}
\PYG{n}{h8\PYGZus{}1} \PYG{o}{=} \PYG{l+m+mi}{290} \PYG{c+c1}{\PYGZsh{} m, head in Well 1}
\PYG{n}{h8\PYGZus{}3} \PYG{o}{=} \PYG{l+m+mi}{275} \PYG{c+c1}{\PYGZsh{} m, head in the river end}
\PYG{n}{m8\PYGZus{}1} \PYG{o}{=} \PYG{l+m+mi}{6} \PYG{c+c1}{\PYGZsh{} m, aquifer thickness at well 1}
\PYG{n}{m8\PYGZus{}3} \PYG{o}{=} \PYG{l+m+mi}{20} \PYG{c+c1}{\PYGZsh{} m, aquifer thickness near river end}
\PYG{n}{K8} \PYG{o}{=} \PYG{l+m+mf}{5.6} \PYG{o}{*} \PYG{l+m+mi}{10}\PYG{o}{*}\PYG{o}{*}\PYG{o}{\PYGZhy{}}\PYG{l+m+mi}{6} \PYG{c+c1}{\PYGZsh{} m/s, conductivity of aquifer}
\PYG{n}{L8} \PYG{o}{=} \PYG{l+m+mi}{700} \PYG{c+c1}{\PYGZsh{} m, length of the river cross\PYGZhy{}section}
\PYG{n}{W8} \PYG{o}{=} \PYG{l+m+mi}{500} \PYG{c+c1}{\PYGZsh{} m, Width of aquifer}

\PYG{c+c1}{\PYGZsh{} solution }
\PYG{c+c1}{\PYGZsh{} Discharge per unit width using eq. 2F}
\PYG{n}{q8} \PYG{o}{=} \PYG{n}{K8}\PYG{o}{*}\PYG{p}{(}\PYG{p}{(}\PYG{n}{h8\PYGZus{}1} \PYG{o}{\PYGZhy{}} \PYG{n}{h8\PYGZus{}3}\PYG{p}{)}\PYG{o}{/}\PYG{n}{L8}\PYG{p}{)}\PYG{o}{*}\PYG{p}{(}\PYG{n}{m8\PYGZus{}3} \PYG{o}{\PYGZhy{}} \PYG{n}{m8\PYGZus{}1}\PYG{p}{)}\PYG{o}{/}\PYG{n}{np}\PYG{o}{.}\PYG{n}{log}\PYG{p}{(}\PYG{n}{m8\PYGZus{}3}\PYG{o}{/}\PYG{n}{m8\PYGZus{}1}\PYG{p}{)} 
\PYG{n}{Q8} \PYG{o}{=} \PYG{n}{q8}\PYG{o}{*}\PYG{n}{W8} 

\PYG{c+c1}{\PYGZsh{}output}
\PYG{n+nb}{print}\PYG{p}{(}\PYG{l+s+s2}{\PYGZdq{}}\PYG{l+s+s2}{Discharge per unit width of aquifer is: }\PYG{l+s+si}{\PYGZob{}0:1.2e\PYGZcb{}}\PYG{l+s+s2}{\PYGZdq{}}\PYG{o}{.}\PYG{n}{format}\PYG{p}{(}\PYG{n}{q8}\PYG{p}{)}\PYG{p}{,} \PYG{l+s+s2}{\PYGZdq{}}\PYG{l+s+s2}{m}\PYG{l+s+se}{\PYGZbs{}u00b2}\PYG{l+s+s2}{/s }\PYG{l+s+se}{\PYGZbs{}n}\PYG{l+s+s2}{\PYGZdq{}}\PYG{p}{)}
\PYG{n+nb}{print}\PYG{p}{(}\PYG{l+s+s2}{\PYGZdq{}}\PYG{l+s+s2}{Discharge from the given width of aquifer is: }\PYG{l+s+si}{\PYGZob{}0:1.2e\PYGZcb{}}\PYG{l+s+s2}{\PYGZdq{}}\PYG{o}{.}\PYG{n}{format}\PYG{p}{(}\PYG{n}{Q8}\PYG{p}{)}\PYG{p}{,} \PYG{l+s+s2}{\PYGZdq{}}\PYG{l+s+s2}{m}\PYG{l+s+se}{\PYGZbs{}u00b3}\PYG{l+s+s2}{/s}\PYG{l+s+s2}{\PYGZdq{}}\PYG{p}{)}
\end{sphinxVerbatim}

\end{sphinxuseclass}\end{sphinxVerbatimInput}
\begin{sphinxVerbatimOutput}

\begin{sphinxuseclass}{cell_output}
\begin{sphinxVerbatim}[commandchars=\\\{\}]
Discharge per unit width of aquifer is: 1.40e\PYGZhy{}06 m²/s 

Discharge from the given width of aquifer is: 6.98e\PYGZhy{}04 m³/s
\end{sphinxVerbatim}

\end{sphinxuseclass}\end{sphinxVerbatimOutput}

\end{sphinxuseclass}
\end{sphinxuseclass}

\subsection{Homework Problem 9 \sphinxhyphen{} Unconfined aquifer}
\label{\detokenize{content/tutorials/T8/tutorial_08:homework-problem-9-unconfined-aquifer}}
\begin{sphinxuseclass}{cell}
\begin{sphinxuseclass}{tag_hide-input}
\begin{sphinxuseclass}{tag_full-width}\begin{sphinxVerbatimOutput}

\begin{sphinxuseclass}{cell_output}
\begin{sphinxVerbatim}[commandchars=\\\{\}]
Column
    [0] LaTeX(str, style=\PYGZob{}\PYGZsq{}font\PYGZhy{}size\PYGZsq{}: \PYGZsq{}12pt\PYGZsq{}\PYGZcb{}, width=900)
    [1] PNG(str, width=450)
\end{sphinxVerbatim}

\end{sphinxuseclass}\end{sphinxVerbatimOutput}

\end{sphinxuseclass}
\end{sphinxuseclass}
\end{sphinxuseclass}

\subsection{Solution of Homework Problem 9}
\label{\detokenize{content/tutorials/T8/tutorial_08:solution-of-homework-problem-9}}
\begin{sphinxuseclass}{cell}
\begin{sphinxuseclass}{tag_hide-input}
\begin{sphinxuseclass}{tag_full-width}\begin{sphinxVerbatimOutput}

\begin{sphinxuseclass}{cell_output}
\begin{sphinxVerbatim}[commandchars=\\\{\}]
LaTeX(str, style=\PYGZob{}\PYGZsq{}font\PYGZhy{}size\PYGZsq{}: \PYGZsq{}13pt\PYGZsq{}\PYGZcb{}, width=900)
\end{sphinxVerbatim}

\end{sphinxuseclass}\end{sphinxVerbatimOutput}

\end{sphinxuseclass}
\end{sphinxuseclass}
\end{sphinxuseclass}
\begin{sphinxuseclass}{cell}
\begin{sphinxuseclass}{tag_full-width}\begin{sphinxVerbatimInput}

\begin{sphinxuseclass}{cell_input}
\begin{sphinxVerbatim}[commandchars=\\\{\}]
\PYG{c+c1}{\PYGZsh{}Solution of Homework Problem 9}
\PYG{c+c1}{\PYGZsh{} Given}

\PYG{n}{h9\PYGZus{}1} \PYG{o}{=} \PYG{l+m+mi}{30} \PYG{c+c1}{\PYGZsh{} m, River 1 stage }
\PYG{n}{h9\PYGZus{}2} \PYG{o}{=} \PYG{l+m+mi}{10} \PYG{c+c1}{\PYGZsh{} m, River 2 stage}
\PYG{n}{K9} \PYG{o}{=} \PYG{l+m+mi}{5} \PYG{o}{*} \PYG{l+m+mi}{10}\PYG{o}{*}\PYG{o}{*}\PYG{o}{\PYGZhy{}}\PYG{l+m+mi}{4} \PYG{c+c1}{\PYGZsh{} m/s uniform conductivity of aquifer}
\PYG{n}{L9} \PYG{o}{=} \PYG{l+m+mi}{50} \PYG{c+c1}{\PYGZsh{} m, length of the aquifer}
\PYG{n}{w9} \PYG{o}{=} \PYG{l+m+mf}{0.01}\PYG{o}{/}\PYG{p}{(}\PYG{l+m+mi}{24}\PYG{o}{*}\PYG{l+m+mi}{3600}\PYG{p}{)} \PYG{c+c1}{\PYGZsh{} m/s recharge rate in the aquifer}
\PYG{n}{x9} \PYG{o}{=} \PYG{l+m+mi}{5} \PYG{c+c1}{\PYGZsh{} m, loaction at which water table is to be found }

\PYG{c+c1}{\PYGZsh{}Calculation part a}
\PYG{n}{h9\PYGZus{}w} \PYG{o}{=} \PYG{n}{np}\PYG{o}{.}\PYG{n}{sqrt}\PYG{p}{(}\PYG{n}{h9\PYGZus{}1}\PYG{o}{*}\PYG{o}{*}\PYG{l+m+mi}{2} \PYG{o}{\PYGZhy{}} \PYG{p}{(}\PYG{p}{(}\PYG{n}{h9\PYGZus{}1}\PYG{o}{*}\PYG{o}{*}\PYG{l+m+mi}{2} \PYG{o}{\PYGZhy{}} \PYG{n}{h9\PYGZus{}2}\PYG{o}{*}\PYG{o}{*}\PYG{l+m+mi}{2}\PYG{p}{)}\PYG{o}{*}\PYG{n}{x9}\PYG{p}{)}\PYG{o}{/}\PYG{n}{L9} \PYG{o}{+} \PYG{p}{(}\PYG{n}{w9}\PYG{o}{/}\PYG{n}{K9}\PYG{p}{)}\PYG{o}{*}\PYG{p}{(}\PYG{n}{L9}\PYG{o}{\PYGZhy{}}\PYG{n}{x9}\PYG{p}{)}\PYG{o}{*}\PYG{n}{x9}\PYG{p}{)} \PYG{c+c1}{\PYGZsh{} head at x = 5 m from eq. 4A}
\PYG{n}{q9} \PYG{o}{=} \PYG{n}{K9}\PYG{o}{*}\PYG{p}{(}\PYG{p}{(}\PYG{n}{h9\PYGZus{}1}\PYG{o}{*}\PYG{o}{*}\PYG{l+m+mi}{2}\PYG{o}{\PYGZhy{}} \PYG{n}{h9\PYGZus{}2}\PYG{o}{*}\PYG{o}{*}\PYG{l+m+mi}{2}\PYG{p}{)}\PYG{o}{/}\PYG{l+m+mi}{2}\PYG{o}{*}\PYG{n}{L9}\PYG{p}{)} \PYG{o}{\PYGZhy{}} \PYG{n}{w9}\PYG{o}{*}\PYG{p}{(}\PYG{p}{(}\PYG{n}{L9}\PYG{o}{/}\PYG{l+m+mi}{2}\PYG{p}{)}\PYG{o}{\PYGZhy{}}\PYG{n}{x9}\PYG{p}{)}     \PYG{c+c1}{\PYGZsh{} m\PYGZca{}2/s, total dischage from given width}

\PYG{c+c1}{\PYGZsh{}Calculation part b}
\PYG{n}{h9\PYGZus{}nw} \PYG{o}{=} \PYG{n}{np}\PYG{o}{.}\PYG{n}{sqrt}\PYG{p}{(}\PYG{n}{h9\PYGZus{}1}\PYG{o}{*}\PYG{o}{*}\PYG{l+m+mi}{2} \PYG{o}{\PYGZhy{}} \PYG{p}{(}\PYG{p}{(}\PYG{n}{h9\PYGZus{}1}\PYG{o}{*}\PYG{o}{*}\PYG{l+m+mi}{2} \PYG{o}{\PYGZhy{}} \PYG{n}{h9\PYGZus{}2}\PYG{o}{*}\PYG{o}{*}\PYG{l+m+mi}{2}\PYG{p}{)}\PYG{o}{*}\PYG{n}{x9}\PYG{p}{)}\PYG{o}{/}\PYG{n}{L9}\PYG{p}{)}


\PYG{c+c1}{\PYGZsh{}output}
\PYG{n+nb}{print}\PYG{p}{(}\PYG{l+s+s2}{\PYGZdq{}}\PYG{l+s+s2}{The water table at the required  location (x) with recharge is: }\PYG{l+s+si}{\PYGZob{}0:1.5f\PYGZcb{}}\PYG{l+s+s2}{\PYGZdq{}}\PYG{o}{.}\PYG{n}{format}\PYG{p}{(}\PYG{n}{h9\PYGZus{}w}\PYG{p}{)}\PYG{p}{,} \PYG{l+s+s2}{\PYGZdq{}}\PYG{l+s+s2}{m }\PYG{l+s+se}{\PYGZbs{}n}\PYG{l+s+s2}{\PYGZdq{}}\PYG{p}{)}
\PYG{n+nb}{print}\PYG{p}{(}\PYG{l+s+s2}{\PYGZdq{}}\PYG{l+s+s2}{Discharge per unit width from the aquifer is: }\PYG{l+s+si}{\PYGZob{}0:1.2f\PYGZcb{}}\PYG{l+s+s2}{\PYGZdq{}}\PYG{o}{.}\PYG{n}{format}\PYG{p}{(}\PYG{n}{q9}\PYG{p}{)}\PYG{p}{,} \PYG{l+s+s2}{\PYGZdq{}}\PYG{l+s+s2}{m}\PYG{l+s+se}{\PYGZbs{}u00b2}\PYG{l+s+s2}{/s }\PYG{l+s+se}{\PYGZbs{}n}\PYG{l+s+s2}{\PYGZdq{}}\PYG{p}{)}
\PYG{n+nb}{print}\PYG{p}{(}\PYG{l+s+s2}{\PYGZdq{}}\PYG{l+s+s2}{The water table at the required  location (x) without recharge is: }\PYG{l+s+si}{\PYGZob{}0:1.5f\PYGZcb{}}\PYG{l+s+s2}{\PYGZdq{}}\PYG{o}{.}\PYG{n}{format}\PYG{p}{(}\PYG{n}{h9\PYGZus{}nw}\PYG{p}{)}\PYG{p}{,} \PYG{l+s+s2}{\PYGZdq{}}\PYG{l+s+s2}{m }\PYG{l+s+s2}{\PYGZdq{}}\PYG{p}{)}
\end{sphinxVerbatim}

\end{sphinxuseclass}\end{sphinxVerbatimInput}
\begin{sphinxVerbatimOutput}

\begin{sphinxuseclass}{cell_output}
\begin{sphinxVerbatim}[commandchars=\\\{\}]
The water table at the required  location (x) with recharge is: 28.63655 m 

Discharge per unit width from the aquifer is: 10.00 m²/s 

The water table at the required  location (x) without recharge is: 28.63564 m 
\end{sphinxVerbatim}

\end{sphinxuseclass}\end{sphinxVerbatimOutput}

\end{sphinxuseclass}
\end{sphinxuseclass}

\subsection{Tutorial Problem 20}
\label{\detokenize{content/tutorials/T8/tutorial_08:tutorial-problem-20}}
\sphinxAtStartPar
A column (\(L = 1\) m and \(\oslash= 5\) cm) was packed with sandy soil (\(n_e= 35\%\),  \(K= 0.0002\) m/s). The hydraulic head at the inlet and the outlet was set to 235 m and 230 m, respectively. The NaCl solution with concentration 10 mg/L was steadily introduced to the column after saturating it with distilled water. The experiment condition was such that the diffusive flow could be neglected.

\sphinxAtStartPar
\sphinxstylestrong{A.} Determine the advective mass flow and the mass flux entering the column?

\sphinxAtStartPar
\sphinxstylestrong{B.} What will be the concentration at the outlet of the column when only advective flow is considered? Consider the NaCl conc. = 0 at the outlet initially.

\sphinxAtStartPar
\sphinxstylestrong{C.} What will be the dispersion coefficient assuming dispersivity is 0.001 m?


\subsection{Solution of Tutorial Problem 20}
\label{\detokenize{content/tutorials/T8/tutorial_08:solution-of-tutorial-problem-20}}
\begin{sphinxuseclass}{cell}
\begin{sphinxuseclass}{tag_full-width}\begin{sphinxVerbatimInput}

\begin{sphinxuseclass}{cell_input}
\begin{sphinxVerbatim}[commandchars=\\\{\}]
\PYG{c+c1}{\PYGZsh{} Solution Tutorial Problem 20 A.}

\PYG{c+c1}{\PYGZsh{} Given are:  }

\PYG{n}{L} \PYG{o}{=} \PYG{l+m+mi}{1} \PYG{c+c1}{\PYGZsh{} m, length of tube}
\PYG{n}{n\PYGZus{}e} \PYG{o}{=} \PYG{l+m+mf}{0.35} \PYG{c+c1}{\PYGZsh{} (\PYGZhy{}), effective porosity}
\PYG{n}{C} \PYG{o}{=} \PYG{l+m+mi}{10} \PYG{c+c1}{\PYGZsh{} mg/L , NaCl concentration}
\PYG{n}{d} \PYG{o}{=} \PYG{l+m+mi}{5} \PYG{c+c1}{\PYGZsh{} cm, diameter of pipe}
\PYG{n}{K} \PYG{o}{=} \PYG{l+m+mf}{0.0002} \PYG{c+c1}{\PYGZsh{} m/s, conductivity}
\PYG{n}{hin} \PYG{o}{=} \PYG{l+m+mi}{230} \PYG{c+c1}{\PYGZsh{} m, head inlet}
\PYG{n}{hout} \PYG{o}{=} \PYG{l+m+mi}{235} \PYG{c+c1}{\PYGZsh{} m, head outlet}

\PYG{c+c1}{\PYGZsh{} intermediate problem}
\PYG{n}{d\PYGZus{}m} \PYG{o}{=} \PYG{n}{d}\PYG{o}{/}\PYG{l+m+mi}{100} \PYG{c+c1}{\PYGZsh{} m, diameter in m}
\PYG{n}{C\PYGZus{}k} \PYG{o}{=} \PYG{n}{C}\PYG{o}{/}\PYG{l+m+mi}{1000} \PYG{c+c1}{\PYGZsh{} kg/m³ conc. unit change}
\PYG{n}{A\PYGZus{}c} \PYG{o}{=} \PYG{n}{np}\PYG{o}{.}\PYG{n}{pi}\PYG{o}{/}\PYG{l+m+mi}{4}\PYG{o}{*}\PYG{p}{(}\PYG{n}{d\PYGZus{}m}\PYG{p}{)}\PYG{o}{*}\PYG{o}{*}\PYG{l+m+mi}{2} \PYG{c+c1}{\PYGZsh{} m², area column}
\PYG{n}{dh} \PYG{o}{=} \PYG{p}{(}\PYG{n}{hout}\PYG{o}{\PYGZhy{}}\PYG{n}{hin}\PYG{p}{)} \PYG{c+c1}{\PYGZsh{} m, head difference}
\PYG{n}{hd} \PYG{o}{=} \PYG{n}{dh}\PYG{o}{/}\PYG{n}{L}
\PYG{n}{v} \PYG{o}{=} \PYG{p}{(}\PYG{n}{K}\PYG{o}{*}\PYG{n}{dh}\PYG{o}{/}\PYG{n}{L}\PYG{p}{)}\PYG{o}{/}\PYG{n}{n\PYGZus{}e} \PYG{c+c1}{\PYGZsh{} m/s velocity = q/ne= (K*dh/L)/n\PYGZus{}e}

\PYG{c+c1}{\PYGZsh{} calculation and print out}
\PYG{n}{J\PYGZus{}adv} \PYG{o}{=} \PYG{n}{n\PYGZus{}e}\PYG{o}{*}\PYG{n}{v}\PYG{o}{*}\PYG{n}{A\PYGZus{}c}\PYG{o}{*}\PYG{n}{C\PYGZus{}k} \PYG{c+c1}{\PYGZsh{} Kg/s, advective flow}
\PYG{n}{j\PYGZus{}adv} \PYG{o}{=} \PYG{n}{J\PYGZus{}adv}\PYG{o}{/}\PYG{n}{A\PYGZus{}c} \PYG{c+c1}{\PYGZsh{} Kg/(m\PYGZca{}2\PYGZhy{}s), advective flux}

\PYG{n+nb}{print}\PYG{p}{(}\PYG{l+s+s1}{\PYGZsq{}}\PYG{l+s+se}{\PYGZbs{}033}\PYG{l+s+s1}{[1m Results are:}\PYG{l+s+se}{\PYGZbs{}033}\PYG{l+s+s1}{[0m }\PYG{l+s+se}{\PYGZbs{}n}\PYG{l+s+s1}{\PYGZsq{}}\PYG{p}{)}  
\PYG{n+nb}{print}\PYG{p}{(}\PYG{l+s+s2}{\PYGZdq{}}\PYG{l+s+s2}{The required advective mass flow is}\PYG{l+s+si}{\PYGZob{}0:0.2f\PYGZcb{}}\PYG{l+s+s2}{\PYGZdq{}}\PYG{o}{.}\PYG{n}{format}\PYG{p}{(}\PYG{n}{J\PYGZus{}adv}\PYG{p}{)}\PYG{p}{,} \PYG{l+s+s2}{\PYGZdq{}}\PYG{l+s+s2}{Kg/s }\PYG{l+s+se}{\PYGZbs{}n}\PYG{l+s+s2}{\PYGZdq{}}\PYG{p}{)}
\PYG{n+nb}{print}\PYG{p}{(}\PYG{l+s+s2}{\PYGZdq{}}\PYG{l+s+s2}{The required advective mass flow is}\PYG{l+s+si}{\PYGZob{}0:0.2f\PYGZcb{}}\PYG{l+s+s2}{\PYGZdq{}}\PYG{o}{.}\PYG{n}{format}\PYG{p}{(}\PYG{n}{J\PYGZus{}adv}\PYG{p}{)}\PYG{p}{,} \PYG{l+s+s2}{\PYGZdq{}}\PYG{l+s+s2}{Kg/m}\PYG{l+s+se}{\PYGZbs{}u00b2}\PYG{l+s+s2}{\PYGZhy{}s}\PYG{l+s+s2}{\PYGZdq{}}\PYG{p}{)}
\end{sphinxVerbatim}

\end{sphinxuseclass}\end{sphinxVerbatimInput}
\begin{sphinxVerbatimOutput}

\begin{sphinxuseclass}{cell_output}
\begin{sphinxVerbatim}[commandchars=\\\{\}]
\PYG{Color+ColorBold}{ Results are:} 

The required advective mass flow is0.00 Kg/s 

The required advective mass flow is0.00 Kg/m²\PYGZhy{}s
\end{sphinxVerbatim}

\end{sphinxuseclass}\end{sphinxVerbatimOutput}

\end{sphinxuseclass}
\end{sphinxuseclass}
\sphinxAtStartPar
\sphinxstylestrong{Solution Tutorial Problem 20 B.}

\sphinxAtStartPar
If only advective flow is considered, i.e., dispersion and diffusion is not present and we get:
\begin{equation*}
\begin{split}
v\frac{dC}{dx} = 0 
\end{split}
\end{equation*}
\sphinxAtStartPar
Integrating the equation between \(C_{in}\) and \(C_{out}\), we get
\begin{equation*}
\begin{split}
C_{in} = C_{out}
\end{split}
\end{equation*}
\sphinxAtStartPar
The concentration at the \sphinxstylestrong{outlet = inlet = 10 mg/L}.

\begin{sphinxuseclass}{cell}\begin{sphinxVerbatimInput}

\begin{sphinxuseclass}{cell_input}
\begin{sphinxVerbatim}[commandchars=\\\{\}]
\PYG{c+c1}{\PYGZsh{} solution Tutorial Problem 20 C.}

\PYG{c+c1}{\PYGZsh{} Given}

\PYG{n}{dy} \PYG{o}{=} \PYG{l+m+mf}{0.001} \PYG{c+c1}{\PYGZsh{} m, dispersivity }

\PYG{c+c1}{\PYGZsh{} Solution and print}
\PYG{n}{Dis} \PYG{o}{=} \PYG{n}{dy}\PYG{o}{*}\PYG{n}{v} \PYG{c+c1}{\PYGZsh{} m²/s, dispersion coeff.}
\PYG{n+nb}{print}\PYG{p}{(}\PYG{l+s+s2}{\PYGZdq{}}\PYG{l+s+se}{\PYGZbs{}n}\PYG{l+s+s2}{ The required dispersion coefficent is }\PYG{l+s+si}{\PYGZob{}0:0.2e\PYGZcb{}}\PYG{l+s+s2}{\PYGZdq{}}\PYG{o}{.}\PYG{n}{format}\PYG{p}{(}\PYG{n}{Dis}\PYG{p}{)}\PYG{p}{,} \PYG{l+s+s2}{\PYGZdq{}}\PYG{l+s+s2}{m}\PYG{l+s+se}{\PYGZbs{}u00b2}\PYG{l+s+s2}{/s}\PYG{l+s+s2}{\PYGZdq{}}\PYG{p}{)}
\end{sphinxVerbatim}

\end{sphinxuseclass}\end{sphinxVerbatimInput}
\begin{sphinxVerbatimOutput}

\begin{sphinxuseclass}{cell_output}
\begin{sphinxVerbatim}[commandchars=\\\{\}]
 The required dispersion coefficent is 2.86e\PYGZhy{}06 m²/s
\end{sphinxVerbatim}

\end{sphinxuseclass}\end{sphinxVerbatimOutput}

\end{sphinxuseclass}

\subsection{Tutorial Problem 21}
\label{\detokenize{content/tutorials/T8/tutorial_08:tutorial-problem-21}}
\sphinxAtStartPar
A 1 m long column (\(\oslash\) = 5 cm) is packed (\(n_e= 30\%\)). The NaCl solution with concentration 70 mg/L is introduced to the column at the flow rate of 100 mL/min. The following NaCl concentrations were measured at the outlet at different times:

\begin{sphinxuseclass}{cell}
\begin{sphinxuseclass}{tag_hide-input}
\begin{sphinxuseclass}{tag_full-width}\begin{sphinxVerbatimOutput}

\begin{sphinxuseclass}{cell_output}
\begin{sphinxVerbatim}[commandchars=\\\{\}]
    Time (min)  Conc (mg/L)
0           60         0.00
1          120         0.00
2          180         0.00
3          240         2.50
4          300         5.40
5          360        10.60
6          420        21.05
7          480        29.00
8          540        41.00
9          600        52.00
10         660        53.00
11         720        52.00
\end{sphinxVerbatim}

\end{sphinxuseclass}\end{sphinxVerbatimOutput}

\end{sphinxuseclass}
\end{sphinxuseclass}
\end{sphinxuseclass}
\sphinxAtStartPar
\sphinxstylestrong{Find:}

\sphinxAtStartPar
\sphinxstylestrong{A.} Normalize outlet concentration with the initial concentration?

\sphinxAtStartPar
\sphinxstylestrong{B.} Plot normalized conc. as a function of time?

\sphinxAtStartPar
\sphinxstylestrong{C.} Find pore volume?

\sphinxAtStartPar
\sphinxstylestrong{D.} Plot normalized conc. as a function of pore volumes.

\sphinxAtStartPar
\sphinxstylestrong{E.} Find breakthrough time and number of flushing required to obtain a breakthrough concentration.


\subsection{Solution Problem 21}
\label{\detokenize{content/tutorials/T8/tutorial_08:solution-problem-21}}
\sphinxAtStartPar
\sphinxstylestrong{21. A \& B}

\sphinxAtStartPar
To normalize with initial concentration, we divide the outlet concentration with the initial concentration = 70 mg/L, we get:

\begin{sphinxuseclass}{cell}
\begin{sphinxuseclass}{tag_full-width}\begin{sphinxVerbatimInput}

\begin{sphinxuseclass}{cell_input}
\begin{sphinxVerbatim}[commandchars=\\\{\}]
\PYG{c+c1}{\PYGZsh{}solution 21.A \PYGZam{} B}

\PYG{n}{c\PYGZus{}ini} \PYG{o}{=} \PYG{l+m+mi}{70} \PYG{c+c1}{\PYGZsh{} mg/L, initail concentration}

\PYG{n}{c\PYGZus{}n} \PYG{o}{=} \PYG{n}{c\PYGZus{}mg}\PYG{o}{/}\PYG{n}{c\PYGZus{}ini} \PYG{c+c1}{\PYGZsh{} (\PYGZhy{}), normalized concentration}

\PYG{n}{d2} \PYG{o}{=} \PYG{p}{\PYGZob{}}\PYG{l+s+s2}{\PYGZdq{}}\PYG{l+s+s2}{Conc (mg/L)}\PYG{l+s+s2}{\PYGZdq{}}\PYG{p}{:} \PYG{n}{c\PYGZus{}mg}\PYG{p}{,} \PYG{l+s+s2}{\PYGZdq{}}\PYG{l+s+s2}{Normalized Conc ()}\PYG{l+s+s2}{\PYGZdq{}}\PYG{p}{:} \PYG{n}{c\PYGZus{}n} \PYG{p}{\PYGZcb{}}
\PYG{n}{df2} \PYG{o}{=} \PYG{n}{pd}\PYG{o}{.}\PYG{n}{DataFrame}\PYG{p}{(}\PYG{n}{d2}\PYG{p}{)}
 \PYG{c+c1}{\PYGZsh{} display top 5 data}

\PYG{n}{plt}\PYG{o}{.}\PYG{n}{plot}\PYG{p}{(}\PYG{n}{t\PYGZus{}m}\PYG{p}{,} \PYG{n}{c\PYGZus{}n}\PYG{p}{,} \PYG{l+s+s2}{\PYGZdq{}}\PYG{l+s+s2}{\PYGZhy{}\PYGZhy{}o}\PYG{l+s+s2}{\PYGZdq{}}\PYG{p}{,}\PYG{p}{)}
\PYG{n}{plt}\PYG{o}{.}\PYG{n}{xlabel}\PYG{p}{(}\PYG{l+s+s2}{\PYGZdq{}}\PYG{l+s+s2}{Time (min)}\PYG{l+s+s2}{\PYGZdq{}}\PYG{p}{)}\PYG{p}{;} \PYG{n}{plt}\PYG{o}{.}\PYG{n}{ylabel}\PYG{p}{(}\PYG{l+s+sa}{r}\PYG{l+s+s2}{\PYGZdq{}}\PYG{l+s+s2}{Normalized Conc. \PYGZdl{}C(t)/C(0)\PYGZdl{} (\PYGZhy{})}\PYG{l+s+s2}{\PYGZdq{}}\PYG{p}{)}\PYG{p}{;}

\PYG{n}{df2}
\end{sphinxVerbatim}

\end{sphinxuseclass}\end{sphinxVerbatimInput}
\begin{sphinxVerbatimOutput}

\begin{sphinxuseclass}{cell_output}
\begin{sphinxVerbatim}[commandchars=\\\{\}]
    Conc (mg/L)  Normalized Conc ()
0          0.00            0.000000
1          0.00            0.000000
2          0.00            0.000000
3          2.50            0.035714
4          5.40            0.077143
5         10.60            0.151429
6         21.05            0.300714
7         29.00            0.414286
8         41.00            0.585714
9         52.00            0.742857
10        53.00            0.757143
11        52.00            0.742857
\end{sphinxVerbatim}

\noindent\sphinxincludegraphics{{C:/Users/vibhu/GWtextbook/_build/jupyter_execute/tutorial_08_21_1}.png}

\end{sphinxuseclass}\end{sphinxVerbatimOutput}

\end{sphinxuseclass}
\end{sphinxuseclass}
\sphinxAtStartPar
\sphinxstylestrong{Solution 21.C}

\sphinxAtStartPar
A pore volume is the volume {[}L\(^3\){]} of water that comes out of the saturated column. It can be obtained from the following equation:
\begin{equation*}
\begin{split}
P_v = n_e \times V 
\end{split}
\end{equation*}
\sphinxAtStartPar
Where, V = Volume of packing

\sphinxAtStartPar
To obtain a number of pore volume, use
\begin{equation*}
\begin{split}
N_{Pv} = V_x \times t/L
\end{split}
\end{equation*}
\sphinxAtStartPar
where, \(V_x\) = Linear velocity {[}L/T{]}, \(t\)  = time {[}T{]}, \(L\) = Column Length {[}L{]}

\sphinxAtStartPar
If volumetric flow (\(V_{vol}\)) is given, use
\begin{equation*}
\begin{split}
N_{Pv} = V_{vol}\times t/V 
\end{split}
\end{equation*}
\begin{sphinxuseclass}{cell}
\begin{sphinxuseclass}{tag_full-width}\begin{sphinxVerbatimInput}

\begin{sphinxuseclass}{cell_input}
\begin{sphinxVerbatim}[commandchars=\\\{\}]
\PYG{c+c1}{\PYGZsh{}solution 21.C}
\PYG{c+c1}{\PYGZsh{} Given}

\PYG{n}{dia} \PYG{o}{=} \PYG{l+m+mi}{5} \PYG{c+c1}{\PYGZsh{} cm, diameter of column}
\PYG{n}{L\PYGZus{}c} \PYG{o}{=} \PYG{l+m+mi}{1} \PYG{c+c1}{\PYGZsh{} m, length of column}
\PYG{n}{v\PYGZus{}vol} \PYG{o}{=} \PYG{l+m+mi}{100} \PYG{c+c1}{\PYGZsh{} mL/min, volumetric flow}

\PYG{c+c1}{\PYGZsh{}intermediate calculation}
\PYG{n}{dia\PYGZus{}m} \PYG{o}{=} \PYG{n}{dia}\PYG{o}{/}\PYG{l+m+mi}{100} \PYG{c+c1}{\PYGZsh{} m, diameter of column }
\PYG{n}{v\PYGZus{}c} \PYG{o}{=} \PYG{n}{np}\PYG{o}{.}\PYG{n}{pi}\PYG{o}{/}\PYG{l+m+mi}{4}\PYG{o}{*}\PYG{n}{dia\PYGZus{}m}\PYG{o}{*}\PYG{o}{*}\PYG{l+m+mi}{2}\PYG{o}{*}\PYG{n}{L\PYGZus{}c}
\PYG{n}{v\PYGZus{}volm} \PYG{o}{=} \PYG{n}{v\PYGZus{}vol}\PYG{o}{/}\PYG{l+m+mi}{10}\PYG{o}{*}\PYG{o}{*}\PYG{l+m+mi}{6} \PYG{c+c1}{\PYGZsh{} m³/min, vol. flow unit change}

\PYG{c+c1}{\PYGZsh{}compute and print}
\PYG{n}{p\PYGZus{}vol} \PYG{o}{=} \PYG{n}{np}\PYG{o}{.}\PYG{n}{round}\PYG{p}{(}\PYG{n}{v\PYGZus{}volm}\PYG{o}{*}\PYG{n}{t\PYGZus{}m}\PYG{o}{/}\PYG{n}{v\PYGZus{}c}\PYG{p}{,}\PYG{l+m+mi}{2}\PYG{p}{)} \PYG{c+c1}{\PYGZsh{} (), nr. of pore volume, rounded to 2 decimal place}

\PYG{n}{d3} \PYG{o}{=} \PYG{p}{\PYGZob{}}\PYG{l+s+s2}{\PYGZdq{}}\PYG{l+s+s2}{Time (min)}\PYG{l+s+s2}{\PYGZdq{}}\PYG{p}{:}\PYG{n}{t\PYGZus{}m}\PYG{p}{,} \PYG{l+s+s2}{\PYGZdq{}}\PYG{l+s+s2}{Nr. of pore volumes }\PYG{l+s+s2}{\PYGZdq{}}\PYG{p}{:} \PYG{n}{p\PYGZus{}vol}\PYG{p}{\PYGZcb{}}
\PYG{n}{df3} \PYG{o}{=} \PYG{n}{pd}\PYG{o}{.}\PYG{n}{DataFrame}\PYG{p}{(}\PYG{n}{d3}\PYG{p}{)}

\PYG{n}{plt}\PYG{o}{.}\PYG{n}{plot}\PYG{p}{(}\PYG{n}{p\PYGZus{}vol}\PYG{p}{,} \PYG{n}{c\PYGZus{}n}\PYG{p}{,}\PYG{l+s+s2}{\PYGZdq{}}\PYG{l+s+s2}{\PYGZhy{}\PYGZhy{}o}\PYG{l+s+s2}{\PYGZdq{}}\PYG{p}{,}\PYG{p}{)}
\PYG{n}{plt}\PYG{o}{.}\PYG{n}{ylabel}\PYG{p}{(}\PYG{l+s+sa}{r}\PYG{l+s+s2}{\PYGZdq{}}\PYG{l+s+s2}{Normalized Conc., \PYGZdl{}C(t)/C(0)\PYGZdl{} (\PYGZhy{})}\PYG{l+s+s2}{\PYGZdq{}}\PYG{p}{)}\PYG{p}{;} \PYG{n}{plt}\PYG{o}{.}\PYG{n}{xlabel}\PYG{p}{(}\PYG{l+s+s2}{\PYGZdq{}}\PYG{l+s+s2}{Nr. of pore volume (\PYGZhy{})}\PYG{l+s+s2}{\PYGZdq{}}\PYG{p}{)}\PYG{p}{;}

\PYG{n}{df3} \PYG{c+c1}{\PYGZsh{} display top 5 data}
\end{sphinxVerbatim}

\end{sphinxuseclass}\end{sphinxVerbatimInput}
\begin{sphinxVerbatimOutput}

\begin{sphinxuseclass}{cell_output}
\begin{sphinxVerbatim}[commandchars=\\\{\}]
    Time (min)  Nr. of pore volumes 
0           60                  3.06
1          120                  6.11
2          180                  9.17
3          240                 12.22
4          300                 15.28
5          360                 18.33
6          420                 21.39
7          480                 24.45
8          540                 27.50
9          600                 30.56
10         660                 33.61
11         720                 36.67
\end{sphinxVerbatim}

\noindent\sphinxincludegraphics{{C:/Users/vibhu/GWtextbook/_build/jupyter_execute/tutorial_08_23_1}.png}

\end{sphinxuseclass}\end{sphinxVerbatimOutput}

\end{sphinxuseclass}
\end{sphinxuseclass}
\sphinxAtStartPar
\sphinxstylestrong{Solution 21.E}

\begin{sphinxuseclass}{cell}
\begin{sphinxuseclass}{tag_full-width}\begin{sphinxVerbatimInput}

\begin{sphinxuseclass}{cell_input}
\begin{sphinxVerbatim}[commandchars=\\\{\}]
\PYG{n}{fig} \PYG{o}{=} \PYG{n}{plt}\PYG{o}{.}\PYG{n}{figure}\PYG{p}{(}\PYG{p}{)}
\PYG{n}{ax1} \PYG{o}{=} \PYG{n}{fig}\PYG{o}{.}\PYG{n}{add\PYGZus{}subplot}\PYG{p}{(}\PYG{l+m+mi}{111}\PYG{p}{)}
\PYG{n}{ax2} \PYG{o}{=} \PYG{n}{ax1}\PYG{o}{.}\PYG{n}{twiny}\PYG{p}{(}\PYG{p}{)}

\PYG{n}{ax1}\PYG{o}{.}\PYG{n}{plot}\PYG{p}{(}\PYG{n}{t\PYGZus{}m}\PYG{p}{,} \PYG{n}{c\PYGZus{}n}\PYG{p}{,} \PYG{l+s+s2}{\PYGZdq{}}\PYG{l+s+s2}{\PYGZhy{}\PYGZhy{}o}\PYG{l+s+s2}{\PYGZdq{}}\PYG{p}{,}\PYG{p}{)}
\PYG{n}{ax1}\PYG{o}{.}\PYG{n}{yaxis}\PYG{o}{.}\PYG{n}{grid}\PYG{p}{(}\PYG{p}{)}
\PYG{n}{ax1}\PYG{o}{.}\PYG{n}{xaxis}\PYG{o}{.}\PYG{n}{grid}\PYG{p}{(}\PYG{p}{)}
\PYG{n}{ax1}\PYG{o}{.}\PYG{n}{set\PYGZus{}xlabel}\PYG{p}{(}\PYG{l+s+s2}{\PYGZdq{}}\PYG{l+s+s2}{Time (min)}\PYG{l+s+s2}{\PYGZdq{}}\PYG{p}{)}
\PYG{n}{ax1}\PYG{o}{.}\PYG{n}{set\PYGZus{}ylabel}\PYG{p}{(}\PYG{l+s+sa}{r}\PYG{l+s+s2}{\PYGZdq{}}\PYG{l+s+s2}{Normalized conc., \PYGZdl{}C(t)/C(0), (\PYGZhy{})\PYGZdl{} }\PYG{l+s+s2}{\PYGZdq{}}\PYG{p}{)}
\PYG{n}{ax2}\PYG{o}{.}\PYG{n}{plot}\PYG{p}{(}\PYG{n}{p\PYGZus{}vol}\PYG{p}{,} \PYG{n}{c\PYGZus{}n}\PYG{p}{,}\PYG{l+s+s2}{\PYGZdq{}}\PYG{l+s+s2}{\PYGZhy{}\PYGZhy{}o}\PYG{l+s+s2}{\PYGZdq{}}\PYG{p}{,}\PYG{p}{)}
\PYG{n}{ax2}\PYG{o}{.}\PYG{n}{set\PYGZus{}xlabel}\PYG{p}{(}\PYG{l+s+s2}{\PYGZdq{}}\PYG{l+s+s2}{Nr. of pore\PYGZhy{}volumes (\PYGZhy{})}\PYG{l+s+s2}{\PYGZdq{}}\PYG{p}{)}
\PYG{n}{plt}\PYG{o}{.}\PYG{n}{axhline}\PYG{p}{(}\PYG{n}{y}\PYG{o}{=}\PYG{l+m+mf}{.5}\PYG{p}{)}
\end{sphinxVerbatim}

\end{sphinxuseclass}\end{sphinxVerbatimInput}
\begin{sphinxVerbatimOutput}

\begin{sphinxuseclass}{cell_output}
\begin{sphinxVerbatim}[commandchars=\\\{\}]
\PYGZlt{}matplotlib.lines.Line2D at 0x13c6d15e7d0\PYGZgt{}
\end{sphinxVerbatim}

\noindent\sphinxincludegraphics{{C:/Users/vibhu/GWtextbook/_build/jupyter_execute/tutorial_08_25_1}.png}

\end{sphinxuseclass}\end{sphinxVerbatimOutput}

\end{sphinxuseclass}
\end{sphinxuseclass}
\sphinxAtStartPar
\sphinxstylestrong{Solution 21.E}

\sphinxAtStartPar
The breakthrough is when outlet concentration is at 50\% of the normalized concentration. In the figure the line is drawn. The breakthrough curve is then interpreted as:

\sphinxAtStartPar
After 510 minutes and 27 pore volumes of the input concentration the breakthrough concentration is obtained at the outlet


\subsection{Homework Problem}
\label{\detokenize{content/tutorials/T8/tutorial_08:homework-problem}}
\sphinxAtStartPar
\sphinxstylestrong{Homework Problem 11} \sphinxhyphen{} Conservative transport

\begin{sphinxuseclass}{cell}
\begin{sphinxuseclass}{tag_hide-input}
\begin{sphinxuseclass}{tag_full-width}\begin{sphinxVerbatimOutput}

\begin{sphinxuseclass}{cell_output}
\begin{sphinxVerbatim}[commandchars=\\\{\}]
Row
    [0] Column
        [0] Markdown(str, style=\PYGZob{}\PYGZsq{}font\PYGZhy{}size\PYGZsq{}: \PYGZsq{}12pt\PYGZsq{}\PYGZcb{}, width=600)
        [1] LaTeX(str, style=\PYGZob{}\PYGZsq{}font\PYGZhy{}size\PYGZsq{}: \PYGZsq{}12pt\PYGZsq{}\PYGZcb{}, width=600)
    [1] Spacer(width=50)
    [2] DataFrame(DataFrame)
\end{sphinxVerbatim}

\end{sphinxuseclass}\end{sphinxVerbatimOutput}

\end{sphinxuseclass}
\end{sphinxuseclass}
\end{sphinxuseclass}
\sphinxstepscope

\begin{sphinxuseclass}{cell}
\begin{sphinxuseclass}{tag_remove-output}\begin{sphinxVerbatimInput}

\begin{sphinxuseclass}{cell_input}
\begin{sphinxVerbatim}[commandchars=\\\{\}]
\PYG{k+kn}{import} \PYG{n+nn}{numpy} \PYG{k}{as} \PYG{n+nn}{np}
\PYG{k+kn}{import} \PYG{n+nn}{matplotlib}\PYG{n+nn}{.}\PYG{n+nn}{pyplot} \PYG{k}{as} \PYG{n+nn}{plt}
\PYG{k+kn}{import} \PYG{n+nn}{pandas} \PYG{k}{as} \PYG{n+nn}{pd} 
\PYG{k+kn}{import} \PYG{n+nn}{panel} \PYG{k}{as} \PYG{n+nn}{pn}
\PYG{k+kn}{from} \PYG{n+nn}{scipy} \PYG{k+kn}{import} \PYG{n}{stats} 
\PYG{n}{pn}\PYG{o}{.}\PYG{n}{extension}\PYG{p}{(}\PYG{l+s+s1}{\PYGZsq{}}\PYG{l+s+s1}{katex}\PYG{l+s+s1}{\PYGZsq{}}\PYG{p}{)} 

\PYG{k+kn}{import} \PYG{n+nn}{warnings}
\PYG{n}{warnings}\PYG{o}{.}\PYG{n}{filterwarnings}\PYG{p}{(}\PYG{l+s+s2}{\PYGZdq{}}\PYG{l+s+s2}{ignore}\PYG{l+s+s2}{\PYGZdq{}}\PYG{p}{)}
\end{sphinxVerbatim}

\end{sphinxuseclass}\end{sphinxVerbatimInput}

\end{sphinxuseclass}
\end{sphinxuseclass}

\chapter{Tutorial 9 \sphinxhyphen{} Reactive transport}
\label{\detokenize{content/tutorials/T9/tutorial_09:tutorial-9-reactive-transport}}\label{\detokenize{content/tutorials/T9/tutorial_09::doc}}

\begin{enumerate}
\sphinxsetlistlabels{\arabic}{enumi}{enumii}{}{.}%
\item {} 
\sphinxAtStartPar
\sphinxstylestrong{Solution of Homework Problems 10 \sphinxhyphen{} 11}


\item {} 
\sphinxAtStartPar
\sphinxstylestrong{Tutorial Problems on Reactive Transport}


\item {} 
\sphinxAtStartPar
\sphinxstylestrong{Homework Problems on Reactive Transport}

\end{enumerate}




\section{Homework Problem 10: Aquifer characterization}
\label{\detokenize{content/tutorials/T9/tutorial_09:homework-problem-10-aquifer-characterization}}
\sphinxAtStartPar
A pumping test is conducted to determine hydraulic properties (storage coefficient \(S\), the transmissivity \(T\) and the hydraulic conductivity \(K\)) of
the aquifer. of a confined aquifer. For this purpose, a constant
pumping rate of 1219 m3/d is established and drawdown is recorded in an observation well. This problem is to be
solved with the Theis method implemented in the code below.

\sphinxAtStartPar
The code generates the typ curve based on your date of birth (ddmmyyyy). To use the code, you will provide different value of \(T\) and \(S\) and make a match of the data with the typ\sphinxhyphen{}curve.

\sphinxAtStartPar
Code (2 cells below)


\subsection{Solution Homework Problem 10}
\label{\detokenize{content/tutorials/T9/tutorial_09:solution-homework-problem-10}}
\begin{sphinxuseclass}{cell}\begin{sphinxVerbatimInput}

\begin{sphinxuseclass}{cell_input}
\begin{sphinxVerbatim}[commandchars=\\\{\}]
\PYG{c+c1}{\PYGZsh{} Functions to generate well\PYGZhy{}function (this is another method based on scipy library)}

\PYG{k+kn}{from} \PYG{n+nn}{scipy}\PYG{n+nn}{.}\PYG{n+nn}{special} \PYG{k+kn}{import} \PYG{n}{expi}
\PYG{k}{def} \PYG{n+nf}{W}\PYG{p}{(}\PYG{n}{u}\PYG{p}{)}\PYG{p}{:} 
    \PYG{k}{return} \PYG{o}{\PYGZhy{}}\PYG{n}{expi}\PYG{p}{(}\PYG{o}{\PYGZhy{}}\PYG{n}{u}\PYG{p}{)}

\PYG{c+c1}{\PYGZsh{}Generate your data and function required to solve}

\PYG{k}{def} \PYG{n+nf}{data}\PYG{p}{(}\PYG{n}{Q}\PYG{p}{,} \PYG{n}{DOB}\PYG{p}{,} \PYG{n}{S}\PYG{p}{,} \PYG{n}{T}\PYG{p}{)}\PYG{p}{:}

    \PYG{l+s+sd}{\PYGZsq{}\PYGZsq{}\PYGZsq{}}
\PYG{l+s+sd}{    Q = pumping rate in m\PYGZca{}3/s, }
\PYG{l+s+sd}{    DOB\PYGZhy{} date of birth (ddmmyyyy), }
\PYG{l+s+sd}{    S = Storage Coeff. and }
\PYG{l+s+sd}{    T = Transmissivity (m\PYGZca{}2/s)}
\PYG{l+s+sd}{    \PYGZsq{}\PYGZsq{}\PYGZsq{}}
    \PYG{n}{S\PYGZus{}dob} \PYG{o}{=} \PYG{n+nb}{sum}\PYG{p}{(}\PYG{n+nb}{int}\PYG{p}{(}\PYG{n}{DOB}\PYG{p}{)} \PYG{k}{for} \PYG{n}{DOB} \PYG{o+ow}{in} \PYG{n+nb}{str}\PYG{p}{(}\PYG{n}{DOB}\PYG{p}{)}\PYG{p}{)} \PYG{c+c1}{\PYGZsh{} add numbers in your DOB}
    \PYG{n}{d\PYGZus{}t} \PYG{o}{=} \PYG{n}{np}\PYG{o}{.}\PYG{n}{array}\PYG{p}{(}\PYG{p}{[}\PYG{l+m+mf}{3.5}\PYG{p}{,} \PYG{l+m+mi}{5}\PYG{p}{,} \PYG{l+m+mf}{6.2}\PYG{p}{,} \PYG{l+m+mi}{8}\PYG{p}{,} \PYG{l+m+mf}{9.2}\PYG{p}{,} \PYG{l+m+mf}{12.4}\PYG{p}{,} \PYG{l+m+mf}{16.5}\PYG{p}{,} \PYG{l+m+mi}{20}\PYG{p}{,} \PYG{l+m+mi}{30}\PYG{p}{,} \PYG{l+m+mi}{60}\PYG{p}{,} \PYG{l+m+mi}{100}\PYG{p}{,} \PYG{l+m+mi}{200}\PYG{p}{,} \PYG{l+m+mi}{320}\PYG{p}{,} \PYG{l+m+mi}{380}\PYG{p}{,} \PYG{l+m+mi}{500}\PYG{p}{]}\PYG{p}{)}
    \PYG{n}{d\PYGZus{}d} \PYG{o}{=} \PYG{n}{np}\PYG{o}{.}\PYG{n}{array}\PYG{p}{(}\PYG{p}{[}\PYG{l+m+mf}{0.12}\PYG{p}{,} \PYG{l+m+mf}{0.23}\PYG{p}{,} \PYG{l+m+mf}{0.31}\PYG{p}{,} \PYG{l+m+mf}{0.41}\PYG{p}{,} \PYG{l+m+mf}{0.47}\PYG{p}{,} \PYG{l+m+mf}{0.64}\PYG{p}{,} \PYG{l+m+mf}{0.82}\PYG{p}{,} \PYG{l+m+mf}{0.92}\PYG{p}{,} \PYG{l+m+mf}{1.2}\PYG{p}{,} \PYG{l+m+mf}{1.74}\PYG{p}{,} \PYG{l+m+mf}{2.14}\PYG{p}{,} \PYG{l+m+mf}{2.57}\PYG{p}{,} \PYG{l+m+mi}{3}\PYG{p}{,} \PYG{l+m+mf}{3.1}\PYG{p}{,} \PYG{l+m+mf}{3.34}\PYG{p}{]}\PYG{p}{)}
    \PYG{n}{data\PYGZus{}t} \PYG{o}{=} \PYG{n}{d\PYGZus{}t}\PYG{o}{/}\PYG{p}{(}\PYG{n}{S\PYGZus{}dob}\PYG{o}{/}\PYG{l+m+mi}{22}\PYG{p}{)}\PYG{o}{*}\PYG{o}{*}\PYG{l+m+mi}{3} \PYG{c+c1}{\PYGZsh{} min, time based on DOB}
    \PYG{n}{data\PYGZus{}d} \PYG{o}{=} \PYG{n}{d\PYGZus{}d}\PYG{o}{/}\PYG{p}{(}\PYG{n}{S\PYGZus{}dob}\PYG{o}{/}\PYG{l+m+mi}{22}\PYG{p}{)} \PYG{c+c1}{\PYGZsh{} m, drawdown data based on DOB}
    \PYG{n}{dist} \PYG{o}{=} \PYG{l+m+mi}{251}\PYG{o}{/}\PYG{p}{(}\PYG{n}{S\PYGZus{}dob}\PYG{o}{/}\PYG{l+m+mi}{22}\PYG{p}{)} \PYG{c+c1}{\PYGZsh{} m, distance to observation well based on DOB}
    \PYG{n}{Aq\PYGZus{}t} \PYG{o}{=} \PYG{l+m+mi}{15}\PYG{o}{/}\PYG{p}{(}\PYG{n}{S\PYGZus{}dob}\PYG{o}{/}\PYG{l+m+mi}{22}\PYG{p}{)} \PYG{c+c1}{\PYGZsh{} m, aquifer thickness based on DOB}
    
    \PYG{n}{i\PYGZus{}u} \PYG{o}{=} \PYG{p}{(}\PYG{l+m+mi}{4}\PYG{o}{*}\PYG{n}{T}\PYG{o}{*}\PYG{n}{data\PYGZus{}t}\PYG{o}{*}\PYG{l+m+mi}{60}\PYG{p}{)}\PYG{o}{/}\PYG{p}{(}\PYG{n}{S}\PYG{o}{*}\PYG{n}{dist}\PYG{o}{*}\PYG{o}{*}\PYG{l+m+mi}{2}\PYG{p}{)} 
    \PYG{n}{W\PYGZus{}u} \PYG{o}{=} \PYG{p}{(}\PYG{l+m+mi}{4}\PYG{o}{*}\PYG{n}{np}\PYG{o}{.}\PYG{n}{pi}\PYG{o}{*}\PYG{n}{data\PYGZus{}d}\PYG{o}{*}\PYG{n}{T}\PYG{p}{)}\PYG{o}{/}\PYG{p}{(}\PYG{n}{Q}\PYG{p}{)}
    \PYG{k}{return} \PYG{n}{i\PYGZus{}u}\PYG{p}{,} \PYG{n}{W\PYGZus{}u}
\end{sphinxVerbatim}

\end{sphinxuseclass}\end{sphinxVerbatimInput}

\end{sphinxuseclass}
\begin{sphinxuseclass}{cell}
\begin{sphinxuseclass}{tag_full-width}\begin{sphinxVerbatimInput}

\begin{sphinxuseclass}{cell_input}
\begin{sphinxVerbatim}[commandchars=\\\{\}]
\PYG{c+c1}{\PYGZsh{}Solution }
\PYG{c+c1}{\PYGZsh{}Q = pumping rate in m\PYGZca{}3/s, DOB\PYGZhy{} date of birth (ddmmyyyy), S = Storage Coeff. and T = Transmissivity (m\PYGZca{}2/s)}
\PYG{c+c1}{\PYGZsh{} Change the value in the bracket to find the fit}

\PYG{n}{i\PYGZus{}u}\PYG{p}{,} \PYG{n}{W\PYGZus{}u} \PYG{o}{=} \PYG{n}{data}\PYG{p}{(}\PYG{n}{Q}\PYG{o}{=}\PYG{l+m+mf}{2.41E\PYGZhy{}02}\PYG{p}{,} \PYG{n}{DOB}\PYG{o}{=}\PYG{l+m+mi}{17071975}\PYG{p}{,} \PYG{n}{S}\PYG{o}{=}\PYG{l+m+mf}{4.0e\PYGZhy{}05}\PYG{p}{,} \PYG{n}{T} \PYG{o}{=} \PYG{l+m+mf}{3.5e\PYGZhy{}03}\PYG{p}{)}

\PYG{c+c1}{\PYGZsh{}interim calculation to get typ\PYGZhy{}curve}
\PYG{n}{u\PYGZus{}1} \PYG{o}{=} \PYG{n}{np}\PYG{o}{.}\PYG{n}{logspace}\PYG{p}{(}\PYG{l+m+mi}{10}\PYG{p}{,}\PYG{o}{\PYGZhy{}}\PYG{l+m+mi}{1}\PYG{p}{,}\PYG{l+m+mi}{250}\PYG{p}{,} \PYG{n}{base}\PYG{o}{=}\PYG{l+m+mf}{10.0}\PYG{p}{)} \PYG{c+c1}{\PYGZsh{} setting the value of u}
\PYG{n}{w\PYGZus{}u} \PYG{o}{=}\PYG{n}{W}\PYG{p}{(}\PYG{l+m+mi}{1}\PYG{o}{/}\PYG{n}{u\PYGZus{}1}\PYG{p}{)} \PYG{c+c1}{\PYGZsh{} finding W(1/u) : as we use 1/u in the typ curce}

\PYG{c+c1}{\PYGZsh{} Output}
\PYG{n}{dx\PYGZus{}1} \PYG{o}{=} \PYG{p}{\PYGZob{}}\PYG{l+s+s2}{\PYGZdq{}}\PYG{l+s+s2}{1/u}\PYG{l+s+s2}{\PYGZdq{}}\PYG{p}{:}\PYG{n}{i\PYGZus{}u}\PYG{p}{,} \PYG{l+s+s2}{\PYGZdq{}}\PYG{l+s+s2}{W(u)}\PYG{l+s+s2}{\PYGZdq{}}\PYG{p}{:}\PYG{n}{W\PYGZus{}u}\PYG{p}{\PYGZcb{}}\PYG{p}{;} \PYG{n}{dfx\PYGZus{}a} \PYG{o}{=} \PYG{n}{pd}\PYG{o}{.}\PYG{n}{DataFrame}\PYG{p}{(}\PYG{n}{dx\PYGZus{}1}\PYG{p}{)}\PYG{p}{;} \PYG{n}{figs} \PYG{o}{=} \PYG{n}{plt}\PYG{o}{.}\PYG{n}{figure}\PYG{p}{(}\PYG{n}{figsize}\PYG{o}{=}\PYG{p}{(}\PYG{l+m+mi}{9}\PYG{p}{,}\PYG{l+m+mi}{6}\PYG{p}{)}\PYG{p}{)} 
\PYG{n}{plt}\PYG{o}{.}\PYG{n}{loglog}\PYG{p}{(}\PYG{n}{u\PYGZus{}1}\PYG{p}{,} \PYG{n}{w\PYGZus{}u}\PYG{p}{)} \PYG{c+c1}{\PYGZsh{} typ curve}
\PYG{n}{plt}\PYG{o}{.}\PYG{n}{loglog}\PYG{p}{(}\PYG{n}{i\PYGZus{}u}\PYG{p}{,} \PYG{n}{W\PYGZus{}u}\PYG{p}{,} \PYG{l+s+s2}{\PYGZdq{}}\PYG{l+s+s2}{ro}\PYG{l+s+s2}{\PYGZdq{}} \PYG{p}{)} \PYG{c+c1}{\PYGZsh{} your data}
\PYG{n}{plt}\PYG{o}{.}\PYG{n}{title}\PYG{p}{(}\PYG{l+s+s2}{\PYGZdq{}}\PYG{l+s+s2}{The typ curve}\PYG{l+s+s2}{\PYGZdq{}}\PYG{p}{)}\PYG{p}{;} \PYG{n}{plt}\PYG{o}{.}\PYG{n}{ylim}\PYG{p}{(}\PYG{p}{(}\PYG{l+m+mf}{0.1}\PYG{p}{,} \PYG{l+m+mi}{10}\PYG{p}{)}\PYG{p}{)}\PYG{p}{;} \PYG{n}{plt}\PYG{o}{.}\PYG{n}{xlim}\PYG{p}{(}\PYG{l+m+mi}{1}\PYG{p}{,} \PYG{l+m+mf}{1e5}\PYG{p}{)}
\PYG{n}{plt}\PYG{o}{.}\PYG{n}{grid}\PYG{p}{(}\PYG{k+kc}{True}\PYG{p}{,} \PYG{n}{which}\PYG{o}{=}\PYG{l+s+s2}{\PYGZdq{}}\PYG{l+s+s2}{both}\PYG{l+s+s2}{\PYGZdq{}}\PYG{p}{,}\PYG{n}{ls}\PYG{o}{=}\PYG{l+s+s2}{\PYGZdq{}}\PYG{l+s+s2}{\PYGZhy{}}\PYG{l+s+s2}{\PYGZdq{}}\PYG{p}{)}\PYG{p}{;} \PYG{n}{plt}\PYG{o}{.}\PYG{n}{ylabel}\PYG{p}{(}\PYG{l+s+sa}{r}\PYG{l+s+s2}{\PYGZdq{}}\PYG{l+s+s2}{W(u)}\PYG{l+s+s2}{\PYGZdq{}}\PYG{p}{)}\PYG{p}{;}\PYG{n}{plt}\PYG{o}{.}\PYG{n}{xlabel}\PYG{p}{(}\PYG{l+s+sa}{r}\PYG{l+s+s2}{\PYGZdq{}}\PYG{l+s+s2}{1/u}\PYG{l+s+s2}{\PYGZdq{}}\PYG{p}{)}\PYG{p}{;} \PYG{n}{plt}\PYG{o}{.}\PYG{n}{close}\PYG{p}{(}\PYG{p}{)}
\PYG{n}{rx\PYGZus{}2} \PYG{o}{=} \PYG{n}{pn}\PYG{o}{.}\PYG{n}{pane}\PYG{o}{.}\PYG{n}{Matplotlib}\PYG{p}{(}\PYG{n}{figs}\PYG{p}{,} \PYG{n}{dpi}\PYG{o}{=}\PYG{l+m+mi}{150}\PYG{p}{)}\PYG{p}{;} \PYG{n}{pn}\PYG{o}{.}\PYG{n}{Row}\PYG{p}{(}\PYG{n}{dfx\PYGZus{}a}\PYG{p}{,} \PYG{n}{rx\PYGZus{}2}\PYG{p}{)} 
\end{sphinxVerbatim}

\end{sphinxuseclass}\end{sphinxVerbatimInput}
\begin{sphinxVerbatimOutput}

\begin{sphinxuseclass}{cell_output}
\begin{sphinxVerbatim}[commandchars=\\\{\}]
Row
    [0] DataFrame(DataFrame)
    [1] Matplotlib(Figure, dpi=150, height=450, width=675)
\end{sphinxVerbatim}

\end{sphinxuseclass}\end{sphinxVerbatimOutput}

\end{sphinxuseclass}
\end{sphinxuseclass}

\subsection{Homework Problem 11 \sphinxhyphen{} Conservative transport}
\label{\detokenize{content/tutorials/T9/tutorial_09:homework-problem-11-conservative-transport}}
\begin{sphinxuseclass}{cell}
\begin{sphinxuseclass}{tag_hide-input}
\begin{sphinxuseclass}{tag_full-width}\begin{sphinxVerbatimOutput}

\begin{sphinxuseclass}{cell_output}
\begin{sphinxVerbatim}[commandchars=\\\{\}]
Row
    [0] Column
        [0] Markdown(str, style=\PYGZob{}\PYGZsq{}font\PYGZhy{}size\PYGZsq{}: \PYGZsq{}12pt\PYGZsq{}\PYGZcb{}, width=600)
        [1] LaTeX(str, style=\PYGZob{}\PYGZsq{}font\PYGZhy{}size\PYGZsq{}: \PYGZsq{}12pt\PYGZsq{}\PYGZcb{}, width=600)
    [1] Spacer(width=50)
    [2] DataFrame(DataFrame)
\end{sphinxVerbatim}

\end{sphinxuseclass}\end{sphinxVerbatimOutput}

\end{sphinxuseclass}
\end{sphinxuseclass}
\end{sphinxuseclass}

\subsection{Solution Homework Problem 11}
\label{\detokenize{content/tutorials/T9/tutorial_09:solution-homework-problem-11}}
\begin{sphinxuseclass}{cell}\begin{sphinxVerbatimInput}

\begin{sphinxuseclass}{cell_input}
\begin{sphinxVerbatim}[commandchars=\\\{\}]
\PYG{c+c1}{\PYGZsh{} solution 11. a}

\PYG{n}{C\PYGZus{}m} \PYG{o}{=} \PYG{l+m+mi}{55} \PYG{c+c1}{\PYGZsh{} mg/L, injected concentration}

\PYG{c+c1}{\PYGZsh{} calculation}
\PYG{n}{dh11\PYGZus{}rc} \PYG{o}{=} \PYG{n}{dh11\PYGZus{}C}\PYG{o}{/}\PYG{n}{C\PYGZus{}m} \PYG{c+c1}{\PYGZsh{} (\PYGZhy{}), Relative conc. Conc Out/Injected Con }

\PYG{c+c1}{\PYGZsh{}output}
\PYG{n}{dh11\PYGZus{}a} \PYG{o}{=} \PYG{n}{dh11} \PYG{o}{=} \PYG{p}{\PYGZob{}}\PYG{l+s+s2}{\PYGZdq{}}\PYG{l+s+s2}{Time [min]}\PYG{l+s+s2}{\PYGZdq{}}\PYG{p}{:}\PYG{n}{dh11\PYGZus{}t}\PYG{p}{,} \PYG{l+s+s2}{\PYGZdq{}}\PYG{l+s+s2}{Conc. [mg/L]}\PYG{l+s+s2}{\PYGZdq{}}\PYG{p}{:}\PYG{n}{dh11\PYGZus{}C}\PYG{p}{,} \PYG{l+s+s2}{\PYGZdq{}}\PYG{l+s+s2}{Rel. Conc [\PYGZhy{}]}\PYG{l+s+s2}{\PYGZdq{}}\PYG{p}{:}\PYG{n}{dh11\PYGZus{}rc}\PYG{p}{\PYGZcb{}}
\PYG{n}{df11\PYGZus{}a} \PYG{o}{=} \PYG{n}{pd}\PYG{o}{.}\PYG{n}{DataFrame}\PYG{p}{(}\PYG{n}{dh11\PYGZus{}a}\PYG{p}{)}
\PYG{n}{df11\PYGZus{}a} 
\end{sphinxVerbatim}

\end{sphinxuseclass}\end{sphinxVerbatimInput}
\begin{sphinxVerbatimOutput}

\begin{sphinxuseclass}{cell_output}
\begin{sphinxVerbatim}[commandchars=\\\{\}]
    Time [min]  Conc. [mg/L]  Rel. Conc [\PYGZhy{}]
0           15           0.0       0.000000
1           30           0.0       0.000000
2           45           0.0       0.000000
3           60           2.5       0.045455
4           75           5.4       0.098182
5           90          10.6       0.192727
6          105          21.0       0.381818
7          120          29.1       0.529091
8          135          40.8       0.741818
9          150          51.7       0.940000
10         165          55.0       1.000000
11         180          55.0       1.000000
\end{sphinxVerbatim}

\end{sphinxuseclass}\end{sphinxVerbatimOutput}

\end{sphinxuseclass}
\begin{sphinxuseclass}{cell}\begin{sphinxVerbatimInput}

\begin{sphinxuseclass}{cell_input}
\begin{sphinxVerbatim}[commandchars=\\\{\}]
\PYG{c+c1}{\PYGZsh{} Solution 11 b}

\PYG{c+c1}{\PYGZsh{} Plotting}
\PYG{n}{fig} \PYG{o}{=} \PYG{n}{plt}\PYG{o}{.}\PYG{n}{figure}\PYG{p}{(}\PYG{p}{)}
\PYG{n}{plt}\PYG{o}{.}\PYG{n}{plot}\PYG{p}{(}\PYG{n}{dh11\PYGZus{}t}\PYG{p}{,} \PYG{n}{dh11\PYGZus{}rc}\PYG{p}{,} \PYG{l+s+s1}{\PYGZsq{}}\PYG{l+s+s1}{x\PYGZhy{}}\PYG{l+s+s1}{\PYGZsq{}}\PYG{p}{,} \PYG{n}{color} \PYG{o}{=} \PYG{l+s+s2}{\PYGZdq{}}\PYG{l+s+s2}{k}\PYG{l+s+s2}{\PYGZdq{}}\PYG{p}{,} \PYG{n}{label}\PYG{o}{=}\PYG{l+s+s1}{\PYGZsq{}}\PYG{l+s+s1}{ Relative Conc.}\PYG{l+s+s1}{\PYGZsq{}}\PYG{p}{)}\PYG{p}{;}
\PYG{n}{plt}\PYG{o}{.}\PYG{n}{xlabel}\PYG{p}{(}\PYG{l+s+sa}{r}\PYG{l+s+s2}{\PYGZdq{}}\PYG{l+s+s2}{Time, (min)}\PYG{l+s+s2}{\PYGZdq{}}\PYG{p}{)}\PYG{p}{;} \PYG{n}{plt}\PYG{o}{.}\PYG{n}{ylabel}\PYG{p}{(}\PYG{l+s+sa}{r}\PYG{l+s+s2}{\PYGZdq{}}\PYG{l+s+s2}{Relative Conc., (\PYGZhy{})}\PYG{l+s+s2}{\PYGZdq{}}\PYG{p}{)}\PYG{p}{;}
\PYG{n}{plt}\PYG{o}{.}\PYG{n}{grid}\PYG{p}{(}\PYG{p}{)}\PYG{p}{;} \PYG{n}{plt}\PYG{o}{.}\PYG{n}{legend}\PYG{p}{(}\PYG{n}{fontsize}\PYG{o}{=}\PYG{l+m+mi}{11}\PYG{p}{)}\PYG{p}{;} 
\end{sphinxVerbatim}

\end{sphinxuseclass}\end{sphinxVerbatimInput}
\begin{sphinxVerbatimOutput}

\begin{sphinxuseclass}{cell_output}
\noindent\sphinxincludegraphics{{C:/Users/vibhu/GWtextbook/_build/jupyter_execute/tutorial_09_10_0}.png}

\end{sphinxuseclass}\end{sphinxVerbatimOutput}

\end{sphinxuseclass}
\begin{sphinxuseclass}{cell}
\begin{sphinxuseclass}{tag_full-width}\begin{sphinxVerbatimInput}

\begin{sphinxuseclass}{cell_input}
\begin{sphinxVerbatim}[commandchars=\\\{\}]
\PYG{c+c1}{\PYGZsh{}Solution HW 11 c}

\PYG{n}{fig} \PYG{o}{=} \PYG{n}{plt}\PYG{o}{.}\PYG{n}{figure}\PYG{p}{(}\PYG{p}{)}
\PYG{n}{plt}\PYG{o}{.}\PYG{n}{plot}\PYG{p}{(}\PYG{n}{dh11\PYGZus{}t}\PYG{p}{,} \PYG{n}{dh11\PYGZus{}rc}\PYG{p}{,} \PYG{l+s+s1}{\PYGZsq{}}\PYG{l+s+s1}{o\PYGZhy{}}\PYG{l+s+s1}{\PYGZsq{}}\PYG{p}{,} \PYG{n}{color} \PYG{o}{=} \PYG{l+s+s2}{\PYGZdq{}}\PYG{l+s+s2}{r}\PYG{l+s+s2}{\PYGZdq{}}\PYG{p}{,} \PYG{n}{label}\PYG{o}{=}\PYG{l+s+s1}{\PYGZsq{}}\PYG{l+s+s1}{ Relative Conc.}\PYG{l+s+s1}{\PYGZsq{}}\PYG{p}{)}\PYG{p}{;}
\PYG{n}{plt}\PYG{o}{.}\PYG{n}{xlabel}\PYG{p}{(}\PYG{l+s+sa}{r}\PYG{l+s+s2}{\PYGZdq{}}\PYG{l+s+s2}{Time, (min)}\PYG{l+s+s2}{\PYGZdq{}}\PYG{p}{)}\PYG{p}{;} \PYG{n}{plt}\PYG{o}{.}\PYG{n}{ylabel}\PYG{p}{(}\PYG{l+s+sa}{r}\PYG{l+s+s2}{\PYGZdq{}}\PYG{l+s+s2}{Relative Conc., (\PYGZhy{})}\PYG{l+s+s2}{\PYGZdq{}}\PYG{p}{)}\PYG{p}{;}
\PYG{n}{plt}\PYG{o}{.}\PYG{n}{grid}\PYG{p}{(}\PYG{p}{)}\PYG{p}{;} \PYG{n}{plt}\PYG{o}{.}\PYG{n}{legend}\PYG{p}{(}\PYG{n}{fontsize}\PYG{o}{=}\PYG{l+m+mi}{11}\PYG{p}{)}\PYG{p}{;} 
\PYG{n}{plt}\PYG{o}{.}\PYG{n}{annotate}\PYG{p}{(}\PYG{l+s+sa}{r}\PYG{l+s+s1}{\PYGZsq{}}\PYG{l+s+s1}{t\PYGZdl{}\PYGZus{}}\PYG{l+s+si}{\PYGZob{}16\PYGZcb{}}\PYG{l+s+s1}{\PYGZdl{}}\PYG{l+s+s1}{\PYGZsq{}}\PYG{p}{,} \PYG{n}{xy}\PYG{o}{=}\PYG{p}{(}\PYG{l+m+mi}{82}\PYG{p}{,} \PYG{l+m+mf}{0.16}\PYG{p}{)}\PYG{p}{,}  \PYG{n}{xycoords}\PYG{o}{=}\PYG{l+s+s1}{\PYGZsq{}}\PYG{l+s+s1}{data}\PYG{l+s+s1}{\PYGZsq{}}\PYG{p}{,}\PYG{n}{xytext}\PYG{o}{=}\PYG{p}{(}\PYG{l+m+mf}{0.0001}\PYG{p}{,} \PYG{l+m+mf}{0.16}\PYG{p}{)}\PYG{p}{,} \PYG{n}{textcoords}\PYG{o}{=}\PYG{l+s+s1}{\PYGZsq{}}\PYG{l+s+s1}{axes fraction}\PYG{l+s+s1}{\PYGZsq{}}\PYG{p}{,} 
             \PYG{n}{arrowprops}\PYG{o}{=}\PYG{n+nb}{dict}\PYG{p}{(}\PYG{n}{facecolor}\PYG{o}{=}\PYG{l+s+s1}{\PYGZsq{}}\PYG{l+s+s1}{green}\PYG{l+s+s1}{\PYGZsq{}}\PYG{p}{,} \PYG{n}{shrink}\PYG{o}{=}\PYG{l+m+mf}{0.01}\PYG{p}{)}\PYG{p}{,}\PYG{n}{horizontalalignment}\PYG{o}{=}\PYG{l+s+s1}{\PYGZsq{}}\PYG{l+s+s1}{left}\PYG{l+s+s1}{\PYGZsq{}}\PYG{p}{,} \PYG{n}{verticalalignment}\PYG{o}{=}\PYG{l+s+s1}{\PYGZsq{}}\PYG{l+s+s1}{bottom}\PYG{l+s+s1}{\PYGZsq{}}\PYG{p}{,}\PYG{p}{)}
\PYG{n}{plt}\PYG{o}{.}\PYG{n}{annotate}\PYG{p}{(}\PYG{l+s+s1}{\PYGZsq{}}\PYG{l+s+s1}{\PYGZsq{}}\PYG{p}{,} \PYG{n}{xy}\PYG{o}{=}\PYG{p}{(}\PYG{l+m+mi}{82}\PYG{p}{,} \PYG{l+m+mf}{0.0}\PYG{p}{)}\PYG{p}{,}  \PYG{n}{xycoords}\PYG{o}{=}\PYG{l+s+s1}{\PYGZsq{}}\PYG{l+s+s1}{data}\PYG{l+s+s1}{\PYGZsq{}}\PYG{p}{,}\PYG{n}{xytext}\PYG{o}{=}\PYG{p}{(}\PYG{l+m+mf}{0.409}\PYG{p}{,} \PYG{l+m+mf}{0.16}\PYG{p}{)}\PYG{p}{,} \PYG{n}{textcoords}\PYG{o}{=}\PYG{l+s+s1}{\PYGZsq{}}\PYG{l+s+s1}{axes fraction}\PYG{l+s+s1}{\PYGZsq{}}\PYG{p}{,} 
             \PYG{n}{arrowprops}\PYG{o}{=}\PYG{n+nb}{dict}\PYG{p}{(}\PYG{n}{facecolor}\PYG{o}{=}\PYG{l+s+s1}{\PYGZsq{}}\PYG{l+s+s1}{green}\PYG{l+s+s1}{\PYGZsq{}}\PYG{p}{,} \PYG{n}{shrink}\PYG{o}{=}\PYG{l+m+mf}{0.01}\PYG{p}{)}\PYG{p}{,}\PYG{n}{horizontalalignment}\PYG{o}{=}\PYG{l+s+s1}{\PYGZsq{}}\PYG{l+s+s1}{left}\PYG{l+s+s1}{\PYGZsq{}}\PYG{p}{,} \PYG{n}{verticalalignment}\PYG{o}{=}\PYG{l+s+s1}{\PYGZsq{}}\PYG{l+s+s1}{bottom}\PYG{l+s+s1}{\PYGZsq{}}\PYG{p}{,}\PYG{p}{)}
\PYG{n}{plt}\PYG{o}{.}\PYG{n}{annotate}\PYG{p}{(}\PYG{l+s+sa}{r}\PYG{l+s+s1}{\PYGZsq{}}\PYG{l+s+s1}{t\PYGZdl{}\PYGZus{}}\PYG{l+s+si}{\PYGZob{}50\PYGZcb{}}\PYG{l+s+s1}{\PYGZdl{}}\PYG{l+s+s1}{\PYGZsq{}}\PYG{p}{,} \PYG{n}{xy}\PYG{o}{=}\PYG{p}{(}\PYG{l+m+mi}{118}\PYG{p}{,} \PYG{l+m+mf}{0.5}\PYG{p}{)}\PYG{p}{,}  \PYG{n}{xycoords}\PYG{o}{=}\PYG{l+s+s1}{\PYGZsq{}}\PYG{l+s+s1}{data}\PYG{l+s+s1}{\PYGZsq{}}\PYG{p}{,}\PYG{n}{xytext}\PYG{o}{=}\PYG{p}{(}\PYG{l+m+mf}{0.0001}\PYG{p}{,} \PYG{l+m+mf}{0.5}\PYG{p}{)}\PYG{p}{,} \PYG{n}{textcoords}\PYG{o}{=}\PYG{l+s+s1}{\PYGZsq{}}\PYG{l+s+s1}{axes fraction}\PYG{l+s+s1}{\PYGZsq{}}\PYG{p}{,} 
             \PYG{n}{arrowprops}\PYG{o}{=}\PYG{n+nb}{dict}\PYG{p}{(}\PYG{n}{facecolor}\PYG{o}{=}\PYG{l+s+s1}{\PYGZsq{}}\PYG{l+s+s1}{green}\PYG{l+s+s1}{\PYGZsq{}}\PYG{p}{,} \PYG{n}{shrink}\PYG{o}{=}\PYG{l+m+mf}{0.01}\PYG{p}{)}\PYG{p}{,}\PYG{n}{horizontalalignment}\PYG{o}{=}\PYG{l+s+s1}{\PYGZsq{}}\PYG{l+s+s1}{left}\PYG{l+s+s1}{\PYGZsq{}}\PYG{p}{,} \PYG{n}{verticalalignment}\PYG{o}{=}\PYG{l+s+s1}{\PYGZsq{}}\PYG{l+s+s1}{bottom}\PYG{l+s+s1}{\PYGZsq{}}\PYG{p}{,}\PYG{p}{)}
\PYG{n}{plt}\PYG{o}{.}\PYG{n}{annotate}\PYG{p}{(}\PYG{l+s+s1}{\PYGZsq{}}\PYG{l+s+s1}{\PYGZsq{}}\PYG{p}{,} \PYG{n}{xy}\PYG{o}{=}\PYG{p}{(}\PYG{l+m+mi}{118}\PYG{p}{,} \PYG{l+m+mf}{0.001}\PYG{p}{)}\PYG{p}{,}  \PYG{n}{xycoords}\PYG{o}{=}\PYG{l+s+s1}{\PYGZsq{}}\PYG{l+s+s1}{data}\PYG{l+s+s1}{\PYGZsq{}}\PYG{p}{,}\PYG{n}{xytext}\PYG{o}{=}\PYG{p}{(}\PYG{l+m+mf}{0.61}\PYG{p}{,} \PYG{l+m+mf}{0.48}\PYG{p}{)}\PYG{p}{,} \PYG{n}{textcoords}\PYG{o}{=}\PYG{l+s+s1}{\PYGZsq{}}\PYG{l+s+s1}{axes fraction}\PYG{l+s+s1}{\PYGZsq{}}\PYG{p}{,} 
             \PYG{n}{arrowprops}\PYG{o}{=}\PYG{n+nb}{dict}\PYG{p}{(}\PYG{n}{facecolor}\PYG{o}{=}\PYG{l+s+s1}{\PYGZsq{}}\PYG{l+s+s1}{green}\PYG{l+s+s1}{\PYGZsq{}}\PYG{p}{,} \PYG{n}{shrink}\PYG{o}{=}\PYG{l+m+mf}{0.01}\PYG{p}{)}\PYG{p}{,}\PYG{n}{horizontalalignment}\PYG{o}{=}\PYG{l+s+s1}{\PYGZsq{}}\PYG{l+s+s1}{left}\PYG{l+s+s1}{\PYGZsq{}}\PYG{p}{,} \PYG{n}{verticalalignment}\PYG{o}{=}\PYG{l+s+s1}{\PYGZsq{}}\PYG{l+s+s1}{bottom}\PYG{l+s+s1}{\PYGZsq{}}\PYG{p}{,}\PYG{p}{)}
\PYG{n}{plt}\PYG{o}{.}\PYG{n}{annotate}\PYG{p}{(}\PYG{l+s+sa}{r}\PYG{l+s+s1}{\PYGZsq{}}\PYG{l+s+s1}{t\PYGZdl{}\PYGZus{}}\PYG{l+s+si}{\PYGZob{}84\PYGZcb{}}\PYG{l+s+s1}{\PYGZdl{}}\PYG{l+s+s1}{\PYGZsq{}}\PYG{p}{,} \PYG{n}{xy}\PYG{o}{=}\PYG{p}{(}\PYG{l+m+mi}{145}\PYG{p}{,} \PYG{l+m+mf}{0.86}\PYG{p}{)}\PYG{p}{,}  \PYG{n}{xycoords}\PYG{o}{=}\PYG{l+s+s1}{\PYGZsq{}}\PYG{l+s+s1}{data}\PYG{l+s+s1}{\PYGZsq{}}\PYG{p}{,}\PYG{n}{xytext}\PYG{o}{=}\PYG{p}{(}\PYG{l+m+mf}{0.0001}\PYG{p}{,} \PYG{l+m+mf}{0.81}\PYG{p}{)}\PYG{p}{,} \PYG{n}{textcoords}\PYG{o}{=}\PYG{l+s+s1}{\PYGZsq{}}\PYG{l+s+s1}{axes fraction}\PYG{l+s+s1}{\PYGZsq{}}\PYG{p}{,} 
             \PYG{n}{arrowprops}\PYG{o}{=}\PYG{n+nb}{dict}\PYG{p}{(}\PYG{n}{facecolor}\PYG{o}{=}\PYG{l+s+s1}{\PYGZsq{}}\PYG{l+s+s1}{green}\PYG{l+s+s1}{\PYGZsq{}}\PYG{p}{,} \PYG{n}{shrink}\PYG{o}{=}\PYG{l+m+mf}{0.01}\PYG{p}{)}\PYG{p}{,}\PYG{n}{horizontalalignment}\PYG{o}{=}\PYG{l+s+s1}{\PYGZsq{}}\PYG{l+s+s1}{left}\PYG{l+s+s1}{\PYGZsq{}}\PYG{p}{,} \PYG{n}{verticalalignment}\PYG{o}{=}\PYG{l+s+s1}{\PYGZsq{}}\PYG{l+s+s1}{bottom}\PYG{l+s+s1}{\PYGZsq{}}\PYG{p}{,}\PYG{p}{)}
\PYG{n}{plt}\PYG{o}{.}\PYG{n}{annotate}\PYG{p}{(}\PYG{l+s+s1}{\PYGZsq{}}\PYG{l+s+s1}{\PYGZsq{}}\PYG{p}{,} \PYG{n}{xy}\PYG{o}{=}\PYG{p}{(}\PYG{l+m+mi}{145}\PYG{p}{,} \PYG{l+m+mf}{0.001}\PYG{p}{)}\PYG{p}{,}  \PYG{n}{xycoords}\PYG{o}{=}\PYG{l+s+s1}{\PYGZsq{}}\PYG{l+s+s1}{data}\PYG{l+s+s1}{\PYGZsq{}}\PYG{p}{,}\PYG{n}{xytext}\PYG{o}{=}\PYG{p}{(}\PYG{l+m+mf}{0.76}\PYG{p}{,} \PYG{l+m+mf}{0.80}\PYG{p}{)}\PYG{p}{,} \PYG{n}{textcoords}\PYG{o}{=}\PYG{l+s+s1}{\PYGZsq{}}\PYG{l+s+s1}{axes fraction}\PYG{l+s+s1}{\PYGZsq{}}\PYG{p}{,} 
             \PYG{n}{arrowprops}\PYG{o}{=}\PYG{n+nb}{dict}\PYG{p}{(}\PYG{n}{facecolor}\PYG{o}{=}\PYG{l+s+s1}{\PYGZsq{}}\PYG{l+s+s1}{green}\PYG{l+s+s1}{\PYGZsq{}}\PYG{p}{,} \PYG{n}{shrink}\PYG{o}{=}\PYG{l+m+mf}{0.01}\PYG{p}{)}\PYG{p}{,}\PYG{n}{horizontalalignment}\PYG{o}{=}\PYG{l+s+s1}{\PYGZsq{}}\PYG{l+s+s1}{left}\PYG{l+s+s1}{\PYGZsq{}}\PYG{p}{,} \PYG{n}{verticalalignment}\PYG{o}{=}\PYG{l+s+s1}{\PYGZsq{}}\PYG{l+s+s1}{bottom}\PYG{l+s+s1}{\PYGZsq{}}\PYG{p}{,}\PYG{p}{)}
\PYG{n}{plt}\PYG{o}{.}\PYG{n}{close}\PYG{p}{(}\PYG{p}{)} \PYG{c+c1}{\PYGZsh{} otherwise we have 2 figure}
\PYG{n}{r6\PYGZus{}8} \PYG{o}{=} \PYG{n}{pn}\PYG{o}{.}\PYG{n}{pane}\PYG{o}{.}\PYG{n}{Matplotlib}\PYG{p}{(}\PYG{n}{fig}\PYG{p}{,} \PYG{n}{dpi}\PYG{o}{=}\PYG{l+m+mi}{150}\PYG{p}{)}

\PYG{n}{r6\PYGZus{}9} \PYG{o}{=} \PYG{n}{pn}\PYG{o}{.}\PYG{n}{pane}\PYG{o}{.}\PYG{n}{LaTeX}\PYG{p}{(}\PYG{l+s+sa}{r}\PYG{l+s+s2}{\PYGZdq{}\PYGZdq{}\PYGZdq{}}\PYG{l+s+s2}{ }
\PYG{l+s+s2}{From the figure:\PYGZlt{}br\PYGZgt{}}
\PYG{l+s+s2}{\PYGZdl{}t\PYGZus{}}\PYG{l+s+si}{\PYGZob{}16\PYGZcb{}}\PYG{l+s+s2}{\PYGZbs{}}\PYG{l+s+s2}{approx 80\PYGZdl{}\PYGZlt{}br\PYGZgt{}}
\PYG{l+s+s2}{\PYGZdl{}t\PYGZus{}}\PYG{l+s+si}{\PYGZob{}50\PYGZcb{}}\PYG{l+s+s2}{\PYGZbs{}}\PYG{l+s+s2}{approx 120\PYGZdl{}\PYGZlt{}br\PYGZgt{}}
\PYG{l+s+s2}{\PYGZdl{}t\PYGZus{}}\PYG{l+s+si}{\PYGZob{}84\PYGZcb{}}\PYG{l+s+s2}{\PYGZbs{}}\PYG{l+s+s2}{approx 145\PYGZdl{}\PYGZlt{}br\PYGZgt{}}
\PYG{l+s+s2}{                     }
\PYG{l+s+s2}{\PYGZdq{}\PYGZdq{}\PYGZdq{}}\PYG{p}{,}\PYG{n}{width} \PYG{o}{=} \PYG{l+m+mi}{300}\PYG{p}{,} \PYG{n}{style}\PYG{o}{=}\PYG{p}{\PYGZob{}}\PYG{l+s+s1}{\PYGZsq{}}\PYG{l+s+s1}{font\PYGZhy{}size}\PYG{l+s+s1}{\PYGZsq{}}\PYG{p}{:} \PYG{l+s+s1}{\PYGZsq{}}\PYG{l+s+s1}{13pt}\PYG{l+s+s1}{\PYGZsq{}}\PYG{p}{\PYGZcb{}}\PYG{p}{)}

\PYG{n}{r6\PYGZus{}10} \PYG{o}{=} \PYG{n}{pn}\PYG{o}{.}\PYG{n}{Column}\PYG{p}{(}\PYG{n}{r6\PYGZus{}9}\PYG{p}{)}

\PYG{n}{pn}\PYG{o}{.}\PYG{n}{Row}\PYG{p}{(}\PYG{n}{r6\PYGZus{}10}\PYG{p}{,} \PYG{n}{r6\PYGZus{}8}\PYG{p}{)} 
\end{sphinxVerbatim}

\end{sphinxuseclass}\end{sphinxVerbatimInput}
\begin{sphinxVerbatimOutput}

\begin{sphinxuseclass}{cell_output}
\begin{sphinxVerbatim}[commandchars=\\\{\}]
Row
    [0] Column
        [0] LaTeX(str, style=\PYGZob{}\PYGZsq{}font\PYGZhy{}size\PYGZsq{}: \PYGZsq{}13pt\PYGZsq{}\PYGZcb{}, width=300)
    [1] Matplotlib(Figure, dpi=150, height=300, width=450)
\end{sphinxVerbatim}

\end{sphinxuseclass}\end{sphinxVerbatimOutput}

\end{sphinxuseclass}
\end{sphinxuseclass}
\begin{sphinxuseclass}{cell}\begin{sphinxVerbatimInput}

\begin{sphinxuseclass}{cell_input}
\begin{sphinxVerbatim}[commandchars=\\\{\}]
\PYG{c+c1}{\PYGZsh{}Solution HW\PYGZhy{}11 d }

\PYG{c+c1}{\PYGZsh{}Given}
\PYG{n}{Q} \PYG{o}{=} \PYG{l+m+mi}{10} \PYG{c+c1}{\PYGZsh{} mL/min, discharge in column}
\PYG{n}{dc} \PYG{o}{=} \PYG{l+m+mf}{7.5} \PYG{c+c1}{\PYGZsh{} cm, diameter of column}
\PYG{n}{Lc} \PYG{o}{=} \PYG{l+m+mi}{85} \PYG{c+c1}{\PYGZsh{} cm, length of column}
\PYG{n}{t\PYGZus{}50} \PYG{o}{=} \PYG{l+m+mi}{120} \PYG{c+c1}{\PYGZsh{} min, obtained from 17c}

\PYG{c+c1}{\PYGZsh{} Calculation}
\PYG{n}{Vc} \PYG{o}{=} \PYG{n}{np}\PYG{o}{.}\PYG{n}{pi}\PYG{o}{*}\PYG{p}{(}\PYG{n}{dc}\PYG{o}{/}\PYG{l+m+mi}{2}\PYG{p}{)}\PYG{o}{*}\PYG{o}{*}\PYG{l+m+mi}{2}\PYG{o}{*}\PYG{n}{Lc} \PYG{c+c1}{\PYGZsh{} cm\PYGZca{}3, Volume of column pi*d\PYGZca{}2/4* h\PYGZhy{}}
\PYG{n}{n\PYGZus{}ef} \PYG{o}{=} \PYG{n}{Q}\PYG{o}{*}\PYG{n}{t\PYGZus{}50}\PYG{o}{/}\PYG{n}{Vc} \PYG{c+c1}{\PYGZsh{} (\PYGZhy{}), effective porosity from given formula}

\PYG{c+c1}{\PYGZsh{}output}
\PYG{n+nb}{print}\PYG{p}{(}\PYG{l+s+s2}{\PYGZdq{}}\PYG{l+s+s2}{The effective porosity in the column is }\PYG{l+s+si}{\PYGZob{}0:1.2f\PYGZcb{}}\PYG{l+s+s2}{\PYGZdq{}}\PYG{o}{.}\PYG{n}{format}\PYG{p}{(}\PYG{n}{n\PYGZus{}ef}\PYG{p}{)}\PYG{p}{)}
\end{sphinxVerbatim}

\end{sphinxuseclass}\end{sphinxVerbatimInput}
\begin{sphinxVerbatimOutput}

\begin{sphinxuseclass}{cell_output}
\begin{sphinxVerbatim}[commandchars=\\\{\}]
The effective porosity in the column is 0.32
\end{sphinxVerbatim}

\end{sphinxuseclass}\end{sphinxVerbatimOutput}

\end{sphinxuseclass}
\begin{sphinxuseclass}{cell}\begin{sphinxVerbatimInput}

\begin{sphinxuseclass}{cell_input}
\begin{sphinxVerbatim}[commandchars=\\\{\}]
\PYG{c+c1}{\PYGZsh{}Solution HW 11 e}

\PYG{c+c1}{\PYGZsh{}Given}
\PYG{n}{t\PYGZus{}16} \PYG{o}{=} \PYG{l+m+mi}{80} \PYG{c+c1}{\PYGZsh{} min, obtained from 17c}
\PYG{n}{t\PYGZus{}84} \PYG{o}{=} \PYG{l+m+mi}{145} \PYG{c+c1}{\PYGZsh{} min, obtained from 17c}
\PYG{n}{Lc} \PYG{o}{=} \PYG{l+m+mi}{85} \PYG{c+c1}{\PYGZsh{} cm, length of column}

\PYG{c+c1}{\PYGZsh{} Calculation}
\PYG{n}{alpha} \PYG{o}{=} \PYG{n}{Lc}\PYG{o}{/}\PYG{l+m+mi}{8}\PYG{o}{*}\PYG{p}{(}\PYG{p}{(}\PYG{n}{t\PYGZus{}84}\PYG{o}{\PYGZhy{}}\PYG{n}{t\PYGZus{}16}\PYG{p}{)}\PYG{o}{/}\PYG{n}{t\PYGZus{}50}\PYG{p}{)}\PYG{o}{*}\PYG{o}{*}\PYG{l+m+mi}{2} 

\PYG{c+c1}{\PYGZsh{}output}
\PYG{n+nb}{print}\PYG{p}{(}\PYG{l+s+s2}{\PYGZdq{}}\PYG{l+s+s2}{The required dispersivity in the column is }\PYG{l+s+si}{\PYGZob{}0:1.2f\PYGZcb{}}\PYG{l+s+s2}{\PYGZdq{}}\PYG{o}{.}\PYG{n}{format}\PYG{p}{(}\PYG{n}{alpha}\PYG{p}{)}\PYG{p}{,} \PYG{l+s+s2}{\PYGZdq{}}\PYG{l+s+s2}{m}\PYG{l+s+s2}{\PYGZdq{}}\PYG{p}{)}
\end{sphinxVerbatim}

\end{sphinxuseclass}\end{sphinxVerbatimInput}
\begin{sphinxVerbatimOutput}

\begin{sphinxuseclass}{cell_output}
\begin{sphinxVerbatim}[commandchars=\\\{\}]
The required dispersivity in the column is 3.12 m
\end{sphinxVerbatim}

\end{sphinxuseclass}\end{sphinxVerbatimOutput}

\end{sphinxuseclass}
\begin{sphinxuseclass}{cell}
\begin{sphinxuseclass}{tag_hide-input}\begin{sphinxVerbatimOutput}

\begin{sphinxuseclass}{cell_output}
\noindent\sphinxincludegraphics{{C:/Users/vibhu/GWtextbook/_build/jupyter_execute/tutorial_09_14_0}.png}

\end{sphinxuseclass}\end{sphinxVerbatimOutput}

\end{sphinxuseclass}
\end{sphinxuseclass}

\subsection{Tutorial Problem 22 \sphinxhyphen{} Conservative transport (additional problem)}
\label{\detokenize{content/tutorials/T9/tutorial_09:tutorial-problem-22-conservative-transport-additional-problem}}
\begin{sphinxuseclass}{cell}
\begin{sphinxuseclass}{tag_hide-input}
\begin{sphinxuseclass}{tag_full-width}\begin{sphinxVerbatimOutput}

\begin{sphinxuseclass}{cell_output}
\begin{sphinxVerbatim}[commandchars=\\\{\}]
Row
    [0] Column
        [0] LaTeX(str, style=\PYGZob{}\PYGZsq{}font\PYGZhy{}size\PYGZsq{}: \PYGZsq{}12pt\PYGZsq{}\PYGZcb{}, width=600)
        [1] LaTeX(str, style=\PYGZob{}\PYGZsq{}font\PYGZhy{}size\PYGZsq{}: \PYGZsq{}12pt\PYGZsq{}\PYGZcb{}, width=600)
    [1] PNG(str, width=150)
\end{sphinxVerbatim}

\end{sphinxuseclass}\end{sphinxVerbatimOutput}

\end{sphinxuseclass}
\end{sphinxuseclass}
\end{sphinxuseclass}
\begin{sphinxuseclass}{cell}
\begin{sphinxuseclass}{tag_hide-input}\begin{sphinxVerbatimOutput}

\begin{sphinxuseclass}{cell_output}
\begin{sphinxVerbatim}[commandchars=\\\{\}]
Column
    [0] PNG(str, width=600)
    [1] LaTeX(str, style=\PYGZob{}\PYGZsq{}font\PYGZhy{}size\PYGZsq{}: \PYGZsq{}12pt\PYGZsq{}\PYGZcb{}, width=800)
\end{sphinxVerbatim}

\end{sphinxuseclass}\end{sphinxVerbatimOutput}

\end{sphinxuseclass}
\end{sphinxuseclass}
\begin{sphinxuseclass}{cell}\begin{sphinxVerbatimInput}

\begin{sphinxuseclass}{cell_input}
\begin{sphinxVerbatim}[commandchars=\\\{\}]
\PYG{c+c1}{\PYGZsh{} Solution of Problem 22, STEP 1}
\PYG{c+c1}{\PYGZsh{}Given}
\PYG{n}{x\PYGZus{}o} \PYG{o}{=} \PYG{l+m+mi}{0} \PYG{c+c1}{\PYGZsh{} m, starting point along x\PYGZhy{}direction}
\PYG{n}{y\PYGZus{}o} \PYG{o}{=} \PYG{l+m+mi}{250} \PYG{c+c1}{\PYGZsh{} m, starting point along y\PYGZhy{}direction}
\PYG{n}{v\PYGZus{}x} \PYG{o}{=} \PYG{l+m+mi}{2}\PYG{o}{*}\PYG{l+m+mf}{1e\PYGZhy{}5} \PYG{c+c1}{\PYGZsh{} m/s Groundwater velocity}
\PYG{n}{t} \PYG{o}{=} \PYG{l+m+mi}{5} \PYG{c+c1}{\PYGZsh{} a, time in year}
\PYG{n}{R} \PYG{o}{=} \PYG{l+m+mi}{1}\PYG{c+c1}{\PYGZsh{} (\PYGZhy{}), retardation factor}

\PYG{c+c1}{\PYGZsh{}calculate}
\PYG{n}{t\PYGZus{}s} \PYG{o}{=} \PYG{n}{t}\PYG{o}{*}\PYG{l+m+mi}{365}\PYG{o}{*}\PYG{l+m+mi}{24}\PYG{o}{*}\PYG{l+m+mi}{3600} \PYG{c+c1}{\PYGZsh{} s, time unit conversion}
\PYG{n}{x\PYGZus{}max} \PYG{o}{=} \PYG{n}{x\PYGZus{}o} \PYG{o}{+} \PYG{n}{v\PYGZus{}x}\PYG{o}{*}\PYG{n}{t\PYGZus{}s}\PYG{o}{/}\PYG{n}{R}
\PYG{n}{y\PYGZus{}max} \PYG{o}{=} \PYG{n}{y\PYGZus{}o}

\PYG{c+c1}{\PYGZsh{}output}
\PYG{n+nb}{print}\PYG{p}{(}\PYG{l+s+s2}{\PYGZdq{}}\PYG{l+s+s2}{The x\PYGZus{}max is located at:}\PYG{l+s+si}{\PYGZob{}0:1.2f\PYGZcb{}}\PYG{l+s+s2}{\PYGZdq{}}\PYG{o}{.}\PYG{n}{format}\PYG{p}{(}\PYG{n}{x\PYGZus{}max}\PYG{p}{)}\PYG{p}{,} \PYG{l+s+s2}{\PYGZdq{}}\PYG{l+s+s2}{m }\PYG{l+s+se}{\PYGZbs{}n}\PYG{l+s+s2}{\PYGZdq{}} \PYG{p}{)}
\PYG{n+nb}{print}\PYG{p}{(}\PYG{l+s+s2}{\PYGZdq{}}\PYG{l+s+s2}{The y\PYGZus{}max is located at:}\PYG{l+s+si}{\PYGZob{}0:1.2f\PYGZcb{}}\PYG{l+s+s2}{\PYGZdq{}}\PYG{o}{.}\PYG{n}{format}\PYG{p}{(}\PYG{n}{y\PYGZus{}max}\PYG{p}{)}\PYG{p}{,} \PYG{l+s+s2}{\PYGZdq{}}\PYG{l+s+s2}{m}\PYG{l+s+s2}{\PYGZdq{}} \PYG{p}{)}
\end{sphinxVerbatim}

\end{sphinxuseclass}\end{sphinxVerbatimInput}
\begin{sphinxVerbatimOutput}

\begin{sphinxuseclass}{cell_output}
\begin{sphinxVerbatim}[commandchars=\\\{\}]
The x\PYGZus{}max is located at:3153.60 m 

The y\PYGZus{}max is located at:250.00 m
\end{sphinxVerbatim}

\end{sphinxuseclass}\end{sphinxVerbatimOutput}

\end{sphinxuseclass}
\begin{sphinxuseclass}{cell}\begin{sphinxVerbatimInput}

\begin{sphinxuseclass}{cell_input}
\begin{sphinxVerbatim}[commandchars=\\\{\}]
\PYG{c+c1}{\PYGZsh{} Solution of Problem 22, STEP 2}
\PYG{c+c1}{\PYGZsh{} Given }
\PYG{n}{M} \PYG{o}{=} \PYG{l+m+mi}{985} \PYG{c+c1}{\PYGZsh{} kg, mass }
\PYG{n}{n\PYGZus{}ef} \PYG{o}{=} \PYG{l+m+mf}{0.2} \PYG{c+c1}{\PYGZsh{} (\PYGZhy{}), effective porosity}
\PYG{n}{m} \PYG{o}{=} \PYG{l+m+mi}{10} \PYG{c+c1}{\PYGZsh{} m, aquifer thickness}
\PYG{n}{a\PYGZus{}L} \PYG{o}{=} \PYG{l+m+mf}{0.5} \PYG{c+c1}{\PYGZsh{} m, longitudinal dispersivity }
\PYG{n}{a\PYGZus{}T} \PYG{o}{=} \PYG{l+m+mf}{0.2} \PYG{c+c1}{\PYGZsh{} m, Transverse dispersivity}
\PYG{n}{L\PYGZus{}a} \PYG{o}{=} \PYG{l+m+mi}{0} \PYG{c+c1}{\PYGZsh{}  (\PYGZhy{}), degradation rate, Lambda}


\PYG{c+c1}{\PYGZsh{} Compute}
\PYG{n}{C\PYGZus{}max} \PYG{o}{=} \PYG{n}{M}\PYG{o}{/}\PYG{p}{(}\PYG{l+m+mi}{4}\PYG{o}{*}\PYG{n}{np}\PYG{o}{.}\PYG{n}{pi}\PYG{o}{*} \PYG{n}{n\PYGZus{}ef}\PYG{o}{*}\PYG{n}{m}\PYG{o}{*} \PYG{n}{np}\PYG{o}{.}\PYG{n}{sqrt}\PYG{p}{(}\PYG{n}{a\PYGZus{}L}\PYG{o}{*}\PYG{n}{a\PYGZus{}T}\PYG{p}{)}\PYG{o}{*}\PYG{n}{v\PYGZus{}x}\PYG{o}{*}\PYG{n}{t\PYGZus{}s}\PYG{p}{)}\PYG{o}{*}\PYG{n}{np}\PYG{o}{.}\PYG{n}{exp}\PYG{p}{(}\PYG{o}{\PYGZhy{}}\PYG{n}{L\PYGZus{}a}\PYG{o}{*}\PYG{n}{t\PYGZus{}s}\PYG{o}{/}\PYG{n}{R}\PYG{p}{)}

\PYG{n+nb}{print}\PYG{p}{(}\PYG{l+s+s2}{\PYGZdq{}}\PYG{l+s+s2}{The C\PYGZus{}max is: }\PYG{l+s+si}{\PYGZob{}0:1.2e\PYGZcb{}}\PYG{l+s+s2}{\PYGZdq{}}\PYG{o}{.}\PYG{n}{format}\PYG{p}{(} \PYG{n}{C\PYGZus{}max}\PYG{p}{)}\PYG{p}{,} \PYG{l+s+s2}{\PYGZdq{}}\PYG{l+s+s2}{Kg/m}\PYG{l+s+se}{\PYGZbs{}u00b3}\PYG{l+s+s2}{ }\PYG{l+s+se}{\PYGZbs{}n}\PYG{l+s+s2}{\PYGZdq{}} \PYG{p}{)} 
\PYG{n+nb}{print}\PYG{p}{(}\PYG{l+s+s2}{\PYGZdq{}}\PYG{l+s+s2}{The C\PYGZus{}max is: }\PYG{l+s+si}{\PYGZob{}0:1.2f\PYGZcb{}}\PYG{l+s+s2}{\PYGZdq{}}\PYG{o}{.}\PYG{n}{format}\PYG{p}{(}\PYG{n}{C\PYGZus{}max}\PYG{o}{*}\PYG{l+m+mi}{1000}\PYG{p}{)}\PYG{p}{,} \PYG{l+s+s2}{\PYGZdq{}}\PYG{l+s+s2}{mg/L}\PYG{l+s+s2}{\PYGZdq{}} \PYG{p}{)} 
\end{sphinxVerbatim}

\end{sphinxuseclass}\end{sphinxVerbatimInput}
\begin{sphinxVerbatimOutput}

\begin{sphinxuseclass}{cell_output}
\begin{sphinxVerbatim}[commandchars=\\\{\}]
The C\PYGZus{}max is: 3.93e\PYGZhy{}02 Kg/m³ 

The C\PYGZus{}max is: 39.30 mg/L
\end{sphinxVerbatim}

\end{sphinxuseclass}\end{sphinxVerbatimOutput}

\end{sphinxuseclass}
\begin{sphinxuseclass}{cell}\begin{sphinxVerbatimInput}

\begin{sphinxuseclass}{cell_input}
\begin{sphinxVerbatim}[commandchars=\\\{\}]
\PYG{c+c1}{\PYGZsh{} Solution of Problem 26, STEP 3 and Step 4}

\PYG{c+c1}{\PYGZsh{}Given}
\PYG{n}{C\PYGZus{}ast} \PYG{o}{=} \PYG{l+m+mf}{4.43} \PYG{c+c1}{\PYGZsh{} mg/L concentration whose location is to be found}
\PYG{n}{C\PYGZus{}maxf} \PYG{o}{=} \PYG{n}{C\PYGZus{}max}\PYG{o}{*}\PYG{l+m+mi}{1000} \PYG{c+c1}{\PYGZsh{} mg/L converting unit of C\PYGZus{}max from Kg/m to mg/L }

\PYG{c+c1}{\PYGZsh{} Compute f}
\PYG{n}{f} \PYG{o}{=} \PYG{n}{C\PYGZus{}ast}\PYG{o}{/}\PYG{n}{C\PYGZus{}maxf}

\PYG{c+c1}{\PYGZsh{} Solution Step 4}

\PYG{c+c1}{\PYGZsh{} compute a and b}
\PYG{n}{a} \PYG{o}{=} \PYG{n}{np}\PYG{o}{.}\PYG{n}{sqrt}\PYG{p}{(}\PYG{o}{\PYGZhy{}}\PYG{l+m+mi}{4}\PYG{o}{*}\PYG{n}{np}\PYG{o}{.}\PYG{n}{log}\PYG{p}{(}\PYG{n}{f}\PYG{p}{)}\PYG{o}{*}\PYG{n}{a\PYGZus{}L}\PYG{o}{*}\PYG{n}{v\PYGZus{}x}\PYG{o}{*}\PYG{n}{t\PYGZus{}s}\PYG{o}{/}\PYG{n}{R}\PYG{p}{)}
\PYG{n}{b} \PYG{o}{=} \PYG{n}{np}\PYG{o}{.}\PYG{n}{sqrt}\PYG{p}{(}\PYG{n}{a\PYGZus{}T}\PYG{o}{/}\PYG{n}{a\PYGZus{}L}\PYG{p}{)}\PYG{o}{*}\PYG{n}{a} 

\PYG{c+c1}{\PYGZsh{}Output}
\PYG{n+nb}{print}\PYG{p}{(}\PYG{l+s+s2}{\PYGZdq{}}\PYG{l+s+s2}{The f is: }\PYG{l+s+si}{\PYGZob{}0:1.4f\PYGZcb{}}\PYG{l+s+s2}{\PYGZdq{}}\PYG{o}{.}\PYG{n}{format}\PYG{p}{(}\PYG{n}{f}\PYG{p}{)} \PYG{p}{)} 
\PYG{n+nb}{print}\PYG{p}{(}\PYG{l+s+s2}{\PYGZdq{}}\PYG{l+s+s2}{The a is: }\PYG{l+s+si}{\PYGZob{}0:1.2f\PYGZcb{}}\PYG{l+s+s2}{\PYGZdq{}}\PYG{o}{.}\PYG{n}{format}\PYG{p}{(}\PYG{n}{a}\PYG{p}{)}\PYG{p}{,} \PYG{l+s+s2}{\PYGZdq{}}\PYG{l+s+s2}{m}\PYG{l+s+s2}{\PYGZdq{}}\PYG{p}{)} 
\PYG{n+nb}{print}\PYG{p}{(}\PYG{l+s+s2}{\PYGZdq{}}\PYG{l+s+s2}{The b is: }\PYG{l+s+si}{\PYGZob{}0:1.2f\PYGZcb{}}\PYG{l+s+s2}{\PYGZdq{}}\PYG{o}{.}\PYG{n}{format}\PYG{p}{(}\PYG{n}{b}\PYG{p}{)}\PYG{p}{,} \PYG{l+s+s2}{\PYGZdq{}}\PYG{l+s+s2}{m}\PYG{l+s+s2}{\PYGZdq{}}\PYG{p}{)} 
\end{sphinxVerbatim}

\end{sphinxuseclass}\end{sphinxVerbatimInput}
\begin{sphinxVerbatimOutput}

\begin{sphinxuseclass}{cell_output}
\begin{sphinxVerbatim}[commandchars=\\\{\}]
The f is: 0.1127
The a is: 117.33 m
The b is: 74.21 m
\end{sphinxVerbatim}

\end{sphinxuseclass}\end{sphinxVerbatimOutput}

\end{sphinxuseclass}

\subsection{Tutorial Problem 23}
\label{\detokenize{content/tutorials/T9/tutorial_09:tutorial-problem-23}}
\begin{sphinxuseclass}{cell}
\begin{sphinxuseclass}{tag_hide-input}
\begin{sphinxuseclass}{tag_full-width}\begin{sphinxVerbatimOutput}

\begin{sphinxuseclass}{cell_output}
\begin{sphinxVerbatim}[commandchars=\\\{\}]
Row
    [0] Column
        [0] LaTeX(str, style=\PYGZob{}\PYGZsq{}font\PYGZhy{}size\PYGZsq{}: \PYGZsq{}12pt\PYGZsq{}\PYGZcb{}, width=600)
        [1] LaTeX(str, style=\PYGZob{}\PYGZsq{}font\PYGZhy{}size\PYGZsq{}: \PYGZsq{}12pt\PYGZsq{}\PYGZcb{}, width=600)
    [1] Spacer(width=50)
    [2] Column
        [0] DataFrame(DataFrame)
        [1] LaTeX(str, style=\PYGZob{}\PYGZsq{}font\PYGZhy{}size\PYGZsq{}: \PYGZsq{}13pt\PYGZsq{}\PYGZcb{})
\end{sphinxVerbatim}

\end{sphinxuseclass}\end{sphinxVerbatimOutput}

\end{sphinxuseclass}
\end{sphinxuseclass}
\end{sphinxuseclass}

\subsection{Solution of Problem 23}
\label{\detokenize{content/tutorials/T9/tutorial_09:solution-of-problem-23}}
\begin{sphinxuseclass}{cell}\begin{sphinxVerbatimInput}

\begin{sphinxuseclass}{cell_input}
\begin{sphinxVerbatim}[commandchars=\\\{\}]
\PYG{c+c1}{\PYGZsh{} solution problem 23 a}
\PYG{c+c1}{\PYGZsh{}Given}
\PYG{n}{Vw} \PYG{o}{=} \PYG{l+m+mi}{25}\PYG{o}{/}\PYG{l+m+mi}{1000} \PYG{c+c1}{\PYGZsh{} L, volume of water in L}
\PYG{n}{Ms} \PYG{o}{=} \PYG{l+m+mi}{10} \PYG{c+c1}{\PYGZsh{} g, mass of Cr(IV)}

\PYG{c+c1}{\PYGZsh{} calculation}
\PYG{n}{d23\PYGZus{}Ca} \PYG{o}{=} \PYG{n}{Vw}\PYG{o}{/}\PYG{n}{Ms}\PYG{o}{*}\PYG{p}{(}\PYG{n}{d23\PYGZus{}Co}\PYG{o}{\PYGZhy{}}\PYG{n}{d23\PYGZus{}Ceq}\PYG{p}{)} \PYG{c+c1}{\PYGZsh{} Ca  = Vw/Ms* (Co\PYGZhy{}Ceq)}

\PYG{c+c1}{\PYGZsh{}output}
\PYG{n}{d23\PYGZus{}a} \PYG{o}{=} \PYG{p}{\PYGZob{}}\PYG{l+s+s2}{\PYGZdq{}}\PYG{l+s+s2}{Co [mg/L]}\PYG{l+s+s2}{\PYGZdq{}}\PYG{p}{:}\PYG{n}{d23\PYGZus{}Co}\PYG{p}{,} \PYG{l+s+s2}{\PYGZdq{}}\PYG{l+s+s2}{Ceq [mg/L]}\PYG{l+s+s2}{\PYGZdq{}}\PYG{p}{:}\PYG{n}{d23\PYGZus{}Ceq}\PYG{p}{,} \PYG{l+s+s2}{\PYGZdq{}}\PYG{l+s+s2}{Ca [mg/g]}\PYG{l+s+s2}{\PYGZdq{}}\PYG{p}{:}\PYG{n}{d23\PYGZus{}Ca}\PYG{p}{\PYGZcb{}}
\PYG{n}{df23\PYGZus{}a} \PYG{o}{=} \PYG{n}{pd}\PYG{o}{.}\PYG{n}{DataFrame}\PYG{p}{(}\PYG{n}{d23\PYGZus{}a}\PYG{p}{)}
\PYG{n}{df23\PYGZus{}a} 
\end{sphinxVerbatim}

\end{sphinxuseclass}\end{sphinxVerbatimInput}
\begin{sphinxVerbatimOutput}

\begin{sphinxuseclass}{cell_output}
\begin{sphinxVerbatim}[commandchars=\\\{\}]
   Co [mg/L]  Ceq [mg/L]  Ca [mg/g]
0         50          15     0.0875
1         75          28     0.1175
2        100          40     0.1500
3        150          61     0.2225
4        200          82     0.2950
5        250         104     0.3650
\end{sphinxVerbatim}

\end{sphinxuseclass}\end{sphinxVerbatimOutput}

\end{sphinxuseclass}
\begin{sphinxuseclass}{cell}
\begin{sphinxuseclass}{tag_full-width}\begin{sphinxVerbatimInput}

\begin{sphinxuseclass}{cell_input}
\begin{sphinxVerbatim}[commandchars=\\\{\}]
\PYG{c+c1}{\PYGZsh{} Solution problem 23b}

\PYG{n}{r23\PYGZus{}7} \PYG{o}{=} \PYG{n}{pn}\PYG{o}{.}\PYG{n}{pane}\PYG{o}{.}\PYG{n}{Markdown}\PYG{p}{(}\PYG{l+s+s2}{\PYGZdq{}\PYGZdq{}\PYGZdq{}}
\PYG{l+s+s2}{The linear isotherm is the regression line through the origin of the \PYGZus{}C\PYGZlt{}sub\PYGZgt{}a\PYGZlt{}/sub\PYGZgt{}\PYGZus{} vs.\PYGZus{}C\PYGZlt{}sub\PYGZgt{}eq\PYGZlt{}/sub\PYGZgt{}\PYGZus{} plot.}
\PYG{l+s+s2}{Its slope is the distribution coefficient \PYGZus{}K\PYGZlt{}sub\PYGZgt{}d\PYGZlt{}/sub\PYGZgt{}\PYGZus{}\PYGZlt{}br\PYGZgt{}\PYGZlt{}br\PYGZgt{}}
\PYG{l+s+s2}{***Here:***\PYGZlt{}br\PYGZgt{}\PYGZlt{}br\PYGZgt{}}
\PYG{l+s+s2}{K\PYGZlt{}sub\PYGZgt{}d\PYGZlt{}/sub\PYGZgt{} = 3.19E\PYGZhy{}03 L/ g\PYGZlt{}br\PYGZgt{}\PYGZlt{}br\PYGZgt{}}
\PYG{l+s+s2}{K\PYGZlt{}sub\PYGZgt{}d\PYGZlt{}/sub\PYGZgt{} = 3.19 cm\PYGZlt{}sup\PYGZgt{}3\PYGZlt{}/sup\PYGZgt{}/ g}
\PYG{l+s+s2}{\PYGZdq{}\PYGZdq{}\PYGZdq{}}\PYG{p}{,}\PYG{n}{width} \PYG{o}{=} \PYG{l+m+mi}{400}\PYG{p}{,} \PYG{n}{style}\PYG{o}{=}\PYG{p}{\PYGZob{}}\PYG{l+s+s1}{\PYGZsq{}}\PYG{l+s+s1}{font\PYGZhy{}size}\PYG{l+s+s1}{\PYGZsq{}}\PYG{p}{:} \PYG{l+s+s1}{\PYGZsq{}}\PYG{l+s+s1}{13pt}\PYG{l+s+s1}{\PYGZsq{}}\PYG{p}{\PYGZcb{}}\PYG{p}{)}

\PYG{c+c1}{\PYGZsh{} Linear fit}
\PYG{n}{slope}\PYG{p}{,} \PYG{n}{intercept}\PYG{p}{,} \PYG{n}{r\PYGZus{}value}\PYG{p}{,} \PYG{n}{p\PYGZus{}value}\PYG{p}{,} \PYG{n}{std\PYGZus{}err} \PYG{o}{=} \PYG{n}{stats}\PYG{o}{.}\PYG{n}{linregress}\PYG{p}{(}\PYG{n}{d23\PYGZus{}Ceq}\PYG{p}{,} \PYG{n}{d23\PYGZus{}Ca}\PYG{p}{)} \PYG{c+c1}{\PYGZsh{} linear regression}

\PYG{c+c1}{\PYGZsh{}output}
\PYG{n}{fig} \PYG{o}{=} \PYG{n}{plt}\PYG{o}{.}\PYG{n}{figure}\PYG{p}{(}\PYG{p}{)}
\PYG{n}{plt}\PYG{o}{.}\PYG{n}{plot}\PYG{p}{(}\PYG{n}{d23\PYGZus{}Ceq}\PYG{p}{,} \PYG{n}{d23\PYGZus{}Ca}\PYG{p}{,} \PYG{l+s+s1}{\PYGZsq{}}\PYG{l+s+s1}{o}\PYG{l+s+s1}{\PYGZsq{}}\PYG{p}{,} \PYG{n}{label}\PYG{o}{=}\PYG{l+s+s1}{\PYGZsq{}}\PYG{l+s+s1}{ provided data}\PYG{l+s+s1}{\PYGZsq{}}\PYG{p}{)}\PYG{p}{;}
\PYG{n}{pred} \PYG{o}{=} \PYG{n}{intercept} \PYG{o}{+} \PYG{n}{slope}\PYG{o}{*}\PYG{n}{d23\PYGZus{}Ceq} \PYG{c+c1}{\PYGZsh{} fit line}
\PYG{n}{plt}\PYG{o}{.}\PYG{n}{plot}\PYG{p}{(}\PYG{n}{d23\PYGZus{}Ceq}\PYG{p}{,} \PYG{n}{pred}\PYG{p}{,} \PYG{l+s+s1}{\PYGZsq{}}\PYG{l+s+s1}{r}\PYG{l+s+s1}{\PYGZsq{}}\PYG{p}{,} \PYG{n}{label}\PYG{o}{=}\PYG{l+s+s1}{\PYGZsq{}}\PYG{l+s+s1}{y=}\PYG{l+s+si}{\PYGZob{}:.2E\PYGZcb{}}\PYG{l+s+s1}{x+}\PYG{l+s+si}{\PYGZob{}:.2E\PYGZcb{}}\PYG{l+s+s1}{\PYGZsq{}}\PYG{o}{.}\PYG{n}{format}\PYG{p}{(}\PYG{n}{slope}\PYG{p}{,}\PYG{n}{intercept}\PYG{p}{)}\PYG{p}{)} \PYG{p}{;}
\PYG{n}{plt}\PYG{o}{.}\PYG{n}{xlabel}\PYG{p}{(}\PYG{l+s+sa}{r}\PYG{l+s+s2}{\PYGZdq{}}\PYG{l+s+s2}{Equilibrium concentration,\PYGZdl{}C\PYGZus{}}\PYG{l+s+si}{\PYGZob{}eq\PYGZcb{}}\PYG{l+s+s2}{ \PYGZdl{} (mg/L)}\PYG{l+s+s2}{\PYGZdq{}}\PYG{p}{)}\PYG{p}{;} \PYG{n}{plt}\PYG{o}{.}\PYG{n}{ylabel}\PYG{p}{(}\PYG{l+s+sa}{r}\PYG{l+s+s2}{\PYGZdq{}}\PYG{l+s+s2}{Mass Ratio, \PYGZdl{}C\PYGZus{}}\PYG{l+s+si}{\PYGZob{}a\PYGZcb{}}\PYG{l+s+s2}{ \PYGZdl{} (mg/L)}\PYG{l+s+s2}{\PYGZdq{}}\PYG{p}{)}\PYG{p}{;}
\PYG{n}{plt}\PYG{o}{.}\PYG{n}{grid}\PYG{p}{(}\PYG{p}{)}\PYG{p}{;} \PYG{n}{plt}\PYG{o}{.}\PYG{n}{legend}\PYG{p}{(}\PYG{n}{fontsize}\PYG{o}{=}\PYG{l+m+mi}{11}\PYG{p}{)}\PYG{p}{;}  \PYG{n}{plt}\PYG{o}{.}\PYG{n}{text}\PYG{p}{(}\PYG{l+m+mi}{20}\PYG{p}{,} \PYG{l+m+mf}{0.30}\PYG{p}{,}\PYG{l+s+s1}{\PYGZsq{}}\PYG{l+s+s1}{\PYGZdl{}R\PYGZca{}2 = }\PYG{l+s+si}{\PYGZpc{}0.2f}\PYG{l+s+s1}{\PYGZdl{}}\PYG{l+s+s1}{\PYGZsq{}} \PYG{o}{\PYGZpc{}} \PYG{n}{r\PYGZus{}value}\PYG{p}{)}
\PYG{n}{plt}\PYG{o}{.}\PYG{n}{close}\PYG{p}{(}\PYG{p}{)} \PYG{c+c1}{\PYGZsh{} otherwise we have 2 figure}
\PYG{n}{r23\PYGZus{}8} \PYG{o}{=} \PYG{n}{pn}\PYG{o}{.}\PYG{n}{pane}\PYG{o}{.}\PYG{n}{Matplotlib}\PYG{p}{(}\PYG{n}{fig}\PYG{p}{,} \PYG{n}{dpi}\PYG{o}{=}\PYG{l+m+mi}{300}\PYG{p}{)}

\PYG{n}{pn}\PYG{o}{.}\PYG{n}{Row}\PYG{p}{(}\PYG{n}{r23\PYGZus{}7}\PYG{p}{,} \PYG{n}{r23\PYGZus{}8}\PYG{p}{)} 
\end{sphinxVerbatim}

\end{sphinxuseclass}\end{sphinxVerbatimInput}
\begin{sphinxVerbatimOutput}

\begin{sphinxuseclass}{cell_output}
\begin{sphinxVerbatim}[commandchars=\\\{\}]
Row
    [0] Markdown(str, style=\PYGZob{}\PYGZsq{}font\PYGZhy{}size\PYGZsq{}: \PYGZsq{}13pt\PYGZsq{}\PYGZcb{}, width=400)
    [1] Matplotlib(Figure, dpi=300, height=600, width=900)
\end{sphinxVerbatim}

\end{sphinxuseclass}\end{sphinxVerbatimOutput}

\end{sphinxuseclass}
\end{sphinxuseclass}
\begin{sphinxuseclass}{cell}\begin{sphinxVerbatimInput}

\begin{sphinxuseclass}{cell_input}
\begin{sphinxVerbatim}[commandchars=\\\{\}]
\PYG{c+c1}{\PYGZsh{} Solution problem 23 c}

\PYG{n}{r23\PYGZus{}10} \PYG{o}{=} \PYG{n}{pn}\PYG{o}{.}\PYG{n}{pane}\PYG{o}{.}\PYG{n}{LaTeX}\PYG{p}{(}\PYG{l+s+sa}{r}\PYG{l+s+s2}{\PYGZdq{}\PYGZdq{}\PYGZdq{}}
\PYG{l+s+s2}{\PYGZdl{}\PYGZdl{} R = 1+ }\PYG{l+s+s2}{\PYGZbs{}}\PYG{l+s+s2}{frac}\PYG{l+s+s2}{\PYGZob{}}\PYG{l+s+s2}{1\PYGZhy{}n\PYGZus{}e\PYGZcb{}}\PYG{l+s+si}{\PYGZob{}n\PYGZus{}e\PYGZcb{}}\PYG{l+s+s2}{\PYGZbs{}}\PYG{l+s+s2}{cdot }\PYG{l+s+s2}{\PYGZbs{}}\PYG{l+s+s2}{rho}\PYG{l+s+s2}{\PYGZbs{}}\PYG{l+s+s2}{cdot K\PYGZus{}d \PYGZdl{}\PYGZdl{}}
\PYG{l+s+s2}{\PYGZdq{}\PYGZdq{}\PYGZdq{}}\PYG{p}{,}\PYG{n}{width} \PYG{o}{=} \PYG{l+m+mi}{400}\PYG{p}{,} \PYG{n}{style}\PYG{o}{=}\PYG{p}{\PYGZob{}}\PYG{l+s+s1}{\PYGZsq{}}\PYG{l+s+s1}{font\PYGZhy{}size}\PYG{l+s+s1}{\PYGZsq{}}\PYG{p}{:} \PYG{l+s+s1}{\PYGZsq{}}\PYG{l+s+s1}{13pt}\PYG{l+s+s1}{\PYGZsq{}}\PYG{p}{\PYGZcb{}}\PYG{p}{)}

\PYG{c+c1}{\PYGZsh{}Given}
\PYG{n}{rho} \PYG{o}{=} \PYG{l+m+mf}{2.7} \PYG{c+c1}{\PYGZsh{} g/cm3 solid density}
\PYG{n}{n\PYGZus{}e} \PYG{o}{=} \PYG{l+m+mf}{0.30} \PYG{c+c1}{\PYGZsh{} (), effective porosity}
\PYG{n}{K\PYGZus{}d} \PYG{o}{=} \PYG{n}{slope}\PYG{o}{*}\PYG{l+m+mi}{1000} \PYG{c+c1}{\PYGZsh{} cm\PYGZca{}3/g, the slope of the plot, *1000 for unit conversion}

\PYG{c+c1}{\PYGZsh{} Calculate}
\PYG{n}{R} \PYG{o}{=} \PYG{l+m+mi}{1} \PYG{o}{+} \PYG{p}{(}\PYG{p}{(}\PYG{l+m+mi}{1}\PYG{o}{\PYGZhy{}}\PYG{n}{n\PYGZus{}e}\PYG{p}{)}\PYG{o}{/}\PYG{n}{n\PYGZus{}e}\PYG{p}{)}\PYG{o}{*}\PYG{n}{rho}\PYG{o}{*}\PYG{n}{K\PYGZus{}d} 

\PYG{c+c1}{\PYGZsh{}output}
\PYG{n+nb}{print}\PYG{p}{(}\PYG{l+s+s2}{\PYGZdq{}}\PYG{l+s+s2}{The Retardation factor of the sample is: }\PYG{l+s+si}{\PYGZob{}0:1.2f\PYGZcb{}}\PYG{l+s+s2}{ }\PYG{l+s+se}{\PYGZbs{}n}\PYG{l+s+s2}{\PYGZdq{}}\PYG{o}{.}\PYG{n}{format}\PYG{p}{(}\PYG{n}{R}\PYG{p}{)}\PYG{p}{)}
\PYG{n+nb}{print}\PYG{p}{(}\PYG{l+s+s2}{\PYGZdq{}}\PYG{l+s+s2}{The Retardation factor is obtained using:}\PYG{l+s+s2}{\PYGZdq{}}\PYG{p}{)}

\PYG{n}{pn}\PYG{o}{.}\PYG{n}{Column}\PYG{p}{(}\PYG{n}{r23\PYGZus{}10}\PYG{p}{)}
\end{sphinxVerbatim}

\end{sphinxuseclass}\end{sphinxVerbatimInput}
\begin{sphinxVerbatimOutput}

\begin{sphinxuseclass}{cell_output}
\begin{sphinxVerbatim}[commandchars=\\\{\}]
The Retardation factor of the sample is: 21.11 

The Retardation factor is obtained using:
\end{sphinxVerbatim}

\begin{sphinxVerbatim}[commandchars=\\\{\}]
Column
    [0] LaTeX(str, style=\PYGZob{}\PYGZsq{}font\PYGZhy{}size\PYGZsq{}: \PYGZsq{}13pt\PYGZsq{}\PYGZcb{}, width=400)
\end{sphinxVerbatim}

\end{sphinxuseclass}\end{sphinxVerbatimOutput}

\end{sphinxuseclass}
\begin{sphinxuseclass}{cell}\begin{sphinxVerbatimInput}

\begin{sphinxuseclass}{cell_input}
\begin{sphinxVerbatim}[commandchars=\\\{\}]
\PYG{c+c1}{\PYGZsh{}\PYGZsh{}\PYGZsh{} Tutorial Problem 24 \PYGZsh{}\PYGZsh{}\PYGZsh{}}
\end{sphinxVerbatim}

\end{sphinxuseclass}\end{sphinxVerbatimInput}

\end{sphinxuseclass}
\begin{sphinxuseclass}{cell}
\begin{sphinxuseclass}{tag_hide-input}
\begin{sphinxuseclass}{tag_full-width}\begin{sphinxVerbatimOutput}

\begin{sphinxuseclass}{cell_output}
\begin{sphinxVerbatim}[commandchars=\\\{\}]
Column
    [0] Markdown(str, style=\PYGZob{}\PYGZsq{}font\PYGZhy{}size\PYGZsq{}: \PYGZsq{}12pt\PYGZsq{}\PYGZcb{}, width=900)
    [1] DataFrame(DataFrame)
\end{sphinxVerbatim}

\end{sphinxuseclass}\end{sphinxVerbatimOutput}

\end{sphinxuseclass}
\end{sphinxuseclass}
\end{sphinxuseclass}

\subsection{solution Tutorial Problem 24}
\label{\detokenize{content/tutorials/T9/tutorial_09:solution-tutorial-problem-24}}
\begin{sphinxuseclass}{cell}
\begin{sphinxuseclass}{tag_full-width}\begin{sphinxVerbatimInput}

\begin{sphinxuseclass}{cell_input}
\begin{sphinxVerbatim}[commandchars=\\\{\}]
\PYG{c+c1}{\PYGZsh{} Solution TP 24}
\PYG{c+c1}{\PYGZsh{}Given}

\PYG{n}{Vw} \PYG{o}{=} \PYG{l+m+mi}{25}\PYG{o}{/}\PYG{l+m+mi}{1000} \PYG{c+c1}{\PYGZsh{} L, volume of water in L}
\PYG{n}{Ms} \PYG{o}{=} \PYG{l+m+mi}{10} \PYG{c+c1}{\PYGZsh{} g, mass of Cr(IV)}


\PYG{n}{r24\PYGZus{}2} \PYG{o}{=} \PYG{n}{pn}\PYG{o}{.}\PYG{n}{pane}\PYG{o}{.}\PYG{n}{LaTeX}\PYG{p}{(}\PYG{l+s+sa}{r}\PYG{l+s+s2}{\PYGZdq{}\PYGZdq{}\PYGZdq{}}
\PYG{l+s+s2}{First step: Calculate \PYGZdl{}C\PYGZus{}a\PYGZdl{} using}

\PYG{l+s+s2}{\PYGZdl{}\PYGZdl{}C\PYGZus{}a = }\PYG{l+s+s2}{\PYGZbs{}}\PYG{l+s+s2}{frac}\PYG{l+s+s2}{\PYGZob{}}\PYG{l+s+s2}{V\PYGZus{}w }\PYG{l+s+s2}{\PYGZbs{}}\PYG{l+s+s2}{cdot (C\PYGZus{}0 \PYGZhy{} C\PYGZus{}}\PYG{l+s+si}{\PYGZob{}eq\PYGZcb{}}\PYG{l+s+s2}{)\PYGZcb{}}\PYG{l+s+si}{\PYGZob{}M\PYGZus{}s\PYGZcb{}}\PYG{l+s+s2}{\PYGZdl{}\PYGZdl{}}

\PYG{l+s+s2}{\PYGZdq{}\PYGZdq{}\PYGZdq{}}\PYG{p}{,}\PYG{n}{width} \PYG{o}{=} \PYG{l+m+mi}{900}\PYG{p}{,} \PYG{n}{style}\PYG{o}{=}\PYG{p}{\PYGZob{}}\PYG{l+s+s1}{\PYGZsq{}}\PYG{l+s+s1}{font\PYGZhy{}size}\PYG{l+s+s1}{\PYGZsq{}}\PYG{p}{:} \PYG{l+s+s1}{\PYGZsq{}}\PYG{l+s+s1}{12pt}\PYG{l+s+s1}{\PYGZsq{}}\PYG{p}{\PYGZcb{}}\PYG{p}{)}

\PYG{c+c1}{\PYGZsh{} Obtain decadic logarithm of C\PYGZus{}eq and C\PYGZus{}a}

\PYG{n}{d24\PYGZus{}Ca} \PYG{o}{=} \PYG{n}{Vw}\PYG{o}{/}\PYG{n}{Ms}\PYG{o}{*}\PYG{p}{(}\PYG{n}{d24\PYGZus{}Co}\PYG{o}{\PYGZhy{}}\PYG{n}{d24\PYGZus{}Ceq}\PYG{p}{)} \PYG{c+c1}{\PYGZsh{} Ca  = Vw/Ms* (Co\PYGZhy{}Ceq)}
\PYG{n}{log\PYGZus{}Ca} \PYG{o}{=} \PYG{n}{np}\PYG{o}{.}\PYG{n}{log10}\PYG{p}{(}\PYG{n}{d24\PYGZus{}Ca}\PYG{p}{)}
\PYG{n}{log\PYGZus{}Ceq} \PYG{o}{=} \PYG{n}{np}\PYG{o}{.}\PYG{n}{log10}\PYG{p}{(}\PYG{n}{d24\PYGZus{}Ceq}\PYG{p}{)}

\PYG{c+c1}{\PYGZsh{} output in table form \PYGZhy{} we use pandas}
\PYG{n}{log\PYGZus{}d24} \PYG{o}{=} \PYG{p}{\PYGZob{}}\PYG{l+s+s2}{\PYGZdq{}}\PYG{l+s+s2}{Co [mg/L]}\PYG{l+s+s2}{\PYGZdq{}}\PYG{p}{:}\PYG{n}{d24\PYGZus{}Co}\PYG{p}{,} \PYG{l+s+s2}{\PYGZdq{}}\PYG{l+s+s2}{Ceq [mg/L]}\PYG{l+s+s2}{\PYGZdq{}}\PYG{p}{:}\PYG{n}{d24\PYGZus{}Ceq}\PYG{p}{,} \PYG{l+s+s2}{\PYGZdq{}}\PYG{l+s+s2}{Ca [mg/g]}\PYG{l+s+s2}{\PYGZdq{}}\PYG{p}{:} \PYG{n}{d24\PYGZus{}Ca}\PYG{p}{,} 
            \PYG{l+s+s2}{\PYGZdq{}}\PYG{l+s+s2}{log\PYGZus{}Ca}\PYG{l+s+s2}{\PYGZdq{}}\PYG{p}{:} \PYG{n}{log\PYGZus{}Ca}\PYG{p}{,} \PYG{l+s+s2}{\PYGZdq{}}\PYG{l+s+s2}{log\PYGZus{}Ceq}\PYG{l+s+s2}{\PYGZdq{}}\PYG{p}{:} \PYG{n}{log\PYGZus{}Ceq}\PYG{p}{\PYGZcb{}}

\PYG{n}{log\PYGZus{}df24} \PYG{o}{=} \PYG{n}{pd}\PYG{o}{.}\PYG{n}{DataFrame}\PYG{p}{(}\PYG{n}{log\PYGZus{}d24}\PYG{p}{)}

\PYG{n}{pn}\PYG{o}{.}\PYG{n}{Column}\PYG{p}{(}\PYG{n}{r24\PYGZus{}2}\PYG{p}{,} \PYG{n}{log\PYGZus{}df24}\PYG{p}{)}
\end{sphinxVerbatim}

\end{sphinxuseclass}\end{sphinxVerbatimInput}
\begin{sphinxVerbatimOutput}

\begin{sphinxuseclass}{cell_output}
\begin{sphinxVerbatim}[commandchars=\\\{\}]
Column
    [0] LaTeX(str, style=\PYGZob{}\PYGZsq{}font\PYGZhy{}size\PYGZsq{}: \PYGZsq{}12pt\PYGZsq{}\PYGZcb{}, width=900)
    [1] DataFrame(DataFrame)
\end{sphinxVerbatim}

\end{sphinxuseclass}\end{sphinxVerbatimOutput}

\end{sphinxuseclass}
\end{sphinxuseclass}
\begin{sphinxuseclass}{cell}
\begin{sphinxuseclass}{tag_hide-input}
\begin{sphinxuseclass}{tag_full-width}\begin{sphinxVerbatimOutput}

\begin{sphinxuseclass}{cell_output}
\begin{sphinxVerbatim}[commandchars=\\\{\}]
LaTeX(str, style=\PYGZob{}\PYGZsq{}font\PYGZhy{}size\PYGZsq{}: \PYGZsq{}12pt\PYGZsq{}\PYGZcb{}, width=600)
\end{sphinxVerbatim}

\end{sphinxuseclass}\end{sphinxVerbatimOutput}

\end{sphinxuseclass}
\end{sphinxuseclass}
\end{sphinxuseclass}
\begin{sphinxuseclass}{cell}\begin{sphinxVerbatimInput}

\begin{sphinxuseclass}{cell_input}
\begin{sphinxVerbatim}[commandchars=\\\{\}]
\PYG{c+c1}{\PYGZsh{} Continue solution P 24}

\PYG{n}{fig1} \PYG{o}{=} \PYG{n}{plt}\PYG{o}{.}\PYG{n}{figure}\PYG{p}{(}\PYG{p}{)}

\PYG{n}{plt}\PYG{o}{.}\PYG{n}{plot}\PYG{p}{(}\PYG{n}{log\PYGZus{}Ceq}\PYG{p}{,} \PYG{n}{log\PYGZus{}Ca}\PYG{p}{,} \PYG{l+s+s1}{\PYGZsq{}}\PYG{l+s+s1}{o}\PYG{l+s+s1}{\PYGZsq{}}\PYG{p}{,} \PYG{n}{label}\PYG{o}{=}\PYG{l+s+s1}{\PYGZsq{}}\PYG{l+s+s1}{ provided data}\PYG{l+s+s1}{\PYGZsq{}}\PYG{p}{)}\PYG{p}{;}
\PYG{n}{plt}\PYG{o}{.}\PYG{n}{xlabel}\PYG{p}{(}\PYG{l+s+sa}{r}\PYG{l+s+s2}{\PYGZdq{}}\PYG{l+s+s2}{Equilibrium concentration,\PYGZdl{}}\PYG{l+s+s2}{\PYGZbs{}}\PYG{l+s+s2}{log C\PYGZus{}}\PYG{l+s+si}{\PYGZob{}eq\PYGZcb{}}\PYG{l+s+s2}{ \PYGZdl{} (mg/L)}\PYG{l+s+s2}{\PYGZdq{}}\PYG{p}{)}\PYG{p}{;} \PYG{n}{plt}\PYG{o}{.}\PYG{n}{ylabel}\PYG{p}{(}\PYG{l+s+sa}{r}\PYG{l+s+s2}{\PYGZdq{}}\PYG{l+s+s2}{Mass Ratio, \PYGZdl{}}\PYG{l+s+s2}{\PYGZbs{}}\PYG{l+s+s2}{log C\PYGZus{}}\PYG{l+s+si}{\PYGZob{}a\PYGZcb{}}\PYG{l+s+s2}{ \PYGZdl{} (mg/L)}\PYG{l+s+s2}{\PYGZdq{}}\PYG{p}{)}\PYG{p}{;}
\PYG{n}{plt}\PYG{o}{.}\PYG{n}{legend}\PYG{p}{(}\PYG{n}{fontsize}\PYG{o}{=}\PYG{l+m+mi}{11}\PYG{p}{)}\PYG{p}{;}
\PYG{n}{plt}\PYG{o}{.}\PYG{n}{close}\PYG{p}{(}\PYG{p}{)} \PYG{c+c1}{\PYGZsh{} otherwise we have 2 figure only when using pn.}
\PYG{n}{r24\PYGZus{}3} \PYG{o}{=} \PYG{n}{pn}\PYG{o}{.}\PYG{n}{pane}\PYG{o}{.}\PYG{n}{Matplotlib}\PYG{p}{(}\PYG{n}{fig1}\PYG{p}{,} \PYG{n}{dpi}\PYG{o}{=}\PYG{l+m+mi}{300}\PYG{p}{)}


\PYG{c+c1}{\PYGZsh{} Linear fit we use scipy.stats.linregress library}
\PYG{n}{slope}\PYG{p}{,} \PYG{n}{intercept}\PYG{p}{,} \PYG{n}{r\PYGZus{}value}\PYG{p}{,} \PYG{n}{p\PYGZus{}value}\PYG{p}{,} \PYG{n}{std\PYGZus{}err} \PYG{o}{=} \PYG{n}{stats}\PYG{o}{.}\PYG{n}{linregress}\PYG{p}{(}\PYG{n}{log\PYGZus{}Ceq}\PYG{p}{,} \PYG{n}{log\PYGZus{}Ca}\PYG{p}{)} \PYG{c+c1}{\PYGZsh{} linear regression}

\PYG{c+c1}{\PYGZsh{} Make a fit plot}
\PYG{n}{fig2} \PYG{o}{=} \PYG{n}{plt}\PYG{o}{.}\PYG{n}{figure}\PYG{p}{(}\PYG{p}{)}
\PYG{n}{plt}\PYG{o}{.}\PYG{n}{plot}\PYG{p}{(}\PYG{n}{log\PYGZus{}Ceq}\PYG{p}{,} \PYG{n}{log\PYGZus{}Ca}\PYG{p}{,} \PYG{l+s+s1}{\PYGZsq{}}\PYG{l+s+s1}{o}\PYG{l+s+s1}{\PYGZsq{}}\PYG{p}{,} \PYG{n}{label}\PYG{o}{=}\PYG{l+s+s1}{\PYGZsq{}}\PYG{l+s+s1}{ provided data}\PYG{l+s+s1}{\PYGZsq{}}\PYG{p}{)}\PYG{p}{;}
\PYG{n}{pred} \PYG{o}{=} \PYG{n}{intercept} \PYG{o}{+} \PYG{n}{slope}\PYG{o}{*}\PYG{n}{log\PYGZus{}Ceq} \PYG{c+c1}{\PYGZsh{} fit line y = mx + C}
\PYG{n}{plt}\PYG{o}{.}\PYG{n}{plot}\PYG{p}{(}\PYG{n}{log\PYGZus{}Ceq}\PYG{p}{,} \PYG{n}{pred}\PYG{p}{,} \PYG{l+s+s1}{\PYGZsq{}}\PYG{l+s+s1}{r}\PYG{l+s+s1}{\PYGZsq{}}\PYG{p}{,} \PYG{n}{label}\PYG{o}{=}\PYG{l+s+s1}{\PYGZsq{}}\PYG{l+s+s1}{y=}\PYG{l+s+si}{\PYGZob{}:.2E\PYGZcb{}}\PYG{l+s+s1}{x+}\PYG{l+s+si}{\PYGZob{}:.2E\PYGZcb{}}\PYG{l+s+s1}{\PYGZsq{}}\PYG{o}{.}\PYG{n}{format}\PYG{p}{(}\PYG{n}{slope}\PYG{p}{,}\PYG{n}{intercept}\PYG{p}{)}\PYG{p}{)} \PYG{p}{;}
\PYG{n}{plt}\PYG{o}{.}\PYG{n}{xlabel}\PYG{p}{(}\PYG{l+s+sa}{r}\PYG{l+s+s2}{\PYGZdq{}}\PYG{l+s+s2}{Equilibrium concentration,\PYGZdl{}}\PYG{l+s+s2}{\PYGZbs{}}\PYG{l+s+s2}{log C\PYGZus{}}\PYG{l+s+si}{\PYGZob{}eq\PYGZcb{}}\PYG{l+s+s2}{ \PYGZdl{} (mg/L)}\PYG{l+s+s2}{\PYGZdq{}}\PYG{p}{)}\PYG{p}{;} \PYG{n}{plt}\PYG{o}{.}\PYG{n}{ylabel}\PYG{p}{(}\PYG{l+s+sa}{r}\PYG{l+s+s2}{\PYGZdq{}}\PYG{l+s+s2}{Mass Ratio, \PYGZdl{}}\PYG{l+s+s2}{\PYGZbs{}}\PYG{l+s+s2}{log C\PYGZus{}}\PYG{l+s+si}{\PYGZob{}a\PYGZcb{}}\PYG{l+s+s2}{ \PYGZdl{} (mg/L)}\PYG{l+s+s2}{\PYGZdq{}}\PYG{p}{)}\PYG{p}{;}
\PYG{n}{plt}\PYG{o}{.}\PYG{n}{grid}\PYG{p}{(}\PYG{p}{)}\PYG{p}{;} \PYG{n}{plt}\PYG{o}{.}\PYG{n}{legend}\PYG{p}{(}\PYG{n}{fontsize}\PYG{o}{=}\PYG{l+m+mi}{11}\PYG{p}{)}\PYG{p}{;}  \PYG{n}{plt}\PYG{o}{.}\PYG{n}{text}\PYG{p}{(}\PYG{l+m+mf}{1.20}\PYG{p}{,} \PYG{o}{\PYGZhy{}}\PYG{l+m+mf}{0.60}\PYG{p}{,}\PYG{l+s+s1}{\PYGZsq{}}\PYG{l+s+s1}{\PYGZdl{}R\PYGZca{}2 = }\PYG{l+s+si}{\PYGZpc{}0.2f}\PYG{l+s+s1}{\PYGZdl{}}\PYG{l+s+s1}{\PYGZsq{}} \PYG{o}{\PYGZpc{}} \PYG{n}{r\PYGZus{}value}\PYG{p}{)}
\PYG{n}{plt}\PYG{o}{.}\PYG{n}{close}\PYG{p}{(}\PYG{p}{)} \PYG{c+c1}{\PYGZsh{} otherwise we have 2 figure}
\PYG{n}{r24\PYGZus{}4} \PYG{o}{=} \PYG{n}{pn}\PYG{o}{.}\PYG{n}{pane}\PYG{o}{.}\PYG{n}{Matplotlib}\PYG{p}{(}\PYG{n}{fig2}\PYG{p}{,} \PYG{n}{dpi}\PYG{o}{=}\PYG{l+m+mi}{300}\PYG{p}{)}

\PYG{c+c1}{\PYGZsh{} solution 10.2}

\PYG{n}{r24\PYGZus{}5} \PYG{o}{=} \PYG{n}{pn}\PYG{o}{.}\PYG{n}{pane}\PYG{o}{.}\PYG{n}{LaTeX}\PYG{p}{(}\PYG{l+s+sa}{r}\PYG{l+s+s2}{\PYGZdq{}\PYGZdq{}\PYGZdq{}}\PYG{l+s+s2}{ }
\PYG{l+s+s2}{The fit is almost perfect \PYGZdl{}R\PYGZca{}2 = 0.99\PYGZdl{}. So we can use linear\PYGZhy{}fit results to get\PYGZlt{}br\PYGZgt{}}
\PYG{l+s+s2}{a. slope = \PYGZdl{}n\PYGZus{}}\PYG{l+s+si}{\PYGZob{}Fr\PYGZcb{}}\PYG{l+s+s2}{\PYGZdl{} = 0.76 \PYGZlt{}br\PYGZgt{}}
\PYG{l+s+s2}{b. intercept = \PYGZdl{}}\PYG{l+s+s2}{\PYGZbs{}}\PYG{l+s+s2}{log K\PYGZus{}}\PYG{l+s+si}{\PYGZob{}Fr\PYGZcb{}}\PYG{l+s+s2}{\PYGZdl{} = \PYGZhy{}1.99, i.e., \PYGZdl{}K\PYGZus{}}\PYG{l+s+si}{\PYGZob{}Fr\PYGZcb{}}\PYG{l+s+s2}{ = 10\PYGZca{}}\PYG{l+s+s2}{\PYGZob{}}\PYG{l+s+s2}{\PYGZhy{}1.99\PYGZcb{}\PYGZdl{}}

\PYG{l+s+s2}{\PYGZdq{}\PYGZdq{}\PYGZdq{}}\PYG{p}{,}\PYG{n}{width} \PYG{o}{=} \PYG{l+m+mi}{600}\PYG{p}{,} \PYG{n}{style}\PYG{o}{=}\PYG{p}{\PYGZob{}}\PYG{l+s+s1}{\PYGZsq{}}\PYG{l+s+s1}{font\PYGZhy{}size}\PYG{l+s+s1}{\PYGZsq{}}\PYG{p}{:} \PYG{l+s+s1}{\PYGZsq{}}\PYG{l+s+s1}{12pt}\PYG{l+s+s1}{\PYGZsq{}}\PYG{p}{\PYGZcb{}}\PYG{p}{)}

\PYG{c+c1}{\PYGZsh{}output}
\PYG{n}{r24\PYGZus{}6} \PYG{o}{=} \PYG{n}{pn}\PYG{o}{.}\PYG{n}{Row}\PYG{p}{(}\PYG{n}{r24\PYGZus{}3}\PYG{p}{,} \PYG{n}{r24\PYGZus{}4}\PYG{p}{)} 
\PYG{n}{pn}\PYG{o}{.}\PYG{n}{Column}\PYG{p}{(}\PYG{n}{r24\PYGZus{}6}\PYG{p}{,} \PYG{n}{r24\PYGZus{}5}\PYG{p}{)}
\end{sphinxVerbatim}

\end{sphinxuseclass}\end{sphinxVerbatimInput}
\begin{sphinxVerbatimOutput}

\begin{sphinxuseclass}{cell_output}
\begin{sphinxVerbatim}[commandchars=\\\{\}]
Column
    [0] Row
        [0] Matplotlib(Figure, dpi=300, height=600, width=900)
        [1] Matplotlib(Figure, dpi=300, height=600, width=900)
    [1] LaTeX(str, style=\PYGZob{}\PYGZsq{}font\PYGZhy{}size\PYGZsq{}: \PYGZsq{}12pt\PYGZsq{}\PYGZcb{}, width=600)
\end{sphinxVerbatim}

\end{sphinxuseclass}\end{sphinxVerbatimOutput}

\end{sphinxuseclass}

\subsection{Tutorial problem 25 \sphinxhyphen{} contaminated site}
\label{\detokenize{content/tutorials/T9/tutorial_09:tutorial-problem-25-contaminated-site}}
\begin{sphinxuseclass}{cell}
\begin{sphinxuseclass}{tag_hide-input}\begin{sphinxVerbatimOutput}

\begin{sphinxuseclass}{cell_output}
\begin{sphinxVerbatim}[commandchars=\\\{\}]
Column
    [0] LaTeX(str, style=\PYGZob{}\PYGZsq{}font\PYGZhy{}size\PYGZsq{}: \PYGZsq{}12pt\PYGZsq{}\PYGZcb{}, width=800)
    [1] PNG(str, width=800)
\end{sphinxVerbatim}

\end{sphinxuseclass}\end{sphinxVerbatimOutput}

\end{sphinxuseclass}
\end{sphinxuseclass}
\begin{sphinxuseclass}{cell}\begin{sphinxVerbatimInput}

\begin{sphinxuseclass}{cell_input}
\begin{sphinxVerbatim}[commandchars=\\\{\}]
\PYG{c+c1}{\PYGZsh{}\PYGZsh{}\PYGZsh{} solution of Problem 25 \PYGZsh{}\PYGZsh{}\PYGZsh{}}
\end{sphinxVerbatim}

\end{sphinxuseclass}\end{sphinxVerbatimInput}

\end{sphinxuseclass}
\begin{sphinxuseclass}{cell}
\begin{sphinxuseclass}{tag_hide-input}\begin{sphinxVerbatimOutput}

\begin{sphinxuseclass}{cell_output}
\begin{sphinxVerbatim}[commandchars=\\\{\}]
Column
    [0] LaTeX(str, style=\PYGZob{}\PYGZsq{}font\PYGZhy{}size\PYGZsq{}: \PYGZsq{}12pt\PYGZsq{}\PYGZcb{}, width=800)
    [1] Markdown(str, style=\PYGZob{}\PYGZsq{}font\PYGZhy{}size\PYGZsq{}: \PYGZsq{}12pt\PYGZsq{}\PYGZcb{}, width=800)
\end{sphinxVerbatim}

\end{sphinxuseclass}\end{sphinxVerbatimOutput}

\end{sphinxuseclass}
\end{sphinxuseclass}

\subsection{Solution of problem 25}
\label{\detokenize{content/tutorials/T9/tutorial_09:solution-of-problem-25}}
\begin{sphinxuseclass}{cell}\begin{sphinxVerbatimInput}

\begin{sphinxuseclass}{cell_input}
\begin{sphinxVerbatim}[commandchars=\\\{\}]
\PYG{c+c1}{\PYGZsh{} Solution problem 25 }

\PYG{c+c1}{\PYGZsh{}input}
\PYG{n}{Dx} \PYG{o}{=} \PYG{l+m+mf}{7.56} \PYG{c+c1}{\PYGZsh{}m\PYGZca{}2/d disp coeff}
\PYG{n}{vx} \PYG{o}{=} \PYG{l+m+mf}{0.252} \PYG{c+c1}{\PYGZsh{} m/d gw velocity}
\PYG{n}{R} \PYG{o}{=} \PYG{l+m+mf}{5.354} \PYG{c+c1}{\PYGZsh{} [] retardation}
\PYG{n}{Co} \PYG{o}{=} \PYG{l+m+mi}{12} \PYG{c+c1}{\PYGZsh{} mg/L in concentration}
\PYG{n}{x} \PYG{o}{=} \PYG{l+m+mi}{30} \PYG{c+c1}{\PYGZsh{} m distance}
\PYG{n}{ld} \PYG{o}{=} \PYG{l+m+mf}{0.01} \PYG{c+c1}{\PYGZsh{} 1/d lambda}
\PYG{n}{t} \PYG{o}{=} \PYG{n}{np}\PYG{o}{.}\PYG{n}{linspace}\PYG{p}{(}\PYG{l+m+mi}{0}\PYG{p}{,} \PYG{l+m+mi}{1000}\PYG{p}{,} \PYG{l+m+mi}{1000}\PYG{p}{)}

\PYG{c+c1}{\PYGZsh{} interim calculations}

\PYG{n}{f1} \PYG{o}{=} \PYG{n}{Dx}\PYG{o}{/}\PYG{n}{R}
\PYG{n}{f2} \PYG{o}{=} \PYG{n}{vx}\PYG{o}{/}\PYG{n}{R}
\PYG{n}{f3} \PYG{o}{=} \PYG{n}{np}\PYG{o}{.}\PYG{n}{sqrt}\PYG{p}{(}\PYG{n}{f2}\PYG{o}{*}\PYG{o}{*}\PYG{l+m+mi}{2}\PYG{o}{+} \PYG{l+m+mi}{4}\PYG{o}{*}\PYG{n}{ld}\PYG{o}{*}\PYG{n}{f1}\PYG{p}{)}

\PYG{k+kn}{import} \PYG{n+nn}{scipy}\PYG{n+nn}{.}\PYG{n+nn}{special} \PYG{k}{as} \PYG{n+nn}{sc} \PYG{c+c1}{\PYGZsh{} Required for getting erfc function}

\PYG{n}{T1} \PYG{o}{=} \PYG{n}{np}\PYG{o}{.}\PYG{n}{exp}\PYG{p}{(}\PYG{n}{x}\PYG{o}{/}\PYG{p}{(}\PYG{l+m+mi}{2}\PYG{o}{*}\PYG{n}{f1}\PYG{p}{)}\PYG{o}{*}\PYG{p}{(}\PYG{n}{f2}\PYG{o}{\PYGZhy{}}\PYG{n}{f3}\PYG{p}{)}\PYG{p}{)}        \PYG{c+c1}{\PYGZsh{} first exp term}
\PYG{n}{T2} \PYG{o}{=} \PYG{n}{sc}\PYG{o}{.}\PYG{n}{erfc}\PYG{p}{(}\PYG{p}{(}\PYG{n}{x}\PYG{o}{\PYGZhy{}}\PYG{n}{t}\PYG{o}{*}\PYG{n}{f3}\PYG{p}{)}\PYG{o}{/}\PYG{p}{(}\PYG{l+m+mi}{2}\PYG{o}{*}\PYG{n}{np}\PYG{o}{.}\PYG{n}{sqrt}\PYG{p}{(}\PYG{n}{t}\PYG{p}{)}\PYG{o}{*}\PYG{n}{f1}\PYG{p}{)}\PYG{p}{)}     \PYG{c+c1}{\PYGZsh{} first erfc term  }
\PYG{n}{T3} \PYG{o}{=} \PYG{n}{np}\PYG{o}{.}\PYG{n}{exp}\PYG{p}{(}\PYG{n}{x}\PYG{o}{/}\PYG{p}{(}\PYG{l+m+mi}{2}\PYG{o}{*}\PYG{n}{f1}\PYG{p}{)}\PYG{o}{*}\PYG{p}{(}\PYG{n}{f2}\PYG{o}{+}\PYG{n}{f3}\PYG{p}{)}\PYG{p}{)}       \PYG{c+c1}{\PYGZsh{} second exp term}
\PYG{n}{T4} \PYG{o}{=} \PYG{n}{sc}\PYG{o}{.}\PYG{n}{erfc}\PYG{p}{(}\PYG{p}{(}\PYG{n}{x}\PYG{o}{+}\PYG{n}{t}\PYG{o}{*}\PYG{n}{f3}\PYG{p}{)}\PYG{o}{/}\PYG{p}{(}\PYG{l+m+mi}{2}\PYG{o}{*}\PYG{n}{np}\PYG{o}{.}\PYG{n}{sqrt}\PYG{p}{(}\PYG{n}{t}\PYG{p}{)}\PYG{o}{*}\PYG{n}{f1}\PYG{p}{)}\PYG{p}{)}      \PYG{c+c1}{\PYGZsh{} second erfc term}
\end{sphinxVerbatim}

\end{sphinxuseclass}\end{sphinxVerbatimInput}

\end{sphinxuseclass}
\begin{sphinxuseclass}{cell}\begin{sphinxVerbatimInput}

\begin{sphinxuseclass}{cell_input}
\begin{sphinxVerbatim}[commandchars=\\\{\}]
\PYG{c+c1}{\PYGZsh{} solution P 27 contd.}

\PYG{c+c1}{\PYGZsh{} Calculation}
\PYG{n}{C} \PYG{o}{=} \PYG{n}{Co}\PYG{o}{/}\PYG{l+m+mi}{2}\PYG{o}{*}\PYG{p}{(}\PYG{n}{T1}\PYG{o}{*}\PYG{n}{T2}\PYG{p}{)}\PYG{o}{+}\PYG{p}{(}\PYG{n}{T3}\PYG{o}{*}\PYG{n}{T4}\PYG{p}{)}

\PYG{c+c1}{\PYGZsh{}plotting}

\PYG{n}{plt}\PYG{o}{.}\PYG{n}{plot}\PYG{p}{(}\PYG{n}{t}\PYG{p}{,}\PYG{n}{C}\PYG{p}{,} \PYG{n}{label} \PYG{o}{=} \PYG{l+s+s2}{\PYGZdq{}}\PYG{l+s+s2}{Concentration at 30 m from the source}\PYG{l+s+s2}{\PYGZdq{}}\PYG{p}{)}
\PYG{n}{plt}\PYG{o}{.}\PYG{n}{grid}\PYG{p}{(}\PYG{p}{)} 
\PYG{n}{plt}\PYG{o}{.}\PYG{n}{ylim}\PYG{p}{(}\PYG{p}{(}\PYG{l+m+mi}{0}\PYG{p}{,}\PYG{l+m+mi}{2}\PYG{p}{)}\PYG{p}{)}
\PYG{n}{plt}\PYG{o}{.}\PYG{n}{xlabel}\PYG{p}{(}\PYG{l+s+sa}{r}\PYG{l+s+s2}{\PYGZdq{}}\PYG{l+s+s2}{\PYGZdl{}t\PYGZdl{} (days)}\PYG{l+s+s2}{\PYGZdq{}}\PYG{p}{)}\PYG{p}{;} \PYG{n}{plt}\PYG{o}{.}\PYG{n}{ylabel}\PYG{p}{(}\PYG{l+s+sa}{r}\PYG{l+s+s2}{\PYGZdq{}}\PYG{l+s+s2}{\PYGZdl{}C\PYGZus{}i\PYGZdl{} (mg/LL)}\PYG{l+s+s2}{\PYGZdq{}}\PYG{p}{)}
\PYG{n}{plt}\PYG{o}{.}\PYG{n}{legend}\PYG{p}{(}\PYG{p}{)} 
\end{sphinxVerbatim}

\end{sphinxuseclass}\end{sphinxVerbatimInput}
\begin{sphinxVerbatimOutput}

\begin{sphinxuseclass}{cell_output}
\begin{sphinxVerbatim}[commandchars=\\\{\}]
\PYGZlt{}matplotlib.legend.Legend at 0x28e9e0feb00\PYGZgt{}
\end{sphinxVerbatim}

\noindent\sphinxincludegraphics{{C:/Users/vibhu/GWtextbook/_build/jupyter_execute/tutorial_09_39_1}.png}

\end{sphinxuseclass}\end{sphinxVerbatimOutput}

\end{sphinxuseclass}
\sphinxAtStartPar
\sphinxstylestrong{Tutorial problems end here.}

\sphinxAtStartPar
Next Tutorial we perform numerical modeling using MODFLOW/MT3DMS in \sphinxstylestrong{modelmuse} interface

\sphinxAtStartPar
In tutorials we solved:
\begin{itemize}
\item {} 
\sphinxAtStartPar
25 Class problems

\item {} 
\sphinxAtStartPar
11 Homework problems

\item {} 
\sphinxAtStartPar
1\sphinxhyphen{}set past exam (self\sphinxhyphen{}learning)

\end{itemize}

\sphinxAtStartPar
We learned a bit on using Python code to solve our problem

\sphinxstepscope


\part{Tools}

\sphinxstepscope


\chapter{Simulating Mass Budget}
\label{\detokenize{content/tools/decay:simulating-mass-budget}}\label{\detokenize{content/tools/decay::doc}}
\sphinxAtStartPar
\sphinxstyleemphasis{(The contents presented in this section were re\sphinxhyphen{}developed principally by Dr. P. K. Yadav. The original tool, Spreadsheet based, was developed by Prof. Rudolf Liedl)}


\section{How to use the tool?}
\label{\detokenize{content/tools/decay:how-to-use-the-tool}}\begin{enumerate}
\sphinxsetlistlabels{\arabic}{enumi}{enumii}{}{.}%
\item {} 
\sphinxAtStartPar
Go to the Binder by clicking the rocket button (top\sphinxhyphen{}right of the page)

\item {} 
\sphinxAtStartPar
Execute the code cell

\item {} 
\sphinxAtStartPar
Change the values of different quantities in the box.

\end{enumerate}

\sphinxAtStartPar
This tool can also be downloaded and run locally. For that download the \sphinxstyleemphasis{deacy.ipynb} file and execute the process in any editor (e.g., JUPYTER notebook, JUPYTER lab) that is able to read and execute this file\sphinxhyphen{}type.

\sphinxAtStartPar
The code may also be executed in the book page.

\sphinxAtStartPar
The codes are licensed under CC by 4.0 \sphinxhref{https://creativecommons.org/licenses/by/4.0/deed.en}{(use anyways, but acknowledge the original work)}

\begin{sphinxuseclass}{cell}\begin{sphinxVerbatimInput}

\begin{sphinxuseclass}{cell_input}
\begin{sphinxVerbatim}[commandchars=\\\{\}]
\PYG{c+c1}{\PYGZsh{} Used library}
\PYG{k+kn}{import} \PYG{n+nn}{numpy} \PYG{k}{as} \PYG{n+nn}{np} \PYG{c+c1}{\PYGZsh{} for calculation}
\PYG{k+kn}{import} \PYG{n+nn}{matplotlib}\PYG{n+nn}{.}\PYG{n+nn}{pyplot} \PYG{k}{as} \PYG{n+nn}{plt} \PYG{c+c1}{\PYGZsh{} for plots}
\PYG{k+kn}{import} \PYG{n+nn}{pandas} \PYG{k}{as} \PYG{n+nn}{pd}  \PYG{c+c1}{\PYGZsh{} for table}
\PYG{k+kn}{import} \PYG{n+nn}{ipywidgets} \PYG{k}{as} \PYG{n+nn}{widgets}  \PYG{c+c1}{\PYGZsh{} for widgets}

\PYG{c+c1}{\PYGZsh{} The main function}

\PYG{k}{def} \PYG{n+nf}{mass\PYGZus{}bal}\PYG{p}{(}\PYG{n}{n\PYGZus{}simulation}\PYG{p}{,} \PYG{n}{MA}\PYG{p}{,} \PYG{n}{MB}\PYG{p}{,} \PYG{n}{MC}\PYG{p}{,} \PYG{n}{R\PYGZus{}A}\PYG{p}{,} \PYG{n}{R\PYGZus{}B}\PYG{p}{)}\PYG{p}{:}
    
    \PYG{n}{A} \PYG{o}{=} \PYG{n}{np}\PYG{o}{.}\PYG{n}{zeros}\PYG{p}{(}\PYG{n}{n\PYGZus{}simulation}\PYG{p}{)} \PYG{c+c1}{\PYGZsh{} creat an array with zros}
    \PYG{n}{B} \PYG{o}{=} \PYG{n}{np}\PYG{o}{.}\PYG{n}{zeros}\PYG{p}{(}\PYG{n}{n\PYGZus{}simulation}\PYG{p}{)}
    \PYG{n}{C} \PYG{o}{=} \PYG{n}{np}\PYG{o}{.}\PYG{n}{zeros}\PYG{p}{(}\PYG{n}{n\PYGZus{}simulation}\PYG{p}{)} 
    \PYG{n}{time}  \PYG{o}{=} \PYG{n}{np}\PYG{o}{.}\PYG{n}{arange}\PYG{p}{(}\PYG{n}{n\PYGZus{}simulation}\PYG{p}{)}
    
    \PYG{k}{for} \PYG{n}{i} \PYG{o+ow}{in} \PYG{n+nb}{range}\PYG{p}{(}\PYG{l+m+mi}{0}\PYG{p}{,}\PYG{n}{n\PYGZus{}simulation}\PYG{o}{\PYGZhy{}}\PYG{l+m+mi}{1}\PYG{p}{)}\PYG{p}{:}
        \PYG{n}{A}\PYG{p}{[}\PYG{l+m+mi}{0}\PYG{p}{]} \PYG{o}{=} \PYG{n}{MA}  \PYG{c+c1}{\PYGZsh{} starting input value}
        
        \PYG{n}{B}\PYG{p}{[}\PYG{l+m+mi}{0}\PYG{p}{]} \PYG{o}{=} \PYG{n}{MB}
        \PYG{n}{C}\PYG{p}{[}\PYG{l+m+mi}{0}\PYG{p}{]} \PYG{o}{=} \PYG{n}{MC}
        \PYG{n}{A}\PYG{p}{[}\PYG{n}{i}\PYG{o}{+}\PYG{l+m+mi}{1}\PYG{p}{]} \PYG{o}{=} \PYG{n}{A}\PYG{p}{[}\PYG{n}{i}\PYG{p}{]}\PYG{o}{\PYGZhy{}}\PYG{n}{R\PYGZus{}A}\PYG{o}{*}\PYG{n}{A}\PYG{p}{[}\PYG{n}{i}\PYG{p}{]}
        \PYG{n}{B}\PYG{p}{[}\PYG{n}{i}\PYG{o}{+}\PYG{l+m+mi}{1}\PYG{p}{]} \PYG{o}{=} \PYG{n}{B}\PYG{p}{[}\PYG{n}{i}\PYG{p}{]}\PYG{o}{+}\PYG{n}{R\PYGZus{}A}\PYG{o}{*}\PYG{n}{A}\PYG{p}{[}\PYG{n}{i}\PYG{p}{]}\PYG{o}{\PYGZhy{}}\PYG{n}{R\PYGZus{}B}\PYG{o}{*}\PYG{n}{B}\PYG{p}{[}\PYG{n}{i}\PYG{p}{]} 
        \PYG{n}{C}\PYG{p}{[}\PYG{n}{i}\PYG{o}{+}\PYG{l+m+mi}{1}\PYG{p}{]} \PYG{o}{=} \PYG{n}{C}\PYG{p}{[}\PYG{n}{i}\PYG{p}{]}\PYG{o}{+}\PYG{n}{R\PYGZus{}B}\PYG{o}{*}\PYG{n}{B}\PYG{p}{[}\PYG{n}{i}\PYG{p}{]}
        \PYG{n}{summ} \PYG{o}{=} \PYG{n}{A}\PYG{p}{[}\PYG{n}{i}\PYG{p}{]}\PYG{o}{+}\PYG{n}{B}\PYG{p}{[}\PYG{n}{i}\PYG{p}{]}\PYG{o}{+}\PYG{n}{C}\PYG{p}{[}\PYG{n}{i}\PYG{p}{]}
        
    \PYG{n}{d} \PYG{o}{=} \PYG{p}{\PYGZob{}}\PYG{l+s+s2}{\PYGZdq{}}\PYG{l+s+s2}{Mass\PYGZus{}A}\PYG{l+s+s2}{\PYGZdq{}}\PYG{p}{:} \PYG{n}{A}\PYG{p}{,} \PYG{l+s+s2}{\PYGZdq{}}\PYG{l+s+s2}{Mass\PYGZus{}B}\PYG{l+s+s2}{\PYGZdq{}}\PYG{p}{:} \PYG{n}{B}\PYG{p}{,} \PYG{l+s+s2}{\PYGZdq{}}\PYG{l+s+s2}{Mass\PYGZus{}C}\PYG{l+s+s2}{\PYGZdq{}}\PYG{p}{:} \PYG{n}{C}\PYG{p}{,} \PYG{l+s+s2}{\PYGZdq{}}\PYG{l+s+s2}{Total Mass}\PYG{l+s+s2}{\PYGZdq{}}\PYG{p}{:} \PYG{n}{summ}\PYG{p}{\PYGZcb{}}
    \PYG{n}{df} \PYG{o}{=} \PYG{n}{pd}\PYG{o}{.}\PYG{n}{DataFrame}\PYG{p}{(}\PYG{n}{d}\PYG{p}{)} \PYG{c+c1}{\PYGZsh{} Generating result table}
    \PYG{n}{label} \PYG{o}{=} \PYG{p}{[}\PYG{l+s+s2}{\PYGZdq{}}\PYG{l+s+s2}{Mass A (g)}\PYG{l+s+s2}{\PYGZdq{}}\PYG{p}{,} \PYG{l+s+s2}{\PYGZdq{}}\PYG{l+s+s2}{Mass B (g)}\PYG{l+s+s2}{\PYGZdq{}}\PYG{p}{,} \PYG{l+s+s2}{\PYGZdq{}}\PYG{l+s+s2}{Mass C (g)}\PYG{l+s+s2}{\PYGZdq{}}\PYG{p}{]}
    \PYG{n}{fig} \PYG{o}{=} \PYG{n}{plt}\PYG{o}{.}\PYG{n}{figure}\PYG{p}{(}\PYG{n}{figsize}\PYG{o}{=}\PYG{p}{(}\PYG{l+m+mi}{6}\PYG{p}{,}\PYG{l+m+mi}{4}\PYG{p}{)}\PYG{p}{)}
    \PYG{n}{plt}\PYG{o}{.}\PYG{n}{plot}\PYG{p}{(}\PYG{n}{time}\PYG{p}{,} \PYG{n}{A}\PYG{p}{,} \PYG{n}{time}\PYG{p}{,} \PYG{n}{B}\PYG{p}{,} \PYG{n}{time}\PYG{p}{,} \PYG{n}{C}\PYG{p}{,} \PYG{n}{linewidth}\PYG{o}{=}\PYG{l+m+mi}{3}\PYG{p}{)}\PYG{p}{;}  \PYG{c+c1}{\PYGZsh{} plotting the results}
    \PYG{n}{plt}\PYG{o}{.}\PYG{n}{xlabel}\PYG{p}{(}\PYG{l+s+s2}{\PYGZdq{}}\PYG{l+s+s2}{Time [Time Unit]}\PYG{l+s+s2}{\PYGZdq{}}\PYG{p}{)}\PYG{p}{;} \PYG{n}{plt}\PYG{o}{.}\PYG{n}{ylabel}\PYG{p}{(}\PYG{l+s+s2}{\PYGZdq{}}\PYG{l+s+s2}{Mass [g]}\PYG{l+s+s2}{\PYGZdq{}}\PYG{p}{)} \PYG{c+c1}{\PYGZsh{} placing axis labels}
    \PYG{n}{plt}\PYG{o}{.}\PYG{n}{legend}\PYG{p}{(}\PYG{n}{label}\PYG{p}{,} \PYG{n}{loc}\PYG{o}{=}\PYG{l+m+mi}{0}\PYG{p}{)}\PYG{p}{;}\PYG{n}{plt}\PYG{o}{.}\PYG{n}{grid}\PYG{p}{(}\PYG{p}{)}\PYG{p}{;} \PYG{n}{plt}\PYG{o}{.}\PYG{n}{xlim}\PYG{p}{(}\PYG{p}{[}\PYG{l+m+mi}{0}\PYG{p}{,}\PYG{n}{n\PYGZus{}simulation}\PYG{p}{]}\PYG{p}{)}\PYG{p}{;} \PYG{n}{plt}\PYG{o}{.}\PYG{n}{ylim}\PYG{p}{(}\PYG{n}{bottom}\PYG{o}{=}\PYG{l+m+mi}{0}\PYG{p}{)} \PYG{c+c1}{\PYGZsh{} legends, grids, x,y limits}
    \PYG{n}{plt}\PYG{o}{.}\PYG{n}{show}\PYG{p}{(}\PYG{p}{)} \PYG{c+c1}{\PYGZsh{} display plot}
    
    \PYG{k}{return} \PYG{n+nb}{print}\PYG{p}{(}\PYG{n}{df}\PYG{o}{.}\PYG{n}{round}\PYG{p}{(}\PYG{l+m+mi}{2}\PYG{p}{)}\PYG{p}{)} 

\PYG{c+c1}{\PYGZsh{} Widgets and interactive}

\PYG{n}{N} \PYG{o}{=} \PYG{n}{widgets}\PYG{o}{.}\PYG{n}{BoundedIntText}\PYG{p}{(}\PYG{n}{value}\PYG{o}{=}\PYG{l+m+mi}{20}\PYG{p}{,}\PYG{n+nb}{min}\PYG{o}{=}\PYG{l+m+mi}{0}\PYG{p}{,}\PYG{n+nb}{max}\PYG{o}{=}\PYG{l+m+mi}{100}\PYG{p}{,}\PYG{n}{step}\PYG{o}{=}\PYG{l+m+mi}{1}\PYG{p}{,}\PYG{n}{description}\PYG{o}{=} \PYG{l+s+s1}{\PYGZsq{}}\PYG{l+s+s1}{\PYGZam{}Delta; t (day)}\PYG{l+s+s1}{\PYGZsq{}}\PYG{p}{,}\PYG{n}{disabled}\PYG{o}{=}\PYG{k+kc}{False}\PYG{p}{)}

\PYG{n}{A} \PYG{o}{=} \PYG{n}{widgets}\PYG{o}{.}\PYG{n}{BoundedFloatText}\PYG{p}{(}\PYG{n}{value}\PYG{o}{=}\PYG{l+m+mi}{100}\PYG{p}{,}\PYG{n+nb}{min}\PYG{o}{=}\PYG{l+m+mi}{0}\PYG{p}{,}\PYG{n+nb}{max}\PYG{o}{=}\PYG{l+m+mf}{1000.0}\PYG{p}{,}\PYG{n}{step}\PYG{o}{=}\PYG{l+m+mi}{1}\PYG{p}{,}\PYG{n}{description}\PYG{o}{=}\PYG{l+s+s1}{\PYGZsq{}}\PYG{l+s+s1}{M\PYGZlt{}sub\PYGZgt{}A\PYGZlt{}/sub\PYGZgt{} (kg)}\PYG{l+s+s1}{\PYGZsq{}}\PYG{p}{,}\PYG{n}{disabled}\PYG{o}{=}\PYG{k+kc}{False}\PYG{p}{)}

\PYG{n}{B} \PYG{o}{=} \PYG{n}{widgets}\PYG{o}{.}\PYG{n}{BoundedFloatText}\PYG{p}{(}\PYG{n}{value}\PYG{o}{=}\PYG{l+m+mi}{5}\PYG{p}{,}\PYG{n+nb}{min}\PYG{o}{=}\PYG{l+m+mi}{0}\PYG{p}{,}\PYG{n+nb}{max}\PYG{o}{=}\PYG{l+m+mf}{1000.0}\PYG{p}{,}\PYG{n}{step}\PYG{o}{=}\PYG{l+m+mi}{1}\PYG{p}{,}\PYG{n}{description}\PYG{o}{=}\PYG{l+s+s1}{\PYGZsq{}}\PYG{l+s+s1}{M\PYGZlt{}sub\PYGZgt{}B\PYGZlt{}/sub\PYGZgt{} (kg)}\PYG{l+s+s1}{\PYGZsq{}}\PYG{p}{,}\PYG{n}{disabled}\PYG{o}{=}\PYG{k+kc}{False}\PYG{p}{)}

\PYG{n}{C} \PYG{o}{=} \PYG{n}{widgets}\PYG{o}{.}\PYG{n}{BoundedFloatText}\PYG{p}{(}\PYG{n}{value}\PYG{o}{=}\PYG{l+m+mi}{10}\PYG{p}{,}\PYG{n+nb}{min}\PYG{o}{=}\PYG{l+m+mi}{0}\PYG{p}{,}\PYG{n+nb}{max}\PYG{o}{=}\PYG{l+m+mi}{1000}\PYG{p}{,}\PYG{n}{step}\PYG{o}{=}\PYG{l+m+mf}{0.1}\PYG{p}{,}\PYG{n}{description}\PYG{o}{=}\PYG{l+s+s1}{\PYGZsq{}}\PYG{l+s+s1}{M\PYGZlt{}sub\PYGZgt{}C\PYGZlt{}/sub\PYGZgt{} (kg)}\PYG{l+s+s1}{\PYGZsq{}}\PYG{p}{,}\PYG{n}{disabled}\PYG{o}{=}\PYG{k+kc}{False}\PYG{p}{)}

\PYG{n}{RA} \PYG{o}{=} \PYG{n}{widgets}\PYG{o}{.}\PYG{n}{BoundedFloatText}\PYG{p}{(}\PYG{n}{value}\PYG{o}{=}\PYG{l+m+mf}{0.2}\PYG{p}{,}\PYG{n+nb}{min}\PYG{o}{=}\PYG{l+m+mi}{0}\PYG{p}{,}\PYG{n+nb}{max}\PYG{o}{=}\PYG{l+m+mi}{100}\PYG{p}{,}\PYG{n}{step}\PYG{o}{=}\PYG{l+m+mf}{0.1}\PYG{p}{,}\PYG{n}{description}\PYG{o}{=}\PYG{l+s+s1}{\PYGZsq{}}\PYG{l+s+s1}{R\PYGZlt{}sub\PYGZgt{}A\PYGZlt{}/sub\PYGZgt{} (day\PYGZlt{}sup\PYGZgt{}\PYGZhy{}1 \PYGZlt{}/sup\PYGZgt{})}\PYG{l+s+s1}{\PYGZsq{}}\PYG{p}{,}\PYG{n}{disabled}\PYG{o}{=}\PYG{k+kc}{False}\PYG{p}{)}

\PYG{n}{RB} \PYG{o}{=} \PYG{n}{widgets}\PYG{o}{.}\PYG{n}{BoundedFloatText}\PYG{p}{(}\PYG{n}{value}\PYG{o}{=}\PYG{l+m+mf}{0.2}\PYG{p}{,}\PYG{n+nb}{min}\PYG{o}{=}\PYG{l+m+mi}{0}\PYG{p}{,}\PYG{n+nb}{max}\PYG{o}{=}\PYG{l+m+mi}{100}\PYG{p}{,}\PYG{n}{step}\PYG{o}{=}\PYG{l+m+mf}{0.1}\PYG{p}{,}\PYG{n}{description}\PYG{o}{=}\PYG{l+s+s1}{\PYGZsq{}}\PYG{l+s+s1}{R\PYGZlt{}sub\PYGZgt{}B\PYGZlt{}/sub\PYGZgt{} (day\PYGZlt{}sup\PYGZgt{}\PYGZhy{}1 \PYGZlt{}/sup\PYGZgt{})}\PYG{l+s+s1}{\PYGZsq{}}\PYG{p}{,}\PYG{n}{disabled}\PYG{o}{=}\PYG{k+kc}{False}\PYG{p}{)}


\PYG{n}{interactive\PYGZus{}plot} \PYG{o}{=} \PYG{n}{widgets}\PYG{o}{.}\PYG{n}{interactive}\PYG{p}{(}\PYG{n}{mass\PYGZus{}bal}\PYG{p}{,} \PYG{n}{n\PYGZus{}simulation} \PYG{o}{=} \PYG{n}{N}\PYG{p}{,} \PYG{n}{MA}\PYG{o}{=}\PYG{n}{A}\PYG{p}{,} \PYG{n}{MB}\PYG{o}{=}\PYG{n}{B}\PYG{p}{,} \PYG{n}{MC}\PYG{o}{=}\PYG{n}{C}\PYG{p}{,} \PYG{n}{R\PYGZus{}A}\PYG{o}{=}\PYG{n}{RA}\PYG{p}{,} \PYG{n}{R\PYGZus{}B}\PYG{o}{=}\PYG{n}{RB}\PYG{p}{,}\PYG{p}{)}
\PYG{n}{output} \PYG{o}{=} \PYG{n}{interactive\PYGZus{}plot}\PYG{o}{.}\PYG{n}{children}\PYG{p}{[}\PYG{o}{\PYGZhy{}}\PYG{l+m+mi}{1}\PYG{p}{]}  
\PYG{c+c1}{\PYGZsh{}output.layout.height = \PYGZsq{}350px\PYGZsq{}}
\PYG{n}{interactive\PYGZus{}plot}
\end{sphinxVerbatim}

\end{sphinxuseclass}\end{sphinxVerbatimInput}
\begin{sphinxVerbatimOutput}

\begin{sphinxuseclass}{cell_output}
\begin{sphinxVerbatim}[commandchars=\\\{\}]
interactive(children=(BoundedIntText(value=20, description=\PYGZsq{}\PYGZam{}Delta; t (day)\PYGZsq{}), BoundedFloatText(value=100.0, d…
\end{sphinxVerbatim}

\end{sphinxuseclass}\end{sphinxVerbatimOutput}

\end{sphinxuseclass}
\sphinxstepscope


\chapter{Simulating Seive Analysis}
\label{\detokenize{content/tools/sieve_analysis:simulating-seive-analysis}}\label{\detokenize{content/tools/sieve_analysis::doc}}

\section{How to use the tool?}
\label{\detokenize{content/tools/sieve_analysis:how-to-use-the-tool}}\begin{enumerate}
\sphinxsetlistlabels{\arabic}{enumi}{enumii}{}{.}%
\item {} 
\sphinxAtStartPar
Go to the Binder by clicking the rocket button (top\sphinxhyphen{}right of the page)

\item {} 
\sphinxAtStartPar
Execute the code cell

\item {} 
\sphinxAtStartPar
Change the values of different quantities in the box and click the \sphinxstylestrong{run interact}.

\item {} 
\sphinxAtStartPar
From the resulting figure, using your mouse and selecting points in the figure obtain d10 and d60.

\item {} 
\sphinxAtStartPar
Execute the second code\sphinxhyphen{}cell and provide d10, d60 and temperature date

\item {} 
\sphinxAtStartPar
Click the exectute button.

\item {} 
\sphinxAtStartPar
For re\sphinxhyphen{}simulations \sphinxhyphen{} changes the input values in the boxes and click the “\sphinxstylestrong{run interact}” button.

\end{enumerate}

\sphinxAtStartPar
This tool can also be downloaded and run locally. For that download the \sphinxstyleemphasis{deacy.ipynb} file and execute the process in any editor (e.g., JUPYTER notebook, JUPYTER lab) that is able to read and execute this file\sphinxhyphen{}type.

\sphinxAtStartPar
The code may also be executed in the book page.

\sphinxAtStartPar
The codes are licensed under CC by 4.0 \sphinxhref{https://creativecommons.org/licenses/by/4.0/deed.en}{(use anyways, but acknowledge the original work)}

\begin{sphinxuseclass}{cell}\begin{sphinxVerbatimInput}

\begin{sphinxuseclass}{cell_input}
\begin{sphinxVerbatim}[commandchars=\\\{\}]
\PYG{c+c1}{\PYGZsh{} used Python library}
\PYG{k+kn}{import} \PYG{n+nn}{numpy} \PYG{k}{as} \PYG{n+nn}{np} \PYG{c+c1}{\PYGZsh{} for calculation }
\PYG{k+kn}{import} \PYG{n+nn}{matplotlib}\PYG{n+nn}{.}\PYG{n+nn}{pyplot} \PYG{k}{as} \PYG{n+nn}{plt}  \PYG{c+c1}{\PYGZsh{} for plotting}
\PYG{k+kn}{import} \PYG{n+nn}{pandas} \PYG{k}{as} \PYG{n+nn}{pd}  \PYG{c+c1}{\PYGZsh{} for data table}
\PYG{k+kn}{import} \PYG{n+nn}{ipywidgets} \PYG{k}{as} \PYG{n+nn}{widgets} \PYG{c+c1}{\PYGZsh{} for widgets}
\PYG{c+c1}{\PYGZsh{}\PYGZpc{}matplotlib widget}
\PYG{k+kn}{import} \PYG{n+nn}{warnings}\PYG{p}{;} \PYG{n}{warnings}\PYG{o}{.}\PYG{n}{simplefilter}\PYG{p}{(}\PYG{l+s+s1}{\PYGZsq{}}\PYG{l+s+s1}{ignore}\PYG{l+s+s1}{\PYGZsq{}}\PYG{p}{)}
\PYG{k+kn}{from} \PYG{n+nn}{scipy}\PYG{n+nn}{.}\PYG{n+nn}{interpolate} \PYG{k+kn}{import} \PYG{n}{interp1d}\PYG{p}{,} \PYG{n}{Rbf} \PYG{p}{,} \PYG{n}{UnivariateSpline}\PYG{p}{,}\PYG{n}{CubicSpline}
\PYG{k+kn}{from} \PYG{n+nn}{scipy} \PYG{k+kn}{import} \PYG{n}{interpolate}

\PYG{c+c1}{\PYGZsh{}from scipy.interpolate import Rbf}

\PYG{n+nb}{print}\PYG{p}{(}\PYG{l+s+s2}{\PYGZdq{}}\PYG{l+s+s2}{Please provide the seive data in the boxes:  }\PYG{l+s+s2}{\PYGZdq{}}\PYG{p}{)}

\PYG{k}{def} \PYG{n+nf}{SA}\PYG{p}{(}\PYG{n}{mu}\PYG{p}{,} \PYG{n}{m1}\PYG{p}{,} \PYG{n}{m2}\PYG{p}{,} \PYG{n}{m3}\PYG{p}{,} \PYG{n}{m4}\PYG{p}{,} \PYG{n}{ml}\PYG{p}{,}\PYG{n}{perdat}\PYG{p}{)}\PYG{p}{:}
    \PYG{n}{dia} \PYG{o}{=} \PYG{p}{[}\PYG{l+m+mi}{6}\PYG{p}{,}\PYG{l+m+mi}{2}\PYG{p}{,}\PYG{l+m+mf}{0.6}\PYG{p}{,}\PYG{l+m+mf}{0.2}\PYG{p}{,} \PYG{l+m+mf}{0.06}\PYG{p}{,} \PYG{l+m+mf}{0.01}\PYG{p}{]} \PYG{c+c1}{\PYGZsh{} mm, diameter \PYGZlt{}0.06 (cup)= 0.01, \PYGZgt{}2 = 6}
    \PYG{n}{mass} \PYG{o}{=} \PYG{p}{[}\PYG{n}{mu}\PYG{p}{,} \PYG{n}{m1}\PYG{p}{,} \PYG{n}{m2}\PYG{p}{,} \PYG{n}{m3}\PYG{p}{,} \PYG{n}{m4}\PYG{p}{,} \PYG{n}{ml}\PYG{p}{]} \PYG{c+c1}{\PYGZsh{} g, the residue in seive }
    \PYG{n}{Total\PYGZus{}mass} \PYG{o}{=} \PYG{n}{np}\PYG{o}{.}\PYG{n}{sum}\PYG{p}{(}\PYG{n}{mass}\PYG{p}{)}  \PYG{c+c1}{\PYGZsh{} add the mass column to get total mass}
    \PYG{n}{retain\PYGZus{}per} \PYG{o}{=} \PYG{n}{np}\PYG{o}{.}\PYG{n}{round}\PYG{p}{(}\PYG{n}{mass}\PYG{o}{/}\PYG{n}{Total\PYGZus{}mass}\PYG{o}{*}\PYG{l+m+mi}{100}\PYG{p}{,}\PYG{l+m+mi}{3}\PYG{p}{)}   \PYG{c+c1}{\PYGZsh{} retain percentage}
    \PYG{n}{retain\PYGZus{}per\PYGZus{}cumsum} \PYG{o}{=} \PYG{n}{np}\PYG{o}{.}\PYG{n}{round}\PYG{p}{(}\PYG{n}{np}\PYG{o}{.}\PYG{n}{cumsum}\PYG{p}{(}\PYG{n}{retain\PYGZus{}per}\PYG{p}{)}\PYG{p}{,}\PYG{l+m+mi}{3}\PYG{p}{)} \PYG{c+c1}{\PYGZsh{} get the cummulative sum of the reatined}
    \PYG{n}{passing\PYGZus{}per} \PYG{o}{=} \PYG{n}{np}\PYG{o}{.}\PYG{n}{round}\PYG{p}{(}\PYG{l+m+mi}{100} \PYG{o}{\PYGZhy{}} \PYG{n}{retain\PYGZus{}per\PYGZus{}cumsum}\PYG{p}{,} \PYG{l+m+mi}{3}\PYG{p}{)} \PYG{c+c1}{\PYGZsh{} substract 100\PYGZhy{}cummsum to get passing \PYGZpc{}}
    \PYG{n}{data} \PYG{o}{=} \PYG{p}{\PYGZob{}}\PYG{l+s+s2}{\PYGZdq{}}\PYG{l+s+s2}{mesh diameter [mm]}\PYG{l+s+s2}{\PYGZdq{}}\PYG{p}{:} \PYG{n}{dia}\PYG{p}{,} \PYG{l+s+s2}{\PYGZdq{}}\PYG{l+s+s2}{residue in the sieve [g]}\PYG{l+s+s2}{\PYGZdq{}}\PYG{p}{:} \PYG{n}{mass}\PYG{p}{,} \PYG{l+s+s2}{\PYGZdq{}}\PYG{l+s+s2}{Σtotal}\PYG{l+s+s2}{\PYGZdq{}}\PYG{p}{:} \PYG{n}{retain\PYGZus{}per}\PYG{p}{,} \PYG{l+s+s2}{\PYGZdq{}}\PYG{l+s+s2}{Σ/Σtotal}\PYG{l+s+s2}{\PYGZdq{}}\PYG{p}{:} \PYG{n}{passing\PYGZus{}per} \PYG{p}{\PYGZcb{}}

    \PYG{n}{df1}\PYG{o}{=} \PYG{n}{pd}\PYG{o}{.}\PYG{n}{DataFrame}\PYG{p}{(}\PYG{n}{data}\PYG{p}{)}
    \PYG{n}{df1} \PYG{o}{=} \PYG{n}{df1}\PYG{o}{.}\PYG{n}{set\PYGZus{}index}\PYG{p}{(}\PYG{l+s+s2}{\PYGZdq{}}\PYG{l+s+s2}{mesh diameter [mm]}\PYG{l+s+s2}{\PYGZdq{}}\PYG{p}{)}
    \PYG{n+nb}{print}\PYG{p}{(}\PYG{n}{df1}\PYG{p}{)}

    \PYG{n}{plt}\PYG{o}{.}\PYG{n}{rcParams}\PYG{p}{[}\PYG{l+s+s1}{\PYGZsq{}}\PYG{l+s+s1}{axes.linewidth}\PYG{l+s+s1}{\PYGZsq{}}\PYG{p}{]}\PYG{o}{=}\PYG{l+m+mi}{2}
    \PYG{c+c1}{\PYGZsh{}plt.rcParams[\PYGZdq{}axes.edgecolor\PYGZdq{}]=\PYGZsq{}white\PYGZsq{}}
    \PYG{n}{plt}\PYG{o}{.}\PYG{n}{rcParams}\PYG{p}{[}\PYG{l+s+s1}{\PYGZsq{}}\PYG{l+s+s1}{grid.linestyle}\PYG{l+s+s1}{\PYGZsq{}}\PYG{p}{]}\PYG{o}{=}\PYG{l+s+s1}{\PYGZsq{}}\PYG{l+s+s1}{\PYGZhy{}\PYGZhy{}}\PYG{l+s+s1}{\PYGZsq{}}
    \PYG{n}{plt}\PYG{o}{.}\PYG{n}{rcParams}\PYG{p}{[}\PYG{l+s+s1}{\PYGZsq{}}\PYG{l+s+s1}{grid.linewidth}\PYG{l+s+s1}{\PYGZsq{}}\PYG{p}{]}\PYG{o}{=}\PYG{l+m+mi}{1}
    \PYG{n}{x} \PYG{o}{=} \PYG{n}{np}\PYG{o}{.}\PYG{n}{append}\PYG{p}{(}\PYG{p}{[}\PYG{l+m+mi}{20}\PYG{p}{]}\PYG{p}{,}\PYG{n}{dia}\PYG{p}{)} \PYG{c+c1}{\PYGZsh{} adding data to extend over 6 mm dia}
    \PYG{n}{y} \PYG{o}{=} \PYG{n}{np}\PYG{o}{.}\PYG{n}{append}\PYG{p}{(}\PYG{p}{[}\PYG{l+m+mi}{100}\PYG{p}{]}\PYG{p}{,}\PYG{n}{passing\PYGZus{}per}\PYG{p}{)} \PYG{c+c1}{\PYGZsh{} adding 100\PYGZpc{} to plot}
    
    
    \PYG{c+c1}{\PYGZsh{}\PYGZsh{}\PYGZsh{}\PYGZsh{}\PYGZsh{}\PYGZsh{}\PYGZsh{}\PYGZsh{}\PYGZsh{}\PYGZsh{}\PYGZsh{}\PYGZsh{}\PYGZsh{}\PYGZsh{}\PYGZsh{}\PYGZsh{}\PYGZsh{}\PYGZsh{}\PYGZsh{}\PYGZsh{}\PYGZsh{}\PYGZsh{}\PYGZsh{}\PYGZsh{}\PYGZsh{}\PYGZsh{}\PYGZsh{}\PYGZsh{}\PYGZsh{}\PYGZsh{}\PYGZsh{}\PYGZsh{}\PYGZsh{}\PYGZsh{}\PYGZsh{}\PYGZsh{}\PYGZsh{}\PYGZsh{}\PYGZsh{}\PYGZsh{}\PYGZsh{}Interpolating }
    \PYG{c+c1}{\PYGZsh{}interp\PYGZus{}func = interp1d(y,x)}
    \PYG{n}{interp\PYGZus{}func} \PYG{o}{=} \PYG{n}{CubicSpline}\PYG{p}{(}\PYG{n}{y}\PYG{p}{,}\PYG{n}{x}\PYG{p}{)}
    
    \PYG{n}{Dd60} \PYG{o}{=} \PYG{n}{interp\PYGZus{}func}\PYG{p}{(}\PYG{l+m+mi}{60}\PYG{p}{)}
    \PYG{n}{Dd10}\PYG{o}{=}\PYG{n}{interp\PYGZus{}func}\PYG{p}{(}\PYG{l+m+mi}{10}\PYG{p}{)}
    \PYG{n}{Ddx}\PYG{o}{=}\PYG{n}{interp\PYGZus{}func}\PYG{p}{(}\PYG{n}{perdat}\PYG{p}{)}

    \PYG{c+c1}{\PYGZsh{}\PYGZsh{}\PYGZsh{}\PYGZsh{}\PYGZsh{}\PYGZsh{}\PYGZsh{}\PYGZsh{}\PYGZsh{}\PYGZsh{}\PYGZsh{}\PYGZsh{}\PYGZsh{}\PYGZsh{}\PYGZsh{}\PYGZsh{}\PYGZsh{}\PYGZsh{}\PYGZsh{}\PYGZsh{}\PYGZsh{}\PYGZsh{}\PYGZsh{}\PYGZsh{}\PYGZsh{}\PYGZsh{}\PYGZsh{}\PYGZsh{}\PYGZsh{}\PYGZsh{}\PYGZsh{}\PYGZsh{}\PYGZsh{}\PYGZsh{}\PYGZsh{}\PYGZsh{}\PYGZsh{}\PYGZsh{}\PYGZsh{}\PYGZsh{}\PYGZsh{} Printing the D60 D10}
    \PYG{n+nb}{print}\PYG{p}{(}\PYG{l+s+s1}{\PYGZsq{}}\PYG{l+s+se}{\PYGZbs{}n}\PYG{l+s+s1}{\PYGZsq{}}\PYG{p}{,}\PYG{l+s+s1}{\PYGZsq{}}\PYG{l+s+se}{\PYGZbs{}n}\PYG{l+s+s1}{\PYGZsq{}}\PYG{p}{)}
    \PYG{n+nb}{print}\PYG{p}{(}\PYG{l+s+s2}{\PYGZdq{}}\PYG{l+s+s2}{d60 =}\PYG{l+s+s2}{\PYGZdq{}}\PYG{p}{,}\PYG{l+s+s2}{\PYGZdq{}}\PYG{l+s+si}{\PYGZpc{}.2f}\PYG{l+s+s2}{\PYGZdq{}} \PYG{o}{\PYGZpc{}} \PYG{n}{Dd60}\PYG{p}{)}
    \PYG{n+nb}{print}\PYG{p}{(}\PYG{l+s+s2}{\PYGZdq{}}\PYG{l+s+s2}{d10 =}\PYG{l+s+s2}{\PYGZdq{}}\PYG{p}{,}\PYG{l+s+s2}{\PYGZdq{}}\PYG{l+s+si}{\PYGZpc{}.2f}\PYG{l+s+s2}{\PYGZdq{}} \PYG{o}{\PYGZpc{}} \PYG{n}{Dd10}\PYG{p}{)}
    \PYG{n+nb}{print} \PYG{p}{(}\PYG{l+s+s1}{\PYGZsq{}}\PYG{l+s+s1}{d}\PYG{l+s+si}{\PYGZpc{}d}\PYG{l+s+s1}{ = }\PYG{l+s+si}{\PYGZpc{}.2f}\PYG{l+s+s1}{\PYGZsq{}} \PYG{o}{\PYGZpc{}} \PYG{p}{(}\PYG{n}{perdat}\PYG{p}{,} \PYG{n}{Ddx}\PYG{p}{)}\PYG{p}{)}
    

    
    
    \PYG{n}{fig}\PYG{p}{,} \PYG{n}{ax} \PYG{o}{=} \PYG{n}{plt}\PYG{o}{.}\PYG{n}{subplots}\PYG{p}{(}\PYG{n}{figsize}\PYG{o}{=}\PYG{p}{(}\PYG{l+m+mi}{15}\PYG{p}{,}\PYG{l+m+mi}{10}\PYG{p}{)}\PYG{p}{)}
    \PYG{n}{fig}\PYG{o}{.}\PYG{n}{canvas}\PYG{o}{.}\PYG{n}{header\PYGZus{}visible} \PYG{o}{=} \PYG{k+kc}{False}
    \PYG{n}{plt}\PYG{o}{.}\PYG{n}{semilogx}\PYG{p}{(}\PYG{n}{x}\PYG{p}{,} \PYG{n}{y}\PYG{p}{,} \PYG{l+s+s1}{\PYGZsq{}}\PYG{l+s+s1}{x\PYGZhy{}}\PYG{l+s+s1}{\PYGZsq{}}\PYG{p}{,} \PYG{n}{color}\PYG{o}{=}\PYG{l+s+s1}{\PYGZsq{}}\PYG{l+s+s1}{red}\PYG{l+s+s1}{\PYGZsq{}}\PYG{p}{)}  
    \PYG{n}{tics}\PYG{o}{=}\PYG{n}{x}\PYG{o}{.}\PYG{n}{tolist}\PYG{p}{(}\PYG{p}{)}

    \PYG{n}{ax}\PYG{o}{.}\PYG{n}{grid}\PYG{p}{(}\PYG{n}{which}\PYG{o}{=}\PYG{l+s+s1}{\PYGZsq{}}\PYG{l+s+s1}{major}\PYG{l+s+s1}{\PYGZsq{}}\PYG{p}{,} \PYG{n}{color}\PYG{o}{=}\PYG{l+s+s1}{\PYGZsq{}}\PYG{l+s+s1}{k}\PYG{l+s+s1}{\PYGZsq{}}\PYG{p}{,} \PYG{n}{alpha}\PYG{o}{=}\PYG{l+m+mf}{0.7}\PYG{p}{)} 
    \PYG{n}{ax}\PYG{o}{.}\PYG{n}{grid}\PYG{p}{(}\PYG{n}{which}\PYG{o}{=}\PYG{l+s+s1}{\PYGZsq{}}\PYG{l+s+s1}{minor}\PYG{l+s+s1}{\PYGZsq{}}\PYG{p}{,} \PYG{n}{color}\PYG{o}{=}\PYG{l+s+s1}{\PYGZsq{}}\PYG{l+s+s1}{k}\PYG{l+s+s1}{\PYGZsq{}}\PYG{p}{,} \PYG{n}{alpha}\PYG{o}{=}\PYG{l+m+mf}{0.3}\PYG{p}{)}
    \PYG{n}{ax}\PYG{o}{.}\PYG{n}{set\PYGZus{}xticks}\PYG{p}{(}\PYG{n}{x}\PYG{p}{)}\PYG{p}{;}  
    \PYG{n}{ax}\PYG{o}{.}\PYG{n}{set\PYGZus{}yticks}\PYG{p}{(}\PYG{n}{np}\PYG{o}{.}\PYG{n}{arange}\PYG{p}{(}\PYG{l+m+mi}{0}\PYG{p}{,}\PYG{l+m+mi}{110}\PYG{p}{,}\PYG{l+m+mi}{10}\PYG{p}{)}\PYG{p}{)}\PYG{p}{;}
    \PYG{n}{plt}\PYG{o}{.}\PYG{n}{title}\PYG{p}{(}\PYG{l+s+s1}{\PYGZsq{}}\PYG{l+s+s1}{grain size distribution}\PYG{l+s+s1}{\PYGZsq{}}\PYG{p}{)}\PYG{p}{;}
    \PYG{n}{plt}\PYG{o}{.}\PYG{n}{xlabel}\PYG{p}{(}\PYG{l+s+s1}{\PYGZsq{}}\PYG{l+s+s1}{grain size d [mm]}\PYG{l+s+s1}{\PYGZsq{}}\PYG{p}{)}\PYG{p}{;}
    \PYG{n}{plt}\PYG{o}{.}\PYG{n}{ylabel}\PYG{p}{(}\PYG{l+s+s1}{\PYGZsq{}}\PYG{l+s+s1}{grain fraction \PYGZlt{} d ins }\PYG{l+s+si}{\PYGZpc{} o}\PYG{l+s+s1}{f total mass}\PYG{l+s+s1}{\PYGZsq{}}\PYG{p}{)}\PYG{p}{;}

    \PYG{c+c1}{\PYGZsh{}\PYGZsh{}\PYGZsh{}\PYGZsh{}\PYGZsh{}\PYGZsh{}\PYGZsh{}\PYGZsh{}\PYGZsh{}\PYGZsh{}\PYGZsh{}\PYGZsh{}\PYGZsh{}\PYGZsh{}\PYGZsh{}\PYGZsh{}\PYGZsh{}\PYGZsh{}\PYGZsh{}\PYGZsh{}\PYGZsh{}\PYGZsh{}\PYGZsh{}\PYGZsh{}\PYGZsh{}\PYGZsh{}\PYGZsh{}\PYGZsh{}\PYGZsh{}\PYGZsh{}\PYGZsh{}\PYGZsh{}\PYGZsh{}\PYGZsh{}\PYGZsh{}\PYGZsh{}\PYGZsh{}\PYGZsh{}\PYGZsh{}\PYGZsh{}\PYGZsh{}\PYGZsh{}\PYGZsh{}\PYGZsh{}\PYGZsh{}\PYGZsh{}\PYGZsh{}\PYGZsh{}\PYGZsh{}\PYGZsh{}\PYGZsh{}\PYGZsh{}\PYGZsh{}\PYGZsh{}\PYGZsh{}\PYGZsh{}\PYGZsh{}\PYGZsh{}\PYGZsh{}\PYGZsh{}\PYGZsh{}\PYGZsh{}\PYGZsh{}\PYGZsh{}\PYGZsh{}\PYGZsh{}\PYGZsh{}\PYGZsh{}\PYGZsh{}\PYGZsh{} Plotting Interpolated Lines}
    \PYG{n}{plt}\PYG{o}{.}\PYG{n}{plot}\PYG{p}{(}\PYG{p}{[}\PYG{l+m+mi}{0}\PYG{p}{,}\PYG{n}{Dd60}\PYG{p}{,}\PYG{n}{Dd60}\PYG{p}{,}\PYG{n}{Dd60}\PYG{p}{]}\PYG{p}{,}\PYG{p}{[}\PYG{l+m+mi}{60}\PYG{p}{,}\PYG{l+m+mi}{60}\PYG{p}{,}\PYG{l+m+mi}{0}\PYG{p}{,}\PYG{l+m+mi}{60}\PYG{p}{]}\PYG{p}{,}\PYG{n}{ls}\PYG{o}{=}\PYG{l+s+s1}{\PYGZsq{}}\PYG{l+s+s1}{\PYGZhy{}}\PYG{l+s+s1}{\PYGZsq{}}\PYG{p}{,}\PYG{n}{color}\PYG{o}{=}\PYG{l+s+s1}{\PYGZsq{}}\PYG{l+s+s1}{g}\PYG{l+s+s1}{\PYGZsq{}}\PYG{p}{)}\PYG{p}{;}
    \PYG{n}{plt}\PYG{o}{.}\PYG{n}{plot}\PYG{p}{(}\PYG{p}{[}\PYG{l+m+mi}{0}\PYG{p}{,}\PYG{n}{Dd10}\PYG{p}{,}\PYG{n}{Dd10}\PYG{p}{,}\PYG{n}{Dd10}\PYG{p}{]}\PYG{p}{,}\PYG{p}{[}\PYG{l+m+mi}{10}\PYG{p}{,}\PYG{l+m+mi}{10}\PYG{p}{,}\PYG{l+m+mi}{0}\PYG{p}{,}\PYG{l+m+mi}{10}\PYG{p}{]}\PYG{p}{,}\PYG{n}{ls}\PYG{o}{=}\PYG{l+s+s1}{\PYGZsq{}}\PYG{l+s+s1}{\PYGZhy{}}\PYG{l+s+s1}{\PYGZsq{}}\PYG{p}{,}\PYG{n}{color}\PYG{o}{=}\PYG{l+s+s1}{\PYGZsq{}}\PYG{l+s+s1}{r}\PYG{l+s+s1}{\PYGZsq{}}\PYG{p}{)}\PYG{p}{;}
    \PYG{n}{plt}\PYG{o}{.}\PYG{n}{plot}\PYG{p}{(}\PYG{p}{[}\PYG{l+m+mi}{0}\PYG{p}{,}\PYG{n}{Ddx}\PYG{p}{,}\PYG{n}{Ddx}\PYG{p}{,}\PYG{n}{Ddx}\PYG{p}{]}\PYG{p}{,}\PYG{p}{[}\PYG{n}{perdat}\PYG{p}{,}\PYG{n}{perdat}\PYG{p}{,}\PYG{l+m+mi}{0}\PYG{p}{,}\PYG{n}{perdat}\PYG{p}{]}\PYG{p}{,}\PYG{n}{ls}\PYG{o}{=}\PYG{l+s+s1}{\PYGZsq{}}\PYG{l+s+s1}{\PYGZhy{}}\PYG{l+s+s1}{\PYGZsq{}}\PYG{p}{,}\PYG{n}{color}\PYG{o}{=}\PYG{l+s+s1}{\PYGZsq{}}\PYG{l+s+s1}{r}\PYG{l+s+s1}{\PYGZsq{}}\PYG{p}{)}
    \PYG{n}{ax}\PYG{o}{.}\PYG{n}{set\PYGZus{}xlim}\PYG{p}{(}\PYG{l+m+mi}{0}\PYG{p}{,} \PYG{l+m+mi}{30}\PYG{p}{)}
    \PYG{k+kn}{from} \PYG{n+nn}{matplotlib}\PYG{n+nn}{.}\PYG{n+nn}{ticker} \PYG{k+kn}{import} \PYG{n}{StrMethodFormatter}
    \PYG{n}{ax}\PYG{o}{.}\PYG{n}{xaxis}\PYG{o}{.}\PYG{n}{set\PYGZus{}major\PYGZus{}formatter}\PYG{p}{(}\PYG{n}{StrMethodFormatter}\PYG{p}{(}\PYG{l+s+s1}{\PYGZsq{}}\PYG{l+s+si}{\PYGZob{}x:0.2f\PYGZcb{}}\PYG{l+s+s1}{\PYGZsq{}}\PYG{p}{)}\PYG{p}{)}
    
    \PYG{k}{return}\PYG{p}{(}\PYG{n}{Dd60}\PYG{p}{,}\PYG{n}{Dd10}\PYG{p}{)}
    
\PYG{n}{style} \PYG{o}{=} \PYG{p}{\PYGZob{}}\PYG{l+s+s1}{\PYGZsq{}}\PYG{l+s+s1}{description\PYGZus{}width}\PYG{l+s+s1}{\PYGZsq{}}\PYG{p}{:} \PYG{l+s+s1}{\PYGZsq{}}\PYG{l+s+s1}{200px}\PYG{l+s+s1}{\PYGZsq{}}\PYG{p}{\PYGZcb{}}    

\PYG{n}{Inter}\PYG{o}{=}\PYG{n}{widgets}\PYG{o}{.}\PYG{n}{interact\PYGZus{}manual}\PYG{p}{(}\PYG{n}{SA}\PYG{p}{,} 
                       \PYG{n}{mu}\PYG{o}{=} \PYG{n}{widgets}\PYG{o}{.}\PYG{n}{FloatText}\PYG{p}{(}\PYG{n}{description}\PYG{o}{=}\PYG{l+s+s2}{\PYGZdq{}}\PYG{l+s+s2}{6 mm}\PYG{l+s+s2}{\PYGZdq{}}\PYG{p}{,} \PYG{n}{style}\PYG{o}{=}\PYG{n}{style}\PYG{p}{)}\PYG{p}{,}
                       \PYG{n}{m1}\PYG{o}{=} \PYG{n}{widgets}\PYG{o}{.}\PYG{n}{FloatText}\PYG{p}{(}\PYG{n}{description}\PYG{o}{=}\PYG{l+s+s2}{\PYGZdq{}}\PYG{l+s+s2}{2 mm}\PYG{l+s+s2}{\PYGZdq{}}\PYG{p}{,}\PYG{n}{style}\PYG{o}{=}\PYG{n}{style}\PYG{p}{)}\PYG{p}{,}
                       \PYG{n}{m2}\PYG{o}{=} \PYG{n}{widgets}\PYG{o}{.}\PYG{n}{FloatText}\PYG{p}{(}\PYG{n}{description}\PYG{o}{=}\PYG{l+s+s2}{\PYGZdq{}}\PYG{l+s+s2}{0.6 mm}\PYG{l+s+s2}{\PYGZdq{}}\PYG{p}{,} \PYG{n}{style}\PYG{o}{=}\PYG{n}{style}\PYG{p}{)}\PYG{p}{,}
                       \PYG{n}{m3}\PYG{o}{=} \PYG{n}{widgets}\PYG{o}{.}\PYG{n}{FloatText}\PYG{p}{(}\PYG{n}{description}\PYG{o}{=}\PYG{l+s+s2}{\PYGZdq{}}\PYG{l+s+s2}{0.2 mm}\PYG{l+s+s2}{\PYGZdq{}}\PYG{p}{,} \PYG{n}{style}\PYG{o}{=}\PYG{n}{style}\PYG{p}{)}\PYG{p}{,}
                       \PYG{n}{m4}\PYG{o}{=} \PYG{n}{widgets}\PYG{o}{.}\PYG{n}{FloatText}\PYG{p}{(}\PYG{n}{description}\PYG{o}{=}\PYG{l+s+s2}{\PYGZdq{}}\PYG{l+s+s2}{0.06 mm}\PYG{l+s+s2}{\PYGZdq{}}\PYG{p}{,} \PYG{n}{style}\PYG{o}{=}\PYG{n}{style}\PYG{p}{)}\PYG{p}{,}
                       \PYG{n}{ml}\PYG{o}{=} \PYG{n}{widgets}\PYG{o}{.}\PYG{n}{FloatText}\PYG{p}{(}\PYG{n}{description}\PYG{o}{=}\PYG{l+s+s2}{\PYGZdq{}}\PYG{l+s+s2}{0.01 mm}\PYG{l+s+s2}{\PYGZdq{}}\PYG{p}{,} \PYG{n}{style}\PYG{o}{=}\PYG{n}{style}\PYG{p}{)}\PYG{p}{,}
                       \PYG{n}{perdat}\PYG{o}{=} \PYG{n}{widgets}\PYG{o}{.}\PYG{n}{FloatText}\PYG{p}{(}\PYG{n}{description}\PYG{o}{=}\PYG{l+s+s2}{\PYGZdq{}}\PYG{l+s+s2}{Enter Fineness }\PYG{l+s+s2}{\PYGZpc{}}\PYG{l+s+s2}{ }\PYG{l+s+s2}{\PYGZdq{}}\PYG{p}{,} \PYG{n}{style}\PYG{o}{=}\PYG{n}{style}\PYG{p}{)}     \PYG{p}{)}
\end{sphinxVerbatim}

\end{sphinxuseclass}\end{sphinxVerbatimInput}
\begin{sphinxVerbatimOutput}

\begin{sphinxuseclass}{cell_output}
\begin{sphinxVerbatim}[commandchars=\\\{\}]
Please provide the seive data in the boxes:  
\end{sphinxVerbatim}

\begin{sphinxVerbatim}[commandchars=\\\{\}]
interactive(children=(FloatText(value=0.0, description=\PYGZsq{}6 mm\PYGZsq{}, style=DescriptionStyle(description\PYGZus{}width=\PYGZsq{}200px…
\end{sphinxVerbatim}

\end{sphinxuseclass}\end{sphinxVerbatimOutput}

\end{sphinxuseclass}
\begin{sphinxuseclass}{cell}\begin{sphinxVerbatimInput}

\begin{sphinxuseclass}{cell_input}
\begin{sphinxVerbatim}[commandchars=\\\{\}]
\PYG{c+c1}{\PYGZsh{} used Python library}
\PYG{k+kn}{import} \PYG{n+nn}{numpy} \PYG{k}{as} \PYG{n+nn}{np} \PYG{c+c1}{\PYGZsh{} for calculation }
\PYG{k+kn}{import} \PYG{n+nn}{matplotlib}\PYG{n+nn}{.}\PYG{n+nn}{pyplot} \PYG{k}{as} \PYG{n+nn}{plt}  \PYG{c+c1}{\PYGZsh{} for plotting}
\PYG{k+kn}{import} \PYG{n+nn}{pandas} \PYG{k}{as} \PYG{n+nn}{pd}  \PYG{c+c1}{\PYGZsh{} for data table}
\PYG{k+kn}{import} \PYG{n+nn}{ipywidgets} \PYG{k}{as} \PYG{n+nn}{widgets} \PYG{c+c1}{\PYGZsh{} for widgets}
\PYG{c+c1}{\PYGZsh{}\PYGZpc{}matplotlib widget}
\PYG{k+kn}{import} \PYG{n+nn}{warnings}\PYG{p}{;} \PYG{n}{warnings}\PYG{o}{.}\PYG{n}{simplefilter}\PYG{p}{(}\PYG{l+s+s1}{\PYGZsq{}}\PYG{l+s+s1}{ignore}\PYG{l+s+s1}{\PYGZsq{}}\PYG{p}{)}
\PYG{k+kn}{from} \PYG{n+nn}{scipy}\PYG{n+nn}{.}\PYG{n+nn}{interpolate} \PYG{k+kn}{import} \PYG{n}{interp1d}\PYG{p}{,}\PYG{n}{CubicSpline}
\PYG{k+kn}{from} \PYG{n+nn}{scipy} \PYG{k+kn}{import} \PYG{n}{interpolate}

\PYG{c+c1}{\PYGZsh{}from scipy.interpolate import Rbf}

\PYG{n+nb}{print}\PYG{p}{(}\PYG{l+s+s2}{\PYGZdq{}}\PYG{l+s+s2}{Please provide the seive data in the boxes:  }\PYG{l+s+s2}{\PYGZdq{}}\PYG{p}{)}

\PYG{k}{def} \PYG{n+nf}{SA}\PYG{p}{(}\PYG{n}{mu}\PYG{p}{,} \PYG{n}{m1}\PYG{p}{,} \PYG{n}{m2}\PYG{p}{,} \PYG{n}{m3}\PYG{p}{,} \PYG{n}{m4}\PYG{p}{,} \PYG{n}{ml}\PYG{p}{,}\PYG{n}{perdat}\PYG{p}{)}\PYG{p}{:}
    \PYG{n}{dia} \PYG{o}{=} \PYG{p}{[}\PYG{l+m+mi}{6}\PYG{p}{,}\PYG{l+m+mi}{2}\PYG{p}{,}\PYG{l+m+mf}{0.6}\PYG{p}{,}\PYG{l+m+mf}{0.2}\PYG{p}{,} \PYG{l+m+mf}{0.06}\PYG{p}{,} \PYG{l+m+mf}{0.01}\PYG{p}{]} \PYG{c+c1}{\PYGZsh{} mm, diameter \PYGZlt{}0.06 (cup)= 0.01, \PYGZgt{}2 = 6}
    \PYG{n}{mass} \PYG{o}{=} \PYG{p}{[}\PYG{n}{mu}\PYG{p}{,} \PYG{n}{m1}\PYG{p}{,} \PYG{n}{m2}\PYG{p}{,} \PYG{n}{m3}\PYG{p}{,} \PYG{n}{m4}\PYG{p}{,} \PYG{n}{ml}\PYG{p}{]} \PYG{c+c1}{\PYGZsh{} g, the residue in seive }
    \PYG{n}{Total\PYGZus{}mass} \PYG{o}{=} \PYG{n}{np}\PYG{o}{.}\PYG{n}{sum}\PYG{p}{(}\PYG{n}{mass}\PYG{p}{)}  \PYG{c+c1}{\PYGZsh{} add the mass column to get total mass}
    \PYG{n}{retain\PYGZus{}per} \PYG{o}{=} \PYG{n}{np}\PYG{o}{.}\PYG{n}{round}\PYG{p}{(}\PYG{n}{mass}\PYG{o}{/}\PYG{n}{Total\PYGZus{}mass}\PYG{o}{*}\PYG{l+m+mi}{100}\PYG{p}{,}\PYG{l+m+mi}{3}\PYG{p}{)}   \PYG{c+c1}{\PYGZsh{} retain percentage}
    \PYG{n}{retain\PYGZus{}per\PYGZus{}cumsum} \PYG{o}{=} \PYG{n}{np}\PYG{o}{.}\PYG{n}{round}\PYG{p}{(}\PYG{n}{np}\PYG{o}{.}\PYG{n}{cumsum}\PYG{p}{(}\PYG{n}{retain\PYGZus{}per}\PYG{p}{)}\PYG{p}{,}\PYG{l+m+mi}{3}\PYG{p}{)} \PYG{c+c1}{\PYGZsh{} get the cummulative sum of the reatined}
    \PYG{n}{passing\PYGZus{}per} \PYG{o}{=} \PYG{n}{np}\PYG{o}{.}\PYG{n}{round}\PYG{p}{(}\PYG{l+m+mi}{100} \PYG{o}{\PYGZhy{}} \PYG{n}{retain\PYGZus{}per\PYGZus{}cumsum}\PYG{p}{,} \PYG{l+m+mi}{3}\PYG{p}{)} \PYG{c+c1}{\PYGZsh{} substract 100\PYGZhy{}cummsum to get passing \PYGZpc{}}
    \PYG{n}{data} \PYG{o}{=} \PYG{p}{\PYGZob{}}\PYG{l+s+s2}{\PYGZdq{}}\PYG{l+s+s2}{mesh diameter [mm]}\PYG{l+s+s2}{\PYGZdq{}}\PYG{p}{:} \PYG{n}{dia}\PYG{p}{,} \PYG{l+s+s2}{\PYGZdq{}}\PYG{l+s+s2}{residue in the sieve [g]}\PYG{l+s+s2}{\PYGZdq{}}\PYG{p}{:} \PYG{n}{mass}\PYG{p}{,} \PYG{l+s+s2}{\PYGZdq{}}\PYG{l+s+s2}{Σtotal}\PYG{l+s+s2}{\PYGZdq{}}\PYG{p}{:} \PYG{n}{retain\PYGZus{}per}\PYG{p}{,} \PYG{l+s+s2}{\PYGZdq{}}\PYG{l+s+s2}{Σ/Σtotal}\PYG{l+s+s2}{\PYGZdq{}}\PYG{p}{:} \PYG{n}{passing\PYGZus{}per} \PYG{p}{\PYGZcb{}}

    \PYG{n}{df1}\PYG{o}{=} \PYG{n}{pd}\PYG{o}{.}\PYG{n}{DataFrame}\PYG{p}{(}\PYG{n}{data}\PYG{p}{)}
    \PYG{n}{df1} \PYG{o}{=} \PYG{n}{df1}\PYG{o}{.}\PYG{n}{set\PYGZus{}index}\PYG{p}{(}\PYG{l+s+s2}{\PYGZdq{}}\PYG{l+s+s2}{mesh diameter [mm]}\PYG{l+s+s2}{\PYGZdq{}}\PYG{p}{)}
    \PYG{n+nb}{print}\PYG{p}{(}\PYG{n}{df1}\PYG{p}{)}

    \PYG{n}{plt}\PYG{o}{.}\PYG{n}{rcParams}\PYG{p}{[}\PYG{l+s+s1}{\PYGZsq{}}\PYG{l+s+s1}{axes.linewidth}\PYG{l+s+s1}{\PYGZsq{}}\PYG{p}{]}\PYG{o}{=}\PYG{l+m+mi}{2}
    \PYG{c+c1}{\PYGZsh{}plt.rcParams[\PYGZdq{}axes.edgecolor\PYGZdq{}]=\PYGZsq{}white\PYGZsq{}}
    \PYG{n}{plt}\PYG{o}{.}\PYG{n}{rcParams}\PYG{p}{[}\PYG{l+s+s1}{\PYGZsq{}}\PYG{l+s+s1}{grid.linestyle}\PYG{l+s+s1}{\PYGZsq{}}\PYG{p}{]}\PYG{o}{=}\PYG{l+s+s1}{\PYGZsq{}}\PYG{l+s+s1}{\PYGZhy{}\PYGZhy{}}\PYG{l+s+s1}{\PYGZsq{}}
    \PYG{n}{plt}\PYG{o}{.}\PYG{n}{rcParams}\PYG{p}{[}\PYG{l+s+s1}{\PYGZsq{}}\PYG{l+s+s1}{grid.linewidth}\PYG{l+s+s1}{\PYGZsq{}}\PYG{p}{]}\PYG{o}{=}\PYG{l+m+mi}{1}
    \PYG{n}{x} \PYG{o}{=} \PYG{n}{np}\PYG{o}{.}\PYG{n}{append}\PYG{p}{(}\PYG{p}{[}\PYG{l+m+mi}{20}\PYG{p}{]}\PYG{p}{,}\PYG{n}{dia}\PYG{p}{)} \PYG{c+c1}{\PYGZsh{} adding data to extend over 6 mm dia}
    \PYG{n}{y} \PYG{o}{=} \PYG{n}{np}\PYG{o}{.}\PYG{n}{append}\PYG{p}{(}\PYG{p}{[}\PYG{l+m+mi}{100}\PYG{p}{]}\PYG{p}{,}\PYG{n}{passing\PYGZus{}per}\PYG{p}{)} \PYG{c+c1}{\PYGZsh{} adding 100\PYGZpc{} to plot}
    
    
    \PYG{n}{y}\PYG{o}{.}\PYG{n}{sort}\PYG{p}{(}\PYG{p}{)}
    \PYG{n}{x}\PYG{o}{.}\PYG{n}{sort}\PYG{p}{(}\PYG{p}{)}
    \PYG{n}{interp\PYGZus{}func} \PYG{o}{=} \PYG{n}{CubicSpline}\PYG{p}{(}\PYG{n}{y}\PYG{p}{,}\PYG{n}{x}\PYG{p}{)}
    
    \PYG{n}{Dd60} \PYG{o}{=} \PYG{n}{interp\PYGZus{}func}\PYG{p}{(}\PYG{l+m+mi}{60}\PYG{p}{)}
    \PYG{n}{Dd10}\PYG{o}{=}\PYG{n}{interp\PYGZus{}func}\PYG{p}{(}\PYG{l+m+mi}{10}\PYG{p}{)}
    \PYG{n}{Ddx}\PYG{o}{=}\PYG{n}{interp\PYGZus{}func}\PYG{p}{(}\PYG{n}{perdat}\PYG{p}{)}

    
    \PYG{n+nb}{print}\PYG{p}{(}\PYG{l+s+s1}{\PYGZsq{}}\PYG{l+s+se}{\PYGZbs{}n}\PYG{l+s+s1}{\PYGZsq{}}\PYG{p}{,}\PYG{l+s+s1}{\PYGZsq{}}\PYG{l+s+se}{\PYGZbs{}n}\PYG{l+s+s1}{\PYGZsq{}}\PYG{p}{)}
    \PYG{n+nb}{print}\PYG{p}{(}\PYG{l+s+s2}{\PYGZdq{}}\PYG{l+s+s2}{d60 =}\PYG{l+s+s2}{\PYGZdq{}}\PYG{p}{,}\PYG{l+s+s2}{\PYGZdq{}}\PYG{l+s+si}{\PYGZpc{}.2f}\PYG{l+s+s2}{\PYGZdq{}} \PYG{o}{\PYGZpc{}} \PYG{n}{Dd60}\PYG{p}{)}
    \PYG{n+nb}{print}\PYG{p}{(}\PYG{l+s+s2}{\PYGZdq{}}\PYG{l+s+s2}{d10 =}\PYG{l+s+s2}{\PYGZdq{}}\PYG{p}{,}\PYG{l+s+s2}{\PYGZdq{}}\PYG{l+s+si}{\PYGZpc{}.2f}\PYG{l+s+s2}{\PYGZdq{}} \PYG{o}{\PYGZpc{}} \PYG{n}{Dd10}\PYG{p}{)}
    \PYG{n+nb}{print} \PYG{p}{(}\PYG{l+s+s1}{\PYGZsq{}}\PYG{l+s+s1}{d}\PYG{l+s+si}{\PYGZpc{}d}\PYG{l+s+s1}{ = }\PYG{l+s+si}{\PYGZpc{}.2f}\PYG{l+s+s1}{\PYGZsq{}} \PYG{o}{\PYGZpc{}} \PYG{p}{(}\PYG{n}{perdat}\PYG{p}{,} \PYG{n}{Ddx}\PYG{p}{)}\PYG{p}{)}
    

    
    
    \PYG{n}{fig}\PYG{p}{,} \PYG{n}{ax} \PYG{o}{=} \PYG{n}{plt}\PYG{o}{.}\PYG{n}{subplots}\PYG{p}{(}\PYG{n}{figsize}\PYG{o}{=}\PYG{p}{(}\PYG{l+m+mi}{15}\PYG{p}{,}\PYG{l+m+mi}{10}\PYG{p}{)}\PYG{p}{)}
    \PYG{n}{fig}\PYG{o}{.}\PYG{n}{canvas}\PYG{o}{.}\PYG{n}{header\PYGZus{}visible} \PYG{o}{=} \PYG{k+kc}{False}
    \PYG{n}{plt}\PYG{o}{.}\PYG{n}{semilogx}\PYG{p}{(}\PYG{n}{x}\PYG{p}{,} \PYG{n}{y}\PYG{p}{,} \PYG{l+s+s1}{\PYGZsq{}}\PYG{l+s+s1}{x\PYGZhy{}}\PYG{l+s+s1}{\PYGZsq{}}\PYG{p}{,} \PYG{n}{color}\PYG{o}{=}\PYG{l+s+s1}{\PYGZsq{}}\PYG{l+s+s1}{red}\PYG{l+s+s1}{\PYGZsq{}}\PYG{p}{)}  
    \PYG{n}{tics}\PYG{o}{=}\PYG{n}{x}\PYG{o}{.}\PYG{n}{tolist}\PYG{p}{(}\PYG{p}{)}

    \PYG{n}{ax}\PYG{o}{.}\PYG{n}{grid}\PYG{p}{(}\PYG{n}{which}\PYG{o}{=}\PYG{l+s+s1}{\PYGZsq{}}\PYG{l+s+s1}{major}\PYG{l+s+s1}{\PYGZsq{}}\PYG{p}{,} \PYG{n}{color}\PYG{o}{=}\PYG{l+s+s1}{\PYGZsq{}}\PYG{l+s+s1}{k}\PYG{l+s+s1}{\PYGZsq{}}\PYG{p}{,} \PYG{n}{alpha}\PYG{o}{=}\PYG{l+m+mf}{0.7}\PYG{p}{)} 
    \PYG{n}{ax}\PYG{o}{.}\PYG{n}{grid}\PYG{p}{(}\PYG{n}{which}\PYG{o}{=}\PYG{l+s+s1}{\PYGZsq{}}\PYG{l+s+s1}{minor}\PYG{l+s+s1}{\PYGZsq{}}\PYG{p}{,} \PYG{n}{color}\PYG{o}{=}\PYG{l+s+s1}{\PYGZsq{}}\PYG{l+s+s1}{k}\PYG{l+s+s1}{\PYGZsq{}}\PYG{p}{,} \PYG{n}{alpha}\PYG{o}{=}\PYG{l+m+mf}{0.3}\PYG{p}{)}
    \PYG{n}{ax}\PYG{o}{.}\PYG{n}{set\PYGZus{}xticks}\PYG{p}{(}\PYG{n}{x}\PYG{p}{)}\PYG{p}{;}  
    \PYG{n}{ax}\PYG{o}{.}\PYG{n}{set\PYGZus{}yticks}\PYG{p}{(}\PYG{n}{np}\PYG{o}{.}\PYG{n}{arange}\PYG{p}{(}\PYG{l+m+mi}{0}\PYG{p}{,}\PYG{l+m+mi}{110}\PYG{p}{,}\PYG{l+m+mi}{10}\PYG{p}{)}\PYG{p}{)}\PYG{p}{;}
    \PYG{n}{plt}\PYG{o}{.}\PYG{n}{title}\PYG{p}{(}\PYG{l+s+s1}{\PYGZsq{}}\PYG{l+s+s1}{grain size distribution}\PYG{l+s+s1}{\PYGZsq{}}\PYG{p}{)}\PYG{p}{;}
    \PYG{n}{plt}\PYG{o}{.}\PYG{n}{xlabel}\PYG{p}{(}\PYG{l+s+s1}{\PYGZsq{}}\PYG{l+s+s1}{grain size d [mm]}\PYG{l+s+s1}{\PYGZsq{}}\PYG{p}{)}\PYG{p}{;}
    \PYG{n}{plt}\PYG{o}{.}\PYG{n}{ylabel}\PYG{p}{(}\PYG{l+s+s1}{\PYGZsq{}}\PYG{l+s+s1}{grain fraction \PYGZlt{} d ins }\PYG{l+s+si}{\PYGZpc{} o}\PYG{l+s+s1}{f total mass}\PYG{l+s+s1}{\PYGZsq{}}\PYG{p}{)}\PYG{p}{;}
    \PYG{c+c1}{\PYGZsh{}x2=[0,0]}
   \PYG{c+c1}{\PYGZsh{} y2=[60,Dd60]}
    \PYG{c+c1}{\PYGZsh{}print(x,y)}
    \PYG{n}{plt}\PYG{o}{.}\PYG{n}{plot}\PYG{p}{(}\PYG{p}{[}\PYG{l+m+mi}{0}\PYG{p}{,}\PYG{n}{Dd60}\PYG{p}{,}\PYG{n}{Dd60}\PYG{p}{,}\PYG{n}{Dd60}\PYG{p}{]}\PYG{p}{,}\PYG{p}{[}\PYG{l+m+mi}{60}\PYG{p}{,}\PYG{l+m+mi}{60}\PYG{p}{,}\PYG{l+m+mi}{0}\PYG{p}{,}\PYG{l+m+mi}{60}\PYG{p}{]}\PYG{p}{,}\PYG{n}{ls}\PYG{o}{=}\PYG{l+s+s1}{\PYGZsq{}}\PYG{l+s+s1}{\PYGZhy{}}\PYG{l+s+s1}{\PYGZsq{}}\PYG{p}{,}\PYG{n}{color}\PYG{o}{=}\PYG{l+s+s1}{\PYGZsq{}}\PYG{l+s+s1}{g}\PYG{l+s+s1}{\PYGZsq{}}\PYG{p}{)}\PYG{p}{;}
    \PYG{n}{plt}\PYG{o}{.}\PYG{n}{plot}\PYG{p}{(}\PYG{p}{[}\PYG{l+m+mi}{0}\PYG{p}{,}\PYG{n}{Dd10}\PYG{p}{,}\PYG{n}{Dd10}\PYG{p}{,}\PYG{n}{Dd10}\PYG{p}{]}\PYG{p}{,}\PYG{p}{[}\PYG{l+m+mi}{10}\PYG{p}{,}\PYG{l+m+mi}{10}\PYG{p}{,}\PYG{l+m+mi}{0}\PYG{p}{,}\PYG{l+m+mi}{10}\PYG{p}{]}\PYG{p}{,}\PYG{n}{ls}\PYG{o}{=}\PYG{l+s+s1}{\PYGZsq{}}\PYG{l+s+s1}{\PYGZhy{}}\PYG{l+s+s1}{\PYGZsq{}}\PYG{p}{,}\PYG{n}{color}\PYG{o}{=}\PYG{l+s+s1}{\PYGZsq{}}\PYG{l+s+s1}{r}\PYG{l+s+s1}{\PYGZsq{}}\PYG{p}{)}\PYG{p}{;}
    \PYG{n}{plt}\PYG{o}{.}\PYG{n}{plot}\PYG{p}{(}\PYG{p}{[}\PYG{l+m+mi}{0}\PYG{p}{,}\PYG{n}{Ddx}\PYG{p}{,}\PYG{n}{Ddx}\PYG{p}{,}\PYG{n}{Ddx}\PYG{p}{]}\PYG{p}{,}\PYG{p}{[}\PYG{n}{perdat}\PYG{p}{,}\PYG{n}{perdat}\PYG{p}{,}\PYG{l+m+mi}{0}\PYG{p}{,}\PYG{n}{perdat}\PYG{p}{]}\PYG{p}{,}\PYG{n}{ls}\PYG{o}{=}\PYG{l+s+s1}{\PYGZsq{}}\PYG{l+s+s1}{\PYGZhy{}}\PYG{l+s+s1}{\PYGZsq{}}\PYG{p}{,}\PYG{n}{color}\PYG{o}{=}\PYG{l+s+s1}{\PYGZsq{}}\PYG{l+s+s1}{\PYGZsh{}8A2BE2}\PYG{l+s+s1}{\PYGZsq{}}\PYG{p}{)}
    \PYG{n}{ax}\PYG{o}{.}\PYG{n}{set\PYGZus{}xlim}\PYG{p}{(}\PYG{l+m+mi}{0}\PYG{p}{,} \PYG{l+m+mi}{30}\PYG{p}{)}
    \PYG{k+kn}{from} \PYG{n+nn}{matplotlib}\PYG{n+nn}{.}\PYG{n+nn}{ticker} \PYG{k+kn}{import} \PYG{n}{StrMethodFormatter}
    \PYG{n}{ax}\PYG{o}{.}\PYG{n}{xaxis}\PYG{o}{.}\PYG{n}{set\PYGZus{}major\PYGZus{}formatter}\PYG{p}{(}\PYG{n}{StrMethodFormatter}\PYG{p}{(}\PYG{l+s+s1}{\PYGZsq{}}\PYG{l+s+si}{\PYGZob{}x:0.2f\PYGZcb{}}\PYG{l+s+s1}{\PYGZsq{}}\PYG{p}{)}\PYG{p}{)}
    
    \PYG{k}{return}\PYG{p}{(}\PYG{n}{Dd60}\PYG{p}{,}\PYG{n}{Dd10}\PYG{p}{)}
    
\PYG{n}{style} \PYG{o}{=} \PYG{p}{\PYGZob{}}\PYG{l+s+s1}{\PYGZsq{}}\PYG{l+s+s1}{description\PYGZus{}width}\PYG{l+s+s1}{\PYGZsq{}}\PYG{p}{:} \PYG{l+s+s1}{\PYGZsq{}}\PYG{l+s+s1}{200px}\PYG{l+s+s1}{\PYGZsq{}}\PYG{p}{\PYGZcb{}}    

\PYG{n}{Inter}\PYG{o}{=}\PYG{n}{widgets}\PYG{o}{.}\PYG{n}{interact\PYGZus{}manual}\PYG{p}{(}\PYG{n}{SA}\PYG{p}{,} 
                       \PYG{n}{mu}\PYG{o}{=} \PYG{n}{widgets}\PYG{o}{.}\PYG{n}{FloatText}\PYG{p}{(}\PYG{n}{description}\PYG{o}{=}\PYG{l+s+s2}{\PYGZdq{}}\PYG{l+s+s2}{6 mm}\PYG{l+s+s2}{\PYGZdq{}}\PYG{p}{,} \PYG{n}{style}\PYG{o}{=}\PYG{n}{style}\PYG{p}{)}\PYG{p}{,}
                       \PYG{n}{m1}\PYG{o}{=} \PYG{n}{widgets}\PYG{o}{.}\PYG{n}{FloatText}\PYG{p}{(}\PYG{n}{description}\PYG{o}{=}\PYG{l+s+s2}{\PYGZdq{}}\PYG{l+s+s2}{2 mm}\PYG{l+s+s2}{\PYGZdq{}}\PYG{p}{,}\PYG{n}{style}\PYG{o}{=}\PYG{n}{style}\PYG{p}{)}\PYG{p}{,}
                       \PYG{n}{m2}\PYG{o}{=} \PYG{n}{widgets}\PYG{o}{.}\PYG{n}{FloatText}\PYG{p}{(}\PYG{n}{description}\PYG{o}{=}\PYG{l+s+s2}{\PYGZdq{}}\PYG{l+s+s2}{0.6 mm}\PYG{l+s+s2}{\PYGZdq{}}\PYG{p}{,} \PYG{n}{style}\PYG{o}{=}\PYG{n}{style}\PYG{p}{)}\PYG{p}{,}
                       \PYG{n}{m3}\PYG{o}{=} \PYG{n}{widgets}\PYG{o}{.}\PYG{n}{FloatText}\PYG{p}{(}\PYG{n}{description}\PYG{o}{=}\PYG{l+s+s2}{\PYGZdq{}}\PYG{l+s+s2}{0.2 mm}\PYG{l+s+s2}{\PYGZdq{}}\PYG{p}{,} \PYG{n}{style}\PYG{o}{=}\PYG{n}{style}\PYG{p}{)}\PYG{p}{,}
                       \PYG{n}{m4}\PYG{o}{=} \PYG{n}{widgets}\PYG{o}{.}\PYG{n}{FloatText}\PYG{p}{(}\PYG{n}{description}\PYG{o}{=}\PYG{l+s+s2}{\PYGZdq{}}\PYG{l+s+s2}{0.06 mm}\PYG{l+s+s2}{\PYGZdq{}}\PYG{p}{,} \PYG{n}{style}\PYG{o}{=}\PYG{n}{style}\PYG{p}{)}\PYG{p}{,}
                       \PYG{n}{ml}\PYG{o}{=} \PYG{n}{widgets}\PYG{o}{.}\PYG{n}{FloatText}\PYG{p}{(}\PYG{n}{description}\PYG{o}{=}\PYG{l+s+s2}{\PYGZdq{}}\PYG{l+s+s2}{0.01 mm}\PYG{l+s+s2}{\PYGZdq{}}\PYG{p}{,} \PYG{n}{style}\PYG{o}{=}\PYG{n}{style}\PYG{p}{)}\PYG{p}{,}
                       \PYG{n}{perdat}\PYG{o}{=} \PYG{n}{widgets}\PYG{o}{.}\PYG{n}{FloatText}\PYG{p}{(}\PYG{n}{description}\PYG{o}{=}\PYG{l+s+s2}{\PYGZdq{}}\PYG{l+s+s2}{Enter Fineness }\PYG{l+s+s2}{\PYGZpc{}}\PYG{l+s+s2}{ }\PYG{l+s+s2}{\PYGZdq{}}\PYG{p}{,} \PYG{n}{style}\PYG{o}{=}\PYG{n}{style}\PYG{p}{)}     \PYG{p}{)}
\end{sphinxVerbatim}

\end{sphinxuseclass}\end{sphinxVerbatimInput}
\begin{sphinxVerbatimOutput}

\begin{sphinxuseclass}{cell_output}
\begin{sphinxVerbatim}[commandchars=\\\{\}]
Please provide the seive data in the boxes:  
\end{sphinxVerbatim}

\begin{sphinxVerbatim}[commandchars=\\\{\}]
interactive(children=(FloatText(value=0.0, description=\PYGZsq{}6 mm\PYGZsq{}, style=DescriptionStyle(description\PYGZus{}width=\PYGZsq{}200px…
\end{sphinxVerbatim}

\end{sphinxuseclass}\end{sphinxVerbatimOutput}

\end{sphinxuseclass}
\begin{sphinxuseclass}{cell}\begin{sphinxVerbatimInput}

\begin{sphinxuseclass}{cell_input}
\begin{sphinxVerbatim}[commandchars=\\\{\}]
\PYG{n+nb}{print}\PYG{p}{(}\PYG{n}{Inter}\PYG{p}{)}
\end{sphinxVerbatim}

\end{sphinxuseclass}\end{sphinxVerbatimInput}
\begin{sphinxVerbatimOutput}

\begin{sphinxuseclass}{cell_output}
\begin{sphinxVerbatim}[commandchars=\\\{\}]
\PYGZlt{}function SA at 0x00000274FF7A9E10\PYGZgt{}
\end{sphinxVerbatim}

\end{sphinxuseclass}\end{sphinxVerbatimOutput}

\end{sphinxuseclass}
\sphinxAtStartPar
\sphinxstylestrong{The plot shown is interactive use the pointer and others tools in the graph to obtain d10 and d60 for the next step}

\begin{sphinxuseclass}{cell}\begin{sphinxVerbatimInput}

\begin{sphinxuseclass}{cell_input}
\begin{sphinxVerbatim}[commandchars=\\\{\}]
\PYG{k}{def} \PYG{n+nf}{SA2}\PYG{p}{(}\PYG{n}{d10}\PYG{p}{,} \PYG{n}{d60}\PYG{p}{,} \PYG{n}{t}\PYG{p}{)}\PYG{p}{:}
    \PYG{n}{U} \PYG{o}{=} \PYG{n}{d60}\PYG{o}{/}\PYG{n}{d10}
    \PYG{n}{K\PYGZus{}h} \PYG{o}{=}  \PYG{l+m+mf}{0.0116}\PYG{o}{*}\PYG{p}{(}\PYG{l+m+mf}{0.7}\PYG{o}{+}\PYG{l+m+mf}{0.03}\PYG{o}{*}\PYG{n}{t}\PYG{p}{)}\PYG{o}{*}\PYG{n}{d10}\PYG{o}{*}\PYG{o}{*}\PYG{l+m+mi}{2}
    \PYG{n+nb}{print}\PYG{p}{(}\PYG{l+s+s2}{\PYGZdq{}}\PYG{l+s+se}{\PYGZbs{}n}\PYG{l+s+s2}{ The coefficient of non\PYGZhy{}uniformity: }\PYG{l+s+si}{\PYGZob{}0:0.2f\PYGZcb{}}\PYG{l+s+s2}{\PYGZdq{}}\PYG{o}{.}\PYG{n}{format}\PYG{p}{(}\PYG{n}{U}\PYG{p}{)}\PYG{p}{,} \PYG{l+s+s2}{\PYGZdq{}}\PYG{l+s+se}{\PYGZbs{}n}\PYG{l+s+s2}{\PYGZdq{}}\PYG{p}{)}
    \PYG{n+nb}{print}\PYG{p}{(}\PYG{l+s+s2}{\PYGZdq{}}\PYG{l+s+s2}{The Hydraulic Conductivity based on Hazen Formula: }\PYG{l+s+si}{\PYGZob{}0:0.2e\PYGZcb{}}\PYG{l+s+s2}{ m/s}\PYG{l+s+s2}{\PYGZdq{}}\PYG{o}{.}\PYG{n}{format}\PYG{p}{(}\PYG{n}{K\PYGZus{}h}\PYG{p}{)}\PYG{p}{)}

\PYG{n}{style} \PYG{o}{=} \PYG{p}{\PYGZob{}}\PYG{l+s+s1}{\PYGZsq{}}\PYG{l+s+s1}{description\PYGZus{}width}\PYG{l+s+s1}{\PYGZsq{}}\PYG{p}{:} \PYG{l+s+s1}{\PYGZsq{}}\PYG{l+s+s1}{200px}\PYG{l+s+s1}{\PYGZsq{}}\PYG{p}{\PYGZcb{}}    

\PYG{n}{Inter}\PYG{o}{=}\PYG{n}{widgets}\PYG{o}{.}\PYG{n}{interact\PYGZus{}manual}\PYG{p}{(}\PYG{n}{SA2}\PYG{p}{,} 
                       \PYG{n}{d10}\PYG{o}{=} \PYG{n}{widgets}\PYG{o}{.}\PYG{n}{FloatText}\PYG{p}{(}\PYG{n}{description}\PYG{o}{=}\PYG{l+s+s2}{\PYGZdq{}}\PYG{l+s+s2}{d10 (mm)}\PYG{l+s+s2}{\PYGZdq{}}\PYG{p}{,} \PYG{n}{style}\PYG{o}{=}\PYG{n}{style}\PYG{p}{)}\PYG{p}{,}
                       \PYG{n}{d60}\PYG{o}{=} \PYG{n}{widgets}\PYG{o}{.}\PYG{n}{FloatText}\PYG{p}{(}\PYG{n}{description}\PYG{o}{=}\PYG{l+s+s2}{\PYGZdq{}}\PYG{l+s+s2}{d60 (mm)}\PYG{l+s+s2}{\PYGZdq{}}\PYG{p}{,}\PYG{n}{style}\PYG{o}{=}\PYG{n}{style}\PYG{p}{)}\PYG{p}{,}
                       \PYG{n}{t}\PYG{o}{=} \PYG{n}{widgets}\PYG{o}{.}\PYG{n}{FloatText}\PYG{p}{(}\PYG{n}{description}\PYG{o}{=}\PYG{l+s+s2}{\PYGZdq{}}\PYG{l+s+s2}{Temperature (°C)}\PYG{l+s+s2}{\PYGZdq{}}\PYG{p}{,} \PYG{n}{style}\PYG{o}{=}\PYG{n}{style}\PYG{p}{)}\PYG{p}{)}
\end{sphinxVerbatim}

\end{sphinxuseclass}\end{sphinxVerbatimInput}
\begin{sphinxVerbatimOutput}

\begin{sphinxuseclass}{cell_output}
\begin{sphinxVerbatim}[commandchars=\\\{\}]
interactive(children=(FloatText(value=0.0, description=\PYGZsq{}d10 (mm)\PYGZsq{}, style=DescriptionStyle(description\PYGZus{}width=\PYGZsq{}2…
\end{sphinxVerbatim}

\end{sphinxuseclass}\end{sphinxVerbatimOutput}

\end{sphinxuseclass}
\sphinxstepscope


\chapter{Simulating Effective hydraulic conductivity}
\label{\detokenize{content/tools/effective_K:simulating-effective-hydraulic-conductivity}}\label{\detokenize{content/tools/effective_K::doc}}

\section{How to use the tool?}
\label{\detokenize{content/tools/effective_K:how-to-use-the-tool}}\begin{enumerate}
\sphinxsetlistlabels{\arabic}{enumi}{enumii}{}{.}%
\item {} 
\sphinxAtStartPar
Go to the Binder by clicking the rocket button (top\sphinxhyphen{}right of the page)

\item {} 
\sphinxAtStartPar
Execute the code cell

\item {} 
\sphinxAtStartPar
Change the values of different quantities (layer thickness and corresponding conductivity) in the box and click the \sphinxstylestrong{run interact}.

\item {} 
\sphinxAtStartPar
For re\sphinxhyphen{}simulations \sphinxhyphen{} changes the input values in the boxes and click the “\sphinxstylestrong{run interact}” button.

\end{enumerate}

\sphinxAtStartPar
This tool can also be downloaded and run locally. For that download the \sphinxstylestrong{\sphinxstyleemphasis{effective\_K.ipynb}} file from the book GitHub site, and execute the process in any editor (e.g., JUPYTER notebook, JUPYTER lab) that is able to read and execute this file\sphinxhyphen{}type.

\sphinxAtStartPar
The code may also be executed in the book page. You may also run this tool (smartphone optimized) from: \sphinxurl{https://keff-app.herokuapp.com/}

\sphinxAtStartPar
The codes are licensed under CC by 4.0 \sphinxhref{https://creativecommons.org/licenses/by/4.0/deed.en}{(use anyways, but acknowledge the original work)}

\begin{sphinxuseclass}{cell}\begin{sphinxVerbatimInput}

\begin{sphinxuseclass}{cell_input}
\begin{sphinxVerbatim}[commandchars=\\\{\}]
\PYG{c+c1}{\PYGZsh{}}
\PYG{k+kn}{import} \PYG{n+nn}{matplotlib}\PYG{n+nn}{.}\PYG{n+nn}{pyplot} \PYG{k}{as} \PYG{n+nn}{plt} 
\PYG{k+kn}{import} \PYG{n+nn}{numpy} \PYG{k}{as} \PYG{n+nn}{np} 
\PYG{k+kn}{import} \PYG{n+nn}{pandas} \PYG{k}{as} \PYG{n+nn}{pd}
\PYG{k+kn}{import} \PYG{n+nn}{ipywidgets} \PYG{k}{as} \PYG{n+nn}{widgets}
\PYG{k+kn}{from} \PYG{n+nn}{ipywidgets} \PYG{k+kn}{import} \PYG{n}{interact}\PYG{p}{,} \PYG{n}{interactive}\PYG{p}{,} \PYG{n}{fixed}\PYG{p}{,} \PYG{n}{interact\PYGZus{}manual}
\PYG{n}{plt}\PYG{o}{.}\PYG{n}{rcParams}\PYG{p}{[}\PYG{l+s+s2}{\PYGZdq{}}\PYG{l+s+s2}{font.weight}\PYG{l+s+s2}{\PYGZdq{}}\PYG{p}{]} \PYG{o}{=} \PYG{l+s+s2}{\PYGZdq{}}\PYG{l+s+s2}{bold}\PYG{l+s+s2}{\PYGZdq{}}
\PYG{n}{plt}\PYG{o}{.}\PYG{n}{rcParams}\PYG{p}{[}\PYG{l+s+s2}{\PYGZdq{}}\PYG{l+s+s2}{font.size}\PYG{l+s+s2}{\PYGZdq{}}\PYG{p}{]} \PYG{o}{=} \PYG{l+m+mi}{8}
\PYG{k+kn}{import} \PYG{n+nn}{warnings}
\PYG{n}{warnings}\PYG{o}{.}\PYG{n}{filterwarnings}\PYG{p}{(}\PYG{l+s+s1}{\PYGZsq{}}\PYG{l+s+s1}{ignore}\PYG{l+s+s1}{\PYGZsq{}}\PYG{p}{)}


\PYG{k}{def} \PYG{n+nf}{eff\PYGZus{}K}\PYG{p}{(}\PYG{n}{M1}\PYG{p}{,} \PYG{n}{M2}\PYG{p}{,} \PYG{n}{M3}\PYG{p}{,} \PYG{n}{K1}\PYG{p}{,} \PYG{n}{K2}\PYG{p}{,} \PYG{n}{K3}\PYG{p}{)}\PYG{p}{:}
    
    \PYG{n}{K} \PYG{o}{=} \PYG{p}{[}\PYG{n}{K1}\PYG{p}{,} \PYG{n}{K2}\PYG{p}{,} \PYG{n}{K3}\PYG{p}{]}
    \PYG{n}{K\PYGZus{}f} \PYG{o}{=} \PYG{p}{[}\PYG{l+s+s2}{\PYGZdq{}}\PYG{l+s+si}{\PYGZpc{}0.2e}\PYG{l+s+s2}{\PYGZdq{}} \PYG{o}{\PYGZpc{}}\PYG{k}{elem} for elem in K]
    \PYG{n}{INPUT} \PYG{o}{=} \PYG{p}{\PYGZob{}}\PYG{l+s+s2}{\PYGZdq{}}\PYG{l+s+s2}{Thickness [L]}\PYG{l+s+s2}{\PYGZdq{}}\PYG{p}{:} \PYG{p}{[}\PYG{n}{M1}\PYG{p}{,} \PYG{n}{M2}\PYG{p}{,} \PYG{n}{M3}\PYG{p}{]}\PYG{p}{,} \PYG{l+s+s2}{\PYGZdq{}}\PYG{l+s+s2}{Hydraulic Conductivity [L/T]}\PYG{l+s+s2}{\PYGZdq{}}\PYG{p}{:} \PYG{n}{K\PYGZus{}f}\PYG{p}{\PYGZcb{}}
    \PYG{n}{index} \PYG{o}{=} \PYG{p}{[}\PYG{l+s+s2}{\PYGZdq{}}\PYG{l+s+s2}{Layer 1}\PYG{l+s+s2}{\PYGZdq{}}\PYG{p}{,} \PYG{l+s+s2}{\PYGZdq{}}\PYG{l+s+s2}{Layer 2}\PYG{l+s+s2}{\PYGZdq{}}\PYG{p}{,} \PYG{l+s+s2}{\PYGZdq{}}\PYG{l+s+s2}{Layer 3}\PYG{l+s+s2}{\PYGZdq{}}\PYG{p}{]}
    \PYG{n}{df} \PYG{o}{=} \PYG{n}{pd}\PYG{o}{.}\PYG{n}{DataFrame}\PYG{p}{(}\PYG{n}{INPUT}\PYG{p}{,} \PYG{n}{index}\PYG{o}{=}\PYG{n}{index}\PYG{p}{)}
    \PYG{n}{tt} \PYG{o}{=} \PYG{n}{M1}\PYG{o}{+}\PYG{n}{M2} \PYG{o}{+} \PYG{n}{M3}  \PYG{c+c1}{\PYGZsh{} m, totial thickness}
    
    \PYG{c+c1}{\PYGZsh{} finding relative thickness, }
    \PYG{n}{RL1}\PYG{p}{,} \PYG{n}{RL2}\PYG{p}{,} \PYG{n}{RL3} \PYG{o}{=} \PYG{n}{M1}\PYG{o}{/}\PYG{n}{tt}\PYG{p}{,} \PYG{n}{M2}\PYG{o}{/}\PYG{n}{tt}\PYG{p}{,} \PYG{n}{M3}\PYG{o}{/}\PYG{n}{tt} 
    \PYG{n}{HRL1}\PYG{p}{,} \PYG{n}{HRL2}\PYG{p}{,} \PYG{n}{HRL3} \PYG{o}{=} \PYG{l+m+mi}{1}\PYG{o}{/}\PYG{n}{K1}\PYG{p}{,} \PYG{l+m+mi}{1}\PYG{o}{/}\PYG{n}{K2}\PYG{p}{,} \PYG{l+m+mi}{1}\PYG{o}{/}\PYG{n}{K3} 
    \PYG{n}{WHK1}\PYG{p}{,} \PYG{n}{WHK2}\PYG{p}{,} \PYG{n}{WHK3} \PYG{o}{=} \PYG{n}{RL1}\PYG{o}{*}\PYG{n}{K1}\PYG{p}{,} \PYG{n}{RL2}\PYG{o}{*}\PYG{n}{K2}\PYG{p}{,}\PYG{n}{RL3}\PYG{o}{*}\PYG{n}{K3}
    \PYG{n}{WHR1}\PYG{p}{,}\PYG{n}{WHR2}\PYG{p}{,} \PYG{n}{WHR3} \PYG{o}{=} \PYG{n}{RL1}\PYG{o}{/}\PYG{n}{K1}\PYG{p}{,} \PYG{n}{RL2}\PYG{o}{/}\PYG{n}{K2}\PYG{p}{,} \PYG{n}{RL3}\PYG{o}{/}\PYG{n}{K3} 
    
    \PYG{c+c1}{\PYGZsh{} creating intermediate table}
    \PYG{n}{RL} \PYG{o}{=}  \PYG{p}{[}\PYG{n}{RL1}\PYG{p}{,} \PYG{n}{RL2}\PYG{p}{,} \PYG{n}{RL3}\PYG{p}{]}
    \PYG{n}{HRL} \PYG{o}{=} \PYG{p}{[}\PYG{n}{HRL1}\PYG{p}{,} \PYG{n}{HRL2}\PYG{p}{,} \PYG{n}{HRL3}\PYG{p}{]}
    \PYG{n}{WHK} \PYG{o}{=} \PYG{p}{[}\PYG{n}{WHK1}\PYG{p}{,} \PYG{n}{WHK2}\PYG{p}{,} \PYG{n}{WHK3}\PYG{p}{]}
    \PYG{n}{WHR} \PYG{o}{=} \PYG{p}{[}\PYG{n}{WHR1}\PYG{p}{,}\PYG{n}{WHR2}\PYG{p}{,} \PYG{n}{WHR3}\PYG{p}{]}
    \PYG{n}{RL\PYGZus{}f} \PYG{o}{=} \PYG{p}{[} \PYG{l+s+s1}{\PYGZsq{}}\PYG{l+s+si}{\PYGZpc{}.2f}\PYG{l+s+s1}{\PYGZsq{}} \PYG{o}{\PYGZpc{}}\PYG{k}{elem} for elem in RL ]
    \PYG{n}{HRL\PYGZus{}f} \PYG{o}{=} \PYG{p}{[} \PYG{l+s+s1}{\PYGZsq{}}\PYG{l+s+si}{\PYGZpc{}.2e}\PYG{l+s+s1}{\PYGZsq{}} \PYG{o}{\PYGZpc{}}\PYG{k}{elem} for elem in HRL ]
    \PYG{n}{WHK\PYGZus{}f} \PYG{o}{=} \PYG{p}{[} \PYG{l+s+s1}{\PYGZsq{}}\PYG{l+s+si}{\PYGZpc{}.2e}\PYG{l+s+s1}{\PYGZsq{}} \PYG{o}{\PYGZpc{}}\PYG{k}{elem} for elem in WHK ]
    \PYG{n}{WHR\PYGZus{}f} \PYG{o}{=} \PYG{p}{[} \PYG{l+s+s1}{\PYGZsq{}}\PYG{l+s+si}{\PYGZpc{}.2e}\PYG{l+s+s1}{\PYGZsq{}} \PYG{o}{\PYGZpc{}}\PYG{k}{elem} for elem in WHR ]
    
    \PYG{n}{index2} \PYG{o}{=} \PYG{p}{[}\PYG{l+s+s2}{\PYGZdq{}}\PYG{l+s+s2}{Layer 1}\PYG{l+s+s2}{\PYGZdq{}}\PYG{p}{,} \PYG{l+s+s2}{\PYGZdq{}}\PYG{l+s+s2}{Layer 2}\PYG{l+s+s2}{\PYGZdq{}}\PYG{p}{,} \PYG{l+s+s2}{\PYGZdq{}}\PYG{l+s+s2}{Layer 3}\PYG{l+s+s2}{\PYGZdq{}}\PYG{p}{,} \PYG{l+s+s2}{\PYGZdq{}}\PYG{l+s+s2}{Sum}\PYG{l+s+s2}{\PYGZdq{}}\PYG{p}{]}
    \PYG{n}{CAL1} \PYG{o}{=} \PYG{p}{\PYGZob{}}\PYG{l+s+s2}{\PYGZdq{}}\PYG{l+s+s2}{Relative Thickness [\PYGZhy{}]}\PYG{l+s+s2}{\PYGZdq{}}\PYG{p}{:}\PYG{n}{RL\PYGZus{}f}\PYG{p}{,} \PYG{l+s+s2}{\PYGZdq{}}\PYG{l+s+s2}{Hydraulic Resistance [T/L]}\PYG{l+s+s2}{\PYGZdq{}}\PYG{p}{:}\PYG{n}{HRL\PYGZus{}f}\PYG{p}{,}
            \PYG{l+s+s2}{\PYGZdq{}}\PYG{l+s+s2}{Weighted Hyd. Cond. [L/T]}\PYG{l+s+s2}{\PYGZdq{}}\PYG{p}{:} \PYG{n}{WHK\PYGZus{}f}\PYG{p}{,} \PYG{l+s+s2}{\PYGZdq{}}\PYG{l+s+s2}{Weighted Hyd. Resistance [T/L]}\PYG{l+s+s2}{\PYGZdq{}}\PYG{p}{:} \PYG{n}{WHR\PYGZus{}f}\PYG{p}{\PYGZcb{}}
    \PYG{n}{df2} \PYG{o}{=} \PYG{n}{pd}\PYG{o}{.}\PYG{n}{DataFrame}\PYG{p}{(}\PYG{n}{CAL1}\PYG{p}{)}
    
    \PYG{n+nb}{print}\PYG{p}{(}\PYG{l+s+s2}{\PYGZdq{}}\PYG{l+s+se}{\PYGZbs{}n}\PYG{l+s+se}{\PYGZbs{}n}\PYG{l+s+se}{\PYGZbs{}033}\PYG{l+s+s2}{[1m Intermediate Calculations: }\PYG{l+s+se}{\PYGZbs{}033}\PYG{l+s+s2}{[0m }\PYG{l+s+se}{\PYGZbs{}n}\PYG{l+s+s2}{\PYGZdq{}}\PYG{p}{)}
    \PYG{n+nb}{print}\PYG{p}{(}\PYG{n}{df2}\PYG{p}{,} \PYG{l+s+s2}{\PYGZdq{}}\PYG{l+s+se}{\PYGZbs{}n}\PYG{l+s+s2}{\PYGZdq{}}\PYG{p}{)}
    
    \PYG{c+c1}{\PYGZsh{} calculations Parallel flow}
    \PYG{n}{HR\PYGZus{}eff} \PYG{o}{=} \PYG{n+nb}{sum}\PYG{p}{(}\PYG{n}{WHR}\PYG{p}{)}
    \PYG{n}{HR\PYGZus{}eff\PYGZus{}a} \PYG{o}{=} \PYG{n+nb}{max}\PYG{p}{(}\PYG{n}{WHR}\PYG{p}{)}

    \PYG{n}{HC\PYGZus{}eff} \PYG{o}{=} \PYG{l+m+mi}{1}\PYG{o}{/}\PYG{n}{HR\PYGZus{}eff}
    \PYG{n}{HC\PYGZus{}eff\PYGZus{}a} \PYG{o}{=} \PYG{l+m+mi}{1}\PYG{o}{/}\PYG{n}{HR\PYGZus{}eff\PYGZus{}a}
    
    \PYG{n}{RT1} \PYG{o}{=} \PYG{l+m+mi}{0} 
    \PYG{n}{RT2} \PYG{o}{=} \PYG{n}{RT1}\PYG{o}{+}\PYG{n}{RL1}
    \PYG{n}{RT3} \PYG{o}{=} \PYG{n}{RT2}\PYG{o}{+}\PYG{n}{RL2}
    \PYG{n}{RT4} \PYG{o}{=} \PYG{l+m+mi}{1}
    
    \PYG{n}{RH1} \PYG{o}{=} \PYG{l+m+mi}{1}
    \PYG{n}{RH2} \PYG{o}{=} \PYG{l+m+mi}{1}\PYG{o}{\PYGZhy{}}\PYG{n}{HC\PYGZus{}eff}\PYG{o}{*}\PYG{n}{WHR1}
    \PYG{n}{RH3} \PYG{o}{=} \PYG{n}{HC\PYGZus{}eff}\PYG{o}{*}\PYG{n}{WHR3} 
    \PYG{n}{RH4} \PYG{o}{=} \PYG{l+m+mi}{0}

      \PYG{c+c1}{\PYGZsh{} creating data table }
    \PYG{n}{RH} \PYG{o}{=} \PYG{p}{[}\PYG{n}{RH1}\PYG{p}{,} \PYG{n}{RH2}\PYG{p}{,} \PYG{n}{RH3}\PYG{p}{,} \PYG{n}{RH4}\PYG{p}{]}
    \PYG{n}{RH\PYGZus{}f} \PYG{o}{=} \PYG{p}{[}\PYG{l+s+s2}{\PYGZdq{}}\PYG{l+s+si}{\PYGZpc{}0.2f}\PYG{l+s+s2}{\PYGZdq{}} \PYG{o}{\PYGZpc{}}\PYG{k}{elem} for elem in RH]
    \PYG{n}{RT} \PYG{o}{=} \PYG{p}{[}\PYG{n}{RT1}\PYG{p}{,} \PYG{n}{RT2}\PYG{p}{,} \PYG{n}{RT3}\PYG{p}{,} \PYG{n}{RT4}\PYG{p}{]}
    \PYG{n}{RT\PYGZus{}f} \PYG{o}{=} \PYG{p}{[}\PYG{l+s+s2}{\PYGZdq{}}\PYG{l+s+si}{\PYGZpc{}0.2f}\PYG{l+s+s2}{\PYGZdq{}} \PYG{o}{\PYGZpc{}}\PYG{k}{elem} for elem in RT] \PYGZsh{} 0.2f is for number format

    \PYG{n}{df3} \PYG{o}{=} \PYG{p}{\PYGZob{}}\PYG{l+s+s2}{\PYGZdq{}}\PYG{l+s+s2}{Relative Thickness [\PYGZhy{}]}\PYG{l+s+s2}{\PYGZdq{}}\PYG{p}{:} \PYG{n}{RT\PYGZus{}f}\PYG{p}{,} \PYG{l+s+s2}{\PYGZdq{}}\PYG{l+s+s2}{Relative Head [\PYGZhy{}]}\PYG{l+s+s2}{\PYGZdq{}}\PYG{p}{:} \PYG{n}{RH\PYGZus{}f}\PYG{p}{\PYGZcb{}}
    \PYG{n}{df3} \PYG{o}{=} \PYG{n}{pd}\PYG{o}{.}\PYG{n}{DataFrame}\PYG{p}{(}\PYG{n}{df3}\PYG{p}{)}
    

    \PYG{n}{fig} \PYG{o}{=} \PYG{n}{plt}\PYG{o}{.}\PYG{n}{figure}\PYG{p}{(}\PYG{p}{)}
    \PYG{n}{ax} \PYG{o}{=} \PYG{n}{fig}\PYG{o}{.}\PYG{n}{add\PYGZus{}subplot}\PYG{p}{(}\PYG{l+m+mi}{1}\PYG{p}{,}\PYG{l+m+mi}{1}\PYG{p}{,}\PYG{l+m+mi}{1}\PYG{p}{)}
    \PYG{n}{ax}\PYG{o}{.}\PYG{n}{set\PYGZus{}xlim}\PYG{p}{(}\PYG{l+m+mi}{0}\PYG{p}{,} \PYG{l+m+mf}{1.01}\PYG{p}{)}\PYG{p}{;} \PYG{n}{ax}\PYG{o}{.}\PYG{n}{set\PYGZus{}ylim}\PYG{p}{(}\PYG{l+m+mi}{0}\PYG{p}{,}\PYG{l+m+mf}{1.01}\PYG{p}{)}
    \PYG{n}{ax}\PYG{o}{.}\PYG{n}{xaxis}\PYG{o}{.}\PYG{n}{set\PYGZus{}ticks\PYGZus{}position}\PYG{p}{(}\PYG{l+s+s1}{\PYGZsq{}}\PYG{l+s+s1}{top}\PYG{l+s+s1}{\PYGZsq{}}\PYG{p}{)} 
    \PYG{n}{ax}\PYG{o}{.}\PYG{n}{xaxis}\PYG{o}{.}\PYG{n}{set\PYGZus{}label\PYGZus{}position}\PYG{p}{(}\PYG{l+s+s1}{\PYGZsq{}}\PYG{l+s+s1}{top}\PYG{l+s+s1}{\PYGZsq{}}\PYG{p}{)} 
    \PYG{n}{ax}\PYG{o}{.}\PYG{n}{set\PYGZus{}xlabel}\PYG{p}{(}\PYG{l+s+s2}{\PYGZdq{}}\PYG{l+s+s2}{Relative head [\PYGZhy{}]}\PYG{l+s+s2}{\PYGZdq{}}\PYG{p}{,} \PYG{n}{fontsize}\PYG{o}{=}\PYG{l+m+mi}{12}\PYG{p}{)}  
    \PYG{n}{ax}\PYG{o}{.}\PYG{n}{set\PYGZus{}ylabel}\PYG{p}{(}\PYG{l+s+s2}{\PYGZdq{}}\PYG{l+s+s2}{Relative thickness [\PYGZhy{}]}\PYG{l+s+s2}{\PYGZdq{}}\PYG{p}{,} \PYG{n}{fontsize}\PYG{o}{=}\PYG{l+m+mi}{12}\PYG{p}{)}  
    \PYG{n}{plt}\PYG{o}{.}\PYG{n}{gca}\PYG{p}{(}\PYG{p}{)}\PYG{o}{.}\PYG{n}{invert\PYGZus{}yaxis}\PYG{p}{(}\PYG{p}{)}
    \PYG{n}{ax}\PYG{o}{.}\PYG{n}{spines}\PYG{p}{[}\PYG{l+s+s1}{\PYGZsq{}}\PYG{l+s+s1}{right}\PYG{l+s+s1}{\PYGZsq{}}\PYG{p}{]}\PYG{o}{.}\PYG{n}{set\PYGZus{}visible}\PYG{p}{(}\PYG{k+kc}{False}\PYG{p}{)}
    \PYG{n}{ax}\PYG{o}{.}\PYG{n}{spines}\PYG{p}{[}\PYG{l+s+s1}{\PYGZsq{}}\PYG{l+s+s1}{bottom}\PYG{l+s+s1}{\PYGZsq{}}\PYG{p}{]}\PYG{o}{.}\PYG{n}{set\PYGZus{}visible}\PYG{p}{(}\PYG{k+kc}{False}\PYG{p}{)}
    
    \PYG{n}{ax}\PYG{o}{.}\PYG{n}{axhline}\PYG{p}{(}\PYG{n}{y}\PYG{o}{=}\PYG{l+m+mi}{0}\PYG{p}{,} \PYG{n}{color}\PYG{o}{=}\PYG{l+s+s1}{\PYGZsq{}}\PYG{l+s+s1}{r}\PYG{l+s+s1}{\PYGZsq{}}\PYG{p}{,} \PYG{n}{linewidth}\PYG{o}{=}\PYG{l+m+mi}{2}\PYG{p}{)}
    \PYG{n}{ax}\PYG{o}{.}\PYG{n}{axhline}\PYG{p}{(}\PYG{n}{y}\PYG{o}{=}\PYG{n}{RT2}\PYG{p}{,} \PYG{n}{color}\PYG{o}{=}\PYG{l+s+s1}{\PYGZsq{}}\PYG{l+s+s1}{r}\PYG{l+s+s1}{\PYGZsq{}}\PYG{p}{,} \PYG{n}{linewidth}\PYG{o}{=}\PYG{l+m+mi}{2}\PYG{p}{)}
    \PYG{n}{ax}\PYG{o}{.}\PYG{n}{axhline}\PYG{p}{(}\PYG{n}{y}\PYG{o}{=}\PYG{n}{RT3}\PYG{p}{,} \PYG{n}{color}\PYG{o}{=}\PYG{l+s+s1}{\PYGZsq{}}\PYG{l+s+s1}{r}\PYG{l+s+s1}{\PYGZsq{}}\PYG{p}{,} \PYG{n}{linewidth}\PYG{o}{=}\PYG{l+m+mi}{2}\PYG{p}{)}
    \PYG{n}{ax}\PYG{o}{.}\PYG{n}{axhline}\PYG{p}{(}\PYG{n}{y}\PYG{o}{=}\PYG{n}{RT4}\PYG{p}{,} \PYG{n}{color}\PYG{o}{=}\PYG{l+s+s1}{\PYGZsq{}}\PYG{l+s+s1}{r}\PYG{l+s+s1}{\PYGZsq{}}\PYG{p}{,} \PYG{n}{linewidth}\PYG{o}{=}\PYG{l+m+mi}{2}\PYG{p}{)}
    \PYG{n}{ax}\PYG{o}{.}\PYG{n}{plot}\PYG{p}{(}\PYG{n}{RH}\PYG{p}{,} \PYG{n}{RT}\PYG{p}{)}

    \PYG{n}{plt}\PYG{o}{.}\PYG{n}{xticks}\PYG{p}{(}\PYG{n}{np}\PYG{o}{.}\PYG{n}{arange}\PYG{p}{(}\PYG{l+m+mi}{0}\PYG{p}{,} \PYG{l+m+mf}{1.1}\PYG{p}{,} \PYG{l+m+mf}{0.1}\PYG{p}{)}\PYG{p}{)}
    \PYG{n}{plt}\PYG{o}{.}\PYG{n}{yticks}\PYG{p}{(}\PYG{n}{np}\PYG{o}{.}\PYG{n}{arange}\PYG{p}{(}\PYG{l+m+mi}{0}\PYG{p}{,} \PYG{l+m+mf}{1.1}\PYG{p}{,} \PYG{l+m+mf}{0.1}\PYG{p}{)}\PYG{p}{)}
    
    \PYG{n+nb}{print}\PYG{p}{(}\PYG{l+s+s2}{\PYGZdq{}}\PYG{l+s+se}{\PYGZbs{}n}\PYG{l+s+se}{\PYGZbs{}n}\PYG{l+s+se}{\PYGZbs{}033}\PYG{l+s+s2}{[1m Parallel flow: }\PYG{l+s+se}{\PYGZbs{}033}\PYG{l+s+s2}{[0m }\PYG{l+s+se}{\PYGZbs{}n}\PYG{l+s+s2}{\PYGZdq{}}\PYG{p}{)}
    
    \PYG{n+nb}{print}\PYG{p}{(}\PYG{l+s+s2}{\PYGZdq{}}\PYG{l+s+s2}{The Effective Hydraulic Conductivity is: }\PYG{l+s+si}{\PYGZob{}0:0.2e\PYGZcb{}}\PYG{l+s+s2}{\PYGZdq{}}\PYG{o}{.}\PYG{n}{format}\PYG{p}{(}\PYG{n}{HC\PYGZus{}eff}\PYG{p}{)}\PYG{p}{,} \PYG{l+s+s2}{\PYGZdq{}}\PYG{l+s+s2}{m/s}\PYG{l+s+se}{\PYGZbs{}n}\PYG{l+s+s2}{\PYGZdq{}}\PYG{p}{)}
    \PYG{n+nb}{print}\PYG{p}{(}\PYG{l+s+s2}{\PYGZdq{}}\PYG{l+s+s2}{The Approximate Effective Hydraulic Conductivity is: }\PYG{l+s+si}{\PYGZob{}0:0.2e\PYGZcb{}}\PYG{l+s+s2}{\PYGZdq{}}\PYG{o}{.}\PYG{n}{format}\PYG{p}{(}\PYG{n}{HC\PYGZus{}eff\PYGZus{}a}\PYG{p}{)}\PYG{p}{,} \PYG{l+s+s2}{\PYGZdq{}}\PYG{l+s+s2}{m/s}\PYG{l+s+se}{\PYGZbs{}n}\PYG{l+s+s2}{\PYGZdq{}}\PYG{p}{)}
    \PYG{n+nb}{print}\PYG{p}{(}\PYG{l+s+s2}{\PYGZdq{}}\PYG{l+s+s2}{The Effective Hydraulic Resistance is: }\PYG{l+s+si}{\PYGZob{}0:0.2e\PYGZcb{}}\PYG{l+s+s2}{\PYGZdq{}}\PYG{o}{.}\PYG{n}{format}\PYG{p}{(}\PYG{n}{HR\PYGZus{}eff}\PYG{p}{)}\PYG{p}{,} \PYG{l+s+s2}{\PYGZdq{}}\PYG{l+s+s2}{s/m}\PYG{l+s+se}{\PYGZbs{}n}\PYG{l+s+s2}{\PYGZdq{}}\PYG{p}{)}
    \PYG{n+nb}{print}\PYG{p}{(}\PYG{l+s+s2}{\PYGZdq{}}\PYG{l+s+s2}{The Approximate Effective Hydraulic Resistance is }\PYG{l+s+si}{\PYGZob{}0:0.2e\PYGZcb{}}\PYG{l+s+s2}{\PYGZdq{}}\PYG{o}{.}\PYG{n}{format}\PYG{p}{(}\PYG{n}{HR\PYGZus{}eff\PYGZus{}a}\PYG{p}{)}\PYG{p}{,} \PYG{l+s+s2}{\PYGZdq{}}\PYG{l+s+s2}{s/m}\PYG{l+s+se}{\PYGZbs{}n}\PYG{l+s+s2}{\PYGZdq{}}\PYG{p}{)}
    
    \PYG{n+nb}{print}\PYG{p}{(}\PYG{n}{df3}\PYG{p}{,} \PYG{l+s+s2}{\PYGZdq{}}\PYG{l+s+se}{\PYGZbs{}n}\PYG{l+s+s2}{\PYGZdq{}}\PYG{p}{)}
    \PYG{n}{plt}\PYG{o}{.}\PYG{n}{show}\PYG{p}{(}\PYG{n}{fig}\PYG{p}{)}
    
    \PYG{c+c1}{\PYGZsh{} Perpendendicular flow}
    
    \PYG{n}{WHK\PYGZus{}eff} \PYG{o}{=} \PYG{n+nb}{sum}\PYG{p}{(}\PYG{n}{WHK}\PYG{p}{)}
    \PYG{n}{WHK\PYGZus{}eff\PYGZus{}a} \PYG{o}{=} \PYG{n+nb}{max}\PYG{p}{(}\PYG{n}{WHK}\PYG{p}{)}

    \PYG{n}{WHR\PYGZus{}eff} \PYG{o}{=} \PYG{l+m+mi}{1}\PYG{o}{/}\PYG{n}{WHK\PYGZus{}eff}
    \PYG{n}{WHR\PYGZus{}eff\PYGZus{}a} \PYG{o}{=} \PYG{l+m+mi}{1}\PYG{o}{/}\PYG{n}{WHK\PYGZus{}eff\PYGZus{}a}

    \PYG{n}{RD1} \PYG{o}{=} \PYG{n}{WHK1}\PYG{o}{/}\PYG{n}{WHK\PYGZus{}eff}
    \PYG{n}{RD2} \PYG{o}{=} \PYG{n}{WHK2}\PYG{o}{/}\PYG{n}{WHK\PYGZus{}eff}
    \PYG{n}{RD3} \PYG{o}{=} \PYG{n}{WHK3}\PYG{o}{/}\PYG{n}{WHK\PYGZus{}eff}

    \PYG{n}{RD} \PYG{o}{=} \PYG{p}{[}\PYG{n}{RD1}\PYG{p}{,} \PYG{n}{RD2}\PYG{p}{,} \PYG{n}{RD3}\PYG{p}{]}
    \PYG{n}{RD\PYGZus{}f} \PYG{o}{=} \PYG{p}{[}\PYG{l+s+s2}{\PYGZdq{}}\PYG{l+s+si}{\PYGZpc{}0.2f}\PYG{l+s+s2}{\PYGZdq{}} \PYG{o}{\PYGZpc{}}\PYG{k}{elem} for elem in RD]

    \PYG{n}{df4} \PYG{o}{=} \PYG{n}{pd}\PYG{o}{.}\PYG{n}{DataFrame}\PYG{p}{(}\PYG{p}{\PYGZob{}}\PYG{l+s+s2}{\PYGZdq{}}\PYG{l+s+s2}{Relative Discharge [\PYGZhy{}]}\PYG{l+s+s2}{\PYGZdq{}}\PYG{p}{:} \PYG{n}{RD\PYGZus{}f}\PYG{p}{\PYGZcb{}}\PYG{p}{,} \PYG{n}{index}\PYG{o}{=} \PYG{n}{index}\PYG{p}{)}
    
    \PYG{n}{fig2} \PYG{o}{=} \PYG{n}{plt}\PYG{o}{.}\PYG{n}{figure}\PYG{p}{(}\PYG{p}{)}
    \PYG{n}{plt}\PYG{o}{.}\PYG{n}{gca}\PYG{p}{(}\PYG{p}{)}\PYG{o}{.}\PYG{n}{invert\PYGZus{}yaxis}\PYG{p}{(}\PYG{p}{)}
    \PYG{n}{ay} \PYG{o}{=} \PYG{n}{fig2}\PYG{o}{.}\PYG{n}{add\PYGZus{}subplot}\PYG{p}{(}\PYG{l+m+mi}{1}\PYG{p}{,}\PYG{l+m+mi}{1}\PYG{p}{,}\PYG{l+m+mi}{1}\PYG{p}{)}
    \PYG{n}{ay}\PYG{o}{.}\PYG{n}{barh}\PYG{p}{(}\PYG{n}{index}\PYG{p}{,} \PYG{n}{RD}\PYG{p}{)} 
    \PYG{n}{plt}\PYG{o}{.}\PYG{n}{xticks}\PYG{p}{(}\PYG{n}{np}\PYG{o}{.}\PYG{n}{arange}\PYG{p}{(}\PYG{l+m+mi}{0}\PYG{p}{,} \PYG{l+m+mf}{1.1}\PYG{p}{,} \PYG{l+m+mf}{0.1}\PYG{p}{)}\PYG{p}{)}
    \PYG{n}{ay}\PYG{o}{.}\PYG{n}{set\PYGZus{}xlabel}\PYG{p}{(}\PYG{l+s+s2}{\PYGZdq{}}\PYG{l+s+s2}{Relative discharge [\PYGZhy{}]}\PYG{l+s+s2}{\PYGZdq{}}\PYG{p}{,} \PYG{n}{fontsize}\PYG{o}{=}\PYG{l+m+mi}{12}\PYG{p}{)}
    \PYG{n}{ay}\PYG{o}{.}\PYG{n}{set\PYGZus{}xlabel}\PYG{p}{(}\PYG{l+s+s2}{\PYGZdq{}}\PYG{l+s+s2}{Layer number}\PYG{l+s+s2}{\PYGZdq{}}\PYG{p}{,} \PYG{n}{fontsize}\PYG{o}{=}\PYG{l+m+mi}{12}\PYG{p}{)}
    

    
    \PYG{n+nb}{print}\PYG{p}{(}\PYG{l+s+s2}{\PYGZdq{}}\PYG{l+s+se}{\PYGZbs{}n}\PYG{l+s+se}{\PYGZbs{}033}\PYG{l+s+s2}{[1m Perpendicular flow: }\PYG{l+s+se}{\PYGZbs{}033}\PYG{l+s+s2}{[0m }\PYG{l+s+se}{\PYGZbs{}n}\PYG{l+s+s2}{\PYGZdq{}}\PYG{p}{)}
    
    \PYG{n+nb}{print}\PYG{p}{(}\PYG{l+s+s2}{\PYGZdq{}}\PYG{l+s+s2}{The Effective Hydraulic Conductivity is: }\PYG{l+s+si}{\PYGZob{}0:0.2e\PYGZcb{}}\PYG{l+s+s2}{\PYGZdq{}}\PYG{o}{.}\PYG{n}{format}\PYG{p}{(}\PYG{n}{WHK\PYGZus{}eff}\PYG{p}{)}\PYG{p}{,} \PYG{l+s+s2}{\PYGZdq{}}\PYG{l+s+s2}{s/m }\PYG{l+s+se}{\PYGZbs{}n}\PYG{l+s+s2}{\PYGZdq{}}\PYG{p}{)}
    \PYG{n+nb}{print}\PYG{p}{(}\PYG{l+s+s2}{\PYGZdq{}}\PYG{l+s+s2}{The Approximate Effective Hydraulic Conductivity is }\PYG{l+s+si}{\PYGZob{}0:0.2e\PYGZcb{}}\PYG{l+s+s2}{\PYGZdq{}}\PYG{o}{.}\PYG{n}{format}\PYG{p}{(}\PYG{n}{WHK\PYGZus{}eff\PYGZus{}a}\PYG{p}{)}\PYG{p}{,} \PYG{l+s+s2}{\PYGZdq{}}\PYG{l+s+s2}{s/m}\PYG{l+s+se}{\PYGZbs{}n}\PYG{l+s+s2}{\PYGZdq{}}\PYG{p}{)}
    \PYG{n+nb}{print}\PYG{p}{(}\PYG{l+s+s2}{\PYGZdq{}}\PYG{l+s+s2}{The Effective Hydraulic Resistance is: }\PYG{l+s+si}{\PYGZob{}0:0.2e\PYGZcb{}}\PYG{l+s+s2}{\PYGZdq{}}\PYG{o}{.}\PYG{n}{format}\PYG{p}{(}\PYG{n}{WHR\PYGZus{}eff}\PYG{p}{)}\PYG{p}{,} \PYG{l+s+s2}{\PYGZdq{}}\PYG{l+s+s2}{m/s}\PYG{l+s+se}{\PYGZbs{}n}\PYG{l+s+s2}{\PYGZdq{}}\PYG{p}{)}
    \PYG{n+nb}{print}\PYG{p}{(}\PYG{l+s+s2}{\PYGZdq{}}\PYG{l+s+s2}{The Approximate Effective Hydraulic Resistance is: }\PYG{l+s+si}{\PYGZob{}0:0.2e\PYGZcb{}}\PYG{l+s+s2}{\PYGZdq{}}\PYG{o}{.}\PYG{n}{format}\PYG{p}{(}\PYG{n}{WHR\PYGZus{}eff\PYGZus{}a}\PYG{p}{)}\PYG{p}{,} \PYG{l+s+s2}{\PYGZdq{}}\PYG{l+s+s2}{m/s}\PYG{l+s+se}{\PYGZbs{}n}\PYG{l+s+se}{\PYGZbs{}n}\PYG{l+s+s2}{\PYGZdq{}}\PYG{p}{)}
    
    \PYG{n+nb}{print}\PYG{p}{(}\PYG{n}{df4}\PYG{p}{,} \PYG{l+s+s2}{\PYGZdq{}}\PYG{l+s+se}{\PYGZbs{}n}\PYG{l+s+s2}{\PYGZdq{}}\PYG{p}{)}
    \PYG{n}{plt}\PYG{o}{.}\PYG{n}{show}\PYG{p}{(}\PYG{n}{fig2}\PYG{p}{)}
    
\PYG{n}{style} \PYG{o}{=} \PYG{p}{\PYGZob{}}\PYG{l+s+s1}{\PYGZsq{}}\PYG{l+s+s1}{description\PYGZus{}width}\PYG{l+s+s1}{\PYGZsq{}}\PYG{p}{:} \PYG{l+s+s1}{\PYGZsq{}}\PYG{l+s+s1}{initial}\PYG{l+s+s1}{\PYGZsq{}}\PYG{p}{\PYGZcb{}}    
\PYG{n}{Inter}\PYG{o}{=}\PYG{n}{widgets}\PYG{o}{.}\PYG{n}{interact\PYGZus{}manual}\PYG{p}{(}\PYG{n}{eff\PYGZus{}K}\PYG{p}{,} 
                       \PYG{n}{M1}\PYG{o}{=} \PYG{n}{widgets}\PYG{o}{.}\PYG{n}{FloatText}\PYG{p}{(}\PYG{n}{description}\PYG{o}{=}\PYG{l+s+s2}{\PYGZdq{}}\PYG{l+s+s2}{Layer Thickness 1}\PYG{l+s+s2}{\PYGZdq{}}\PYG{p}{,} \PYG{n}{style}\PYG{o}{=}\PYG{n}{style}\PYG{p}{)}\PYG{p}{,}
                       \PYG{n}{K1}\PYG{o}{=} \PYG{n}{widgets}\PYG{o}{.}\PYG{n}{FloatText}\PYG{p}{(}\PYG{n}{description}\PYG{o}{=}\PYG{l+s+s2}{\PYGZdq{}}\PYG{l+s+s2}{Hydraulic Conductivity 1}\PYG{l+s+s2}{\PYGZdq{}}\PYG{p}{,}\PYG{n}{style}\PYG{o}{=}\PYG{n}{style}\PYG{p}{)}\PYG{p}{,}
                       \PYG{n}{M2}\PYG{o}{=} \PYG{n}{widgets}\PYG{o}{.}\PYG{n}{FloatText}\PYG{p}{(}\PYG{n}{description}\PYG{o}{=}\PYG{l+s+s2}{\PYGZdq{}}\PYG{l+s+s2}{Layer Thickness 2}\PYG{l+s+s2}{\PYGZdq{}}\PYG{p}{,} \PYG{n}{style}\PYG{o}{=}\PYG{n}{style}\PYG{p}{)}\PYG{p}{,}
                       \PYG{n}{K2}\PYG{o}{=} \PYG{n}{widgets}\PYG{o}{.}\PYG{n}{FloatText}\PYG{p}{(}\PYG{n}{description}\PYG{o}{=}\PYG{l+s+s2}{\PYGZdq{}}\PYG{l+s+s2}{Hydraulic Conductivity 2}\PYG{l+s+s2}{\PYGZdq{}}\PYG{p}{,} \PYG{n}{style}\PYG{o}{=}\PYG{n}{style}\PYG{p}{)}\PYG{p}{,}
                       \PYG{n}{M3}\PYG{o}{=} \PYG{n}{widgets}\PYG{o}{.}\PYG{n}{FloatText}\PYG{p}{(}\PYG{n}{description}\PYG{o}{=}\PYG{l+s+s2}{\PYGZdq{}}\PYG{l+s+s2}{Layer Thickness 3}\PYG{l+s+s2}{\PYGZdq{}}\PYG{p}{,} \PYG{n}{style}\PYG{o}{=}\PYG{n}{style}\PYG{p}{)}\PYG{p}{,}
                       \PYG{n}{K3}\PYG{o}{=} \PYG{n}{widgets}\PYG{o}{.}\PYG{n}{FloatText}\PYG{p}{(}\PYG{n}{description}\PYG{o}{=}\PYG{l+s+s2}{\PYGZdq{}}\PYG{l+s+s2}{Hydraulic Conductivity 3}\PYG{l+s+s2}{\PYGZdq{}}\PYG{p}{,} \PYG{n}{style}\PYG{o}{=}\PYG{n}{style}\PYG{p}{)}\PYG{p}{)}
\end{sphinxVerbatim}

\end{sphinxuseclass}\end{sphinxVerbatimInput}
\begin{sphinxVerbatimOutput}

\begin{sphinxuseclass}{cell_output}
\begin{sphinxVerbatim}[commandchars=\\\{\}]
interactive(children=(FloatText(value=0.0, description=\PYGZsq{}Layer Thickness 1\PYGZsq{}, style=DescriptionStyle(description…
\end{sphinxVerbatim}

\end{sphinxuseclass}\end{sphinxVerbatimOutput}

\end{sphinxuseclass}
\sphinxstepscope


\chapter{Uniform Flow and Well*}
\label{\detokenize{content/tools/uniform_flow_and_well:uniform-flow-and-well}}\label{\detokenize{content/tools/uniform_flow_and_well::doc}}
\sphinxAtStartPar
The worksheet addresses the superposition of uniform and radial steady\sphinxhyphen{}state groundwater flow 
in a homogeneous, confined aquifer of uniform thickness without recharge. 
The radial flow component may represent an extraction or injection well. 

\sphinxAtStartPar
The worksheet calculates hydraulic head isolines (red), streamlines (blue / black), and isochrones (green) 
by using an analytical solution. The set of streamlines includes the dividing streamline (black). 
In addition, the capture width of the well (dashed lines) and the position of the stagnation point are determined. 
Three travel time values representing isochrones can be selected by the user. 


\begin{savenotes}\sphinxattablestart
\centering
\begin{tabulary}{\linewidth}[t]{|T|T|T|}
\hline
\sphinxstyletheadfamily 
\sphinxAtStartPar
input parameters
&\sphinxstyletheadfamily 
\sphinxAtStartPar
units
&\sphinxstyletheadfamily 
\sphinxAtStartPar
remarks
\\
\hline
\sphinxAtStartPar
hydraulic conductivity
&
\sphinxAtStartPar
m/s
&
\sphinxAtStartPar
enter positive number
\\
\hline
\sphinxAtStartPar
effective porosity
&
\sphinxAtStartPar
\sphinxhyphen{}
&
\sphinxAtStartPar
enter number between 0 and 1
\\
\hline
\sphinxAtStartPar
thickness
&
\sphinxAtStartPar
mm/a
&
\sphinxAtStartPar
enter positive number
\\
\hline
\sphinxAtStartPar
uniform velocity
&
\sphinxAtStartPar
m/d
&
\sphinxAtStartPar
enter number different from zero\(*\)
\\
\hline
\sphinxAtStartPar
pumping rate
&
\sphinxAtStartPar
m³/d
&
\sphinxAtStartPar
enter number different from zero\(**\)
\\
\hline
\sphinxAtStartPar
travel time
&
\sphinxAtStartPar
d
&
\sphinxAtStartPar
enter positive number
\\
\hline
\end{tabulary}
\par
\sphinxattableend\end{savenotes}

\sphinxAtStartPar
\(*\) Positive or negative numbers correspond to uniform flow in parallel with or antiparallel to the x\sphinxhyphen{}axis, resp. 
\(**\) Positive or negative numbers correspond to water extraction or injection, resp.

\sphinxAtStartPar
\sphinxstylestrong{\sphinxstyleemphasis{Contributed by Ms. Anne Pförtner and Sophie Pförtner. The original concept from Prof. R. Liedl spreasheet code.}}

\sphinxAtStartPar
The codes are licensed under CC by 4.0 \sphinxhref{https://creativecommons.org/licenses/by/4.0/deed.en}{(use anyways, but acknowledge the original work)}

\begin{sphinxuseclass}{cell}\begin{sphinxVerbatimInput}

\begin{sphinxuseclass}{cell_input}
\begin{sphinxVerbatim}[commandchars=\\\{\}]
\PYG{k+kn}{import} \PYG{n+nn}{numpy} \PYG{k}{as} \PYG{n+nn}{np}
\PYG{k+kn}{from} \PYG{n+nn}{ipywidgets} \PYG{k+kn}{import} \PYG{o}{*}
\PYG{k+kn}{import} \PYG{n+nn}{matplotlib}\PYG{n+nn}{.}\PYG{n+nn}{pyplot} \PYG{k}{as} \PYG{n+nn}{plt}

\PYG{k+kn}{import} \PYG{n+nn}{warnings}
\PYG{n}{warnings}\PYG{o}{.}\PYG{n}{filterwarnings}\PYG{p}{(}\PYG{l+s+s1}{\PYGZsq{}}\PYG{l+s+s1}{ignore}\PYG{l+s+s1}{\PYGZsq{}}\PYG{p}{)}

\PYG{c+c1}{\PYGZsh{}definition of the function }
\PYG{k}{def} \PYG{n+nf}{uniform\PYGZus{}flow}\PYG{p}{(}\PYG{n}{K}\PYG{p}{,} \PYG{n}{ne}\PYG{p}{,} \PYG{n}{m}\PYG{p}{,} \PYG{n}{v}\PYG{p}{,} \PYG{n}{Q}\PYG{p}{,} \PYG{n}{t1}\PYG{p}{,} \PYG{n}{t2}\PYG{p}{,} \PYG{n}{t3}\PYG{p}{)}\PYG{p}{:}

    \PYG{c+c1}{\PYGZsh{}intermediate results }
    \PYG{n}{K\PYGZus{}1} \PYG{o}{=} \PYG{n}{K}\PYG{o}{*}\PYG{l+m+mi}{86400}                       \PYG{c+c1}{\PYGZsh{}hydraulic conductivity [m/d]}
    \PYG{n}{capture\PYGZus{}width} \PYG{o}{=} \PYG{n}{Q}\PYG{o}{/}\PYG{n}{m}\PYG{o}{/}\PYG{n}{v}               \PYG{c+c1}{\PYGZsh{}capture width [m]}
    \PYG{n}{L\PYGZus{}ref} \PYG{o}{=} \PYG{n}{Q}\PYG{o}{/}\PYG{p}{(}\PYG{l+m+mi}{2}\PYG{o}{*}\PYG{n}{np}\PYG{o}{.}\PYG{n}{pi}\PYG{o}{*}\PYG{n}{np}\PYG{o}{.}\PYG{n}{exp}\PYG{p}{(}\PYG{l+m+mi}{1}\PYG{p}{)}\PYG{o}{*}\PYG{n}{m}\PYG{o}{*}\PYG{n}{v}\PYG{p}{)}   \PYG{c+c1}{\PYGZsh{}[m]}
    \PYG{n}{h\PYGZus{}ref} \PYG{o}{=} \PYG{n}{v}\PYG{o}{/}\PYG{n}{K\PYGZus{}1}\PYG{o}{*}\PYG{n}{L\PYGZus{}ref}                 \PYG{c+c1}{\PYGZsh{}[m]}
    \PYG{n}{t\PYGZus{}ref} \PYG{o}{=} \PYG{l+m+mf}{0.5}\PYG{o}{*}\PYG{n}{ne}\PYG{o}{*}\PYG{n}{Q}\PYG{o}{/}\PYG{n}{np}\PYG{o}{.}\PYG{n}{pi}\PYG{o}{/}\PYG{n}{m}\PYG{o}{/}\PYG{n}{v}\PYG{o}{*}\PYG{o}{*}\PYG{l+m+mi}{2}       \PYG{c+c1}{\PYGZsh{}[d]}
    \PYG{n}{stagnation\PYGZus{}x} \PYG{o}{=} \PYG{n}{np}\PYG{o}{.}\PYG{n}{exp}\PYG{p}{(}\PYG{l+m+mi}{1}\PYG{p}{)}\PYG{o}{*}\PYG{n}{L\PYGZus{}ref}      \PYG{c+c1}{\PYGZsh{}stagnation point (x) [m]}
    \PYG{n}{stagnation\PYGZus{}y} \PYG{o}{=} \PYG{l+m+mi}{0}                    \PYG{c+c1}{\PYGZsh{}stagnation point (y) [m]}

    \PYG{c+c1}{\PYGZsh{}isolines; Syntax: isolines\PYGZus{}[X or Y]\PYGZus{}plot\PYGZus{}[h\PYGZus{}ref]\PYGZus{}[optional: 1 or 2]}
    \PYG{n}{isolines\PYGZus{}x\PYGZus{}plot\PYGZus{}n5\PYGZus{}1}\PYG{o}{=}\PYG{p}{[}\PYG{p}{]} 
    \PYG{n}{isolines\PYGZus{}y\PYGZus{}plot\PYGZus{}n5\PYGZus{}1}\PYG{o}{=}\PYG{p}{[}\PYG{p}{]}
    \PYG{n}{isolines\PYGZus{}x\PYGZus{}plot\PYGZus{}n5\PYGZus{}2} \PYG{o}{=} \PYG{p}{[}\PYG{p}{]}
    \PYG{n}{isolines\PYGZus{}y\PYGZus{}plot\PYGZus{}n5\PYGZus{}2} \PYG{o}{=} \PYG{p}{[}\PYG{p}{]}
    \PYG{n}{isolines\PYGZus{}x\PYGZus{}plot\PYGZus{}n2\PYGZus{}5\PYGZus{}1} \PYG{o}{=} \PYG{p}{[}\PYG{p}{]}
    \PYG{n}{isolines\PYGZus{}y\PYGZus{}plot\PYGZus{}n2\PYGZus{}5\PYGZus{}1} \PYG{o}{=} \PYG{p}{[}\PYG{p}{]}
    \PYG{n}{isolines\PYGZus{}x\PYGZus{}plot\PYGZus{}n2\PYGZus{}5\PYGZus{}2} \PYG{o}{=} \PYG{p}{[}\PYG{p}{]}
    \PYG{n}{isolines\PYGZus{}y\PYGZus{}plot\PYGZus{}n2\PYGZus{}5\PYGZus{}2} \PYG{o}{=} \PYG{p}{[}\PYG{p}{]}
    \PYG{n}{isolines\PYGZus{}x\PYGZus{}plot\PYGZus{}0\PYGZus{}1} \PYG{o}{=} \PYG{p}{[}\PYG{p}{]}
    \PYG{n}{isolines\PYGZus{}y\PYGZus{}plot\PYGZus{}0\PYGZus{}1} \PYG{o}{=} \PYG{p}{[}\PYG{p}{]}
    \PYG{n}{isolines\PYGZus{}x\PYGZus{}plot\PYGZus{}0\PYGZus{}2} \PYG{o}{=} \PYG{p}{[}\PYG{p}{]}
    \PYG{n}{isolines\PYGZus{}y\PYGZus{}plot\PYGZus{}0\PYGZus{}2} \PYG{o}{=} \PYG{p}{[}\PYG{p}{]}
    \PYG{n}{isolines\PYGZus{}x\PYGZus{}plot\PYGZus{}2\PYGZus{}5} \PYG{o}{=} \PYG{p}{[}\PYG{p}{]}
    \PYG{n}{isolines\PYGZus{}y\PYGZus{}plot\PYGZus{}2\PYGZus{}5} \PYG{o}{=} \PYG{p}{[}\PYG{p}{]}
    \PYG{n}{isolines\PYGZus{}x\PYGZus{}plot\PYGZus{}5} \PYG{o}{=} \PYG{p}{[}\PYG{p}{]}
    \PYG{n}{isolines\PYGZus{}y\PYGZus{}plot\PYGZus{}5} \PYG{o}{=} \PYG{p}{[}\PYG{p}{]}
    \PYG{n}{isolines\PYGZus{}x\PYGZus{}plot\PYGZus{}7\PYGZus{}5} \PYG{o}{=} \PYG{p}{[}\PYG{p}{]}
    \PYG{n}{isolines\PYGZus{}y\PYGZus{}plot\PYGZus{}7\PYGZus{}5} \PYG{o}{=} \PYG{p}{[}\PYG{p}{]}
    \PYG{n}{isolines\PYGZus{}x\PYGZus{}plot\PYGZus{}10} \PYG{o}{=} \PYG{p}{[}\PYG{p}{]}
    \PYG{n}{isolines\PYGZus{}y\PYGZus{}plot\PYGZus{}10} \PYG{o}{=} \PYG{p}{[}\PYG{p}{]}
    \PYG{n}{isolines\PYGZus{}x\PYGZus{}plot\PYGZus{}12\PYGZus{}5} \PYG{o}{=} \PYG{p}{[}\PYG{p}{]}
    \PYG{n}{isolines\PYGZus{}y\PYGZus{}plot\PYGZus{}12\PYGZus{}5} \PYG{o}{=} \PYG{p}{[}\PYG{p}{]}
    \PYG{n}{isolines\PYGZus{}x\PYGZus{}plot\PYGZus{}15} \PYG{o}{=} \PYG{p}{[}\PYG{p}{]}
    \PYG{n}{isolines\PYGZus{}y\PYGZus{}plot\PYGZus{}15} \PYG{o}{=} \PYG{p}{[}\PYG{p}{]}
    
    
    \PYG{k}{for} \PYG{n}{x} \PYG{o+ow}{in} \PYG{n+nb}{range}\PYG{p}{(}\PYG{l+m+mi}{0}\PYG{p}{,} \PYG{l+m+mi}{100}\PYG{p}{)}\PYG{p}{:}
        \PYG{n}{isolines\PYGZus{}x\PYGZus{}n5\PYGZus{}1}\PYG{o}{=}\PYG{n}{L\PYGZus{}ref}\PYG{o}{*}\PYG{p}{(}\PYG{p}{(}\PYG{n}{x}\PYG{o}{*}\PYG{l+m+mf}{0.169103048517306}\PYG{o}{+}\PYG{p}{(}\PYG{l+m+mi}{100}\PYG{o}{\PYGZhy{}}\PYG{n}{x}\PYG{p}{)}\PYG{o}{*}\PYG{o}{\PYGZhy{}}\PYG{l+m+mf}{0.150360933444141}\PYG{p}{)}\PYG{o}{/}\PYG{l+m+mi}{100}\PYG{p}{)}
        \PYG{n}{isolines\PYGZus{}x\PYGZus{}n5\PYGZus{}2}\PYG{o}{=}\PYG{n}{L\PYGZus{}ref}\PYG{o}{*}\PYG{p}{(}\PYG{p}{(}\PYG{n}{x}\PYG{o}{*}\PYG{l+m+mf}{12.35}\PYG{o}{+}\PYG{p}{(}\PYG{l+m+mi}{100}\PYG{o}{\PYGZhy{}}\PYG{n}{x}\PYG{p}{)}\PYG{o}{*}\PYG{l+m+mf}{11.6815653622516}\PYG{p}{)}\PYG{o}{/}\PYG{l+m+mi}{100}\PYG{p}{)}
        \PYG{n}{isolines\PYGZus{}x\PYGZus{}n2\PYGZus{}5\PYGZus{}1}\PYG{o}{=}\PYG{n}{L\PYGZus{}ref}\PYG{o}{*}\PYG{p}{(}\PYG{p}{(}\PYG{n}{x}\PYG{o}{*}\PYG{l+m+mf}{0.474722923528955}\PYG{o}{+}\PYG{p}{(}\PYG{l+m+mi}{100}\PYG{o}{\PYGZhy{}}\PYG{n}{x}\PYG{p}{)}\PYG{o}{*}\PYG{o}{\PYGZhy{}}\PYG{l+m+mf}{0.350418198256065}\PYG{p}{)}\PYG{o}{/}\PYG{l+m+mi}{100}\PYG{p}{)}
        \PYG{n}{isolines\PYGZus{}x\PYGZus{}n2\PYGZus{}5\PYGZus{}2}\PYG{o}{=}\PYG{n}{L\PYGZus{}ref}\PYG{o}{*}\PYG{p}{(}\PYG{p}{(}\PYG{n}{x}\PYG{o}{*}\PYG{l+m+mf}{9.45}\PYG{o}{+}\PYG{p}{(}\PYG{l+m+mi}{100}\PYG{o}{\PYGZhy{}}\PYG{n}{x}\PYG{p}{)}\PYG{o}{*}\PYG{l+m+mf}{8.22933315122817}\PYG{p}{)}\PYG{o}{/}\PYG{l+m+mi}{100}\PYG{p}{)}
        \PYG{n}{isolines\PYGZus{}x\PYGZus{}0\PYGZus{}1}\PYG{o}{=}\PYG{n}{L\PYGZus{}ref}\PYG{o}{*}\PYG{p}{(}\PYG{p}{(}\PYG{n}{x}\PYG{o}{*}\PYG{n}{np}\PYG{o}{.}\PYG{n}{exp}\PYG{p}{(}\PYG{l+m+mi}{1}\PYG{p}{)}\PYG{o}{+}\PYG{p}{(}\PYG{l+m+mi}{100}\PYG{o}{\PYGZhy{}}\PYG{n}{x}\PYG{p}{)}\PYG{o}{*}\PYG{o}{\PYGZhy{}}\PYG{l+m+mf}{0.75695357132717}\PYG{p}{)}\PYG{o}{/}\PYG{l+m+mi}{100}\PYG{p}{)}
        \PYG{n}{isolines\PYGZus{}x\PYGZus{}0\PYGZus{}2}\PYG{o}{=}\PYG{n}{L\PYGZus{}ref}\PYG{o}{*}\PYG{p}{(}\PYG{p}{(}\PYG{n}{x}\PYG{o}{*}\PYG{l+m+mf}{6.65}\PYG{o}{+}\PYG{p}{(}\PYG{l+m+mi}{100}\PYG{o}{\PYGZhy{}}\PYG{n}{x}\PYG{p}{)}\PYG{o}{*}\PYG{n}{np}\PYG{o}{.}\PYG{n}{exp}\PYG{p}{(}\PYG{l+m+mi}{1}\PYG{p}{)}\PYG{p}{)}\PYG{o}{/}\PYG{l+m+mi}{100}\PYG{p}{)}
        \PYG{n}{isolines\PYGZus{}x\PYGZus{}2\PYGZus{}5}\PYG{o}{=}\PYG{n}{L\PYGZus{}ref}\PYG{o}{*}\PYG{p}{(}\PYG{p}{(}\PYG{n}{x}\PYG{o}{*}\PYG{l+m+mi}{4}\PYG{o}{+}\PYG{p}{(}\PYG{l+m+mi}{100}\PYG{o}{\PYGZhy{}}\PYG{n}{x}\PYG{p}{)}\PYG{o}{*}\PYG{o}{\PYGZhy{}}\PYG{l+m+mf}{1.46395392968976}\PYG{p}{)}\PYG{o}{/}\PYG{l+m+mi}{100}\PYG{p}{)}
        \PYG{n}{isolines\PYGZus{}x\PYGZus{}5}\PYG{o}{=}\PYG{n}{L\PYGZus{}ref}\PYG{o}{*}\PYG{p}{(}\PYG{p}{(}\PYG{n}{x}\PYG{o}{*}\PYG{l+m+mf}{1.5}\PYG{o}{+}\PYG{p}{(}\PYG{l+m+mi}{100}\PYG{o}{\PYGZhy{}}\PYG{n}{x}\PYG{p}{)}\PYG{o}{*}\PYG{o}{\PYGZhy{}}\PYG{l+m+mf}{2.50444142220744}\PYG{p}{)}\PYG{o}{/}\PYG{l+m+mi}{100}\PYG{p}{)}
        \PYG{n}{isolines\PYGZus{}x\PYGZus{}7\PYGZus{}5}\PYG{o}{=}\PYG{n}{L\PYGZus{}ref}\PYG{o}{*}\PYG{p}{(}\PYG{p}{(}\PYG{n}{x}\PYG{o}{*}\PYG{o}{\PYGZhy{}}\PYG{l+m+mf}{0.875}\PYG{o}{+}\PYG{p}{(}\PYG{l+m+mi}{100}\PYG{o}{\PYGZhy{}}\PYG{n}{x}\PYG{p}{)}\PYG{o}{*}\PYG{o}{\PYGZhy{}}\PYG{l+m+mf}{3.84154019983304}\PYG{p}{)}\PYG{o}{/}\PYG{l+m+mi}{100}\PYG{p}{)}
        \PYG{n}{isolines\PYGZus{}x\PYGZus{}10}\PYG{o}{=}\PYG{n}{L\PYGZus{}ref}\PYG{o}{*}\PYG{p}{(}\PYG{p}{(}\PYG{n}{x}\PYG{o}{*}\PYG{o}{\PYGZhy{}}\PYG{l+m+mf}{3.15}\PYG{o}{+}\PYG{p}{(}\PYG{l+m+mi}{100}\PYG{o}{\PYGZhy{}}\PYG{n}{x}\PYG{p}{)}\PYG{o}{*}\PYG{o}{\PYGZhy{}}\PYG{l+m+mf}{5.4105773228373}\PYG{p}{)}\PYG{o}{/}\PYG{l+m+mi}{100}\PYG{p}{)}
        \PYG{n}{isolines\PYGZus{}x\PYGZus{}12\PYGZus{}5}\PYG{o}{=}\PYG{n}{L\PYGZus{}ref}\PYG{o}{*}\PYG{p}{(}\PYG{p}{(}\PYG{n}{x}\PYG{o}{*}\PYG{o}{\PYGZhy{}}\PYG{l+m+mf}{5.4}\PYG{o}{+}\PYG{p}{(}\PYG{l+m+mi}{100}\PYG{o}{\PYGZhy{}}\PYG{n}{x}\PYG{p}{)}\PYG{o}{*}\PYG{o}{\PYGZhy{}}\PYG{l+m+mf}{7.15205676143326}\PYG{p}{)}\PYG{o}{/}\PYG{l+m+mi}{100}\PYG{p}{)}
        \PYG{n}{isolines\PYGZus{}x\PYGZus{}15}\PYG{o}{=}\PYG{n}{L\PYGZus{}ref}\PYG{o}{*}\PYG{p}{(}\PYG{p}{(}\PYG{n}{x}\PYG{o}{*}\PYG{o}{\PYGZhy{}}\PYG{l+m+mf}{7.65}\PYG{o}{+}\PYG{p}{(}\PYG{l+m+mi}{100}\PYG{o}{\PYGZhy{}}\PYG{n}{x}\PYG{p}{)}\PYG{o}{*}\PYG{o}{\PYGZhy{}}\PYG{l+m+mf}{9.02099613666581}\PYG{p}{)}\PYG{o}{/}\PYG{l+m+mi}{100}\PYG{p}{)}
        
        \PYG{k}{if} \PYG{n}{x} \PYG{o}{==} \PYG{l+m+mi}{0}\PYG{p}{:}
            \PYG{n}{isolines\PYGZus{}y\PYGZus{}n5\PYGZus{}1} \PYG{o}{=} \PYG{l+m+mi}{0}
            \PYG{n}{isolines\PYGZus{}y\PYGZus{}n5\PYGZus{}2} \PYG{o}{=} \PYG{l+m+mi}{0}
            \PYG{n}{isolines\PYGZus{}y\PYGZus{}n2\PYGZus{}5\PYGZus{}1} \PYG{o}{=} \PYG{l+m+mi}{0}
            \PYG{n}{isolines\PYGZus{}y\PYGZus{}n2\PYGZus{}5\PYGZus{}2} \PYG{o}{=} \PYG{l+m+mi}{0}
            \PYG{n}{isolines\PYGZus{}y\PYGZus{}0\PYGZus{}1} \PYG{o}{=} \PYG{l+m+mi}{0}
            \PYG{n}{isolines\PYGZus{}y\PYGZus{}0\PYGZus{}2} \PYG{o}{=} \PYG{l+m+mi}{0}
            \PYG{n}{isolines\PYGZus{}y\PYGZus{}2\PYGZus{}5} \PYG{o}{=} \PYG{l+m+mi}{0}
            \PYG{n}{isolines\PYGZus{}y\PYGZus{}5} \PYG{o}{=} \PYG{l+m+mi}{0}
            \PYG{n}{isolines\PYGZus{}y\PYGZus{}7\PYGZus{}5} \PYG{o}{=} \PYG{l+m+mi}{0}
            \PYG{n}{isolines\PYGZus{}y\PYGZus{}10} \PYG{o}{=} \PYG{l+m+mi}{0}
            \PYG{n}{isolines\PYGZus{}y\PYGZus{}12\PYGZus{}5} \PYG{o}{=} \PYG{l+m+mi}{0}
            \PYG{n}{isolines\PYGZus{}y\PYGZus{}15} \PYG{o}{=} \PYG{l+m+mi}{0}
            
        \PYG{k}{else}\PYG{p}{:}
            \PYG{n}{isolines\PYGZus{}y\PYGZus{}n5\PYGZus{}1} \PYG{o}{=} \PYG{n}{np}\PYG{o}{.}\PYG{n}{sqrt}\PYG{p}{(}\PYG{p}{(}\PYG{n}{L\PYGZus{}ref}\PYG{o}{*}\PYG{n}{np}\PYG{o}{.}\PYG{n}{exp}\PYG{p}{(}\PYG{o}{\PYGZhy{}}\PYG{l+m+mi}{5}\PYG{o}{/}\PYG{n}{np}\PYG{o}{.}\PYG{n}{exp}\PYG{p}{(}\PYG{l+m+mi}{1}\PYG{p}{)}\PYG{o}{+}\PYG{n}{isolines\PYGZus{}x\PYGZus{}n5\PYGZus{}1}\PYG{o}{/}\PYG{n}{L\PYGZus{}ref}\PYG{o}{/}\PYG{n}{np}\PYG{o}{.}\PYG{n}{exp}\PYG{p}{(}\PYG{l+m+mi}{1}\PYG{p}{)}\PYG{p}{)}\PYG{p}{)}\PYG{o}{*}\PYG{o}{*}\PYG{l+m+mi}{2}\PYG{o}{\PYGZhy{}}\PYG{n}{isolines\PYGZus{}x\PYGZus{}n5\PYGZus{}1}\PYG{o}{*}\PYG{o}{*}\PYG{l+m+mi}{2}\PYG{p}{)}
            \PYG{n}{isolines\PYGZus{}y\PYGZus{}n5\PYGZus{}2} \PYG{o}{=} \PYG{n}{np}\PYG{o}{.}\PYG{n}{sqrt}\PYG{p}{(}\PYG{p}{(}\PYG{n}{L\PYGZus{}ref}\PYG{o}{*}\PYG{n}{np}\PYG{o}{.}\PYG{n}{exp}\PYG{p}{(}\PYG{o}{\PYGZhy{}}\PYG{l+m+mi}{5}\PYG{o}{/}\PYG{n}{np}\PYG{o}{.}\PYG{n}{exp}\PYG{p}{(}\PYG{l+m+mi}{1}\PYG{p}{)}\PYG{o}{+}\PYG{n}{isolines\PYGZus{}x\PYGZus{}n5\PYGZus{}2}\PYG{o}{/}\PYG{n}{L\PYGZus{}ref}\PYG{o}{/}\PYG{n}{np}\PYG{o}{.}\PYG{n}{exp}\PYG{p}{(}\PYG{l+m+mi}{1}\PYG{p}{)}\PYG{p}{)}\PYG{p}{)}\PYG{o}{*}\PYG{o}{*}\PYG{l+m+mi}{2}\PYG{o}{\PYGZhy{}}\PYG{n}{isolines\PYGZus{}x\PYGZus{}n5\PYGZus{}2}\PYG{o}{*}\PYG{o}{*}\PYG{l+m+mi}{2}\PYG{p}{)}
            \PYG{n}{isolines\PYGZus{}y\PYGZus{}n2\PYGZus{}5\PYGZus{}1} \PYG{o}{=} \PYG{n}{np}\PYG{o}{.}\PYG{n}{sqrt}\PYG{p}{(}\PYG{p}{(}\PYG{n}{L\PYGZus{}ref}\PYG{o}{*}\PYG{n}{np}\PYG{o}{.}\PYG{n}{exp}\PYG{p}{(}\PYG{o}{\PYGZhy{}}\PYG{l+m+mf}{2.5}\PYG{o}{/}\PYG{n}{np}\PYG{o}{.}\PYG{n}{exp}\PYG{p}{(}\PYG{l+m+mi}{1}\PYG{p}{)}\PYG{o}{+}\PYG{n}{isolines\PYGZus{}x\PYGZus{}n2\PYGZus{}5\PYGZus{}1}\PYG{o}{/}\PYG{n}{L\PYGZus{}ref}\PYG{o}{/}\PYG{n}{np}\PYG{o}{.}\PYG{n}{exp}\PYG{p}{(}\PYG{l+m+mi}{1}\PYG{p}{)}\PYG{p}{)}\PYG{p}{)}\PYG{o}{*}\PYG{o}{*}\PYG{l+m+mi}{2}\PYG{o}{\PYGZhy{}}\PYG{n}{isolines\PYGZus{}x\PYGZus{}n2\PYGZus{}5\PYGZus{}1}\PYG{o}{*}\PYG{o}{*}\PYG{l+m+mi}{2}\PYG{p}{)}
            \PYG{n}{isolines\PYGZus{}y\PYGZus{}n2\PYGZus{}5\PYGZus{}2} \PYG{o}{=} \PYG{n}{np}\PYG{o}{.}\PYG{n}{sqrt}\PYG{p}{(}\PYG{p}{(}\PYG{n}{L\PYGZus{}ref}\PYG{o}{*}\PYG{n}{np}\PYG{o}{.}\PYG{n}{exp}\PYG{p}{(}\PYG{o}{\PYGZhy{}}\PYG{l+m+mf}{2.5}\PYG{o}{/}\PYG{n}{np}\PYG{o}{.}\PYG{n}{exp}\PYG{p}{(}\PYG{l+m+mi}{1}\PYG{p}{)}\PYG{o}{+}\PYG{n}{isolines\PYGZus{}x\PYGZus{}n2\PYGZus{}5\PYGZus{}2}\PYG{o}{/}\PYG{n}{L\PYGZus{}ref}\PYG{o}{/}\PYG{n}{np}\PYG{o}{.}\PYG{n}{exp}\PYG{p}{(}\PYG{l+m+mi}{1}\PYG{p}{)}\PYG{p}{)}\PYG{p}{)}\PYG{o}{*}\PYG{o}{*}\PYG{l+m+mi}{2}\PYG{o}{\PYGZhy{}}\PYG{n}{isolines\PYGZus{}x\PYGZus{}n2\PYGZus{}5\PYGZus{}2}\PYG{o}{*}\PYG{o}{*}\PYG{l+m+mi}{2}\PYG{p}{)}
            \PYG{n}{isolines\PYGZus{}y\PYGZus{}0\PYGZus{}1} \PYG{o}{=} \PYG{n}{np}\PYG{o}{.}\PYG{n}{sqrt}\PYG{p}{(}\PYG{p}{(}\PYG{n}{L\PYGZus{}ref}\PYG{o}{*}\PYG{n}{np}\PYG{o}{.}\PYG{n}{exp}\PYG{p}{(}\PYG{l+m+mi}{0}\PYG{o}{/}\PYG{n}{np}\PYG{o}{.}\PYG{n}{exp}\PYG{p}{(}\PYG{l+m+mi}{1}\PYG{p}{)}\PYG{o}{+}\PYG{n}{isolines\PYGZus{}x\PYGZus{}0\PYGZus{}1}\PYG{o}{/}\PYG{n}{L\PYGZus{}ref}\PYG{o}{/}\PYG{n}{np}\PYG{o}{.}\PYG{n}{exp}\PYG{p}{(}\PYG{l+m+mi}{1}\PYG{p}{)}\PYG{p}{)}\PYG{p}{)}\PYG{o}{*}\PYG{o}{*}\PYG{l+m+mi}{2}\PYG{o}{\PYGZhy{}}\PYG{n}{isolines\PYGZus{}x\PYGZus{}0\PYGZus{}1}\PYG{o}{*}\PYG{o}{*}\PYG{l+m+mi}{2}\PYG{p}{)}
            \PYG{n}{isolines\PYGZus{}y\PYGZus{}0\PYGZus{}2} \PYG{o}{=} \PYG{n}{np}\PYG{o}{.}\PYG{n}{sqrt}\PYG{p}{(}\PYG{p}{(}\PYG{n}{L\PYGZus{}ref}\PYG{o}{*}\PYG{n}{np}\PYG{o}{.}\PYG{n}{exp}\PYG{p}{(}\PYG{l+m+mi}{0}\PYG{o}{/}\PYG{n}{np}\PYG{o}{.}\PYG{n}{exp}\PYG{p}{(}\PYG{l+m+mi}{1}\PYG{p}{)}\PYG{o}{+}\PYG{n}{isolines\PYGZus{}x\PYGZus{}0\PYGZus{}2}\PYG{o}{/}\PYG{n}{L\PYGZus{}ref}\PYG{o}{/}\PYG{n}{np}\PYG{o}{.}\PYG{n}{exp}\PYG{p}{(}\PYG{l+m+mi}{1}\PYG{p}{)}\PYG{p}{)}\PYG{p}{)}\PYG{o}{*}\PYG{o}{*}\PYG{l+m+mi}{2}\PYG{o}{\PYGZhy{}}\PYG{n}{isolines\PYGZus{}x\PYGZus{}0\PYGZus{}2}\PYG{o}{*}\PYG{o}{*}\PYG{l+m+mi}{2}\PYG{p}{)}
            \PYG{n}{isolines\PYGZus{}y\PYGZus{}2\PYGZus{}5} \PYG{o}{=} \PYG{n}{np}\PYG{o}{.}\PYG{n}{sqrt}\PYG{p}{(}\PYG{p}{(}\PYG{n}{L\PYGZus{}ref}\PYG{o}{*}\PYG{n}{np}\PYG{o}{.}\PYG{n}{exp}\PYG{p}{(}\PYG{l+m+mf}{2.5}\PYG{o}{/}\PYG{n}{np}\PYG{o}{.}\PYG{n}{exp}\PYG{p}{(}\PYG{l+m+mi}{1}\PYG{p}{)}\PYG{o}{+}\PYG{n}{isolines\PYGZus{}x\PYGZus{}2\PYGZus{}5}\PYG{o}{/}\PYG{n}{L\PYGZus{}ref}\PYG{o}{/}\PYG{n}{np}\PYG{o}{.}\PYG{n}{exp}\PYG{p}{(}\PYG{l+m+mi}{1}\PYG{p}{)}\PYG{p}{)}\PYG{p}{)}\PYG{o}{*}\PYG{o}{*}\PYG{l+m+mi}{2}\PYG{o}{\PYGZhy{}}\PYG{n}{isolines\PYGZus{}x\PYGZus{}2\PYGZus{}5}\PYG{o}{*}\PYG{o}{*}\PYG{l+m+mi}{2}\PYG{p}{)}
            \PYG{n}{isolines\PYGZus{}y\PYGZus{}5} \PYG{o}{=} \PYG{n}{np}\PYG{o}{.}\PYG{n}{sqrt}\PYG{p}{(}\PYG{p}{(}\PYG{n}{L\PYGZus{}ref}\PYG{o}{*}\PYG{n}{np}\PYG{o}{.}\PYG{n}{exp}\PYG{p}{(}\PYG{l+m+mi}{5}\PYG{o}{/}\PYG{n}{np}\PYG{o}{.}\PYG{n}{exp}\PYG{p}{(}\PYG{l+m+mi}{1}\PYG{p}{)}\PYG{o}{+}\PYG{n}{isolines\PYGZus{}x\PYGZus{}5}\PYG{o}{/}\PYG{n}{L\PYGZus{}ref}\PYG{o}{/}\PYG{n}{np}\PYG{o}{.}\PYG{n}{exp}\PYG{p}{(}\PYG{l+m+mi}{1}\PYG{p}{)}\PYG{p}{)}\PYG{p}{)}\PYG{o}{*}\PYG{o}{*}\PYG{l+m+mi}{2}\PYG{o}{\PYGZhy{}}\PYG{n}{isolines\PYGZus{}x\PYGZus{}5}\PYG{o}{*}\PYG{o}{*}\PYG{l+m+mi}{2}\PYG{p}{)}
            \PYG{n}{isolines\PYGZus{}y\PYGZus{}7\PYGZus{}5} \PYG{o}{=} \PYG{n}{np}\PYG{o}{.}\PYG{n}{sqrt}\PYG{p}{(}\PYG{p}{(}\PYG{n}{L\PYGZus{}ref}\PYG{o}{*}\PYG{n}{np}\PYG{o}{.}\PYG{n}{exp}\PYG{p}{(}\PYG{l+m+mf}{7.5}\PYG{o}{/}\PYG{n}{np}\PYG{o}{.}\PYG{n}{exp}\PYG{p}{(}\PYG{l+m+mi}{1}\PYG{p}{)}\PYG{o}{+}\PYG{n}{isolines\PYGZus{}x\PYGZus{}7\PYGZus{}5}\PYG{o}{/}\PYG{n}{L\PYGZus{}ref}\PYG{o}{/}\PYG{n}{np}\PYG{o}{.}\PYG{n}{exp}\PYG{p}{(}\PYG{l+m+mi}{1}\PYG{p}{)}\PYG{p}{)}\PYG{p}{)}\PYG{o}{*}\PYG{o}{*}\PYG{l+m+mi}{2}\PYG{o}{\PYGZhy{}}\PYG{n}{isolines\PYGZus{}x\PYGZus{}7\PYGZus{}5}\PYG{o}{*}\PYG{o}{*}\PYG{l+m+mi}{2}\PYG{p}{)}
            \PYG{n}{isolines\PYGZus{}y\PYGZus{}10} \PYG{o}{=} \PYG{n}{np}\PYG{o}{.}\PYG{n}{sqrt}\PYG{p}{(}\PYG{p}{(}\PYG{n}{L\PYGZus{}ref}\PYG{o}{*}\PYG{n}{np}\PYG{o}{.}\PYG{n}{exp}\PYG{p}{(}\PYG{l+m+mi}{10}\PYG{o}{/}\PYG{n}{np}\PYG{o}{.}\PYG{n}{exp}\PYG{p}{(}\PYG{l+m+mi}{1}\PYG{p}{)}\PYG{o}{+}\PYG{n}{isolines\PYGZus{}x\PYGZus{}10}\PYG{o}{/}\PYG{n}{L\PYGZus{}ref}\PYG{o}{/}\PYG{n}{np}\PYG{o}{.}\PYG{n}{exp}\PYG{p}{(}\PYG{l+m+mi}{1}\PYG{p}{)}\PYG{p}{)}\PYG{p}{)}\PYG{o}{*}\PYG{o}{*}\PYG{l+m+mi}{2}\PYG{o}{\PYGZhy{}}\PYG{n}{isolines\PYGZus{}x\PYGZus{}10}\PYG{o}{*}\PYG{o}{*}\PYG{l+m+mi}{2}\PYG{p}{)}
            \PYG{n}{isolines\PYGZus{}y\PYGZus{}12\PYGZus{}5} \PYG{o}{=} \PYG{n}{np}\PYG{o}{.}\PYG{n}{sqrt}\PYG{p}{(}\PYG{p}{(}\PYG{n}{L\PYGZus{}ref}\PYG{o}{*}\PYG{n}{np}\PYG{o}{.}\PYG{n}{exp}\PYG{p}{(}\PYG{l+m+mf}{12.5}\PYG{o}{/}\PYG{n}{np}\PYG{o}{.}\PYG{n}{exp}\PYG{p}{(}\PYG{l+m+mi}{1}\PYG{p}{)}\PYG{o}{+}\PYG{n}{isolines\PYGZus{}x\PYGZus{}12\PYGZus{}5}\PYG{o}{/}\PYG{n}{L\PYGZus{}ref}\PYG{o}{/}\PYG{n}{np}\PYG{o}{.}\PYG{n}{exp}\PYG{p}{(}\PYG{l+m+mi}{1}\PYG{p}{)}\PYG{p}{)}\PYG{p}{)}\PYG{o}{*}\PYG{o}{*}\PYG{l+m+mi}{2}\PYG{o}{\PYGZhy{}}\PYG{n}{isolines\PYGZus{}x\PYGZus{}12\PYGZus{}5}\PYG{o}{*}\PYG{o}{*}\PYG{l+m+mi}{2}\PYG{p}{)}
            \PYG{n}{isolines\PYGZus{}y\PYGZus{}15} \PYG{o}{=} \PYG{n}{np}\PYG{o}{.}\PYG{n}{sqrt}\PYG{p}{(}\PYG{p}{(}\PYG{n}{L\PYGZus{}ref}\PYG{o}{*}\PYG{n}{np}\PYG{o}{.}\PYG{n}{exp}\PYG{p}{(}\PYG{l+m+mi}{15}\PYG{o}{/}\PYG{n}{np}\PYG{o}{.}\PYG{n}{exp}\PYG{p}{(}\PYG{l+m+mi}{1}\PYG{p}{)}\PYG{o}{+}\PYG{n}{isolines\PYGZus{}x\PYGZus{}15}\PYG{o}{/}\PYG{n}{L\PYGZus{}ref}\PYG{o}{/}\PYG{n}{np}\PYG{o}{.}\PYG{n}{exp}\PYG{p}{(}\PYG{l+m+mi}{1}\PYG{p}{)}\PYG{p}{)}\PYG{p}{)}\PYG{o}{*}\PYG{o}{*}\PYG{l+m+mi}{2}\PYG{o}{\PYGZhy{}}\PYG{n}{isolines\PYGZus{}x\PYGZus{}15}\PYG{o}{*}\PYG{o}{*}\PYG{l+m+mi}{2}\PYG{p}{)}
        
        \PYG{n}{isolines\PYGZus{}x\PYGZus{}plot\PYGZus{}n5\PYGZus{}1}\PYG{o}{.}\PYG{n}{append}\PYG{p}{(}\PYG{n}{isolines\PYGZus{}x\PYGZus{}n5\PYGZus{}1}\PYG{p}{)}
        \PYG{n}{isolines\PYGZus{}y\PYGZus{}plot\PYGZus{}n5\PYGZus{}1}\PYG{o}{.}\PYG{n}{append}\PYG{p}{(}\PYG{n}{isolines\PYGZus{}y\PYGZus{}n5\PYGZus{}1}\PYG{p}{)}
        \PYG{n}{isolines\PYGZus{}x\PYGZus{}plot\PYGZus{}n5\PYGZus{}2}\PYG{o}{.}\PYG{n}{append}\PYG{p}{(}\PYG{n}{isolines\PYGZus{}x\PYGZus{}n5\PYGZus{}2}\PYG{p}{)}
        \PYG{n}{isolines\PYGZus{}y\PYGZus{}plot\PYGZus{}n5\PYGZus{}2}\PYG{o}{.}\PYG{n}{append}\PYG{p}{(}\PYG{n}{isolines\PYGZus{}y\PYGZus{}n5\PYGZus{}2}\PYG{p}{)}
        \PYG{n}{isolines\PYGZus{}x\PYGZus{}plot\PYGZus{}n2\PYGZus{}5\PYGZus{}1}\PYG{o}{.}\PYG{n}{append}\PYG{p}{(}\PYG{n}{isolines\PYGZus{}x\PYGZus{}n2\PYGZus{}5\PYGZus{}1}\PYG{p}{)}
        \PYG{n}{isolines\PYGZus{}y\PYGZus{}plot\PYGZus{}n2\PYGZus{}5\PYGZus{}1}\PYG{o}{.}\PYG{n}{append}\PYG{p}{(}\PYG{n}{isolines\PYGZus{}y\PYGZus{}n2\PYGZus{}5\PYGZus{}1}\PYG{p}{)}
        \PYG{n}{isolines\PYGZus{}x\PYGZus{}plot\PYGZus{}n2\PYGZus{}5\PYGZus{}2}\PYG{o}{.}\PYG{n}{append}\PYG{p}{(}\PYG{n}{isolines\PYGZus{}x\PYGZus{}n2\PYGZus{}5\PYGZus{}2}\PYG{p}{)}
        \PYG{n}{isolines\PYGZus{}y\PYGZus{}plot\PYGZus{}n2\PYGZus{}5\PYGZus{}2}\PYG{o}{.}\PYG{n}{append}\PYG{p}{(}\PYG{n}{isolines\PYGZus{}y\PYGZus{}n2\PYGZus{}5\PYGZus{}2}\PYG{p}{)}
        \PYG{n}{isolines\PYGZus{}x\PYGZus{}plot\PYGZus{}0\PYGZus{}1}\PYG{o}{.}\PYG{n}{append}\PYG{p}{(}\PYG{n}{isolines\PYGZus{}x\PYGZus{}0\PYGZus{}1}\PYG{p}{)}
        \PYG{n}{isolines\PYGZus{}y\PYGZus{}plot\PYGZus{}0\PYGZus{}1}\PYG{o}{.}\PYG{n}{append}\PYG{p}{(}\PYG{n}{isolines\PYGZus{}y\PYGZus{}0\PYGZus{}1}\PYG{p}{)}
        \PYG{n}{isolines\PYGZus{}x\PYGZus{}plot\PYGZus{}0\PYGZus{}2}\PYG{o}{.}\PYG{n}{append}\PYG{p}{(}\PYG{n}{isolines\PYGZus{}x\PYGZus{}0\PYGZus{}2}\PYG{p}{)}
        \PYG{n}{isolines\PYGZus{}y\PYGZus{}plot\PYGZus{}0\PYGZus{}2}\PYG{o}{.}\PYG{n}{append}\PYG{p}{(}\PYG{n}{isolines\PYGZus{}y\PYGZus{}0\PYGZus{}2}\PYG{p}{)}
        \PYG{n}{isolines\PYGZus{}x\PYGZus{}plot\PYGZus{}2\PYGZus{}5}\PYG{o}{.}\PYG{n}{append}\PYG{p}{(}\PYG{n}{isolines\PYGZus{}x\PYGZus{}2\PYGZus{}5}\PYG{p}{)}
        \PYG{n}{isolines\PYGZus{}y\PYGZus{}plot\PYGZus{}2\PYGZus{}5}\PYG{o}{.}\PYG{n}{append}\PYG{p}{(}\PYG{n}{isolines\PYGZus{}y\PYGZus{}2\PYGZus{}5}\PYG{p}{)}
        \PYG{n}{isolines\PYGZus{}x\PYGZus{}plot\PYGZus{}5}\PYG{o}{.}\PYG{n}{append}\PYG{p}{(}\PYG{n}{isolines\PYGZus{}x\PYGZus{}5}\PYG{p}{)}
        \PYG{n}{isolines\PYGZus{}y\PYGZus{}plot\PYGZus{}5}\PYG{o}{.}\PYG{n}{append}\PYG{p}{(}\PYG{n}{isolines\PYGZus{}y\PYGZus{}5}\PYG{p}{)}
        \PYG{n}{isolines\PYGZus{}x\PYGZus{}plot\PYGZus{}7\PYGZus{}5}\PYG{o}{.}\PYG{n}{append}\PYG{p}{(}\PYG{n}{isolines\PYGZus{}x\PYGZus{}7\PYGZus{}5}\PYG{p}{)}
        \PYG{n}{isolines\PYGZus{}y\PYGZus{}plot\PYGZus{}7\PYGZus{}5}\PYG{o}{.}\PYG{n}{append}\PYG{p}{(}\PYG{n}{isolines\PYGZus{}y\PYGZus{}7\PYGZus{}5}\PYG{p}{)}
        \PYG{n}{isolines\PYGZus{}x\PYGZus{}plot\PYGZus{}10}\PYG{o}{.}\PYG{n}{append}\PYG{p}{(}\PYG{n}{isolines\PYGZus{}x\PYGZus{}10}\PYG{p}{)}
        \PYG{n}{isolines\PYGZus{}y\PYGZus{}plot\PYGZus{}10}\PYG{o}{.}\PYG{n}{append}\PYG{p}{(}\PYG{n}{isolines\PYGZus{}y\PYGZus{}10}\PYG{p}{)}
        \PYG{n}{isolines\PYGZus{}x\PYGZus{}plot\PYGZus{}12\PYGZus{}5}\PYG{o}{.}\PYG{n}{append}\PYG{p}{(}\PYG{n}{isolines\PYGZus{}x\PYGZus{}12\PYGZus{}5}\PYG{p}{)}
        \PYG{n}{isolines\PYGZus{}y\PYGZus{}plot\PYGZus{}12\PYGZus{}5}\PYG{o}{.}\PYG{n}{append}\PYG{p}{(}\PYG{n}{isolines\PYGZus{}y\PYGZus{}12\PYGZus{}5}\PYG{p}{)}
        \PYG{n}{isolines\PYGZus{}x\PYGZus{}plot\PYGZus{}15}\PYG{o}{.}\PYG{n}{append}\PYG{p}{(}\PYG{n}{isolines\PYGZus{}x\PYGZus{}15}\PYG{p}{)}
        \PYG{n}{isolines\PYGZus{}y\PYGZus{}plot\PYGZus{}15}\PYG{o}{.}\PYG{n}{append}\PYG{p}{(}\PYG{n}{isolines\PYGZus{}y\PYGZus{}15}\PYG{p}{)}
    
    \PYG{c+c1}{\PYGZsh{}streamlines; synthax: streamlines\PYGZus{}[X or Y]\PYGZus{}plot\PYGZus{}[psi]}
    \PYG{n}{streamlines\PYGZus{}x\PYGZus{}plot\PYGZus{}0} \PYG{o}{=} \PYG{p}{[}\PYG{l+m+mi}{0}\PYG{p}{,} \PYG{p}{(}\PYG{n}{L\PYGZus{}ref}\PYG{o}{*}\PYG{o}{\PYGZhy{}}\PYG{l+m+mi}{10}\PYG{p}{)}\PYG{p}{]}
    \PYG{n}{streamlines\PYGZus{}y\PYGZus{}plot\PYGZus{}0} \PYG{o}{=} \PYG{p}{[}\PYG{l+m+mi}{0}\PYG{p}{,} \PYG{l+m+mi}{0}\PYG{p}{]}
    \PYG{n}{streamlines\PYGZus{}x\PYGZus{}plot\PYGZus{}0\PYGZus{}2} \PYG{o}{=} \PYG{p}{[}\PYG{p}{]}
    \PYG{n}{streamlines\PYGZus{}y\PYGZus{}plot\PYGZus{}0\PYGZus{}2} \PYG{o}{=} \PYG{p}{[}\PYG{p}{]}
    \PYG{n}{streamlines\PYGZus{}x\PYGZus{}plot\PYGZus{}0\PYGZus{}4} \PYG{o}{=} \PYG{p}{[}\PYG{p}{]}
    \PYG{n}{streamlines\PYGZus{}y\PYGZus{}plot\PYGZus{}0\PYGZus{}4} \PYG{o}{=} \PYG{p}{[}\PYG{p}{]}
    \PYG{n}{streamlines\PYGZus{}x\PYGZus{}plot\PYGZus{}0\PYGZus{}6} \PYG{o}{=} \PYG{p}{[}\PYG{p}{]}
    \PYG{n}{streamlines\PYGZus{}y\PYGZus{}plot\PYGZus{}0\PYGZus{}6} \PYG{o}{=} \PYG{p}{[}\PYG{p}{]}
    \PYG{n}{streamlines\PYGZus{}x\PYGZus{}plot\PYGZus{}0\PYGZus{}8} \PYG{o}{=} \PYG{p}{[}\PYG{p}{]}
    \PYG{n}{streamlines\PYGZus{}y\PYGZus{}plot\PYGZus{}0\PYGZus{}8} \PYG{o}{=} \PYG{p}{[}\PYG{p}{]}
    \PYG{n}{streamlines\PYGZus{}x\PYGZus{}plot\PYGZus{}1} \PYG{o}{=} \PYG{p}{[}\PYG{p}{]}
    \PYG{n}{streamlines\PYGZus{}y\PYGZus{}plot\PYGZus{}1} \PYG{o}{=} \PYG{p}{[}\PYG{p}{]}
    \PYG{n}{streamlines\PYGZus{}x\PYGZus{}plot\PYGZus{}1\PYGZus{}2} \PYG{o}{=} \PYG{p}{[}\PYG{p}{]}
    \PYG{n}{streamlines\PYGZus{}y\PYGZus{}plot\PYGZus{}1\PYGZus{}2} \PYG{o}{=} \PYG{p}{[}\PYG{p}{]}
    \PYG{n}{streamlines\PYGZus{}x\PYGZus{}plot\PYGZus{}1\PYGZus{}4} \PYG{o}{=} \PYG{p}{[}\PYG{p}{]}
    \PYG{n}{streamlines\PYGZus{}y\PYGZus{}plot\PYGZus{}1\PYGZus{}4} \PYG{o}{=} \PYG{p}{[}\PYG{p}{]}
    \PYG{n}{streamlines\PYGZus{}x\PYGZus{}plot\PYGZus{}1\PYGZus{}6} \PYG{o}{=} \PYG{p}{[}\PYG{p}{]}
    \PYG{n}{streamlines\PYGZus{}y\PYGZus{}plot\PYGZus{}1\PYGZus{}6} \PYG{o}{=} \PYG{p}{[}\PYG{p}{]}
    
    \PYG{k}{for} \PYG{n}{x} \PYG{o+ow}{in} \PYG{n+nb}{range}\PYG{p}{(}\PYG{l+m+mi}{0}\PYG{p}{,}\PYG{l+m+mi}{100}\PYG{p}{)}\PYG{p}{:}
        \PYG{n}{streamlines\PYGZus{}y\PYGZus{}0\PYGZus{}2} \PYG{o}{=} \PYG{n}{L\PYGZus{}ref}\PYG{o}{*}\PYG{p}{(}\PYG{p}{(}\PYG{n}{x}\PYG{o}{*}\PYG{l+m+mi}{0}\PYG{o}{+}\PYG{p}{(}\PYG{l+m+mi}{100}\PYG{o}{\PYGZhy{}}\PYG{n}{x}\PYG{p}{)}\PYG{o}{*}\PYG{l+m+mf}{1.34462005667342}\PYG{p}{)}\PYG{o}{/}\PYG{l+m+mi}{100}\PYG{p}{)}
        \PYG{n}{streamlines\PYGZus{}y\PYGZus{}0\PYGZus{}4} \PYG{o}{=} \PYG{n}{L\PYGZus{}ref}\PYG{o}{*}\PYG{p}{(}\PYG{p}{(}\PYG{n}{x}\PYG{o}{*}\PYG{l+m+mi}{0}\PYG{o}{+}\PYG{p}{(}\PYG{l+m+mi}{100}\PYG{o}{\PYGZhy{}}\PYG{n}{x}\PYG{p}{)}\PYG{o}{*}\PYG{l+m+mf}{2.6992421745751}\PYG{p}{)}\PYG{o}{/}\PYG{l+m+mi}{100}\PYG{p}{)}
        \PYG{n}{streamlines\PYGZus{}y\PYGZus{}0\PYGZus{}6} \PYG{o}{=} \PYG{n}{L\PYGZus{}ref}\PYG{o}{*}\PYG{p}{(}\PYG{p}{(}\PYG{n}{x}\PYG{o}{*}\PYG{l+m+mi}{0}\PYG{o}{+}\PYG{p}{(}\PYG{l+m+mi}{100}\PYG{o}{\PYGZhy{}}\PYG{n}{x}\PYG{p}{)}\PYG{o}{*}\PYG{l+m+mf}{4.07255559164565}\PYG{p}{)}\PYG{o}{/}\PYG{l+m+mi}{100}\PYG{p}{)}
        \PYG{n}{streamlines\PYGZus{}y\PYGZus{}0\PYGZus{}8} \PYG{o}{=} \PYG{n}{L\PYGZus{}ref}\PYG{o}{*}\PYG{p}{(}\PYG{p}{(}\PYG{n}{x}\PYG{o}{*}\PYG{l+m+mi}{0}\PYG{o}{+}\PYG{p}{(}\PYG{l+m+mi}{100}\PYG{o}{\PYGZhy{}}\PYG{n}{x}\PYG{p}{)}\PYG{o}{*}\PYG{l+m+mf}{5.47097889806004}\PYG{p}{)}\PYG{o}{/}\PYG{l+m+mi}{100}\PYG{p}{)}
        \PYG{n}{streamlines\PYGZus{}y\PYGZus{}1} \PYG{o}{=} \PYG{n}{L\PYGZus{}ref}\PYG{o}{*}\PYG{p}{(}\PYG{p}{(}\PYG{n}{x}\PYG{o}{*}\PYG{l+m+mi}{0}\PYG{o}{+}\PYG{p}{(}\PYG{l+m+mi}{100}\PYG{o}{\PYGZhy{}}\PYG{n}{x}\PYG{p}{)}\PYG{o}{*}\PYG{l+m+mf}{6.89826117541355}\PYG{p}{)}\PYG{o}{/}\PYG{l+m+mi}{100}\PYG{p}{)}
        \PYG{n}{streamlines\PYGZus{}y\PYGZus{}1\PYGZus{}2} \PYG{o}{=} \PYG{n}{L\PYGZus{}ref}\PYG{o}{*}\PYG{p}{(}\PYG{p}{(}\PYG{n}{x}\PYG{o}{*}\PYG{l+m+mf}{2.13413758353342}\PYG{o}{+}\PYG{p}{(}\PYG{l+m+mi}{100}\PYG{o}{\PYGZhy{}}\PYG{n}{x}\PYG{p}{)}\PYG{o}{*}\PYG{l+m+mf}{8.35561532789609}\PYG{p}{)}\PYG{o}{/}\PYG{l+m+mi}{100}\PYG{p}{)}
        \PYG{n}{streamlines\PYGZus{}y\PYGZus{}1\PYGZus{}4} \PYG{o}{=} \PYG{n}{L\PYGZus{}ref}\PYG{o}{*}\PYG{p}{(}\PYG{p}{(}\PYG{n}{x}\PYG{o}{*}\PYG{l+m+mf}{4.24381382643103}\PYG{o}{+}\PYG{p}{(}\PYG{l+m+mi}{100}\PYG{o}{\PYGZhy{}}\PYG{n}{x}\PYG{p}{)}\PYG{o}{*}\PYG{l+m+mf}{9.8422946888304}\PYG{p}{)}\PYG{o}{/}\PYG{l+m+mi}{100}\PYG{p}{)}
        \PYG{n}{streamlines\PYGZus{}y\PYGZus{}1\PYGZus{}6} \PYG{o}{=} \PYG{n}{L\PYGZus{}ref}\PYG{o}{*}\PYG{p}{(}\PYG{p}{(}\PYG{n}{x}\PYG{o}{*}\PYG{l+m+mf}{6.31283612436048}\PYG{o}{+}\PYG{p}{(}\PYG{l+m+mi}{100}\PYG{o}{\PYGZhy{}}\PYG{n}{x}\PYG{p}{)}\PYG{o}{*}\PYG{l+m+mf}{11.3562442221618}\PYG{p}{)}\PYG{o}{/}\PYG{l+m+mi}{100}\PYG{p}{)}

        \PYG{n}{streamlines\PYGZus{}x\PYGZus{}0\PYGZus{}2} \PYG{o}{=} \PYG{n}{streamlines\PYGZus{}y\PYGZus{}0\PYGZus{}2}\PYG{o}{/}\PYG{n}{np}\PYG{o}{.}\PYG{n}{tan}\PYG{p}{(}\PYG{n}{streamlines\PYGZus{}y\PYGZus{}0\PYGZus{}2}\PYG{o}{/}\PYG{n}{L\PYGZus{}ref}\PYG{o}{/}\PYG{n}{np}\PYG{o}{.}\PYG{n}{exp}\PYG{p}{(}\PYG{l+m+mi}{1}\PYG{p}{)}\PYG{o}{\PYGZhy{}}\PYG{n}{np}\PYG{o}{.}\PYG{n}{pi}\PYG{o}{*}\PYG{l+m+mf}{0.2}\PYG{p}{)}
        \PYG{n}{streamlines\PYGZus{}x\PYGZus{}0\PYGZus{}4} \PYG{o}{=} \PYG{n}{streamlines\PYGZus{}y\PYGZus{}0\PYGZus{}4}\PYG{o}{/}\PYG{n}{np}\PYG{o}{.}\PYG{n}{tan}\PYG{p}{(}\PYG{n}{streamlines\PYGZus{}y\PYGZus{}0\PYGZus{}4}\PYG{o}{/}\PYG{n}{L\PYGZus{}ref}\PYG{o}{/}\PYG{n}{np}\PYG{o}{.}\PYG{n}{exp}\PYG{p}{(}\PYG{l+m+mi}{1}\PYG{p}{)}\PYG{o}{\PYGZhy{}}\PYG{n}{np}\PYG{o}{.}\PYG{n}{pi}\PYG{o}{*}\PYG{l+m+mf}{0.4}\PYG{p}{)}
        \PYG{n}{streamlines\PYGZus{}x\PYGZus{}0\PYGZus{}6} \PYG{o}{=} \PYG{n}{streamlines\PYGZus{}y\PYGZus{}0\PYGZus{}6}\PYG{o}{/}\PYG{n}{np}\PYG{o}{.}\PYG{n}{tan}\PYG{p}{(}\PYG{n}{streamlines\PYGZus{}y\PYGZus{}0\PYGZus{}6}\PYG{o}{/}\PYG{n}{L\PYGZus{}ref}\PYG{o}{/}\PYG{n}{np}\PYG{o}{.}\PYG{n}{exp}\PYG{p}{(}\PYG{l+m+mi}{1}\PYG{p}{)}\PYG{o}{\PYGZhy{}}\PYG{n}{np}\PYG{o}{.}\PYG{n}{pi}\PYG{o}{*}\PYG{l+m+mf}{0.6}\PYG{p}{)}
        \PYG{n}{streamlines\PYGZus{}x\PYGZus{}0\PYGZus{}8} \PYG{o}{=} \PYG{n}{streamlines\PYGZus{}y\PYGZus{}0\PYGZus{}8}\PYG{o}{/}\PYG{n}{np}\PYG{o}{.}\PYG{n}{tan}\PYG{p}{(}\PYG{n}{streamlines\PYGZus{}y\PYGZus{}0\PYGZus{}8}\PYG{o}{/}\PYG{n}{L\PYGZus{}ref}\PYG{o}{/}\PYG{n}{np}\PYG{o}{.}\PYG{n}{exp}\PYG{p}{(}\PYG{l+m+mi}{1}\PYG{p}{)}\PYG{o}{\PYGZhy{}}\PYG{n}{np}\PYG{o}{.}\PYG{n}{pi}\PYG{o}{*}\PYG{l+m+mf}{0.8}\PYG{p}{)}
        \PYG{n}{streamlines\PYGZus{}x\PYGZus{}1} \PYG{o}{=} \PYG{n}{streamlines\PYGZus{}y\PYGZus{}1}\PYG{o}{/}\PYG{n}{np}\PYG{o}{.}\PYG{n}{tan}\PYG{p}{(}\PYG{n}{streamlines\PYGZus{}y\PYGZus{}1}\PYG{o}{/}\PYG{n}{L\PYGZus{}ref}\PYG{o}{/}\PYG{n}{np}\PYG{o}{.}\PYG{n}{exp}\PYG{p}{(}\PYG{l+m+mi}{1}\PYG{p}{)}\PYG{o}{\PYGZhy{}}\PYG{n}{np}\PYG{o}{.}\PYG{n}{pi}\PYG{o}{*}\PYG{l+m+mi}{1}\PYG{p}{)}
        \PYG{n}{streamlines\PYGZus{}x\PYGZus{}1\PYGZus{}2} \PYG{o}{=} \PYG{n}{streamlines\PYGZus{}y\PYGZus{}1\PYGZus{}2}\PYG{o}{/}\PYG{n}{np}\PYG{o}{.}\PYG{n}{tan}\PYG{p}{(}\PYG{n}{streamlines\PYGZus{}y\PYGZus{}1\PYGZus{}2}\PYG{o}{/}\PYG{n}{L\PYGZus{}ref}\PYG{o}{/}\PYG{n}{np}\PYG{o}{.}\PYG{n}{exp}\PYG{p}{(}\PYG{l+m+mi}{1}\PYG{p}{)}\PYG{o}{\PYGZhy{}}\PYG{n}{np}\PYG{o}{.}\PYG{n}{pi}\PYG{o}{*}\PYG{l+m+mf}{1.2}\PYG{p}{)}
        \PYG{n}{streamlines\PYGZus{}x\PYGZus{}1\PYGZus{}4} \PYG{o}{=} \PYG{n}{streamlines\PYGZus{}y\PYGZus{}1\PYGZus{}4}\PYG{o}{/}\PYG{n}{np}\PYG{o}{.}\PYG{n}{tan}\PYG{p}{(}\PYG{n}{streamlines\PYGZus{}y\PYGZus{}1\PYGZus{}4}\PYG{o}{/}\PYG{n}{L\PYGZus{}ref}\PYG{o}{/}\PYG{n}{np}\PYG{o}{.}\PYG{n}{exp}\PYG{p}{(}\PYG{l+m+mi}{1}\PYG{p}{)}\PYG{o}{\PYGZhy{}}\PYG{n}{np}\PYG{o}{.}\PYG{n}{pi}\PYG{o}{*}\PYG{l+m+mf}{1.4}\PYG{p}{)}
        \PYG{n}{streamlines\PYGZus{}x\PYGZus{}1\PYGZus{}6} \PYG{o}{=} \PYG{n}{streamlines\PYGZus{}y\PYGZus{}1\PYGZus{}6}\PYG{o}{/}\PYG{n}{np}\PYG{o}{.}\PYG{n}{tan}\PYG{p}{(}\PYG{n}{streamlines\PYGZus{}y\PYGZus{}1\PYGZus{}6}\PYG{o}{/}\PYG{n}{L\PYGZus{}ref}\PYG{o}{/}\PYG{n}{np}\PYG{o}{.}\PYG{n}{exp}\PYG{p}{(}\PYG{l+m+mi}{1}\PYG{p}{)}\PYG{o}{\PYGZhy{}}\PYG{n}{np}\PYG{o}{.}\PYG{n}{pi}\PYG{o}{*}\PYG{l+m+mf}{1.6}\PYG{p}{)}
        
        \PYG{n}{streamlines\PYGZus{}x\PYGZus{}plot\PYGZus{}0\PYGZus{}2}\PYG{o}{.}\PYG{n}{append}\PYG{p}{(}\PYG{n}{streamlines\PYGZus{}x\PYGZus{}0\PYGZus{}2}\PYG{p}{)}
        \PYG{n}{streamlines\PYGZus{}y\PYGZus{}plot\PYGZus{}0\PYGZus{}2}\PYG{o}{.}\PYG{n}{append}\PYG{p}{(}\PYG{n}{streamlines\PYGZus{}y\PYGZus{}0\PYGZus{}2}\PYG{p}{)}
        \PYG{n}{streamlines\PYGZus{}x\PYGZus{}plot\PYGZus{}0\PYGZus{}4}\PYG{o}{.}\PYG{n}{append}\PYG{p}{(}\PYG{n}{streamlines\PYGZus{}x\PYGZus{}0\PYGZus{}4}\PYG{p}{)}
        \PYG{n}{streamlines\PYGZus{}y\PYGZus{}plot\PYGZus{}0\PYGZus{}4}\PYG{o}{.}\PYG{n}{append}\PYG{p}{(}\PYG{n}{streamlines\PYGZus{}y\PYGZus{}0\PYGZus{}4}\PYG{p}{)}
        \PYG{n}{streamlines\PYGZus{}x\PYGZus{}plot\PYGZus{}0\PYGZus{}6}\PYG{o}{.}\PYG{n}{append}\PYG{p}{(}\PYG{n}{streamlines\PYGZus{}x\PYGZus{}0\PYGZus{}6}\PYG{p}{)}
        \PYG{n}{streamlines\PYGZus{}y\PYGZus{}plot\PYGZus{}0\PYGZus{}6}\PYG{o}{.}\PYG{n}{append}\PYG{p}{(}\PYG{n}{streamlines\PYGZus{}y\PYGZus{}0\PYGZus{}6}\PYG{p}{)}
        \PYG{n}{streamlines\PYGZus{}x\PYGZus{}plot\PYGZus{}0\PYGZus{}8}\PYG{o}{.}\PYG{n}{append}\PYG{p}{(}\PYG{n}{streamlines\PYGZus{}x\PYGZus{}0\PYGZus{}8}\PYG{p}{)}
        \PYG{n}{streamlines\PYGZus{}y\PYGZus{}plot\PYGZus{}0\PYGZus{}8}\PYG{o}{.}\PYG{n}{append}\PYG{p}{(}\PYG{n}{streamlines\PYGZus{}y\PYGZus{}0\PYGZus{}8}\PYG{p}{)}
        \PYG{n}{streamlines\PYGZus{}x\PYGZus{}plot\PYGZus{}1}\PYG{o}{.}\PYG{n}{append}\PYG{p}{(}\PYG{n}{streamlines\PYGZus{}x\PYGZus{}1}\PYG{p}{)}
        \PYG{n}{streamlines\PYGZus{}y\PYGZus{}plot\PYGZus{}1}\PYG{o}{.}\PYG{n}{append}\PYG{p}{(}\PYG{n}{streamlines\PYGZus{}y\PYGZus{}1}\PYG{p}{)}
        \PYG{n}{streamlines\PYGZus{}x\PYGZus{}plot\PYGZus{}1\PYGZus{}2}\PYG{o}{.}\PYG{n}{append}\PYG{p}{(}\PYG{n}{streamlines\PYGZus{}x\PYGZus{}1\PYGZus{}2}\PYG{p}{)}
        \PYG{n}{streamlines\PYGZus{}y\PYGZus{}plot\PYGZus{}1\PYGZus{}2}\PYG{o}{.}\PYG{n}{append}\PYG{p}{(}\PYG{n}{streamlines\PYGZus{}y\PYGZus{}1\PYGZus{}2}\PYG{p}{)}
        \PYG{n}{streamlines\PYGZus{}x\PYGZus{}plot\PYGZus{}1\PYGZus{}4}\PYG{o}{.}\PYG{n}{append}\PYG{p}{(}\PYG{n}{streamlines\PYGZus{}x\PYGZus{}1\PYGZus{}4}\PYG{p}{)}
        \PYG{n}{streamlines\PYGZus{}y\PYGZus{}plot\PYGZus{}1\PYGZus{}4}\PYG{o}{.}\PYG{n}{append}\PYG{p}{(}\PYG{n}{streamlines\PYGZus{}y\PYGZus{}1\PYGZus{}4}\PYG{p}{)}
        \PYG{n}{streamlines\PYGZus{}x\PYGZus{}plot\PYGZus{}1\PYGZus{}6}\PYG{o}{.}\PYG{n}{append}\PYG{p}{(}\PYG{n}{streamlines\PYGZus{}x\PYGZus{}1\PYGZus{}6}\PYG{p}{)}
        \PYG{n}{streamlines\PYGZus{}y\PYGZus{}plot\PYGZus{}1\PYGZus{}6}\PYG{o}{.}\PYG{n}{append}\PYG{p}{(}\PYG{n}{streamlines\PYGZus{}y\PYGZus{}1\PYGZus{}6}\PYG{p}{)}

    \PYG{c+c1}{\PYGZsh{}isochrones}
    \PYG{n}{isochrones\PYGZus{}x\PYGZus{}plot\PYGZus{}t1} \PYG{o}{=} \PYG{p}{[}\PYG{p}{]}
    \PYG{n}{isochrones\PYGZus{}y\PYGZus{}plot\PYGZus{}t1} \PYG{o}{=} \PYG{p}{[}\PYG{p}{]}
    \PYG{n}{isochrones\PYGZus{}x\PYGZus{}plot\PYGZus{}t2} \PYG{o}{=} \PYG{p}{[}\PYG{p}{]}
    \PYG{n}{isochrones\PYGZus{}y\PYGZus{}plot\PYGZus{}t2} \PYG{o}{=} \PYG{p}{[}\PYG{p}{]}
    \PYG{n}{isochrones\PYGZus{}x\PYGZus{}plot\PYGZus{}t3} \PYG{o}{=} \PYG{p}{[}\PYG{p}{]}
    \PYG{n}{isochrones\PYGZus{}y\PYGZus{}plot\PYGZus{}t3} \PYG{o}{=} \PYG{p}{[}\PYG{p}{]}
    
    \PYG{c+c1}{\PYGZsh{}iterate 5 times with start value t\PYGZus{}xmin1/ t\PYGZus{}xmax1}
    
    \PYG{n}{t1\PYGZus{}xmin1}\PYG{o}{=}\PYG{o}{\PYGZhy{}}\PYG{n}{np}\PYG{o}{.}\PYG{n}{exp}\PYG{p}{(}\PYG{l+m+mi}{1}\PYG{p}{)}\PYG{o}{*}\PYG{n}{np}\PYG{o}{.}\PYG{n}{sqrt}\PYG{p}{(}\PYG{n}{np}\PYG{o}{.}\PYG{n}{exp}\PYG{p}{(}\PYG{l+m+mi}{2}\PYG{o}{*}\PYG{p}{(}\PYG{n}{t1}\PYG{o}{/}\PYG{n}{t\PYGZus{}ref}\PYG{p}{)}\PYG{p}{)}\PYG{o}{\PYGZhy{}}\PYG{l+m+mi}{1}\PYG{p}{)}
    \PYG{n}{t1\PYGZus{}xmin6}\PYG{o}{=} \PYG{n}{t1\PYGZus{}xmin1}\PYG{o}{+}\PYG{p}{(}\PYG{n}{np}\PYG{o}{.}\PYG{n}{exp}\PYG{p}{(}\PYG{l+m+mi}{1}\PYG{p}{)}\PYG{o}{\PYGZhy{}}\PYG{n}{t1\PYGZus{}xmin1}\PYG{p}{)}\PYG{o}{*}\PYG{p}{(}\PYG{l+m+mi}{1}\PYG{o}{+}\PYG{n}{np}\PYG{o}{.}\PYG{n}{exp}\PYG{p}{(}\PYG{l+m+mi}{1}\PYG{p}{)}\PYG{o}{/}\PYG{n}{t1\PYGZus{}xmin1}\PYG{o}{*}\PYG{p}{(}\PYG{n}{np}\PYG{o}{.}\PYG{n}{log}\PYG{p}{(}\PYG{l+m+mi}{1}\PYG{o}{\PYGZhy{}}\PYG{n}{t1\PYGZus{}xmin1}\PYG{o}{/}\PYG{n}{np}\PYG{o}{.}\PYG{n}{exp}\PYG{p}{(}\PYG{l+m+mi}{1}\PYG{p}{)}\PYG{p}{)}\PYG{o}{+}\PYG{p}{(}\PYG{n}{t1}\PYG{o}{/}\PYG{n}{t\PYGZus{}ref}\PYG{p}{)}\PYG{p}{)}\PYG{p}{)}
    \PYG{n}{t1\PYGZus{}xmax1}\PYG{o}{=} \PYG{n}{np}\PYG{o}{.}\PYG{n}{exp}\PYG{p}{(}\PYG{l+m+mi}{1}\PYG{p}{)}\PYG{o}{*}\PYG{n}{np}\PYG{o}{.}\PYG{n}{sqrt}\PYG{p}{(}\PYG{l+m+mi}{1}\PYG{o}{\PYGZhy{}}\PYG{n}{np}\PYG{o}{.}\PYG{n}{exp}\PYG{p}{(}\PYG{o}{\PYGZhy{}}\PYG{l+m+mi}{2}\PYG{o}{*}\PYG{p}{(}\PYG{n}{t1}\PYG{o}{/}\PYG{n}{t\PYGZus{}ref}\PYG{p}{)}\PYG{p}{)}\PYG{p}{)}
    \PYG{n}{t1\PYGZus{}xmax6}\PYG{o}{=} \PYG{n}{t1\PYGZus{}xmax1}\PYG{o}{+}\PYG{p}{(}\PYG{n}{np}\PYG{o}{.}\PYG{n}{exp}\PYG{p}{(}\PYG{l+m+mi}{1}\PYG{p}{)}\PYG{o}{\PYGZhy{}}\PYG{n}{t1\PYGZus{}xmax1}\PYG{p}{)}\PYG{o}{*}\PYG{p}{(}\PYG{l+m+mi}{1}\PYG{o}{+}\PYG{n}{np}\PYG{o}{.}\PYG{n}{exp}\PYG{p}{(}\PYG{l+m+mi}{1}\PYG{p}{)}\PYG{o}{/}\PYG{n}{t1\PYGZus{}xmax1}\PYG{o}{*}\PYG{p}{(}\PYG{n}{np}\PYG{o}{.}\PYG{n}{log}\PYG{p}{(}\PYG{l+m+mi}{1}\PYG{o}{\PYGZhy{}}\PYG{n}{t1\PYGZus{}xmax1}\PYG{o}{/}\PYG{n}{np}\PYG{o}{.}\PYG{n}{exp}\PYG{p}{(}\PYG{l+m+mi}{1}\PYG{p}{)}\PYG{p}{)}\PYG{o}{+}\PYG{p}{(}\PYG{n}{t1}\PYG{o}{/}\PYG{n}{t\PYGZus{}ref}\PYG{p}{)}\PYG{p}{)}\PYG{p}{)}
    \PYG{n}{t2\PYGZus{}xmin1}\PYG{o}{=}\PYG{o}{\PYGZhy{}}\PYG{n}{np}\PYG{o}{.}\PYG{n}{exp}\PYG{p}{(}\PYG{l+m+mi}{1}\PYG{p}{)}\PYG{o}{*}\PYG{n}{np}\PYG{o}{.}\PYG{n}{sqrt}\PYG{p}{(}\PYG{n}{np}\PYG{o}{.}\PYG{n}{exp}\PYG{p}{(}\PYG{l+m+mi}{2}\PYG{o}{*}\PYG{p}{(}\PYG{n}{t2}\PYG{o}{/}\PYG{n}{t\PYGZus{}ref}\PYG{p}{)}\PYG{p}{)}\PYG{o}{\PYGZhy{}}\PYG{l+m+mi}{1}\PYG{p}{)}
    \PYG{n}{t2\PYGZus{}xmin6}\PYG{o}{=} \PYG{n}{t2\PYGZus{}xmin1}\PYG{o}{+}\PYG{p}{(}\PYG{n}{np}\PYG{o}{.}\PYG{n}{exp}\PYG{p}{(}\PYG{l+m+mi}{1}\PYG{p}{)}\PYG{o}{\PYGZhy{}}\PYG{n}{t2\PYGZus{}xmin1}\PYG{p}{)}\PYG{o}{*}\PYG{p}{(}\PYG{l+m+mi}{1}\PYG{o}{+}\PYG{n}{np}\PYG{o}{.}\PYG{n}{exp}\PYG{p}{(}\PYG{l+m+mi}{1}\PYG{p}{)}\PYG{o}{/}\PYG{n}{t2\PYGZus{}xmin1}\PYG{o}{*}\PYG{p}{(}\PYG{n}{np}\PYG{o}{.}\PYG{n}{log}\PYG{p}{(}\PYG{l+m+mi}{1}\PYG{o}{\PYGZhy{}}\PYG{n}{t2\PYGZus{}xmin1}\PYG{o}{/}\PYG{n}{np}\PYG{o}{.}\PYG{n}{exp}\PYG{p}{(}\PYG{l+m+mi}{1}\PYG{p}{)}\PYG{p}{)}\PYG{o}{+}\PYG{p}{(}\PYG{n}{t2}\PYG{o}{/}\PYG{n}{t\PYGZus{}ref}\PYG{p}{)}\PYG{p}{)}\PYG{p}{)}
    \PYG{n}{t2\PYGZus{}xmax1}\PYG{o}{=}\PYG{n}{np}\PYG{o}{.}\PYG{n}{exp}\PYG{p}{(}\PYG{l+m+mi}{1}\PYG{p}{)}\PYG{o}{*}\PYG{n}{np}\PYG{o}{.}\PYG{n}{sqrt}\PYG{p}{(}\PYG{l+m+mi}{1}\PYG{o}{\PYGZhy{}}\PYG{n}{np}\PYG{o}{.}\PYG{n}{exp}\PYG{p}{(}\PYG{o}{\PYGZhy{}}\PYG{l+m+mi}{2}\PYG{o}{*}\PYG{p}{(}\PYG{n}{t2}\PYG{o}{/}\PYG{n}{t\PYGZus{}ref}\PYG{p}{)}\PYG{p}{)}\PYG{p}{)}
    \PYG{n}{t2\PYGZus{}xmax6}\PYG{o}{=} \PYG{n}{t2\PYGZus{}xmax1}\PYG{o}{+}\PYG{p}{(}\PYG{n}{np}\PYG{o}{.}\PYG{n}{exp}\PYG{p}{(}\PYG{l+m+mi}{1}\PYG{p}{)}\PYG{o}{\PYGZhy{}}\PYG{n}{t2\PYGZus{}xmax1}\PYG{p}{)}\PYG{o}{*}\PYG{p}{(}\PYG{l+m+mi}{1}\PYG{o}{+}\PYG{n}{np}\PYG{o}{.}\PYG{n}{exp}\PYG{p}{(}\PYG{l+m+mi}{1}\PYG{p}{)}\PYG{o}{/}\PYG{n}{t2\PYGZus{}xmax1}\PYG{o}{*}\PYG{p}{(}\PYG{n}{np}\PYG{o}{.}\PYG{n}{log}\PYG{p}{(}\PYG{l+m+mi}{1}\PYG{o}{\PYGZhy{}}\PYG{n}{t2\PYGZus{}xmax1}\PYG{o}{/}\PYG{n}{np}\PYG{o}{.}\PYG{n}{exp}\PYG{p}{(}\PYG{l+m+mi}{1}\PYG{p}{)}\PYG{p}{)}\PYG{o}{+}\PYG{p}{(}\PYG{n}{t2}\PYG{o}{/}\PYG{n}{t\PYGZus{}ref}\PYG{p}{)}\PYG{p}{)}\PYG{p}{)}
    \PYG{n}{t3\PYGZus{}xmin1}\PYG{o}{=}\PYG{o}{\PYGZhy{}}\PYG{n}{np}\PYG{o}{.}\PYG{n}{exp}\PYG{p}{(}\PYG{l+m+mi}{1}\PYG{p}{)}\PYG{o}{*}\PYG{n}{np}\PYG{o}{.}\PYG{n}{sqrt}\PYG{p}{(}\PYG{n}{np}\PYG{o}{.}\PYG{n}{exp}\PYG{p}{(}\PYG{l+m+mi}{2}\PYG{o}{*}\PYG{p}{(}\PYG{n}{t3}\PYG{o}{/}\PYG{n}{t\PYGZus{}ref}\PYG{p}{)}\PYG{p}{)}\PYG{o}{\PYGZhy{}}\PYG{l+m+mi}{1}\PYG{p}{)}
    \PYG{n}{t3\PYGZus{}xmin6}\PYG{o}{=} \PYG{n}{t3\PYGZus{}xmin1}\PYG{o}{+}\PYG{p}{(}\PYG{n}{np}\PYG{o}{.}\PYG{n}{exp}\PYG{p}{(}\PYG{l+m+mi}{1}\PYG{p}{)}\PYG{o}{\PYGZhy{}}\PYG{n}{t3\PYGZus{}xmin1}\PYG{p}{)}\PYG{o}{*}\PYG{p}{(}\PYG{l+m+mi}{1}\PYG{o}{+}\PYG{n}{np}\PYG{o}{.}\PYG{n}{exp}\PYG{p}{(}\PYG{l+m+mi}{1}\PYG{p}{)}\PYG{o}{/}\PYG{n}{t3\PYGZus{}xmin1}\PYG{o}{*}\PYG{p}{(}\PYG{n}{np}\PYG{o}{.}\PYG{n}{log}\PYG{p}{(}\PYG{l+m+mi}{1}\PYG{o}{\PYGZhy{}}\PYG{n}{t3\PYGZus{}xmin1}\PYG{o}{/}\PYG{n}{np}\PYG{o}{.}\PYG{n}{exp}\PYG{p}{(}\PYG{l+m+mi}{1}\PYG{p}{)}\PYG{p}{)}\PYG{o}{+}\PYG{p}{(}\PYG{n}{t3}\PYG{o}{/}\PYG{n}{t\PYGZus{}ref}\PYG{p}{)}\PYG{p}{)}\PYG{p}{)}
    \PYG{n}{t3\PYGZus{}xmax1}\PYG{o}{=}\PYG{n}{np}\PYG{o}{.}\PYG{n}{exp}\PYG{p}{(}\PYG{l+m+mi}{1}\PYG{p}{)}\PYG{o}{*}\PYG{n}{np}\PYG{o}{.}\PYG{n}{sqrt}\PYG{p}{(}\PYG{l+m+mi}{1}\PYG{o}{\PYGZhy{}}\PYG{n}{np}\PYG{o}{.}\PYG{n}{exp}\PYG{p}{(}\PYG{o}{\PYGZhy{}}\PYG{l+m+mi}{2}\PYG{o}{*}\PYG{p}{(}\PYG{n}{t3}\PYG{o}{/}\PYG{n}{t\PYGZus{}ref}\PYG{p}{)}\PYG{p}{)}\PYG{p}{)}
    \PYG{n}{t3\PYGZus{}xmax6}\PYG{o}{=} \PYG{n}{t3\PYGZus{}xmax1}\PYG{o}{+}\PYG{p}{(}\PYG{n}{np}\PYG{o}{.}\PYG{n}{exp}\PYG{p}{(}\PYG{l+m+mi}{1}\PYG{p}{)}\PYG{o}{\PYGZhy{}}\PYG{n}{t3\PYGZus{}xmax1}\PYG{p}{)}\PYG{o}{*}\PYG{p}{(}\PYG{l+m+mi}{1}\PYG{o}{+}\PYG{n}{np}\PYG{o}{.}\PYG{n}{exp}\PYG{p}{(}\PYG{l+m+mi}{1}\PYG{p}{)}\PYG{o}{/}\PYG{n}{t3\PYGZus{}xmax1}\PYG{o}{*}\PYG{p}{(}\PYG{n}{np}\PYG{o}{.}\PYG{n}{log}\PYG{p}{(}\PYG{l+m+mi}{1}\PYG{o}{\PYGZhy{}}\PYG{n}{t3\PYGZus{}xmax1}\PYG{o}{/}\PYG{n}{np}\PYG{o}{.}\PYG{n}{exp}\PYG{p}{(}\PYG{l+m+mi}{1}\PYG{p}{)}\PYG{p}{)}\PYG{o}{+}\PYG{p}{(}\PYG{n}{t3}\PYG{o}{/}\PYG{n}{t\PYGZus{}ref}\PYG{p}{)}\PYG{p}{)}\PYG{p}{)}
    \PYG{k}{for} \PYG{n}{i} \PYG{o+ow}{in} \PYG{n+nb}{range}\PYG{p}{(}\PYG{l+m+mi}{4}\PYG{p}{)}\PYG{p}{:}     
        \PYG{n}{t1\PYGZus{}xmin6}\PYG{o}{=} \PYG{n}{t1\PYGZus{}xmin6}\PYG{o}{+}\PYG{p}{(}\PYG{n}{np}\PYG{o}{.}\PYG{n}{exp}\PYG{p}{(}\PYG{l+m+mi}{1}\PYG{p}{)}\PYG{o}{\PYGZhy{}}\PYG{n}{t1\PYGZus{}xmin6}\PYG{p}{)}\PYG{o}{*}\PYG{p}{(}\PYG{l+m+mi}{1}\PYG{o}{+}\PYG{n}{np}\PYG{o}{.}\PYG{n}{exp}\PYG{p}{(}\PYG{l+m+mi}{1}\PYG{p}{)}\PYG{o}{/}\PYG{n}{t1\PYGZus{}xmin6}\PYG{o}{*}\PYG{p}{(}\PYG{n}{np}\PYG{o}{.}\PYG{n}{log}\PYG{p}{(}\PYG{l+m+mi}{1}\PYG{o}{\PYGZhy{}}\PYG{n}{t1\PYGZus{}xmin6}\PYG{o}{/}\PYG{n}{np}\PYG{o}{.}\PYG{n}{exp}\PYG{p}{(}\PYG{l+m+mi}{1}\PYG{p}{)}\PYG{p}{)}\PYG{o}{+}\PYG{p}{(}\PYG{n}{t1}\PYG{o}{/}\PYG{n}{t\PYGZus{}ref}\PYG{p}{)}\PYG{p}{)}\PYG{p}{)}
        \PYG{n}{t1\PYGZus{}xmax6}\PYG{o}{=} \PYG{n}{t1\PYGZus{}xmax6}\PYG{o}{+}\PYG{p}{(}\PYG{n}{np}\PYG{o}{.}\PYG{n}{exp}\PYG{p}{(}\PYG{l+m+mi}{1}\PYG{p}{)}\PYG{o}{\PYGZhy{}}\PYG{n}{t1\PYGZus{}xmax6}\PYG{p}{)}\PYG{o}{*}\PYG{p}{(}\PYG{l+m+mi}{1}\PYG{o}{+}\PYG{n}{np}\PYG{o}{.}\PYG{n}{exp}\PYG{p}{(}\PYG{l+m+mi}{1}\PYG{p}{)}\PYG{o}{/}\PYG{n}{t1\PYGZus{}xmax6}\PYG{o}{*}\PYG{p}{(}\PYG{n}{np}\PYG{o}{.}\PYG{n}{log}\PYG{p}{(}\PYG{l+m+mi}{1}\PYG{o}{\PYGZhy{}}\PYG{n}{t1\PYGZus{}xmax6}\PYG{o}{/}\PYG{n}{np}\PYG{o}{.}\PYG{n}{exp}\PYG{p}{(}\PYG{l+m+mi}{1}\PYG{p}{)}\PYG{p}{)}\PYG{o}{+}\PYG{p}{(}\PYG{n}{t1}\PYG{o}{/}\PYG{n}{t\PYGZus{}ref}\PYG{p}{)}\PYG{p}{)}\PYG{p}{)}
        \PYG{n}{t2\PYGZus{}xmin6}\PYG{o}{=} \PYG{n}{t2\PYGZus{}xmin6}\PYG{o}{+}\PYG{p}{(}\PYG{n}{np}\PYG{o}{.}\PYG{n}{exp}\PYG{p}{(}\PYG{l+m+mi}{1}\PYG{p}{)}\PYG{o}{\PYGZhy{}}\PYG{n}{t2\PYGZus{}xmin6}\PYG{p}{)}\PYG{o}{*}\PYG{p}{(}\PYG{l+m+mi}{1}\PYG{o}{+}\PYG{n}{np}\PYG{o}{.}\PYG{n}{exp}\PYG{p}{(}\PYG{l+m+mi}{1}\PYG{p}{)}\PYG{o}{/}\PYG{n}{t2\PYGZus{}xmin6}\PYG{o}{*}\PYG{p}{(}\PYG{n}{np}\PYG{o}{.}\PYG{n}{log}\PYG{p}{(}\PYG{l+m+mi}{1}\PYG{o}{\PYGZhy{}}\PYG{n}{t2\PYGZus{}xmin6}\PYG{o}{/}\PYG{n}{np}\PYG{o}{.}\PYG{n}{exp}\PYG{p}{(}\PYG{l+m+mi}{1}\PYG{p}{)}\PYG{p}{)}\PYG{o}{+}\PYG{p}{(}\PYG{n}{t2}\PYG{o}{/}\PYG{n}{t\PYGZus{}ref}\PYG{p}{)}\PYG{p}{)}\PYG{p}{)}
        \PYG{n}{t2\PYGZus{}xmax6}\PYG{o}{=} \PYG{n}{t2\PYGZus{}xmax6}\PYG{o}{+}\PYG{p}{(}\PYG{n}{np}\PYG{o}{.}\PYG{n}{exp}\PYG{p}{(}\PYG{l+m+mi}{1}\PYG{p}{)}\PYG{o}{\PYGZhy{}}\PYG{n}{t2\PYGZus{}xmax6}\PYG{p}{)}\PYG{o}{*}\PYG{p}{(}\PYG{l+m+mi}{1}\PYG{o}{+}\PYG{n}{np}\PYG{o}{.}\PYG{n}{exp}\PYG{p}{(}\PYG{l+m+mi}{1}\PYG{p}{)}\PYG{o}{/}\PYG{n}{t2\PYGZus{}xmax6}\PYG{o}{*}\PYG{p}{(}\PYG{n}{np}\PYG{o}{.}\PYG{n}{log}\PYG{p}{(}\PYG{l+m+mi}{1}\PYG{o}{\PYGZhy{}}\PYG{n}{t2\PYGZus{}xmax6}\PYG{o}{/}\PYG{n}{np}\PYG{o}{.}\PYG{n}{exp}\PYG{p}{(}\PYG{l+m+mi}{1}\PYG{p}{)}\PYG{p}{)}\PYG{o}{+}\PYG{p}{(}\PYG{n}{t2}\PYG{o}{/}\PYG{n}{t\PYGZus{}ref}\PYG{p}{)}\PYG{p}{)}\PYG{p}{)}
        \PYG{n}{t3\PYGZus{}xmin6}\PYG{o}{=} \PYG{n}{t3\PYGZus{}xmin6}\PYG{o}{+}\PYG{p}{(}\PYG{n}{np}\PYG{o}{.}\PYG{n}{exp}\PYG{p}{(}\PYG{l+m+mi}{1}\PYG{p}{)}\PYG{o}{\PYGZhy{}}\PYG{n}{t3\PYGZus{}xmin6}\PYG{p}{)}\PYG{o}{*}\PYG{p}{(}\PYG{l+m+mi}{1}\PYG{o}{+}\PYG{n}{np}\PYG{o}{.}\PYG{n}{exp}\PYG{p}{(}\PYG{l+m+mi}{1}\PYG{p}{)}\PYG{o}{/}\PYG{n}{t3\PYGZus{}xmin6}\PYG{o}{*}\PYG{p}{(}\PYG{n}{np}\PYG{o}{.}\PYG{n}{log}\PYG{p}{(}\PYG{l+m+mi}{1}\PYG{o}{\PYGZhy{}}\PYG{n}{t3\PYGZus{}xmin6}\PYG{o}{/}\PYG{n}{np}\PYG{o}{.}\PYG{n}{exp}\PYG{p}{(}\PYG{l+m+mi}{1}\PYG{p}{)}\PYG{p}{)}\PYG{o}{+}\PYG{p}{(}\PYG{n}{t3}\PYG{o}{/}\PYG{n}{t\PYGZus{}ref}\PYG{p}{)}\PYG{p}{)}\PYG{p}{)}
        \PYG{n}{t3\PYGZus{}xmax6}\PYG{o}{=} \PYG{n}{t3\PYGZus{}xmax6}\PYG{o}{+}\PYG{p}{(}\PYG{n}{np}\PYG{o}{.}\PYG{n}{exp}\PYG{p}{(}\PYG{l+m+mi}{1}\PYG{p}{)}\PYG{o}{\PYGZhy{}}\PYG{n}{t3\PYGZus{}xmax6}\PYG{p}{)}\PYG{o}{*}\PYG{p}{(}\PYG{l+m+mi}{1}\PYG{o}{+}\PYG{n}{np}\PYG{o}{.}\PYG{n}{exp}\PYG{p}{(}\PYG{l+m+mi}{1}\PYG{p}{)}\PYG{o}{/}\PYG{n}{t3\PYGZus{}xmax6}\PYG{o}{*}\PYG{p}{(}\PYG{n}{np}\PYG{o}{.}\PYG{n}{log}\PYG{p}{(}\PYG{l+m+mi}{1}\PYG{o}{\PYGZhy{}}\PYG{n}{t3\PYGZus{}xmax6}\PYG{o}{/}\PYG{n}{np}\PYG{o}{.}\PYG{n}{exp}\PYG{p}{(}\PYG{l+m+mi}{1}\PYG{p}{)}\PYG{p}{)}\PYG{o}{+}\PYG{p}{(}\PYG{n}{t3}\PYG{o}{/}\PYG{n}{t\PYGZus{}ref}\PYG{p}{)}\PYG{p}{)}\PYG{p}{)}

    \PYG{k}{for} \PYG{n}{x} \PYG{o+ow}{in} \PYG{n+nb}{range} \PYG{p}{(}\PYG{l+m+mi}{0}\PYG{p}{,}\PYG{l+m+mi}{100}\PYG{p}{)}\PYG{p}{:}
    
        \PYG{n}{isochrones\PYGZus{}x\PYGZus{}t1} \PYG{o}{=} \PYG{l+m+mf}{0.5}\PYG{o}{*}\PYG{n}{L\PYGZus{}ref}\PYG{o}{*}\PYG{p}{(}\PYG{n}{t1\PYGZus{}xmin6}\PYG{o}{+}\PYG{n}{t1\PYGZus{}xmax6}\PYG{o}{+}\PYG{p}{(}\PYG{n}{t1\PYGZus{}xmax6}\PYG{o}{\PYGZhy{}}\PYG{n}{t1\PYGZus{}xmin6}\PYG{p}{)}\PYG{o}{*}\PYG{n}{np}\PYG{o}{.}\PYG{n}{cos}\PYG{p}{(}\PYG{n}{np}\PYG{o}{.}\PYG{n}{pi}\PYG{o}{*}\PYG{p}{(}\PYG{l+m+mi}{100}\PYG{o}{\PYGZhy{}}\PYG{n}{x}\PYG{p}{)}\PYG{o}{/}\PYG{l+m+mi}{100}\PYG{p}{)}\PYG{p}{)}
        \PYG{n}{isochrones\PYGZus{}x\PYGZus{}t2} \PYG{o}{=} \PYG{l+m+mf}{0.5}\PYG{o}{*}\PYG{n}{L\PYGZus{}ref}\PYG{o}{*}\PYG{p}{(}\PYG{n}{t2\PYGZus{}xmin6}\PYG{o}{+}\PYG{n}{t2\PYGZus{}xmax6}\PYG{o}{+}\PYG{p}{(}\PYG{n}{t2\PYGZus{}xmax6}\PYG{o}{\PYGZhy{}}\PYG{n}{t2\PYGZus{}xmin6}\PYG{p}{)}\PYG{o}{*}\PYG{n}{np}\PYG{o}{.}\PYG{n}{cos}\PYG{p}{(}\PYG{n}{np}\PYG{o}{.}\PYG{n}{pi}\PYG{o}{*}\PYG{p}{(}\PYG{l+m+mi}{100}\PYG{o}{\PYGZhy{}}\PYG{n}{x}\PYG{p}{)}\PYG{o}{/}\PYG{l+m+mi}{100}\PYG{p}{)}\PYG{p}{)}
        \PYG{n}{isochrones\PYGZus{}x\PYGZus{}t3} \PYG{o}{=} \PYG{l+m+mf}{0.5}\PYG{o}{*}\PYG{n}{L\PYGZus{}ref}\PYG{o}{*}\PYG{p}{(}\PYG{n}{t3\PYGZus{}xmin6}\PYG{o}{+}\PYG{n}{t3\PYGZus{}xmax6}\PYG{o}{+}\PYG{p}{(}\PYG{n}{t3\PYGZus{}xmax6}\PYG{o}{\PYGZhy{}}\PYG{n}{t3\PYGZus{}xmin6}\PYG{p}{)}\PYG{o}{*}\PYG{n}{np}\PYG{o}{.}\PYG{n}{cos}\PYG{p}{(}\PYG{n}{np}\PYG{o}{.}\PYG{n}{pi}\PYG{o}{*}\PYG{p}{(}\PYG{l+m+mi}{100}\PYG{o}{\PYGZhy{}}\PYG{n}{x}\PYG{p}{)}\PYG{o}{/}\PYG{l+m+mi}{100}\PYG{p}{)}\PYG{p}{)}

        \PYG{k}{if} \PYG{n}{x} \PYG{o}{==} \PYG{l+m+mi}{0}\PYG{p}{:}
            \PYG{n}{isochrones\PYGZus{}y\PYGZus{}t1} \PYG{o}{=} \PYG{l+m+mi}{0}
            \PYG{n}{isochrones\PYGZus{}y\PYGZus{}t2} \PYG{o}{=} \PYG{l+m+mi}{0}
            \PYG{n}{isochrones\PYGZus{}y\PYGZus{}t3} \PYG{o}{=} \PYG{l+m+mi}{0}
        \PYG{k}{else}\PYG{p}{:}
        
            \PYG{n}{isochrones\PYGZus{}y1\PYGZus{}t1} \PYG{o}{=} \PYG{n}{L\PYGZus{}ref}\PYG{o}{*}\PYG{n}{np}\PYG{o}{.}\PYG{n}{exp}\PYG{p}{(}\PYG{l+m+mi}{1}\PYG{p}{)}\PYG{o}{*}\PYG{n}{np}\PYG{o}{.}\PYG{n}{arccos}\PYG{p}{(}\PYG{p}{(}\PYG{l+m+mf}{0.5}\PYG{o}{*}\PYG{n}{isochrones\PYGZus{}x\PYGZus{}t1}\PYG{o}{/}\PYG{n}{np}\PYG{o}{.}\PYG{n}{exp}\PYG{p}{(}\PYG{l+m+mi}{1}\PYG{p}{)}\PYG{o}{/}\PYG{n}{L\PYGZus{}ref}\PYG{o}{+}\PYG{n}{np}\PYG{o}{.}\PYG{n}{exp}\PYG{p}{(}\PYG{o}{\PYGZhy{}}\PYG{p}{(}\PYG{n}{t1}\PYG{o}{/}\PYG{n}{t\PYGZus{}ref}\PYG{p}{)}\PYG{o}{\PYGZhy{}}\PYG{n}{isochrones\PYGZus{}x\PYGZus{}t1}\PYG{o}{/}\PYG{n}{np}\PYG{o}{.}\PYG{n}{exp}\PYG{p}{(}\PYG{l+m+mi}{1}\PYG{p}{)}\PYG{o}{/}\PYG{n}{L\PYGZus{}ref}\PYG{p}{)}\PYG{p}{)}\PYG{o}{/}\PYG{p}{(}\PYG{l+m+mi}{1}\PYG{o}{\PYGZhy{}}\PYG{l+m+mf}{0.5}\PYG{o}{*}\PYG{n}{isochrones\PYGZus{}x\PYGZus{}t1}\PYG{o}{/}\PYG{n}{np}\PYG{o}{.}\PYG{n}{exp}\PYG{p}{(}\PYG{l+m+mi}{1}\PYG{p}{)}\PYG{o}{/}\PYG{n}{L\PYGZus{}ref}\PYG{p}{)}\PYG{p}{)}
            \PYG{n}{isochrones\PYGZus{}y\PYGZus{}t1} \PYG{o}{=} \PYG{n}{L\PYGZus{}ref}\PYG{o}{*}\PYG{n}{np}\PYG{o}{.}\PYG{n}{exp}\PYG{p}{(}\PYG{l+m+mi}{1}\PYG{p}{)}\PYG{o}{*}\PYG{n}{np}\PYG{o}{.}\PYG{n}{arccos}\PYG{p}{(}\PYG{p}{(}\PYG{n}{isochrones\PYGZus{}x\PYGZus{}t1}\PYG{o}{/}\PYG{n}{L\PYGZus{}ref}\PYG{o}{*}\PYG{p}{(}\PYG{n}{np}\PYG{o}{.}\PYG{n}{sin}\PYG{p}{(}\PYG{n}{isochrones\PYGZus{}y1\PYGZus{}t1}\PYG{o}{/}\PYG{n}{np}\PYG{o}{.}\PYG{n}{exp}\PYG{p}{(}\PYG{l+m+mi}{1}\PYG{p}{)}\PYG{o}{/}\PYG{n}{L\PYGZus{}ref}\PYG{p}{)}\PYG{o}{/}\PYG{p}{(}\PYG{n}{isochrones\PYGZus{}y1\PYGZus{}t1}\PYG{o}{/}\PYG{n}{L\PYGZus{}ref}\PYG{p}{)}\PYG{o}{\PYGZhy{}}\PYG{l+m+mf}{0.5}\PYG{o}{*}\PYG{n}{np}\PYG{o}{.}\PYG{n}{cos}\PYG{p}{(}\PYG{n}{isochrones\PYGZus{}y1\PYGZus{}t1}\PYG{o}{/}\PYG{n}{np}\PYG{o}{.}\PYG{n}{exp}\PYG{p}{(}\PYG{l+m+mi}{1}\PYG{p}{)}\PYG{o}{/}\PYG{n}{L\PYGZus{}ref}\PYG{p}{)}\PYG{o}{/}\PYG{n}{np}\PYG{o}{.}\PYG{n}{exp}\PYG{p}{(}\PYG{l+m+mi}{1}\PYG{p}{)}\PYG{p}{)}\PYG{o}{+}\PYG{n}{np}\PYG{o}{.}\PYG{n}{exp}\PYG{p}{(}\PYG{o}{\PYGZhy{}}\PYG{p}{(}\PYG{n}{t1}\PYG{o}{/}\PYG{n}{t\PYGZus{}ref}\PYG{p}{)}\PYG{o}{\PYGZhy{}}\PYG{n}{isochrones\PYGZus{}x\PYGZus{}t1}\PYG{o}{/}\PYG{n}{np}\PYG{o}{.}\PYG{n}{exp}\PYG{p}{(}\PYG{l+m+mi}{1}\PYG{p}{)}\PYG{o}{/}\PYG{n}{L\PYGZus{}ref}\PYG{p}{)}\PYG{p}{)}\PYG{o}{/}\PYG{p}{(}\PYG{l+m+mi}{1}\PYG{o}{\PYGZhy{}}\PYG{l+m+mf}{0.5}\PYG{o}{*}\PYG{n}{isochrones\PYGZus{}x\PYGZus{}t1}\PYG{o}{/}\PYG{n}{np}\PYG{o}{.}\PYG{n}{exp}\PYG{p}{(}\PYG{l+m+mi}{1}\PYG{p}{)}\PYG{o}{/}\PYG{n}{L\PYGZus{}ref}\PYG{p}{)}\PYG{p}{)}

            \PYG{n}{isochrones\PYGZus{}y1\PYGZus{}t2} \PYG{o}{=} \PYG{n}{L\PYGZus{}ref}\PYG{o}{*}\PYG{n}{np}\PYG{o}{.}\PYG{n}{exp}\PYG{p}{(}\PYG{l+m+mi}{1}\PYG{p}{)}\PYG{o}{*}\PYG{n}{np}\PYG{o}{.}\PYG{n}{arccos}\PYG{p}{(}\PYG{p}{(}\PYG{l+m+mf}{0.5}\PYG{o}{*}\PYG{n}{isochrones\PYGZus{}x\PYGZus{}t2}\PYG{o}{/}\PYG{n}{np}\PYG{o}{.}\PYG{n}{exp}\PYG{p}{(}\PYG{l+m+mi}{1}\PYG{p}{)}\PYG{o}{/}\PYG{n}{L\PYGZus{}ref}\PYG{o}{+}\PYG{n}{np}\PYG{o}{.}\PYG{n}{exp}\PYG{p}{(}\PYG{o}{\PYGZhy{}}\PYG{p}{(}\PYG{n}{t2}\PYG{o}{/}\PYG{n}{t\PYGZus{}ref}\PYG{p}{)}\PYG{o}{\PYGZhy{}}\PYG{n}{isochrones\PYGZus{}x\PYGZus{}t2}\PYG{o}{/}\PYG{n}{np}\PYG{o}{.}\PYG{n}{exp}\PYG{p}{(}\PYG{l+m+mi}{1}\PYG{p}{)}\PYG{o}{/}\PYG{n}{L\PYGZus{}ref}\PYG{p}{)}\PYG{p}{)}\PYG{o}{/}\PYG{p}{(}\PYG{l+m+mi}{1}\PYG{o}{\PYGZhy{}}\PYG{l+m+mf}{0.5}\PYG{o}{*}\PYG{n}{isochrones\PYGZus{}x\PYGZus{}t2}\PYG{o}{/}\PYG{n}{np}\PYG{o}{.}\PYG{n}{exp}\PYG{p}{(}\PYG{l+m+mi}{1}\PYG{p}{)}\PYG{o}{/}\PYG{n}{L\PYGZus{}ref}\PYG{p}{)}\PYG{p}{)}
            \PYG{n}{isochrones\PYGZus{}y\PYGZus{}t2} \PYG{o}{=} \PYG{n}{L\PYGZus{}ref}\PYG{o}{*}\PYG{n}{np}\PYG{o}{.}\PYG{n}{exp}\PYG{p}{(}\PYG{l+m+mi}{1}\PYG{p}{)}\PYG{o}{*}\PYG{n}{np}\PYG{o}{.}\PYG{n}{arccos}\PYG{p}{(}\PYG{p}{(}\PYG{n}{isochrones\PYGZus{}x\PYGZus{}t2}\PYG{o}{/}\PYG{n}{L\PYGZus{}ref}\PYG{o}{*}\PYG{p}{(}\PYG{n}{np}\PYG{o}{.}\PYG{n}{sin}\PYG{p}{(}\PYG{n}{isochrones\PYGZus{}y1\PYGZus{}t2}\PYG{o}{/}\PYG{n}{np}\PYG{o}{.}\PYG{n}{exp}\PYG{p}{(}\PYG{l+m+mi}{1}\PYG{p}{)}\PYG{o}{/}\PYG{n}{L\PYGZus{}ref}\PYG{p}{)}\PYG{o}{/}\PYG{p}{(}\PYG{n}{isochrones\PYGZus{}y1\PYGZus{}t2}\PYG{o}{/}\PYG{n}{L\PYGZus{}ref}\PYG{p}{)}\PYG{o}{\PYGZhy{}}\PYG{l+m+mf}{0.5}\PYG{o}{*}\PYG{n}{np}\PYG{o}{.}\PYG{n}{cos}\PYG{p}{(}\PYG{n}{isochrones\PYGZus{}y1\PYGZus{}t2}\PYG{o}{/}\PYG{n}{np}\PYG{o}{.}\PYG{n}{exp}\PYG{p}{(}\PYG{l+m+mi}{1}\PYG{p}{)}\PYG{o}{/}\PYG{n}{L\PYGZus{}ref}\PYG{p}{)}\PYG{o}{/}\PYG{n}{np}\PYG{o}{.}\PYG{n}{exp}\PYG{p}{(}\PYG{l+m+mi}{1}\PYG{p}{)}\PYG{p}{)}\PYG{o}{+}\PYG{n}{np}\PYG{o}{.}\PYG{n}{exp}\PYG{p}{(}\PYG{o}{\PYGZhy{}}\PYG{p}{(}\PYG{n}{t2}\PYG{o}{/}\PYG{n}{t\PYGZus{}ref}\PYG{p}{)}\PYG{o}{\PYGZhy{}}\PYG{n}{isochrones\PYGZus{}x\PYGZus{}t2}\PYG{o}{/}\PYG{n}{np}\PYG{o}{.}\PYG{n}{exp}\PYG{p}{(}\PYG{l+m+mi}{1}\PYG{p}{)}\PYG{o}{/}\PYG{n}{L\PYGZus{}ref}\PYG{p}{)}\PYG{p}{)}\PYG{o}{/}\PYG{p}{(}\PYG{l+m+mi}{1}\PYG{o}{\PYGZhy{}}\PYG{l+m+mf}{0.5}\PYG{o}{*}\PYG{n}{isochrones\PYGZus{}x\PYGZus{}t2}\PYG{o}{/}\PYG{n}{np}\PYG{o}{.}\PYG{n}{exp}\PYG{p}{(}\PYG{l+m+mi}{1}\PYG{p}{)}\PYG{o}{/}\PYG{n}{L\PYGZus{}ref}\PYG{p}{)}\PYG{p}{)}

            \PYG{n}{isochrones\PYGZus{}y1\PYGZus{}t3} \PYG{o}{=} \PYG{n}{L\PYGZus{}ref}\PYG{o}{*}\PYG{n}{np}\PYG{o}{.}\PYG{n}{exp}\PYG{p}{(}\PYG{l+m+mi}{1}\PYG{p}{)}\PYG{o}{*}\PYG{n}{np}\PYG{o}{.}\PYG{n}{arccos}\PYG{p}{(}\PYG{p}{(}\PYG{l+m+mf}{0.5}\PYG{o}{*}\PYG{n}{isochrones\PYGZus{}x\PYGZus{}t3}\PYG{o}{/}\PYG{n}{np}\PYG{o}{.}\PYG{n}{exp}\PYG{p}{(}\PYG{l+m+mi}{1}\PYG{p}{)}\PYG{o}{/}\PYG{n}{L\PYGZus{}ref}\PYG{o}{+}\PYG{n}{np}\PYG{o}{.}\PYG{n}{exp}\PYG{p}{(}\PYG{o}{\PYGZhy{}}\PYG{p}{(}\PYG{n}{t3}\PYG{o}{/}\PYG{n}{t\PYGZus{}ref}\PYG{p}{)}\PYG{o}{\PYGZhy{}}\PYG{n}{isochrones\PYGZus{}x\PYGZus{}t3}\PYG{o}{/}\PYG{n}{np}\PYG{o}{.}\PYG{n}{exp}\PYG{p}{(}\PYG{l+m+mi}{1}\PYG{p}{)}\PYG{o}{/}\PYG{n}{L\PYGZus{}ref}\PYG{p}{)}\PYG{p}{)}\PYG{o}{/}\PYG{p}{(}\PYG{l+m+mi}{1}\PYG{o}{\PYGZhy{}}\PYG{l+m+mf}{0.5}\PYG{o}{*}\PYG{n}{isochrones\PYGZus{}x\PYGZus{}t3}\PYG{o}{/}\PYG{n}{np}\PYG{o}{.}\PYG{n}{exp}\PYG{p}{(}\PYG{l+m+mi}{1}\PYG{p}{)}\PYG{o}{/}\PYG{n}{L\PYGZus{}ref}\PYG{p}{)}\PYG{p}{)}
            \PYG{n}{isochrones\PYGZus{}y\PYGZus{}t3} \PYG{o}{=} \PYG{n}{L\PYGZus{}ref}\PYG{o}{*}\PYG{n}{np}\PYG{o}{.}\PYG{n}{exp}\PYG{p}{(}\PYG{l+m+mi}{1}\PYG{p}{)}\PYG{o}{*}\PYG{n}{np}\PYG{o}{.}\PYG{n}{arccos}\PYG{p}{(}\PYG{p}{(}\PYG{n}{isochrones\PYGZus{}x\PYGZus{}t3}\PYG{o}{/}\PYG{n}{L\PYGZus{}ref}\PYG{o}{*}\PYG{p}{(}\PYG{n}{np}\PYG{o}{.}\PYG{n}{sin}\PYG{p}{(}\PYG{n}{isochrones\PYGZus{}y1\PYGZus{}t3}\PYG{o}{/}\PYG{n}{np}\PYG{o}{.}\PYG{n}{exp}\PYG{p}{(}\PYG{l+m+mi}{1}\PYG{p}{)}\PYG{o}{/}\PYG{n}{L\PYGZus{}ref}\PYG{p}{)}\PYG{o}{/}\PYG{p}{(}\PYG{n}{isochrones\PYGZus{}y\PYGZus{}t3}\PYG{o}{/}\PYG{n}{L\PYGZus{}ref}\PYG{p}{)}\PYG{o}{\PYGZhy{}}\PYG{l+m+mf}{0.5}\PYG{o}{*}\PYG{n}{np}\PYG{o}{.}\PYG{n}{cos}\PYG{p}{(}\PYG{n}{isochrones\PYGZus{}y1\PYGZus{}t3}\PYG{o}{/}\PYG{n}{np}\PYG{o}{.}\PYG{n}{exp}\PYG{p}{(}\PYG{l+m+mi}{1}\PYG{p}{)}\PYG{o}{/}\PYG{n}{L\PYGZus{}ref}\PYG{p}{)}\PYG{o}{/}\PYG{n}{np}\PYG{o}{.}\PYG{n}{exp}\PYG{p}{(}\PYG{l+m+mi}{1}\PYG{p}{)}\PYG{p}{)}\PYG{o}{+}\PYG{n}{np}\PYG{o}{.}\PYG{n}{exp}\PYG{p}{(}\PYG{o}{\PYGZhy{}}\PYG{p}{(}\PYG{n}{t3}\PYG{o}{/}\PYG{n}{t\PYGZus{}ref}\PYG{p}{)}\PYG{o}{\PYGZhy{}}\PYG{n}{isochrones\PYGZus{}x\PYGZus{}t3}\PYG{o}{/}\PYG{n}{np}\PYG{o}{.}\PYG{n}{exp}\PYG{p}{(}\PYG{l+m+mi}{1}\PYG{p}{)}\PYG{o}{/}\PYG{n}{L\PYGZus{}ref}\PYG{p}{)}\PYG{p}{)}\PYG{o}{/}\PYG{p}{(}\PYG{l+m+mi}{1}\PYG{o}{\PYGZhy{}}\PYG{l+m+mf}{0.5}\PYG{o}{*}\PYG{n}{isochrones\PYGZus{}x\PYGZus{}t3}\PYG{o}{/}\PYG{n}{np}\PYG{o}{.}\PYG{n}{exp}\PYG{p}{(}\PYG{l+m+mi}{1}\PYG{p}{)}\PYG{o}{/}\PYG{n}{L\PYGZus{}ref}\PYG{p}{)}\PYG{p}{)}

            \PYG{k}{for} \PYG{n}{i} \PYG{o+ow}{in} \PYG{n+nb}{range}\PYG{p}{(}\PYG{l+m+mi}{4}\PYG{p}{)}\PYG{p}{:}
                \PYG{n}{isochrones\PYGZus{}y\PYGZus{}t1} \PYG{o}{=} \PYG{n}{L\PYGZus{}ref}\PYG{o}{*}\PYG{n}{np}\PYG{o}{.}\PYG{n}{exp}\PYG{p}{(}\PYG{l+m+mi}{1}\PYG{p}{)}\PYG{o}{*}\PYG{n}{np}\PYG{o}{.}\PYG{n}{arccos}\PYG{p}{(}\PYG{p}{(}\PYG{n}{isochrones\PYGZus{}x\PYGZus{}t1}\PYG{o}{/}\PYG{n}{L\PYGZus{}ref}\PYG{o}{*}\PYG{p}{(}\PYG{n}{np}\PYG{o}{.}\PYG{n}{sin}\PYG{p}{(}\PYG{n}{isochrones\PYGZus{}y\PYGZus{}t1}\PYG{o}{/}\PYG{n}{np}\PYG{o}{.}\PYG{n}{exp}\PYG{p}{(}\PYG{l+m+mi}{1}\PYG{p}{)}\PYG{o}{/}\PYG{n}{L\PYGZus{}ref}\PYG{p}{)}\PYG{o}{/}\PYG{p}{(}\PYG{n}{isochrones\PYGZus{}y\PYGZus{}t1}\PYG{o}{/}\PYG{n}{L\PYGZus{}ref}\PYG{p}{)}\PYG{o}{\PYGZhy{}}\PYG{l+m+mf}{0.5}\PYG{o}{*}\PYG{n}{np}\PYG{o}{.}\PYG{n}{cos}\PYG{p}{(}\PYG{n}{isochrones\PYGZus{}y\PYGZus{}t1}\PYG{o}{/}\PYG{n}{np}\PYG{o}{.}\PYG{n}{exp}\PYG{p}{(}\PYG{l+m+mi}{1}\PYG{p}{)}\PYG{o}{/}\PYG{n}{L\PYGZus{}ref}\PYG{p}{)}\PYG{o}{/}\PYG{n}{np}\PYG{o}{.}\PYG{n}{exp}\PYG{p}{(}\PYG{l+m+mi}{1}\PYG{p}{)}\PYG{p}{)}\PYG{o}{+}\PYG{n}{np}\PYG{o}{.}\PYG{n}{exp}\PYG{p}{(}\PYG{o}{\PYGZhy{}}\PYG{p}{(}\PYG{n}{t1}\PYG{o}{/}\PYG{n}{t\PYGZus{}ref}\PYG{p}{)}\PYG{o}{\PYGZhy{}}\PYG{n}{isochrones\PYGZus{}x\PYGZus{}t1}\PYG{o}{/}\PYG{n}{np}\PYG{o}{.}\PYG{n}{exp}\PYG{p}{(}\PYG{l+m+mi}{1}\PYG{p}{)}\PYG{o}{/}\PYG{n}{L\PYGZus{}ref}\PYG{p}{)}\PYG{p}{)}\PYG{o}{/}\PYG{p}{(}\PYG{l+m+mi}{1}\PYG{o}{\PYGZhy{}}\PYG{l+m+mf}{0.5}\PYG{o}{*}\PYG{n}{isochrones\PYGZus{}x\PYGZus{}t1}\PYG{o}{/}\PYG{n}{np}\PYG{o}{.}\PYG{n}{exp}\PYG{p}{(}\PYG{l+m+mi}{1}\PYG{p}{)}\PYG{o}{/}\PYG{n}{L\PYGZus{}ref}\PYG{p}{)}\PYG{p}{)}
            
                \PYG{n}{isochrones\PYGZus{}y\PYGZus{}t2} \PYG{o}{=} \PYG{n}{L\PYGZus{}ref}\PYG{o}{*}\PYG{n}{np}\PYG{o}{.}\PYG{n}{exp}\PYG{p}{(}\PYG{l+m+mi}{1}\PYG{p}{)}\PYG{o}{*}\PYG{n}{np}\PYG{o}{.}\PYG{n}{arccos}\PYG{p}{(}\PYG{p}{(}\PYG{n}{isochrones\PYGZus{}x\PYGZus{}t2}\PYG{o}{/}\PYG{n}{L\PYGZus{}ref}\PYG{o}{*}\PYG{p}{(}\PYG{n}{np}\PYG{o}{.}\PYG{n}{sin}\PYG{p}{(}\PYG{n}{isochrones\PYGZus{}y\PYGZus{}t2}\PYG{o}{/}\PYG{n}{np}\PYG{o}{.}\PYG{n}{exp}\PYG{p}{(}\PYG{l+m+mi}{1}\PYG{p}{)}\PYG{o}{/}\PYG{n}{L\PYGZus{}ref}\PYG{p}{)}\PYG{o}{/}\PYG{p}{(}\PYG{n}{isochrones\PYGZus{}y\PYGZus{}t2}\PYG{o}{/}\PYG{n}{L\PYGZus{}ref}\PYG{p}{)}\PYG{o}{\PYGZhy{}}\PYG{l+m+mf}{0.5}\PYG{o}{*}\PYG{n}{np}\PYG{o}{.}\PYG{n}{cos}\PYG{p}{(}\PYG{n}{isochrones\PYGZus{}y\PYGZus{}t2}\PYG{o}{/}\PYG{n}{np}\PYG{o}{.}\PYG{n}{exp}\PYG{p}{(}\PYG{l+m+mi}{1}\PYG{p}{)}\PYG{o}{/}\PYG{n}{L\PYGZus{}ref}\PYG{p}{)}\PYG{o}{/}\PYG{n}{np}\PYG{o}{.}\PYG{n}{exp}\PYG{p}{(}\PYG{l+m+mi}{1}\PYG{p}{)}\PYG{p}{)}\PYG{o}{+}\PYG{n}{np}\PYG{o}{.}\PYG{n}{exp}\PYG{p}{(}\PYG{o}{\PYGZhy{}}\PYG{p}{(}\PYG{n}{t2}\PYG{o}{/}\PYG{n}{t\PYGZus{}ref}\PYG{p}{)}\PYG{o}{\PYGZhy{}}\PYG{n}{isochrones\PYGZus{}x\PYGZus{}t2}\PYG{o}{/}\PYG{n}{np}\PYG{o}{.}\PYG{n}{exp}\PYG{p}{(}\PYG{l+m+mi}{1}\PYG{p}{)}\PYG{o}{/}\PYG{n}{L\PYGZus{}ref}\PYG{p}{)}\PYG{p}{)}\PYG{o}{/}\PYG{p}{(}\PYG{l+m+mi}{1}\PYG{o}{\PYGZhy{}}\PYG{l+m+mf}{0.5}\PYG{o}{*}\PYG{n}{isochrones\PYGZus{}x\PYGZus{}t2}\PYG{o}{/}\PYG{n}{np}\PYG{o}{.}\PYG{n}{exp}\PYG{p}{(}\PYG{l+m+mi}{1}\PYG{p}{)}\PYG{o}{/}\PYG{n}{L\PYGZus{}ref}\PYG{p}{)}\PYG{p}{)}

                \PYG{n}{isochrones\PYGZus{}y\PYGZus{}t3} \PYG{o}{=} \PYG{n}{L\PYGZus{}ref}\PYG{o}{*}\PYG{n}{np}\PYG{o}{.}\PYG{n}{exp}\PYG{p}{(}\PYG{l+m+mi}{1}\PYG{p}{)}\PYG{o}{*}\PYG{n}{np}\PYG{o}{.}\PYG{n}{arccos}\PYG{p}{(}\PYG{p}{(}\PYG{n}{isochrones\PYGZus{}x\PYGZus{}t3}\PYG{o}{/}\PYG{n}{L\PYGZus{}ref}\PYG{o}{*}\PYG{p}{(}\PYG{n}{np}\PYG{o}{.}\PYG{n}{sin}\PYG{p}{(}\PYG{n}{isochrones\PYGZus{}y\PYGZus{}t3}\PYG{o}{/}\PYG{n}{np}\PYG{o}{.}\PYG{n}{exp}\PYG{p}{(}\PYG{l+m+mi}{1}\PYG{p}{)}\PYG{o}{/}\PYG{n}{L\PYGZus{}ref}\PYG{p}{)}\PYG{o}{/}\PYG{p}{(}\PYG{n}{isochrones\PYGZus{}y\PYGZus{}t3}\PYG{o}{/}\PYG{n}{L\PYGZus{}ref}\PYG{p}{)}\PYG{o}{\PYGZhy{}}\PYG{l+m+mf}{0.5}\PYG{o}{*}\PYG{n}{np}\PYG{o}{.}\PYG{n}{cos}\PYG{p}{(}\PYG{n}{isochrones\PYGZus{}y\PYGZus{}t3}\PYG{o}{/}\PYG{n}{np}\PYG{o}{.}\PYG{n}{exp}\PYG{p}{(}\PYG{l+m+mi}{1}\PYG{p}{)}\PYG{o}{/}\PYG{n}{L\PYGZus{}ref}\PYG{p}{)}\PYG{o}{/}\PYG{n}{np}\PYG{o}{.}\PYG{n}{exp}\PYG{p}{(}\PYG{l+m+mi}{1}\PYG{p}{)}\PYG{p}{)}\PYG{o}{+}\PYG{n}{np}\PYG{o}{.}\PYG{n}{exp}\PYG{p}{(}\PYG{o}{\PYGZhy{}}\PYG{p}{(}\PYG{n}{t3}\PYG{o}{/}\PYG{n}{t\PYGZus{}ref}\PYG{p}{)}\PYG{o}{\PYGZhy{}}\PYG{n}{isochrones\PYGZus{}x\PYGZus{}t3}\PYG{o}{/}\PYG{n}{np}\PYG{o}{.}\PYG{n}{exp}\PYG{p}{(}\PYG{l+m+mi}{1}\PYG{p}{)}\PYG{o}{/}\PYG{n}{L\PYGZus{}ref}\PYG{p}{)}\PYG{p}{)}\PYG{o}{/}\PYG{p}{(}\PYG{l+m+mi}{1}\PYG{o}{\PYGZhy{}}\PYG{l+m+mf}{0.5}\PYG{o}{*}\PYG{n}{isochrones\PYGZus{}x\PYGZus{}t3}\PYG{o}{/}\PYG{n}{np}\PYG{o}{.}\PYG{n}{exp}\PYG{p}{(}\PYG{l+m+mi}{1}\PYG{p}{)}\PYG{o}{/}\PYG{n}{L\PYGZus{}ref}\PYG{p}{)}\PYG{p}{)}

    
        \PYG{n}{isochrones\PYGZus{}x\PYGZus{}plot\PYGZus{}t1}\PYG{o}{.}\PYG{n}{append}\PYG{p}{(}\PYG{n}{isochrones\PYGZus{}x\PYGZus{}t1}\PYG{p}{)}
        \PYG{n}{isochrones\PYGZus{}y\PYGZus{}plot\PYGZus{}t1}\PYG{o}{.}\PYG{n}{append}\PYG{p}{(}\PYG{n}{isochrones\PYGZus{}y\PYGZus{}t1}\PYG{p}{)}
        \PYG{n}{isochrones\PYGZus{}x\PYGZus{}plot\PYGZus{}t2}\PYG{o}{.}\PYG{n}{append}\PYG{p}{(}\PYG{n}{isochrones\PYGZus{}x\PYGZus{}t2}\PYG{p}{)}
        \PYG{n}{isochrones\PYGZus{}y\PYGZus{}plot\PYGZus{}t2}\PYG{o}{.}\PYG{n}{append}\PYG{p}{(}\PYG{n}{isochrones\PYGZus{}y\PYGZus{}t2}\PYG{p}{)}
        \PYG{n}{isochrones\PYGZus{}x\PYGZus{}plot\PYGZus{}t3}\PYG{o}{.}\PYG{n}{append}\PYG{p}{(}\PYG{n}{isochrones\PYGZus{}x\PYGZus{}t3}\PYG{p}{)}
        \PYG{n}{isochrones\PYGZus{}y\PYGZus{}plot\PYGZus{}t3}\PYG{o}{.}\PYG{n}{append}\PYG{p}{(}\PYG{n}{isochrones\PYGZus{}y\PYGZus{}t3}\PYG{p}{)}

    \PYG{c+c1}{\PYGZsh{}still necessary: mirror on x\PYGZhy{}axis }
    \PYG{n}{isolines\PYGZus{}y\PYGZus{}plot\PYGZus{}n5\PYGZus{}1\PYGZus{}mirror} \PYG{o}{=} \PYG{o}{\PYGZhy{}}\PYG{l+m+mi}{1}\PYG{o}{*}\PYG{p}{(}\PYG{n}{np}\PYG{o}{.}\PYG{n}{asarray}\PYG{p}{(}\PYG{n}{isolines\PYGZus{}y\PYGZus{}plot\PYGZus{}n5\PYGZus{}1}\PYG{p}{)}\PYG{p}{)}
    \PYG{n}{isolines\PYGZus{}y\PYGZus{}plot\PYGZus{}n5\PYGZus{}2\PYGZus{}mirror} \PYG{o}{=} \PYG{o}{\PYGZhy{}}\PYG{l+m+mi}{1}\PYG{o}{*}\PYG{p}{(}\PYG{n}{np}\PYG{o}{.}\PYG{n}{asarray}\PYG{p}{(}\PYG{n}{isolines\PYGZus{}y\PYGZus{}plot\PYGZus{}n5\PYGZus{}2}\PYG{p}{)}\PYG{p}{)}
    \PYG{n}{isolines\PYGZus{}y\PYGZus{}plot\PYGZus{}n2\PYGZus{}5\PYGZus{}1\PYGZus{}mirror} \PYG{o}{=} \PYG{o}{\PYGZhy{}}\PYG{l+m+mi}{1}\PYG{o}{*}\PYG{p}{(}\PYG{n}{np}\PYG{o}{.}\PYG{n}{asarray}\PYG{p}{(}\PYG{n}{isolines\PYGZus{}y\PYGZus{}plot\PYGZus{}n2\PYGZus{}5\PYGZus{}1}\PYG{p}{)}\PYG{p}{)}
    \PYG{n}{isolines\PYGZus{}y\PYGZus{}plot\PYGZus{}n2\PYGZus{}5\PYGZus{}2\PYGZus{}mirror} \PYG{o}{=} \PYG{o}{\PYGZhy{}}\PYG{l+m+mi}{1}\PYG{o}{*}\PYG{p}{(}\PYG{n}{np}\PYG{o}{.}\PYG{n}{asarray}\PYG{p}{(}\PYG{n}{isolines\PYGZus{}y\PYGZus{}plot\PYGZus{}n2\PYGZus{}5\PYGZus{}2}\PYG{p}{)}\PYG{p}{)}
    \PYG{n}{isolines\PYGZus{}y\PYGZus{}plot\PYGZus{}0\PYGZus{}1\PYGZus{}mirror} \PYG{o}{=} \PYG{o}{\PYGZhy{}}\PYG{l+m+mi}{1}\PYG{o}{*}\PYG{p}{(}\PYG{n}{np}\PYG{o}{.}\PYG{n}{asarray}\PYG{p}{(}\PYG{n}{isolines\PYGZus{}y\PYGZus{}plot\PYGZus{}0\PYGZus{}1}\PYG{p}{)}\PYG{p}{)}
    \PYG{n}{isolines\PYGZus{}y\PYGZus{}plot\PYGZus{}0\PYGZus{}2\PYGZus{}mirror} \PYG{o}{=} \PYG{o}{\PYGZhy{}}\PYG{l+m+mi}{1}\PYG{o}{*}\PYG{p}{(}\PYG{n}{np}\PYG{o}{.}\PYG{n}{asarray}\PYG{p}{(}\PYG{n}{isolines\PYGZus{}y\PYGZus{}plot\PYGZus{}0\PYGZus{}2}\PYG{p}{)}\PYG{p}{)}
    \PYG{n}{isolines\PYGZus{}y\PYGZus{}plot\PYGZus{}2\PYGZus{}5\PYGZus{}mirror} \PYG{o}{=} \PYG{o}{\PYGZhy{}}\PYG{l+m+mi}{1}\PYG{o}{*}\PYG{p}{(}\PYG{n}{np}\PYG{o}{.}\PYG{n}{asarray}\PYG{p}{(}\PYG{n}{isolines\PYGZus{}y\PYGZus{}plot\PYGZus{}2\PYGZus{}5}\PYG{p}{)}\PYG{p}{)}
    \PYG{n}{isolines\PYGZus{}y\PYGZus{}plot\PYGZus{}5\PYGZus{}mirror} \PYG{o}{=} \PYG{o}{\PYGZhy{}}\PYG{l+m+mi}{1}\PYG{o}{*}\PYG{p}{(}\PYG{n}{np}\PYG{o}{.}\PYG{n}{asarray}\PYG{p}{(}\PYG{n}{isolines\PYGZus{}y\PYGZus{}plot\PYGZus{}5}\PYG{p}{)}\PYG{p}{)}
    \PYG{n}{isolines\PYGZus{}y\PYGZus{}plot\PYGZus{}7\PYGZus{}5\PYGZus{}mirror} \PYG{o}{=} \PYG{o}{\PYGZhy{}}\PYG{l+m+mi}{1}\PYG{o}{*}\PYG{p}{(}\PYG{n}{np}\PYG{o}{.}\PYG{n}{asarray}\PYG{p}{(}\PYG{n}{isolines\PYGZus{}y\PYGZus{}plot\PYGZus{}7\PYGZus{}5}\PYG{p}{)}\PYG{p}{)}
    \PYG{n}{isolines\PYGZus{}y\PYGZus{}plot\PYGZus{}10\PYGZus{}mirror} \PYG{o}{=} \PYG{o}{\PYGZhy{}}\PYG{l+m+mi}{1}\PYG{o}{*}\PYG{p}{(}\PYG{n}{np}\PYG{o}{.}\PYG{n}{asarray}\PYG{p}{(}\PYG{n}{isolines\PYGZus{}y\PYGZus{}plot\PYGZus{}10}\PYG{p}{)}\PYG{p}{)}
    \PYG{n}{isolines\PYGZus{}y\PYGZus{}plot\PYGZus{}12\PYGZus{}5\PYGZus{}mirror} \PYG{o}{=} \PYG{o}{\PYGZhy{}}\PYG{l+m+mi}{1}\PYG{o}{*}\PYG{p}{(}\PYG{n}{np}\PYG{o}{.}\PYG{n}{asarray}\PYG{p}{(}\PYG{n}{isolines\PYGZus{}y\PYGZus{}plot\PYGZus{}12\PYGZus{}5}\PYG{p}{)}\PYG{p}{)}
    \PYG{n}{isolines\PYGZus{}y\PYGZus{}plot\PYGZus{}15\PYGZus{}mirror} \PYG{o}{=} \PYG{o}{\PYGZhy{}}\PYG{l+m+mi}{1}\PYG{o}{*}\PYG{p}{(}\PYG{n}{np}\PYG{o}{.}\PYG{n}{asarray}\PYG{p}{(}\PYG{n}{isolines\PYGZus{}y\PYGZus{}plot\PYGZus{}15}\PYG{p}{)}\PYG{p}{)}
    
    \PYG{n}{streamlines\PYGZus{}y\PYGZus{}plot\PYGZus{}0\PYGZus{}2\PYGZus{}mirror} \PYG{o}{=} \PYG{o}{\PYGZhy{}}\PYG{l+m+mi}{1}\PYG{o}{*}\PYG{p}{(}\PYG{n}{np}\PYG{o}{.}\PYG{n}{asarray}\PYG{p}{(}\PYG{n}{streamlines\PYGZus{}y\PYGZus{}plot\PYGZus{}0\PYGZus{}2}\PYG{p}{)}\PYG{p}{)}
    \PYG{n}{streamlines\PYGZus{}y\PYGZus{}plot\PYGZus{}0\PYGZus{}4\PYGZus{}mirror} \PYG{o}{=} \PYG{o}{\PYGZhy{}}\PYG{l+m+mi}{1}\PYG{o}{*}\PYG{p}{(}\PYG{n}{np}\PYG{o}{.}\PYG{n}{asarray}\PYG{p}{(}\PYG{n}{streamlines\PYGZus{}y\PYGZus{}plot\PYGZus{}0\PYGZus{}4}\PYG{p}{)}\PYG{p}{)}
    \PYG{n}{streamlines\PYGZus{}y\PYGZus{}plot\PYGZus{}0\PYGZus{}6\PYGZus{}mirror} \PYG{o}{=} \PYG{o}{\PYGZhy{}}\PYG{l+m+mi}{1}\PYG{o}{*}\PYG{p}{(}\PYG{n}{np}\PYG{o}{.}\PYG{n}{asarray}\PYG{p}{(}\PYG{n}{streamlines\PYGZus{}y\PYGZus{}plot\PYGZus{}0\PYGZus{}6}\PYG{p}{)}\PYG{p}{)}
    \PYG{n}{streamlines\PYGZus{}y\PYGZus{}plot\PYGZus{}0\PYGZus{}8\PYGZus{}mirror} \PYG{o}{=} \PYG{o}{\PYGZhy{}}\PYG{l+m+mi}{1}\PYG{o}{*}\PYG{p}{(}\PYG{n}{np}\PYG{o}{.}\PYG{n}{asarray}\PYG{p}{(}\PYG{n}{streamlines\PYGZus{}y\PYGZus{}plot\PYGZus{}0\PYGZus{}8}\PYG{p}{)}\PYG{p}{)}
    \PYG{n}{streamlines\PYGZus{}y\PYGZus{}plot\PYGZus{}1\PYGZus{}mirror} \PYG{o}{=} \PYG{o}{\PYGZhy{}}\PYG{l+m+mi}{1}\PYG{o}{*}\PYG{p}{(}\PYG{n}{np}\PYG{o}{.}\PYG{n}{asarray}\PYG{p}{(}\PYG{n}{streamlines\PYGZus{}y\PYGZus{}plot\PYGZus{}1}\PYG{p}{)}\PYG{p}{)}
    \PYG{n}{streamlines\PYGZus{}y\PYGZus{}plot\PYGZus{}1\PYGZus{}2\PYGZus{}mirror} \PYG{o}{=} \PYG{o}{\PYGZhy{}}\PYG{l+m+mi}{1}\PYG{o}{*}\PYG{p}{(}\PYG{n}{np}\PYG{o}{.}\PYG{n}{asarray}\PYG{p}{(}\PYG{n}{streamlines\PYGZus{}y\PYGZus{}plot\PYGZus{}1\PYGZus{}2}\PYG{p}{)}\PYG{p}{)}
    \PYG{n}{streamlines\PYGZus{}y\PYGZus{}plot\PYGZus{}1\PYGZus{}4\PYGZus{}mirror} \PYG{o}{=} \PYG{o}{\PYGZhy{}}\PYG{l+m+mi}{1}\PYG{o}{*}\PYG{p}{(}\PYG{n}{np}\PYG{o}{.}\PYG{n}{asarray}\PYG{p}{(}\PYG{n}{streamlines\PYGZus{}y\PYGZus{}plot\PYGZus{}1\PYGZus{}4}\PYG{p}{)}\PYG{p}{)}
    \PYG{n}{streamlines\PYGZus{}y\PYGZus{}plot\PYGZus{}1\PYGZus{}6\PYGZus{}mirror} \PYG{o}{=} \PYG{o}{\PYGZhy{}}\PYG{l+m+mi}{1}\PYG{o}{*}\PYG{p}{(}\PYG{n}{np}\PYG{o}{.}\PYG{n}{asarray}\PYG{p}{(}\PYG{n}{streamlines\PYGZus{}y\PYGZus{}plot\PYGZus{}1\PYGZus{}6}\PYG{p}{)}\PYG{p}{)}

    
    \PYG{n}{isochrones\PYGZus{}y\PYGZus{}plot\PYGZus{}t1\PYGZus{}mirror} \PYG{o}{=} \PYG{o}{\PYGZhy{}}\PYG{l+m+mi}{1}\PYG{o}{*}\PYG{p}{(}\PYG{n}{np}\PYG{o}{.}\PYG{n}{asarray}\PYG{p}{(}\PYG{n}{isochrones\PYGZus{}y\PYGZus{}plot\PYGZus{}t1}\PYG{p}{)}\PYG{p}{)}
    \PYG{n}{isochrones\PYGZus{}y\PYGZus{}plot\PYGZus{}t2\PYGZus{}mirror} \PYG{o}{=} \PYG{o}{\PYGZhy{}}\PYG{l+m+mi}{1}\PYG{o}{*}\PYG{p}{(}\PYG{n}{np}\PYG{o}{.}\PYG{n}{asarray}\PYG{p}{(}\PYG{n}{isochrones\PYGZus{}y\PYGZus{}plot\PYGZus{}t2}\PYG{p}{)}\PYG{p}{)}
    \PYG{n}{isochrones\PYGZus{}y\PYGZus{}plot\PYGZus{}t3\PYGZus{}mirror} \PYG{o}{=} \PYG{o}{\PYGZhy{}}\PYG{l+m+mi}{1}\PYG{o}{*}\PYG{p}{(}\PYG{n}{np}\PYG{o}{.}\PYG{n}{asarray}\PYG{p}{(}\PYG{n}{isochrones\PYGZus{}y\PYGZus{}plot\PYGZus{}t3}\PYG{p}{)}\PYG{p}{)}
    

    \PYG{n}{fig}\PYG{p}{,} \PYG{p}{(}\PYG{n}{ax1}\PYG{p}{,} \PYG{n}{ax2}\PYG{p}{)} \PYG{o}{=} \PYG{n}{plt}\PYG{o}{.}\PYG{n}{subplots}\PYG{p}{(}\PYG{l+m+mi}{2}\PYG{p}{,} \PYG{n}{figsize}\PYG{o}{=}\PYG{p}{(}\PYG{l+m+mi}{10}\PYG{p}{,} \PYG{l+m+mi}{10}\PYG{p}{)}\PYG{p}{)}

    \PYG{c+c1}{\PYGZsh{}plotten incl. mirror on x\PYGZhy{}axis}

    \PYG{n}{ax1}\PYG{o}{.}\PYG{n}{plot}\PYG{p}{(}\PYG{n}{isolines\PYGZus{}x\PYGZus{}plot\PYGZus{}n5\PYGZus{}1}\PYG{p}{,} \PYG{n}{isolines\PYGZus{}y\PYGZus{}plot\PYGZus{}n5\PYGZus{}1}\PYG{p}{,} \PYG{l+s+s1}{\PYGZsq{}}\PYG{l+s+s1}{r}\PYG{l+s+s1}{\PYGZsq{}}\PYG{p}{)}
    \PYG{n}{ax1}\PYG{o}{.}\PYG{n}{plot}\PYG{p}{(}\PYG{n}{isolines\PYGZus{}x\PYGZus{}plot\PYGZus{}n5\PYGZus{}2}\PYG{p}{,} \PYG{n}{isolines\PYGZus{}y\PYGZus{}plot\PYGZus{}n5\PYGZus{}2}\PYG{p}{,} \PYG{l+s+s1}{\PYGZsq{}}\PYG{l+s+s1}{r}\PYG{l+s+s1}{\PYGZsq{}}\PYG{p}{)}
    \PYG{n}{ax1}\PYG{o}{.}\PYG{n}{plot}\PYG{p}{(}\PYG{n}{isolines\PYGZus{}x\PYGZus{}plot\PYGZus{}n2\PYGZus{}5\PYGZus{}1}\PYG{p}{,} \PYG{n}{isolines\PYGZus{}y\PYGZus{}plot\PYGZus{}n2\PYGZus{}5\PYGZus{}1}\PYG{p}{,} \PYG{l+s+s1}{\PYGZsq{}}\PYG{l+s+s1}{r}\PYG{l+s+s1}{\PYGZsq{}}\PYG{p}{)}
    \PYG{n}{ax1}\PYG{o}{.}\PYG{n}{plot}\PYG{p}{(}\PYG{n}{isolines\PYGZus{}x\PYGZus{}plot\PYGZus{}n2\PYGZus{}5\PYGZus{}2}\PYG{p}{,} \PYG{n}{isolines\PYGZus{}y\PYGZus{}plot\PYGZus{}n2\PYGZus{}5\PYGZus{}2}\PYG{p}{,} \PYG{l+s+s1}{\PYGZsq{}}\PYG{l+s+s1}{r}\PYG{l+s+s1}{\PYGZsq{}}\PYG{p}{)}
    \PYG{n}{ax1}\PYG{o}{.}\PYG{n}{plot}\PYG{p}{(}\PYG{n}{isolines\PYGZus{}x\PYGZus{}plot\PYGZus{}0\PYGZus{}1}\PYG{p}{,} \PYG{n}{isolines\PYGZus{}y\PYGZus{}plot\PYGZus{}0\PYGZus{}1}\PYG{p}{,} \PYG{l+s+s1}{\PYGZsq{}}\PYG{l+s+s1}{r}\PYG{l+s+s1}{\PYGZsq{}}\PYG{p}{)}
    \PYG{n}{ax1}\PYG{o}{.}\PYG{n}{plot}\PYG{p}{(}\PYG{n}{isolines\PYGZus{}x\PYGZus{}plot\PYGZus{}0\PYGZus{}2}\PYG{p}{,} \PYG{n}{isolines\PYGZus{}y\PYGZus{}plot\PYGZus{}0\PYGZus{}2}\PYG{p}{,} \PYG{l+s+s1}{\PYGZsq{}}\PYG{l+s+s1}{r}\PYG{l+s+s1}{\PYGZsq{}}\PYG{p}{)}
    \PYG{n}{ax1}\PYG{o}{.}\PYG{n}{plot}\PYG{p}{(}\PYG{n}{isolines\PYGZus{}x\PYGZus{}plot\PYGZus{}2\PYGZus{}5}\PYG{p}{,} \PYG{n}{isolines\PYGZus{}y\PYGZus{}plot\PYGZus{}2\PYGZus{}5}\PYG{p}{,} \PYG{l+s+s1}{\PYGZsq{}}\PYG{l+s+s1}{r}\PYG{l+s+s1}{\PYGZsq{}}\PYG{p}{)}
    \PYG{n}{ax1}\PYG{o}{.}\PYG{n}{plot}\PYG{p}{(}\PYG{n}{isolines\PYGZus{}x\PYGZus{}plot\PYGZus{}5}\PYG{p}{,} \PYG{n}{isolines\PYGZus{}y\PYGZus{}plot\PYGZus{}5}\PYG{p}{,} \PYG{l+s+s1}{\PYGZsq{}}\PYG{l+s+s1}{r}\PYG{l+s+s1}{\PYGZsq{}}\PYG{p}{)}
    \PYG{n}{ax1}\PYG{o}{.}\PYG{n}{plot}\PYG{p}{(}\PYG{n}{isolines\PYGZus{}x\PYGZus{}plot\PYGZus{}7\PYGZus{}5}\PYG{p}{,} \PYG{n}{isolines\PYGZus{}y\PYGZus{}plot\PYGZus{}7\PYGZus{}5}\PYG{p}{,} \PYG{l+s+s1}{\PYGZsq{}}\PYG{l+s+s1}{r}\PYG{l+s+s1}{\PYGZsq{}}\PYG{p}{)}
    \PYG{n}{ax1}\PYG{o}{.}\PYG{n}{plot}\PYG{p}{(}\PYG{n}{isolines\PYGZus{}x\PYGZus{}plot\PYGZus{}10}\PYG{p}{,} \PYG{n}{isolines\PYGZus{}y\PYGZus{}plot\PYGZus{}10}\PYG{p}{,} \PYG{l+s+s1}{\PYGZsq{}}\PYG{l+s+s1}{r}\PYG{l+s+s1}{\PYGZsq{}}\PYG{p}{)}
    \PYG{n}{ax1}\PYG{o}{.}\PYG{n}{plot}\PYG{p}{(}\PYG{n}{isolines\PYGZus{}x\PYGZus{}plot\PYGZus{}12\PYGZus{}5}\PYG{p}{,} \PYG{n}{isolines\PYGZus{}y\PYGZus{}plot\PYGZus{}12\PYGZus{}5}\PYG{p}{,} \PYG{l+s+s1}{\PYGZsq{}}\PYG{l+s+s1}{r}\PYG{l+s+s1}{\PYGZsq{}}\PYG{p}{)}
    \PYG{n}{ax1}\PYG{o}{.}\PYG{n}{plot}\PYG{p}{(}\PYG{n}{isolines\PYGZus{}x\PYGZus{}plot\PYGZus{}15}\PYG{p}{,} \PYG{n}{isolines\PYGZus{}y\PYGZus{}plot\PYGZus{}15}\PYG{p}{,} \PYG{l+s+s1}{\PYGZsq{}}\PYG{l+s+s1}{r}\PYG{l+s+s1}{\PYGZsq{}}\PYG{p}{)}

    \PYG{n}{ax1}\PYG{o}{.}\PYG{n}{plot}\PYG{p}{(}\PYG{n}{isolines\PYGZus{}x\PYGZus{}plot\PYGZus{}n5\PYGZus{}1}\PYG{p}{,} \PYG{n}{isolines\PYGZus{}y\PYGZus{}plot\PYGZus{}n5\PYGZus{}1\PYGZus{}mirror}\PYG{p}{,} \PYG{l+s+s1}{\PYGZsq{}}\PYG{l+s+s1}{r}\PYG{l+s+s1}{\PYGZsq{}}\PYG{p}{)}
    \PYG{n}{ax1}\PYG{o}{.}\PYG{n}{plot}\PYG{p}{(}\PYG{n}{isolines\PYGZus{}x\PYGZus{}plot\PYGZus{}n5\PYGZus{}2}\PYG{p}{,} \PYG{n}{isolines\PYGZus{}y\PYGZus{}plot\PYGZus{}n5\PYGZus{}2\PYGZus{}mirror}\PYG{p}{,} \PYG{l+s+s1}{\PYGZsq{}}\PYG{l+s+s1}{r}\PYG{l+s+s1}{\PYGZsq{}}\PYG{p}{)}
    \PYG{n}{ax1}\PYG{o}{.}\PYG{n}{plot}\PYG{p}{(}\PYG{n}{isolines\PYGZus{}x\PYGZus{}plot\PYGZus{}n2\PYGZus{}5\PYGZus{}1}\PYG{p}{,} \PYG{n}{isolines\PYGZus{}y\PYGZus{}plot\PYGZus{}n2\PYGZus{}5\PYGZus{}1\PYGZus{}mirror}\PYG{p}{,} \PYG{l+s+s1}{\PYGZsq{}}\PYG{l+s+s1}{r}\PYG{l+s+s1}{\PYGZsq{}}\PYG{p}{)}
    \PYG{n}{ax1}\PYG{o}{.}\PYG{n}{plot}\PYG{p}{(}\PYG{n}{isolines\PYGZus{}x\PYGZus{}plot\PYGZus{}n2\PYGZus{}5\PYGZus{}2}\PYG{p}{,} \PYG{n}{isolines\PYGZus{}y\PYGZus{}plot\PYGZus{}n2\PYGZus{}5\PYGZus{}2\PYGZus{}mirror}\PYG{p}{,} \PYG{l+s+s1}{\PYGZsq{}}\PYG{l+s+s1}{r}\PYG{l+s+s1}{\PYGZsq{}}\PYG{p}{)}
    \PYG{n}{ax1}\PYG{o}{.}\PYG{n}{plot}\PYG{p}{(}\PYG{n}{isolines\PYGZus{}x\PYGZus{}plot\PYGZus{}0\PYGZus{}1}\PYG{p}{,} \PYG{n}{isolines\PYGZus{}y\PYGZus{}plot\PYGZus{}0\PYGZus{}1\PYGZus{}mirror}\PYG{p}{,} \PYG{l+s+s1}{\PYGZsq{}}\PYG{l+s+s1}{r}\PYG{l+s+s1}{\PYGZsq{}}\PYG{p}{)}
    \PYG{n}{ax1}\PYG{o}{.}\PYG{n}{plot}\PYG{p}{(}\PYG{n}{isolines\PYGZus{}x\PYGZus{}plot\PYGZus{}0\PYGZus{}2}\PYG{p}{,} \PYG{n}{isolines\PYGZus{}y\PYGZus{}plot\PYGZus{}0\PYGZus{}2\PYGZus{}mirror}\PYG{p}{,} \PYG{l+s+s1}{\PYGZsq{}}\PYG{l+s+s1}{r}\PYG{l+s+s1}{\PYGZsq{}}\PYG{p}{)}
    \PYG{n}{ax1}\PYG{o}{.}\PYG{n}{plot}\PYG{p}{(}\PYG{n}{isolines\PYGZus{}x\PYGZus{}plot\PYGZus{}2\PYGZus{}5}\PYG{p}{,} \PYG{n}{isolines\PYGZus{}y\PYGZus{}plot\PYGZus{}2\PYGZus{}5\PYGZus{}mirror}\PYG{p}{,} \PYG{l+s+s1}{\PYGZsq{}}\PYG{l+s+s1}{r}\PYG{l+s+s1}{\PYGZsq{}}\PYG{p}{)}
    \PYG{n}{ax1}\PYG{o}{.}\PYG{n}{plot}\PYG{p}{(}\PYG{n}{isolines\PYGZus{}x\PYGZus{}plot\PYGZus{}5}\PYG{p}{,} \PYG{n}{isolines\PYGZus{}y\PYGZus{}plot\PYGZus{}5\PYGZus{}mirror}\PYG{p}{,} \PYG{l+s+s1}{\PYGZsq{}}\PYG{l+s+s1}{r}\PYG{l+s+s1}{\PYGZsq{}}\PYG{p}{)}
    \PYG{n}{ax1}\PYG{o}{.}\PYG{n}{plot}\PYG{p}{(}\PYG{n}{isolines\PYGZus{}x\PYGZus{}plot\PYGZus{}7\PYGZus{}5}\PYG{p}{,} \PYG{n}{isolines\PYGZus{}y\PYGZus{}plot\PYGZus{}7\PYGZus{}5\PYGZus{}mirror}\PYG{p}{,} \PYG{l+s+s1}{\PYGZsq{}}\PYG{l+s+s1}{r}\PYG{l+s+s1}{\PYGZsq{}}\PYG{p}{)}
    \PYG{n}{ax1}\PYG{o}{.}\PYG{n}{plot}\PYG{p}{(}\PYG{n}{isolines\PYGZus{}x\PYGZus{}plot\PYGZus{}10}\PYG{p}{,} \PYG{n}{isolines\PYGZus{}y\PYGZus{}plot\PYGZus{}10\PYGZus{}mirror}\PYG{p}{,} \PYG{l+s+s1}{\PYGZsq{}}\PYG{l+s+s1}{r}\PYG{l+s+s1}{\PYGZsq{}}\PYG{p}{)}
    \PYG{n}{ax1}\PYG{o}{.}\PYG{n}{plot}\PYG{p}{(}\PYG{n}{isolines\PYGZus{}x\PYGZus{}plot\PYGZus{}12\PYGZus{}5}\PYG{p}{,} \PYG{n}{isolines\PYGZus{}y\PYGZus{}plot\PYGZus{}12\PYGZus{}5\PYGZus{}mirror}\PYG{p}{,} \PYG{l+s+s1}{\PYGZsq{}}\PYG{l+s+s1}{r}\PYG{l+s+s1}{\PYGZsq{}}\PYG{p}{)}
    \PYG{n}{ax1}\PYG{o}{.}\PYG{n}{plot}\PYG{p}{(}\PYG{n}{isolines\PYGZus{}x\PYGZus{}plot\PYGZus{}15}\PYG{p}{,} \PYG{n}{isolines\PYGZus{}y\PYGZus{}plot\PYGZus{}15\PYGZus{}mirror}\PYG{p}{,} \PYG{l+s+s1}{\PYGZsq{}}\PYG{l+s+s1}{r}\PYG{l+s+s1}{\PYGZsq{}}\PYG{p}{)}

    \PYG{n}{ax1}\PYG{o}{.}\PYG{n}{plot}\PYG{p}{(}\PYG{n}{streamlines\PYGZus{}x\PYGZus{}plot\PYGZus{}0}\PYG{p}{,}\PYG{n}{streamlines\PYGZus{}y\PYGZus{}plot\PYGZus{}0}\PYG{p}{,} \PYG{l+s+s1}{\PYGZsq{}}\PYG{l+s+s1}{b}\PYG{l+s+s1}{\PYGZsq{}}\PYG{p}{)}
    \PYG{n}{ax1}\PYG{o}{.}\PYG{n}{plot}\PYG{p}{(}\PYG{n}{streamlines\PYGZus{}x\PYGZus{}plot\PYGZus{}0\PYGZus{}2}\PYG{p}{,}\PYG{n}{streamlines\PYGZus{}y\PYGZus{}plot\PYGZus{}0\PYGZus{}2}\PYG{p}{,} \PYG{l+s+s1}{\PYGZsq{}}\PYG{l+s+s1}{b}\PYG{l+s+s1}{\PYGZsq{}}\PYG{p}{)}
    \PYG{n}{ax1}\PYG{o}{.}\PYG{n}{plot}\PYG{p}{(}\PYG{n}{streamlines\PYGZus{}x\PYGZus{}plot\PYGZus{}0\PYGZus{}4}\PYG{p}{,}\PYG{n}{streamlines\PYGZus{}y\PYGZus{}plot\PYGZus{}0\PYGZus{}4}\PYG{p}{,} \PYG{l+s+s1}{\PYGZsq{}}\PYG{l+s+s1}{b}\PYG{l+s+s1}{\PYGZsq{}}\PYG{p}{)}
    \PYG{n}{ax1}\PYG{o}{.}\PYG{n}{plot}\PYG{p}{(}\PYG{n}{streamlines\PYGZus{}x\PYGZus{}plot\PYGZus{}0\PYGZus{}6}\PYG{p}{,}\PYG{n}{streamlines\PYGZus{}y\PYGZus{}plot\PYGZus{}0\PYGZus{}6}\PYG{p}{,} \PYG{l+s+s1}{\PYGZsq{}}\PYG{l+s+s1}{b}\PYG{l+s+s1}{\PYGZsq{}}\PYG{p}{)}
    \PYG{n}{ax1}\PYG{o}{.}\PYG{n}{plot}\PYG{p}{(}\PYG{n}{streamlines\PYGZus{}x\PYGZus{}plot\PYGZus{}0\PYGZus{}8}\PYG{p}{,}\PYG{n}{streamlines\PYGZus{}y\PYGZus{}plot\PYGZus{}0\PYGZus{}8}\PYG{p}{,} \PYG{l+s+s1}{\PYGZsq{}}\PYG{l+s+s1}{b}\PYG{l+s+s1}{\PYGZsq{}}\PYG{p}{)}
    \PYG{n}{ax1}\PYG{o}{.}\PYG{n}{plot}\PYG{p}{(}\PYG{n}{streamlines\PYGZus{}x\PYGZus{}plot\PYGZus{}1}\PYG{p}{,}\PYG{n}{streamlines\PYGZus{}y\PYGZus{}plot\PYGZus{}1}\PYG{p}{,} \PYG{n}{color} \PYG{o}{=} \PYG{l+s+s1}{\PYGZsq{}}\PYG{l+s+s1}{black}\PYG{l+s+s1}{\PYGZsq{}}\PYG{p}{)}
    \PYG{n}{ax1}\PYG{o}{.}\PYG{n}{plot}\PYG{p}{(}\PYG{n}{streamlines\PYGZus{}x\PYGZus{}plot\PYGZus{}1\PYGZus{}2}\PYG{p}{,}\PYG{n}{streamlines\PYGZus{}y\PYGZus{}plot\PYGZus{}1\PYGZus{}2}\PYG{p}{,} \PYG{l+s+s1}{\PYGZsq{}}\PYG{l+s+s1}{b}\PYG{l+s+s1}{\PYGZsq{}}\PYG{p}{)}
    \PYG{n}{ax1}\PYG{o}{.}\PYG{n}{plot}\PYG{p}{(}\PYG{n}{streamlines\PYGZus{}x\PYGZus{}plot\PYGZus{}1\PYGZus{}4}\PYG{p}{,}\PYG{n}{streamlines\PYGZus{}y\PYGZus{}plot\PYGZus{}1\PYGZus{}4}\PYG{p}{,} \PYG{l+s+s1}{\PYGZsq{}}\PYG{l+s+s1}{b}\PYG{l+s+s1}{\PYGZsq{}}\PYG{p}{)}
    \PYG{n}{ax1}\PYG{o}{.}\PYG{n}{plot}\PYG{p}{(}\PYG{n}{streamlines\PYGZus{}x\PYGZus{}plot\PYGZus{}1\PYGZus{}6}\PYG{p}{,}\PYG{n}{streamlines\PYGZus{}y\PYGZus{}plot\PYGZus{}1\PYGZus{}6}\PYG{p}{,} \PYG{l+s+s1}{\PYGZsq{}}\PYG{l+s+s1}{b}\PYG{l+s+s1}{\PYGZsq{}}\PYG{p}{)}

    
    \PYG{n}{ax1}\PYG{o}{.}\PYG{n}{plot}\PYG{p}{(}\PYG{n}{streamlines\PYGZus{}x\PYGZus{}plot\PYGZus{}0\PYGZus{}2}\PYG{p}{,}\PYG{n}{streamlines\PYGZus{}y\PYGZus{}plot\PYGZus{}0\PYGZus{}2\PYGZus{}mirror}\PYG{p}{,} \PYG{l+s+s1}{\PYGZsq{}}\PYG{l+s+s1}{b}\PYG{l+s+s1}{\PYGZsq{}}\PYG{p}{)}
    \PYG{n}{ax1}\PYG{o}{.}\PYG{n}{plot}\PYG{p}{(}\PYG{n}{streamlines\PYGZus{}x\PYGZus{}plot\PYGZus{}0\PYGZus{}4}\PYG{p}{,}\PYG{n}{streamlines\PYGZus{}y\PYGZus{}plot\PYGZus{}0\PYGZus{}4\PYGZus{}mirror}\PYG{p}{,} \PYG{l+s+s1}{\PYGZsq{}}\PYG{l+s+s1}{b}\PYG{l+s+s1}{\PYGZsq{}}\PYG{p}{)}
    \PYG{n}{ax1}\PYG{o}{.}\PYG{n}{plot}\PYG{p}{(}\PYG{n}{streamlines\PYGZus{}x\PYGZus{}plot\PYGZus{}0\PYGZus{}6}\PYG{p}{,}\PYG{n}{streamlines\PYGZus{}y\PYGZus{}plot\PYGZus{}0\PYGZus{}6\PYGZus{}mirror}\PYG{p}{,} \PYG{l+s+s1}{\PYGZsq{}}\PYG{l+s+s1}{b}\PYG{l+s+s1}{\PYGZsq{}}\PYG{p}{)}
    \PYG{n}{ax1}\PYG{o}{.}\PYG{n}{plot}\PYG{p}{(}\PYG{n}{streamlines\PYGZus{}x\PYGZus{}plot\PYGZus{}0\PYGZus{}8}\PYG{p}{,}\PYG{n}{streamlines\PYGZus{}y\PYGZus{}plot\PYGZus{}0\PYGZus{}8\PYGZus{}mirror}\PYG{p}{,} \PYG{l+s+s1}{\PYGZsq{}}\PYG{l+s+s1}{b}\PYG{l+s+s1}{\PYGZsq{}}\PYG{p}{)}
    \PYG{n}{ax1}\PYG{o}{.}\PYG{n}{plot}\PYG{p}{(}\PYG{n}{streamlines\PYGZus{}x\PYGZus{}plot\PYGZus{}1}\PYG{p}{,}\PYG{n}{streamlines\PYGZus{}y\PYGZus{}plot\PYGZus{}1\PYGZus{}mirror}\PYG{p}{,} \PYG{n}{color} \PYG{o}{=} \PYG{l+s+s1}{\PYGZsq{}}\PYG{l+s+s1}{black}\PYG{l+s+s1}{\PYGZsq{}}\PYG{p}{)}
    \PYG{n}{ax1}\PYG{o}{.}\PYG{n}{plot}\PYG{p}{(}\PYG{n}{streamlines\PYGZus{}x\PYGZus{}plot\PYGZus{}1\PYGZus{}2}\PYG{p}{,}\PYG{n}{streamlines\PYGZus{}y\PYGZus{}plot\PYGZus{}1\PYGZus{}2\PYGZus{}mirror}\PYG{p}{,} \PYG{l+s+s1}{\PYGZsq{}}\PYG{l+s+s1}{b}\PYG{l+s+s1}{\PYGZsq{}}\PYG{p}{)}
    \PYG{n}{ax1}\PYG{o}{.}\PYG{n}{plot}\PYG{p}{(}\PYG{n}{streamlines\PYGZus{}x\PYGZus{}plot\PYGZus{}1\PYGZus{}4}\PYG{p}{,}\PYG{n}{streamlines\PYGZus{}y\PYGZus{}plot\PYGZus{}1\PYGZus{}4\PYGZus{}mirror}\PYG{p}{,} \PYG{l+s+s1}{\PYGZsq{}}\PYG{l+s+s1}{b}\PYG{l+s+s1}{\PYGZsq{}}\PYG{p}{)}
    \PYG{n}{ax1}\PYG{o}{.}\PYG{n}{plot}\PYG{p}{(}\PYG{n}{streamlines\PYGZus{}x\PYGZus{}plot\PYGZus{}1\PYGZus{}6}\PYG{p}{,}\PYG{n}{streamlines\PYGZus{}y\PYGZus{}plot\PYGZus{}1\PYGZus{}6\PYGZus{}mirror}\PYG{p}{,} \PYG{l+s+s1}{\PYGZsq{}}\PYG{l+s+s1}{b}\PYG{l+s+s1}{\PYGZsq{}}\PYG{p}{)}


    \PYG{n}{ax1}\PYG{o}{.}\PYG{n}{set}\PYG{p}{(}\PYG{n}{xlabel}\PYG{o}{=}\PYG{l+s+s1}{\PYGZsq{}}\PYG{l+s+s1}{x [m]}\PYG{l+s+s1}{\PYGZsq{}}\PYG{p}{,} \PYG{n}{ylabel} \PYG{o}{=}\PYG{l+s+s1}{\PYGZsq{}}\PYG{l+s+s1}{y [m]}\PYG{l+s+s1}{\PYGZsq{}}\PYG{p}{,} \PYG{n}{xlim} \PYG{o}{=} \PYG{p}{[}\PYG{o}{\PYGZhy{}}\PYG{l+m+mi}{175}\PYG{p}{,} \PYG{l+m+mi}{225}\PYG{p}{]}\PYG{p}{,} \PYG{n}{ylim} \PYG{o}{=} \PYG{p}{[}\PYG{o}{\PYGZhy{}}\PYG{l+m+mi}{175}\PYG{p}{,}\PYG{l+m+mi}{175}\PYG{p}{]}\PYG{p}{)}
    \PYG{n}{fig}\PYG{o}{.}\PYG{n}{savefig}\PYG{p}{(}\PYG{l+s+s2}{\PYGZdq{}}\PYG{l+s+s2}{isolines.png}\PYG{l+s+s2}{\PYGZdq{}}\PYG{p}{,} \PYG{n}{dpi}\PYG{o}{=}\PYG{l+m+mi}{300}\PYG{p}{)}

    \PYG{n}{ax2}\PYG{o}{.}\PYG{n}{plot}\PYG{p}{(}\PYG{n}{isochrones\PYGZus{}x\PYGZus{}plot\PYGZus{}t1}\PYG{p}{,} \PYG{n}{isochrones\PYGZus{}y\PYGZus{}plot\PYGZus{}t1}\PYG{p}{,} \PYG{l+s+s1}{\PYGZsq{}}\PYG{l+s+s1}{g}\PYG{l+s+s1}{\PYGZsq{}}\PYG{p}{)}
    \PYG{n}{ax2}\PYG{o}{.}\PYG{n}{plot}\PYG{p}{(}\PYG{n}{isochrones\PYGZus{}x\PYGZus{}plot\PYGZus{}t2}\PYG{p}{,} \PYG{n}{isochrones\PYGZus{}y\PYGZus{}plot\PYGZus{}t2}\PYG{p}{,} \PYG{l+s+s1}{\PYGZsq{}}\PYG{l+s+s1}{g}\PYG{l+s+s1}{\PYGZsq{}}\PYG{p}{)}
    \PYG{n}{ax2}\PYG{o}{.}\PYG{n}{plot}\PYG{p}{(}\PYG{n}{isochrones\PYGZus{}x\PYGZus{}plot\PYGZus{}t3}\PYG{p}{,} \PYG{n}{isochrones\PYGZus{}y\PYGZus{}plot\PYGZus{}t3}\PYG{p}{,} \PYG{l+s+s1}{\PYGZsq{}}\PYG{l+s+s1}{g}\PYG{l+s+s1}{\PYGZsq{}}\PYG{p}{)}

    \PYG{n}{ax2}\PYG{o}{.}\PYG{n}{plot}\PYG{p}{(}\PYG{n}{isochrones\PYGZus{}x\PYGZus{}plot\PYGZus{}t1}\PYG{p}{,} \PYG{n}{isochrones\PYGZus{}y\PYGZus{}plot\PYGZus{}t1\PYGZus{}mirror}\PYG{p}{,} \PYG{l+s+s1}{\PYGZsq{}}\PYG{l+s+s1}{g}\PYG{l+s+s1}{\PYGZsq{}}\PYG{p}{)}
    \PYG{n}{ax2}\PYG{o}{.}\PYG{n}{plot}\PYG{p}{(}\PYG{n}{isochrones\PYGZus{}x\PYGZus{}plot\PYGZus{}t2}\PYG{p}{,} \PYG{n}{isochrones\PYGZus{}y\PYGZus{}plot\PYGZus{}t2\PYGZus{}mirror}\PYG{p}{,} \PYG{l+s+s1}{\PYGZsq{}}\PYG{l+s+s1}{g}\PYG{l+s+s1}{\PYGZsq{}}\PYG{p}{)}
    \PYG{n}{ax2}\PYG{o}{.}\PYG{n}{plot}\PYG{p}{(}\PYG{n}{isochrones\PYGZus{}x\PYGZus{}plot\PYGZus{}t3}\PYG{p}{,} \PYG{n}{isochrones\PYGZus{}y\PYGZus{}plot\PYGZus{}t3\PYGZus{}mirror}\PYG{p}{,} \PYG{l+s+s1}{\PYGZsq{}}\PYG{l+s+s1}{g}\PYG{l+s+s1}{\PYGZsq{}}\PYG{p}{)}
    
    \PYG{n}{ax2}\PYG{o}{.}\PYG{n}{plot}\PYG{p}{(}\PYG{n}{streamlines\PYGZus{}x\PYGZus{}plot\PYGZus{}0}\PYG{p}{,}\PYG{n}{streamlines\PYGZus{}y\PYGZus{}plot\PYGZus{}0}\PYG{p}{,} \PYG{l+s+s1}{\PYGZsq{}}\PYG{l+s+s1}{b}\PYG{l+s+s1}{\PYGZsq{}}\PYG{p}{)}
    \PYG{n}{ax2}\PYG{o}{.}\PYG{n}{plot}\PYG{p}{(}\PYG{n}{streamlines\PYGZus{}x\PYGZus{}plot\PYGZus{}0\PYGZus{}2}\PYG{p}{,}\PYG{n}{streamlines\PYGZus{}y\PYGZus{}plot\PYGZus{}0\PYGZus{}2}\PYG{p}{,} \PYG{l+s+s1}{\PYGZsq{}}\PYG{l+s+s1}{b}\PYG{l+s+s1}{\PYGZsq{}}\PYG{p}{)}
    \PYG{n}{ax2}\PYG{o}{.}\PYG{n}{plot}\PYG{p}{(}\PYG{n}{streamlines\PYGZus{}x\PYGZus{}plot\PYGZus{}0\PYGZus{}4}\PYG{p}{,}\PYG{n}{streamlines\PYGZus{}y\PYGZus{}plot\PYGZus{}0\PYGZus{}4}\PYG{p}{,} \PYG{l+s+s1}{\PYGZsq{}}\PYG{l+s+s1}{b}\PYG{l+s+s1}{\PYGZsq{}}\PYG{p}{)}
    \PYG{n}{ax2}\PYG{o}{.}\PYG{n}{plot}\PYG{p}{(}\PYG{n}{streamlines\PYGZus{}x\PYGZus{}plot\PYGZus{}0\PYGZus{}6}\PYG{p}{,}\PYG{n}{streamlines\PYGZus{}y\PYGZus{}plot\PYGZus{}0\PYGZus{}6}\PYG{p}{,} \PYG{l+s+s1}{\PYGZsq{}}\PYG{l+s+s1}{b}\PYG{l+s+s1}{\PYGZsq{}}\PYG{p}{)}
    \PYG{n}{ax2}\PYG{o}{.}\PYG{n}{plot}\PYG{p}{(}\PYG{n}{streamlines\PYGZus{}x\PYGZus{}plot\PYGZus{}0\PYGZus{}8}\PYG{p}{,}\PYG{n}{streamlines\PYGZus{}y\PYGZus{}plot\PYGZus{}0\PYGZus{}8}\PYG{p}{,} \PYG{l+s+s1}{\PYGZsq{}}\PYG{l+s+s1}{b}\PYG{l+s+s1}{\PYGZsq{}}\PYG{p}{)}
    \PYG{n}{ax2}\PYG{o}{.}\PYG{n}{plot}\PYG{p}{(}\PYG{n}{streamlines\PYGZus{}x\PYGZus{}plot\PYGZus{}1}\PYG{p}{,}\PYG{n}{streamlines\PYGZus{}y\PYGZus{}plot\PYGZus{}1}\PYG{p}{,} \PYG{n}{color} \PYG{o}{=} \PYG{l+s+s1}{\PYGZsq{}}\PYG{l+s+s1}{black}\PYG{l+s+s1}{\PYGZsq{}}\PYG{p}{)}
    \PYG{n}{ax2}\PYG{o}{.}\PYG{n}{plot}\PYG{p}{(}\PYG{n}{streamlines\PYGZus{}x\PYGZus{}plot\PYGZus{}1\PYGZus{}2}\PYG{p}{,}\PYG{n}{streamlines\PYGZus{}y\PYGZus{}plot\PYGZus{}1\PYGZus{}2}\PYG{p}{,} \PYG{l+s+s1}{\PYGZsq{}}\PYG{l+s+s1}{b}\PYG{l+s+s1}{\PYGZsq{}}\PYG{p}{)}
    \PYG{n}{ax2}\PYG{o}{.}\PYG{n}{plot}\PYG{p}{(}\PYG{n}{streamlines\PYGZus{}x\PYGZus{}plot\PYGZus{}1\PYGZus{}4}\PYG{p}{,}\PYG{n}{streamlines\PYGZus{}y\PYGZus{}plot\PYGZus{}1\PYGZus{}4}\PYG{p}{,} \PYG{l+s+s1}{\PYGZsq{}}\PYG{l+s+s1}{b}\PYG{l+s+s1}{\PYGZsq{}}\PYG{p}{)}
    \PYG{n}{ax2}\PYG{o}{.}\PYG{n}{plot}\PYG{p}{(}\PYG{n}{streamlines\PYGZus{}x\PYGZus{}plot\PYGZus{}1\PYGZus{}6}\PYG{p}{,}\PYG{n}{streamlines\PYGZus{}y\PYGZus{}plot\PYGZus{}1\PYGZus{}6}\PYG{p}{,} \PYG{l+s+s1}{\PYGZsq{}}\PYG{l+s+s1}{b}\PYG{l+s+s1}{\PYGZsq{}}\PYG{p}{)}
    
    
    \PYG{n}{ax2}\PYG{o}{.}\PYG{n}{plot}\PYG{p}{(}\PYG{n}{streamlines\PYGZus{}x\PYGZus{}plot\PYGZus{}0\PYGZus{}2}\PYG{p}{,}\PYG{n}{streamlines\PYGZus{}y\PYGZus{}plot\PYGZus{}0\PYGZus{}2\PYGZus{}mirror}\PYG{p}{,} \PYG{l+s+s1}{\PYGZsq{}}\PYG{l+s+s1}{b}\PYG{l+s+s1}{\PYGZsq{}}\PYG{p}{)}
    \PYG{n}{ax2}\PYG{o}{.}\PYG{n}{plot}\PYG{p}{(}\PYG{n}{streamlines\PYGZus{}x\PYGZus{}plot\PYGZus{}0\PYGZus{}4}\PYG{p}{,}\PYG{n}{streamlines\PYGZus{}y\PYGZus{}plot\PYGZus{}0\PYGZus{}4\PYGZus{}mirror}\PYG{p}{,} \PYG{l+s+s1}{\PYGZsq{}}\PYG{l+s+s1}{b}\PYG{l+s+s1}{\PYGZsq{}}\PYG{p}{)}
    \PYG{n}{ax2}\PYG{o}{.}\PYG{n}{plot}\PYG{p}{(}\PYG{n}{streamlines\PYGZus{}x\PYGZus{}plot\PYGZus{}0\PYGZus{}6}\PYG{p}{,}\PYG{n}{streamlines\PYGZus{}y\PYGZus{}plot\PYGZus{}0\PYGZus{}6\PYGZus{}mirror}\PYG{p}{,} \PYG{l+s+s1}{\PYGZsq{}}\PYG{l+s+s1}{b}\PYG{l+s+s1}{\PYGZsq{}}\PYG{p}{)}
    \PYG{n}{ax2}\PYG{o}{.}\PYG{n}{plot}\PYG{p}{(}\PYG{n}{streamlines\PYGZus{}x\PYGZus{}plot\PYGZus{}0\PYGZus{}8}\PYG{p}{,}\PYG{n}{streamlines\PYGZus{}y\PYGZus{}plot\PYGZus{}0\PYGZus{}8\PYGZus{}mirror}\PYG{p}{,} \PYG{l+s+s1}{\PYGZsq{}}\PYG{l+s+s1}{b}\PYG{l+s+s1}{\PYGZsq{}}\PYG{p}{)}
    \PYG{n}{ax2}\PYG{o}{.}\PYG{n}{plot}\PYG{p}{(}\PYG{n}{streamlines\PYGZus{}x\PYGZus{}plot\PYGZus{}1}\PYG{p}{,}\PYG{n}{streamlines\PYGZus{}y\PYGZus{}plot\PYGZus{}1\PYGZus{}mirror}\PYG{p}{,} \PYG{n}{color} \PYG{o}{=} \PYG{l+s+s1}{\PYGZsq{}}\PYG{l+s+s1}{black}\PYG{l+s+s1}{\PYGZsq{}}\PYG{p}{)}
    \PYG{n}{ax2}\PYG{o}{.}\PYG{n}{plot}\PYG{p}{(}\PYG{n}{streamlines\PYGZus{}x\PYGZus{}plot\PYGZus{}1\PYGZus{}2}\PYG{p}{,}\PYG{n}{streamlines\PYGZus{}y\PYGZus{}plot\PYGZus{}1\PYGZus{}2\PYGZus{}mirror}\PYG{p}{,} \PYG{l+s+s1}{\PYGZsq{}}\PYG{l+s+s1}{b}\PYG{l+s+s1}{\PYGZsq{}}\PYG{p}{)}
    \PYG{n}{ax2}\PYG{o}{.}\PYG{n}{plot}\PYG{p}{(}\PYG{n}{streamlines\PYGZus{}x\PYGZus{}plot\PYGZus{}1\PYGZus{}4}\PYG{p}{,}\PYG{n}{streamlines\PYGZus{}y\PYGZus{}plot\PYGZus{}1\PYGZus{}4\PYGZus{}mirror}\PYG{p}{,} \PYG{l+s+s1}{\PYGZsq{}}\PYG{l+s+s1}{b}\PYG{l+s+s1}{\PYGZsq{}}\PYG{p}{)}
    \PYG{n}{ax2}\PYG{o}{.}\PYG{n}{plot}\PYG{p}{(}\PYG{n}{streamlines\PYGZus{}x\PYGZus{}plot\PYGZus{}1\PYGZus{}6}\PYG{p}{,}\PYG{n}{streamlines\PYGZus{}y\PYGZus{}plot\PYGZus{}1\PYGZus{}6\PYGZus{}mirror}\PYG{p}{,} \PYG{l+s+s1}{\PYGZsq{}}\PYG{l+s+s1}{b}\PYG{l+s+s1}{\PYGZsq{}}\PYG{p}{)}
    
    

    \PYG{n}{ax2}\PYG{o}{.}\PYG{n}{set}\PYG{p}{(}\PYG{n}{xlabel}\PYG{o}{=}\PYG{l+s+s1}{\PYGZsq{}}\PYG{l+s+s1}{x [m]}\PYG{l+s+s1}{\PYGZsq{}}\PYG{p}{,} \PYG{n}{ylabel} \PYG{o}{=}\PYG{l+s+s1}{\PYGZsq{}}\PYG{l+s+s1}{y [m]}\PYG{l+s+s1}{\PYGZsq{}}\PYG{p}{,} \PYG{n}{xlim} \PYG{o}{=} \PYG{p}{[}\PYG{o}{\PYGZhy{}}\PYG{l+m+mi}{175}\PYG{p}{,} \PYG{l+m+mi}{225}\PYG{p}{]}\PYG{p}{,} \PYG{n}{ylim} \PYG{o}{=} \PYG{p}{[}\PYG{o}{\PYGZhy{}}\PYG{l+m+mi}{175}\PYG{p}{,}\PYG{l+m+mi}{175}\PYG{p}{]}\PYG{p}{)}
    \PYG{n}{fig}\PYG{o}{.}\PYG{n}{savefig}\PYG{p}{(}\PYG{l+s+s2}{\PYGZdq{}}\PYG{l+s+s2}{isochrones.png}\PYG{l+s+s2}{\PYGZdq{}}\PYG{p}{,} \PYG{n}{dpi}\PYG{o}{=}\PYG{l+m+mi}{300}\PYG{p}{)}

\PYG{n}{interact}\PYG{p}{(}\PYG{n}{uniform\PYGZus{}flow}\PYG{p}{,}
         \PYG{n}{K}\PYG{o}{=}\PYG{n}{widgets}\PYG{o}{.}\PYG{n}{FloatLogSlider}\PYG{p}{(}\PYG{n}{value}\PYG{o}{=}\PYG{l+m+mf}{3e\PYGZhy{}4}\PYG{p}{,} \PYG{n}{base}\PYG{o}{=}\PYG{l+m+mi}{10}\PYG{p}{,} \PYG{n+nb}{min}\PYG{o}{=}\PYG{o}{\PYGZhy{}}\PYG{l+m+mi}{10}\PYG{p}{,} \PYG{n+nb}{max}\PYG{o}{=}\PYG{l+m+mi}{0}\PYG{p}{,} \PYG{n}{step}\PYG{o}{=}\PYG{l+m+mf}{0.1}\PYG{p}{,} \PYG{n}{description}\PYG{o}{=}\PYG{l+s+s1}{\PYGZsq{}}\PYG{l+s+s1}{hydraulic conductivity [m/s]:}\PYG{l+s+s1}{\PYGZsq{}}\PYG{p}{,} \PYG{n}{disabled}\PYG{o}{=}\PYG{k+kc}{False}\PYG{p}{)}\PYG{p}{,}
         \PYG{n}{ne}\PYG{o}{=}\PYG{n}{widgets}\PYG{o}{.}\PYG{n}{FloatSlider}\PYG{p}{(}\PYG{n}{value}\PYG{o}{=}\PYG{l+m+mf}{0.2}\PYG{p}{,} \PYG{n+nb}{min}\PYG{o}{=}\PYG{l+m+mf}{0.001}\PYG{p}{,} \PYG{n+nb}{max}\PYG{o}{=}\PYG{l+m+mi}{1}\PYG{p}{,} \PYG{n}{step}\PYG{o}{=}\PYG{l+m+mf}{0.05}\PYG{p}{,} \PYG{n}{description}\PYG{o}{=}\PYG{l+s+s1}{\PYGZsq{}}\PYG{l+s+s1}{effective porosity [\PYGZhy{}]:}\PYG{l+s+s1}{\PYGZsq{}}\PYG{p}{,} \PYG{n}{disabled}\PYG{o}{=}\PYG{k+kc}{False}\PYG{p}{)}\PYG{p}{,}
         \PYG{n}{m}\PYG{o}{=} \PYG{n}{widgets}\PYG{o}{.}\PYG{n}{FloatSlider}\PYG{p}{(}\PYG{n}{value}\PYG{o}{=}\PYG{l+m+mi}{7}\PYG{p}{,}\PYG{n+nb}{min}\PYG{o}{=}\PYG{l+m+mi}{0}\PYG{p}{,} \PYG{n+nb}{max}\PYG{o}{=}\PYG{l+m+mi}{30}\PYG{p}{,}\PYG{n}{step}\PYG{o}{=}\PYG{l+m+mi}{1}\PYG{p}{,} \PYG{n}{description}\PYG{o}{=}\PYG{l+s+s1}{\PYGZsq{}}\PYG{l+s+s1}{thickness [m]:}\PYG{l+s+s1}{\PYGZsq{}} \PYG{p}{,} \PYG{n}{disabled}\PYG{o}{=}\PYG{k+kc}{False}\PYG{p}{)}\PYG{p}{,}
         \PYG{n}{v}\PYG{o}{=}\PYG{n}{widgets}\PYG{o}{.}\PYG{n}{FloatSlider}\PYG{p}{(}\PYG{n}{value}\PYG{o}{=}\PYG{l+m+mf}{0.4}\PYG{p}{,} \PYG{n+nb}{min}\PYG{o}{=}\PYG{l+m+mf}{0.00001}\PYG{p}{,} \PYG{n+nb}{max}\PYG{o}{=}\PYG{l+m+mi}{5}\PYG{p}{,} \PYG{n}{step}\PYG{o}{=}\PYG{l+m+mf}{0.1}\PYG{p}{,} \PYG{n}{description}\PYG{o}{=}\PYG{l+s+s1}{\PYGZsq{}}\PYG{l+s+s1}{uniform velocity [m/d]:}\PYG{l+s+s1}{\PYGZsq{}}\PYG{p}{,} \PYG{n}{disabled}\PYG{o}{=}\PYG{k+kc}{False}\PYG{p}{)}\PYG{p}{,}
         \PYG{n}{Q}\PYG{o}{=}\PYG{n}{widgets}\PYG{o}{.}\PYG{n}{FloatSlider}\PYG{p}{(}\PYG{n}{value}\PYG{o}{=}\PYG{l+m+mi}{800}\PYG{p}{,} \PYG{n+nb}{min}\PYG{o}{=}\PYG{l+m+mi}{100}\PYG{p}{,} \PYG{n+nb}{max}\PYG{o}{=}\PYG{l+m+mi}{1000}\PYG{p}{,} \PYG{n}{step}\PYG{o}{=}\PYG{l+m+mf}{0.1}\PYG{p}{,} \PYG{n}{description}\PYG{o}{=}\PYG{l+s+s1}{\PYGZsq{}}\PYG{l+s+s1}{pumping rate [m\PYGZca{}3/d]:}\PYG{l+s+s1}{\PYGZsq{}}\PYG{p}{,} \PYG{n}{disabled}\PYG{o}{=}\PYG{k+kc}{False}\PYG{p}{)}\PYG{p}{,}
         \PYG{n}{t1}\PYG{o}{=}\PYG{n}{widgets}\PYG{o}{.}\PYG{n}{BoundedFloatText}\PYG{p}{(}\PYG{n}{value}\PYG{o}{=}\PYG{l+m+mi}{10}\PYG{p}{,} \PYG{n+nb}{min}\PYG{o}{=}\PYG{l+m+mi}{1}\PYG{p}{,} \PYG{n+nb}{max}\PYG{o}{=}\PYG{l+m+mi}{100}\PYG{p}{,} \PYG{n}{step}\PYG{o}{=}\PYG{l+m+mf}{0.1}\PYG{p}{,} \PYG{n}{description}\PYG{o}{=}\PYG{l+s+s1}{\PYGZsq{}}\PYG{l+s+s1}{t1 [d]:}\PYG{l+s+s1}{\PYGZsq{}}\PYG{p}{,} \PYG{n}{disabled}\PYG{o}{=}\PYG{k+kc}{False}\PYG{p}{)}\PYG{p}{,}
         \PYG{n}{t2}\PYG{o}{=}\PYG{n}{widgets}\PYG{o}{.}\PYG{n}{BoundedFloatText}\PYG{p}{(}\PYG{n}{value}\PYG{o}{=}\PYG{l+m+mi}{30}\PYG{p}{,} \PYG{n+nb}{min}\PYG{o}{=}\PYG{l+m+mi}{1}\PYG{p}{,} \PYG{n+nb}{max}\PYG{o}{=}\PYG{l+m+mi}{100}\PYG{p}{,} \PYG{n}{step}\PYG{o}{=}\PYG{l+m+mf}{0.1}\PYG{p}{,} \PYG{n}{description}\PYG{o}{=}\PYG{l+s+s1}{\PYGZsq{}}\PYG{l+s+s1}{t2 [d]:}\PYG{l+s+s1}{\PYGZsq{}}\PYG{p}{,} \PYG{n}{disabled}\PYG{o}{=}\PYG{k+kc}{False}\PYG{p}{)}\PYG{p}{,}
         \PYG{n}{t3}\PYG{o}{=}\PYG{n}{widgets}\PYG{o}{.}\PYG{n}{BoundedFloatText}\PYG{p}{(}\PYG{n}{value}\PYG{o}{=}\PYG{l+m+mi}{50}\PYG{p}{,} \PYG{n+nb}{min}\PYG{o}{=}\PYG{l+m+mi}{1}\PYG{p}{,} \PYG{n+nb}{max}\PYG{o}{=}\PYG{l+m+mi}{100}\PYG{p}{,} \PYG{n}{step}\PYG{o}{=}\PYG{l+m+mf}{0.1}\PYG{p}{,} \PYG{n}{description}\PYG{o}{=}\PYG{l+s+s1}{\PYGZsq{}}\PYG{l+s+s1}{t3 [d]:}\PYG{l+s+s1}{\PYGZsq{}}\PYG{p}{,} \PYG{n}{disabled}\PYG{o}{=}\PYG{k+kc}{False}\PYG{p}{)}\PYG{p}{,}
         \PYG{p}{)}\PYG{p}{;}
         
\end{sphinxVerbatim}

\end{sphinxuseclass}\end{sphinxVerbatimInput}
\begin{sphinxVerbatimOutput}

\begin{sphinxuseclass}{cell_output}
\begin{sphinxVerbatim}[commandchars=\\\{\}]
interactive(children=(FloatLogSlider(value=0.0003, description=\PYGZsq{}hydraulic conductivity [m/s]:\PYGZsq{}, max=0.0, min=\PYGZhy{}…
\end{sphinxVerbatim}

\end{sphinxuseclass}\end{sphinxVerbatimOutput}

\end{sphinxuseclass}
\sphinxstepscope


\chapter{Simulating the Anisotropy and flow direction}
\label{\detokenize{content/tools/aniso2D:simulating-the-anisotropy-and-flow-direction}}\label{\detokenize{content/tools/aniso2D::doc}}

\section{How to use the tool?}
\label{\detokenize{content/tools/aniso2D:how-to-use-the-tool}}\begin{enumerate}
\sphinxsetlistlabels{\arabic}{enumi}{enumii}{}{.}%
\item {} 
\sphinxAtStartPar
Go to the Binder by clicking the rocket button (top\sphinxhyphen{}right of the page)

\item {} 
\sphinxAtStartPar
Execute the code cell

\item {} 
\sphinxAtStartPar
Change the values of different quantities in the box.

\item {} 
\sphinxAtStartPar
For re\sphinxhyphen{}simulations \sphinxhyphen{} changes the input values in the boxes.

\end{enumerate}

\sphinxAtStartPar
This tool can also be downloaded and run locally. For that download the \sphinxstyleemphasis{aniso2D.ipynb} file and execute the process in any editor (e.g., JUPYTER notebook, JUPYTER lab) that is able to read and execute this file\sphinxhyphen{}type.

\sphinxAtStartPar
The code may also be executed in the book page.

\sphinxAtStartPar
The codes are licensed under CC by 4.0 \sphinxhref{https://creativecommons.org/licenses/by/4.0/deed.en}{(use anyways, but acknowledge the original work)}

\begin{sphinxuseclass}{cell}\begin{sphinxVerbatimInput}

\begin{sphinxuseclass}{cell_input}
\begin{sphinxVerbatim}[commandchars=\\\{\}]
\PYG{c+c1}{\PYGZsh{} The library used }

\PYG{k+kn}{import} \PYG{n+nn}{numpy} \PYG{k}{as} \PYG{n+nn}{np}
\PYG{k+kn}{import} \PYG{n+nn}{matplotlib}\PYG{n+nn}{.}\PYG{n+nn}{pyplot} \PYG{k}{as} \PYG{n+nn}{plt}
\PYG{k+kn}{import} \PYG{n+nn}{pandas} \PYG{k}{as} \PYG{n+nn}{pd}
\PYG{k+kn}{from} \PYG{n+nn}{ipywidgets} \PYG{k+kn}{import} \PYG{n}{widgets}\PYG{p}{,} \PYG{n}{interactive}



\PYG{c+c1}{\PYGZsh{} the main programme}
\PYG{k}{def} \PYG{n+nf}{aniso}\PYG{p}{(}\PYG{n}{a\PYGZus{}d}\PYG{p}{,} \PYG{n}{ani\PYGZus{}r}\PYG{p}{)}\PYG{p}{:}

\PYG{c+c1}{\PYGZsh{} interim calculation}
    \PYG{n}{a\PYGZus{}r} \PYG{o}{=} \PYG{n}{a\PYGZus{}d}\PYG{o}{*}\PYG{n}{np}\PYG{o}{.}\PYG{n}{pi}\PYG{o}{/}\PYG{l+m+mi}{180}

    \PYG{n}{i\PYGZus{}xr} \PYG{o}{=} \PYG{n}{np}\PYG{o}{.}\PYG{n}{cos}\PYG{p}{(}\PYG{n}{a\PYGZus{}r}\PYG{p}{)} \PYG{c+c1}{\PYGZsh{} (\PYGZhy{}), rel. hyd grad. along x}
    \PYG{n}{i\PYGZus{}zr} \PYG{o}{=} \PYG{n}{np}\PYG{o}{.}\PYG{n}{sin}\PYG{p}{(}\PYG{n}{a\PYGZus{}r}\PYG{p}{)} \PYG{c+c1}{\PYGZsh{} (\PYGZhy{}), rel. hyd grad. along z}
    \PYG{n}{K\PYGZus{}h} \PYG{o}{=} \PYG{l+m+mi}{1} \PYG{c+c1}{\PYGZsh{} (\PYGZhy{}), m/s K\PYGZus{}h}
    \PYG{n}{K\PYGZus{}v} \PYG{o}{=} \PYG{l+m+mi}{1}\PYG{o}{/}\PYG{n}{ani\PYGZus{}r} \PYG{c+c1}{\PYGZsh{} m/s, rel K\PYGZus{}v}
    \PYG{n}{f\PYGZus{}x} \PYG{o}{=} \PYG{o}{\PYGZhy{}}\PYG{n}{i\PYGZus{}xr}\PYG{o}{*}\PYG{n}{K\PYGZus{}h} \PYG{c+c1}{\PYGZsh{} m/s}
    \PYG{n}{f\PYGZus{}z} \PYG{o}{=} \PYG{o}{\PYGZhy{}}\PYG{n}{i\PYGZus{}zr}\PYG{o}{*}\PYG{n}{K\PYGZus{}v} \PYG{c+c1}{\PYGZsh{} m/s}
    \PYG{n}{f\PYGZus{}m} \PYG{o}{=} \PYG{n}{np}\PYG{o}{.}\PYG{n}{sqrt}\PYG{p}{(}\PYG{n}{f\PYGZus{}x}\PYG{o}{*}\PYG{n}{f\PYGZus{}x}\PYG{o}{+}\PYG{n}{f\PYGZus{}z}\PYG{o}{*}\PYG{n}{f\PYGZus{}z}\PYG{p}{)} \PYG{c+c1}{\PYGZsh{} m/s}

    \PYG{n}{args} \PYG{o}{=} \PYG{p}{(}\PYG{n}{K\PYGZus{}h}\PYG{o}{*}\PYG{n}{i\PYGZus{}xr}\PYG{o}{*}\PYG{n}{i\PYGZus{}xr} \PYG{o}{+} \PYG{n}{K\PYGZus{}v}\PYG{o}{*}\PYG{n}{i\PYGZus{}zr}\PYG{o}{*}\PYG{n}{i\PYGZus{}zr}\PYG{p}{)}\PYG{o}{/}\PYG{n}{f\PYGZus{}m}
    \PYG{n}{an\PYGZus{}i\PYGZus{}f} \PYG{o}{=} \PYG{p}{(}\PYG{p}{(}\PYG{n}{np}\PYG{o}{.}\PYG{n}{pi}\PYG{o}{\PYGZhy{}}\PYG{n}{np}\PYG{o}{.}\PYG{n}{arccos}\PYG{p}{(}\PYG{n}{args}\PYG{p}{)}\PYG{p}{)}\PYG{o}{*}\PYG{l+m+mi}{180}\PYG{o}{/}\PYG{n}{np}\PYG{o}{.}\PYG{n}{pi}\PYG{p}{)}\PYG{c+c1}{\PYGZsh{} deg,}

\PYG{c+c1}{\PYGZsh{} plots axes}

    \PYG{n}{fig}\PYG{p}{,} \PYG{p}{(}\PYG{n}{ax1}\PYG{p}{,} \PYG{n}{ax2}\PYG{p}{)} \PYG{o}{=} \PYG{n}{plt}\PYG{o}{.}\PYG{n}{subplots}\PYG{p}{(}\PYG{l+m+mi}{1}\PYG{p}{,} \PYG{l+m+mi}{2}\PYG{p}{,}\PYG{n}{figsize}\PYG{o}{=}\PYG{p}{(}\PYG{l+m+mi}{10}\PYG{p}{,}\PYG{l+m+mi}{6}\PYG{p}{)}\PYG{p}{,} \PYG{n}{gridspec\PYGZus{}kw}\PYG{o}{=}\PYG{p}{\PYGZob{}}\PYG{l+s+s1}{\PYGZsq{}}\PYG{l+s+s1}{width\PYGZus{}ratios}\PYG{l+s+s1}{\PYGZsq{}}\PYG{p}{:} \PYG{p}{[}\PYG{l+m+mi}{3}\PYG{p}{,} \PYG{l+m+mi}{1}\PYG{p}{]}\PYG{p}{\PYGZcb{}}\PYG{p}{)}

\PYG{c+c1}{\PYGZsh{} points for gradient and flux}

    \PYG{n}{grad\PYGZus{}px} \PYG{o}{=} \PYG{p}{[}\PYG{l+m+mi}{0}\PYG{p}{,} \PYG{n}{i\PYGZus{}xr}\PYG{p}{]}\PYG{c+c1}{\PYGZsh{}i\PYGZus{}xr}
    \PYG{n}{grad\PYGZus{}pz} \PYG{o}{=} \PYG{p}{[}\PYG{l+m+mi}{0}\PYG{p}{,} \PYG{n}{i\PYGZus{}zr}\PYG{p}{]}\PYG{c+c1}{\PYGZsh{}i\PYGZus{}zr}

    \PYG{n}{flux\PYGZus{}px} \PYG{o}{=} \PYG{p}{[}\PYG{l+m+mi}{0}\PYG{p}{,} \PYG{n}{f\PYGZus{}x}\PYG{p}{]}
    \PYG{n}{flux\PYGZus{}pz} \PYG{o}{=} \PYG{p}{[}\PYG{l+m+mi}{0}\PYG{p}{,} \PYG{n}{f\PYGZus{}z}\PYG{p}{]}

\PYG{c+c1}{\PYGZsh{} creating points for intersect lines (5 of them)}

    \PYG{n}{p1}\PYG{o}{=}\PYG{o}{\PYGZhy{}}\PYG{l+m+mf}{0.5}\PYG{o}{*}\PYG{n}{i\PYGZus{}zr}\PYG{p}{;} \PYG{n}{p2} \PYG{o}{=} \PYG{l+m+mf}{0.5}\PYG{o}{*}\PYG{n}{i\PYGZus{}zr}\PYG{p}{;} \PYG{n}{p3} \PYG{o}{=}\PYG{o}{\PYGZhy{}}\PYG{l+m+mf}{0.5}\PYG{o}{*}\PYG{n}{i\PYGZus{}zr}\PYG{o}{+}\PYG{l+m+mf}{0.35}\PYG{o}{*}\PYG{n}{i\PYGZus{}xr}\PYG{p}{;} \PYG{n}{p4} \PYG{o}{=} \PYG{l+m+mf}{0.5}\PYG{o}{*}\PYG{n}{i\PYGZus{}zr}\PYG{o}{+}\PYG{l+m+mf}{0.35}\PYG{o}{*}\PYG{n}{i\PYGZus{}xr}\PYG{p}{;} \PYG{n}{p5} \PYG{o}{=} \PYG{o}{\PYGZhy{}}\PYG{l+m+mf}{0.5}\PYG{o}{*}\PYG{n}{i\PYGZus{}zr}\PYG{o}{+}\PYG{l+m+mf}{0.7}\PYG{o}{*}\PYG{n}{i\PYGZus{}xr}
    \PYG{n}{p6} \PYG{o}{=} \PYG{l+m+mf}{0.5}\PYG{o}{*}\PYG{n}{i\PYGZus{}zr}\PYG{o}{+}\PYG{l+m+mf}{0.7}\PYG{o}{*}\PYG{n}{i\PYGZus{}xr}\PYG{p}{;}\PYG{n}{p7} \PYG{o}{=} \PYG{o}{\PYGZhy{}}\PYG{l+m+mf}{0.5}\PYG{o}{*}\PYG{n}{i\PYGZus{}zr}\PYG{o}{\PYGZhy{}}\PYG{l+m+mf}{0.35}\PYG{o}{*}\PYG{n}{i\PYGZus{}xr}\PYG{p}{;} \PYG{n}{p8} \PYG{o}{=} \PYG{l+m+mf}{0.5}\PYG{o}{*}\PYG{n}{i\PYGZus{}zr}\PYG{o}{\PYGZhy{}}\PYG{l+m+mf}{0.35}\PYG{o}{*}\PYG{n}{i\PYGZus{}xr}\PYG{p}{;} \PYG{n}{p9} \PYG{o}{=} \PYG{o}{\PYGZhy{}}\PYG{l+m+mf}{0.5}\PYG{o}{*}\PYG{n}{i\PYGZus{}zr}\PYG{o}{\PYGZhy{}}\PYG{l+m+mf}{0.7}\PYG{o}{*}\PYG{n}{i\PYGZus{}xr}\PYG{p}{;}\PYG{n}{p10} \PYG{o}{=}\PYG{l+m+mf}{0.5}\PYG{o}{*}\PYG{n}{i\PYGZus{}zr}\PYG{o}{\PYGZhy{}}\PYG{l+m+mf}{0.7}\PYG{o}{*}\PYG{n}{i\PYGZus{}xr}

    \PYG{n}{q1}\PYG{o}{=}\PYG{l+m+mf}{0.5}\PYG{o}{*}\PYG{n}{i\PYGZus{}xr}\PYG{p}{;} \PYG{n}{q2} \PYG{o}{=}\PYG{o}{\PYGZhy{}} \PYG{l+m+mf}{0.5}\PYG{o}{*}\PYG{n}{i\PYGZus{}xr}\PYG{p}{;} \PYG{n}{q3} \PYG{o}{=}\PYG{l+m+mf}{0.5}\PYG{o}{*}\PYG{n}{i\PYGZus{}xr}\PYG{o}{+}\PYG{l+m+mf}{0.35}\PYG{o}{*}\PYG{n}{i\PYGZus{}zr}\PYG{p}{;} \PYG{n}{q4} \PYG{o}{=} \PYG{o}{\PYGZhy{}}\PYG{l+m+mf}{0.5}\PYG{o}{*}\PYG{n}{i\PYGZus{}xr}\PYG{o}{+}\PYG{l+m+mf}{0.35}\PYG{o}{*}\PYG{n}{i\PYGZus{}zr}\PYG{p}{;} \PYG{n}{q5} \PYG{o}{=} \PYG{l+m+mf}{0.5}\PYG{o}{*}\PYG{n}{i\PYGZus{}xr}\PYG{o}{+}\PYG{l+m+mf}{0.7}\PYG{o}{*}\PYG{n}{i\PYGZus{}zr}
    \PYG{n}{q6} \PYG{o}{=} \PYG{o}{\PYGZhy{}}\PYG{l+m+mf}{0.5}\PYG{o}{*}\PYG{n}{i\PYGZus{}xr}\PYG{o}{+}\PYG{l+m+mf}{0.7}\PYG{o}{*}\PYG{n}{i\PYGZus{}zr}\PYG{p}{;}\PYG{n}{q7} \PYG{o}{=} \PYG{l+m+mf}{0.5}\PYG{o}{*}\PYG{n}{i\PYGZus{}xr}\PYG{o}{\PYGZhy{}}\PYG{l+m+mf}{0.35}\PYG{o}{*}\PYG{n}{i\PYGZus{}zr}\PYG{p}{;}\PYG{n}{q8} \PYG{o}{=} \PYG{o}{\PYGZhy{}}\PYG{l+m+mf}{0.5}\PYG{o}{*}\PYG{n}{i\PYGZus{}xr}\PYG{o}{\PYGZhy{}}\PYG{l+m+mf}{0.35}\PYG{o}{*}\PYG{n}{i\PYGZus{}zr}\PYG{p}{;} \PYG{n}{q9} \PYG{o}{=} \PYG{l+m+mf}{0.5}\PYG{o}{*}\PYG{n}{i\PYGZus{}xr}\PYG{o}{\PYGZhy{}}\PYG{l+m+mf}{0.7}\PYG{o}{*}\PYG{n}{i\PYGZus{}zr}\PYG{p}{;}\PYG{n}{q10} \PYG{o}{=} \PYG{o}{\PYGZhy{}}\PYG{l+m+mf}{0.5}\PYG{o}{*}\PYG{n}{i\PYGZus{}xr}\PYG{o}{\PYGZhy{}}\PYG{l+m+mf}{0.7}\PYG{o}{*}\PYG{n}{i\PYGZus{}zr}

\PYG{c+c1}{\PYGZsh{} plotted points}
    \PYG{n}{l\PYGZus{}1x} \PYG{o}{=}\PYG{p}{[}\PYG{n}{p1}\PYG{p}{,} \PYG{n}{p2}\PYG{p}{]}\PYG{p}{;} \PYG{n}{l\PYGZus{}1y} \PYG{o}{=} \PYG{p}{[}\PYG{n}{q1}\PYG{p}{,} \PYG{n}{q2}\PYG{p}{]}
    \PYG{n}{l\PYGZus{}2x} \PYG{o}{=}\PYG{p}{[}\PYG{n}{p3}\PYG{p}{,} \PYG{n}{p4}\PYG{p}{]}\PYG{p}{;} \PYG{n}{l\PYGZus{}2y} \PYG{o}{=} \PYG{p}{[}\PYG{n}{q3}\PYG{p}{,} \PYG{n}{q4}\PYG{p}{]}
    \PYG{n}{l\PYGZus{}3x} \PYG{o}{=}\PYG{p}{[}\PYG{n}{p5}\PYG{p}{,} \PYG{n}{p6}\PYG{p}{]}\PYG{p}{;} \PYG{n}{l\PYGZus{}3y} \PYG{o}{=} \PYG{p}{[}\PYG{n}{q5}\PYG{p}{,} \PYG{n}{q6}\PYG{p}{]}
    \PYG{n}{l\PYGZus{}4x} \PYG{o}{=}\PYG{p}{[}\PYG{n}{p7}\PYG{p}{,} \PYG{n}{p8}\PYG{p}{]}\PYG{p}{;} \PYG{n}{l\PYGZus{}4y} \PYG{o}{=} \PYG{p}{[}\PYG{n}{q7}\PYG{p}{,} \PYG{n}{q8}\PYG{p}{]}
    \PYG{n}{l\PYGZus{}5x} \PYG{o}{=}\PYG{p}{[}\PYG{n}{p9}\PYG{p}{,} \PYG{n}{p10}\PYG{p}{]}\PYG{p}{;} \PYG{n}{l\PYGZus{}5y} \PYG{o}{=} \PYG{p}{[}\PYG{n}{q9}\PYG{p}{,} \PYG{n}{q10}\PYG{p}{]}


\PYG{c+c1}{\PYGZsh{} creating points for anisotropy}
    \PYG{n}{r1} \PYG{o}{=}\PYG{l+m+mf}{1.05} \PYG{k}{if} \PYG{n}{ani\PYGZus{}r} \PYG{o}{\PYGZgt{}}\PYG{o}{=} \PYG{l+m+mi}{1} \PYG{k}{else} \PYG{l+m+mf}{1.5}\PYG{o}{\PYGZhy{}}\PYG{l+m+mf}{0.45}\PYG{o}{*}\PYG{n}{ani\PYGZus{}r}
    \PYG{n}{r2} \PYG{o}{=} \PYG{l+m+mf}{1.95} \PYG{k}{if} \PYG{n}{ani\PYGZus{}r} \PYG{o}{\PYGZgt{}}\PYG{o}{=} \PYG{l+m+mi}{1} \PYG{k}{else} \PYG{l+m+mf}{1.5}\PYG{o}{+}\PYG{l+m+mf}{0.45}\PYG{o}{*}\PYG{n}{ani\PYGZus{}r}
    \PYG{n}{r3} \PYG{o}{=} \PYG{l+m+mf}{0.5}\PYG{o}{*}\PYG{p}{(}\PYG{n}{r1}\PYG{o}{+}\PYG{n}{r2}\PYG{p}{)}\PYG{p}{;} \PYG{n}{r4} \PYG{o}{=} \PYG{n}{r3}

    \PYG{n}{s1} \PYG{o}{=} \PYG{o}{\PYGZhy{}}\PYG{l+m+mf}{0.5}\PYG{p}{;} \PYG{n}{s2} \PYG{o}{=} \PYG{n}{s1}
    \PYG{n}{s3} \PYG{o}{=} \PYG{o}{\PYGZhy{}}\PYG{l+m+mf}{0.05} \PYG{k}{if} \PYG{n}{ani\PYGZus{}r}\PYG{o}{\PYGZlt{}}\PYG{o}{=}\PYG{l+m+mi}{1} \PYG{k}{else} \PYG{o}{\PYGZhy{}}\PYG{l+m+mf}{0.5}\PYG{o}{+}\PYG{l+m+mf}{0.45}\PYG{o}{/}\PYG{n}{ani\PYGZus{}r}
    \PYG{n}{s4} \PYG{o}{=} \PYG{o}{\PYGZhy{}}\PYG{l+m+mf}{0.95} \PYG{k}{if} \PYG{n}{ani\PYGZus{}r}\PYG{o}{\PYGZlt{}}\PYG{o}{=}\PYG{l+m+mi}{1} \PYG{k}{else} \PYG{o}{\PYGZhy{}}\PYG{l+m+mf}{0.5}\PYG{o}{\PYGZhy{}}\PYG{l+m+mf}{0.45}\PYG{o}{/}\PYG{n}{ani\PYGZus{}r}

\PYG{c+c1}{\PYGZsh{} plotted points}
    \PYG{n}{Iso\PYGZus{}1x} \PYG{o}{=} \PYG{p}{[}\PYG{n}{r1}\PYG{p}{,} \PYG{n}{r2}\PYG{p}{]}\PYG{p}{;} \PYG{n}{Iso\PYGZus{}1y} \PYG{o}{=} \PYG{p}{[}\PYG{n}{s1}\PYG{p}{,} \PYG{n}{s2}\PYG{p}{]}
    \PYG{n}{Iso\PYGZus{}2x} \PYG{o}{=} \PYG{p}{[}\PYG{n}{r3}\PYG{p}{,} \PYG{n}{r4}\PYG{p}{]}\PYG{p}{;} \PYG{n}{Iso\PYGZus{}2y} \PYG{o}{=} \PYG{p}{[}\PYG{n}{s3}\PYG{p}{,} \PYG{n}{s4}\PYG{p}{]}
    \PYG{n}{Iso\PYGZus{}x} \PYG{o}{=}\PYG{p}{[}\PYG{n}{r1}\PYG{p}{,} \PYG{n}{r2}\PYG{p}{,} \PYG{n}{r3}\PYG{p}{,} \PYG{n}{r4}\PYG{p}{]}\PYG{p}{;} \PYG{n}{Iso\PYGZus{}y} \PYG{o}{=} \PYG{p}{[}\PYG{n}{s1}\PYG{p}{,}\PYG{n}{s2}\PYG{p}{,}\PYG{n}{s3}\PYG{p}{,}\PYG{n}{s4}\PYG{p}{]}

\PYG{c+c1}{\PYGZsh{} plotting all points}

\PYG{c+c1}{\PYGZsh{} plotting gradient/flux lines}

    \PYG{n}{ax1}\PYG{o}{.}\PYG{n}{plot}\PYG{p}{(}\PYG{n}{grad\PYGZus{}px}\PYG{p}{,} \PYG{n}{grad\PYGZus{}pz}\PYG{p}{,} \PYG{l+s+s2}{\PYGZdq{}}\PYG{l+s+s2}{g}\PYG{l+s+s2}{\PYGZdq{}}\PYG{p}{,} \PYG{n}{label}\PYG{o}{=}\PYG{l+s+s2}{\PYGZdq{}}\PYG{l+s+s2}{ gradient}\PYG{l+s+s2}{\PYGZdq{}}\PYG{p}{)} \PYG{c+c1}{\PYGZsh{} plotting gradient}
    \PYG{n}{ax1}\PYG{o}{.}\PYG{n}{plot}\PYG{p}{(}\PYG{n}{flux\PYGZus{}px}\PYG{p}{,} \PYG{n}{flux\PYGZus{}pz}\PYG{p}{,} \PYG{l+s+s2}{\PYGZdq{}}\PYG{l+s+s2}{r}\PYG{l+s+s2}{\PYGZdq{}}\PYG{p}{,} \PYG{n}{label}\PYG{o}{=}\PYG{l+s+s2}{\PYGZdq{}}\PYG{l+s+s2}{ flux}\PYG{l+s+s2}{\PYGZdq{}}\PYG{p}{)} \PYG{c+c1}{\PYGZsh{} plotting flux}

\PYG{c+c1}{\PYGZsh{} plotting intersect lines }
    \PYG{n}{ax1}\PYG{o}{.}\PYG{n}{plot}\PYG{p}{(}\PYG{n}{l\PYGZus{}1x}\PYG{p}{,} \PYG{n}{l\PYGZus{}1y}\PYG{p}{,} \PYG{l+s+s2}{\PYGZdq{}}\PYG{l+s+s2}{b}\PYG{l+s+s2}{\PYGZdq{}}\PYG{p}{,} \PYG{n}{label} \PYG{o}{=} \PYG{l+s+s2}{\PYGZdq{}}\PYG{l+s+s2}{head isoline}\PYG{l+s+s2}{\PYGZdq{}}\PYG{p}{)}
    \PYG{n}{ax1}\PYG{o}{.}\PYG{n}{plot}\PYG{p}{(}\PYG{n}{l\PYGZus{}2x}\PYG{p}{,} \PYG{n}{l\PYGZus{}2y}\PYG{p}{,} \PYG{l+s+s2}{\PYGZdq{}}\PYG{l+s+s2}{b}\PYG{l+s+s2}{\PYGZdq{}}\PYG{p}{)}
    \PYG{n}{ax1}\PYG{o}{.}\PYG{n}{plot}\PYG{p}{(}\PYG{n}{l\PYGZus{}3x}\PYG{p}{,} \PYG{n}{l\PYGZus{}3y}\PYG{p}{,} \PYG{l+s+s2}{\PYGZdq{}}\PYG{l+s+s2}{b}\PYG{l+s+s2}{\PYGZdq{}}\PYG{p}{)}
    \PYG{n}{ax1}\PYG{o}{.}\PYG{n}{plot}\PYG{p}{(}\PYG{n}{l\PYGZus{}4x}\PYG{p}{,} \PYG{n}{l\PYGZus{}4y}\PYG{p}{,} \PYG{l+s+s2}{\PYGZdq{}}\PYG{l+s+s2}{b}\PYG{l+s+s2}{\PYGZdq{}}\PYG{p}{)}
    \PYG{n}{ax1}\PYG{o}{.}\PYG{n}{plot}\PYG{p}{(}\PYG{n}{l\PYGZus{}5x}\PYG{p}{,} \PYG{n}{l\PYGZus{}5y}\PYG{p}{,} \PYG{l+s+s2}{\PYGZdq{}}\PYG{l+s+s2}{b}\PYG{l+s+s2}{\PYGZdq{}}\PYG{p}{)}
    \PYG{n}{ax1}\PYG{o}{.}\PYG{n}{legend}\PYG{p}{(}\PYG{p}{)}


    \PYG{n}{ax1}\PYG{o}{.}\PYG{n}{spines}\PYG{p}{[}\PYG{l+s+s1}{\PYGZsq{}}\PYG{l+s+s1}{left}\PYG{l+s+s1}{\PYGZsq{}}\PYG{p}{]}\PYG{o}{.}\PYG{n}{set\PYGZus{}position}\PYG{p}{(}\PYG{l+s+s1}{\PYGZsq{}}\PYG{l+s+s1}{center}\PYG{l+s+s1}{\PYGZsq{}}\PYG{p}{)} \PYG{c+c1}{\PYGZsh{} bring the axis lines in center}
    \PYG{n}{ax1}\PYG{o}{.}\PYG{n}{spines}\PYG{p}{[}\PYG{l+s+s1}{\PYGZsq{}}\PYG{l+s+s1}{bottom}\PYG{l+s+s1}{\PYGZsq{}}\PYG{p}{]}\PYG{o}{.}\PYG{n}{set\PYGZus{}position}\PYG{p}{(}\PYG{l+s+s1}{\PYGZsq{}}\PYG{l+s+s1}{center}\PYG{l+s+s1}{\PYGZsq{}}\PYG{p}{)}
    \PYG{n}{ax1}\PYG{o}{.}\PYG{n}{spines}\PYG{p}{[}\PYG{l+s+s1}{\PYGZsq{}}\PYG{l+s+s1}{right}\PYG{l+s+s1}{\PYGZsq{}}\PYG{p}{]}\PYG{o}{.}\PYG{n}{set\PYGZus{}color}\PYG{p}{(}\PYG{l+s+s1}{\PYGZsq{}}\PYG{l+s+s1}{none}\PYG{l+s+s1}{\PYGZsq{}}\PYG{p}{)} \PYG{c+c1}{\PYGZsh{} remove the top box}
    \PYG{n}{ax1}\PYG{o}{.}\PYG{n}{spines}\PYG{p}{[}\PYG{l+s+s1}{\PYGZsq{}}\PYG{l+s+s1}{top}\PYG{l+s+s1}{\PYGZsq{}}\PYG{p}{]}\PYG{o}{.}\PYG{n}{set\PYGZus{}color}\PYG{p}{(}\PYG{l+s+s1}{\PYGZsq{}}\PYG{l+s+s1}{none}\PYG{l+s+s1}{\PYGZsq{}}\PYG{p}{)} 
    \PYG{n}{ax1}\PYG{o}{.}\PYG{n}{set\PYGZus{}xticks}\PYG{p}{(}\PYG{p}{[}\PYG{p}{]}\PYG{p}{)}\PYG{p}{;}\PYG{n}{ax1}\PYG{o}{.}\PYG{n}{set\PYGZus{}yticks}\PYG{p}{(}\PYG{p}{[}\PYG{p}{]}\PYG{p}{)}\PYG{p}{;} \PYG{c+c1}{\PYGZsh{} remove the ticks}
    \PYG{n}{ax1}\PYG{o}{.}\PYG{n}{set\PYGZus{}title}\PYG{p}{(}\PYG{l+s+s2}{\PYGZdq{}}\PYG{l+s+s2}{Anisotropy flux and gradient}\PYG{l+s+s2}{\PYGZdq{}}\PYG{p}{,} \PYG{n}{y}\PYG{o}{=}\PYG{l+m+mi}{0}\PYG{p}{,} \PYG{n}{pad}\PYG{o}{=}\PYG{o}{\PYGZhy{}}\PYG{l+m+mi}{25}\PYG{p}{,} \PYG{n}{verticalalignment}\PYG{o}{=}\PYG{l+s+s2}{\PYGZdq{}}\PYG{l+s+s2}{top}\PYG{l+s+s2}{\PYGZdq{}}\PYG{p}{)}


\PYG{c+c1}{\PYGZsh{} plotting Anisotropy}
    \PYG{n}{ax2}\PYG{o}{.}\PYG{n}{plot}\PYG{p}{(}\PYG{n}{Iso\PYGZus{}1x}\PYG{p}{,} \PYG{n}{Iso\PYGZus{}1y}\PYG{p}{,} \PYG{l+s+s2}{\PYGZdq{}}\PYG{l+s+s2}{k}\PYG{l+s+s2}{\PYGZdq{}}\PYG{p}{,} \PYG{n}{label} \PYG{o}{=} \PYG{l+s+sa}{r}\PYG{l+s+s2}{\PYGZdq{}}\PYG{l+s+s2}{\PYGZdl{}K\PYGZus{}h: K\PYGZus{}v\PYGZdl{}}\PYG{l+s+s2}{\PYGZdq{}}\PYG{p}{)}
    \PYG{n}{ax2}\PYG{o}{.}\PYG{n}{plot}\PYG{p}{(}\PYG{n}{Iso\PYGZus{}2x}\PYG{p}{,} \PYG{n}{Iso\PYGZus{}2y}\PYG{p}{,} \PYG{l+s+s2}{\PYGZdq{}}\PYG{l+s+s2}{k}\PYG{l+s+s2}{\PYGZdq{}}\PYG{p}{)}
    \PYG{n}{ax2}\PYG{o}{.}\PYG{n}{legend}\PYG{p}{(}\PYG{n}{bbox\PYGZus{}to\PYGZus{}anchor}\PYG{o}{=}\PYG{p}{(}\PYG{o}{\PYGZhy{}}\PYG{l+m+mf}{0.4}\PYG{p}{,} \PYG{o}{\PYGZhy{}}\PYG{l+m+mf}{0.05}\PYG{p}{)}\PYG{p}{,} \PYG{n}{loc}\PYG{o}{=}\PYG{l+s+s1}{\PYGZsq{}}\PYG{l+s+s1}{lower left}\PYG{l+s+s1}{\PYGZsq{}}\PYG{p}{)}
    \PYG{n}{ax2}\PYG{o}{.}\PYG{n}{set\PYGZus{}xlim}\PYG{p}{(}\PYG{l+m+mi}{1}\PYG{p}{,} \PYG{l+m+mi}{2}\PYG{p}{)}
    \PYG{n}{ax2}\PYG{o}{.}\PYG{n}{set\PYGZus{}ylim}\PYG{p}{(}\PYG{o}{\PYGZhy{}}\PYG{l+m+mi}{1}\PYG{p}{,} \PYG{l+m+mi}{1}\PYG{p}{)}
    \PYG{n}{ax2}\PYG{o}{.}\PYG{n}{axis}\PYG{p}{(}\PYG{l+s+s1}{\PYGZsq{}}\PYG{l+s+s1}{off}\PYG{l+s+s1}{\PYGZsq{}}\PYG{p}{)}
    \PYG{n}{ax2}\PYG{o}{.}\PYG{n}{set\PYGZus{}title}\PYG{p}{(}\PYG{l+s+s2}{\PYGZdq{}}\PYG{l+s+s2}{Anisotropy ratio}\PYG{l+s+s2}{\PYGZdq{}}\PYG{p}{,} \PYG{n}{y}\PYG{o}{=}\PYG{l+m+mi}{0}\PYG{p}{,} \PYG{n}{pad}\PYG{o}{=}\PYG{o}{\PYGZhy{}}\PYG{l+m+mi}{25}\PYG{p}{,} \PYG{n}{verticalalignment}\PYG{o}{=}\PYG{l+s+s2}{\PYGZdq{}}\PYG{l+s+s2}{top}\PYG{l+s+s2}{\PYGZdq{}}\PYG{p}{)}\PYG{p}{;}

\PYG{n}{interactive}\PYG{p}{(}\PYG{n}{aniso}\PYG{p}{,}
         \PYG{n}{a\PYGZus{}d}\PYG{o}{=}\PYG{n}{widgets}\PYG{o}{.}\PYG{n}{BoundedFloatText}\PYG{p}{(}\PYG{n}{value}\PYG{o}{=}\PYG{l+m+mi}{45}\PYG{p}{,} \PYG{n+nb}{min}\PYG{o}{=}\PYG{l+m+mi}{0}\PYG{p}{,} \PYG{n+nb}{max}\PYG{o}{=}\PYG{l+m+mi}{360}\PYG{p}{,} \PYG{n}{step}\PYG{o}{=}\PYG{l+m+mf}{0.5}\PYG{p}{,} \PYG{n}{description}\PYG{o}{=}\PYG{l+s+sa}{r}\PYG{l+s+s1}{\PYGZsq{}}\PYG{l+s+s1}{angle (°)}\PYG{l+s+s1}{\PYGZsq{}}\PYG{p}{,} \PYG{n}{disabled}\PYG{o}{=}\PYG{k+kc}{False}\PYG{p}{)}\PYG{p}{,}
         \PYG{n}{ani\PYGZus{}r}\PYG{o}{=}\PYG{n}{widgets}\PYG{o}{.}\PYG{n}{BoundedIntText}\PYG{p}{(}\PYG{n}{value}\PYG{o}{=}\PYG{l+m+mi}{1}\PYG{p}{,} \PYG{n+nb}{min}\PYG{o}{=}\PYG{l+m+mi}{1}\PYG{p}{,} \PYG{n+nb}{max}\PYG{o}{=}\PYG{l+m+mi}{100}\PYG{p}{,} \PYG{n}{step}\PYG{o}{=}\PYG{l+m+mi}{1}\PYG{p}{,} \PYG{n}{description}\PYG{o}{=}\PYG{l+s+s1}{\PYGZsq{}}\PYG{l+s+s1}{K\PYGZlt{}sub\PYGZgt{}h\PYGZlt{}/sub\PYGZgt{}/K\PYGZlt{}sub\PYGZgt{}v\PYGZlt{}/sub\PYGZgt{}}\PYG{l+s+s1}{\PYGZsq{}}\PYG{p}{,} \PYG{n}{disabled}\PYG{o}{=}\PYG{k+kc}{False}\PYG{p}{)}\PYG{p}{)}
    
\end{sphinxVerbatim}

\end{sphinxuseclass}\end{sphinxVerbatimInput}
\begin{sphinxVerbatimOutput}

\begin{sphinxuseclass}{cell_output}
\begin{sphinxVerbatim}[commandchars=\\\{\}]
interactive(children=(BoundedFloatText(value=45.0, description=\PYGZsq{}angle (°)\PYGZsq{}, max=360.0, step=0.5), BoundedIntTe…
\end{sphinxVerbatim}

\end{sphinxuseclass}\end{sphinxVerbatimOutput}

\end{sphinxuseclass}
\sphinxstepscope


\chapter{Type curve and fitting pumping data tool}
\label{\detokenize{content/tools/type_curve_fit:type-curve-and-fitting-pumping-data-tool}}\label{\detokenize{content/tools/type_curve_fit::doc}}

\section{How to use this tool}
\label{\detokenize{content/tools/type_curve_fit:how-to-use-this-tool}}\begin{enumerate}
\sphinxsetlistlabels{\arabic}{enumi}{enumii}{}{.}%
\item {} 
\sphinxAtStartPar
Go to the Binder by clicking the rocket button (top\sphinxhyphen{}right of the page)

\item {} 
\sphinxAtStartPar
Execute the code cell with libraries

\item {} 
\sphinxAtStartPar
Provide pumping data: \sphinxstylestrong{t\_m} for time in \sphinxstyleemphasis{minutes} and \sphinxstylestrong{s\_m} for drawdown in \sphinxstyleemphasis{meters}. Pls. do not change the name \sphinxstylestrong{t\_m} and \sphinxstylestrong{s\_m}.

\item {} 
\sphinxAtStartPar
Execute the data code cell \sphinxhyphen{} and in the table check top 5 data points.

\item {} 
\sphinxAtStartPar
Execute the next code cell \sphinxhyphen{} 4 interactive boxes will appear change\sphinxhyphen{} discharge \(Q\) and distant of observation \(r\) value are known value. Change the default value in the box with your own values.

\item {} 
\sphinxAtStartPar
Change the value of Transmissivity (\(T\)) and Storage coefficient (\(S\)) and check the fits in the graph.

\item {} 
\sphinxAtStartPar
Step 6 should be continued until desired fit is observed in the graph.

\end{enumerate}


\subsection{Running the tool offline}
\label{\detokenize{content/tools/type_curve_fit:running-the-tool-offline}}\begin{itemize}
\item {} 
\sphinxAtStartPar
In the offline mode, you can use your own data (user\_data.csv). You should use the sample data file provided \sphinxhref{https://prabhasyadav.github.io/iGW-I/data/user\_data.csv}{here:}

\item {} 
\sphinxAtStartPar
In the cell where data is put, uncomment cells with this forms \#1\sphinxhyphen{}3. And comment out (use \#) the uncommented line t\_m and s\_m

\item {} 
\sphinxAtStartPar
Do not change the name of the csv file \sphinxstylestrong{user\_data.csv} and also the column titles (\sphinxstylestrong{Time (min)} and \sphinxstylestrong{Drawdown (m)} ).

\item {} 
\sphinxAtStartPar
Follow steps 4\sphinxhyphen{}7 from above. \DUrole{xref,std,std-doc}{/contents/flow/lecture\_03/13\_gw\_storage}

\end{itemize}

\sphinxAtStartPar
The codes are licensed under CC by 4.0 \sphinxhref{https://creativecommons.org/licenses/by/4.0/deed.en}{(use anyways, but acknowledge the original work)}


\subsection{Python \sphinxstyleliteralintitle{\sphinxupquote{Libraries}} Cell}
\label{\detokenize{content/tools/type_curve_fit:python-libraries-cell}}
\sphinxAtStartPar
\sphinxcode{\sphinxupquote{expi}} function from \sphinxcode{\sphinxupquote{scipy.special}}s that provides easy calculation of well function, and \sphinxcode{\sphinxupquote{interactive}}, \sphinxcode{\sphinxupquote{widgets}} and \sphinxcode{\sphinxupquote{Layout}} from \sphinxcode{\sphinxupquote{ipywidget}} \sphinxhyphen{} for interactive activities, are \sphinxstylestrong{special} functions used in this tool.

\sphinxAtStartPar
\sphinxcode{\sphinxupquote{numpy}} for computation, \sphinxcode{\sphinxupquote{matplotlib.pyplot}} for plotting and \sphinxcode{\sphinxupquote{pandas}} for tabulation, are most general libraries for our works.

\sphinxAtStartPar
Please execute the cell before moving to the next step.

\begin{sphinxuseclass}{cell}\begin{sphinxVerbatimInput}

\begin{sphinxuseclass}{cell_input}
\begin{sphinxVerbatim}[commandchars=\\\{\}]
\PYG{c+c1}{\PYGZsh{} used library}
\PYG{c+c1}{\PYGZsh{}usual libraries}
\PYG{k+kn}{import} \PYG{n+nn}{numpy} \PYG{k}{as} \PYG{n+nn}{np}
\PYG{k+kn}{import} \PYG{n+nn}{pandas} \PYG{k}{as} \PYG{n+nn}{pd}
\PYG{k+kn}{import} \PYG{n+nn}{matplotlib}\PYG{n+nn}{.}\PYG{n+nn}{pyplot} \PYG{k}{as} \PYG{n+nn}{plt}

\PYG{c+c1}{\PYGZsh{} specific libraries}
\PYG{k+kn}{from} \PYG{n+nn}{ipywidgets} \PYG{k+kn}{import} \PYG{n}{interactive}\PYG{p}{,} \PYG{n}{widgets}\PYG{p}{,} \PYG{n}{Layout} \PYG{c+c1}{\PYGZsh{} for interactive plot with slider}
\PYG{k+kn}{from} \PYG{n+nn}{scipy}\PYG{n+nn}{.}\PYG{n+nn}{special} \PYG{k+kn}{import} \PYG{n}{expi} \PYG{c+c1}{\PYGZsh{} for well function}
\end{sphinxVerbatim}

\end{sphinxuseclass}\end{sphinxVerbatimInput}

\end{sphinxuseclass}

\subsection{\sphinxstyleliteralintitle{\sphinxupquote{Input Data}} Cell}
\label{\detokenize{content/tools/type_curve_fit:input-data-cell}}
\sphinxAtStartPar
The next cell is for providing pumping data. You can change the value of variables \sphinxstylestrong{t\_m} and \sphinxstylestrong{s\_m}. Please do not change the name of the variable. Also, for offline mode \sphinxhyphen{} you have an option to upload your \sphinxcode{\sphinxupquote{.csv}} data.

\sphinxAtStartPar
Make sure to execute this cell below and check the output table before moving to the next step.

\sphinxAtStartPar
(\sphinxstyleemphasis{Default data are from \DUrole{xref,std,std-doc}{/contents/tutorials/tutorial\_07/tutorial\_07}})

\begin{sphinxuseclass}{cell}\begin{sphinxVerbatimInput}

\begin{sphinxuseclass}{cell_input}
\begin{sphinxVerbatim}[commandchars=\\\{\}]
\PYG{c+c1}{\PYGZsh{} input data must be in *.csv format. Time data must be in \PYGZdq{}min\PYGZdq{}, and Drawdown in \PYGZdq{}m\PYGZdq{}. }
\PYG{c+c1}{\PYGZsh{}This can only be done in offline mode currently. Remove numbered comments \PYGZsh{}1, from below}

\PYG{c+c1}{\PYGZsh{}1 data = pd.read\PYGZus{}csv(\PYGZdq{}user\PYGZus{}data.csv\PYGZdq{}, sep = \PYGZdq{},\PYGZdq{}, usecols =[\PYGZdq{}Time (min)\PYGZdq{}, \PYGZdq{}Drawdown (m)\PYGZdq{}])}

\PYG{c+c1}{\PYGZsh{}2 t\PYGZus{}m= data19.values[:,0] \PYGZsh{} extracting time data and converting to numpy array}
\PYG{c+c1}{\PYGZsh{}3 s\PYGZus{}m= data19.values[:,1]}

\PYG{c+c1}{\PYGZsh{} You can change the data in t\PYGZus{}m and s\PYGZus{}m. Pls. comment if you are using offline and import your date (csv file)}
\PYG{n}{t\PYGZus{}m} \PYG{o}{=}\PYG{n}{np}\PYG{o}{.}\PYG{n}{array}\PYG{p}{(}\PYG{p}{[}\PYG{l+m+mi}{1}\PYG{p}{,} \PYG{l+m+mi}{2}\PYG{p}{,}    \PYG{l+m+mi}{3}\PYG{p}{,}    \PYG{l+m+mi}{4}\PYG{p}{,}    \PYG{l+m+mi}{5}\PYG{p}{,}    \PYG{l+m+mi}{7}\PYG{p}{,}    \PYG{l+m+mi}{9}\PYG{p}{,}   \PYG{l+m+mi}{12}\PYG{p}{,}   \PYG{l+m+mi}{18}\PYG{p}{,}   \PYG{l+m+mi}{23}\PYG{p}{,}   \PYG{l+m+mi}{33}\PYG{p}{,} \PYG{l+m+mi}{41}\PYG{p}{,}   \PYG{l+m+mi}{56}\PYG{p}{,}  \PYG{l+m+mi}{126}\PYG{p}{,}  \PYG{l+m+mi}{636}\PYG{p}{,} \PYG{l+m+mi}{1896}\PYG{p}{]}\PYG{p}{)}
\PYG{n}{s\PYGZus{}m} \PYG{o}{=} \PYG{n}{np}\PYG{o}{.}\PYG{n}{array}\PYG{p}{(}\PYG{p}{[}\PYG{l+m+mf}{0.01}\PYG{p}{,} \PYG{l+m+mf}{0.03}\PYG{p}{,} \PYG{l+m+mf}{0.05}\PYG{p}{,} \PYG{l+m+mf}{0.06}\PYG{p}{,} \PYG{l+m+mf}{0.07}\PYG{p}{,} \PYG{l+m+mf}{0.09}\PYG{p}{,} \PYG{l+m+mf}{0.12}\PYG{p}{,} \PYG{l+m+mf}{0.14}\PYG{p}{,} \PYG{l+m+mf}{0.16}\PYG{p}{,} \PYG{l+m+mf}{0.17}\PYG{p}{,} \PYG{l+m+mf}{0.18}\PYG{p}{,} \PYG{l+m+mf}{0.19}\PYG{p}{,} \PYG{l+m+mf}{0.2} \PYG{p}{,} \PYG{l+m+mf}{0.22}\PYG{p}{,} \PYG{l+m+mf}{0.3} \PYG{p}{,} \PYG{l+m+mf}{0.32}\PYG{p}{]}\PYG{p}{)}


\PYG{n}{t\PYGZus{}s} \PYG{o}{=} \PYG{n}{t\PYGZus{}m}\PYG{o}{*}\PYG{l+m+mi}{60} \PYG{c+c1}{\PYGZsh{} sec\PYGZhy{} converting time to sec.}

\PYG{n}{d} \PYG{o}{=} \PYG{p}{\PYGZob{}}\PYG{l+s+s1}{\PYGZsq{}}\PYG{l+s+s1}{time (s)}\PYG{l+s+s1}{\PYGZsq{}}\PYG{p}{:} \PYG{n}{t\PYGZus{}s}\PYG{p}{,} \PYG{l+s+s1}{\PYGZsq{}}\PYG{l+s+s1}{drawdown (m)}\PYG{l+s+s1}{\PYGZsq{}}\PYG{p}{:} \PYG{n}{s\PYGZus{}m}\PYG{p}{\PYGZcb{}}
\PYG{n}{df} \PYG{o}{=} \PYG{n}{pd}\PYG{o}{.}\PYG{n}{DataFrame}\PYG{p}{(}\PYG{n}{data}\PYG{o}{=}\PYG{n}{d}\PYG{p}{,} \PYG{n}{index}\PYG{o}{=}\PYG{k+kc}{None}\PYG{p}{)} 
\PYG{n}{df}\PYG{o}{.}\PYG{n}{head}\PYG{p}{(}\PYG{l+m+mi}{5}\PYG{p}{)} \PYG{c+c1}{\PYGZsh{} change 5 to larger number if you want to see more data in the table.}
\end{sphinxVerbatim}

\end{sphinxuseclass}\end{sphinxVerbatimInput}
\begin{sphinxVerbatimOutput}

\begin{sphinxuseclass}{cell_output}
\begin{sphinxVerbatim}[commandchars=\\\{\}]
   time (s)  drawdown (m)
0        60          0.01
1       120          0.03
2       180          0.05
3       240          0.06
4       300          0.07
\end{sphinxVerbatim}

\end{sphinxuseclass}\end{sphinxVerbatimOutput}

\end{sphinxuseclass}

\subsection{The \sphinxstylestrong{main} \sphinxstyleliteralintitle{\sphinxupquote{function}} cell}
\label{\detokenize{content/tools/type_curve_fit:the-main-function-cell}}
\sphinxAtStartPar
The cell provide the main function \sphinxcode{\sphinxupquote{well\_f}} for running the tool. \sphinxcode{\sphinxupquote{well\_f}} requires 4 inputs in the order: \sphinxstylestrong{Transmissivity(m\textbackslash{}u00b2/s)}, \sphinxstylestrong{Storage coefficient (\sphinxhyphen{})}, \sphinxstylestrong{distance to observation well (m)}, and \sphinxstylestrong{discharge (m\textbackslash{}u00b3/s)}

\sphinxAtStartPar
These value should be appropriately modified to make data fit the Type curve.

\sphinxAtStartPar
After the cell is executed, 4 boxes with default value of the arguments will appear. You can interactively change the values in the boxes and visually see the fit.

\begin{sphinxuseclass}{cell}\begin{sphinxVerbatimInput}

\begin{sphinxuseclass}{cell_input}
\begin{sphinxVerbatim}[commandchars=\\\{\}]
\PYG{k}{def} \PYG{n+nf}{W}\PYG{p}{(}\PYG{n}{u}\PYG{p}{)}\PYG{p}{:}  
    \PYG{k}{return}  \PYG{o}{\PYGZhy{}}\PYG{n}{expi}\PYG{p}{(}\PYG{o}{\PYGZhy{}}\PYG{n}{u}\PYG{p}{)} \PYG{c+c1}{\PYGZsh{} provides the well function}

\PYG{k}{def} \PYG{n+nf}{well\PYGZus{}f}\PYG{p}{(}\PYG{n}{T}\PYG{p}{,} \PYG{n}{S\PYGZus{}c}\PYG{p}{,} \PYG{n}{r}\PYG{p}{,} \PYG{n}{Q}\PYG{p}{)}\PYG{p}{:} \PYG{c+c1}{\PYGZsh{} provides the fit curve for given r and Q}
    
        
    \PYG{c+c1}{\PYGZsh{} calculated function see L07\PYGZhy{}slide 31}
    \PYG{n}{u\PYGZus{}1d} \PYG{o}{=} \PYG{l+m+mi}{4}\PYG{o}{*}\PYG{n}{T}\PYG{o}{*}\PYG{n}{t\PYGZus{}s}\PYG{o}{/}\PYG{p}{(}\PYG{n}{S\PYGZus{}c}\PYG{o}{*}\PYG{n}{r}\PYG{o}{*}\PYG{o}{*}\PYG{l+m+mi}{2}\PYG{p}{)} \PYG{c+c1}{\PYGZsh{} calculating 1/u}
    \PYG{n}{w\PYGZus{}ud} \PYG{o}{=} \PYG{l+m+mi}{4}\PYG{o}{*}\PYG{n}{np}\PYG{o}{.}\PYG{n}{pi}\PYG{o}{*}\PYG{n}{s\PYGZus{}m}\PYG{o}{*}\PYG{n}{T}\PYG{o}{/}\PYG{n}{Q}   \PYG{c+c1}{\PYGZsh{} well function}

    \PYG{c+c1}{\PYGZsh{} plots}
    \PYG{n}{u\PYGZus{}1} \PYG{o}{=} \PYG{n}{np}\PYG{o}{.}\PYG{n}{logspace}\PYG{p}{(}\PYG{l+m+mi}{10}\PYG{p}{,}\PYG{o}{\PYGZhy{}}\PYG{l+m+mi}{1}\PYG{p}{,}\PYG{l+m+mi}{250}\PYG{p}{,} \PYG{n}{base}\PYG{o}{=}\PYG{l+m+mf}{10.0}\PYG{p}{)}
    \PYG{n}{w\PYGZus{}u} \PYG{o}{=}\PYG{n}{W}\PYG{p}{(}\PYG{l+m+mi}{1}\PYG{o}{/}\PYG{n}{u\PYGZus{}1}\PYG{p}{)} 
    
    \PYG{n}{plt}\PYG{o}{.}\PYG{n}{figure}\PYG{p}{(}\PYG{n}{figsize}\PYG{o}{=}\PYG{p}{(}\PYG{l+m+mi}{9}\PYG{p}{,}\PYG{l+m+mi}{6}\PYG{p}{)}\PYG{p}{)}\PYG{p}{;}
    \PYG{n}{plt}\PYG{o}{.}\PYG{n}{loglog}\PYG{p}{(}\PYG{n}{u\PYGZus{}1}\PYG{p}{,} \PYG{n}{w\PYGZus{}u}\PYG{p}{,} \PYG{n}{label} \PYG{o}{=} \PYG{l+s+s2}{\PYGZdq{}}\PYG{l+s+s2}{Type curve}\PYG{l+s+s2}{\PYGZdq{}}\PYG{p}{)}\PYG{p}{;} 
    \PYG{n}{plt}\PYG{o}{.}\PYG{n}{loglog}\PYG{p}{(}\PYG{n}{u\PYGZus{}1d}\PYG{p}{,} \PYG{n}{w\PYGZus{}ud}\PYG{p}{,} \PYG{l+s+s2}{\PYGZdq{}}\PYG{l+s+s2}{o}\PYG{l+s+s2}{\PYGZdq{}}\PYG{p}{,} \PYG{n}{color}\PYG{o}{=}\PYG{l+s+s2}{\PYGZdq{}}\PYG{l+s+s2}{red}\PYG{l+s+s2}{\PYGZdq{}}\PYG{p}{,} \PYG{n}{label} \PYG{o}{=} \PYG{l+s+s2}{\PYGZdq{}}\PYG{l+s+s2}{data}\PYG{l+s+s2}{\PYGZdq{}}\PYG{p}{)}
    \PYG{n}{plt}\PYG{o}{.}\PYG{n}{ylim}\PYG{p}{(}\PYG{p}{(}\PYG{l+m+mf}{0.1}\PYG{p}{,} \PYG{l+m+mi}{10}\PYG{p}{)}\PYG{p}{)}\PYG{p}{;}\PYG{n}{plt}\PYG{o}{.}\PYG{n}{xlim}\PYG{p}{(}\PYG{l+m+mi}{1}\PYG{p}{,} \PYG{l+m+mf}{1e5}\PYG{p}{)}
    \PYG{n}{plt}\PYG{o}{.}\PYG{n}{grid}\PYG{p}{(}\PYG{k+kc}{True}\PYG{p}{,} \PYG{n}{which}\PYG{o}{=}\PYG{l+s+s2}{\PYGZdq{}}\PYG{l+s+s2}{both}\PYG{l+s+s2}{\PYGZdq{}}\PYG{p}{,}\PYG{n}{ls}\PYG{o}{=}\PYG{l+s+s2}{\PYGZdq{}}\PYG{l+s+s2}{\PYGZhy{}}\PYG{l+s+s2}{\PYGZdq{}}\PYG{p}{)} 
    \PYG{n}{plt}\PYG{o}{.}\PYG{n}{ylabel}\PYG{p}{(}\PYG{l+s+sa}{r}\PYG{l+s+s2}{\PYGZdq{}}\PYG{l+s+s2}{W(u)}\PYG{l+s+s2}{\PYGZdq{}}\PYG{p}{)}\PYG{p}{;}\PYG{n}{plt}\PYG{o}{.}\PYG{n}{xlabel}\PYG{p}{(}\PYG{l+s+sa}{r}\PYG{l+s+s2}{\PYGZdq{}}\PYG{l+s+s2}{1/u}\PYG{l+s+s2}{\PYGZdq{}}\PYG{p}{)}
    \PYG{n}{plt}\PYG{o}{.}\PYG{n}{legend}\PYG{p}{(}\PYG{p}{)}
    
\PYG{n}{style} \PYG{o}{=} \PYG{p}{\PYGZob{}}\PYG{l+s+s1}{\PYGZsq{}}\PYG{l+s+s1}{description\PYGZus{}width}\PYG{l+s+s1}{\PYGZsq{}}\PYG{p}{:} \PYG{l+s+s1}{\PYGZsq{}}\PYG{l+s+s1}{initial}\PYG{l+s+s1}{\PYGZsq{}}\PYG{p}{\PYGZcb{}}
\PYG{n}{layout}\PYG{o}{=}\PYG{n}{Layout}\PYG{p}{(}\PYG{n}{width}\PYG{o}{=}\PYG{l+s+s1}{\PYGZsq{}}\PYG{l+s+s1}{250px}\PYG{l+s+s1}{\PYGZsq{}}\PYG{p}{)}
\PYG{n}{interactive\PYGZus{}plot} \PYG{o}{=} \PYG{n}{interactive}\PYG{p}{(}\PYG{n}{well\PYGZus{}f}\PYG{p}{,} 
                            \PYG{n}{T}   \PYG{o}{=} \PYG{n}{widgets}\PYG{o}{.}\PYG{n}{FloatText}\PYG{p}{(}\PYG{n}{value}\PYG{o}{=} \PYG{l+m+mf}{0.00322}\PYG{p}{,} \PYG{n}{description}\PYG{o}{=}\PYG{l+s+s1}{\PYGZsq{}}\PYG{l+s+s1}{Transmissivity  (m}\PYG{l+s+se}{\PYGZbs{}u00b2}\PYG{l+s+s1}{/s):}\PYG{l+s+s1}{\PYGZsq{}}\PYG{p}{,} \PYG{n}{disabled}\PYG{o}{=}\PYG{k+kc}{False}\PYG{p}{,} \PYG{n}{style}\PYG{o}{=}\PYG{n}{style}\PYG{p}{,} \PYG{n}{layout}\PYG{o}{=}\PYG{n}{layout}\PYG{p}{,}\PYG{p}{)}\PYG{p}{,} 
                            \PYG{n}{S\PYGZus{}c} \PYG{o}{=} \PYG{n}{widgets}\PYG{o}{.}\PYG{n}{FloatText}\PYG{p}{(}\PYG{n}{value}\PYG{o}{=} \PYG{l+m+mf}{7.97e\PYGZhy{}03}\PYG{p}{,} \PYG{n}{description}\PYG{o}{=}\PYG{l+s+s1}{\PYGZsq{}}\PYG{l+s+s1}{Storage Coefficient (\PYGZhy{}):}\PYG{l+s+s1}{\PYGZsq{}}\PYG{p}{,} \PYG{n}{disabled}\PYG{o}{=}\PYG{k+kc}{False}\PYG{p}{,} \PYG{n}{style}\PYG{o}{=}\PYG{n}{style}\PYG{p}{,} \PYG{n}{layout}\PYG{o}{=}\PYG{n}{layout}\PYG{p}{)}\PYG{p}{,} 
                            \PYG{n}{r}   \PYG{o}{=} \PYG{n}{widgets}\PYG{o}{.}\PYG{n}{FloatText}\PYG{p}{(}\PYG{n}{value}\PYG{o}{=} \PYG{l+m+mf}{9.85}\PYG{p}{,} \PYG{n}{description}\PYG{o}{=}\PYG{l+s+s1}{\PYGZsq{}}\PYG{l+s+s1}{Obs. well location (m):}\PYG{l+s+s1}{\PYGZsq{}}\PYG{p}{,} \PYG{n}{disabled}\PYG{o}{=}\PYG{k+kc}{False}\PYG{p}{,} \PYG{n}{style}\PYG{o}{=}\PYG{n}{style}\PYG{p}{,} \PYG{n}{layout}\PYG{o}{=}\PYG{n}{layout}\PYG{p}{)}\PYG{p}{,} 
                            \PYG{n}{Q}   \PYG{o}{=} \PYG{n}{widgets}\PYG{o}{.}\PYG{n}{FloatText}\PYG{p}{(}\PYG{n}{value}\PYG{o}{=} \PYG{l+m+mf}{0.0025}\PYG{p}{,} \PYG{n}{description}\PYG{o}{=}\PYG{l+s+s1}{\PYGZsq{}}\PYG{l+s+s1}{Discharge (m}\PYG{l+s+se}{\PYGZbs{}u00b3}\PYG{l+s+s1}{/s):}\PYG{l+s+s1}{\PYGZsq{}}\PYG{p}{,} \PYG{n}{disabled}\PYG{o}{=}\PYG{k+kc}{False}\PYG{p}{,} \PYG{n}{style}\PYG{o}{=}\PYG{n}{style}\PYG{p}{,} \PYG{n}{layout}\PYG{o}{=}\PYG{n}{layout}\PYG{p}{)}\PYG{p}{)} 
\PYG{n}{display}\PYG{p}{(}\PYG{n}{interactive\PYGZus{}plot}\PYG{p}{)}
\end{sphinxVerbatim}

\end{sphinxuseclass}\end{sphinxVerbatimInput}
\begin{sphinxVerbatimOutput}

\begin{sphinxuseclass}{cell_output}
\begin{sphinxVerbatim}[commandchars=\\\{\}]
interactive(children=(FloatText(value=0.00322, description=\PYGZsq{}Transmissivity  (m²/s):\PYGZsq{}, layout=Layout(width=\PYGZsq{}250…
\end{sphinxVerbatim}

\end{sphinxuseclass}\end{sphinxVerbatimOutput}

\end{sphinxuseclass}
\sphinxstepscope


\chapter{Tool for simulating 1D Conservative Transport (Laboratory Column)}
\label{\detokenize{content/tools/1D_advection_dispersion:tool-for-simulating-1d-conservative-transport-laboratory-column}}\label{\detokenize{content/tools/1D_advection_dispersion::doc}}

\section{Info on the tool}
\label{\detokenize{content/tools/1D_advection_dispersion:info-on-the-tool}}
\sphinxAtStartPar
The worksheet addresses \sphinxstylestrong{1D conservative transport} of a solute through a porous medium, e.g. in a laboratory column. 
Water flow through the porous medium is assumed to be steady\sphinxhyphen{}state. 
\sphinxstylestrong{Advection} and \sphinxstylestrong{dispersion} are considered as transport processes. 
\sphinxstylestrong{Dirac injection}, \sphinxstylestrong{finite pulse injection} or \sphinxstylestrong{continuous injection} may be used as upgradient boundary condition. 

\sphinxAtStartPar
The worksheet calculates solute breakthrough at the column outlet (sheet “model”) and allows for comparison with measured data (to be provided in sheet “data”). 
Computations are based on analytical solutions involving the complementary error function.


\begin{savenotes}\sphinxattablestart
\centering
\begin{tabulary}{\linewidth}[t]{|T|T|T|}
\hline
\sphinxstyletheadfamily 
\sphinxAtStartPar
input parameters
&\sphinxstyletheadfamily 
\sphinxAtStartPar
dimension
&\sphinxstyletheadfamily 
\sphinxAtStartPar
remarks
\\
\hline
\sphinxAtStartPar
column length
&
\sphinxAtStartPar
{[}L{]}
&
\sphinxAtStartPar
enter positive number
\\
\hline
\sphinxAtStartPar
column radius
&
\sphinxAtStartPar
{[}L{]}
&
\sphinxAtStartPar
enter positive number
\\
\hline
\sphinxAtStartPar
effective porosity
&
\sphinxAtStartPar
{[}\sphinxhyphen{}{]}
&
\sphinxAtStartPar
enter positive number not bigger than 1
\\
\hline
\sphinxAtStartPar
dispersivity
&
\sphinxAtStartPar
{[}L{]}
&
\sphinxAtStartPar
enter non\sphinxhyphen{}negative number
\\
\hline
\sphinxAtStartPar
flow rate
&
\sphinxAtStartPar
{[}L³/T{]}
&
\sphinxAtStartPar
enter positive number
\\
\hline
\sphinxAtStartPar
initial concentration
&
\sphinxAtStartPar
{[}M/L³{]}
&
\sphinxAtStartPar
enter non\sphinxhyphen{}negative number
\\
\hline
\sphinxAtStartPar
input mass
&
\sphinxAtStartPar
{[}M{]}
&
\sphinxAtStartPar
enter positive number or “inf”
\\
\hline
\sphinxAtStartPar
input concentration
&
\sphinxAtStartPar
{[}M/L³{]}
&
\sphinxAtStartPar
enter non\sphinxhyphen{}negative number or “inf”
\\
\hline
\sphinxAtStartPar
bulk density
&
\sphinxAtStartPar
{[}M/L³{]}
&
\sphinxAtStartPar
leave empty or enter positive number
\\
\hline
\sphinxAtStartPar
starting time
&
\sphinxAtStartPar
{[}T{]}
&
\sphinxAtStartPar
enter non\sphinxhyphen{}negative number
\\
\hline
\end{tabulary}
\par
\sphinxattableend\end{savenotes}

\sphinxAtStartPar
\sphinxstylestrong{\sphinxstyleemphasis{Contributed by Ms. Anne Pförtner and Sophie Pförtner. The original concept from Prof. R. Liedl spreasheet code.}}

\sphinxAtStartPar
The codes are licensed under CC by 4.0 \sphinxhref{https://creativecommons.org/licenses/by/4.0/deed.en}{(use anyways, but acknowledge the original work)}


\subsection{Python \sphinxstyleliteralintitle{\sphinxupquote{Libraries}} Cell}
\label{\detokenize{content/tools/1D_advection_dispersion:python-libraries-cell}}
\sphinxAtStartPar
\sphinxcode{\sphinxupquote{numpy}} for computation, \sphinxcode{\sphinxupquote{matplotlib.pyplot}} for plotting and \sphinxcode{\sphinxupquote{pandas}} for tabulation, are most general libraries for our works.

\sphinxAtStartPar
\sphinxcode{\sphinxupquote{ipywidget}} \sphinxhyphen{} for interactive activities, are \sphinxstylestrong{special} functions used in this tool.

\sphinxAtStartPar
\sphinxstylestrong{Please execute the cell before moving to the next step.}

\begin{sphinxuseclass}{cell}\begin{sphinxVerbatimInput}

\begin{sphinxuseclass}{cell_input}
\begin{sphinxVerbatim}[commandchars=\\\{\}]
\PYG{k+kn}{import} \PYG{n+nn}{math}
\PYG{k+kn}{import} \PYG{n+nn}{numpy} \PYG{k}{as} \PYG{n+nn}{np}
\PYG{k+kn}{from} \PYG{n+nn}{ipywidgets} \PYG{k+kn}{import} \PYG{o}{*}
\PYG{k+kn}{import} \PYG{n+nn}{matplotlib}\PYG{n+nn}{.}\PYG{n+nn}{pyplot} \PYG{k}{as} \PYG{n+nn}{plt}
\end{sphinxVerbatim}

\end{sphinxuseclass}\end{sphinxVerbatimInput}

\end{sphinxuseclass}

\subsection{\sphinxstyleliteralintitle{\sphinxupquote{Input Data}} Cell}
\label{\detokenize{content/tools/1D_advection_dispersion:input-data-cell}}
\sphinxAtStartPar
In the cell below you can change input values. Pls. read the \sphinxstylestrong{info} above and the table before making the changes the input. Also check the boundary type after executing the cell.

\sphinxAtStartPar
\sphinxstylestrong{Make sure to execute the cell if you change any input value}

\begin{sphinxuseclass}{cell}\begin{sphinxVerbatimInput}

\begin{sphinxuseclass}{cell_input}
\begin{sphinxVerbatim}[commandchars=\\\{\}]
\PYG{c+c1}{\PYGZsh{}input data}
\PYG{n}{L} \PYG{o}{=} \PYG{l+m+mi}{50}          \PYG{c+c1}{\PYGZsh{}cm, column lenght}
\PYG{n}{R} \PYG{o}{=} \PYG{l+m+mi}{3}           \PYG{c+c1}{\PYGZsh{}cm, column radius}
\PYG{n}{ne} \PYG{o}{=} \PYG{l+m+mf}{0.25}       \PYG{c+c1}{\PYGZsh{}eff. porosity}
\PYG{n}{alpha} \PYG{o}{=} \PYG{l+m+mf}{0.1}     \PYG{c+c1}{\PYGZsh{}dispersivity}
\PYG{n}{Q} \PYG{o}{=} \PYG{l+m+mf}{0.167}       \PYG{c+c1}{\PYGZsh{}cm³/h, flow rate}
\PYG{n}{c0} \PYG{o}{=} \PYG{l+m+mi}{0}          \PYG{c+c1}{\PYGZsh{}mg/cm³, initital concentration}
\PYG{n}{mi} \PYG{o}{=} \PYG{l+m+mi}{2000}       \PYG{c+c1}{\PYGZsh{}mg, input mass}
\PYG{n}{ci} \PYG{o}{=} \PYG{l+m+mf}{1.25e+1}    \PYG{c+c1}{\PYGZsh{}mg/cm³, input concentration}
\PYG{n}{delta\PYGZus{}t} \PYG{o}{=} \PYG{l+m+mi}{70}    \PYG{c+c1}{\PYGZsh{}h, time increment}

\PYG{n}{A} \PYG{o}{=} \PYG{n}{math}\PYG{o}{.}\PYG{n}{pi} \PYG{o}{*} \PYG{n}{R} \PYG{o}{*} \PYG{n}{R} \PYG{c+c1}{\PYGZsh{}cm², area\PYGZhy{}cross section}
\PYG{n}{vf} \PYG{o}{=} \PYG{n}{Q}\PYG{o}{/}\PYG{n}{A}            \PYG{c+c1}{\PYGZsh{}cm/h, darcy velocity}
\PYG{n}{va} \PYG{o}{=} \PYG{n}{vf}\PYG{o}{/}\PYG{n}{ne}          \PYG{c+c1}{\PYGZsh{}cm/h, average linear velocity}
\PYG{n}{D} \PYG{o}{=} \PYG{n}{alpha} \PYG{o}{*} \PYG{n}{va}      \PYG{c+c1}{\PYGZsh{}cm²/h, dispersion coeff.}
\PYG{n}{Vp} \PYG{o}{=} \PYG{n}{L}\PYG{o}{/}\PYG{n}{va}           \PYG{c+c1}{\PYGZsh{}pore volume}

\PYG{c+c1}{\PYGZsh{}intermediate Results}
\PYG{c+c1}{\PYGZsh{}\PYGZsh{}boundary condition}
\PYG{k}{if} \PYG{n}{mi} \PYG{o}{==} \PYG{n}{math}\PYG{o}{.}\PYG{n}{inf}\PYG{p}{:}
    \PYG{n+nb}{print}\PYG{p}{(}\PYG{l+s+s2}{\PYGZdq{}}\PYG{l+s+s2}{The type of boundary condition is a continuous injection.}\PYG{l+s+s2}{\PYGZdq{}}\PYG{p}{)}
\PYG{k}{elif} \PYG{n}{ci} \PYG{o}{==} \PYG{n}{math}\PYG{o}{.}\PYG{n}{inf}\PYG{p}{:}
    \PYG{n+nb}{print}\PYG{p}{(}\PYG{l+s+s2}{\PYGZdq{}}\PYG{l+s+s2}{The type of boundary condition is a dirac pulse injection.}\PYG{l+s+s2}{\PYGZdq{}}\PYG{p}{)}
\PYG{k}{else}\PYG{p}{:}
    \PYG{n+nb}{print}\PYG{p}{(}\PYG{l+s+s2}{\PYGZdq{}}\PYG{l+s+s2}{The type of boundary condition is a finite pulse injection.}\PYG{l+s+s2}{\PYGZdq{}}\PYG{p}{)}
\end{sphinxVerbatim}

\end{sphinxuseclass}\end{sphinxVerbatimInput}
\begin{sphinxVerbatimOutput}

\begin{sphinxuseclass}{cell_output}
\begin{sphinxVerbatim}[commandchars=\\\{\}]
The type of boundary condition is a finite pulse injection.
\end{sphinxVerbatim}

\end{sphinxuseclass}\end{sphinxVerbatimOutput}

\end{sphinxuseclass}

\subsection{The \sphinxstylestrong{main} \sphinxstyleliteralintitle{\sphinxupquote{function}} cell}
\label{\detokenize{content/tools/1D_advection_dispersion:the-main-function-cell}}
\sphinxAtStartPar
You do not have to change the two cells below only have to execute only once.

\begin{sphinxuseclass}{cell}\begin{sphinxVerbatimInput}

\begin{sphinxuseclass}{cell_input}
\begin{sphinxVerbatim}[commandchars=\\\{\}]
\PYG{c+c1}{\PYGZsh{}Definition of the function}
\PYG{k}{def} \PYG{n+nf}{transport}\PYG{p}{(}\PYG{n}{L}\PYG{p}{,} \PYG{n}{R}\PYG{p}{,} \PYG{n}{ne}\PYG{p}{,} \PYG{n}{alpha}\PYG{p}{,} \PYG{n}{Q}\PYG{p}{,} \PYG{n}{c0}\PYG{p}{,} \PYG{n}{mi}\PYG{p}{,} \PYG{n}{ci}\PYG{p}{,} \PYG{n}{delta\PYGZus{}t}\PYG{p}{,} \PYG{n}{t}\PYG{p}{)}\PYG{p}{:}
    
    \PYG{n}{A} \PYG{o}{=} \PYG{n}{np}\PYG{o}{.}\PYG{n}{pi} \PYG{o}{*} \PYG{n}{R} \PYG{o}{*} \PYG{n}{R}   \PYG{c+c1}{\PYGZsh{}cm², area\PYGZhy{}cross section}
    \PYG{n}{vf} \PYG{o}{=} \PYG{n}{Q}\PYG{o}{/}\PYG{n}{A}            \PYG{c+c1}{\PYGZsh{}cm/h, darcy velocity}
    \PYG{n}{va} \PYG{o}{=} \PYG{n}{vf}\PYG{o}{/}\PYG{n}{ne}          \PYG{c+c1}{\PYGZsh{}cm/h, average linear velocity}
    \PYG{n}{D} \PYG{o}{=} \PYG{n}{alpha} \PYG{o}{*} \PYG{n}{va}      \PYG{c+c1}{\PYGZsh{}cm²/h, dispersion coeff.}
    \PYG{n}{Vp} \PYG{o}{=} \PYG{n}{L}\PYG{o}{/}\PYG{n}{va}           \PYG{c+c1}{\PYGZsh{}pore volume}


    \PYG{c+c1}{\PYGZsh{}\PYGZsh{}peclet}
    \PYG{k}{if} \PYG{n}{alpha} \PYG{o}{==} \PYG{l+m+mi}{0}\PYG{p}{:}
        \PYG{n}{Pe} \PYG{o}{=} \PYG{n}{np}\PYG{o}{.}\PYG{n}{inf}
    \PYG{k}{else}\PYG{p}{:}
        \PYG{n}{Pe} \PYG{o}{=} \PYG{n}{L}\PYG{o}{/}\PYG{n}{alpha}

    \PYG{c+c1}{\PYGZsh{}\PYGZsh{}input duration}
    \PYG{k}{if} \PYG{n}{mi}\PYG{o}{==}\PYG{n}{np}\PYG{o}{.}\PYG{n}{inf} \PYG{o+ow}{or} \PYG{n}{ci}\PYG{o}{==}\PYG{n}{c0}\PYG{p}{:}
        \PYG{n}{ti}\PYG{o}{=}\PYG{n}{np}\PYG{o}{.}\PYG{n}{inf}
    \PYG{k}{else}\PYG{p}{:}
        \PYG{k}{if} \PYG{n}{ci}\PYG{o}{==}\PYG{n}{np}\PYG{o}{.}\PYG{n}{inf}\PYG{p}{:}
            \PYG{n}{ti}\PYG{o}{=}\PYG{l+m+mi}{0}
        \PYG{k}{else}\PYG{p}{:}
            \PYG{n}{ti}\PYG{o}{=}\PYG{n}{mi}\PYG{o}{/}\PYG{n}{Q}\PYG{o}{/}\PYG{n+nb}{abs}\PYG{p}{(}\PYG{n}{ci}\PYG{o}{\PYGZhy{}}\PYG{n}{c0}\PYG{p}{)}
    
    \PYG{n}{Vp} \PYG{o}{=} \PYG{n}{L}\PYG{o}{/}\PYG{n}{va} \PYG{c+c1}{\PYGZsh{}pore volume}

    \PYG{c+c1}{\PYGZsh{}\PYGZsh{}rel. input duration}
    \PYG{k}{if} \PYG{n}{ti} \PYG{o}{==} \PYG{k+kc}{None} \PYG{o+ow}{or} \PYG{n}{Vp} \PYG{o}{==} \PYG{k+kc}{None}\PYG{p}{:}
        \PYG{n}{ti\PYGZus{}rel} \PYG{o}{=} \PYG{k+kc}{None}
    \PYG{k}{else}\PYG{p}{:}    
        \PYG{k}{if} \PYG{n}{ti} \PYG{o}{==} \PYG{n}{np}\PYG{o}{.}\PYG{n}{inf}\PYG{p}{:}
            \PYG{n}{ti\PYGZus{}rel} \PYG{o}{=} \PYG{n}{np}\PYG{o}{.}\PYG{n}{inf}
        \PYG{k}{else}\PYG{p}{:}
            \PYG{n}{ti\PYGZus{}rel}\PYG{o}{=}\PYG{n}{ti}\PYG{o}{/}\PYG{n}{Vp}

    \PYG{k}{if} \PYG{n}{Vp} \PYG{o}{==} \PYG{k+kc}{None}\PYG{p}{:}
        \PYG{n}{delta\PYGZus{}t\PYGZus{}rel} \PYG{o}{=} \PYG{k+kc}{None}
    \PYG{k}{else}\PYG{p}{:}
        \PYG{n}{delta\PYGZus{}t\PYGZus{}rel} \PYG{o}{=} \PYG{n}{delta\PYGZus{}t} \PYG{o}{/} \PYG{n}{Vp} \PYG{c+c1}{\PYGZsh{}rel. time increment}

    \PYG{c+c1}{\PYGZsh{}rel. time}

    \PYG{k}{if} \PYG{n}{Vp} \PYG{o}{==} \PYG{k+kc}{None}\PYG{p}{:}
        \PYG{n}{t\PYGZus{}rel} \PYG{o}{=} \PYG{k+kc}{None}
    \PYG{k}{else}\PYG{p}{:}
        \PYG{n}{t\PYGZus{}rel}\PYG{o}{=}\PYG{n}{t}\PYG{o}{/}\PYG{n}{Vp}

  
    \PYG{c+c1}{\PYGZsh{}\PYGZsh{}initial condition }
    \PYG{c+c1}{\PYGZsh{}\PYGZsh{}\PYGZsh{}arg\PYGZhy{}}
    \PYG{k}{if} \PYG{n}{Pe} \PYG{o}{==} \PYG{k+kc}{None} \PYG{o+ow}{or} \PYG{n}{t\PYGZus{}rel} \PYG{o}{==} \PYG{l+m+mi}{0} \PYG{o+ow}{or} \PYG{n}{Pe} \PYG{o}{==} \PYG{n}{np}\PYG{o}{.}\PYG{n}{inf} \PYG{o+ow}{or} \PYG{n}{t\PYGZus{}rel} \PYG{o}{==} \PYG{k+kc}{None}\PYG{p}{:}
        \PYG{n}{arg\PYGZus{}n1\PYGZus{}IC}\PYG{o}{=}\PYG{k+kc}{None}
    \PYG{k}{else}\PYG{p}{:}
        \PYG{n}{arg\PYGZus{}n1\PYGZus{}IC}\PYG{o}{=}\PYG{n}{np}\PYG{o}{.}\PYG{n}{sqrt}\PYG{p}{(}\PYG{l+m+mf}{0.25}\PYG{o}{*}\PYG{n}{Pe}\PYG{o}{/}\PYG{n}{t\PYGZus{}rel}\PYG{p}{)}\PYG{o}{*}\PYG{p}{(}\PYG{l+m+mi}{1}\PYG{o}{\PYGZhy{}}\PYG{n}{t\PYGZus{}rel}\PYG{p}{)}

    \PYG{k}{if} \PYG{n}{Pe} \PYG{o}{==} \PYG{k+kc}{None} \PYG{o+ow}{or} \PYG{n}{arg\PYGZus{}n1\PYGZus{}IC} \PYG{o}{==} \PYG{k+kc}{None} \PYG{o+ow}{or} \PYG{n}{Pe} \PYG{o}{==} \PYG{n}{np}\PYG{o}{.}\PYG{n}{inf} \PYG{o+ow}{or} \PYG{n}{t\PYGZus{}rel} \PYG{o}{==} \PYG{l+m+mi}{0}\PYG{p}{:}
        \PYG{n}{arg\PYGZus{}n2\PYGZus{}IC} \PYG{o}{=} \PYG{k+kc}{None}
    \PYG{k}{else}\PYG{p}{:}    
        \PYG{k}{if} \PYG{n}{arg\PYGZus{}n1\PYGZus{}IC} \PYG{o}{\PYGZgt{}} \PYG{l+m+mi}{0}\PYG{p}{:}
            \PYG{n}{arg\PYGZus{}n2\PYGZus{}IC}\PYG{o}{=} \PYG{l+m+mi}{1}\PYG{o}{\PYGZhy{}}\PYG{l+m+mi}{1}\PYG{o}{*}\PYG{p}{(}\PYG{l+m+mi}{1}\PYG{o}{\PYGZhy{}}\PYG{n}{math}\PYG{o}{.}\PYG{n}{erfc}\PYG{p}{(}\PYG{n+nb}{min}\PYG{p}{(}\PYG{n+nb}{abs}\PYG{p}{(}\PYG{n}{arg\PYGZus{}n1\PYGZus{}IC}\PYG{p}{)}\PYG{p}{,}\PYG{l+m+mi}{27}\PYG{p}{)}\PYG{p}{)}\PYG{p}{)}
        \PYG{k}{elif} \PYG{n}{arg\PYGZus{}n1\PYGZus{}IC} \PYG{o}{\PYGZlt{}} \PYG{l+m+mi}{0}\PYG{p}{:}
            \PYG{n}{arg\PYGZus{}n2\PYGZus{}IC}\PYG{o}{=} \PYG{l+m+mi}{1}\PYG{o}{+}\PYG{l+m+mi}{1}\PYG{o}{*}\PYG{p}{(}\PYG{l+m+mi}{1}\PYG{o}{\PYGZhy{}}\PYG{n}{math}\PYG{o}{.}\PYG{n}{erfc}\PYG{p}{(}\PYG{n+nb}{min}\PYG{p}{(}\PYG{n+nb}{abs}\PYG{p}{(}\PYG{n}{arg\PYGZus{}n1\PYGZus{}IC}\PYG{p}{)}\PYG{p}{,}\PYG{l+m+mi}{27}\PYG{p}{)}\PYG{p}{)}\PYG{p}{)}
            
        \PYG{k}{else}\PYG{p}{:}
            \PYG{n}{arg\PYGZus{}n2\PYGZus{}IC}\PYG{o}{=} \PYG{l+m+mi}{1}\PYG{o}{\PYGZhy{}}\PYG{l+m+mi}{0}\PYG{o}{*}\PYG{p}{(}\PYG{l+m+mi}{1}\PYG{o}{\PYGZhy{}}\PYG{n}{math}\PYG{o}{.}\PYG{n}{erfc}\PYG{p}{(}\PYG{n+nb}{min}\PYG{p}{(}\PYG{n+nb}{abs}\PYG{p}{(}\PYG{n}{arg\PYGZus{}n1\PYGZus{}IC}\PYG{p}{)}\PYG{p}{,}\PYG{l+m+mi}{27}\PYG{p}{)}\PYG{p}{)}\PYG{p}{)}  
        

    
    
    \PYG{c+c1}{\PYGZsh{}\PYGZsh{}\PYGZsh{}arg+}
    \PYG{k}{if} \PYG{n}{Pe} \PYG{o}{==} \PYG{k+kc}{None} \PYG{o+ow}{or} \PYG{n}{t\PYGZus{}rel} \PYG{o}{==} \PYG{k+kc}{None} \PYG{o+ow}{or} \PYG{n}{Pe} \PYG{o}{==} \PYG{n}{np}\PYG{o}{.}\PYG{n}{inf} \PYG{o+ow}{or} \PYG{n}{t\PYGZus{}rel} \PYG{o}{==} \PYG{l+m+mi}{0}\PYG{p}{:}
        \PYG{n}{arg\PYGZus{}p1\PYGZus{}IC} \PYG{o}{=} \PYG{k+kc}{None}
    \PYG{k}{else}\PYG{p}{:}
        \PYG{n}{arg\PYGZus{}p1\PYGZus{}IC} \PYG{o}{=} \PYG{n}{np}\PYG{o}{.}\PYG{n}{sqrt}\PYG{p}{(}\PYG{l+m+mf}{0.25}\PYG{o}{*}\PYG{n}{Pe}\PYG{o}{/}\PYG{n}{t\PYGZus{}rel}\PYG{p}{)}\PYG{o}{*}\PYG{p}{(}\PYG{l+m+mi}{1}\PYG{o}{+}\PYG{n}{t\PYGZus{}rel}\PYG{p}{)}
        
    \PYG{k}{if} \PYG{n}{Pe} \PYG{o}{==} \PYG{k+kc}{None} \PYG{o+ow}{or} \PYG{n}{arg\PYGZus{}p1\PYGZus{}IC} \PYG{o}{==} \PYG{k+kc}{None} \PYG{o+ow}{or} \PYG{n}{Pe} \PYG{o}{==} \PYG{n}{np}\PYG{o}{.}\PYG{n}{inf} \PYG{o+ow}{or} \PYG{n}{t\PYGZus{}rel} \PYG{o}{==} \PYG{l+m+mi}{0}\PYG{p}{:}
        \PYG{n}{arg\PYGZus{}p2\PYGZus{}IC} \PYG{o}{=} \PYG{k+kc}{None} 
    \PYG{k}{else}\PYG{p}{:}
        \PYG{k}{if} \PYG{n}{arg\PYGZus{}p1\PYGZus{}IC}\PYG{o}{\PYGZgt{}}\PYG{l+m+mi}{0}\PYG{p}{:}
            \PYG{n}{arg\PYGZus{}p2\PYGZus{}IC} \PYG{o}{=} \PYG{l+m+mi}{1}\PYG{o}{\PYGZhy{}}\PYG{l+m+mi}{1}\PYG{o}{*}\PYG{p}{(}\PYG{l+m+mi}{1}\PYG{o}{\PYGZhy{}}\PYG{n}{math}\PYG{o}{.}\PYG{n}{erfc}\PYG{p}{(}\PYG{n+nb}{min}\PYG{p}{(}\PYG{n+nb}{abs}\PYG{p}{(}\PYG{n}{arg\PYGZus{}p1\PYGZus{}IC}\PYG{p}{)}\PYG{p}{,}\PYG{l+m+mi}{27}\PYG{p}{)}\PYG{p}{)}\PYG{p}{)}
        \PYG{k}{elif} \PYG{n}{arg\PYGZus{}p1\PYGZus{}IC}\PYG{o}{\PYGZlt{}}\PYG{l+m+mi}{0}\PYG{p}{:}
            \PYG{n}{arg\PYGZus{}p2\PYGZus{}IC} \PYG{o}{=} \PYG{l+m+mi}{1}\PYG{o}{+}\PYG{l+m+mi}{1}\PYG{o}{*}\PYG{p}{(}\PYG{l+m+mi}{1}\PYG{o}{\PYGZhy{}}\PYG{n}{math}\PYG{o}{.}\PYG{n}{erfc}\PYG{p}{(}\PYG{n+nb}{min}\PYG{p}{(}\PYG{n+nb}{abs}\PYG{p}{(}\PYG{n}{arg\PYGZus{}p1\PYGZus{}IC}\PYG{p}{)}\PYG{p}{,}\PYG{l+m+mi}{27}\PYG{p}{)}\PYG{p}{)}\PYG{p}{)}
        \PYG{k}{else}\PYG{p}{:}
            \PYG{n}{arg\PYGZus{}p2\PYGZus{}IC} \PYG{o}{=} \PYG{l+m+mi}{1}\PYG{o}{\PYGZhy{}}\PYG{l+m+mi}{0}\PYG{o}{*}\PYG{p}{(}\PYG{l+m+mi}{1}\PYG{o}{\PYGZhy{}}\PYG{n}{math}\PYG{o}{.}\PYG{n}{erfc}\PYG{p}{(}\PYG{n+nb}{min}\PYG{p}{(}\PYG{n+nb}{abs}\PYG{p}{(}\PYG{n}{arg\PYGZus{}p1\PYGZus{}IC}\PYG{p}{)}\PYG{p}{,}\PYG{l+m+mi}{27}\PYG{p}{)}\PYG{p}{)}\PYG{p}{)}

    
  
    \PYG{c+c1}{\PYGZsh{}\PYGZsh{}\PYGZsh{}arg\PYGZus{}IC}
    \PYG{k}{if} \PYG{n}{Pe} \PYG{o}{==} \PYG{k+kc}{None} \PYG{o+ow}{or} \PYG{n}{arg\PYGZus{}n2\PYGZus{}IC} \PYG{o}{==} \PYG{k+kc}{None} \PYG{o+ow}{or} \PYG{n}{arg\PYGZus{}p2\PYGZus{}IC} \PYG{o}{==} \PYG{k+kc}{None} \PYG{o+ow}{or} \PYG{n}{Pe} \PYG{o}{==} \PYG{n}{np}\PYG{o}{.}\PYG{n}{inf} \PYG{o+ow}{or} \PYG{n}{t\PYGZus{}rel} \PYG{o}{==} \PYG{l+m+mi}{0}\PYG{p}{:}
        \PYG{n}{arg\PYGZus{}IC} \PYG{o}{=} \PYG{k+kc}{None}
    \PYG{k}{else}\PYG{p}{:}
        \PYG{k}{if} \PYG{n}{arg\PYGZus{}p2\PYGZus{}IC} \PYG{o}{==} \PYG{l+m+mi}{0}\PYG{p}{:}
            \PYG{n}{arg\PYGZus{}IC} \PYG{o}{=} \PYG{n}{arg\PYGZus{}n2\PYGZus{}IC}
        \PYG{k}{else}\PYG{p}{:}
            \PYG{n}{arg\PYGZus{}IC} \PYG{o}{=} \PYG{n}{arg\PYGZus{}n2\PYGZus{}IC} \PYG{o}{+} \PYG{n}{np}\PYG{o}{.}\PYG{n}{exp}\PYG{p}{(}\PYG{n}{Pe}\PYG{p}{)} \PYG{o}{*} \PYG{n}{arg\PYGZus{}p2\PYGZus{}IC}
    
   
    \PYG{c+c1}{\PYGZsh{}\PYGZsh{}boundary condition}
    \PYG{c+c1}{\PYGZsh{}\PYGZsh{}\PYGZsh{}positive pulse}
    \PYG{c+c1}{\PYGZsh{}\PYGZsh{}\PYGZsh{}\PYGZsh{}arg\PYGZhy{}}
    \PYG{k}{if} \PYG{n}{Pe} \PYG{o}{==} \PYG{n}{np}\PYG{o}{.}\PYG{n}{inf} \PYG{o+ow}{or} \PYG{n}{t\PYGZus{}rel}\PYG{o}{==}\PYG{l+m+mi}{0} \PYG{o+ow}{or} \PYG{n}{Pe} \PYG{o}{==} \PYG{k+kc}{None} \PYG{o+ow}{or} \PYG{n}{ti\PYGZus{}rel} \PYG{o}{==} \PYG{k+kc}{None} \PYG{o+ow}{or} \PYG{n}{t\PYGZus{}rel} \PYG{o}{==} \PYG{k+kc}{None} \PYG{o+ow}{or} \PYG{n}{ti\PYGZus{}rel} \PYG{o}{==} \PYG{l+m+mi}{0}\PYG{p}{:}
        \PYG{n}{arg\PYGZus{}n1\PYGZus{}BC\PYGZus{}pp} \PYG{o}{=} \PYG{k+kc}{None}
    \PYG{k}{else}\PYG{p}{:}
        \PYG{n}{arg\PYGZus{}n1\PYGZus{}BC\PYGZus{}pp} \PYG{o}{=} \PYG{n}{np}\PYG{o}{.}\PYG{n}{sqrt}\PYG{p}{(}\PYG{l+m+mf}{0.25}\PYG{o}{*}\PYG{n}{Pe}\PYG{o}{/}\PYG{n}{t\PYGZus{}rel}\PYG{p}{)}\PYG{o}{*}\PYG{p}{(}\PYG{l+m+mi}{1}\PYG{o}{\PYGZhy{}}\PYG{n}{t\PYGZus{}rel}\PYG{p}{)}        
    
    \PYG{k}{if} \PYG{n}{Pe} \PYG{o}{==} \PYG{k+kc}{None} \PYG{o+ow}{or} \PYG{n}{Pe} \PYG{o}{==} \PYG{n}{np}\PYG{o}{.}\PYG{n}{inf} \PYG{o+ow}{or} \PYG{n}{ti\PYGZus{}rel} \PYG{o}{==} \PYG{k+kc}{None} \PYG{o+ow}{or} \PYG{n}{ti\PYGZus{}rel} \PYG{o}{==} \PYG{l+m+mi}{0} \PYG{o+ow}{or} \PYG{n}{arg\PYGZus{}n1\PYGZus{}BC\PYGZus{}pp} \PYG{o}{==} \PYG{k+kc}{None} \PYG{o+ow}{or} \PYG{n}{t\PYGZus{}rel} \PYG{o}{==} \PYG{l+m+mi}{0}\PYG{p}{:}
        \PYG{n}{arg\PYGZus{}n2\PYGZus{}BC\PYGZus{}pp} \PYG{o}{=} \PYG{k+kc}{None}
    \PYG{k}{else}\PYG{p}{:}
        \PYG{k}{if} \PYG{n}{arg\PYGZus{}n1\PYGZus{}BC\PYGZus{}pp}\PYG{o}{\PYGZgt{}}\PYG{l+m+mi}{0}\PYG{p}{:}
            \PYG{n}{arg\PYGZus{}n2\PYGZus{}BC\PYGZus{}pp} \PYG{o}{=} \PYG{l+m+mi}{1}\PYG{o}{\PYGZhy{}}\PYG{l+m+mi}{1}\PYG{o}{*}\PYG{p}{(}\PYG{l+m+mi}{1}\PYG{o}{\PYGZhy{}}\PYG{n}{math}\PYG{o}{.}\PYG{n}{erfc}\PYG{p}{(}\PYG{n+nb}{min}\PYG{p}{(}\PYG{n+nb}{abs}\PYG{p}{(}\PYG{n}{arg\PYGZus{}n1\PYGZus{}BC\PYGZus{}pp}\PYG{p}{)}\PYG{p}{,}\PYG{l+m+mi}{27}\PYG{p}{)}\PYG{p}{)}\PYG{p}{)}
        \PYG{k}{elif} \PYG{n}{arg\PYGZus{}n1\PYGZus{}BC\PYGZus{}pp}\PYG{o}{\PYGZlt{}}\PYG{l+m+mi}{0}\PYG{p}{:}
            \PYG{n}{arg\PYGZus{}n2\PYGZus{}BC\PYGZus{}pp} \PYG{o}{=} \PYG{l+m+mi}{1}\PYG{o}{+}\PYG{l+m+mi}{1}\PYG{o}{*}\PYG{p}{(}\PYG{l+m+mi}{1}\PYG{o}{\PYGZhy{}}\PYG{n}{math}\PYG{o}{.}\PYG{n}{erfc}\PYG{p}{(}\PYG{n+nb}{min}\PYG{p}{(}\PYG{n+nb}{abs}\PYG{p}{(}\PYG{n}{arg\PYGZus{}n1\PYGZus{}BC\PYGZus{}pp}\PYG{p}{)}\PYG{p}{,}\PYG{l+m+mi}{27}\PYG{p}{)}\PYG{p}{)}\PYG{p}{)}
        \PYG{k}{else}\PYG{p}{:}
            \PYG{n}{arg\PYGZus{}n2\PYGZus{}BC\PYGZus{}pp} \PYG{o}{=} \PYG{l+m+mi}{1}\PYG{o}{\PYGZhy{}}\PYG{l+m+mi}{1}\PYG{o}{*}\PYG{p}{(}\PYG{l+m+mi}{0}\PYG{o}{\PYGZhy{}}\PYG{n}{math}\PYG{o}{.}\PYG{n}{erfc}\PYG{p}{(}\PYG{n+nb}{min}\PYG{p}{(}\PYG{n+nb}{abs}\PYG{p}{(}\PYG{n}{arg\PYGZus{}n1\PYGZus{}BC\PYGZus{}pp}\PYG{p}{)}\PYG{p}{,}\PYG{l+m+mi}{27}\PYG{p}{)}\PYG{p}{)}\PYG{p}{)}

    
 
    
 
    \PYG{c+c1}{\PYGZsh{}\PYGZsh{}\PYGZsh{}\PYGZsh{}arg+}
    \PYG{k}{if} \PYG{n}{Pe} \PYG{o}{==} \PYG{k+kc}{None} \PYG{o+ow}{or} \PYG{n}{Pe} \PYG{o}{==}  \PYG{n}{np}\PYG{o}{.}\PYG{n}{inf} \PYG{o+ow}{or} \PYG{n}{ti\PYGZus{}rel} \PYG{o}{==} \PYG{k+kc}{None} \PYG{o+ow}{or} \PYG{n}{ti\PYGZus{}rel} \PYG{o}{==} \PYG{l+m+mi}{0} \PYG{o+ow}{or} \PYG{n}{t\PYGZus{}rel} \PYG{o}{==} \PYG{k+kc}{None} \PYG{o+ow}{or} \PYG{n}{t\PYGZus{}rel} \PYG{o}{==}\PYG{l+m+mi}{0}\PYG{p}{:}
        \PYG{n}{arg\PYGZus{}p1\PYGZus{}BC\PYGZus{}pp} \PYG{o}{=} \PYG{k+kc}{None}
    \PYG{k}{else}\PYG{p}{:} 
        \PYG{n}{arg\PYGZus{}p1\PYGZus{}BC\PYGZus{}pp} \PYG{o}{=} \PYG{n}{np}\PYG{o}{.}\PYG{n}{sqrt}\PYG{p}{(}\PYG{l+m+mf}{0.25}\PYG{o}{*}\PYG{n}{Pe}\PYG{o}{/} \PYG{n}{t\PYGZus{}rel}\PYG{p}{)}\PYG{o}{*}\PYG{p}{(}\PYG{l+m+mi}{1}\PYG{o}{+}\PYG{n}{t\PYGZus{}rel}\PYG{p}{)} 

    \PYG{k}{if} \PYG{n}{Pe} \PYG{o}{==} \PYG{k+kc}{None} \PYG{o+ow}{or} \PYG{n}{Pe} \PYG{o}{==} \PYG{n}{np}\PYG{o}{.}\PYG{n}{inf} \PYG{o+ow}{or} \PYG{n}{ti\PYGZus{}rel} \PYG{o}{==} \PYG{k+kc}{None} \PYG{o+ow}{or} \PYG{n}{ti\PYGZus{}rel} \PYG{o}{==} \PYG{l+m+mi}{0} \PYG{o+ow}{or} \PYG{n}{arg\PYGZus{}n1\PYGZus{}BC\PYGZus{}pp} \PYG{o}{==} \PYG{k+kc}{None} \PYG{o+ow}{or} \PYG{n}{t\PYGZus{}rel} \PYG{o}{==} \PYG{l+m+mi}{0}\PYG{p}{:}
        \PYG{n}{arg\PYGZus{}p2\PYGZus{}BC\PYGZus{}pp} \PYG{o}{=} \PYG{k+kc}{None}
    \PYG{k}{else}\PYG{p}{:}
        \PYG{n}{arg\PYGZus{}p2\PYGZus{}BC\PYGZus{}pp} \PYG{o}{=} \PYG{n}{math}\PYG{o}{.}\PYG{n}{erfc}\PYG{p}{(}\PYG{n+nb}{min}\PYG{p}{(}\PYG{n}{arg\PYGZus{}p1\PYGZus{}BC\PYGZus{}pp}\PYG{p}{,}\PYG{l+m+mi}{27}\PYG{p}{)}\PYG{p}{)}
  
    \PYG{c+c1}{\PYGZsh{}\PYGZsh{}\PYGZsh{}\PYGZsh{}arg\PYGZus{}BC\PYGZus{}pp}
    \PYG{k}{if} \PYG{n}{Pe}\PYG{o}{==}\PYG{n}{np}\PYG{o}{.}\PYG{n}{inf} \PYG{o+ow}{or} \PYG{n}{t\PYGZus{}rel}\PYG{o}{==}\PYG{l+m+mi}{0} \PYG{o+ow}{or} \PYG{n}{Pe}\PYG{o}{==}\PYG{k+kc}{None} \PYG{o+ow}{or} \PYG{n}{ti\PYGZus{}rel}\PYG{o}{==}\PYG{k+kc}{None} \PYG{o+ow}{or} \PYG{n}{t\PYGZus{}rel}\PYG{o}{==}\PYG{k+kc}{None} \PYG{o+ow}{or} \PYG{n}{ti\PYGZus{}rel}\PYG{o}{==}\PYG{l+m+mi}{0} \PYG{o+ow}{or} \PYG{n}{arg\PYGZus{}n2\PYGZus{}BC\PYGZus{}pp} \PYG{o}{==} \PYG{k+kc}{None} \PYG{o+ow}{or} \PYG{n}{arg\PYGZus{}p2\PYGZus{}BC\PYGZus{}pp} \PYG{o}{==} \PYG{k+kc}{None} \PYG{p}{:}
        \PYG{n}{arg\PYGZus{}BC\PYGZus{}pp}\PYG{o}{=}\PYG{k+kc}{None}
    \PYG{k}{else}\PYG{p}{:}
        \PYG{k}{if} \PYG{n}{arg\PYGZus{}p2\PYGZus{}BC\PYGZus{}pp}\PYG{o}{==}\PYG{l+m+mi}{0}\PYG{p}{:}
            \PYG{n}{arg\PYGZus{}BC\PYGZus{}pp} \PYG{o}{=} \PYG{n}{arg\PYGZus{}n2\PYGZus{}BC\PYGZus{}pp}
        \PYG{k}{else}\PYG{p}{:}
            \PYG{n}{arg\PYGZus{}BC\PYGZus{}pp} \PYG{o}{=} \PYG{n}{arg\PYGZus{}n2\PYGZus{}BC\PYGZus{}pp} \PYG{o}{+} \PYG{p}{(}\PYG{n}{np}\PYG{o}{.}\PYG{n}{exp}\PYG{p}{(}\PYG{n}{Pe}\PYG{p}{)}\PYG{o}{*}\PYG{n}{arg\PYGZus{}p2\PYGZus{}BC\PYGZus{}pp}\PYG{p}{)}  
   
   
    \PYG{c+c1}{\PYGZsh{}\PYGZsh{}\PYGZsh{}negative pulse}
    \PYG{c+c1}{\PYGZsh{}\PYGZsh{}\PYGZsh{}\PYGZsh{}arg\PYGZhy{}}
    \PYG{n}{arg\PYGZus{}n1\PYGZus{}BC\PYGZus{}np}\PYG{o}{=}\PYG{k+kc}{None}
    \PYG{n}{arg\PYGZus{}n2\PYGZus{}BC\PYGZus{}np}\PYG{o}{=}\PYG{k+kc}{None}

    \PYG{k}{if} \PYG{n}{Pe}\PYG{o}{==}\PYG{k+kc}{None} \PYG{o+ow}{or} \PYG{n}{ti\PYGZus{}rel}\PYG{o}{==}\PYG{k+kc}{None} \PYG{o+ow}{or} \PYG{n}{Pe}\PYG{o}{==}\PYG{n}{np}\PYG{o}{.}\PYG{n}{inf} \PYG{o+ow}{or} \PYG{n}{ti\PYGZus{}rel}\PYG{o}{==}\PYG{n}{np}\PYG{o}{.}\PYG{n}{inf} \PYG{o+ow}{or} \PYG{n}{ti\PYGZus{}rel} \PYG{o}{==} \PYG{l+m+mi}{0} \PYG{o+ow}{or} \PYG{n}{t\PYGZus{}rel}\PYG{o}{==}\PYG{l+m+mi}{0} \PYG{o+ow}{or} \PYG{n}{t\PYGZus{}rel} \PYG{o}{==} \PYG{k+kc}{None}\PYG{p}{:}
        \PYG{n}{arg\PYGZus{}n1\PYGZus{}BC\PYGZus{}np} \PYG{o}{=} \PYG{k+kc}{None}
    \PYG{k}{else}\PYG{p}{:} 
        \PYG{k}{if}  \PYG{n}{t\PYGZus{}rel}\PYG{o}{\PYGZgt{}}\PYG{n}{ti\PYGZus{}rel}\PYG{p}{:}
            \PYG{n}{arg\PYGZus{}n1\PYGZus{}BC\PYGZus{}np} \PYG{o}{=} \PYG{n}{np}\PYG{o}{.}\PYG{n}{sqrt}\PYG{p}{(}\PYG{l+m+mf}{0.25}\PYG{o}{*}\PYG{n}{Pe}\PYG{o}{/}\PYG{p}{(}\PYG{n}{t\PYGZus{}rel}\PYG{o}{\PYGZhy{}}\PYG{n}{ti\PYGZus{}rel}\PYG{p}{)}\PYG{p}{)}\PYG{o}{*}\PYG{p}{(}\PYG{l+m+mi}{1}\PYG{o}{\PYGZhy{}}\PYG{p}{(}\PYG{n}{t\PYGZus{}rel}\PYG{o}{\PYGZhy{}}\PYG{n}{ti\PYGZus{}rel}\PYG{p}{)}\PYG{p}{)}
        \PYG{k}{else}\PYG{p}{:}
            \PYG{n}{arg\PYGZus{}n1\PYGZus{}BC\PYGZus{}np} \PYG{o}{=} \PYG{k+kc}{None}
    
    \PYG{k}{if} \PYG{n}{Pe}\PYG{o}{==}\PYG{k+kc}{None} \PYG{o+ow}{or} \PYG{n}{ti\PYGZus{}rel}\PYG{o}{==}\PYG{k+kc}{None} \PYG{o+ow}{or} \PYG{n}{arg\PYGZus{}n1\PYGZus{}BC\PYGZus{}np} \PYG{o}{==} \PYG{k+kc}{None} \PYG{o+ow}{or} \PYG{n}{Pe}\PYG{o}{==} \PYG{n}{np}\PYG{o}{.}\PYG{n}{inf} \PYG{o+ow}{or} \PYG{n}{ti\PYGZus{}rel}\PYG{o}{==}\PYG{l+m+mi}{0} \PYG{o+ow}{or} \PYG{n}{ti\PYGZus{}rel} \PYG{o}{==} \PYG{n}{np}\PYG{o}{.}\PYG{n}{inf} \PYG{o+ow}{or} \PYG{n}{t\PYGZus{}rel} \PYG{o}{==} \PYG{l+m+mi}{0}\PYG{p}{:}
        \PYG{n}{arg\PYGZus{}n2\PYGZus{}BC\PYGZus{}np} \PYG{o}{=} \PYG{k+kc}{None}
    \PYG{k}{else}\PYG{p}{:} 
        \PYG{k}{if} \PYG{n}{arg\PYGZus{}n1\PYGZus{}BC\PYGZus{}np} \PYG{o}{\PYGZgt{}} \PYG{l+m+mi}{0}\PYG{p}{:}
            \PYG{n}{arg\PYGZus{}n2\PYGZus{}BC\PYGZus{}np} \PYG{o}{=} \PYG{l+m+mi}{1}\PYG{o}{\PYGZhy{}}\PYG{p}{(}\PYG{l+m+mi}{1}\PYG{o}{\PYGZhy{}}\PYG{n}{math}\PYG{o}{.}\PYG{n}{erfc}\PYG{p}{(}\PYG{n+nb}{min}\PYG{p}{(}\PYG{n+nb}{abs}\PYG{p}{(}\PYG{n}{arg\PYGZus{}n1\PYGZus{}BC\PYGZus{}np}\PYG{p}{)}\PYG{p}{,}\PYG{l+m+mi}{27}\PYG{p}{)}\PYG{p}{)}\PYG{p}{)}
        \PYG{k}{elif} \PYG{n}{arg\PYGZus{}n1\PYGZus{}BC\PYGZus{}np} \PYG{o}{\PYGZlt{}} \PYG{l+m+mi}{0}\PYG{p}{:}
            \PYG{n}{arg\PYGZus{}n2\PYGZus{}BC\PYGZus{}np} \PYG{o}{=} \PYG{l+m+mi}{1}\PYG{o}{+}\PYG{p}{(}\PYG{l+m+mi}{1}\PYG{o}{\PYGZhy{}}\PYG{n}{math}\PYG{o}{.}\PYG{n}{erfc}\PYG{p}{(}\PYG{n+nb}{min}\PYG{p}{(}\PYG{n+nb}{abs}\PYG{p}{(}\PYG{n}{arg\PYGZus{}n1\PYGZus{}BC\PYGZus{}np}\PYG{p}{)}\PYG{p}{,}\PYG{l+m+mi}{27}\PYG{p}{)}\PYG{p}{)}\PYG{p}{)}
        \PYG{k}{else}\PYG{p}{:}
            \PYG{n}{arg\PYGZus{}n2\PYGZus{}BC\PYGZus{}np} \PYG{o}{=} \PYG{l+m+mi}{1}\PYG{o}{\PYGZhy{}}\PYG{p}{(}\PYG{l+m+mi}{0}\PYG{o}{\PYGZhy{}}\PYG{n}{math}\PYG{o}{.}\PYG{n}{erfc}\PYG{p}{(}\PYG{n+nb}{min}\PYG{p}{(}\PYG{n+nb}{abs}\PYG{p}{(}\PYG{n}{arg\PYGZus{}n1\PYGZus{}BC\PYGZus{}np}\PYG{p}{)}\PYG{p}{,}\PYG{l+m+mi}{27}\PYG{p}{)}\PYG{p}{)}\PYG{p}{)}
       
    \PYG{c+c1}{\PYGZsh{}\PYGZsh{}\PYGZsh{}\PYGZsh{}arg+}

    \PYG{k}{if} \PYG{n}{Pe}\PYG{o}{==}\PYG{n}{np}\PYG{o}{.}\PYG{n}{inf} \PYG{o+ow}{or} \PYG{n}{Pe}\PYG{o}{==}\PYG{k+kc}{None} \PYG{o+ow}{or} \PYG{n}{ti\PYGZus{}rel}\PYG{o}{==}\PYG{l+m+mi}{0} \PYG{o+ow}{or} \PYG{n}{ti\PYGZus{}rel}\PYG{o}{==}\PYG{n}{np}\PYG{o}{.}\PYG{n}{inf} \PYG{o+ow}{or} \PYG{n}{t\PYGZus{}rel}\PYG{o}{==}\PYG{l+m+mi}{0} \PYG{o+ow}{or} \PYG{n}{ti\PYGZus{}rel}\PYG{o}{==}\PYG{k+kc}{None} \PYG{o+ow}{or} \PYG{n}{t\PYGZus{}rel}\PYG{o}{==}\PYG{k+kc}{None}\PYG{p}{:}
        \PYG{n}{arg\PYGZus{}p1\PYGZus{}BC\PYGZus{}np}\PYG{o}{=}\PYG{k+kc}{None}
    \PYG{k}{else}\PYG{p}{:}
        \PYG{k}{if} \PYG{n}{t\PYGZus{}rel}\PYG{o}{\PYGZgt{}}\PYG{n}{ti\PYGZus{}rel}\PYG{p}{:}
            \PYG{n}{arg\PYGZus{}p1\PYGZus{}BC\PYGZus{}np}\PYG{o}{=}\PYG{n}{np}\PYG{o}{.}\PYG{n}{sqrt}\PYG{p}{(}\PYG{l+m+mf}{0.25}\PYG{o}{*}\PYG{n}{Pe}\PYG{o}{/}\PYG{p}{(}\PYG{n}{t\PYGZus{}rel}\PYG{o}{\PYGZhy{}}\PYG{n}{ti\PYGZus{}rel}\PYG{p}{)}\PYG{p}{)}\PYG{o}{*}\PYG{p}{(}\PYG{l+m+mi}{1}\PYG{o}{+}\PYG{p}{(}\PYG{n}{t\PYGZus{}rel}\PYG{o}{\PYGZhy{}}\PYG{n}{ti\PYGZus{}rel}\PYG{p}{)}\PYG{p}{)}
        \PYG{k}{else}\PYG{p}{:}
            \PYG{n}{arg\PYGZus{}p1\PYGZus{}BC\PYGZus{}np} \PYG{o}{=} \PYG{k+kc}{None}

    \PYG{k}{if} \PYG{n}{Pe}\PYG{o}{==}\PYG{k+kc}{None} \PYG{o+ow}{or} \PYG{n}{ti\PYGZus{}rel}\PYG{o}{==}\PYG{k+kc}{None} \PYG{o+ow}{or} \PYG{n}{arg\PYGZus{}p1\PYGZus{}BC\PYGZus{}np} \PYG{o}{==} \PYG{k+kc}{None} \PYG{o+ow}{or} \PYG{n}{Pe} \PYG{o}{==} \PYG{n}{np}\PYG{o}{.}\PYG{n}{inf} \PYG{o+ow}{or} \PYG{n}{ti\PYGZus{}rel} \PYG{o}{==} \PYG{n}{np}\PYG{o}{.}\PYG{n}{inf} \PYG{o+ow}{or} \PYG{n}{ti\PYGZus{}rel} \PYG{o}{==} \PYG{l+m+mi}{0} \PYG{o+ow}{or} \PYG{n}{t\PYGZus{}rel} \PYG{o}{==}\PYG{l+m+mi}{0}\PYG{p}{:}
        \PYG{n}{arg\PYGZus{}p2\PYGZus{}BC\PYGZus{}np} \PYG{o}{=} \PYG{k+kc}{None}
    \PYG{k}{else}\PYG{p}{:}
        \PYG{k}{if} \PYG{n}{t\PYGZus{}rel} \PYG{o}{\PYGZgt{}} \PYG{n}{ti\PYGZus{}rel}\PYG{p}{:}
            \PYG{n}{arg\PYGZus{}p2\PYGZus{}BC\PYGZus{}np} \PYG{o}{=} \PYG{n}{math}\PYG{o}{.}\PYG{n}{erfc}\PYG{p}{(}\PYG{n+nb}{min}\PYG{p}{(}\PYG{p}{(}\PYG{n}{arg\PYGZus{}p1\PYGZus{}BC\PYGZus{}np}\PYG{p}{)}\PYG{p}{,}\PYG{l+m+mi}{27}\PYG{p}{)}\PYG{p}{)}
        \PYG{k}{else}\PYG{p}{:}
            \PYG{n}{arg\PYGZus{}p2\PYGZus{}BC\PYGZus{}np} \PYG{o}{=} \PYG{k+kc}{None} 


    \PYG{c+c1}{\PYGZsh{}\PYGZsh{}\PYGZsh{}\PYGZsh{}arg\PYGZus{}BC\PYGZus{}np}
    \PYG{n}{arg\PYGZus{}BC\PYGZus{}np} \PYG{o}{=} \PYG{k+kc}{None}

    \PYG{k}{if} \PYG{n}{Pe} \PYG{o}{==} \PYG{k+kc}{None} \PYG{o+ow}{or} \PYG{n}{ti\PYGZus{}rel} \PYG{o}{==} \PYG{k+kc}{None} \PYG{o+ow}{or} \PYG{n}{arg\PYGZus{}n2\PYGZus{}BC\PYGZus{}np} \PYG{o}{==} \PYG{k+kc}{None}  \PYG{o+ow}{or} \PYG{n}{arg\PYGZus{}p2\PYGZus{}BC\PYGZus{}np} \PYG{o}{==} \PYG{k+kc}{None} \PYG{o+ow}{or} \PYG{n}{Pe} \PYG{o}{==} \PYG{n}{np}\PYG{o}{.}\PYG{n}{inf} \PYG{o+ow}{or} \PYG{n}{ti\PYGZus{}rel} \PYG{o}{==} \PYG{l+m+mi}{0} \PYG{o+ow}{or} \PYG{n}{ti\PYGZus{}rel} \PYG{o}{==} \PYG{n}{np}\PYG{o}{.}\PYG{n}{inf} \PYG{o+ow}{or} \PYG{n}{t\PYGZus{}rel} \PYG{o}{==} \PYG{l+m+mi}{0}\PYG{p}{:}
        \PYG{n}{arg\PYGZus{}BC\PYGZus{}np} \PYG{o}{=} \PYG{k+kc}{None}
    \PYG{k}{else}\PYG{p}{:}
        \PYG{k}{if} \PYG{n}{t\PYGZus{}rel} \PYG{o}{\PYGZgt{}} \PYG{n}{ti\PYGZus{}rel}\PYG{p}{:}
             \PYG{k}{if} \PYG{n}{arg\PYGZus{}p2\PYGZus{}BC\PYGZus{}np} \PYG{o}{==} \PYG{l+m+mi}{0}\PYG{p}{:}
                \PYG{n}{arg\PYGZus{}BC\PYGZus{}np} \PYG{o}{=} \PYG{n}{arg\PYGZus{}n2\PYGZus{}BC\PYGZus{}np} 
             \PYG{k}{else}\PYG{p}{:}   
                \PYG{n}{arg\PYGZus{}BC\PYGZus{}np} \PYG{o}{=} \PYG{n}{arg\PYGZus{}n2\PYGZus{}BC\PYGZus{}np}\PYG{o}{+} \PYG{n}{np}\PYG{o}{.}\PYG{n}{exp}\PYG{p}{(}\PYG{n}{Pe}\PYG{p}{)} \PYG{o}{*} \PYG{n}{arg\PYGZus{}p2\PYGZus{}BC\PYGZus{}np}
        \PYG{k}{else}\PYG{p}{:}
            \PYG{n}{arg\PYGZus{}BC\PYGZus{}np} \PYG{o}{=} \PYG{k+kc}{None}
            
      
    \PYG{c+c1}{\PYGZsh{}\PYGZsh{}rel conc due initial condition }
    \PYG{k}{if} \PYG{n}{Pe} \PYG{o}{==} \PYG{k+kc}{None} \PYG{o+ow}{or} \PYG{n}{t\PYGZus{}rel} \PYG{o}{==} \PYG{k+kc}{None}\PYG{p}{:}
        \PYG{n}{c\PYGZus{}rel\PYGZus{}IC} \PYG{o}{=} \PYG{k+kc}{None} 
    \PYG{k}{else}\PYG{p}{:}
        \PYG{k}{if} \PYG{n}{t\PYGZus{}rel}\PYG{o}{\PYGZgt{}}\PYG{l+m+mi}{0}\PYG{p}{:}
            \PYG{k}{if} \PYG{n}{Pe}\PYG{o}{==}\PYG{n}{np}\PYG{o}{.}\PYG{n}{inf}\PYG{p}{:}
                \PYG{k}{if} \PYG{n}{t\PYGZus{}rel}\PYG{o}{\PYGZlt{}}\PYG{l+m+mi}{1}\PYG{p}{:}
                    \PYG{n}{c\PYGZus{}rel\PYGZus{}IC} \PYG{o}{=} \PYG{l+m+mi}{1}
                \PYG{k}{else}\PYG{p}{:}
                    \PYG{n}{c\PYGZus{}rel\PYGZus{}IC} \PYG{o}{=} \PYG{l+m+mi}{0}
            \PYG{k}{else}\PYG{p}{:}
                \PYG{n}{c\PYGZus{}rel\PYGZus{}IC} \PYG{o}{=} \PYG{l+m+mi}{1}\PYG{o}{\PYGZhy{}}\PYG{l+m+mf}{0.5}\PYG{o}{*}\PYG{n}{arg\PYGZus{}IC}
        \PYG{k}{else}\PYG{p}{:}
            \PYG{n}{c\PYGZus{}rel\PYGZus{}IC} \PYG{o}{=} \PYG{k+kc}{None}
    
    
   
    \PYG{c+c1}{\PYGZsh{}\PYGZsh{}rel conc due boundary condition   }
    \PYG{k}{if} \PYG{n}{Pe} \PYG{o}{==} \PYG{k+kc}{None} \PYG{o+ow}{or} \PYG{n}{ti\PYGZus{}rel} \PYG{o}{==} \PYG{k+kc}{None} \PYG{o+ow}{or} \PYG{n}{t\PYGZus{}rel} \PYG{o}{==} \PYG{k+kc}{None}\PYG{p}{:}
        \PYG{n}{c\PYGZus{}rel\PYGZus{}BC} \PYG{o}{=} \PYG{k+kc}{None}
    \PYG{k}{else}\PYG{p}{:}
        \PYG{k}{if} \PYG{n}{t\PYGZus{}rel} \PYG{o}{\PYGZgt{}} \PYG{l+m+mi}{0}\PYG{p}{:}
            \PYG{k}{if} \PYG{n}{Pe} \PYG{o}{==} \PYG{n}{np}\PYG{o}{.}\PYG{n}{inf}\PYG{p}{:}
                \PYG{k}{if} \PYG{n}{t\PYGZus{}rel} \PYG{o}{\PYGZgt{}} \PYG{l+m+mi}{1}\PYG{p}{:}
                    \PYG{k}{if} \PYG{n}{ti\PYGZus{}rel} \PYG{o}{==} \PYG{n}{np}\PYG{o}{.}\PYG{n}{inf}\PYG{p}{:}
                        \PYG{k}{if} \PYG{n}{t\PYGZus{}rel}\PYG{o}{\PYGZgt{}}\PYG{l+m+mi}{1}\PYG{p}{:}
                            \PYG{n}{c\PYGZus{}rel\PYGZus{}BC} \PYG{o}{=} \PYG{l+m+mi}{1}
                        \PYG{k}{else}\PYG{p}{:}
                            \PYG{n}{c\PYGZus{}rel\PYGZus{}BC} \PYG{o}{=} \PYG{l+m+mi}{0}
                    \PYG{k}{elif} \PYG{n}{t\PYGZus{}rel} \PYG{o}{\PYGZlt{}}\PYG{o}{=} \PYG{l+m+mi}{1}\PYG{o}{+}\PYG{n}{ti\PYGZus{}rel}\PYG{p}{:}
                        \PYG{n}{c\PYGZus{}rel\PYGZus{}BC} \PYG{o}{=} \PYG{l+m+mi}{1}
                    \PYG{k}{else}\PYG{p}{:}
                        \PYG{n}{c\PYGZus{}rel\PYGZus{}BC} \PYG{o}{=} \PYG{l+m+mi}{0}
                \PYG{k}{else}\PYG{p}{:}
                    \PYG{n}{c\PYGZus{}rel\PYGZus{}BC} \PYG{o}{=} \PYG{l+m+mi}{0}    
            \PYG{k}{else}\PYG{p}{:} 
                \PYG{k}{if} \PYG{n}{ti\PYGZus{}rel} \PYG{o}{==} \PYG{l+m+mi}{0}\PYG{p}{:}
                    \PYG{n}{c\PYGZus{}rel\PYGZus{}BC} \PYG{o}{=} \PYG{n}{np}\PYG{o}{.}\PYG{n}{sqrt}\PYG{p}{(}\PYG{l+m+mf}{0.25}\PYG{o}{/}\PYG{n}{np}\PYG{o}{.}\PYG{n}{pi}\PYG{o}{*}\PYG{n}{Pe}\PYG{o}{/}\PYG{n}{t\PYGZus{}rel}\PYG{o}{\PYGZca{}}\PYG{l+m+mi}{3}\PYG{p}{)}\PYG{o}{*}\PYG{n}{np}\PYG{o}{.}\PYG{n}{exp}\PYG{p}{(}\PYG{o}{\PYGZhy{}}\PYG{l+m+mf}{0.25}\PYG{o}{*}\PYG{n}{Pe}\PYG{o}{/}\PYG{n}{t\PYGZus{}rel}\PYG{o}{*}\PYG{p}{(}\PYG{l+m+mi}{1}\PYG{o}{\PYGZhy{}}\PYG{n}{t\PYGZus{}rel}\PYG{p}{)}\PYG{o}{\PYGZca{}}\PYG{l+m+mi}{2}\PYG{p}{)}
                \PYG{k}{else}\PYG{p}{:}
                    \PYG{k}{if} \PYG{n}{ti\PYGZus{}rel}\PYG{o}{==}\PYG{n}{np}\PYG{o}{.}\PYG{n}{inf} \PYG{o+ow}{or} \PYG{n}{t\PYGZus{}rel}\PYG{o}{\PYGZlt{}}\PYG{o}{=}\PYG{n}{ti\PYGZus{}rel}\PYG{p}{:}
                        \PYG{n}{c\PYGZus{}rel\PYGZus{}BC} \PYG{o}{=} \PYG{l+m+mf}{0.5} \PYG{o}{*} \PYG{n}{arg\PYGZus{}BC\PYGZus{}pp}
                    \PYG{k}{else}\PYG{p}{:}
                        \PYG{n}{c\PYGZus{}rel\PYGZus{}BC} \PYG{o}{=} \PYG{l+m+mf}{0.5}\PYG{o}{*}\PYG{p}{(}\PYG{n}{arg\PYGZus{}BC\PYGZus{}pp}\PYG{o}{\PYGZhy{}}\PYG{n}{arg\PYGZus{}BC\PYGZus{}np}\PYG{p}{)}
        \PYG{k}{else}\PYG{p}{:}
            \PYG{n}{c\PYGZus{}rel\PYGZus{}BC} \PYG{o}{=} \PYG{k+kc}{None}    



    \PYG{k}{if} \PYG{n}{ti\PYGZus{}rel} \PYG{o}{==} \PYG{k+kc}{None} \PYG{o+ow}{or} \PYG{n}{c\PYGZus{}rel\PYGZus{}IC} \PYG{o}{==} \PYG{k+kc}{None} \PYG{o+ow}{or} \PYG{n}{c\PYGZus{}rel\PYGZus{}BC} \PYG{o}{==} \PYG{k+kc}{None}\PYG{p}{:}
        \PYG{n}{c}\PYG{o}{=}\PYG{k+kc}{None}
    \PYG{k}{else}\PYG{p}{:}
        \PYG{k}{if} \PYG{n}{t\PYGZus{}rel}\PYG{o}{==}\PYG{l+m+mi}{0}\PYG{p}{:}
            \PYG{n}{c}\PYG{o}{=}\PYG{n}{c0}
        \PYG{k}{else}\PYG{p}{:}
            \PYG{k}{if} \PYG{n}{ti\PYGZus{}rel} \PYG{o}{==} \PYG{l+m+mi}{0}\PYG{p}{:}
                \PYG{n}{c}\PYG{o}{=}\PYG{n}{c0}\PYG{o}{*}\PYG{n}{c\PYGZus{}rel\PYGZus{}IC}\PYG{o}{+}\PYG{n}{mi}\PYG{o}{/}\PYG{p}{(}\PYG{n}{ne}\PYG{o}{*}\PYG{n}{A}\PYG{o}{*}\PYG{n}{L}\PYG{p}{)}\PYG{o}{*}\PYG{n}{c\PYGZus{}rel\PYGZus{}BC}
              
            \PYG{k}{else}\PYG{p}{:}
                \PYG{n}{c}\PYG{o}{=}\PYG{n}{c0}\PYG{o}{*}\PYG{n}{c\PYGZus{}rel\PYGZus{}IC}\PYG{o}{+}\PYG{n}{ci}\PYG{o}{*}\PYG{n}{c\PYGZus{}rel\PYGZus{}BC} 
    
   
    \PYG{k}{return} \PYG{n}{c}
   
\end{sphinxVerbatim}

\end{sphinxuseclass}\end{sphinxVerbatimInput}

\end{sphinxuseclass}

\subsection{The \sphinxstyleliteralintitle{\sphinxupquote{Interactive}} cell}
\label{\detokenize{content/tools/1D_advection_dispersion:the-interactive-cell}}
\sphinxAtStartPar
Just execute it

\begin{sphinxuseclass}{cell}\begin{sphinxVerbatimInput}

\begin{sphinxuseclass}{cell_input}
\begin{sphinxVerbatim}[commandchars=\\\{\}]
\PYG{k}{def} \PYG{n+nf}{curve\PYGZus{}data}\PYG{p}{(}\PYG{n}{L}\PYG{p}{,} \PYG{n}{R}\PYG{p}{,} \PYG{n}{ne}\PYG{p}{,} \PYG{n}{alpha}\PYG{p}{,} \PYG{n}{Q}\PYG{p}{,} \PYG{n}{c0}\PYG{p}{,} \PYG{n}{mi}\PYG{p}{,} \PYG{n}{ci}\PYG{p}{,} \PYG{n}{delta\PYGZus{}t}\PYG{p}{,} \PYG{n}{t\PYGZus{}max}\PYG{p}{)}\PYG{p}{:}    
   \PYG{n}{plot\PYGZus{}c} \PYG{o}{=} \PYG{p}{[}\PYG{p}{]}
   \PYG{n}{plot\PYGZus{}t} \PYG{o}{=} \PYG{n}{np}\PYG{o}{.}\PYG{n}{arange}\PYG{p}{(}\PYG{l+m+mi}{0}\PYG{p}{,} \PYG{n}{t\PYGZus{}max}\PYG{p}{,} \PYG{n}{delta\PYGZus{}t}\PYG{p}{)}
  
   \PYG{k}{for} \PYG{n}{t} \PYG{o+ow}{in} \PYG{n}{np}\PYG{o}{.}\PYG{n}{arange}\PYG{p}{(}\PYG{l+m+mi}{0}\PYG{p}{,} \PYG{n}{t\PYGZus{}max}\PYG{p}{,} \PYG{n}{delta\PYGZus{}t}\PYG{p}{)}\PYG{p}{:}
       \PYG{n}{plot\PYGZus{}c}\PYG{o}{.}\PYG{n}{append}\PYG{p}{(}\PYG{n}{transport}\PYG{p}{(}\PYG{n}{L}\PYG{p}{,} \PYG{n}{R}\PYG{p}{,} \PYG{n}{ne}\PYG{p}{,} \PYG{n}{alpha}\PYG{p}{,} \PYG{n}{Q}\PYG{p}{,} \PYG{n}{c0}\PYG{p}{,} \PYG{n}{mi}\PYG{p}{,} \PYG{n}{ci}\PYG{p}{,} \PYG{n}{delta\PYGZus{}t}\PYG{p}{,} \PYG{n}{t}\PYG{p}{)}\PYG{p}{)}  
   \PYG{k}{return} \PYG{n}{plot\PYGZus{}t}\PYG{p}{,} \PYG{n}{plot\PYGZus{}c}
    
   

\PYG{k}{def} \PYG{n+nf}{plot}\PYG{p}{(}\PYG{n}{L}\PYG{p}{,} \PYG{n}{R}\PYG{p}{,} \PYG{n}{ne}\PYG{p}{,} \PYG{n}{alpha}\PYG{p}{,} \PYG{n}{Q}\PYG{p}{,} \PYG{n}{c0}\PYG{p}{,} \PYG{n}{mi}\PYG{p}{,} \PYG{n}{ci}\PYG{p}{,} \PYG{n}{delta\PYGZus{}t}\PYG{p}{,} \PYG{n}{t\PYGZus{}max}\PYG{p}{)}\PYG{p}{:}

    \PYG{n}{plot\PYGZus{}t}\PYG{p}{,} \PYG{n}{plot\PYGZus{}c} \PYG{o}{=} \PYG{n}{curve\PYGZus{}data}\PYG{p}{(}\PYG{n}{L}\PYG{p}{,} \PYG{n}{R}\PYG{p}{,} \PYG{n}{ne}\PYG{p}{,} \PYG{n}{alpha}\PYG{p}{,} \PYG{n}{Q}\PYG{p}{,} \PYG{n}{c0}\PYG{p}{,} \PYG{n}{mi}\PYG{p}{,} \PYG{n}{ci}\PYG{p}{,} \PYG{n}{delta\PYGZus{}t}\PYG{p}{,} \PYG{n}{t\PYGZus{}max}\PYG{p}{)}
   
  
    \PYG{n}{plt}\PYG{o}{.}\PYG{n}{plot}\PYG{p}{(}\PYG{n}{plot\PYGZus{}t}\PYG{p}{,} \PYG{n}{plot\PYGZus{}c}\PYG{p}{)}
    \PYG{n}{plt}\PYG{o}{.}\PYG{n}{ylabel}\PYG{p}{(}\PYG{l+s+s1}{\PYGZsq{}}\PYG{l+s+s1}{concentration [mg/cm³]}\PYG{l+s+s1}{\PYGZsq{}}\PYG{p}{)}
    \PYG{n}{plt}\PYG{o}{.}\PYG{n}{ylim}\PYG{p}{(}\PYG{o}{\PYGZhy{}}\PYG{l+m+mi}{10}\PYG{p}{,}\PYG{l+m+mi}{30}\PYG{p}{)}
    \PYG{n}{plt}\PYG{o}{.}\PYG{n}{xlabel}\PYG{p}{(}\PYG{l+s+s1}{\PYGZsq{}}\PYG{l+s+s1}{Time [h]}\PYG{l+s+s1}{\PYGZsq{}}\PYG{p}{)}
    \PYG{n}{plt}\PYG{o}{.}\PYG{n}{xlim}\PYG{p}{(}\PYG{o}{\PYGZhy{}}\PYG{l+m+mi}{1}\PYG{p}{,} \PYG{n}{t\PYGZus{}max}\PYG{p}{)}
    \PYG{n}{plt}\PYG{o}{.}\PYG{n}{show}

\PYG{n}{interact}\PYG{p}{(}\PYG{n}{plot}\PYG{p}{,}
         \PYG{n}{L}\PYG{o}{=}\PYG{n}{widgets}\PYG{o}{.}\PYG{n}{FloatSlider}\PYG{p}{(}\PYG{n}{value}\PYG{o}{=}\PYG{l+m+mi}{50}\PYG{p}{,} \PYG{n+nb}{min}\PYG{o}{=}\PYG{l+m+mi}{0}\PYG{p}{,} \PYG{n+nb}{max}\PYG{o}{=}\PYG{l+m+mi}{500}\PYG{p}{,} \PYG{n}{step}\PYG{o}{=}\PYG{l+m+mi}{1}\PYG{p}{,} \PYG{n}{description}\PYG{o}{=}\PYG{l+s+s1}{\PYGZsq{}}\PYG{l+s+s1}{column lenght [cm]:}\PYG{l+s+s1}{\PYGZsq{}}\PYG{p}{,} \PYG{n}{disabled}\PYG{o}{=}\PYG{k+kc}{False}\PYG{p}{)}\PYG{p}{,}
         \PYG{n}{R}\PYG{o}{=}\PYG{n}{widgets}\PYG{o}{.}\PYG{n}{FloatSlider}\PYG{p}{(}\PYG{n}{value}\PYG{o}{=}\PYG{l+m+mi}{3}\PYG{p}{,} \PYG{n+nb}{min}\PYG{o}{=}\PYG{l+m+mi}{0}\PYG{p}{,} \PYG{n+nb}{max}\PYG{o}{=}\PYG{l+m+mi}{250}\PYG{p}{,} \PYG{n}{step}\PYG{o}{=}\PYG{l+m+mf}{0.1}\PYG{p}{,} \PYG{n}{description}\PYG{o}{=}\PYG{l+s+s1}{\PYGZsq{}}\PYG{l+s+s1}{column radius [cm]:}\PYG{l+s+s1}{\PYGZsq{}}\PYG{p}{,} \PYG{n}{disabled}\PYG{o}{=}\PYG{k+kc}{False}\PYG{p}{)}\PYG{p}{,}
         \PYG{n}{ne}\PYG{o}{=} \PYG{n}{widgets}\PYG{o}{.}\PYG{n}{FloatSlider}\PYG{p}{(}\PYG{n}{value}\PYG{o}{=}\PYG{l+m+mf}{0.25}\PYG{p}{,}\PYG{n+nb}{min}\PYG{o}{=}\PYG{l+m+mi}{0}\PYG{p}{,} \PYG{n+nb}{max}\PYG{o}{=}\PYG{l+m+mi}{1}\PYG{p}{,}\PYG{n}{step}\PYG{o}{=}\PYG{l+m+mf}{0.05}\PYG{p}{,} \PYG{n}{description}\PYG{o}{=}\PYG{l+s+s1}{\PYGZsq{}}\PYG{l+s+s1}{eff. porosity [\PYGZhy{}]:}\PYG{l+s+s1}{\PYGZsq{}} \PYG{p}{,} \PYG{n}{disabled}\PYG{o}{=}\PYG{k+kc}{False}\PYG{p}{)}\PYG{p}{,}
         \PYG{n}{alpha}\PYG{o}{=}\PYG{n}{widgets}\PYG{o}{.}\PYG{n}{FloatSlider}\PYG{p}{(}\PYG{n}{value}\PYG{o}{=}\PYG{l+m+mf}{0.1}\PYG{p}{,} \PYG{n+nb}{min}\PYG{o}{=}\PYG{l+m+mi}{0}\PYG{p}{,} \PYG{n+nb}{max}\PYG{o}{=}\PYG{l+m+mi}{100}\PYG{p}{,} \PYG{n}{step}\PYG{o}{=}\PYG{l+m+mf}{0.01}\PYG{p}{,} \PYG{n}{description}\PYG{o}{=}\PYG{l+s+s1}{\PYGZsq{}}\PYG{l+s+s1}{dispersivity [cm]:}\PYG{l+s+s1}{\PYGZsq{}}\PYG{p}{,} \PYG{n}{disabled}\PYG{o}{=}\PYG{k+kc}{False}\PYG{p}{)}\PYG{p}{,}
         \PYG{n}{Q}\PYG{o}{=}\PYG{n}{widgets}\PYG{o}{.}\PYG{n}{FloatSlider}\PYG{p}{(}\PYG{n}{value}\PYG{o}{=}\PYG{l+m+mf}{0.167}\PYG{p}{,} \PYG{n+nb}{min}\PYG{o}{=}\PYG{l+m+mi}{0}\PYG{p}{,} \PYG{n+nb}{max}\PYG{o}{=}\PYG{l+m+mi}{10}\PYG{p}{,} \PYG{n}{step}\PYG{o}{=}\PYG{l+m+mf}{0.05}\PYG{p}{,} \PYG{n}{description}\PYG{o}{=}\PYG{l+s+s1}{\PYGZsq{}}\PYG{l+s+s1}{flow rate [cm³/h]:}\PYG{l+s+s1}{\PYGZsq{}}\PYG{p}{,} \PYG{n}{disabled}\PYG{o}{=}\PYG{k+kc}{False}\PYG{p}{)}\PYG{p}{,}
         \PYG{n}{c0}\PYG{o}{=} \PYG{n}{widgets}\PYG{o}{.}\PYG{n}{FloatSlider}\PYG{p}{(}\PYG{n}{value}\PYG{o}{=}\PYG{l+m+mi}{0}\PYG{p}{,}\PYG{n+nb}{min}\PYG{o}{=}\PYG{l+m+mi}{0}\PYG{p}{,} \PYG{n+nb}{max}\PYG{o}{=}\PYG{l+m+mi}{1000}\PYG{p}{,}\PYG{n}{step}\PYG{o}{=}\PYG{l+m+mf}{0.5}\PYG{p}{,} \PYG{n}{description}\PYG{o}{=}\PYG{l+s+s1}{\PYGZsq{}}\PYG{l+s+s1}{initital concentration [mg/cm³]:}\PYG{l+s+s1}{\PYGZsq{}}\PYG{p}{,} \PYG{n}{disabled}\PYG{o}{=}\PYG{k+kc}{True}\PYG{p}{)}\PYG{p}{,}
         \PYG{n}{mi}\PYG{o}{=}\PYG{n}{widgets}\PYG{o}{.}\PYG{n}{FloatSlider}\PYG{p}{(}\PYG{n}{value}\PYG{o}{=}\PYG{l+m+mi}{2000}\PYG{p}{,} \PYG{n+nb}{min}\PYG{o}{=}\PYG{l+m+mi}{0}\PYG{p}{,} \PYG{n+nb}{max}\PYG{o}{=}\PYG{l+m+mi}{10000}\PYG{p}{,} \PYG{n}{step}\PYG{o}{=}\PYG{l+m+mi}{10}\PYG{p}{,} \PYG{n}{description}\PYG{o}{=}\PYG{l+s+s1}{\PYGZsq{}}\PYG{l+s+s1}{input mass [mg]:}\PYG{l+s+s1}{\PYGZsq{}}\PYG{p}{,} \PYG{n}{disabled}\PYG{o}{=}\PYG{k+kc}{False}\PYG{p}{)}\PYG{p}{,}
         \PYG{n}{ci}\PYG{o}{=}\PYG{n}{widgets}\PYG{o}{.}\PYG{n}{FloatSlider}\PYG{p}{(}\PYG{n}{value}\PYG{o}{=}\PYG{l+m+mf}{12.5}\PYG{p}{,} \PYG{n+nb}{min}\PYG{o}{=}\PYG{l+m+mi}{0}\PYG{p}{,} \PYG{n+nb}{max}\PYG{o}{=}\PYG{l+m+mi}{1000}\PYG{p}{,} \PYG{n}{step}\PYG{o}{=}\PYG{l+m+mf}{0.5}\PYG{p}{,} \PYG{n}{description}\PYG{o}{=}\PYG{l+s+s1}{\PYGZsq{}}\PYG{l+s+s1}{input concentration [mg/cm³]:}\PYG{l+s+s1}{\PYGZsq{}}\PYG{p}{,} \PYG{n}{disabled}\PYG{o}{=}\PYG{k+kc}{False}\PYG{p}{)}\PYG{p}{,}
         \PYG{n}{delta\PYGZus{}t}\PYG{o}{=} \PYG{n}{widgets}\PYG{o}{.}\PYG{n}{FloatSlider}\PYG{p}{(}\PYG{n}{value}\PYG{o}{=}\PYG{l+m+mi}{70}\PYG{p}{,}\PYG{n+nb}{min}\PYG{o}{=}\PYG{l+m+mi}{0}\PYG{p}{,} \PYG{n+nb}{max}\PYG{o}{=}\PYG{l+m+mi}{100}\PYG{p}{,}\PYG{n}{step}\PYG{o}{=}\PYG{l+m+mf}{0.5}\PYG{p}{,} \PYG{n}{description}\PYG{o}{=}\PYG{l+s+s1}{\PYGZsq{}}\PYG{l+s+s1}{time increment [h]:}\PYG{l+s+s1}{\PYGZsq{}} \PYG{p}{,} \PYG{n}{disabled}\PYG{o}{=}\PYG{k+kc}{False}\PYG{p}{)}\PYG{p}{,}
         \PYG{n}{t\PYGZus{}max} \PYG{o}{=} \PYG{n}{widgets}\PYG{o}{.}\PYG{n}{FloatSlider}\PYG{p}{(}\PYG{n}{value}\PYG{o}{=}\PYG{l+m+mi}{8400}\PYG{p}{,}\PYG{n+nb}{min}\PYG{o}{=}\PYG{l+m+mi}{0}\PYG{p}{,} \PYG{n+nb}{max}\PYG{o}{=}\PYG{l+m+mi}{10000}\PYG{p}{,}\PYG{n}{step}\PYG{o}{=}\PYG{l+m+mi}{24}\PYG{p}{,} \PYG{n}{description}\PYG{o}{=}\PYG{l+s+s1}{\PYGZsq{}}\PYG{l+s+s1}{time [h]:}\PYG{l+s+s1}{\PYGZsq{}} \PYG{p}{,} \PYG{n}{disabled}\PYG{o}{=}\PYG{k+kc}{False}\PYG{p}{)}\PYG{p}{,}
        \PYG{p}{)}
\end{sphinxVerbatim}

\end{sphinxuseclass}\end{sphinxVerbatimInput}
\begin{sphinxVerbatimOutput}

\begin{sphinxuseclass}{cell_output}
\begin{sphinxVerbatim}[commandchars=\\\{\}]
interactive(children=(FloatSlider(value=50.0, description=\PYGZsq{}column lenght [cm]:\PYGZsq{}, max=500.0, step=1.0), FloatSl…
\end{sphinxVerbatim}

\begin{sphinxVerbatim}[commandchars=\\\{\}]
\PYGZlt{}function \PYGZus{}\PYGZus{}main\PYGZus{}\PYGZus{}.plot(L, R, ne, alpha, Q, c0, mi, ci, delta\PYGZus{}t, t\PYGZus{}max)\PYGZgt{}
\end{sphinxVerbatim}

\end{sphinxuseclass}\end{sphinxVerbatimOutput}

\end{sphinxuseclass}
\sphinxstepscope


\chapter{Simulating Kinetics and Degradation}
\label{\detokenize{content/tools/Kinetics_degradation:simulating-kinetics-and-degradation}}\label{\detokenize{content/tools/Kinetics_degradation::doc}}
\sphinxAtStartPar
The tool simulates:
\begin{enumerate}
\sphinxsetlistlabels{\arabic}{enumi}{enumii}{}{.}%
\item {} 
\sphinxAtStartPar
Enzymatic degradation of ethanol e.g., in the human and organism.

\item {} 
\sphinxAtStartPar
Decay of radioactive elements e.g., Cobalt, Strontium

\item {} 
\sphinxAtStartPar
Michaelis–Menten Kinetics

\end{enumerate}


\section{How to use this tool}
\label{\detokenize{content/tools/Kinetics_degradation:how-to-use-this-tool}}\begin{enumerate}
\sphinxsetlistlabels{\arabic}{enumi}{enumii}{}{.}%
\item {} 
\sphinxAtStartPar
Go to the Binder by clicking the rocket button (top\sphinxhyphen{}right of the page)

\item {} 
\sphinxAtStartPar
Execute the code cell with libraries

\item {} 
\sphinxAtStartPar
Interact with the sliders.

\end{enumerate}

\sphinxAtStartPar
The codes are licensed under CC by 4.0 \sphinxhref{https://creativecommons.org/licenses/by/4.0/deed.en}{(use anyways, but acknowledge the original work)}

\begin{sphinxuseclass}{cell}\begin{sphinxVerbatimInput}

\begin{sphinxuseclass}{cell_input}
\begin{sphinxVerbatim}[commandchars=\\\{\}]
\PYG{k+kn}{import} \PYG{n+nn}{numpy} \PYG{k}{as} \PYG{n+nn}{np} 
\PYG{k+kn}{import} \PYG{n+nn}{matplotlib}\PYG{n+nn}{.}\PYG{n+nn}{pyplot} \PYG{k}{as} \PYG{n+nn}{plt} 
\PYG{k+kn}{import} \PYG{n+nn}{pandas} \PYG{k}{as} \PYG{n+nn}{pd}
\PYG{k+kn}{from} \PYG{n+nn}{ipywidgets} \PYG{k+kn}{import} \PYG{n}{interact}\PYG{p}{,} \PYG{n}{fixed}
\PYG{k+kn}{import} \PYG{n+nn}{ipywidgets} \PYG{k}{as} \PYG{n+nn}{widgets}
\end{sphinxVerbatim}

\end{sphinxuseclass}\end{sphinxVerbatimInput}

\end{sphinxuseclass}
\sphinxAtStartPar
The enzymatic degradation of ethanol in the human and organism.

\begin{sphinxuseclass}{cell}\begin{sphinxVerbatimInput}

\begin{sphinxuseclass}{cell_input}
\begin{sphinxVerbatim}[commandchars=\\\{\}]
\PYG{k}{def} \PYG{n+nf}{alcohol}\PYG{p}{(}\PYG{n}{C0}\PYG{p}{,}\PYG{n}{λ}\PYG{p}{,}\PYG{n}{t}\PYG{p}{)}\PYG{p}{:}
    
    \PYG{l+s+sd}{\PYGZdq{}\PYGZdq{}\PYGZdq{}}
\PYG{l+s+sd}{    C0 = concentration [M/L\PYGZca{}3]}
\PYG{l+s+sd}{    λ = degradation constant [M/L\PYGZca{}3/T]}
\PYG{l+s+sd}{    t = time [T]}
\PYG{l+s+sd}{    }
\PYG{l+s+sd}{    \PYGZdq{}\PYGZdq{}\PYGZdq{}}
    
    \PYG{n}{plt}\PYG{o}{.}\PYG{n}{figure}\PYG{p}{(}\PYG{p}{)}
    \PYG{n}{t} \PYG{o}{=} \PYG{n}{np}\PYG{o}{.}\PYG{n}{linspace}\PYG{p}{(}\PYG{l+m+mi}{0}\PYG{p}{,} \PYG{n}{t}\PYG{p}{,} \PYG{l+m+mi}{1000}\PYG{p}{)}
    \PYG{n}{plt}\PYG{o}{.}\PYG{n}{plot}\PYG{p}{(}\PYG{n}{t}\PYG{p}{,} \PYG{n}{C0} \PYG{o}{\PYGZhy{}} \PYG{n}{λ} \PYG{o}{*} \PYG{n}{t}\PYG{p}{)}
    \PYG{n}{plt}\PYG{o}{.}\PYG{n}{ylabel}\PYG{p}{(}\PYG{l+s+s1}{\PYGZsq{}}\PYG{l+s+s1}{Blood alcohol concentration [‰]}\PYG{l+s+s1}{\PYGZsq{}}\PYG{p}{)}
    \PYG{n}{plt}\PYG{o}{.}\PYG{n}{title}\PYG{p}{(}\PYG{l+s+s1}{\PYGZsq{}}\PYG{l+s+s1}{enzymatic degradation of ethanol}\PYG{l+s+s1}{\PYGZsq{}}\PYG{p}{)}
    \PYG{n}{plt}\PYG{o}{.}\PYG{n}{ylim}\PYG{p}{(}\PYG{l+m+mi}{0}\PYG{p}{,}\PYG{l+m+mf}{8.1}\PYG{p}{)} \PYG{c+c1}{\PYGZsh{}the highest measured blood alcohol concentration in germany }
    \PYG{n}{plt}\PYG{o}{.}\PYG{n}{xlabel}\PYG{p}{(}\PYG{l+s+s1}{\PYGZsq{}}\PYG{l+s+s1}{time [h]}\PYG{l+s+s1}{\PYGZsq{}}\PYG{p}{)}
    \PYG{n}{plt}\PYG{o}{.}\PYG{n}{xlim}\PYG{p}{(}\PYG{l+m+mi}{0}\PYG{p}{,}\PYG{l+m+mi}{100}\PYG{p}{)}
    \PYG{n}{plt}\PYG{o}{.}\PYG{n}{show}\PYG{p}{(}\PYG{p}{)}

    
\PYG{n}{interact}\PYG{p}{(}\PYG{n}{alcohol}\PYG{p}{,} \PYG{n}{C0} \PYG{o}{=} \PYG{n}{widgets}\PYG{o}{.}\PYG{n}{FloatSlider}\PYG{p}{(}\PYG{n+nb}{min}\PYG{o}{=}\PYG{l+m+mi}{0}\PYG{p}{,} \PYG{n+nb}{max}\PYG{o}{=} \PYG{l+m+mf}{8.1}\PYG{p}{,} \PYG{n}{step}\PYG{o}{=}\PYG{l+m+mf}{0.1}\PYG{p}{,} \PYG{n}{value}\PYG{o}{=}\PYG{l+m+mi}{3}\PYG{p}{)}\PYG{p}{,} 
         \PYG{n}{t}\PYG{o}{=} \PYG{n}{fixed}\PYG{p}{(}\PYG{l+m+mi}{200}\PYG{p}{)}\PYG{p}{,} \PYG{n}{λ}\PYG{o}{=}\PYG{n}{widgets}\PYG{o}{.}\PYG{n}{FloatSlider}\PYG{p}{(}\PYG{n+nb}{min}\PYG{o}{=}\PYG{l+m+mi}{0}\PYG{p}{,}\PYG{n+nb}{max}\PYG{o}{=}\PYG{l+m+mi}{1}\PYG{p}{,} \PYG{n}{value}\PYG{o}{=}\PYG{l+m+mf}{0.15}\PYG{p}{,} \PYG{n}{step}\PYG{o}{=}\PYG{l+m+mf}{0.01}\PYG{p}{,} \PYG{n}{readout\PYGZus{}format}\PYG{o}{=}\PYG{l+s+s1}{\PYGZsq{}}\PYG{l+s+s1}{.2f}\PYG{l+s+s1}{\PYGZsq{}}\PYG{p}{)}\PYG{p}{)}
\end{sphinxVerbatim}

\end{sphinxuseclass}\end{sphinxVerbatimInput}
\begin{sphinxVerbatimOutput}

\begin{sphinxuseclass}{cell_output}
\begin{sphinxVerbatim}[commandchars=\\\{\}]
interactive(children=(FloatSlider(value=3.0, description=\PYGZsq{}C0\PYGZsq{}, max=8.1), FloatSlider(value=0.15, description=\PYGZsq{}…
\end{sphinxVerbatim}

\begin{sphinxVerbatim}[commandchars=\\\{\}]
\PYGZlt{}function \PYGZus{}\PYGZus{}main\PYGZus{}\PYGZus{}.alcohol(C0, λ, t)\PYGZgt{}
\end{sphinxVerbatim}

\end{sphinxuseclass}\end{sphinxVerbatimOutput}

\end{sphinxuseclass}
\sphinxAtStartPar
The decay of the elements cobalt and strontium.

\begin{sphinxuseclass}{cell}\begin{sphinxVerbatimInput}

\begin{sphinxuseclass}{cell_input}
\begin{sphinxVerbatim}[commandchars=\\\{\}]
\PYG{k}{def} \PYG{n+nf}{Co}\PYG{p}{(}\PYG{n}{C0}\PYG{p}{,}\PYG{n}{λ}\PYG{p}{,}\PYG{n}{t}\PYG{p}{)}\PYG{p}{:} \PYG{c+c1}{\PYGZsh{}define the funtion for the decay of cobalt}
   
    \PYG{n}{plt}\PYG{o}{.}\PYG{n}{figure}\PYG{p}{(}\PYG{p}{)}
    \PYG{n}{t} \PYG{o}{=} \PYG{n}{np}\PYG{o}{.}\PYG{n}{linspace}\PYG{p}{(}\PYG{l+m+mi}{0}\PYG{p}{,} \PYG{n}{t}\PYG{p}{,} \PYG{l+m+mi}{1000}\PYG{p}{)}
    \PYG{n}{y}\PYG{o}{=} \PYG{n}{C0} \PYG{o}{*} \PYG{n}{np}\PYG{o}{.}\PYG{n}{exp}\PYG{p}{(}\PYG{o}{\PYGZhy{}}\PYG{p}{(}\PYG{n}{λ} \PYG{o}{*} \PYG{n}{t}\PYG{p}{)}\PYG{p}{)} \PYG{c+c1}{\PYGZsh{}equation for 0th\PYGZhy{}degradation kinetics}
    \PYG{n}{plt}\PYG{o}{.}\PYG{n}{plot}\PYG{p}{(}\PYG{n}{t}\PYG{p}{,} \PYG{n}{y}\PYG{p}{)}
    \PYG{n}{plt}\PYG{o}{.}\PYG{n}{ylabel}\PYG{p}{(}\PYG{l+s+s1}{\PYGZsq{}}\PYG{l+s+s1}{solute concentration [mg/l]}\PYG{l+s+s1}{\PYGZsq{}}\PYG{p}{)}
    \PYG{n}{plt}\PYG{o}{.}\PYG{n}{title}\PYG{p}{(}\PYG{l+s+s1}{\PYGZsq{}}\PYG{l+s+s1}{radioactive decay of cobalt}\PYG{l+s+s1}{\PYGZsq{}}\PYG{p}{)}
    \PYG{n}{plt}\PYG{o}{.}\PYG{n}{ylim}\PYG{p}{(}\PYG{l+m+mi}{0}\PYG{p}{,}\PYG{l+m+mi}{100}\PYG{p}{)}
    \PYG{n}{plt}\PYG{o}{.}\PYG{n}{xlabel}\PYG{p}{(}\PYG{l+s+s1}{\PYGZsq{}}\PYG{l+s+s1}{time [a]}\PYG{l+s+s1}{\PYGZsq{}}\PYG{p}{)}
    \PYG{n}{plt}\PYG{o}{.}\PYG{n}{xlim}\PYG{p}{(}\PYG{l+m+mi}{0}\PYG{p}{,}\PYG{l+m+mi}{200}\PYG{p}{)}
    \PYG{n}{plt}\PYG{o}{.}\PYG{n}{show}\PYG{p}{(}\PYG{p}{)}

    
\PYG{n}{interact}\PYG{p}{(}\PYG{n}{Co}\PYG{p}{,} \PYG{n}{C0} \PYG{o}{=} \PYG{n}{widgets}\PYG{o}{.}\PYG{n}{IntSlider}\PYG{p}{(}\PYG{n+nb}{min}\PYG{o}{=}\PYG{l+m+mi}{0}\PYG{p}{,} \PYG{n+nb}{max}\PYG{o}{=} \PYG{l+m+mi}{100}\PYG{p}{,} \PYG{n}{step}\PYG{o}{=}\PYG{l+m+mi}{1}\PYG{p}{,} \PYG{n}{value}\PYG{o}{=}\PYG{l+m+mi}{100}\PYG{p}{)}\PYG{p}{,}
         \PYG{n}{t} \PYG{o}{=} \PYG{n}{fixed}\PYG{p}{(}\PYG{l+m+mi}{1000}\PYG{p}{)}\PYG{p}{,}
         \PYG{n}{λ} \PYG{o}{=} \PYG{n}{widgets}\PYG{o}{.}\PYG{n}{FloatSlider}\PYG{p}{(}\PYG{n}{value}\PYG{o}{=}\PYG{l+m+mf}{0.132}\PYG{p}{,} \PYG{n+nb}{min}\PYG{o}{=}\PYG{l+m+mi}{0}\PYG{p}{,} \PYG{n+nb}{max}\PYG{o}{=}\PYG{l+m+mi}{1}\PYG{p}{,} \PYG{n}{step}\PYG{o}{=}\PYG{l+m+mf}{0.001}\PYG{p}{,} \PYG{n}{readout\PYGZus{}format}\PYG{o}{=}\PYG{l+s+s1}{\PYGZsq{}}\PYG{l+s+s1}{.3f}\PYG{l+s+s1}{\PYGZsq{}}\PYG{p}{)}\PYG{p}{)}

\PYG{k}{def} \PYG{n+nf}{Sr}\PYG{p}{(}\PYG{n}{C0}\PYG{p}{,}\PYG{n}{λ}\PYG{p}{,}\PYG{n}{t}\PYG{p}{)}\PYG{p}{:} \PYG{c+c1}{\PYGZsh{}define the function for the decay of strontium}
    \PYG{n}{plt}\PYG{o}{.}\PYG{n}{figure}\PYG{p}{(}\PYG{p}{)}
    \PYG{n}{t} \PYG{o}{=} \PYG{n}{np}\PYG{o}{.}\PYG{n}{linspace}\PYG{p}{(}\PYG{l+m+mi}{0}\PYG{p}{,} \PYG{n}{t}\PYG{p}{,} \PYG{l+m+mi}{1000}\PYG{p}{)}
    \PYG{n}{y}\PYG{o}{=} \PYG{n}{C0} \PYG{o}{*} \PYG{n}{np}\PYG{o}{.}\PYG{n}{exp}\PYG{p}{(}\PYG{o}{\PYGZhy{}}\PYG{p}{(}\PYG{n}{λ} \PYG{o}{*} \PYG{n}{t}\PYG{p}{)}\PYG{p}{)}
    \PYG{n}{plt}\PYG{o}{.}\PYG{n}{plot}\PYG{p}{(}\PYG{n}{t}\PYG{p}{,} \PYG{n}{y}\PYG{p}{)}
    \PYG{n}{plt}\PYG{o}{.}\PYG{n}{ylabel}\PYG{p}{(}\PYG{l+s+s1}{\PYGZsq{}}\PYG{l+s+s1}{solute concentration [mg/l]}\PYG{l+s+s1}{\PYGZsq{}}\PYG{p}{)}
    \PYG{n}{plt}\PYG{o}{.}\PYG{n}{title}\PYG{p}{(}\PYG{l+s+s1}{\PYGZsq{}}\PYG{l+s+s1}{radioactive decay of strontium}\PYG{l+s+s1}{\PYGZsq{}}\PYG{p}{)}
    \PYG{n}{plt}\PYG{o}{.}\PYG{n}{ylim}\PYG{p}{(}\PYG{l+m+mi}{0}\PYG{p}{,}\PYG{l+m+mi}{100}\PYG{p}{)}
    \PYG{n}{plt}\PYG{o}{.}\PYG{n}{xlabel}\PYG{p}{(}\PYG{l+s+s1}{\PYGZsq{}}\PYG{l+s+s1}{time [a]}\PYG{l+s+s1}{\PYGZsq{}}\PYG{p}{)}
    \PYG{n}{plt}\PYG{o}{.}\PYG{n}{xlim}\PYG{p}{(}\PYG{l+m+mi}{0}\PYG{p}{,}\PYG{l+m+mi}{200}\PYG{p}{)}
    

\PYG{n}{interact}\PYG{p}{(}\PYG{n}{Sr}\PYG{p}{,} \PYG{n}{C0} \PYG{o}{=} \PYG{n}{widgets}\PYG{o}{.}\PYG{n}{IntSlider}\PYG{p}{(}\PYG{n+nb}{min}\PYG{o}{=}\PYG{l+m+mi}{0}\PYG{p}{,} \PYG{n+nb}{max}\PYG{o}{=} \PYG{l+m+mi}{100}\PYG{p}{,} \PYG{n}{step}\PYG{o}{=}\PYG{l+m+mi}{1}\PYG{p}{,} \PYG{n}{value}\PYG{o}{=}\PYG{l+m+mi}{100}\PYG{p}{)}\PYG{p}{,}
         \PYG{n}{t}\PYG{o}{=} \PYG{n}{fixed}\PYG{p}{(}\PYG{l+m+mi}{1000}\PYG{p}{)}\PYG{p}{,}
         \PYG{n}{λ}\PYG{o}{=}\PYG{n}{widgets}\PYG{o}{.}\PYG{n}{FloatSlider}\PYG{p}{(}\PYG{n+nb}{min}\PYG{o}{=}\PYG{l+m+mi}{0}\PYG{p}{,}\PYG{n+nb}{max}\PYG{o}{=}\PYG{l+m+mi}{1}\PYG{p}{,} \PYG{n}{value}\PYG{o}{=}\PYG{l+m+mf}{0.025}\PYG{p}{,} \PYG{n}{step}\PYG{o}{=}\PYG{l+m+mf}{0.001}\PYG{p}{,} \PYG{n}{readout\PYGZus{}format}\PYG{o}{=}\PYG{l+s+s1}{\PYGZsq{}}\PYG{l+s+s1}{.3f}\PYG{l+s+s1}{\PYGZsq{}}\PYG{p}{)}\PYG{p}{)}
\end{sphinxVerbatim}

\end{sphinxuseclass}\end{sphinxVerbatimInput}
\begin{sphinxVerbatimOutput}

\begin{sphinxuseclass}{cell_output}
\begin{sphinxVerbatim}[commandchars=\\\{\}]
interactive(children=(IntSlider(value=100, description=\PYGZsq{}C0\PYGZsq{}), FloatSlider(value=0.132, description=\PYGZsq{}λ\PYGZsq{}, max=1.…
\end{sphinxVerbatim}

\begin{sphinxVerbatim}[commandchars=\\\{\}]
interactive(children=(IntSlider(value=100, description=\PYGZsq{}C0\PYGZsq{}), FloatSlider(value=0.025, description=\PYGZsq{}λ\PYGZsq{}, max=1.…
\end{sphinxVerbatim}

\begin{sphinxVerbatim}[commandchars=\\\{\}]
\PYGZlt{}function \PYGZus{}\PYGZus{}main\PYGZus{}\PYGZus{}.Sr(C0, λ, t)\PYGZgt{}
\end{sphinxVerbatim}

\end{sphinxuseclass}\end{sphinxVerbatimOutput}

\end{sphinxuseclass}

\section{Michaelis\sphinxhyphen{}Menten\sphinxhyphen{}Kinetics}
\label{\detokenize{content/tools/Kinetics_degradation:michaelis-menten-kinetics}}
\sphinxAtStartPar
The Michaelis\sphinxhyphen{}Menten degradation kinetics behaves like a \(0^{th}\)\sphinxhyphen{}order kinetics for \sphinxstylestrong{“short” times} and like a \(1^{st}\)\sphinxhyphen{}order kinetics for \sphinxstylestrong{“long” times.} It describes the dependence of the speed of an enzyme\sphinxhyphen{}catalyzed reaction on the substrate concentration.

\begin{sphinxuseclass}{cell}\begin{sphinxVerbatimInput}

\begin{sphinxuseclass}{cell_input}
\begin{sphinxVerbatim}[commandchars=\\\{\}]
\PYG{k}{def} \PYG{n+nf}{MM1}\PYG{p}{(}\PYG{n}{C\PYGZus{}i}\PYG{p}{,} \PYG{n}{P\PYGZus{}s}\PYG{p}{,} \PYG{n}{H\PYGZus{}l}\PYG{p}{,} \PYG{n}{R\PYGZus{}f}\PYG{p}{,} \PYG{n}{n\PYGZus{}sim}\PYG{p}{)}\PYG{p}{:}
    
    \PYG{l+s+sd}{\PYGZdq{}\PYGZdq{}\PYGZdq{}}
\PYG{l+s+sd}{    C\PYGZus{}i = Initial concentration [M/L\PYGZca{}3]}
\PYG{l+s+sd}{    P\PYGZus{}s = Coefficient\PYGZhy{} shape factor [M/L\PYGZca{}3]}
\PYG{l+s+sd}{    t}
\PYG{l+s+sd}{    \PYGZdq{}\PYGZdq{}\PYGZdq{}}
    
    \PYG{c+c1}{\PYGZsh{}intermediate calculation}
    \PYG{n}{MM\PYGZus{}rc} \PYG{o}{=} \PYG{p}{(}\PYG{l+m+mf}{0.5}\PYG{o}{*}\PYG{n}{C\PYGZus{}i}\PYG{o}{+}\PYG{n}{P\PYGZus{}s}\PYG{o}{*}\PYG{n}{np}\PYG{o}{.}\PYG{n}{log}\PYG{p}{(}\PYG{l+m+mi}{2}\PYG{p}{)}\PYG{p}{)}\PYG{o}{/}\PYG{n}{H\PYGZus{}l} \PYG{c+c1}{\PYGZsh{} [M/L\PYGZca{}3/T], Michaelis\PYGZhy{}Menten rate constant}

    \PYG{n}{ZO\PYGZus{}rc} \PYG{o}{=} \PYG{n}{MM\PYGZus{}rc}\PYG{o}{*}\PYG{n}{C\PYGZus{}i}\PYG{o}{/}\PYG{p}{(}\PYG{n}{P\PYGZus{}s}\PYG{o}{+}\PYG{n}{C\PYGZus{}i}\PYG{p}{)} \PYG{c+c1}{\PYGZsh{} [M/L\PYGZca{}3/T], Zero order rate constant}

    \PYG{n}{ZO\PYGZus{}hl} \PYG{o}{=} \PYG{l+m+mf}{0.5}\PYG{o}{*}\PYG{n}{C\PYGZus{}i}\PYG{o}{/}\PYG{n}{ZO\PYGZus{}rc} \PYG{c+c1}{\PYGZsh{} [T], half\PYGZhy{}life}
    \PYG{n}{t\PYGZus{}c0} \PYG{o}{=} \PYG{l+m+mi}{2}\PYG{o}{*}\PYG{n}{ZO\PYGZus{}hl} \PYG{c+c1}{\PYGZsh{} [T] time to reach C=0}

    \PYG{n}{FO\PYGZus{}rc} \PYG{o}{=} \PYG{n}{MM\PYGZus{}rc}\PYG{o}{/}\PYG{n}{P\PYGZus{}s}
    \PYG{n}{FO\PYGZus{}hl} \PYG{o}{=} \PYG{n}{np}\PYG{o}{.}\PYG{n}{log}\PYG{p}{(}\PYG{l+m+mi}{2}\PYG{p}{)}\PYG{o}{/}\PYG{n}{FO\PYGZus{}rc} \PYG{c+c1}{\PYGZsh{} [T], half\PYGZhy{}life}
    \PYG{n}{FO\PYGZus{}ci} \PYG{o}{=} \PYG{n}{C\PYGZus{}i}\PYG{o}{*}\PYG{n}{np}\PYG{o}{.}\PYG{n}{exp}\PYG{p}{(}\PYG{n}{C\PYGZus{}i}\PYG{o}{/}\PYG{l+m+mi}{2}\PYG{p}{)} \PYG{c+c1}{\PYGZsh{} [M/L\PYGZca{}3]}

    \PYG{c+c1}{\PYGZsh{} Main Computing}

    \PYG{n}{MM} \PYG{o}{=} \PYG{n}{np}\PYG{o}{.}\PYG{n}{zeros}\PYG{p}{(}\PYG{n}{n\PYGZus{}sim}\PYG{p}{)}  \PYG{c+c1}{\PYGZsh{} creat an array with zros}

    \PYG{k}{for} \PYG{n}{i} \PYG{o+ow}{in} \PYG{n+nb}{range}\PYG{p}{(}\PYG{l+m+mi}{0}\PYG{p}{,} \PYG{n}{n\PYGZus{}sim}\PYG{o}{\PYGZhy{}}\PYG{l+m+mi}{1}\PYG{p}{)}\PYG{p}{:}
        \PYG{n}{MM}\PYG{p}{[}\PYG{l+m+mi}{0}\PYG{p}{]} \PYG{o}{=} \PYG{n}{C\PYGZus{}i}
        \PYG{n}{MM}\PYG{p}{[}\PYG{n}{i}\PYG{o}{+}\PYG{l+m+mi}{1}\PYG{p}{]} \PYG{o}{=} \PYG{n}{MM}\PYG{p}{[}\PYG{n}{i}\PYG{p}{]}\PYG{o}{*}\PYG{n}{R\PYGZus{}f}
\PYG{c+c1}{\PYGZsh{}        MM[i]}

    \PYG{n}{time} \PYG{o}{=} \PYG{p}{(}\PYG{n}{C\PYGZus{}i}\PYG{o}{\PYGZhy{}}\PYG{n}{MM}\PYG{p}{)}\PYG{o}{/}\PYG{n}{MM\PYGZus{}rc} \PYG{o}{\PYGZhy{}} \PYG{n}{P\PYGZus{}s}\PYG{o}{/}\PYG{n}{MM\PYGZus{}rc} \PYG{o}{*} \PYG{n}{np}\PYG{o}{.}\PYG{n}{log}\PYG{p}{(}\PYG{n}{MM}\PYG{o}{/}\PYG{n}{C\PYGZus{}i}\PYG{p}{)}

    \PYG{n}{ZO} \PYG{o}{=} \PYG{n}{C\PYGZus{}i}\PYG{o}{\PYGZhy{}}\PYG{n}{ZO\PYGZus{}rc}\PYG{o}{*}\PYG{n}{time}
    \PYG{n}{ZO}\PYG{p}{[}\PYG{n}{ZO} \PYG{o}{\PYGZlt{}} \PYG{l+m+mi}{0}\PYG{p}{]} \PYG{o}{=} \PYG{l+m+mi}{0} \PYG{c+c1}{\PYGZsh{} forcing \PYGZhy{}ve conc. to be zero}

    \PYG{n}{FO} \PYG{o}{=} \PYG{n}{FO\PYGZus{}ci}\PYG{o}{*}\PYG{n}{np}\PYG{o}{.}\PYG{n}{exp}\PYG{p}{(}\PYG{o}{\PYGZhy{}}\PYG{n}{FO\PYGZus{}rc}\PYG{o}{*}\PYG{n}{time}\PYG{p}{)}
    \PYG{n}{FO}\PYG{p}{[}\PYG{n}{FO} \PYG{o}{\PYGZlt{}} \PYG{l+m+mi}{0}\PYG{p}{]} \PYG{o}{=} \PYG{l+m+mi}{0} \PYG{c+c1}{\PYGZsh{} forcing \PYGZhy{}ve conc. to be zero}

    
\PYG{c+c1}{\PYGZsh{}    dict1 = \PYGZob{}\PYGZdq{}time [T]\PYGZdq{}: time, \PYGZdq{}Michaelis\PYGZhy{}Menten\PYGZdq{}: MM[i], \PYGZdq{}Zero\PYGZhy{}Order\PYGZdq{}:ZO, \PYGZdq{}First\PYGZhy{}Order\PYGZdq{}: FO\PYGZcb{}}
\PYG{c+c1}{\PYGZsh{}    pd.DataFrame(dict1)}

    \PYG{n}{plt}\PYG{o}{.}\PYG{n}{plot}\PYG{p}{(}\PYG{n}{time}\PYG{p}{,} \PYG{n}{MM}\PYG{p}{,} \PYG{l+s+s2}{\PYGZdq{}}\PYG{l+s+s2}{\PYGZhy{}r}\PYG{l+s+s2}{\PYGZdq{}}\PYG{p}{,} \PYG{n}{label}\PYG{o}{=}\PYG{l+s+s2}{\PYGZdq{}}\PYG{l+s+s2}{Michaelis Menten}\PYG{l+s+s2}{\PYGZdq{}}\PYG{p}{)}
    \PYG{n}{plt}\PYG{o}{.}\PYG{n}{plot}\PYG{p}{(}\PYG{n}{time}\PYG{p}{,} \PYG{n}{FO}\PYG{p}{,}\PYG{l+s+s2}{\PYGZdq{}}\PYG{l+s+s2}{\PYGZhy{}g}\PYG{l+s+s2}{\PYGZdq{}}\PYG{p}{,} \PYG{n}{label}\PYG{o}{=}\PYG{l+s+s2}{\PYGZdq{}}\PYG{l+s+s2}{First Order}\PYG{l+s+s2}{\PYGZdq{}}\PYG{p}{)}
    \PYG{n}{plt}\PYG{o}{.}\PYG{n}{plot}\PYG{p}{(}\PYG{n}{time}\PYG{p}{,} \PYG{n}{ZO}\PYG{p}{,}\PYG{l+s+s2}{\PYGZdq{}}\PYG{l+s+s2}{\PYGZhy{}b}\PYG{l+s+s2}{\PYGZdq{}}\PYG{p}{,} \PYG{n}{label}\PYG{o}{=}\PYG{l+s+s2}{\PYGZdq{}}\PYG{l+s+s2}{Zero Order}\PYG{l+s+s2}{\PYGZdq{}}\PYG{p}{)}
    \PYG{n}{plt}\PYG{o}{.}\PYG{n}{xlabel}\PYG{p}{(}\PYG{l+s+s2}{\PYGZdq{}}\PYG{l+s+s2}{time [T]}\PYG{l+s+s2}{\PYGZdq{}}\PYG{p}{)}\PYG{p}{;} \PYG{n}{plt}\PYG{o}{.}\PYG{n}{ylabel}\PYG{p}{(}\PYG{l+s+sa}{r}\PYG{l+s+s2}{\PYGZdq{}}\PYG{l+s+s2}{Concentration [M/L\PYGZdl{}\PYGZca{}3\PYGZdl{}]}\PYG{l+s+s2}{\PYGZdq{}}\PYG{p}{)}
    \PYG{n}{plt}\PYG{o}{.}\PYG{n}{legend}\PYG{p}{(}\PYG{p}{)}    
    \PYG{n}{plt}\PYG{o}{.}\PYG{n}{grid}\PYG{p}{(}\PYG{p}{)}


\PYG{n}{interact}\PYG{p}{(}\PYG{n}{MM1}\PYG{p}{,} 
         \PYG{n}{C\PYGZus{}i} \PYG{o}{=} \PYG{n}{widgets}\PYG{o}{.}\PYG{n}{FloatSlider}\PYG{p}{(}\PYG{n+nb}{min}\PYG{o}{=}\PYG{l+m+mf}{0.001}\PYG{p}{,} \PYG{n+nb}{max}\PYG{o}{=} \PYG{l+m+mf}{10.}\PYG{p}{,} \PYG{n}{step}\PYG{o}{=}\PYG{l+m+mf}{0.01}\PYG{p}{,} \PYG{n}{value}\PYG{o}{=}\PYG{l+m+mf}{1.0}\PYG{p}{,} \PYG{n}{readout\PYGZus{}format}\PYG{o}{=}\PYG{l+s+s1}{\PYGZsq{}}\PYG{l+s+s1}{.2f}\PYG{l+s+s1}{\PYGZsq{}}\PYG{p}{)}\PYG{p}{,}
         \PYG{n}{P\PYGZus{}s} \PYG{o}{=} \PYG{n}{widgets}\PYG{o}{.}\PYG{n}{IntSlider}\PYG{p}{(}\PYG{n+nb}{min}\PYG{o}{=}\PYG{l+m+mi}{1}\PYG{p}{,} \PYG{n+nb}{max}\PYG{o}{=} \PYG{l+m+mi}{10}\PYG{p}{,} \PYG{n}{step}\PYG{o}{=}\PYG{l+m+mi}{1}\PYG{p}{,} \PYG{n}{value}\PYG{o}{=}\PYG{l+m+mi}{2}\PYG{p}{)}\PYG{p}{,}
         \PYG{n}{H\PYGZus{}l} \PYG{o}{=} \PYG{n}{widgets}\PYG{o}{.}\PYG{n}{FloatSlider}\PYG{p}{(}\PYG{n+nb}{min}\PYG{o}{=}\PYG{l+m+mf}{1.}\PYG{p}{,} \PYG{n+nb}{max}\PYG{o}{=} \PYG{l+m+mf}{1000.}\PYG{p}{,} \PYG{n}{step}\PYG{o}{=}\PYG{l+m+mf}{1.}\PYG{p}{,} \PYG{n}{value}\PYG{o}{=}\PYG{l+m+mf}{300.}\PYG{p}{,} \PYG{n}{readout\PYGZus{}format}\PYG{o}{=}\PYG{l+s+s1}{\PYGZsq{}}\PYG{l+s+s1}{.1f}\PYG{l+s+s1}{\PYGZsq{}}\PYG{p}{)}\PYG{p}{,}
         \PYG{n}{R\PYGZus{}f} \PYG{o}{=} \PYG{n}{widgets}\PYG{o}{.}\PYG{n}{FloatSlider}\PYG{p}{(}\PYG{n+nb}{min}\PYG{o}{=}\PYG{l+m+mf}{0.01}\PYG{p}{,} \PYG{n+nb}{max}\PYG{o}{=} \PYG{l+m+mf}{1.}\PYG{p}{,} \PYG{n}{step}\PYG{o}{=}\PYG{l+m+mf}{0.1}\PYG{p}{,} \PYG{n}{value}\PYG{o}{=}\PYG{l+m+mf}{0.97}\PYG{p}{,} \PYG{n}{readout\PYGZus{}format}\PYG{o}{=}\PYG{l+s+s1}{\PYGZsq{}}\PYG{l+s+s1}{.2f}\PYG{l+s+s1}{\PYGZsq{}}\PYG{p}{)}\PYG{p}{,}
         \PYG{n}{n\PYGZus{}sim} \PYG{o}{=} \PYG{n}{widgets}\PYG{o}{.}\PYG{n}{IntSlider}\PYG{p}{(}\PYG{n+nb}{min}\PYG{o}{=}\PYG{l+m+mi}{1}\PYG{p}{,} \PYG{n+nb}{max}\PYG{o}{=} \PYG{l+m+mi}{1000}\PYG{p}{,} \PYG{n}{step}\PYG{o}{=}\PYG{l+m+mi}{1}\PYG{p}{,} \PYG{n}{value}\PYG{o}{=}\PYG{l+m+mi}{150}\PYG{p}{)}\PYG{p}{)}
\end{sphinxVerbatim}

\end{sphinxuseclass}\end{sphinxVerbatimInput}
\begin{sphinxVerbatimOutput}

\begin{sphinxuseclass}{cell_output}
\begin{sphinxVerbatim}[commandchars=\\\{\}]
interactive(children=(FloatSlider(value=1.0, description=\PYGZsq{}C\PYGZus{}i\PYGZsq{}, max=10.0, min=0.001, step=0.01), IntSlider(val…
\end{sphinxVerbatim}

\begin{sphinxVerbatim}[commandchars=\\\{\}]
\PYGZlt{}function \PYGZus{}\PYGZus{}main\PYGZus{}\PYGZus{}.MM1(C\PYGZus{}i, P\PYGZus{}s, H\PYGZus{}l, R\PYGZus{}f, n\PYGZus{}sim)\PYGZgt{}
\end{sphinxVerbatim}

\end{sphinxuseclass}\end{sphinxVerbatimOutput}

\end{sphinxuseclass}
\sphinxstepscope


\chapter{Simulating 1D Trench Flow*}
\label{\detokenize{content/tools/1D_ditchflow:simulating-1d-trench-flow}}\label{\detokenize{content/tools/1D_ditchflow::doc}}
\sphinxAtStartPar
The tool simulates the effect on the water table due to model parameters.


\section{Scenario and Equation}
\label{\detokenize{content/tools/1D_ditchflow:scenario-and-equation}}
\begin{figure}[htbp]
\centering
\capstart

\noindent\sphinxincludegraphics[scale=0.2]{{contents/modeling/lecture_11/images/M11_f4}.png}
\caption{Conceptual model of a flow between two water bodies separated by unconfined aquifer}\label{\detokenize{content/tools/1D_ditchflow:ditch}}\end{figure}

\sphinxAtStartPar
You can calculate the steady flow in an unconfined aquifer with this Equations%
\begin{footnote}[1]\sphinxAtStartFootnote
C. W. Fetter, Thomas Boving, David Kreamer (2017), \sphinxstyleemphasis{Contaminant Hydrogeology}: Third Edition, Waveland Press
%
\end{footnote} :
\begin{equation*}
\begin{split}q' = \frac{1}{2} \cdot K \cdot \frac{H_o^2-H_u^2}{L}\end{split}
\end{equation*}\begin{equation*}
\begin{split}h(x)=\sqrt{H_o^2 - \frac{H_o^2-H_u^2}{L} \cdot x+\frac{R}{K} \cdot x \cdot(L-x)}\end{split}
\end{equation*}
\sphinxAtStartPar
with\(q'\) = flow per unit width \([m^2/s]\),
\(h\) = head at x \([m]\),\(x\) = distance from the origin \([m]\),
\(H_o\) = head at the origin \([m]\),
\(H_u\) = head at L \([m]\),
\(L\) = distance from the origin at the point \(H_u\) is measured \([m]\),
\(K\) = hydraulic conductivity \([m/s]\),
\(R\) = recharge rate \([m/s]\)

\sphinxAtStartPar
\sphinxstylestrong{\sphinxstyleemphasis{Contributed by Ms. Anne Pförtner and Sophie Pförtner. The original concept from Prof. R. Liedl spreasheet code.}}


\subsection{How to use this tool}
\label{\detokenize{content/tools/1D_ditchflow:how-to-use-this-tool}}\begin{enumerate}
\sphinxsetlistlabels{\arabic}{enumi}{enumii}{}{.}%
\item {} 
\sphinxAtStartPar
Go to the Binder by clicking the rocket button (top\sphinxhyphen{}right of the page)

\item {} 
\sphinxAtStartPar
Execute the code cell with libraries and the code cell

\item {} 
\sphinxAtStartPar
Interact with the sliders.

\end{enumerate}

\sphinxAtStartPar
The codes are licensed under CC by 4.0 \sphinxhref{https://creativecommons.org/licenses/by/4.0/deed.en}{(use anyways, but acknowledge the original work)}

\begin{sphinxuseclass}{cell}\begin{sphinxVerbatimInput}

\begin{sphinxuseclass}{cell_input}
\begin{sphinxVerbatim}[commandchars=\\\{\}]
\PYG{c+c1}{\PYGZsh{} Initialize librarys}
\PYG{k+kn}{import} \PYG{n+nn}{matplotlib}\PYG{n+nn}{.}\PYG{n+nn}{pyplot} \PYG{k}{as} \PYG{n+nn}{plt}
\PYG{k+kn}{import} \PYG{n+nn}{numpy} \PYG{k}{as} \PYG{n+nn}{np}
\PYG{k+kn}{import} \PYG{n+nn}{math}
\PYG{k+kn}{from} \PYG{n+nn}{ipywidgets} \PYG{k+kn}{import} \PYG{o}{*}
\end{sphinxVerbatim}

\end{sphinxuseclass}\end{sphinxVerbatimInput}

\end{sphinxuseclass}
\begin{sphinxuseclass}{cell}\begin{sphinxVerbatimInput}

\begin{sphinxuseclass}{cell_input}
\begin{sphinxVerbatim}[commandchars=\\\{\}]
\PYG{c+c1}{\PYGZsh{} Definition of the function}
\PYG{k}{def} \PYG{n+nf}{head}\PYG{p}{(}\PYG{n}{Ho}\PYG{p}{,} \PYG{n}{Hu}\PYG{p}{,} \PYG{n}{L}\PYG{p}{,} \PYG{n}{R}\PYG{p}{,} \PYG{n}{K}\PYG{p}{)}\PYG{p}{:}

    \PYG{l+s+sd}{\PYGZdq{}\PYGZdq{}\PYGZdq{}}
\PYG{l+s+sd}{    Ho: inflow head in [m]}
\PYG{l+s+sd}{    Hu: outflow head in [m]}
\PYG{l+s+sd}{    L:  Domain length in [m]}
\PYG{l+s+sd}{    R:  Recharge rate in [mm/d]}
\PYG{l+s+sd}{    K: Hydraulic conductivity in [m/s]}
\PYG{l+s+sd}{    \PYGZdq{}\PYGZdq{}\PYGZdq{}}
    \PYG{n}{x} \PYG{o}{=} \PYG{n}{np}\PYG{o}{.}\PYG{n}{arange}\PYG{p}{(}\PYG{l+m+mi}{0}\PYG{p}{,} \PYG{n}{L}\PYG{p}{,}\PYG{n}{L}\PYG{o}{/}\PYG{l+m+mi}{1000}\PYG{p}{)}
    \PYG{n}{R}\PYG{o}{=}\PYG{n}{R}\PYG{o}{/}\PYG{l+m+mi}{1000}\PYG{o}{/}\PYG{l+m+mf}{365.25}\PYG{o}{/}\PYG{l+m+mi}{86400}
    \PYG{n}{h}\PYG{o}{=}\PYG{p}{(}\PYG{n}{Ho}\PYG{o}{*}\PYG{o}{*}\PYG{l+m+mi}{2}\PYG{o}{\PYGZhy{}}\PYG{p}{(}\PYG{n}{Ho}\PYG{o}{*}\PYG{o}{*}\PYG{l+m+mi}{2}\PYG{o}{\PYGZhy{}}\PYG{n}{Hu}\PYG{o}{*}\PYG{o}{*}\PYG{l+m+mi}{2}\PYG{p}{)}\PYG{o}{/}\PYG{n}{L}\PYG{o}{*}\PYG{n}{x}\PYG{o}{+}\PYG{p}{(}\PYG{n}{R}\PYG{o}{/}\PYG{n}{K}\PYG{o}{*}\PYG{n}{x}\PYG{o}{*}\PYG{p}{(}\PYG{n}{L}\PYG{o}{\PYGZhy{}}\PYG{n}{x}\PYG{p}{)}\PYG{p}{)}\PYG{p}{)}\PYG{o}{*}\PYG{o}{*}\PYG{l+m+mf}{0.5}
    \PYG{n}{plt}\PYG{o}{.}\PYG{n}{plot}\PYG{p}{(}\PYG{n}{x}\PYG{p}{,} \PYG{n}{h}\PYG{p}{)}
    \PYG{n}{plt}\PYG{o}{.}\PYG{n}{ylabel}\PYG{p}{(}\PYG{l+s+s1}{\PYGZsq{}}\PYG{l+s+s1}{head [m]}\PYG{l+s+s1}{\PYGZsq{}}\PYG{p}{)}
    \PYG{n}{plt}\PYG{o}{.}\PYG{n}{ylim}\PYG{p}{(}\PYG{l+m+mi}{0}\PYG{p}{,}\PYG{l+m+mf}{1.5}\PYG{o}{*}\PYG{n}{Ho}\PYG{p}{)}
    \PYG{n}{plt}\PYG{o}{.}\PYG{n}{xlabel}\PYG{p}{(}\PYG{l+s+s1}{\PYGZsq{}}\PYG{l+s+s1}{x [m]}\PYG{l+s+s1}{\PYGZsq{}}\PYG{p}{)}
    \PYG{n}{plt}\PYG{o}{.}\PYG{n}{xlim}\PYG{p}{(}\PYG{l+m+mi}{0}\PYG{p}{,}\PYG{n}{L}\PYG{p}{)}
    \PYG{n}{plt}\PYG{o}{.}\PYG{n}{show}\PYG{p}{(}\PYG{p}{)}

\PYG{n}{style} \PYG{o}{=} \PYG{p}{\PYGZob{}}\PYG{l+s+s1}{\PYGZsq{}}\PYG{l+s+s1}{description\PYGZus{}width}\PYG{l+s+s1}{\PYGZsq{}}\PYG{p}{:} \PYG{l+s+s1}{\PYGZsq{}}\PYG{l+s+s1}{initial}\PYG{l+s+s1}{\PYGZsq{}}\PYG{p}{\PYGZcb{}}  
\PYG{n}{interact}\PYG{p}{(}\PYG{n}{head}\PYG{p}{,}
         \PYG{n}{Ho}\PYG{o}{=}\PYG{n}{widgets}\PYG{o}{.}\PYG{n}{BoundedFloatText}\PYG{p}{(}\PYG{n}{value}\PYG{o}{=}\PYG{l+m+mi}{10}\PYG{p}{,} \PYG{n+nb}{min}\PYG{o}{=}\PYG{l+m+mi}{0}\PYG{p}{,} \PYG{n+nb}{max}\PYG{o}{=}\PYG{l+m+mi}{1000}\PYG{p}{,} \PYG{n}{step}\PYG{o}{=}\PYG{l+m+mf}{0.1}\PYG{p}{,} \PYG{n}{description}\PYG{o}{=}\PYG{l+s+s1}{\PYGZsq{}}\PYG{l+s+s1}{Ho:}\PYG{l+s+s1}{\PYGZsq{}}\PYG{p}{,} \PYG{n}{disabled}\PYG{o}{=}\PYG{k+kc}{False}\PYG{p}{)}\PYG{p}{,}
         \PYG{n}{Hu}\PYG{o}{=}\PYG{n}{widgets}\PYG{o}{.}\PYG{n}{BoundedFloatText}\PYG{p}{(}\PYG{n}{value}\PYG{o}{=}\PYG{l+m+mf}{7.5}\PYG{p}{,} \PYG{n+nb}{min}\PYG{o}{=}\PYG{l+m+mi}{0}\PYG{p}{,} \PYG{n+nb}{max}\PYG{o}{=}\PYG{l+m+mi}{1000}\PYG{p}{,} \PYG{n}{step}\PYG{o}{=}\PYG{l+m+mf}{0.1}\PYG{p}{,} \PYG{n}{description}\PYG{o}{=}\PYG{l+s+s1}{\PYGZsq{}}\PYG{l+s+s1}{Hu:}\PYG{l+s+s1}{\PYGZsq{}}\PYG{p}{,} \PYG{n}{disabled}\PYG{o}{=}\PYG{k+kc}{False}\PYG{p}{)}\PYG{p}{,}
         \PYG{n}{L}\PYG{o}{=} \PYG{n}{widgets}\PYG{o}{.}\PYG{n}{BoundedFloatText}\PYG{p}{(}\PYG{n}{value}\PYG{o}{=}\PYG{l+m+mi}{175}\PYG{p}{,}\PYG{n+nb}{min}\PYG{o}{=}\PYG{l+m+mi}{0}\PYG{p}{,} \PYG{n+nb}{max}\PYG{o}{=}\PYG{l+m+mi}{10000}\PYG{p}{,}\PYG{n}{step}\PYG{o}{=}\PYG{l+m+mi}{1}\PYG{p}{,} \PYG{n}{description}\PYG{o}{=}\PYG{l+s+s1}{\PYGZsq{}}\PYG{l+s+s1}{L:}\PYG{l+s+s1}{\PYGZsq{}} \PYG{p}{,} \PYG{n}{disabled}\PYG{o}{=}\PYG{k+kc}{False}\PYG{p}{)}\PYG{p}{,}
         \PYG{n}{R}\PYG{o}{=}\PYG{p}{(}\PYG{o}{\PYGZhy{}}\PYG{l+m+mi}{500}\PYG{p}{,}\PYG{l+m+mi}{2500}\PYG{p}{,}\PYG{l+m+mi}{10}\PYG{p}{)}\PYG{p}{,}
         \PYG{n}{K}\PYG{o}{=}\PYG{n}{widgets}\PYG{o}{.}\PYG{n}{FloatLogSlider}\PYG{p}{(}\PYG{n}{value}\PYG{o}{=}\PYG{l+m+mf}{0.0002}\PYG{p}{,}\PYG{n}{base}\PYG{o}{=}\PYG{l+m+mi}{10}\PYG{p}{,}\PYG{n+nb}{min}\PYG{o}{=}\PYG{o}{\PYGZhy{}}\PYG{l+m+mi}{6}\PYG{p}{,} \PYG{n+nb}{max}\PYG{o}{=}\PYG{o}{\PYGZhy{}}\PYG{l+m+mi}{2}\PYG{p}{,} \PYG{n}{step}\PYG{o}{=}\PYG{l+m+mf}{0.0001}\PYG{p}{,}\PYG{n}{readout\PYGZus{}format}\PYG{o}{=}\PYG{l+s+s1}{\PYGZsq{}}\PYG{l+s+s1}{.2e}\PYG{l+s+s1}{\PYGZsq{}}\PYG{p}{)}\PYG{p}{)}
\end{sphinxVerbatim}

\end{sphinxuseclass}\end{sphinxVerbatimInput}
\begin{sphinxVerbatimOutput}

\begin{sphinxuseclass}{cell_output}
\begin{sphinxVerbatim}[commandchars=\\\{\}]
interactive(children=(BoundedFloatText(value=10.0, description=\PYGZsq{}Ho:\PYGZsq{}, max=1000.0, step=0.1), BoundedFloatText(…
\end{sphinxVerbatim}

\begin{sphinxVerbatim}[commandchars=\\\{\}]
\PYGZlt{}function \PYGZus{}\PYGZus{}main\PYGZus{}\PYGZus{}.head(Ho, Hu, L, R, K)\PYGZgt{}
\end{sphinxVerbatim}

\end{sphinxuseclass}\end{sphinxVerbatimOutput}

\end{sphinxuseclass}

\bigskip\hrule\bigskip


\sphinxstepscope


\chapter{Testing the code for run}
\label{\detokenize{content/testcodeer:testing-the-code-for-run}}\label{\detokenize{content/testcodeer::doc}}
















\sphinxstepscope


\chapter{NEw Run}
\label{\detokenize{content/tester:new-run}}\label{\detokenize{content/tester::doc}}


\sphinxstepscope


\part{Previous Years Q\&A}

\sphinxstepscope

\begin{sphinxuseclass}{cell}
\begin{sphinxuseclass}{tag_remove-output}
\begin{sphinxuseclass}{tag_remove-input}
\end{sphinxuseclass}
\end{sphinxuseclass}
\end{sphinxuseclass}

\chapter{Groundwater Exam Solution \sphinxhyphen{}  2019\sphinxhyphen{}2020}
\label{\detokenize{content/Q_and_A/GW_exam_2019_20:groundwater-exam-solution-2019-2020}}\label{\detokenize{content/Q_and_A/GW_exam_2019_20::doc}}
\sphinxAtStartPar
\sphinxstyleemphasis{(The contents presented in this section were re\sphinxhyphen{}developed principally by Dr. P. K. Yadav with supervision from Prof. Rudolf Liedl)}

\sphinxAtStartPar
\sphinxstylestrong{Q1. Aquifer Types}   (ca. 5 pts.)

\sphinxAtStartPar
a. Differentiate between Aquifer, Aquitard and Aquiclude (3 points)

\sphinxAtStartPar
b. Schematically present a confined aquifer (vertical cross\sphinxhyphen{}section) providing essential features with their legends (2 points)

\sphinxAtStartPar
\sphinxstylestrong{Solution 1. a.}

\sphinxAtStartPar
See slide: L03/08

\sphinxAtStartPar
An \sphinxstylestrong{aquifer} or a groundwater reservoir can store and transmit significant (= exploitable) amounts of groundwater.

\sphinxAtStartPar
An \sphinxstylestrong{aquitard} can store and transmit groundwater but to a much lesser extent than an (adjacent) aquifer.

\sphinxAtStartPar
An \sphinxstylestrong{aquiclude} can store groundwater but cannot transmit groundwater.

\sphinxAtStartPar
\sphinxstylestrong{Solution 1b} \sphinxhyphen{} (L03/11)

\sphinxAtStartPar
 



\sphinxAtStartPar
\sphinxstylestrong{Confined Aquifer}
\begin{enumerate}
\sphinxsetlistlabels{\arabic}{enumi}{enumii}{}{.}%
\item {} 
\sphinxAtStartPar
The essential feature of confined aquifer is provided in the figure above.

\end{enumerate}

\begin{sphinxuseclass}{cell}
\begin{sphinxuseclass}{tag_remove-input}
\begin{sphinxuseclass}{tag_remove-output}
\end{sphinxuseclass}
\end{sphinxuseclass}
\end{sphinxuseclass}
\begin{sphinxuseclass}{cell}
\begin{sphinxuseclass}{tag_remove-input}\begin{sphinxVerbatimOutput}

\begin{sphinxuseclass}{cell_output}
\begin{sphinxVerbatim}[commandchars=\\\{\}]

\end{sphinxVerbatim}

\begin{sphinxVerbatim}[commandchars=\\\{\}]
\PYGZlt{}IPython.core.display.HTML object\PYGZgt{}
\end{sphinxVerbatim}

\begin{sphinxVerbatim}[commandchars=\\\{\}]
\PYGZlt{}IPython.core.display.HTML object\PYGZgt{}
\end{sphinxVerbatim}

\begin{sphinxVerbatim}[commandchars=\\\{\}]
\PYGZlt{}IPython.core.display.Javascript object\PYGZgt{}
\end{sphinxVerbatim}

\end{sphinxuseclass}\end{sphinxVerbatimOutput}

\end{sphinxuseclass}
\end{sphinxuseclass}
\sphinxAtStartPar
\sphinxstylestrong{Q2. Groundwater storage (3 pts.)}

\sphinxAtStartPar
Dry season in Dresden (2018\sphinxhyphen{}19) led to intense extraction of groundawater in rural areas. At one location in a confined aquifer (storage coeff. 4·10\sphinxhyphen{}4, total porosity = 30\%),  the pressure head was lowered by 150 m. The thickness of the aquifer was measured to be 90 m before the beginning of extraction and the compressibility of the porous medium in that region is estimated 6·10\sphinxhyphen{}8 m2/N. Density of water can be assumed to be 1000 kg/m3.

\sphinxAtStartPar
(Hint: \(\Delta V_T = \alpha_{pm}\cdot\rho_w \cdot g \cdot  V_T \cdot  \Delta \psi \)).

\sphinxAtStartPar
a. Approximately how much water was extracted? (1 point)

\sphinxAtStartPar
b. How much land subsidence due to water extraction is expected? (2 point)

\sphinxAtStartPar
\sphinxstylestrong{Solution 2}

\sphinxAtStartPar
Given relation:

\sphinxAtStartPar
For part a. (see Tut 02/P4)

\sphinxAtStartPar
\(
S_s  = \frac{\Delta V_w}{V_T\cdot \Delta \psi}
\)

\sphinxAtStartPar
In confined aquifer \(S\) is used, which is obtained from:

\sphinxAtStartPar
\(S = S_s \cdot m\)

\sphinxAtStartPar
\(
 \frac{S}{m} = \frac{\Delta V_w}{A_T\cdot m \cdot \Delta \psi}
\)

\sphinxAtStartPar
So,

\sphinxAtStartPar
\(  \frac{\Delta V_w}{A}  = S \cdot \Delta \psi\)

\sphinxAtStartPar
For part b.

\sphinxAtStartPar
\(\Delta V_T = \alpha_{pm}\cdot\rho_w \cdot g \cdot  V_T \cdot  \Delta \psi \)

\sphinxAtStartPar
\(V_T = A\times h\), with \(A\) surface area and \(h\) aquifer thickness.

\sphinxAtStartPar
\(\Delta V_T = A \times \Delta h\), with \(\Delta h\) change in thickness.

\sphinxAtStartPar
\(A \times \Delta h= \alpha_{pm}\cdot\rho_w \cdot g \cdot  A \cdot h \cdot  \Delta \psi \)

\sphinxAtStartPar
\(\Delta h = \alpha_{pm}\cdot\rho_w \cdot g \cdot h \cdot  \Delta \psi\), with \(\Delta h\) being the land subsidence

\begin{sphinxuseclass}{cell}
\begin{sphinxuseclass}{tag_remove-input}
\begin{sphinxuseclass}{tag_remove-output}
\end{sphinxuseclass}
\end{sphinxuseclass}
\end{sphinxuseclass}
\begin{sphinxuseclass}{cell}
\begin{sphinxuseclass}{tag_remove-input}\begin{sphinxVerbatimOutput}

\begin{sphinxuseclass}{cell_output}
\begin{sphinxVerbatim}[commandchars=\\\{\}]

\end{sphinxVerbatim}

\begin{sphinxVerbatim}[commandchars=\\\{\}]
\PYGZlt{}IPython.core.display.HTML object\PYGZgt{}
\end{sphinxVerbatim}

\begin{sphinxVerbatim}[commandchars=\\\{\}]
\PYGZlt{}IPython.core.display.HTML object\PYGZgt{}
\end{sphinxVerbatim}

\begin{sphinxVerbatim}[commandchars=\\\{\}]
\PYGZlt{}IPython.core.display.Javascript object\PYGZgt{}
\end{sphinxVerbatim}

\end{sphinxuseclass}\end{sphinxVerbatimOutput}

\end{sphinxuseclass}
\end{sphinxuseclass}
\begin{sphinxuseclass}{cell}\begin{sphinxVerbatimInput}

\begin{sphinxuseclass}{cell_input}
\begin{sphinxVerbatim}[commandchars=\\\{\}]
\PYG{c+c1}{\PYGZsh{}  solution 2 a}

\PYG{c+c1}{\PYGZsh{} Given}

\PYG{n}{A} \PYG{o}{=} \PYG{l+m+mi}{1} \PYG{c+c1}{\PYGZsh{} m², assuming 1 m² aquifer area}
\PYG{n}{h} \PYG{o}{=} \PYG{l+m+mi}{90} \PYG{c+c1}{\PYGZsh{} m, aquifer height before extraction}
\PYG{n}{d\PYGZus{}psi} \PYG{o}{=} \PYG{l+m+mi}{150} \PYG{c+c1}{\PYGZsh{} m, change in pressure head}
\PYG{n}{S\PYGZus{}2} \PYG{o}{=} \PYG{l+m+mi}{4}\PYG{o}{*}\PYG{l+m+mi}{10}\PYG{o}{*}\PYG{o}{*}\PYG{o}{\PYGZhy{}}\PYG{l+m+mi}{4} \PYG{c+c1}{\PYGZsh{}  specific storage}
\PYG{n}{rho\PYGZus{}w} \PYG{o}{=} \PYG{l+m+mi}{1000} \PYG{c+c1}{\PYGZsh{} Kg/m³, density of water}
\PYG{n}{a\PYGZus{}pm} \PYG{o}{=} \PYG{l+m+mi}{6}\PYG{o}{*} \PYG{l+m+mi}{10}\PYG{o}{*}\PYG{o}{*}\PYG{o}{\PYGZhy{}}\PYG{l+m+mi}{8} \PYG{c+c1}{\PYGZsh{} m²/N = m\PYGZhy{}s²/kg, compressibility of porous medium}
\PYG{n}{g} \PYG{o}{=} \PYG{l+m+mf}{9.81} \PYG{c+c1}{\PYGZsh{} m/s², gravity factor}

\PYG{c+c1}{\PYGZsh{}Solution}

\PYG{n}{d\PYGZus{}V\PYGZus{}w}  \PYG{o}{=} \PYG{n}{S\PYGZus{}2}\PYG{o}{*}\PYG{n}{A}\PYG{o}{*}\PYG{n}{d\PYGZus{}psi}

\PYG{n+nb}{print}\PYG{p}{(}\PYG{l+s+s2}{\PYGZdq{}}\PYG{l+s+s2}{The water abstraction volume per m}\PYG{l+s+se}{\PYGZbs{}u00b2}\PYG{l+s+s2}{ aquifer is }\PYG{l+s+si}{\PYGZob{}0:0.3f\PYGZcb{}}\PYG{l+s+s2}{\PYGZdq{}}\PYG{o}{.}\PYG{n}{format}\PYG{p}{(}\PYG{n}{d\PYGZus{}V\PYGZus{}w}\PYG{p}{)}\PYG{p}{,} \PYG{l+s+s2}{\PYGZdq{}}\PYG{l+s+s2}{m}\PYG{l+s+se}{\PYGZbs{}u00b3}\PYG{l+s+s2}{\PYGZdq{}}\PYG{p}{)}
\end{sphinxVerbatim}

\end{sphinxuseclass}\end{sphinxVerbatimInput}
\begin{sphinxVerbatimOutput}

\begin{sphinxuseclass}{cell_output}
\begin{sphinxVerbatim}[commandchars=\\\{\}]
The water abstraction volume per m² aquifer is 0.060 m³
\end{sphinxVerbatim}

\end{sphinxuseclass}\end{sphinxVerbatimOutput}

\end{sphinxuseclass}
\begin{sphinxuseclass}{cell}\begin{sphinxVerbatimInput}

\begin{sphinxuseclass}{cell_input}
\begin{sphinxVerbatim}[commandchars=\\\{\}]
\PYG{c+c1}{\PYGZsh{}  solution 2 b}


\PYG{c+c1}{\PYGZsh{} Given}

\PYG{n}{A} \PYG{o}{=} \PYG{l+m+mi}{1} \PYG{c+c1}{\PYGZsh{} m², assuming 1 m² aquifer area}
\PYG{n}{h} \PYG{o}{=} \PYG{l+m+mi}{90} \PYG{c+c1}{\PYGZsh{} m, aquifer height before extraction}
\PYG{n}{d\PYGZus{}psi} \PYG{o}{=} \PYG{l+m+mi}{150} \PYG{c+c1}{\PYGZsh{} m, change in pressure head}
\PYG{n}{Ss} \PYG{o}{=} \PYG{l+m+mi}{4}\PYG{o}{*}\PYG{l+m+mi}{10}\PYG{o}{*}\PYG{o}{*}\PYG{o}{\PYGZhy{}}\PYG{l+m+mi}{4} \PYG{c+c1}{\PYGZsh{} specific storage}
\PYG{n}{rho\PYGZus{}w} \PYG{o}{=} \PYG{l+m+mi}{1000} \PYG{c+c1}{\PYGZsh{} Kg/m³, density of water}
\PYG{n}{a\PYGZus{}pm} \PYG{o}{=} \PYG{l+m+mi}{6}\PYG{o}{*} \PYG{l+m+mi}{10}\PYG{o}{*}\PYG{o}{*}\PYG{o}{\PYGZhy{}}\PYG{l+m+mi}{8} \PYG{c+c1}{\PYGZsh{} m²/N = m\PYGZhy{}s²/kg, compressibility of porous medium}
\PYG{n}{g} \PYG{o}{=} \PYG{l+m+mf}{9.81} \PYG{c+c1}{\PYGZsh{} m/s², gravity factor}

\PYG{c+c1}{\PYGZsh{}interim calculation}
\PYG{n}{V\PYGZus{}T} \PYG{o}{=} \PYG{n}{A}\PYG{o}{*}\PYG{n}{h} \PYG{c+c1}{\PYGZsh{} m³, Aquifer volume before extraction }

\PYG{c+c1}{\PYGZsh{}Solution}

\PYG{n}{d\PYGZus{}h}  \PYG{o}{=} \PYG{n}{a\PYGZus{}pm}\PYG{o}{*}\PYG{n}{rho\PYGZus{}w}\PYG{o}{*}\PYG{n}{g}\PYG{o}{*}\PYG{n}{h}\PYG{o}{*}\PYG{n}{d\PYGZus{}psi}

\PYG{n+nb}{print}\PYG{p}{(}\PYG{l+s+s2}{\PYGZdq{}}\PYG{l+s+s2}{The water abstraction volume is }\PYG{l+s+si}{\PYGZob{}0:0.2f\PYGZcb{}}\PYG{l+s+s2}{\PYGZdq{}}\PYG{o}{.}\PYG{n}{format}\PYG{p}{(}\PYG{n}{d\PYGZus{}h}\PYG{p}{)}\PYG{p}{,} \PYG{l+s+s2}{\PYGZdq{}}\PYG{l+s+s2}{m}\PYG{l+s+s2}{\PYGZdq{}}\PYG{p}{)}
\end{sphinxVerbatim}

\end{sphinxuseclass}\end{sphinxVerbatimInput}
\begin{sphinxVerbatimOutput}

\begin{sphinxuseclass}{cell_output}
\begin{sphinxVerbatim}[commandchars=\\\{\}]
The water abstraction volume is 7.95 m
\end{sphinxVerbatim}

\end{sphinxuseclass}\end{sphinxVerbatimOutput}

\end{sphinxuseclass}
\sphinxAtStartPar
\sphinxstylestrong{Q3. Aquifer Properties} (ca. 10 pts.)

\sphinxAtStartPar
The hydraulic conductivity of a sample (length 15 cm, diameter 5 cm) is to be determined using a constant\sphinxhyphen{}head permeameter. For that 250 ml water is passed through the sample in 30 s while maintaining the head difference of 2.5 cm. Properties of water provided are:
density of water at 20°C: 1000 kg/m3;                                                  dynamic viscosity of water at 20°C: 1.0087·10\sphinxhyphen{}3 Pa·s

\sphinxAtStartPar
a.  Sketch the problem as accurately as possible providing essential features with legends (3 points)

\sphinxAtStartPar
b. What will be the conductivity of the sample? (4 points)

\sphinxAtStartPar
What is the intrinsic permeability of the sample? (2 points)

\sphinxAtStartPar
c. What soil type is likely the sample? (1 point)

\sphinxAtStartPar
(Hint: For the calculation of permeability, dynamic viscosity/density ratio is required.)

\sphinxAtStartPar
\sphinxstylestrong{Solution 3} \sphinxhyphen{}

\sphinxAtStartPar
\sphinxstylestrong{Solution 3a} (L05/15)

\sphinxAtStartPar
 



\begin{sphinxuseclass}{cell}\begin{sphinxVerbatimInput}

\begin{sphinxuseclass}{cell_input}
\begin{sphinxVerbatim}[commandchars=\\\{\}]
\PYG{c+c1}{\PYGZsh{}Solution 3b** (L05/15)}

\PYG{c+c1}{\PYGZsh{} Given}

\PYG{n}{L\PYGZus{}c} \PYG{o}{=} \PYG{l+m+mi}{15} \PYG{c+c1}{\PYGZsh{} cm, column length}
\PYG{n}{Dia\PYGZus{}c} \PYG{o}{=} \PYG{l+m+mi}{5} \PYG{c+c1}{\PYGZsh{} cm, diameter column}
\PYG{n}{V\PYGZus{}in}\PYG{o}{=} \PYG{l+m+mi}{250} \PYG{c+c1}{\PYGZsh{} mL, water entering the column}
\PYG{n}{t\PYGZus{}c} \PYG{o}{=} \PYG{l+m+mi}{30} \PYG{c+c1}{\PYGZsh{} s, time required to pass}
\PYG{n}{d\PYGZus{}3h} \PYG{o}{=} \PYG{l+m+mf}{2.5} \PYG{c+c1}{\PYGZsh{} cm, head difference}

\PYG{c+c1}{\PYGZsh{} interim calculation}
\PYG{n}{A\PYGZus{}c} \PYG{o}{=} \PYG{n}{np}\PYG{o}{.}\PYG{n}{pi}\PYG{o}{*}\PYG{n}{Dia\PYGZus{}c}\PYG{o}{*}\PYG{o}{*}\PYG{l+m+mi}{2}\PYG{o}{/}\PYG{l+m+mi}{4} \PYG{c+c1}{\PYGZsh{} cm², Area of column}
\PYG{n}{Q\PYGZus{}c} \PYG{o}{=} \PYG{n}{V\PYGZus{}in}\PYG{o}{/}\PYG{n}{t\PYGZus{}c} \PYG{c+c1}{\PYGZsh{} cm³/s, assume 1mL = 1 cm³, Discharge out of column}

\PYG{c+c1}{\PYGZsh{}solution }
\PYG{n}{K\PYGZus{}c} \PYG{o}{=} \PYG{p}{(}\PYG{n}{Q\PYGZus{}c}\PYG{o}{*}\PYG{n}{L\PYGZus{}c}\PYG{p}{)}\PYG{o}{/}\PYG{p}{(}\PYG{n}{A\PYGZus{}c}\PYG{o}{*}\PYG{n}{d\PYGZus{}3h}\PYG{p}{)} \PYG{c+c1}{\PYGZsh{} cm/s, conductivity }


\PYG{n+nb}{print}\PYG{p}{(}\PYG{l+s+s2}{\PYGZdq{}}\PYG{l+s+s2}{The area of the column is }\PYG{l+s+si}{\PYGZob{}0:0.2f\PYGZcb{}}\PYG{l+s+s2}{\PYGZdq{}}\PYG{o}{.}\PYG{n}{format}\PYG{p}{(}\PYG{n}{A\PYGZus{}c}\PYG{p}{)}\PYG{p}{,} \PYG{l+s+s2}{\PYGZdq{}}\PYG{l+s+s2}{cm}\PYG{l+s+se}{\PYGZbs{}u00b2}\PYG{l+s+s2}{\PYGZdq{}}\PYG{p}{)}
\PYG{n+nb}{print}\PYG{p}{(}\PYG{l+s+s2}{\PYGZdq{}}\PYG{l+s+s2}{The discharge from the aquifer is }\PYG{l+s+si}{\PYGZob{}0:0.2f\PYGZcb{}}\PYG{l+s+s2}{\PYGZdq{}}\PYG{o}{.}\PYG{n}{format}\PYG{p}{(}\PYG{n}{Q\PYGZus{}c}\PYG{p}{)}\PYG{p}{,} \PYG{l+s+s2}{\PYGZdq{}}\PYG{l+s+s2}{cm}\PYG{l+s+se}{\PYGZbs{}u00b3}\PYG{l+s+s2}{/s}\PYG{l+s+s2}{\PYGZdq{}}\PYG{p}{)}
\PYG{n+nb}{print}\PYG{p}{(}\PYG{l+s+s2}{\PYGZdq{}}\PYG{l+s+s2}{The conductivity of the sample is }\PYG{l+s+si}{\PYGZob{}0:0.2f\PYGZcb{}}\PYG{l+s+s2}{\PYGZdq{}}\PYG{o}{.}\PYG{n}{format}\PYG{p}{(}\PYG{n}{K\PYGZus{}c}\PYG{p}{)}\PYG{p}{,} \PYG{l+s+s2}{\PYGZdq{}}\PYG{l+s+s2}{cm/s}\PYG{l+s+s2}{\PYGZdq{}}\PYG{p}{)}
\PYG{n+nb}{print}\PYG{p}{(}\PYG{l+s+s2}{\PYGZdq{}}\PYG{l+s+s2}{The conductivity of the sample is }\PYG{l+s+si}{\PYGZob{}0:0.4f\PYGZcb{}}\PYG{l+s+s2}{\PYGZdq{}}\PYG{o}{.}\PYG{n}{format}\PYG{p}{(}\PYG{n}{K\PYGZus{}c}\PYG{o}{/}\PYG{l+m+mi}{100}\PYG{p}{)}\PYG{p}{,} \PYG{l+s+s2}{\PYGZdq{}}\PYG{l+s+s2}{m/s}\PYG{l+s+s2}{\PYGZdq{}}\PYG{p}{)}
\end{sphinxVerbatim}

\end{sphinxuseclass}\end{sphinxVerbatimInput}
\begin{sphinxVerbatimOutput}

\begin{sphinxuseclass}{cell_output}
\begin{sphinxVerbatim}[commandchars=\\\{\}]
The area of the column is 19.63 cm²
The discharge from the aquifer is 8.33 cm³/s
The conductivity of the sample is 2.55 cm/s
The conductivity of the sample is 0.0255 m/s
\end{sphinxVerbatim}

\end{sphinxuseclass}\end{sphinxVerbatimOutput}

\end{sphinxuseclass}
\begin{sphinxuseclass}{cell}\begin{sphinxVerbatimInput}

\begin{sphinxuseclass}{cell_input}
\begin{sphinxVerbatim}[commandchars=\\\{\}]
\PYG{c+c1}{\PYGZsh{}Solution 3c** (L05/18)}

\PYG{c+c1}{\PYGZsh{} Given}

\PYG{n}{K\PYGZus{}m} \PYG{o}{=} \PYG{n}{K\PYGZus{}c}\PYG{o}{/}\PYG{l+m+mi}{100} \PYG{c+c1}{\PYGZsh{} m/s, conductivity}
\PYG{n}{rho\PYGZus{}3w} \PYG{o}{=} \PYG{l+m+mi}{1000} \PYG{c+c1}{\PYGZsh{} Kg/m³, density of water}
\PYG{n}{nu\PYGZus{}w} \PYG{o}{=} \PYG{l+m+mf}{1.0087}\PYG{o}{*}\PYG{l+m+mi}{10}\PYG{o}{*}\PYG{o}{*}\PYG{o}{\PYGZhy{}}\PYG{l+m+mi}{3} \PYG{c+c1}{\PYGZsh{} Pa\PYGZhy{}s = Kg/m\PYGZhy{}s, dynamic viscosity }
\PYG{n}{g} \PYG{o}{=} \PYG{l+m+mf}{9.81} \PYG{c+c1}{\PYGZsh{} m/s², gravity factor}

\PYG{c+c1}{\PYGZsh{}solution}
\PYG{n}{k\PYGZus{}c} \PYG{o}{=} \PYG{n}{K\PYGZus{}m}\PYG{o}{*}\PYG{n}{nu\PYGZus{}w}\PYG{o}{/}\PYG{p}{(}\PYG{n}{rho\PYGZus{}3w}\PYG{o}{*}\PYG{n}{g}\PYG{p}{)}

\PYG{n+nb}{print}\PYG{p}{(}\PYG{l+s+s2}{\PYGZdq{}}\PYG{l+s+s2}{The permeability of the sample is }\PYG{l+s+si}{\PYGZob{}0:0.10f\PYGZcb{}}\PYG{l+s+s2}{\PYGZdq{}}\PYG{o}{.}\PYG{n}{format}\PYG{p}{(}\PYG{n}{k\PYGZus{}c}\PYG{p}{)}\PYG{p}{,} \PYG{l+s+s2}{\PYGZdq{}}\PYG{l+s+s2}{m}\PYG{l+s+se}{\PYGZbs{}u00b2}\PYG{l+s+s2}{\PYGZdq{}}\PYG{p}{)}
\PYG{n+nb}{print}\PYG{p}{(}\PYG{l+s+s2}{\PYGZdq{}}\PYG{l+s+s2}{The permeability of the sample is }\PYG{l+s+si}{\PYGZob{}0:0.2E\PYGZcb{}}\PYG{l+s+s2}{\PYGZdq{}}\PYG{o}{.}\PYG{n}{format}\PYG{p}{(}\PYG{n}{k\PYGZus{}c}\PYG{p}{)}\PYG{p}{,} \PYG{l+s+s2}{\PYGZdq{}}\PYG{l+s+s2}{m}\PYG{l+s+se}{\PYGZbs{}u00b2}\PYG{l+s+s2}{\PYGZdq{}}\PYG{p}{)}
\end{sphinxVerbatim}

\end{sphinxuseclass}\end{sphinxVerbatimInput}
\begin{sphinxVerbatimOutput}

\begin{sphinxuseclass}{cell_output}
\begin{sphinxVerbatim}[commandchars=\\\{\}]
The permeability of the sample is 0.0000000026 m²
The permeability of the sample is 2.62E\PYGZhy{}09 m²
\end{sphinxVerbatim}

\end{sphinxuseclass}\end{sphinxVerbatimOutput}

\end{sphinxuseclass}
\sphinxAtStartPar
\sphinxstylestrong{Solution 3d}
(L05/11)

\sphinxAtStartPar
The sample in the column is likely gravel or coarse sand.

\sphinxAtStartPar
\sphinxstylestrong{Q4. Sieve Analysis (ca. 6 pts.)}

\sphinxAtStartPar
Sieve experiments were performed with the bore samples and the following observations were obtained:


\begin{savenotes}\sphinxattablestart
\centering
\begin{tabulary}{\linewidth}[t]{|T|T|T|T|}
\hline
\sphinxstyletheadfamily 
\sphinxAtStartPar
mesh diameter {[}mm{]}
&\sphinxstyletheadfamily 
\sphinxAtStartPar
residue in the sieve {[}g{]}
&\sphinxstyletheadfamily 
\sphinxAtStartPar
Σ total
&\sphinxstyletheadfamily 
\sphinxAtStartPar
Σ/Σtotal
\\
\hline
\sphinxAtStartPar
6
&
\sphinxAtStartPar
0
&
\sphinxAtStartPar

&
\sphinxAtStartPar

\\
\hline
\sphinxAtStartPar
2
&
\sphinxAtStartPar
40
&
\sphinxAtStartPar

&
\sphinxAtStartPar

\\
\hline
\sphinxAtStartPar
0.6
&
\sphinxAtStartPar
250
&
\sphinxAtStartPar

&
\sphinxAtStartPar

\\
\hline
\sphinxAtStartPar
0.2
&
\sphinxAtStartPar
150
&
\sphinxAtStartPar

&
\sphinxAtStartPar

\\
\hline
\sphinxAtStartPar
0.06
&
\sphinxAtStartPar
60
&
\sphinxAtStartPar

&
\sphinxAtStartPar

\\
\hline
\sphinxAtStartPar
<0.06 (cup)
&
\sphinxAtStartPar
10
&
\sphinxAtStartPar

&
\sphinxAtStartPar

\\
\hline
\end{tabulary}
\par
\sphinxattableend\end{savenotes}

\sphinxAtStartPar
a. Draw the granulometric curve in the diagram below.			(ca. 5 pts.)

\sphinxAtStartPar
b. Briefly characterise the sediment.						(ca. 1 pt.)

\begin{sphinxuseclass}{cell}\begin{sphinxVerbatimInput}

\begin{sphinxuseclass}{cell_input}
\begin{sphinxVerbatim}[commandchars=\\\{\}]
\PYG{c+c1}{\PYGZsh{}Solution 4 \PYGZhy{}(L03/18)}

\PYG{n}{dia} \PYG{o}{=} \PYG{p}{[}\PYG{l+m+mi}{6}\PYG{p}{,}\PYG{l+m+mi}{2}\PYG{p}{,}\PYG{l+m+mf}{0.6}\PYG{p}{,}\PYG{l+m+mf}{0.2}\PYG{p}{,} \PYG{l+m+mf}{0.06}\PYG{p}{,} \PYG{l+m+mf}{0.001}\PYG{p}{]} \PYG{c+c1}{\PYGZsh{} mm, diameter \PYGZlt{}0.06 (cup)= 0.001}
\PYG{n}{mass} \PYG{o}{=} \PYG{p}{[}\PYG{l+m+mi}{0}\PYG{p}{,} \PYG{l+m+mi}{40}\PYG{p}{,} \PYG{l+m+mi}{250}\PYG{p}{,} \PYG{l+m+mi}{150}\PYG{p}{,} \PYG{l+m+mi}{60}\PYG{p}{,} \PYG{l+m+mi}{10}\PYG{p}{]} \PYG{c+c1}{\PYGZsh{} g, the residue in seive }

\PYG{c+c1}{\PYGZsh{} Calculation steps \PYGZhy{} filling table}
\PYG{n}{Total\PYGZus{}mass} \PYG{o}{=} \PYG{n}{np}\PYG{o}{.}\PYG{n}{sum}\PYG{p}{(}\PYG{n}{mass}\PYG{p}{)}  \PYG{c+c1}{\PYGZsh{} add the mass column to get total mass}
\PYG{n}{retain\PYGZus{}per} \PYG{o}{=} \PYG{n}{mass}\PYG{o}{/}\PYG{n}{Total\PYGZus{}mass}\PYG{o}{*}\PYG{l+m+mi}{100}   \PYG{c+c1}{\PYGZsh{} retain percentage}
\PYG{n}{retain\PYGZus{}per\PYGZus{}cumsum} \PYG{o}{=} \PYG{n}{np}\PYG{o}{.}\PYG{n}{cumsum}\PYG{p}{(}\PYG{n}{retain\PYGZus{}per}\PYG{p}{)} \PYG{c+c1}{\PYGZsh{} get the cummulative sum of the reatined}
\PYG{n}{passing\PYGZus{}per} \PYG{o}{=} \PYG{l+m+mi}{100} \PYG{o}{\PYGZhy{}} \PYG{n}{retain\PYGZus{}per\PYGZus{}cumsum} \PYG{c+c1}{\PYGZsh{} substract 100\PYGZhy{}cummsum to get passing \PYGZpc{}}

\PYG{n}{data} \PYG{o}{=} \PYG{p}{\PYGZob{}}\PYG{l+s+s2}{\PYGZdq{}}\PYG{l+s+s2}{mesh diameter [mm]}\PYG{l+s+s2}{\PYGZdq{}}\PYG{p}{:} \PYG{n}{dia}\PYG{p}{,} \PYG{l+s+s2}{\PYGZdq{}}\PYG{l+s+s2}{residue in the sieve [g]}\PYG{l+s+s2}{\PYGZdq{}}\PYG{p}{:} \PYG{n}{mass}\PYG{p}{,} \PYG{l+s+s2}{\PYGZdq{}}\PYG{l+s+s2}{Σtotal}\PYG{l+s+s2}{\PYGZdq{}}\PYG{p}{:} \PYG{n}{retain\PYGZus{}per}\PYG{p}{,} \PYG{l+s+s2}{\PYGZdq{}}\PYG{l+s+s2}{Σ/Σtotal}\PYG{l+s+s2}{\PYGZdq{}}\PYG{p}{:} \PYG{n}{passing\PYGZus{}per} \PYG{p}{\PYGZcb{}}

\PYG{n}{df1}\PYG{o}{=} \PYG{n}{pd}\PYG{o}{.}\PYG{n}{DataFrame}\PYG{p}{(}\PYG{n}{data}\PYG{p}{)}
\PYG{n}{df1} 
\end{sphinxVerbatim}

\end{sphinxuseclass}\end{sphinxVerbatimInput}
\begin{sphinxVerbatimOutput}

\begin{sphinxuseclass}{cell_output}
\begin{sphinxVerbatim}[commandchars=\\\{\}]
   mesh diameter [mm]  residue in the sieve [g]     Σtotal    Σ/Σtotal
0               6.000                         0   0.000000  100.000000
1               2.000                        40   7.843137   92.156863
2               0.600                       250  49.019608   43.137255
3               0.200                       150  29.411765   13.725490
4               0.060                        60  11.764706    1.960784
5               0.001                        10   1.960784    0.000000
\end{sphinxVerbatim}

\end{sphinxuseclass}\end{sphinxVerbatimOutput}

\end{sphinxuseclass}
\begin{sphinxuseclass}{cell}\begin{sphinxVerbatimInput}

\begin{sphinxuseclass}{cell_input}
\begin{sphinxVerbatim}[commandchars=\\\{\}]
\PYG{c+c1}{\PYGZsh{} plotting}
\PYG{n}{plt}\PYG{o}{.}\PYG{n}{rcParams}\PYG{p}{[}\PYG{l+s+s1}{\PYGZsq{}}\PYG{l+s+s1}{axes.linewidth}\PYG{l+s+s1}{\PYGZsq{}}\PYG{p}{]}\PYG{o}{=}\PYG{l+m+mi}{2}
\PYG{c+c1}{\PYGZsh{}plt.rcParams[\PYGZdq{}axes.edgecolor\PYGZdq{}]=\PYGZsq{}white\PYGZsq{}}
\PYG{n}{plt}\PYG{o}{.}\PYG{n}{rcParams}\PYG{p}{[}\PYG{l+s+s1}{\PYGZsq{}}\PYG{l+s+s1}{grid.linestyle}\PYG{l+s+s1}{\PYGZsq{}}\PYG{p}{]}\PYG{o}{=}\PYG{l+s+s1}{\PYGZsq{}}\PYG{l+s+s1}{\PYGZhy{}\PYGZhy{}}\PYG{l+s+s1}{\PYGZsq{}}
\PYG{n}{plt}\PYG{o}{.}\PYG{n}{rcParams}\PYG{p}{[}\PYG{l+s+s1}{\PYGZsq{}}\PYG{l+s+s1}{grid.linewidth}\PYG{l+s+s1}{\PYGZsq{}}\PYG{p}{]}\PYG{o}{=}\PYG{l+m+mi}{1}
\PYG{n}{x} \PYG{o}{=} \PYG{n}{np}\PYG{o}{.}\PYG{n}{append}\PYG{p}{(}\PYG{p}{[}\PYG{l+m+mi}{10}\PYG{p}{]}\PYG{p}{,}\PYG{n}{dia}\PYG{p}{)} \PYG{c+c1}{\PYGZsh{} adding data to extend over 6 mm dia}
\PYG{n}{y} \PYG{o}{=} \PYG{n}{np}\PYG{o}{.}\PYG{n}{append}\PYG{p}{(}\PYG{p}{[}\PYG{l+m+mi}{100}\PYG{p}{]}\PYG{p}{,}\PYG{n}{passing\PYGZus{}per}\PYG{p}{)} \PYG{c+c1}{\PYGZsh{} adding 100\PYGZpc{} to plot}

\PYG{n}{fig} \PYG{o}{=} \PYG{n}{plt}\PYG{o}{.}\PYG{n}{figure}\PYG{p}{(}\PYG{n}{figsize}\PYG{o}{=}\PYG{p}{(}\PYG{l+m+mi}{9}\PYG{p}{,}\PYG{l+m+mi}{6}\PYG{p}{)}\PYG{p}{)}
\PYG{n}{plt}\PYG{o}{.}\PYG{n}{semilogx}\PYG{p}{(}\PYG{n}{x}\PYG{p}{,} \PYG{n}{y}\PYG{p}{,} \PYG{l+s+s1}{\PYGZsq{}}\PYG{l+s+s1}{x\PYGZhy{}}\PYG{l+s+s1}{\PYGZsq{}}\PYG{p}{,} \PYG{n}{color}\PYG{o}{=}\PYG{l+s+s1}{\PYGZsq{}}\PYG{l+s+s1}{red}\PYG{l+s+s1}{\PYGZsq{}}\PYG{p}{)}  
\PYG{n}{tics}\PYG{o}{=}\PYG{n}{x}\PYG{o}{.}\PYG{n}{tolist}\PYG{p}{(}\PYG{p}{)}

\PYG{n}{plt}\PYG{o}{.}\PYG{n}{grid}\PYG{p}{(}\PYG{n}{which}\PYG{o}{=}\PYG{l+s+s1}{\PYGZsq{}}\PYG{l+s+s1}{major}\PYG{l+s+s1}{\PYGZsq{}}\PYG{p}{,} \PYG{n}{color}\PYG{o}{=}\PYG{l+s+s1}{\PYGZsq{}}\PYG{l+s+s1}{k}\PYG{l+s+s1}{\PYGZsq{}}\PYG{p}{,} \PYG{n}{alpha}\PYG{o}{=}\PYG{l+m+mf}{0.7}\PYG{p}{)} 
\PYG{n}{plt}\PYG{o}{.}\PYG{n}{grid}\PYG{p}{(}\PYG{n}{which}\PYG{o}{=}\PYG{l+s+s1}{\PYGZsq{}}\PYG{l+s+s1}{minor}\PYG{l+s+s1}{\PYGZsq{}}\PYG{p}{,} \PYG{n}{color}\PYG{o}{=}\PYG{l+s+s1}{\PYGZsq{}}\PYG{l+s+s1}{k}\PYG{l+s+s1}{\PYGZsq{}}\PYG{p}{,} \PYG{n}{alpha}\PYG{o}{=}\PYG{l+m+mf}{0.3}\PYG{p}{)}
\PYG{n}{plt}\PYG{o}{.}\PYG{n}{xticks}\PYG{p}{(}\PYG{n}{x}\PYG{p}{,} \PYG{n}{tics}\PYG{p}{)}\PYG{p}{;}  
\PYG{n}{plt}\PYG{o}{.}\PYG{n}{yticks}\PYG{p}{(}\PYG{n}{np}\PYG{o}{.}\PYG{n}{arange}\PYG{p}{(}\PYG{l+m+mi}{0}\PYG{p}{,}\PYG{l+m+mi}{110}\PYG{p}{,}\PYG{l+m+mi}{10}\PYG{p}{)}\PYG{p}{)}\PYG{p}{;}
\PYG{n}{plt}\PYG{o}{.}\PYG{n}{title}\PYG{p}{(}\PYG{l+s+s1}{\PYGZsq{}}\PYG{l+s+s1}{grain size distribution}\PYG{l+s+s1}{\PYGZsq{}}\PYG{p}{)}\PYG{p}{;}
\PYG{n}{plt}\PYG{o}{.}\PYG{n}{xlabel}\PYG{p}{(}\PYG{l+s+s1}{\PYGZsq{}}\PYG{l+s+s1}{grain size d [mm]}\PYG{l+s+s1}{\PYGZsq{}}\PYG{p}{)}\PYG{p}{;}
\PYG{n}{plt}\PYG{o}{.}\PYG{n}{ylabel}\PYG{p}{(}\PYG{l+s+s1}{\PYGZsq{}}\PYG{l+s+s1}{grain fraction \PYGZlt{} d ins }\PYG{l+s+si}{\PYGZpc{} o}\PYG{l+s+s1}{f total mass}\PYG{l+s+s1}{\PYGZsq{}}\PYG{p}{)}\PYG{p}{;}
\end{sphinxVerbatim}

\end{sphinxuseclass}\end{sphinxVerbatimInput}
\begin{sphinxVerbatimOutput}

\begin{sphinxuseclass}{cell_output}
\noindent\sphinxincludegraphics{{C:/Users/vibhu/GWtextbook/_build/jupyter_execute/GW_exam_2019_20_20_0}.png}

\end{sphinxuseclass}\end{sphinxVerbatimOutput}

\end{sphinxuseclass}
\sphinxAtStartPar
\sphinxstylestrong{solution 4b}

\sphinxAtStartPar
The sample can be considered uniformly distributed as over 70\% of sample falls in the sand size (0.2 mm\sphinxhyphen{}2 mm). Therefore, the sample can be considered sandy.

\sphinxAtStartPar
\sphinxstylestrong{Q5. Aquifer characterization} (ca. 8 pts.)

\sphinxAtStartPar
Water levels in m a.s.l. were measured at three observation wells (see figure).



\sphinxAtStartPar
a. Sketch hydraulic head isolines for increments of 0.5 m. (ca. 3 points.)

\sphinxAtStartPar
b. Gravel layer (thickness (t1) = 1.5 m, and conductivity (K1) = 3.7 10\sphinxhyphen{}3 is embedded between two sandy layers (t2 = 2 m, K2 = 3·10\sphinxhyphen{}4 m/s; and t3 = 3 m, K3 = 4·10\sphinxhyphen{}4 m/s). If the hydraulic gradient is 1\% and overall discharge is 1 m3/d per unit width of the aquifer, find the effective hydraulic conductivity considering a parallelly layered aquifer.

\sphinxAtStartPar
(Hint: \(K_{eff} = \frac{m}{\sum_{i=1}^n \frac{m_i}{K_i}}\) or \(K_{eff} =  \sum_{i=1}^n\frac{m_i\cdot K_i}{m}\)  ) (ca. 2 points)

\sphinxAtStartPar
c. Distinguish between homogeneity and heterogeneity, and isotropy and anisotropy (ca. 3 points)

\sphinxAtStartPar
\sphinxstylestrong{Solution 5a}
(L07/08\sphinxhyphen{}09)

\sphinxAtStartPar
The isolines and flow direction is provided in the figure below.



\begin{sphinxuseclass}{cell}\begin{sphinxVerbatimInput}

\begin{sphinxuseclass}{cell_input}
\begin{sphinxVerbatim}[commandchars=\\\{\}]
\PYG{c+c1}{\PYGZsh{} Solution 5b (L06/08\PYGZhy{}13)}

\PYG{c+c1}{\PYGZsh{} Given:}

\PYG{n}{G\PYGZus{}t1} \PYG{o}{=} \PYG{l+m+mi}{2} \PYG{c+c1}{\PYGZsh{} m, sandy layer top}
\PYG{n}{G\PYGZus{}t2} \PYG{o}{=} \PYG{l+m+mf}{1.5} \PYG{c+c1}{\PYGZsh{} m, gravel layer middle}
\PYG{n}{G\PYGZus{}t3} \PYG{o}{=} \PYG{l+m+mi}{3} \PYG{c+c1}{\PYGZsh{} m, sandy layer bottom }
\PYG{n}{K\PYGZus{}1} \PYG{o}{=}  \PYG{l+m+mf}{3.0}\PYG{o}{*}\PYG{l+m+mi}{10}\PYG{o}{*}\PYG{o}{*}\PYG{o}{\PYGZhy{}}\PYG{l+m+mi}{4} \PYG{c+c1}{\PYGZsh{} m/s cond. in G\PYGZus{}t1}
\PYG{n}{K\PYGZus{}2} \PYG{o}{=} \PYG{l+m+mf}{3.7}\PYG{o}{*}\PYG{l+m+mi}{10}\PYG{o}{*}\PYG{o}{*}\PYG{o}{\PYGZhy{}}\PYG{l+m+mi}{3} \PYG{c+c1}{\PYGZsh{} m/s cond. in G\PYGZus{}t2}
\PYG{n}{K\PYGZus{}3} \PYG{o}{=}  \PYG{l+m+mf}{4.0}\PYG{o}{*}\PYG{l+m+mi}{10}\PYG{o}{*}\PYG{o}{*}\PYG{o}{\PYGZhy{}}\PYG{l+m+mi}{4} \PYG{c+c1}{\PYGZsh{} m/s cond. in G\PYGZus{}t3}
\PYG{n}{i} \PYG{o}{=} \PYG{l+m+mi}{1}\PYG{o}{/}\PYG{l+m+mi}{100} \PYG{c+c1}{\PYGZsh{} (), hydraulic gradient 1\PYGZpc{}}
\PYG{n}{Q\PYGZus{}5} \PYG{o}{=} \PYG{l+m+mi}{1} \PYG{c+c1}{\PYGZsh{} m³/d per\PYGZhy{}W, discharge per unit width}

\PYG{c+c1}{\PYGZsh{}intermediate calculation}
\PYG{n}{m} \PYG{o}{=} \PYG{n}{G\PYGZus{}t1}\PYG{o}{+}\PYG{n}{G\PYGZus{}t2}\PYG{o}{+}\PYG{n}{G\PYGZus{}t3} \PYG{c+c1}{\PYGZsh{} m, total aq. thickness}

\PYG{n}{K\PYGZus{}ef\PYGZus{}h} \PYG{o}{=} \PYG{p}{(}\PYG{l+m+mi}{1}\PYG{o}{/}\PYG{n}{m}\PYG{p}{)} \PYG{o}{*} \PYG{p}{(}\PYG{n}{G\PYGZus{}t1}\PYG{o}{*}\PYG{n}{K\PYGZus{}1} \PYG{o}{+} \PYG{n}{G\PYGZus{}t2}\PYG{o}{*}\PYG{n}{K\PYGZus{}2} \PYG{o}{+} \PYG{n}{G\PYGZus{}t3}\PYG{o}{*}\PYG{n}{K\PYGZus{}3}\PYG{p}{)} \PYG{c+c1}{\PYGZsh{} m/s, eff. horizontal cond.}
\PYG{n}{K\PYGZus{}ef\PYGZus{}v} \PYG{o}{=} \PYG{n}{m}\PYG{o}{/}\PYG{p}{(}\PYG{n}{G\PYGZus{}t1}\PYG{o}{/}\PYG{n}{K\PYGZus{}1} \PYG{o}{+} \PYG{n}{G\PYGZus{}t2}\PYG{o}{/}\PYG{n}{K\PYGZus{}2} \PYG{o}{+} \PYG{n}{G\PYGZus{}t3}\PYG{o}{/}\PYG{n}{K\PYGZus{}3}\PYG{p}{)} \PYG{c+c1}{\PYGZsh{} \PYGZsh{} m/s, eff. vertical cond.}

\PYG{n+nb}{print}\PYG{p}{(}\PYG{l+s+s2}{\PYGZdq{}}\PYG{l+s+s2}{The thickness of the aquifer is }\PYG{l+s+si}{\PYGZob{}0:0.3f\PYGZcb{}}\PYG{l+s+s2}{\PYGZdq{}}\PYG{o}{.}\PYG{n}{format}\PYG{p}{(}\PYG{n}{m}\PYG{p}{)}\PYG{p}{,} \PYG{l+s+s2}{\PYGZdq{}}\PYG{l+s+s2}{m}\PYG{l+s+s2}{\PYGZdq{}}\PYG{p}{)}
\PYG{n+nb}{print}\PYG{p}{(}\PYG{l+s+s2}{\PYGZdq{}}\PYG{l+s+s2}{The effective horizontal conductivity of the aquifer is }\PYG{l+s+si}{\PYGZob{}0:0.2E\PYGZcb{}}\PYG{l+s+s2}{\PYGZdq{}}\PYG{o}{.}\PYG{n}{format}\PYG{p}{(}\PYG{n}{K\PYGZus{}ef\PYGZus{}h}\PYG{p}{)}\PYG{p}{,} \PYG{l+s+s2}{\PYGZdq{}}\PYG{l+s+s2}{m/s}\PYG{l+s+s2}{\PYGZdq{}}\PYG{p}{)}
\PYG{n+nb}{print}\PYG{p}{(}\PYG{l+s+s2}{\PYGZdq{}}\PYG{l+s+s2}{The effective vertical conductivity of the aquifer is }\PYG{l+s+si}{\PYGZob{}0:0.2E\PYGZcb{}}\PYG{l+s+s2}{\PYGZdq{}}\PYG{o}{.}\PYG{n}{format}\PYG{p}{(}\PYG{n}{K\PYGZus{}ef\PYGZus{}v}\PYG{p}{)}\PYG{p}{,} \PYG{l+s+s2}{\PYGZdq{}}\PYG{l+s+s2}{m/s}\PYG{l+s+s2}{\PYGZdq{}}\PYG{p}{)}
\end{sphinxVerbatim}

\end{sphinxuseclass}\end{sphinxVerbatimInput}
\begin{sphinxVerbatimOutput}

\begin{sphinxuseclass}{cell_output}
\begin{sphinxVerbatim}[commandchars=\\\{\}]
The thickness of the aquifer is 6.500 m
The effective horizontal conductivity of the aquifer is 1.13E\PYGZhy{}03 m/s
The effective vertical conductivity of the aquifer is 4.46E\PYGZhy{}04 m/s
\end{sphinxVerbatim}

\end{sphinxuseclass}\end{sphinxVerbatimOutput}

\end{sphinxuseclass}
\sphinxAtStartPar
\sphinxstylestrong{Solution 5c} (L06/23)

\sphinxAtStartPar
\sphinxstylestrong{Homogeneity}: An aquifer is homogeneous when its parameters are constant throughout the porous medium, i.e. the properties of the medium are independent of space

\sphinxAtStartPar
\sphinxstylestrong{Heterogeneity}: Heterogeneous aquifer have its properties varies in space or the properties are space dependent.

\sphinxAtStartPar
\sphinxstylestrong{Isotropy}: This relates to properties of aquifer being independent of direction, i.e., \(K_v = K_h\)

\sphinxAtStartPar
\sphinxstylestrong{Anisotropy}: In this case the aquifer properties are direction dependent, i.e., \(K_v \neq K_h\).

\sphinxAtStartPar
\sphinxstylestrong{Q6. Well} (ca. 5 pts.)

\sphinxAtStartPar
a. Sketch the pumping scenario of an unconfined aquifer (vertical cross section) and label all possible quantities (ca 3 pts.)

\sphinxAtStartPar
b. The conductivity of a confined aquifer (8 m thick) is estimated to be \(4\cdot 10^{-4}\) m/s. If the steady\sphinxhyphen{}state discharge 50 m3/s, using the Theis equation (\(s = Q/4\pi T·W(u)\), with \(W(u) = 15\)), find the drawdown in the aquifer. (2 points)

\sphinxAtStartPar
\sphinxstylestrong{Solution 6a}
(L08/16)

\sphinxAtStartPar
Figure below presents the scenario of a well in an \sphinxstyleemphasis{unconfined} aquifer.



\begin{sphinxuseclass}{cell}\begin{sphinxVerbatimInput}

\begin{sphinxuseclass}{cell_input}
\begin{sphinxVerbatim}[commandchars=\\\{\}]
\PYG{c+c1}{\PYGZsh{}Solution 6b}

\PYG{c+c1}{\PYGZsh{}Given}

\PYG{n}{Q\PYGZus{}6} \PYG{o}{=} \PYG{l+m+mi}{50} \PYG{c+c1}{\PYGZsh{} m³/s, discharge}
\PYG{n}{K\PYGZus{}6} \PYG{o}{=} \PYG{l+m+mi}{4}\PYG{o}{*}\PYG{l+m+mi}{10}\PYG{o}{*}\PYG{o}{*}\PYG{o}{\PYGZhy{}}\PYG{l+m+mi}{4} \PYG{c+c1}{\PYGZsh{} m/s, conductivity}
\PYG{n}{m\PYGZus{}6} \PYG{o}{=} \PYG{l+m+mi}{8} \PYG{c+c1}{\PYGZsh{} m, thickness}
\PYG{n}{W\PYGZus{}u} \PYG{o}{=} \PYG{l+m+mi}{15} \PYG{c+c1}{\PYGZsh{} (), well function}

\PYG{c+c1}{\PYGZsh{} interim cal.}
\PYG{n}{T\PYGZus{}6} \PYG{o}{=} \PYG{n}{K\PYGZus{}6} \PYG{o}{*} \PYG{n}{m\PYGZus{}6} \PYG{c+c1}{\PYGZsh{} m²/s, Transmissivity T = K*m}

\PYG{c+c1}{\PYGZsh{} solution}
\PYG{n}{s\PYGZus{}6} \PYG{o}{=} \PYG{p}{(}\PYG{n}{Q\PYGZus{}6}\PYG{o}{/}\PYG{p}{(}\PYG{l+m+mi}{4}\PYG{o}{*}\PYG{n}{np}\PYG{o}{.}\PYG{n}{pi}\PYG{o}{*}\PYG{n}{T\PYGZus{}6}\PYG{p}{)}\PYG{p}{)} \PYG{o}{*} \PYG{n}{W\PYGZus{}u} \PYG{c+c1}{\PYGZsh{} m, drawdown}

\PYG{n+nb}{print}\PYG{p}{(}\PYG{l+s+s2}{\PYGZdq{}}\PYG{l+s+s2}{The Transmissivity of the aquifer is }\PYG{l+s+si}{\PYGZob{}0:0.5f\PYGZcb{}}\PYG{l+s+s2}{\PYGZdq{}}\PYG{o}{.}\PYG{n}{format}\PYG{p}{(}\PYG{n}{T\PYGZus{}6}\PYG{p}{)}\PYG{p}{,} \PYG{l+s+s2}{\PYGZdq{}}\PYG{l+s+s2}{m}\PYG{l+s+se}{\PYGZbs{}u00b2}\PYG{l+s+s2}{/s}\PYG{l+s+s2}{\PYGZdq{}}\PYG{p}{)}
\PYG{n+nb}{print}\PYG{p}{(}\PYG{l+s+s2}{\PYGZdq{}}\PYG{l+s+s2}{The drawdown in the well is }\PYG{l+s+si}{\PYGZob{}0:0.2f\PYGZcb{}}\PYG{l+s+s2}{\PYGZdq{}}\PYG{o}{.}\PYG{n}{format}\PYG{p}{(}\PYG{n}{s\PYGZus{}6}\PYG{p}{)}\PYG{p}{,} \PYG{l+s+s2}{\PYGZdq{}}\PYG{l+s+s2}{m}\PYG{l+s+s2}{\PYGZdq{}}\PYG{p}{)}
\end{sphinxVerbatim}

\end{sphinxuseclass}\end{sphinxVerbatimInput}
\begin{sphinxVerbatimOutput}

\begin{sphinxuseclass}{cell_output}
\begin{sphinxVerbatim}[commandchars=\\\{\}]
The Transmissivity of the aquifer is 0.00320 m²/s
The drawdown in the well is 18650.97 m
\end{sphinxVerbatim}

\end{sphinxuseclass}\end{sphinxVerbatimOutput}

\end{sphinxuseclass}
\sphinxAtStartPar
\sphinxstylestrong{Q7. Conservative Transport}  (ca. 7 pts.)

\sphinxAtStartPar
a. How is reactive transport different to conservative transport in the aquifers. (2 points)

\sphinxAtStartPar
b. With suitable sketch distinguish between advective flux and dispersive flux. (2 points)

\sphinxAtStartPar
c. A column (L = 1.2 m and Ø = 5 cm) was packed with sandy soil (ne= 35\%  K= 0,0002 m/s). The hydraulic head at the inlet and the outlet was set to 230 m and 235 m, resp. The NaCl solution with conc. 10 mg/L was steadily introduced to the column after saturating it with distilled water. The experiment condition was such that diffusive flow could be neglected.  You may make justified assumption for any missing information.

\sphinxAtStartPar
c.i. What will be the advective mass flux at the outlet of the column? (1.5 points)

\sphinxAtStartPar
c.ii. Considering initial concentration difference between inlet and outlet to be 10 mg/L, what    will be the dispersive mass flux at the outlet? (1.5 points)

\sphinxAtStartPar
(Hint: Dispersive and Advective fluxes are either of \( n_e \cdot v\cdot C\) and \(n_e\cdot \alpha \cdot v \cdot \Delta C/L\))

\sphinxAtStartPar
\sphinxstylestrong{Solution 7a} (L09/05)

\sphinxAtStartPar
A chemical in groundwater is subject to conservative transport processes if there is:
\begin{itemize}
\item {} 
\sphinxAtStartPar
no interaction with the solid material,

\item {} 
\sphinxAtStartPar
no interaction with other chemicals,

\item {} 
\sphinxAtStartPar
no interaction with microbes.

\end{itemize}

\sphinxAtStartPar
When either of the above are part of the groundwater, the transport process is reactive.

\sphinxAtStartPar
\sphinxstylestrong{Solution 7b} (L09/09)

\sphinxAtStartPar
The sketch below distinguish between advective and dispersive fluxes. The figure in the left is of advective process and that in the right results to dispersive flux.



\begin{sphinxuseclass}{cell}\begin{sphinxVerbatimInput}

\begin{sphinxuseclass}{cell_input}
\begin{sphinxVerbatim}[commandchars=\\\{\}]
\PYG{c+c1}{\PYGZsh{}Solution 7c}

\PYG{n}{L\PYGZus{}7} \PYG{o}{=} \PYG{l+m+mf}{1.2} \PYG{c+c1}{\PYGZsh{} m, col. length}
\PYG{n}{Dia\PYGZus{}7} \PYG{o}{=} \PYG{l+m+mi}{5} \PYG{c+c1}{\PYGZsh{} cm, col. diameter }
\PYG{n}{ne\PYGZus{}7} \PYG{o}{=} \PYG{l+m+mf}{0.35} \PYG{c+c1}{\PYGZsh{} (), effective porosity}
\PYG{n}{K\PYGZus{}7} \PYG{o}{=} \PYG{l+m+mf}{0.0002} \PYG{c+c1}{\PYGZsh{} m/s, conductivity}
\PYG{n}{H\PYGZus{}7in} \PYG{o}{=} \PYG{l+m+mi}{235} \PYG{c+c1}{\PYGZsh{} m, head inlet}
\PYG{n}{H\PYGZus{}7out} \PYG{o}{=} \PYG{l+m+mi}{230} \PYG{c+c1}{\PYGZsh{} m, head outlet}
\PYG{n}{C\PYGZus{}7} \PYG{o}{=} \PYG{l+m+mi}{10} \PYG{c+c1}{\PYGZsh{} mg/L, NaCl concentration}
\PYG{n}{al\PYGZus{}7} \PYG{o}{=} \PYG{l+m+mi}{1} \PYG{c+c1}{\PYGZsh{} m, assumed}
\PYG{n}{C\PYGZus{}7d} \PYG{o}{=} \PYG{l+m+mi}{10} \PYG{c+c1}{\PYGZsh{} mg/L}

\PYG{c+c1}{\PYGZsh{}intermediate calc.}
\PYG{n}{i\PYGZus{}7} \PYG{o}{=} \PYG{p}{(}\PYG{n}{H\PYGZus{}7in}\PYG{o}{\PYGZhy{}}\PYG{n}{H\PYGZus{}7out}\PYG{p}{)}\PYG{o}{/}\PYG{n}{L\PYGZus{}7} \PYG{c+c1}{\PYGZsh{} (), head gradient}
\PYG{n}{v\PYGZus{}7dar} \PYG{o}{=} \PYG{n}{K\PYGZus{}7}\PYG{o}{*}\PYG{n}{i\PYGZus{}7} \PYG{c+c1}{\PYGZsh{} m/s, darcy velocity}
\PYG{n}{v\PYGZus{}7av} \PYG{o}{=} \PYG{n}{v\PYGZus{}7dar}\PYG{o}{/}\PYG{n}{ne\PYGZus{}7} \PYG{c+c1}{\PYGZsh{} m/s, average linear velocity}


\PYG{c+c1}{\PYGZsh{} Solution}
\PYG{n}{F\PYGZus{}7ad} \PYG{o}{=} \PYG{n}{ne\PYGZus{}7}\PYG{o}{*}\PYG{n}{v\PYGZus{}7av}\PYG{o}{*}\PYG{n}{C\PYGZus{}7} \PYG{c+c1}{\PYGZsh{} mg\PYGZhy{}m/L\PYGZhy{}s, advective flux }
\PYG{n}{F\PYGZus{}7dis} \PYG{o}{=} \PYG{n}{ne\PYGZus{}7}\PYG{o}{*}\PYG{n}{al\PYGZus{}7}\PYG{o}{*}\PYG{n}{v\PYGZus{}7av}\PYG{o}{*}\PYG{n}{C\PYGZus{}7d}\PYG{o}{/}\PYG{n}{L\PYGZus{}7} \PYG{c+c1}{\PYGZsh{} mg\PYGZhy{}m/L\PYGZhy{}s, dispersive flux }

\PYG{n+nb}{print}\PYG{p}{(}\PYG{l+s+s2}{\PYGZdq{}}\PYG{l+s+s2}{The hydraulic gradient is }\PYG{l+s+si}{\PYGZob{}0:0.4f\PYGZcb{}}\PYG{l+s+s2}{\PYGZdq{}}\PYG{o}{.}\PYG{n}{format}\PYG{p}{(}\PYG{n}{i\PYGZus{}7}\PYG{p}{)}\PYG{p}{,} \PYG{l+s+s2}{\PYGZdq{}}\PYG{l+s+s2}{\PYGZdq{}}\PYG{p}{)}
\PYG{n+nb}{print}\PYG{p}{(}\PYG{l+s+s2}{\PYGZdq{}}\PYG{l+s+s2}{The Darcy velocity is }\PYG{l+s+si}{\PYGZob{}0:0.4f\PYGZcb{}}\PYG{l+s+s2}{\PYGZdq{}}\PYG{o}{.}\PYG{n}{format}\PYG{p}{(}\PYG{n}{v\PYGZus{}7dar}\PYG{p}{)}\PYG{p}{,} \PYG{l+s+s2}{\PYGZdq{}}\PYG{l+s+s2}{m/s}\PYG{l+s+s2}{\PYGZdq{}}\PYG{p}{)}
\PYG{n+nb}{print}\PYG{p}{(}\PYG{l+s+s2}{\PYGZdq{}}\PYG{l+s+s2}{The average linear velocity is }\PYG{l+s+si}{\PYGZob{}0:0.4f\PYGZcb{}}\PYG{l+s+s2}{\PYGZdq{}}\PYG{o}{.}\PYG{n}{format}\PYG{p}{(}\PYG{n}{v\PYGZus{}7av}\PYG{p}{)}\PYG{p}{,} \PYG{l+s+s2}{\PYGZdq{}}\PYG{l+s+s2}{m/s}\PYG{l+s+s2}{\PYGZdq{}}\PYG{p}{)}
\PYG{n+nb}{print}\PYG{p}{(}\PYG{l+s+s2}{\PYGZdq{}}\PYG{l+s+s2}{The advective flux is }\PYG{l+s+si}{\PYGZob{}0:0.10f\PYGZcb{}}\PYG{l+s+s2}{\PYGZdq{}}\PYG{o}{.}\PYG{n}{format}\PYG{p}{(}\PYG{n}{F\PYGZus{}7ad}\PYG{p}{)}\PYG{p}{,} \PYG{l+s+s2}{\PYGZdq{}}\PYG{l+s+s2}{mg\PYGZhy{}m/L\PYGZhy{}s}\PYG{l+s+s2}{\PYGZdq{}}\PYG{p}{)}
\PYG{n+nb}{print}\PYG{p}{(}\PYG{l+s+s2}{\PYGZdq{}}\PYG{l+s+s2}{The dispersive flux is }\PYG{l+s+si}{\PYGZob{}0:0.10f\PYGZcb{}}\PYG{l+s+s2}{\PYGZdq{}}\PYG{o}{.}\PYG{n}{format}\PYG{p}{(}\PYG{n}{F\PYGZus{}7dis}\PYG{p}{)}\PYG{p}{,} \PYG{l+s+s2}{\PYGZdq{}}\PYG{l+s+s2}{mg\PYGZhy{}m/L\PYGZhy{}s}\PYG{l+s+s2}{\PYGZdq{}}\PYG{p}{)}
\end{sphinxVerbatim}

\end{sphinxuseclass}\end{sphinxVerbatimInput}
\begin{sphinxVerbatimOutput}

\begin{sphinxuseclass}{cell_output}
\begin{sphinxVerbatim}[commandchars=\\\{\}]
The hydraulic gradient is 4.1667 
The Darcy velocity is 0.0008 m/s
The average linear velocity is 0.0024 m/s
The advective flux is 0.0083333333 mg\PYGZhy{}m/L\PYGZhy{}s
The dispersive flux is 0.0069444444 mg\PYGZhy{}m/L\PYGZhy{}s
\end{sphinxVerbatim}

\end{sphinxuseclass}\end{sphinxVerbatimOutput}

\end{sphinxuseclass}
\sphinxAtStartPar
\sphinxstylestrong{Q8. Sorption Isotherms} (ca. 10 pts)

\sphinxAtStartPar
Five batch tests (different initial concentrations – see table below) were performed to determine the sorption properties of a sediment. For each batch 20 g of sediment in 30 mL of water were used. The measured equilibrium solute concentrations are also provided in the table.

\sphinxAtStartPar
a)	Complete the above table								(ca. 3 pts.)

\sphinxAtStartPar
b)	Plot the results in the diagram below and draw a Henry isotherm			(ca. 3 pts.)

\sphinxAtStartPar
c)	How is retardation related to isotherm (ca. 2 points)
(value and unit!)		(ca. 2 pts.).

\begin{sphinxuseclass}{cell}\begin{sphinxVerbatimInput}

\begin{sphinxuseclass}{cell_input}
\begin{sphinxVerbatim}[commandchars=\\\{\}]
\PYG{n}{head} \PYG{o}{=} \PYG{p}{[}\PYG{l+s+s2}{\PYGZdq{}}\PYG{l+s+s2}{Batch nr. }\PYG{l+s+s2}{\PYGZdq{}}\PYG{p}{,} \PYG{l+s+s2}{\PYGZdq{}}\PYG{l+s+s2}{Initial Conc. (mg/L) }\PYG{l+s+s2}{\PYGZdq{}}\PYG{p}{,} \PYG{l+s+s2}{\PYGZdq{}}\PYG{l+s+s2}{Equi. Conc. (mg/L)}\PYG{l+s+s2}{\PYGZdq{}}\PYG{p}{,} \PYG{l+s+s2}{\PYGZdq{}}\PYG{l+s+s2}{Sorbed mass (g)}\PYG{l+s+s2}{\PYGZdq{}}\PYG{p}{,} \PYG{l+s+s2}{\PYGZdq{}}\PYG{l+s+s2}{Sorbed mass/solid (mg/g)}\PYG{l+s+s2}{\PYGZdq{}} \PYG{p}{]}
\PYG{n}{bn} \PYG{o}{=} \PYG{n}{np}\PYG{o}{.}\PYG{n}{array}\PYG{p}{(}\PYG{p}{[}\PYG{l+m+mi}{1}\PYG{p}{,}\PYG{l+m+mi}{2}\PYG{p}{,}\PYG{l+m+mi}{3}\PYG{p}{,}\PYG{l+m+mi}{4}\PYG{p}{,}\PYG{l+m+mi}{5}\PYG{p}{]}\PYG{p}{)}
\PYG{n}{C\PYGZus{}0} \PYG{o}{=} \PYG{n}{np}\PYG{o}{.}\PYG{n}{array}\PYG{p}{(}\PYG{p}{[}\PYG{l+m+mi}{5}\PYG{p}{,} \PYG{l+m+mi}{10}\PYG{p}{,} \PYG{l+m+mi}{15}\PYG{p}{,} \PYG{l+m+mi}{20}\PYG{p}{,} \PYG{l+m+mi}{25}\PYG{p}{]}\PYG{p}{)}\PYG{c+c1}{\PYGZsh{} mg/L, initial conc.}
\PYG{n}{C\PYGZus{}eq} \PYG{o}{=} \PYG{n}{np}\PYG{o}{.}\PYG{n}{array}\PYG{p}{(}\PYG{p}{[}\PYG{l+m+mf}{2.5}\PYG{p}{,} \PYG{l+m+mf}{4.9}\PYG{p}{,} \PYG{l+m+mi}{8}\PYG{p}{,} \PYG{l+m+mf}{9.8}\PYG{p}{,} \PYG{l+m+mf}{13.2}\PYG{p}{]}\PYG{p}{)}\PYG{c+c1}{\PYGZsh{} mg/L, equilibrium conc.}
\PYG{n}{s2} \PYG{o}{=} \PYG{n}{ips}\PYG{o}{.}\PYG{n}{sheet}\PYG{p}{(}\PYG{n}{rows}\PYG{o}{=}\PYG{l+m+mi}{6}\PYG{p}{,} \PYG{n}{columns}\PYG{o}{=}\PYG{l+m+mi}{5}\PYG{p}{,} \PYG{n}{row\PYGZus{}headers}\PYG{o}{=}\PYG{k+kc}{False}\PYG{p}{,} \PYG{n}{column\PYGZus{}headers}\PYG{o}{=}\PYG{n}{head}\PYG{p}{)}
\PYG{n}{ips}\PYG{o}{.}\PYG{n}{column}\PYG{p}{(}\PYG{l+m+mi}{0}\PYG{p}{,} \PYG{n}{bn}\PYG{p}{,} \PYG{n}{row\PYGZus{}start}\PYG{o}{=}\PYG{l+m+mi}{0}\PYG{p}{)} 
\PYG{n}{ips}\PYG{o}{.}\PYG{n}{column}\PYG{p}{(}\PYG{l+m+mi}{1}\PYG{p}{,} \PYG{n}{C\PYGZus{}0}\PYG{p}{,} \PYG{n}{row\PYGZus{}start}\PYG{o}{=}\PYG{l+m+mi}{0}\PYG{p}{)}
\PYG{n}{ips}\PYG{o}{.}\PYG{n}{column}\PYG{p}{(}\PYG{l+m+mi}{2}\PYG{p}{,} \PYG{n}{C\PYGZus{}eq}\PYG{p}{,} \PYG{n}{row\PYGZus{}start}\PYG{o}{=}\PYG{l+m+mi}{0}\PYG{p}{)}\PYG{p}{;} 
\PYG{n}{s2}
\end{sphinxVerbatim}

\end{sphinxuseclass}\end{sphinxVerbatimInput}
\begin{sphinxVerbatimOutput}

\begin{sphinxuseclass}{cell_output}
\begin{sphinxVerbatim}[commandchars=\\\{\}]
Sheet(cells=(Cell(column\PYGZus{}end=0, column\PYGZus{}start=0, row\PYGZus{}end=4, row\PYGZus{}start=0, squeeze\PYGZus{}row=False, type=\PYGZsq{}numeric\PYGZsq{}, val…
\end{sphinxVerbatim}

\end{sphinxuseclass}\end{sphinxVerbatimOutput}

\end{sphinxuseclass}
\begin{sphinxuseclass}{cell}\begin{sphinxVerbatimInput}

\begin{sphinxuseclass}{cell_input}
\begin{sphinxVerbatim}[commandchars=\\\{\}]
\PYG{c+c1}{\PYGZsh{} SOlution of Problem 10 a (T07/HP9)}

\PYG{c+c1}{\PYGZsh{} Given}
\PYG{n}{v\PYGZus{}ml} \PYG{o}{=} \PYG{l+m+mi}{30} \PYG{c+c1}{\PYGZsh{} ml of water used in expt.}
\PYG{n}{v\PYGZus{}l} \PYG{o}{=} \PYG{n}{v\PYGZus{}ml}\PYG{o}{/}\PYG{l+m+mi}{1000} \PYG{c+c1}{\PYGZsh{} L, unit conversion}
\PYG{n}{m\PYGZus{}s} \PYG{o}{=} \PYG{l+m+mi}{20} \PYG{c+c1}{\PYGZsh{} g, solid mass used in expt.}

\PYG{n}{bn} \PYG{o}{=} \PYG{n}{np}\PYG{o}{.}\PYG{n}{array}\PYG{p}{(}\PYG{p}{[}\PYG{l+m+mi}{1}\PYG{p}{,}\PYG{l+m+mi}{2}\PYG{p}{,}\PYG{l+m+mi}{3}\PYG{p}{,}\PYG{l+m+mi}{4}\PYG{p}{,}\PYG{l+m+mi}{5}\PYG{p}{]}\PYG{p}{)}
\PYG{n}{C\PYGZus{}0} \PYG{o}{=} \PYG{n}{np}\PYG{o}{.}\PYG{n}{array}\PYG{p}{(}\PYG{p}{[}\PYG{l+m+mi}{5}\PYG{p}{,} \PYG{l+m+mi}{10}\PYG{p}{,} \PYG{l+m+mi}{15}\PYG{p}{,} \PYG{l+m+mi}{20}\PYG{p}{,} \PYG{l+m+mi}{25}\PYG{p}{]}\PYG{p}{)}\PYG{c+c1}{\PYGZsh{} mg/L, initial conc.}
\PYG{n}{C\PYGZus{}eq} \PYG{o}{=} \PYG{n}{np}\PYG{o}{.}\PYG{n}{array}\PYG{p}{(}\PYG{p}{[}\PYG{l+m+mf}{2.5}\PYG{p}{,} \PYG{l+m+mf}{4.9}\PYG{p}{,} \PYG{l+m+mi}{8}\PYG{p}{,} \PYG{l+m+mf}{9.8}\PYG{p}{,} \PYG{l+m+mf}{13.2}\PYG{p}{]}\PYG{p}{)}\PYG{c+c1}{\PYGZsh{} mg/L, equilibrium conc.}
\PYG{n}{s\PYGZus{}m} \PYG{o}{=} \PYG{p}{(}\PYG{n}{C\PYGZus{}0}\PYG{o}{\PYGZhy{}}\PYG{n}{C\PYGZus{}eq}\PYG{p}{)}\PYG{o}{*}\PYG{n}{v\PYGZus{}l}
\PYG{n}{m\PYGZus{}m} \PYG{o}{=} \PYG{n}{s\PYGZus{}m}\PYG{o}{/}\PYG{n}{m\PYGZus{}s}\PYG{c+c1}{\PYGZsh{} mg/g, mass ratio}

\PYG{c+c1}{\PYGZsh{}output}
\PYG{n}{d8} \PYG{o}{=} \PYG{p}{\PYGZob{}}\PYG{l+s+s2}{\PYGZdq{}}\PYG{l+s+s2}{Batch Nr}\PYG{l+s+s2}{\PYGZdq{}}\PYG{p}{:} \PYG{n}{bn}\PYG{p}{,} \PYG{l+s+s2}{\PYGZdq{}}\PYG{l+s+s2}{Initial Conc. (mg/L)}\PYG{l+s+s2}{\PYGZdq{}}\PYG{p}{:} \PYG{n}{C\PYGZus{}0}\PYG{p}{,} \PYG{l+s+s2}{\PYGZdq{}}\PYG{l+s+s2}{Equi. Conc. (mg/L)}\PYG{l+s+s2}{\PYGZdq{}}\PYG{p}{:} \PYG{n}{C\PYGZus{}eq}\PYG{p}{,} \PYG{l+s+s2}{\PYGZdq{}}\PYG{l+s+s2}{Sorbed mass (g)}\PYG{l+s+s2}{\PYGZdq{}}\PYG{p}{:} \PYG{n}{s\PYGZus{}m}\PYG{p}{,} \PYG{l+s+s2}{\PYGZdq{}}\PYG{l+s+s2}{Sorbed mass/solid (mg/g)}\PYG{l+s+s2}{\PYGZdq{}} \PYG{p}{:}\PYG{n}{m\PYGZus{}m}\PYG{p}{\PYGZcb{}}
\PYG{n}{df9} \PYG{o}{=} \PYG{n}{pd}\PYG{o}{.}\PYG{n}{DataFrame}\PYG{p}{(}\PYG{n}{d8}\PYG{p}{)}\PYG{p}{;} \PYG{n}{df9}
\end{sphinxVerbatim}

\end{sphinxuseclass}\end{sphinxVerbatimInput}
\begin{sphinxVerbatimOutput}

\begin{sphinxuseclass}{cell_output}
\begin{sphinxVerbatim}[commandchars=\\\{\}]
   Batch Nr  Initial Conc. (mg/L)  Equi. Conc. (mg/L)  Sorbed mass (g)  \PYGZbs{}
0         1                     5                 2.5            0.075   
1         2                    10                 4.9            0.153   
2         3                    15                 8.0            0.210   
3         4                    20                 9.8            0.306   
4         5                    25                13.2            0.354   

   Sorbed mass/solid (mg/g)  
0                   0.00375  
1                   0.00765  
2                   0.01050  
3                   0.01530  
4                   0.01770  
\end{sphinxVerbatim}

\end{sphinxuseclass}\end{sphinxVerbatimOutput}

\end{sphinxuseclass}
\begin{sphinxuseclass}{cell}\begin{sphinxVerbatimInput}

\begin{sphinxuseclass}{cell_input}
\begin{sphinxVerbatim}[commandchars=\\\{\}]
\PYG{c+c1}{\PYGZsh{} Solution of proble 10 (b) (T07/HP9)}
\PYG{c+c1}{\PYGZsh{} fit}
\PYG{n}{slope}\PYG{p}{,} \PYG{n}{intercept}\PYG{p}{,} \PYG{n}{r\PYGZus{}value}\PYG{p}{,} \PYG{n}{p\PYGZus{}value}\PYG{p}{,} \PYG{n}{std\PYGZus{}err} \PYG{o}{=} \PYG{n}{stats}\PYG{o}{.}\PYG{n}{linregress}\PYG{p}{(}\PYG{n}{C\PYGZus{}eq}\PYG{p}{,} \PYG{n}{m\PYGZus{}m}\PYG{p}{)} \PYG{c+c1}{\PYGZsh{} linear regression}

\PYG{c+c1}{\PYGZsh{}plot and fit}
\PYG{n}{fig} \PYG{o}{=} \PYG{n}{plt}\PYG{o}{.}\PYG{n}{figure}\PYG{p}{(}\PYG{p}{)}\PYG{p}{;} \PYG{n}{plt}\PYG{o}{.}\PYG{n}{plot}\PYG{p}{(}\PYG{n}{C\PYGZus{}eq}\PYG{p}{,} \PYG{n}{m\PYGZus{}m}\PYG{p}{,} \PYG{l+s+s1}{\PYGZsq{}}\PYG{l+s+s1}{bo}\PYG{l+s+s1}{\PYGZsq{}}\PYG{p}{,} \PYG{n}{label}\PYG{o}{=}\PYG{l+s+s1}{\PYGZsq{}}\PYG{l+s+s1}{ provided data}\PYG{l+s+s1}{\PYGZsq{}}\PYG{p}{)}\PYG{p}{;}
\PYG{n}{pred} \PYG{o}{=} \PYG{n}{intercept} \PYG{o}{+} \PYG{n}{slope}\PYG{o}{*}\PYG{n}{C\PYGZus{}eq} \PYG{c+c1}{\PYGZsh{} fit line}
\PYG{n}{plt}\PYG{o}{.}\PYG{n}{plot}\PYG{p}{(}\PYG{n}{C\PYGZus{}eq}\PYG{p}{,} \PYG{n}{pred}\PYG{p}{,} \PYG{l+s+s1}{\PYGZsq{}}\PYG{l+s+s1}{r}\PYG{l+s+s1}{\PYGZsq{}}\PYG{p}{,} \PYG{n}{label}\PYG{o}{=}\PYG{l+s+s1}{\PYGZsq{}}\PYG{l+s+s1}{y=}\PYG{l+s+si}{\PYGZob{}:.2E\PYGZcb{}}\PYG{l+s+s1}{x+}\PYG{l+s+si}{\PYGZob{}:.2f\PYGZcb{}}\PYG{l+s+s1}{\PYGZsq{}}\PYG{o}{.}\PYG{n}{format}\PYG{p}{(}\PYG{n}{slope}\PYG{p}{,}\PYG{n}{intercept}\PYG{p}{)}\PYG{p}{)} \PYG{p}{;}
\PYG{n}{plt}\PYG{o}{.}\PYG{n}{xlabel}\PYG{p}{(}\PYG{l+s+sa}{r}\PYG{l+s+s2}{\PYGZdq{}}\PYG{l+s+s2}{\PYGZdl{}C\PYGZus{}}\PYG{l+s+si}{\PYGZob{}eq\PYGZcb{}}\PYG{l+s+s2}{ \PYGZdl{} mg/L}\PYG{l+s+s2}{\PYGZdq{}}\PYG{p}{)}\PYG{p}{;} \PYG{n}{plt}\PYG{o}{.}\PYG{n}{ylabel}\PYG{p}{(}\PYG{l+s+sa}{r}\PYG{l+s+s2}{\PYGZdq{}}\PYG{l+s+s2}{\PYGZdl{}C\PYGZus{}}\PYG{l+s+si}{\PYGZob{}a\PYGZcb{}}\PYG{l+s+s2}{ \PYGZdl{} mg/g}\PYG{l+s+s2}{\PYGZdq{}}\PYG{p}{)}\PYG{p}{;}
\PYG{n}{plt}\PYG{o}{.}\PYG{n}{grid}\PYG{p}{(}\PYG{p}{)}\PYG{p}{;} \PYG{n}{plt}\PYG{o}{.}\PYG{n}{legend}\PYG{p}{(}\PYG{n}{fontsize}\PYG{o}{=}\PYG{l+m+mi}{11}\PYG{p}{)}\PYG{p}{;}  \PYG{n}{plt}\PYG{o}{.}\PYG{n}{text}\PYG{p}{(}\PYG{l+m+mf}{2.2}\PYG{p}{,} \PYG{l+m+mf}{0.014}\PYG{p}{,}\PYG{l+s+s1}{\PYGZsq{}}\PYG{l+s+s1}{\PYGZdl{}R\PYGZca{}2 = }\PYG{l+s+si}{\PYGZpc{}0.2f}\PYG{l+s+s1}{\PYGZdl{}}\PYG{l+s+s1}{\PYGZsq{}} \PYG{o}{\PYGZpc{}} \PYG{n}{r\PYGZus{}value}\PYG{p}{)}
\PYG{n}{plt}\PYG{o}{.}\PYG{n}{text}\PYG{p}{(}\PYG{l+m+mf}{2.2}\PYG{p}{,} \PYG{l+m+mf}{0.012}\PYG{p}{,}\PYG{l+s+s1}{\PYGZsq{}}\PYG{l+s+s1}{\PYGZdl{}C\PYGZus{}a = K\PYGZus{}}\PYG{l+s+si}{\PYGZob{}d\PYGZcb{}}\PYG{l+s+s1}{\PYGZbs{}}\PYG{l+s+s1}{cdot C\PYGZus{}}\PYG{l+s+si}{\PYGZob{}eq\PYGZcb{}}\PYG{l+s+s1}{\PYGZdl{}}\PYG{l+s+s1}{\PYGZsq{}}\PYG{p}{)}\PYG{p}{;} 
\end{sphinxVerbatim}

\end{sphinxuseclass}\end{sphinxVerbatimInput}
\begin{sphinxVerbatimOutput}

\begin{sphinxuseclass}{cell_output}
\noindent\sphinxincludegraphics{{C:/Users/vibhu/GWtextbook/_build/jupyter_execute/GW_exam_2019_20_36_0}.png}

\end{sphinxuseclass}\end{sphinxVerbatimOutput}

\end{sphinxuseclass}
\sphinxAtStartPar
\sphinxstylestrong{solution 8c}
(L10/13)

\sphinxAtStartPar
The following relation relates Retardation (\(R\)) with linear isotherm (\(K_d\))
\begin{equation*}
\begin{split}
R = 1+ \frac{1-n_e}{n_e}\rho_s K_d
\end{split}
\end{equation*}
\sphinxAtStartPar
with effective porosity \(n_e\), solid density \(\rho_s\).

\sphinxAtStartPar
\sphinxstylestrong{Q9. Groundwater Modelling} (ca. 8 points.)

\sphinxAtStartPar
a. Distinguish between conceptual model and mathematical model; and between analytical solution and empirical solution (ca. 4 points).

\sphinxAtStartPar
b. Draw a conceptual model for a rectangular aquifer 100 m long and 20 m wide. Discretize the domain with 1/10 of the length length\sphinxhyphen{}wise and 1/5 of the width width\sphinxhyphen{}wise. Assure that flow in the model is from left to right direction (ca. 3 points).

\sphinxAtStartPar
c. How is a no\sphinxhyphen{}flow boundary condition mathematically defined? (ca. 1 point)

\sphinxAtStartPar
\sphinxstylestrong{Solution 9a}
(L11/04\sphinxhyphen{}06)

\sphinxAtStartPar
A model or also a \sphinxstyleemphasis{conceptual model} is a representation, an image or a description of a real system.

\sphinxAtStartPar
example for a real system: porous medium with water flowing through the pores (Darcy experiment)

\sphinxAtStartPar
A \sphinxstyleemphasis{mathematical model} provides a quantitative representation of the relevant system components, processes and impacts in the area of investigation. The quantitative representation is based on mathematical equations.

\sphinxAtStartPar
\sphinxstyleemphasis{Analytical solution} : These are exact mathematical expressions solving the model equations.

\sphinxAtStartPar
\sphinxstyleemphasis{Emperical solution} : These are solution based on experimental results.

\sphinxAtStartPar
\sphinxstylestrong{Solution 9b} \sphinxhyphen{} (L14/12)



\sphinxAtStartPar
\sphinxstylestrong{Solution 9c}
(L13/16)

\sphinxAtStartPar
A no\sphinxhyphen{}flow boundary condition is special case of second type or Neumann boundary condition. For no flow condition head gradient is equated to zero, i.e., there is no gradient and thus no flow (water flows from high to low head). Mathematically, this is:

\sphinxAtStartPar
\(\frac{dh}{dx} = 0 \) for no\sphinxhyphen{}flow along \(x-\)axis, with \(h\) representing head.

\sphinxAtStartPar
Good Luck.







\renewcommand{\indexname}{Index}
\printindex
\end{document}